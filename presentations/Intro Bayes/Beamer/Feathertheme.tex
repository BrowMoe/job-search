\documentclass[10pt]{beamer}
\usetheme[
%%% option passed to the outer theme
%    progressstyle=fixedCircCnt,   % fixedCircCnt, movingCircCnt (moving is deault)
  ]{Feather}
  
% If you want to change the colors of the various elements in the theme, edit and uncomment the following lines

% Change the bar colors:
\setbeamercolor{Feather}{fg=red!80!black,bg=red!60!black}

% Change the color of the structural elements:
\setbeamercolor{structure}{fg=black}

% Change the frame title text color:
\setbeamercolor{frametitle}{fg=yellow}

% Change the normal text color background:
%\setbeamercolor{normal text}{fg=black,bg=gray!10}

%-------------------------------------------------------
% INCLUDE PACKAGES
%-------------------------------------------------------

\usepackage{ifpdf}
\usepackage{graphicx} 
\ifpdf
\DeclareGraphicsRule{*}{mps}{*}{}
\else
% Recent LaTeX versions donât require the next line
% \DeclareGraphicsRule{*}{eps}{*}{}
\fi

\usepackage[english, french]{babel}
\usepackage[utf8]{inputenc}
%\usepackage[TS1,T1]{fontenc}
%\usepackage{helvet}
\usepackage{fourier}
\usepackage{hyperref}
%\usepackage{lmodern}
%\usepackage{mathrsfs}
%\usepackage{natbib}
\usepackage{amsmath,amssymb,amsfonts,graphicx,shorttoc,textpos,caption,here, yfonts}
\usepackage{verbatim,enumerate,hyperref,dsfont,fancyhdr,setspace,array}
\usepackage[citestyle=verbose]{biblatex}

\bibliography{Feathertheme.bib}
\DeclareCiteCommand{\cite}
{\usebibmacro{prenote}}
{\usebibmacro{citeindex}%
\usebibmacro{cite}}
{\multicitedelim}
{\usebibmacro{cite:postnote}}
\usefonttheme[onlymath]{serif}
\graphicspath{{Feathergraphics/}}
\newbibmacro*{cite}{%
\usebibmacro{cite:citepages}%
\ifciteseen
	{\iffieldundef{shorthand}
		{\usebibmacro{cite:short}}
		{\usebibmacro{cite:shorthand}}}
	{\usebibmacro{cite:full}}}

\renewbibmacro*{cite}{%
\usebibmacro{cite:citepages}%
{\usebibmacro{cite:full}}}
%-------------------------------------------------------
% DEFFINING AND REDEFINING COMMANDS
%-------------------------------------------------------

% colored hyperlinks
\newcommand{\chref}[2]{
  \href{#1}{{\usebeamercolor[bg]{Feather}#2}}
}
%\bibliographystyle{apalike}
\renewcommand\bibfont{\scriptsize}
\setbeamertemplate{frametitle continuation}[from second]

%-------------------------------------------------------
% INFORMATION IN THE TITLE PAGE
%-------------------------------------------------------

\title[Pre-talk : Bayesian asymptotics and nonparametric inverse problems] % [] is optional - is placed on the bottom of the sidebar on every slide
{ % is placed on the title page
     \textbf{Pre-talk : Bayesian asymptotics and nonparametric inverse problems}
}






\institute[Ruprecht-Karls-Universität Heidelberg]
{
  
  Ruprecht-Karls-Universität Heidelberg
  %there must be an empty line above this line - otherwise some unwanted space is added between the university and the country (I do not know why;( )
}

\date{$30^{th}$ of June 2016}

\author[Xavier Loizeau] % [] is optional - is placed on the bottom of the sidebar on every slide
{ % is placed on the title page
      \textit{In anticipation of Bartek Knapik talk\\      Xavier Loizeau}
}
%-------------------------------------------------------
% THE BODY OF THE PRESENTATION
%-------------------------------------------------------

\begin{document}
%-------------------------------------------------------
% THE TITLEPAGE
%-------------------------------------------------------

{\1% % this is the name of the PDF file for the background
\begin{frame}[plain,noframenumbering] % the plain option removes the header from the title page, noframenumbering removes the numbering of this frame only
  \titlepage % call the title page information from above
\end{frame}}

\AtBeginSection[]{
\begin{frame}{Contents}
\setcounter{tocdepth}{2}
\tableofcontents[sectionstyle=show/shaded, subsectionstyle=show/show/hide]
\end{frame}
}

\begin{frame}{Contents}
\setcounter{tocdepth}{1}
\tableofcontents
\end{frame}

%-------------------------------------------------------
\section{Frequentist paradigm : convergence rates}
%-------------------------------------------------------

\begin{frame}{Point-wise comparison of frequentist estimators}
\begin{columns}
\begin{column}[T]{.4\textwidth}%
\hspace*{8ex}
\includegraphics<1>[scale=.8]{inv-niv-minimax.8}%
\includegraphics<2>[scale=.8]{inv-niv-minimax.9}% lower
\end{column}
\begin{column}[T]{.7\textwidth}%
\begin{itemize}
\item<2->
Measure the performance of a frequentist estimator $\widehat{\theta}_{n}$ using {\textcolor{red}{quadratic risk}} 
for a given parameter $\theta^{\circ}$
\begin{equation*}
	\mathbb{E}_{\theta^{\circ}} \left[d\left(\widehat{\theta}_{n}, \theta^{\circ}\right)^{2}\right]
\end{equation*}
\end{itemize}
\end{column}
\end{columns}
\end{frame}


\begin{frame}{Comparison of frequentist estimators over a subset}
\begin{columns}
\begin{column}[T]{.4\textwidth}%
\hspace*{8ex}
\includegraphics<1-2>[scale=.8]{inv-gssm-minimax.1}%
\includegraphics<3>[scale=.8]{inv-gssm-minimax.2}% lower
\includegraphics<4>[scale=.8]{inv-gssm-minimax.3}%
\includegraphics<5>[scale=.8]{inv-gssm-minimax.4}%
\includegraphics<6>[scale=.8]{inv-gssm-minimax.5}% upper + lower
\includegraphics<7>[scale=.8]{inv-gssm-minimax.6}%
\includegraphics<8->[scale=.8]{inv-gssm-minimax.7}%
\end{column}\begin{column}[T]{.7\textwidth}%
\begin{itemize}
\item<2->
Measure the performance of a frequentist estimator using {\textcolor{red}{maximal risk}} 
over a  class $\Theta^{\circ}$  of parameters
  \begin{equation*}
\sup\limits_{\theta^{\circ} \in \Theta^{\circ}} \mathbb{E}_{\theta^{\circ}} \left[d\left(\widehat{\theta}, \theta^{\circ}\right)^{2}\right]
\end{equation*}
 \item<3->  Goal : finding a lower bound $\mathcal{R}_{n}^{\star}(\Theta^{\circ})$ ...
\begin{equation*}
\inf\limits_{\tilde{\theta}}\sup\limits_{\theta^{\circ} \in \Theta^{\circ}} \mathbb{E}_{\theta^{\circ}}\left[d\left(\tilde{\theta}, \theta^{\circ}\right)^{2}\right] \geq C_{1} \cdot \mathcal{R}_{n}^{\star}\left(\Theta^{\circ}\right)
\end{equation*}
\item<8-> ... which is reached by an estimator $\widehat{\theta}$
  \begin{equation*}
\sup\limits_{\theta^{0}\in \Theta^{\circ}} \mathbb{E}_{\theta^{\circ}}\left[d\left(\widehat{\theta}, \theta^{\circ}\right)^{2}\right] \leq C_{2} \cdot \mathcal{R}_{n}^{\star}\left(\Theta^{\circ}\right)
\end{equation*}
\end{itemize}
\end{column}
\end{columns}
\end{frame}


\section{Bayesian formulation : concentration rate}

\begin{frame}{Bayesian paradigm}{Founding principles of the Bayesian paradigm}

Let $(\Theta, \mathfrak{A})$ be a measurable space

$\boldsymbol{\theta}$ is a random variable $(\Omega, \mathfrak{F}) \rightarrow (\Theta, \mathfrak{A})$

\[ \boldsymbol{\theta} \sim \Pi \]

\bigskip

Denote by $p_{\theta}$ the density of $\Pi$ with respect to a measure $\mu$

Posterior distribution

\[ \forall \mathfrak{B} \in \mathfrak{A} \quad \Pi_{\boldsymbol{\theta}\vert Y}(\mathfrak{B}) = \frac{\int_{\mathfrak{B}} p_{\boldsymbol{\theta}}(Y)d\Pi(\boldsymbol{\theta})}{\int_{\Theta} p_{\boldsymbol{\theta}}(X)d\Pi(\boldsymbol{\theta})}\]

\end{frame}

\begin{frame}{Bayesian paradigm}{Comparing prior choices}

Taking a frequentist point of view
\begin{itemize}
\item $\theta^{\circ}$ the true parameter
\item Is $\Pi_{\boldsymbol{\theta}\vert Y}$ shrinking around $\theta^{\circ}$ as $\epsilon$ tends to $0$?
\item How fast ?

\end{itemize}

\end{frame}


\begin{frame}{Bayesian paradigm}{Comparing prior choices}
\begin{columns}
\begin{column}[T]{.4\textwidth}%
\hspace*{4ex}
\includegraphics<1>[scale=.8]{reg.31}%
\includegraphics<2>[scale=.8]{reg.32}%
\includegraphics<3>[scale=.8]{reg.33}%
\includegraphics<4>[scale=.8]{reg.34}%
\includegraphics<5>[scale=.8]{inv-gssm-minimax.3}%
\includegraphics<6>[scale=.8]{inv-gssm-minimax.4}%
\includegraphics<7>[scale=.8]{inv-gssm-minimax.5}% upper + lower
\includegraphics<8>[scale=.8]{inv-gssm-minimax.6}%
\end{column}\begin{column}[T]{.7\textwidth}%

\begin{itemize}

\item<1-> Concentration rate $(\phi_{\epsilon})_{\epsilon \in \mathbb{R}_{+}}$
\[\exists c \in \mathbb{R}_{+}, \quad \lim_{\epsilon \rightarrow 0} \mathbb{E}_{\theta^{\circ}}\left[ \Pi_{\boldsymbol{\theta}\vert Y} \left(d\left(\boldsymbol{\theta}, \theta^{\circ} \right)^{2} \geq c \, \phi_{\epsilon} \right) \right] = 0 \]

\item<5-> Uniform concentration rate $(\phi_{\epsilon})_{\epsilon \in \mathbb{R}_{+}}$ over a subset on $\Theta^{\circ} \subset \Theta$
\[  \lim_{\epsilon \rightarrow 0} \sup_{\theta^{\circ} \in \Theta^{\circ}} \mathbb{E}_{\theta^{\circ}}\left[ \Pi_{\boldsymbol{\theta}\vert Y} \left(d\left( \boldsymbol{\theta}, \theta^{\circ} \right)^{2} \geq c \, \phi_{\epsilon} \right) \right] = 0\]

\end{itemize}
\end{column}
\end{columns}

\end{frame}

%-------------------------------------------------------
\section{Inverse problems}
%-------------------------------------------------------
\begin{frame}{Inverse problems}

\hspace*{4ex}
\centerline{%
\includegraphics<1>[scale=.8]{inv.2}%
\includegraphics<2>[scale=.8]{inv.3}%
\includegraphics<3>[scale=.8]{inv.4}%
\includegraphics<4>[scale=.8]{inv.8}%
\includegraphics<5>[scale=.8]{inv.9}%
\includegraphics<6>[scale=.8]{inv.10}%
\includegraphics<7>[scale=.8]{inv.11}%
\includegraphics<8>[scale=.8]{inv.12}%
}



\end{frame}

\end{document}









