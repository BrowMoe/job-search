\section{On the shape of the posterior mean}\label{2.6}

We have hence seen that in a general case, considering the asymptotic iteration , the posterior distribution using a sieve prior contracts around the projection estimator and while using a hierarchical prior, the posterior contracts around some penalised contrast model selection estimator.

It is also interesting to note that for any number of iteration $\eta$, the posterior mean can be written both as a shrinkage and as an aggregation estimator.
Indeed, we have
\textcolor{red}{shapes}
Using the same methodology as in \ncite{JJASRS} we can obtain, in the Gaussian case, the following optimality properties for the posterior mean of the iterated hierarchical prior.
\textcolor{red}{properties}

\medskip

Aggregation estimates gathered a lot of interest, see for example \textcolor{red}{Tsybakov}.
While considering such estimators, optimality is formulated in the following way:

We can see that our estimator reaches this optimality criterion.
We will hence consider this shape of estimator in another inverse problem, the circular density deconvolution.