\documentclass[a4paper,11pt]{book}
\synctex=1

\usepackage{graphicx}
% put all the other packages here:

\usepackage{command}
\usepackage{packages}
\usepackage{style}
\usepackage{personal}
\usepackage{theorem}
\usepackage{hyperref} 

\begin{document}

%%%%%%%%%%%%%%%%%%%%%%%% Page 1 %%%%%%%%%%%%%%%%%%%%%%%%%
\pagestyle{empty}

\begin{center}
\huge	\textbf{INAUGURAL-DISSERTATION}
\end{center} 

\bigskip

\bigskip

\begin{center}
	\Large zur\\
	Erlangung der Doktorw\"urde\\
	der\\
	Naturwissenschaftlich-Mathematischen Gesamtfakult\"at\\
	der Ruprecht-Karls-Universit\"at\\
	Heidelberg
\end{center}

\vspace*{\fill}

\begin{center}
	\large vorgelegt von 
\end{center}

\bigskip

\begin{center}
\Large \myDegree \myName

\bigskip

aus \myBirthPlace
\end{center}

\vspace*{\fill}

\begin{flushleft}
\Large Tag der m\"undlichen Pr\"ufung: % do not print the date !
\end{flushleft}

\newpage

%%%%%%%%%%%%%%%%%%%%%%%% Page 2 %%%%%%%%%%%%%%%%%%%%%%%%%
~

\bigskip

\begin{center}
\huge	\textbf{\myTitle}
\end{center} 

\vspace*{\fill}

\Large
\begin{center}
\noindent\begin{tabular}{@{}l@{\enskip}l}
Betreuer:& \mySupervisor\\
& \myCoSupervisor
\end{tabular}
\end{center}

\bigskip

~

\newpage 

%%%%%%%%%%%%%%%%%%%%%%%% Page 3 %%%%%%%%%%%%%%%%%%%%%%%%%

~

\bigskip

\huge \textbf{Acknowledgments}

\vspace*{\fill}

\large 
\myAcknowledgements


\bigskip

~

\newpage 

%%%%%%%%%%%%%%%%%%%%%%%% Page 4 %%%%%%%%%%%%%%%%%%%%%%%%%

\huge \textbf{Zusammenfassung}

\vspace*{\fill}

\normalsize
\myZusammenfassung



\newpage

%%%%%%%%%%%%%%%%%%%%%%%% Page 5 %%%%%%%%%%%%%%%%%%%%%%%%%


\huge \textbf{Abstract}

\vspace*{\fill}

\normalsize
\myAbstract


\bigskip

~

\newpage 

%%%%%%%%%%%%%%%%%%%%%%%% Page 6 %%%%%%%%%%%%%%%%%%%%%%%%%


\thispagestyle{empty}  % one empty page before printing a dedication (you may do that) or contents
\quad 
\newpage
%
\tableofcontents
\chapter*{List of notations}

\section*{Spaces}
\subsection*{General case}
\begin{alignat*}{3}
& \left(\mathds{Y}, \mathcal{Y}\right) &&: && \quad \text{the measurable space of observations};\\
& \mathds{T} \subset \R &&: && \quad \text{time domain};\\
& \mathds{D} \subset \R &&: && \quad \text{intensity domain};\\
& \Xi = \left\{f : \mathds{T} \rightarrow \mathds{D}\right\} &&: && \quad \text{the parameter space represented in time domain};\\
& \mathds{F} \subset \R &&: && \quad \text{frequency domain};\\
& \mathds{A} \subset \C &&: && \quad \text{amplitude in frequency domain};\\
& \left(\Theta, \mathcal{B}\right) = \left(\left\{\theta : \mathds{F} \rightarrow \mathds{A}\right\}, \mathcal{B}\right) &&: && \quad \text{the parameter space represented in frequency domain};
\end{alignat*}

\subsection*{Inverse Gaussian sequence space model}
\begin{alignat*}{3}
%& \left([0, 1[, \mathcal{A}\right) &&: && \quad \text{the measurable space of observations};\\
%& \mathcal{D}_{\mu}([0,1[) && : && \quad \text{space of densities on } [0, 1[ \text{ with respect to } \mu ;\\
& \mathcal{M}([0, 1[) && : && \quad \text{space of probability measures on } [0, 1[\\
& \mathcal{S}^{+}(\mathds{Z}) && : && \quad \text{ set of all positive definite, complex valued functions } [f] \text{ on } \mathds{Z} \text{ with } [f](0) = 1;\\
\end{alignat*}

\subsection*{Circular density deconvolution model}
\begin{alignat*}{3}
%& \left([0, 1[, \mathcal{A}\right) &&: && \quad \text{the measurable space of observations};\\
%& \mathcal{D}_{\mu}([0,1[) && : && \quad \text{space of densities on } [0, 1[ \text{ with respect to } \mu ;\\
& \mathcal{M}([0, 1[) && : && \quad \text{space of probability measures on } [0, 1[\\
& \mathcal{S}^{+}(\mathds{Z}) && : && \quad \text{ set of all positive definite, complex valued functions } [f] \text{ on } \mathds{Z} \text{ with } [f](0) = 1;\\
\end{alignat*}

\section*{Measures and densities}

\subsection*{General case}
\begin{alignat*}{5}
& \P_{\boldsymbol{\theta}^{m}} &&:&& \mathcal{B} \rightarrow [0,1])_{f \in \Xi} &&:&& \quad \text{Gaussian sieve prior with threshold parameter }m\\
\end{alignat*}


$\P_{\boldsymbol{\theta}^{M}}$

$\P_{M}$

$\P_{M \vert Y^{n}}^{n, (\eta)}$

$\P_{\boldsymbol{\theta}^{m}\vert Y^{n}}^{n, (\eta)}$

$\P_{\boldsymbol{\theta}^{M}\vert Y^{n}}^{n, (\eta)}$

\subsection*{Inverse Gaussian sequence space model}
\subsection*{Circular density deconvolution model}

\begin{alignat*}{5}
& (\mathds{P}_{Y \vert f} &&:&& \mathcal{Y} \rightarrow [0,1])_{f \in \Xi} &&:&& \quad \text{family to which the data distribution belongs}\\
& \mathds{P}^{X} &&:&& \mathcal{A} \rightarrow [0,1] &&: && \quad \text{distribution of interest};\\
& \mathds{P}^{\epsilon}&&:&& \mathcal{A} \rightarrow [0,1] &&: && \quad \text{distribution of the noise};\\
& \mu&&:&& \mathcal{A} \rightarrow \mathds{R}_{+} &&: && \quad \text{ a sigma-finite measure, dominating both } \mathds{P}^{X} \text{ and } \mathds{P}^{\epsilon};\\
& f^{X}&&:&& [0, 1[ \rightarrow \overline{\mathds{R}_{+}} &&: && \quad \text{density of } \mathds{P}^{X} \text{ with respect to } \mu;\\
& f^{\epsilon}&&:&& [0, 1[ \rightarrow \overline{\mathds{R}_{+}} &&: && \quad \text{density of } \mathds{P}^{\epsilon} \text{ with respect to } \mu;\\
& \mathds{P}^{Y} (=\mathds{P}^{X}*\mathds{P}^{\epsilon})&&:&& \mathcal{A} \rightarrow [0,1] &&: && \quad \text{distribution of the observations};\\
& f^{Y}(=f^{X}*f^{\epsilon})&&:&& [0, 1[ \rightarrow \overline{\mathds{R}_{+}} &&: && \quad \text{density of } \mathds{P}^{Y} \text{ with respect to } \mu;\\
& \# && : && \sigma(\mathcal{P}(G)) \rightarrow \N && : && \text{counting measure on the sigma-algebra generated by the set of subsets of} G
\end{alignat*}


\section*{Random variables}
\subsection*{General case}
\subsection*{Inverse Gaussian sequence space model}
\subsection*{Circular density deconvolution model}
\begin{alignat*}{5}
& X &&:&& (\Omega, \mathcal{B}) \rightarrow ([0,1[, \mathcal{A}) &&: && \quad \text{a random variable with distribution } \mathds{P}^{X};\\
& \epsilon &&:&& (\Omega, \mathcal{B}) \rightarrow ([0,1[, \mathcal{A})  &&: && \quad \text{a random variable with distribution } \mathds{P}^{\epsilon};\\
& Y (=X \Box \epsilon) &&:&& (\Omega, \mathcal{B}) \rightarrow ([0,1[, \mathcal{A})  &&: && \quad \text{a random variable with distribution } \mathds{P}^{Y};\\
& X^{n} (=(X^{n}_{i})_{i \in \llbracket 1, n \rrbracket}) &&:&& (\Omega, \mathcal{B}) \rightarrow ([0,1[^{n}, \mathcal{A}^{\otimes n}) &&: && \quad \text{a } n \text{-vector of i.i.d. replications of } X;\\
& \epsilon^{n} (=(\epsilon^{n}_{i})_{i \in \llbracket 1, n \rrbracket})&&:&& (\Omega, \mathcal{B}) \rightarrow ([0,1[^{n}, \mathcal{A}^{\otimes n}) &&: && \quad \text{a } n \text{-vector of i.i.d. replications of } \epsilon;\\
& Y^{n} (=(Y^{n}_{i})_{i \in \llbracket 1, n \rrbracket}) &&:&& (\Omega, \mathcal{B}) \rightarrow ([0,1[^{n}, \mathcal{A}^{\otimes n})  &&: && \quad \text{a } n \text{-vector of i.i.d. replications of } Y;\\
& \boldsymbol{\theta}^{m} &&:&& (\Omega, \mathcal{B}) \rightarrow ([0,1[, \mathcal{A}) &&: && \quad \text{a random variable with distribution } \mathds{P}^{X};\\
& \boldsymbol{\theta}^{M} &&:&& (\Omega, \mathcal{B}) \rightarrow ([0,1[, \mathcal{A}) &&: && \quad \text{a random variable with distribution } \mathds{P}^{X};\\
& \widetilde{\theta}^{m, (\eta)} &&:&& (\Omega, \mathcal{B}) \rightarrow ([0,1[, \mathcal{A}) &&: && \quad \text{a random variable with distribution } \mathds{P}^{X};\\
& \widehat{\theta}^{(\eta)} &&:&& (\Omega, \mathcal{B}) \rightarrow ([0,1[, \mathcal{A}) &&: && \quad \text{a random variable with distribution } \mathds{P}^{X};\\
& \bar{\theta}^{m, (\eta)} &&:&& (\Omega, \mathcal{B}) \rightarrow ([0,1[, \mathcal{A}) &&: && \quad \text{a random variable with distribution } \mathds{P}^{X};
\end{alignat*}

\section*{Unary operators}
\subsection*{General case}
\subsection*{Inverse Gaussian sequence space model}
\subsection*{Circular density deconvolution model}
\begin{alignat*}{6}
& &&\mathds{E}[\cdot] && && &&:&& \quad \text{the expectation operator when the distribution is obvious};\\
& \forall \mathds{P} \text{ distribution } && \mathds{E}_{\mathds{P}}[\cdot] && && &&:&& \quad \text{the expected value under } \mathds{P};\\
& && ^{*}\cdot &&:&& \mathcal{M}([0, 1[) && \rightarrow && \mathcal{M}([0, 1[);\\
& && && && \mathds{P} && \mapsto && ^{*}\mathds{P} = \mathds{P}*\mathds{P}^{\epsilon}\\
& && ^{*}\cdot &&:&& \mathcal{D}_{\mu}([0, 1[) && \rightarrow && \mathcal{D}_{\mu}([0, 1[);\\
& && && && f && \mapsto && ^{*}f = f*f^{\epsilon}\\
& \forall j \in \mathds{Z} && e_{j}(\cdot) &&:&& [0,1[ && \rightarrow && \mathds{C};\\
& && && && x && \mapsto && \exp[-2 i \pi j x]\\
& && \mathcal{F}_{\mu}(\cdot) &&:&& \mathcal{D}_{\mu}([0, 1[) && \rightarrow && \mathcal{S}^{+}(\mathds{Z});\\
& && && && f && \mapsto && [f] = \left(j \mapsto \int_{0}^{1} f(x) e_{j}(x) d\mu(x)\right)\\
& && \mathcal{F}(\cdot) &&:&& \mathcal{M}([0, 1[) && \rightarrow && \mathcal{S}^{+}(\mathds{Z});\\
& && && && \mathds{P} && \mapsto && [\mathds{P}] = \left(j \mapsto \int_{0}^{1} e_{j}(x) d\mathds{P}(x)\right)\\
& && \mathcal{F}_{\mu}^{-1}(\cdot) &&:&& \mathcal{S}^{+}(\mathds{Z}) && \rightarrow && \mathcal{D}_{\mu}([0, 1[);\\
& && && && [f] && \mapsto && f = \left(x \mapsto \sum\limits_{j \in \mathds{Z}} [f]_{j} e_{j}(x) \right)\\
& && \mathcal{F}^{-1}(\cdot) &&:&& \mathcal{S}^{+}(\mathds{Z}) && \rightarrow && \mathcal{M}([0, 1[);\\
& && && && [\mathds{P}] && \mapsto && \mathds{P} = \left(A \mapsto \int_{A} \sum\limits_{j \in \mathds{Z}} [\mathds{P}]_{j} e_{j}(x) dx\right)\\
\end{alignat*}

\section*{Binary operators}
\subsection*{General case}
\subsection*{Inverse Gaussian sequence space model}
\subsection*{Circular density deconvolution model}
\begin{alignat*}{7}
& \cdot \Box \cdot &&:&& [0,1[^{2} &&\rightarrow&& [0,1[ &&: && \quad \text{the modular addition binary operator on the unit segment};\\
& && && (x,y) &&\mapsto&& x+y-\lfloor x+y \rfloor && &&\\
\end{alignat*}

\section*{Miscellaneous}
\subsection*{General case}
\subsection*{Inverse Gaussian sequence space model}
\subsection*{Circular density deconvolution model}
For any set $S$, subset $s \subseteq S$ we note $\mathds{1}_{s}$ the indicatrix function
\begin{alignat*}{5}
	&\mathds{1}_{s} &&:&& S &&\rightarrow&& \{0, 1\};\\
	& && && x && \mapsto &&
		\begin{cases}
			0 \text{ if } x \notin s\\
			1 \text{ if } x \in s
		\end{cases}
\end{alignat*}
















%
\chapter{Introduction}\label{1}
\section{Inverse problems}\label{INTRO_INVERSEPROBLEMS}
\textit{We introduce here some fundamentals of inverse problem theory. This section builds upon results which can be found, for example, in \ncite{engle1996regularization}}.

Consider the situation when one wishes to estimate an object, say $f$ belonging to a space $\Xi$.
The object $f$ will be referred to as "parameter of interest" and the space $\Xi$ as "parameter space".
We assume that this parameter has some influence on a system which we are able to observe.
Hence, recording observation of this system allows us to learn about this parameter.
These observations will be referred to as "data" and denoted by $Y$.
Our ability to learn in such a way is central as it underpins our ability to understand the behaviour of a system, to predict it and to influence it.
This is a wide family of problems and we shall give more precision about the specific subfamily we consider.

We will give particular interest to inverse problems, a family of models where one wants to infer on $f$ but the data we observe comes from a system led by a different parameter $g$ which can be written $g := T(f)$ where $T$ is an mapping from $\Xi$ to itself.

These models gathered interest for a long time due to their numerous applications, theoretical physics, astrophysics, medical imaging, econometrics, or acoustics are just a few of the countless examples of such applications.
Many of those models have the particularity to be ill-posed in the sense of \citet{cite:hadamard}.
That is to say, if we build an estimator $\widehat{g}$ of $g = T(f)$ from the data $Y$ and try to apply the inverse $T^{-1}$ of $T$ to this estimator in order to estimate $f$, one of the following problems might arise:
\begin{itemize}
\item non existence (the equation $T(x) = \widehat{g}$ does not have a solution);
\item non unicity (the equation $T(x) = \widehat{g}$ has multiple solutions);
\item non stability (the solutions to the equations $T(x) = \widehat{g}$ does not depend continuously on $\widehat{g}$).
\end{itemize}

Though Hadamard thought that inverse problems do not arise in practical situations and that problems of our realm only are of the well-posed kind.
Evolution of science proved him wrong and ill-posed problems now have many applications.
The specific challenges they represent has since gathered ever increasing interest.
We will use two examples throughout this thesis, respectively introduced in \nref{INTRO_IGSSM} and \nref{INTRO_CIRCULARDECONVOLUTION}.

\bigskip

From now on, we will assume that $\Xi$ is an infinite dimensional vector space on $\mathds{K}$ (standing for either $\R$ or $\mathds{C}$), equipped with a norm $\Vert \cdot \Vert_{\Xi}$ which is derived from an inner product $\langle \cdot \vert \cdot \rangle_{\Xi}$ and $\Xi$ is hence an infinite dimensional Hilbert space.
We denote by $\mathcal{L}(\Xi)$ the set of bounded endomorphisms on $\Xi$, that is to say linear operators $S$ from $\Xi$ onto itself such that there exists $M$ in $\R_{+}$ verifying, for any $x$ in $\Xi$, the following inequality $\Vert S(x) \Vert_{\Xi} \leq M \Vert x \Vert_{\Xi}$.
In addition, we denote, for any $S$ in $\mathcal{L}(\Xi)$, $\mathcal{D}(S)$ its definition domain, $\mathcal{R}(S)$ its range, and $\mathcal{N}(S)$ its kernel.
Assume, from now on, that $T$ is an element of $\mathcal{L}(\Xi)$.

In this case, the following property gives us sufficient and necessary conditions under which the two first forms of ill-posedness do not happen.

\begin{pr*}
For any $S$ in $\mathcal{L}(\Xi)$, and any element $x$ of $\Xi$, there exists an unique solution to the equation $S(y) = \widehat{S(x)}$ for any estimate $\widehat{S(x)}$ of $S(x)$ in $\Xi$ if and only if
\item[\mylabel{BACKGROUND_INVERSEPROBLEMS_EXISTENCE}{\dgrau{\bfseries{(existence): }}}] $\widehat{S(x)}$ belongs to the range $\mathcal{R}(S)$ of $S$;
\item[\mylabel{BACKGROUND_INVERSEPROBLEMS_UNIQUENESS}{\dgrau\bfseries{(uniqueness): }}] the operator $S$ is injective, i.e. $\mathcal{N}(S) = \{0\}$.
\reEnd
\end{pr*}

In the case where the existence condition is not fulfilled, one would look for an approximate solution $\widetilde{f}$ minimising an objective function which could be the distance with respect to $\Vert \cdot \Vert_{\Xi}$, that is to say, if it exists, $\widetilde{f} \in \argmin_{x \in \mathcal{D}(T)}\Vert T(x) - \widehat{g} \Vert_{\Xi}$.
If the uniqueness condition is not fulfilled then we can look for the solution with minimal norm, once again, assuming that it exists.

We will see that the orthogonal projection operators, with respect to $\langle \cdot \vert \cdot \rangle_{\Xi}$, plays an important role.
Indeed, one can show how the last property relates to the  orthogonal projection onto the closure of the range of $T$, $\overline{\mathcal{R}}(T)$.
First introduce the following notations.

\begin{de}
For any $S$ in $\mathcal{L}(\Xi)$, denote by $S^{\star}$ its adjoint operator with respect to $\langle \cdot \vert \cdot \rangle_{\Xi}$, that is to say the unique operator such that for any $x$ and $y$ in $\Xi$ we have $ \langle S(x) \vert y \rangle_{\Xi} = \langle x \vert S^{\star}(y) \rangle_{\Xi}$.
For any subspace $\mathds{U}$ of $\Xi$, denote by $\Pi_{\mathds{U}}$ the orthogonal projection onto $\mathds{U}$ with respect to $\langle \cdot \vert \cdot \rangle_{\Xi}$.
\assEnd
\end{de}

We can now formulate the following property linking the distance minimising criteria with the orthogonal projection onto the closure of the range of $T$.

\begin{pr*}
For any $S$ in $\mathcal{L}(\Xi)$; any element $x$ of $\Xi$; any estimate $\widehat{S(x)}$ of $S(x)$ in $\Xi$; and any estimate $\widetilde{x}$ of $x$ which lies within $\mathcal{D}(S)$, the following assertions are equivalent:
\item[\mylabel{BACKGROUND_INVERSEPROBLEMS_PROJECTION_i}{\dgrau{\bfseries{i (distance to the target minimisation)}}}]: $\widetilde{x}$ minimises the function $y \mapsto \Vert \widehat{S(x)} - S(y) \Vert_{\Xi}$;
\item[\mylabel{BACKGROUND_INVERSEPROBLEMS_PROJECTION_ii}{\dgrau\bfseries{ii}}]: $\Pi_{\overline{R}(S)}(\widehat{S(x)}) = S(\widetilde{x})$;
\item[\mylabel{BACKGROUND_INVERSEPROBLEMS_PROJECTION_iii}{\dgrau\bfseries{iii (normal equation)}}]: $S^{\star}(\widehat{S(x)}) = S^{\star}(S(\widetilde{x}))$.
\reEnd
\end{pr*}

Given those considerations, it is naturally that one defines the generalised inverse (also called pseudo inverse or Moore-Penrose inverse).

\begin{de}
For any linear subspace $\mathds{U}$ of $\Xi$, denote $\mathds{U}^{\perp}$ its orthogonal complement with respect to $\langle \cdot \vert \cdot \rangle_{\Xi}$ that is $\mathds{U}^{\perp} := \{x \in \Xi: \forall u \in \mathds{U}, \langle x \vert u \rangle_{\Xi} = 0\}$.
Moreover, denote $\oplus$ the direct sum binary operator.
Then, for any linear operator $S$, define its generalised inverse $S^{+}$ as the unique linear extension of $S^{-1}: \mathcal{R}(S) \rightarrow \mathcal{N}(S)^{\perp}$ to the domain $\mathcal{D}(S^{+}) := \mathcal{R}(S) \oplus \mathcal{R}(S)^{\perp}$ with $\mathcal{N}(S^{+}) = \mathcal{R}(S)^{\perp}$ satisfying for any $x$ in $\mathcal{D}(S^{+})$ the equality $S^{+}(x) := S^{-1}(\Pi_{\overline{\mathcal{R}}(S)}(x))$.
\assEnd
\end{de}

One should note that the generalised inverse has the following important properties.

\begin{rmk}
For any $S$ in $\mathcal{L}(\Xi)$, the following equalities stand: $S S^{+} S = S$, $S^{+} S S^{+} = S^{+}$, $S^{+} S = \Pi_{\mathcal{N}(S)^{\perp}}$ and for any $x$ in $\mathcal{D}(S^{+})$, $S S^{+}(x) = \Pi_{\overline{\mathcal{R}}(S)}(x)$.
In addition, one should notice that if $S$ is injective, so is $S^{\star}S$ and as a consequence, $S^{\star}S : \Xi \rightarrow \mathcal{R}(S^{\star}S)$ is invertible which implies that for any $x$ in $\mathcal{R}(S) \oplus \mathcal{R}(S)^{\perp}$ we have that $(S^{\star} S)^{+} S^{\star} x$ is the unique solution of \ref{BACKGROUND_INVERSEPROBLEMS_PROJECTION_iii} which implies that $S^{-1}(\Pi_{\overline{\mathcal{R}}(S)}x) = \{S^{+} x\} = \{(S^{\star} S)^{+} S^{\star} x\}$.
Moreover, if $S$ is invertible, $S^{+}$ and $S^{-1}$ coincide.
\remEnd
\end{rmk}

We hence see that the Moore-Penrose inverse offers a solution to the two first sources of ill-posedness.

\begin{pr*}
For any linear operator $S$ from $\Xi$ onto itself and $x$ in $\mathcal{D}(S^{+})$, $S^{+}(x)$ is an element of $S^{-1}(\Pi_{\overline{R}(S)} x)$ and, hence fulfils \ref{BACKGROUND_INVERSEPROBLEMS_PROJECTION_i}.
Moreover, $S^{+}(x)$ is the unique element fulfilling this condition with minimal $\Vert \cdot \Vert_{\Xi}$-norm, that is $\Vert S^{+} x \Vert_{\Xi} = \inf \{\Vert h \Vert_{\Xi}: h \in S^{-1}(\Pi_{\overline{\mathcal{R}}(S)} x)\}$.
\reEnd
\end{pr*}

We will work under a set of assumptions where the two first kinds of ill-posedness do not happen.
However, we give more attention to the third source of ill-posedness.
The next property gives a general condition under which it occurs.

\begin{pr*}
Let $\Xi$ be infinite dimensional and $S$ be an injective compact linear operator from $\Xi$ onto itself.
Then $\inf_{h \in \Xi} \{\Vert S(h) \Vert_{\Xi}: \Vert h \Vert_{\Xi} = 1 \} = 0$ which implies that $S^{-1}$ (and hence $S^{+}$) are not continuous.
\reEnd
\end{pr*}

This discontinuity property highlights the need to define a so called regularised version of the Moore-Penrose inverse.
Indeed, it implies that there exists $\epsilon$ in $\R_{+}^{\star}$ such that for any $\delta$ in $\R_{+}^{\star}$, there exists a couple $(x, y)$ of elements of $\Xi$ with $\Vert x - y \Vert_{\Xi} \leq \delta$, such that $\Vert S^{+}(x) - S^{+}(y) \Vert_{\Xi} \geq \epsilon$.
Taking $x = g$ and $(y_{n})_{n \in \N} = (\widehat{g}_{n})_{n \in \N}$ a sequence of estimators, it means that even if $(\widehat{g}_{n})_{n \in \N}$ is a consistent sequence of estimations for $g$, $S^{+}(\widehat{g})$ would still not be a consistent estimator of $f$.

\medskip

We will see later in this overview that depending on the approach one uses, the strategy to overcome this difficulty will not be the same.
Namely, in the frequentist paradigm, one introduces the notion of regularisation in order to define a continuous approximation of $T^{+}$ whereas in the Bayesian paradigm, this regularisation occurs naturally in this derivation of the posterior distribution.

To make this clearer, we will first introduce the shape that our data will take.
\section{Frequentist approach}\label{INTRO_FREQ}
In the two previous sections, we have first introduced inverse problems in a general context and highlighted some difficulties which are inherent to this kind of problem.
We then introduced the type of data we will have at hand.
Now, we aim to introduce the methods we will use and more generally the paradigm they conform to, what motivates their construction and how to justify satisfaction or dissatisfaction regarding their properties.
As explained in the introduction, our methods will be of two kinds, namely frequentist and Bayesian.
In this section, we present the frequentist paradigm and the notions of decision theory which allow to quantify the quality of frequentist estimation methods.

\subsection{Estimation}\label{INTRO_FREQ_ESTIMATION}

Remind that, given a family of probability distributions on $\Xi$ and indexed by $\Xi$ itself $(\P_{x})_{x \in \Xi}$ we are interested in estimating an object $f$ in $\Xi$ while observing some data $Y$ from $\P_{g}$ where $g = Tf$ with $T$ a linear operator from $\Xi$ onto itself.
Then, the frequentist approach consists in defining an estimator of the parameter of interest using the data where an estimator is an application as defined hereafter.

\begin{de}
Given a parameter space $(\Xi, \mathcal{A})$ and an observation space $(\mathds{Y}, \mathcal{Y})$, an estimator is a measurable application from $(\mathds{Y}, \mathcal{Y})$ to $(\Xi, \mathcal{A})$.
\assEnd
\end{de}

Hence, in our particular case, an estimator would be any measurable application from $(\Xi^{n}, \mathcal{A}^{\otimes n})$ to $(\Xi, \mathcal{A})$.
As mentioned earlier, using the generalised Fourier transform, we will go through the space of sequences $\Theta$, equipped with the Borel sigma algebra generated by the $l^{2}$-norm, say $\mathcal{B}$.
In our context, some naive estimators for relevant objects of the model we consider are the so-called "empirical estimators" or "orthogonal series estimator" (OSE).

\begin{de}
Keeping in mind that we observe $Y^{n} = (Y_{p})_{p \in \llbracket 1, n \rrbracket}$ where $(Y_{p})_{p \in \mathds{Z}}$ is a stationary process such that, for any $p$ in $\mathds{Z}$, we have $Y_{p}$ follows $\P_{g}$, where $(\P_{x})_{x \in \Xi}$ is a probability distribution on $\Xi$.
Define, for any $s$ in $\mathds{F}$
\begin{alignat*}{10}
&\phi_{n}(s) && : && (\Xi^{n}, \mathcal{A}^{\otimes n}) && \rightarrow && (\Theta, \mathcal{B}); && \quad \theta_{n}(s) && : && (\Xi^{n}, \mathcal{A}^{\otimes n}) && \rightarrow && (\Theta, \mathcal{B});\\
& && && Y^{n} && \mapsto && n^{-1} \sum\nolimits_{p = 1}^{n} \langle Y_{p} \vert e_{s} \rangle_{\Xi} && \quad && && Y^{n} && \mapsto && \phi_{n}(s) \lambda^{-1}(s)
\end{alignat*}
where $\lambda^{-1}$ is well defined as we assumed $\lambda(s) \neq 0$ for any $s$.
If it were not the case, one would use the generalised inverse $\lambda^{+}(s) = \lambda(s)^{-1} \mathds{1}_{\{\lambda(s) \neq 0\}}$.

Note that this definition is suitable under assumption \nref{AS_INTRO_DATA_KNOWN} but not \nref{AS_INTRO_DATA_UNKNOWN} as it relies on the knowledge of $\lambda$ to be computed; in this case we would consider
\begin{alignat*}{5}
& \theta_{n, n_{\lambda}}(s) && : && (\Xi^{n + n_{\lambda}}, \mathcal{A}^{\otimes (n + n_{\lambda})}) && \rightarrow && (\Theta, \mathcal{B});\\
& && && (Y^{n}, \epsilon^{n_{\lambda}}) && \mapsto && \phi_{n}(s) \lambda_{n_{\lambda}}^{+}(s)
\end{alignat*}
where we define, for any $s$ in $\mathds{F}$ the estimator $\lambda_{n_{\lambda}}(s) := n_{\lambda}^{-1} \sum_{p = 1}^{n_{\lambda}} \langle \epsilon_{p} \vert e_{s} \rangle_{\Xi}$ of $\lambda(s)$ and $\lambda_{n_{\lambda}}^{+}(s) := \mathds{1}_{\{\vert \lambda_{n_{\lambda}}(s) \vert^{2} > n_{\lambda}^{-1} \}} \lambda_{n_{\lambda}}^{-1}(s)$ which hence does not rely on the knowledge of $\lambda$ but the information we have about it through the observation of $\epsilon^{n_{\lambda}}$.

From these estimators one can naturally build their counterparts
\[g_{n}: Y^{n} \mapsto \mathcal{F}^{-1}(\phi_{n}) ; \quad h_{n_{\lambda}}: \epsilon^{n_{\lambda}} \mapsto \mathcal{F}^{-1}(\lambda_{n_{\lambda}}); \quad f_{n, n_{\lambda}}; (Y^{n}, \epsilon^{n_{\lambda}}) \mapsto \mathcal{F}^{-1}(\theta_{n, n_{\lambda}}).\]
\assEnd
\end{de}

However, we have seen in \nref{INTRO_INVERSEPROBLEMS} that inverse problems define a class of statistical models which has three major characteristics.
We have also seen that two of them (non-existence or non-unicity of the solution) can be addressed thanks to the generalised inverse construction.
However, we also pointed out that even once one has addressed those two issues, they can still face the difficulty of instability of the solution.

We will see that the estimators we just defined do not escape this phenomenon.

It is in order to address this issue that one defines the family of operators called regularisations.

\begin{de}\label{INTRO_FREQ_ESTIMATION_REGULARISATION}
Given $S$ in $\mathcal{L}(\Xi)$, a family of elements of $\mathcal{L}(\Xi)$, say $\{S_{m}^{+}, m \in \R_{+} \}$ is called regularisation of $S^{+}$ if, for any $x$ in $\mathcal{D}(S^{+})$ holds $\lim_{m \rightarrow \infty} \Vert S_{m}^{+} x - S^{+} x \Vert_{\Xi} = 0$.
\assEnd
\end{de}

Note that the definition of such a family does not solve the problem by itself.
Indeed, define the operator norm such that, for any $S$ in $\mathcal{L}(\Xi)$ we have, $\Vert S \Vert_{\mathcal{L}(\Xi)} := \sup\{\Vert S(x) \Vert_{\Xi}, x \in \Xi, \Vert x \Vert_{\Xi} \leq 1\}$.
Then, if $S^{+}$ is not bounded, then, for any regularisation of $S^{+}$, we have $\lim_{m \rightarrow \infty} \Vert S^{+}_{m} \Vert_{\mathcal{L}(\Xi)} = \infty$ and hence the limit itself is not an element of $\mathcal{L}(\Xi)$.

However, for any $S$ in $\mathcal{L}(\Xi)$, and $x$ in $\Xi$, if we have a sequence of estimates, indexed by an integer $n$, say, $(\widehat{S(x)}_{n})_{n \in \N}$ of $S(x)$ such that $\lim_{n \rightarrow \infty} \Vert \widehat{S(x)}_{n} - S(x) \Vert_{\Xi} = 0$, then, there exist a sequence $m_{n}$ such that $\lim_{n \rightarrow \infty} \Vert S^{+}_{m_{n}}(\widehat{S(x)}_{n}) - S^{+}(S(x)) \Vert_{\Xi} = 0$ and hence there exists a consistent estimation procedure.

Hence, we see that the selection of the parameter $m$, which we will call regularisation parameter, is primordial.
Depending on it, the estimation procedure could be consistent or not.
In addition, within the choices leading to consistent estimation, one can obtain various convergence rates.

In this thesis, the so called regularisation by dimension reduction plays a central role.

The regularisation consists in projecting our estimate onto the "lower frequencies" from $\mathcal{U}$.
To do so, consider the following definition.

\begin{de}
Consider an index set $\mathds{M}$ (here $\N$), and a sequence of measurable subsets of $\mathds{F}$ indexed by $\mathds{M}$, say, $(\mathds{F}_{m})_{m \in \mathds{M}}$.
This sequence is called a nested sieve if:
\item[\mylabel{BACKGROUND_REGULARISATION_NESTEDSIEVE_i}{\dgrau{\bfseries{i: }}}] for any $k$ and $m$  in $\mathds{M}$ such that $k \leq m$, we have $\mathds{F}_{k} \subset \mathds{F}_{m}$;
\item[\mylabel{BACKGROUND_REGULARISATION_NESTEDSIEVE_ii}{\dgrau{\bfseries{ii: }}}] for any $m$ in $\mathds{M}$, we have $\mu(\mathds{F}_{m}) < \infty$;
\item[\mylabel{BACKGROUND_REGULARISATION_NESTEDSIEVE_iii}{\dgrau{\bfseries{iii: }}}] $\cup_{m \in \mathds{M}} \mathds{F}_{m} = \mathds{F}$.
\assEnd
\end{de}

Similarly, for any $m$ in $\mathds{M}$, we define $\mathds{U}_{\overline{m}}$ the linear subspace of $\mathds{U}$ generated by $(e_{s})_{s \in \mathds{F}_{m}}$.

For any $m$ in $\mathds{M}$, we will denote  the set $\mathds{F} \setminus \mathds{F}_{m}$ by $\mathds{F}_{m}^{c}$.
In all the examples in this thesis, $\mathds{F}$ will be either $\N$ or $\mathds{Z}$; $\mathds{M}$ will be $\mathds{N}$; and for any $m$ in $\mathds{N}$, $\mathds{F}_{m}$ will be $\{s \in \mathds{F}: \vert s \vert \leq m \}$.
The following notation will hence be regularly used: for any $s_{1}$ and $s_{2}$ in $\mathds{Z}$ with $s_{1} \leq s_{2}$ we denote $\llbracket s_{1}, s_{2} \rrbracket$ the set $[s_{1}, s_{2}] \cap \mathds{Z}$.

By extension, for any $m_{1}$ and $m_{2}$ in $\mathds{M}$, we will denote $\mathds{U}_{\underline{m_{1}}}$ the linear subspace of $\mathds{U}$ generated by $(e_{s})_{s \in \mathds{F}_{m_{1}}^{c}}$; and $\mathds{U}_{\underline{m_{1}}, \overline{m_{2}}}$ the linear subspace of $\mathds{U}$ generated by $(e_{s})_{s \in \mathds{F}_{m_{1}}^{c} \cap \mathds{F}_{m_{2}}}$.
One should note that for any $m$ in $\mathds{M}$, $\mathds{U}_{\underline{m}}$ is the orthogonal complement of $\mathds{U}_{\overline{m}}$ in $\mathds{U}$.

Then, the following operators appear naturally.

\begin{de}
We define the following family of projection operators on $\Theta$.
For any $m_{1}$ and $m_{2}$ in $\mathds{M}$ denote by $\Pi_{\overline{m_{1}}}$, $\Pi_{\underline{m_{1}}}$, and $\Pi_{\underline{m_{1}}, \overline{m_{2}}}$ the following projection operators:
\begin{alignat*}{10}
& \Pi_{\overline{m_{1}}} && : && \Theta && \rightarrow && \Theta; \quad && \Pi_{\underline{m_{1}}} && : && \Theta && \rightarrow && \Theta;\\
& && && [x] && \mapsto && (s \mapsto [x](s) \mathds{1}_{\{ s \in \mathds{F}_{m_{1}}\}}) \quad && && && [x] && \mapsto && (s \mapsto [x](s) \mathds{1}_{\{ s \in \mathds{F}_{m_{1}}^{c}\}})\\
& \Pi_{\underline{m_{1}}, \overline{m_{2}}} && : && \Theta && \rightarrow && \Theta && && && && && \\
& && && [x] && \mapsto && (s \mapsto [x](s) \mathds{1}_{\{ s \in \mathds{F}_{m_{1}}^{c} \cap \mathds{F}_{m_{2}}^{c}\}}) && && && && &&
\end{alignat*}

By extension, we define, for any $m$ in $\mathds{M}$ the truncated Fourier transform $\mathcal{F}_{\overline{m}}$
\[ \mathcal{F}_{\overline{m}} : \Xi \rightarrow \Theta; \quad [x] \mapsto [x]_{\overline{m}} = (\Pi_{\overline{m}} [x] : s \mapsto [x](s) \mathds{1}_{\{s \in \mathds{F}_{m}\}}).\]

We see that $\mathcal{F}_{\overline{m}}$ is a unitary mapping between Hilbert spaces and we should highlight that its conjugates is given by
\[\mathcal{F}_{\overline{m}}^{\star} : \Theta \rightarrow \Xi, \quad [x] \mapsto \sum_{s \in \mathds{F}_{m}} [x](s) e_{s} = \Pi_{\mathds{U}_{\overline{m}}} \mathcal{F}^{\star}([x]) = \mathcal{F}^{\star}(\Pi_{\overline{m}}[x]).\]
 \assEnd
 \end{de}

Hence, considering an inverse problem where one is interested in estimating $f$ in $\Xi$ when having at hand an estimate $\widehat{T(f)}$ of $T(f)$ where $T$ is a bounded linear operator from $\Xi$ onto itself, we will consider the family of so called projection estimators defined by $\{\widehat{f}_{\overline{m}} = \Pi_{\mathds{U}_{\overline{m}}}(T^{+}\widehat{T(f)}), m \in \mathds{M}\}$.

We will see that, often, it will be easier to approximate objects in $\Theta$ and then apply $\mathcal{F}^{\star}$.
In this perspective we extend the definition of $\mathcal{F}$ in the following way.

\begin{de}
Denote $\mathcal{L}(\Theta)$ the space of linear application from $\Theta$ onto itself.
Then, for any $S$ in $\mathcal{L}(\Xi)$, we define $[S]$ to be
\begin{alignat*}{5}
& [S] && : && \mathds{F}^{2} && \rightarrow && \mathds{K}. \\
& && && (s_{1}, s_{2}) && \mapsto && [S](s_{1}, s_{2}) = \langle e_{s_{1}} \vert S(e_{s_{2}}) \rangle_{\Xi}
\end{alignat*}
Notice that $[S]$ defines an element of $\mathcal{L}(\Theta)$ such that, for any $[x]$ in $\Theta$, $[S][x]$ is such that, for any $s$ in $\mathds{F}$, $[S][x](s)$ is given by $\sum_{s' \in \mathds{F}} [S](s, s')[x](s')$.

In addition, we define, for any $m$ in $\mathds{M}$ and $[S]$ in $\mathcal{L}(\Theta)$, the operator $[S]_{\overline{m}}$ such that for any $s_{1}$ and $s_{2}$ in $\mathds{F}$, we have $[S]_{\overline{m}}(s_{1}, s_{2}) = [S](s_{1}, s_{2}) \mathds{1}_{\{ \{s_{1} \in \mathds{F}_{m}\} \cap \{s_{2}  \in \mathds{F}_{m}\} \}}$.
It is interesting to note that for any $S$ in $\mathcal{L}(\Xi)$ and $m$ in $\mathds{M}$, if we denote $S_{\overline{m}} = \Pi_{\mathds{U}_{\overline{m}}} S \Pi_{\mathds{U}_{\overline{m}}}$, we have $[S]_{\overline{m}} = [S_{\overline{m}}]$.

We note that the adjoint operator of $[S]$ is represented for any $s_{1}$ and $s_{2}$ in $\mathds{F}$ by $[S]^{\star}(s_{1}, s_{2}) = [S^{\star}](s_{1}, s_{2}) = \overline{[S](s_{2}, s_{1})}$.
\assEnd
\end{de}

Notice that, for the operator $T$ appearing in our model, due to \nref{INTRO_DATA_DIAGONAL}, we have for any $s$ and $s'$ in $\mathds{F}$ that $[T](s, s') = \mathds{1}_{\{s = s'\}} \lambda(s)$.
Considering the objects we just introduced, the following notations will be convenient throughout the thesis.

\begin{nota}\label{INTRO_FREQ_PROJEST}
For any $m$ in $\mathds{M}$ let be the following objects:
\begin{alignat*}{15}
& \lambda_{\overline{m}} && : && \mathds{F} && \rightarrow && \mathds{K}; && \quad \theta^{\circ}_{\overline{m}} && : && \mathds{F} && \rightarrow && \mathds{K}; && \quad \phi_{\overline{m}} && : && \mathds{F} && \rightarrow && \mathds{K}.\\
& && && s && \mapsto && \Pi_{\overline{m}} \lambda(s) && && && s && \mapsto && \Pi_{\overline{m}} \theta^{\circ}(s) && && && s && \mapsto && \Pi_{\overline{m}} \phi(s) = \lambda_{\overline{m}}(s) \theta^{\circ}(s)
\end{alignat*}
as well as their counterparts in $\Xi$
\[h_{\overline{m}} := \mathcal{F}^{-1}(\lambda_{\overline{m}}) \, ; \quad f_{\overline{m}} := \mathcal{F}^{-1}(\theta^{\circ}_{\overline{m}}); \quad g_{\overline{m}} := \mathcal{F}^{-1}(\phi^{\circ}_{\overline{m}}).\]

We also define their empirical counterparts which are called "projection estimators".
Under \nref{AS_INTRO_DATA_KNOWN} they take the following form:
\begin{alignat*}{10}
& \phi_{n, \overline{m}} && : && \mathds{F} && \rightarrow && \mathds{K}; && \quad \theta_{n, \overline{m}} && : && \mathds{F} && \rightarrow && \mathds{K};\\
& && && s && \mapsto && \Pi_{\overline{m}} \phi_{n}(s) && && && s && \mapsto && \Pi_{\overline{m}} \theta_{n}(s) = \lambda_{\overline{m}}^{-1}(s) \phi_{n}(s)
\end{alignat*}
and their counterparts in $\Xi$ are
\[g_{n, \overline{m}} := \mathcal{F}^{-1}(\phi_{n, \overline{m}}); \quad f_{n, \overline{m}} := \mathcal{F}^{-1}(\theta_{n, \overline{m}}).\]
On the other hand, under \nref{AS_INTRO_DATA_UNKNOWN} they take the form
\begin{alignat*}{15}
& \phi_{n, \overline{m}} && : && \mathds{F} && \rightarrow && \mathds{K}; && \quad \lambda_{n_{\lambda}} && : && \mathds{F} && \rightarrow && \mathds{K}; && \theta_{n, n_{\lambda}, \overline{m}} && : && \mathds{F} && \rightarrow && \mathds{K}.\\
& && && s && \mapsto && \Pi_{\overline{m}} \phi_{n}(s) && && && s && \mapsto && \lambda_{n_{\lambda}}(s) && && && s && \mapsto && \Pi_{\overline{m}} \theta_{n, n_{\lambda}}(s) = \lambda_{n_{\lambda}}^{+}(s) \phi_{n, \overline{m}}(s)
\end{alignat*}
where $\lambda_{n_{\lambda}}^{+}(s) = \mathds{1}_{\{ \vert\lambda_{n_{\lambda}}(s)\vert^{2} \geq n_{\lambda}^{-1}\}} \lambda_{n_{\lambda}}^{-1}(s)$, for any $s$ in $\mathds{F}$.
Their counterparts in $\Xi$ are
\[g_{n, \overline{m}} := \mathcal{F}^{-1}(\phi_{n, \overline{m}}); \quad f_{n, \overline{m}} := \mathcal{F}^{-1}(\theta_{n, \overline{m}}).\]
\assEnd
\end{nota}

The family $\{\lambda_{\overline{m}}, m \in \mathds{M}\}$ defines a regularisation as defined in \nref{INTRO_FREQ_ESTIMATION_REGULARISATION}.
We hence have at hand a family of estimators, called projection estimators, arising from the empirical estimators based on our data while using the dimension reduction regularisation technic.

Note that many other types of regularisations have gathered interest along the years.
For example \ncite{engl1989convergence} consider the convergence rate of Tikhonov regularisation; while \ncite{cavalier2007wavelet} consider the Galerkin regularisation.

The estimation technics we will study in this thesis are deeply linked to the family of projection estimators.
As one might notice, given a set of observations, the number of potential estimators for $f$ is infinite, and it can be easily seen that most of them do not lead to a consistent estimation.
Hence, we will be interested in properties which can objectively indicate if a given estimator is satisfying.

\subsection{Decision theory}\label{INTRO_FREQ_DECISION}

As we have seen previously, for a given model, one could chose among a variety of estimators.
This choice is in general not obvious and decision theory can be used to help in this process.

To make this part more illustrative for the remaining of this script let us first introduce the following set of assumptions about the parameter space that will hold true for all of our examples.

\begin{as}\label{INTRO_FREQ_DECISION_PARAMETERSPACE}
Assume that $\Xi$ is a subset of the space of functions from $[0, 1]$ to $\mathds{C}$, equipped with the scalar product $(x, y) \mapsto \langle x \vert y \rangle_{L^{2}} = \int_{[0, 1]} x(t) \cdot \overline{y(t)} \, \text{d}t$.
Then we consider $(e_{s})_{s \in \mathds{Z}} = ([0, 1] \rightarrow \mathds{C}, t \mapsto \exp[2 \cdot \imath \cdot \pi \cdot s \cdot t])_{s \in \mathds{Z}}$.
One can see that it is an orthonormal system in $\Xi$.
Hence, $\Theta$ is a subset of $\mathds{C}^{\mathds{Z}}$ equipped with the scalar product $([x], [y]) \mapsto \langle [x] \vert [y] \rangle_{l^{2}} = \sum_{s \in \mathds{Z}} [x](s) \cdot \overline{[y](s)}$.
\assEnd
\end{as}

\medskip

We have used, in the past sections, the distance between an estimate of an object of interest and the said object as an argument about whether one should be satisfied about the said estimate.
We formalise now the criteria under which one can qualify an estimator as satisfying.

\subsubsection{The loss function $l: (\{\mathds{Y} \rightarrow \Xi\} \times \mathds{Y} \times \Xi) \rightarrow \R_{+}$}\label{INTRO_FREQ_DECISION_LOSSFUNCION}
this function represents the error made by using a certain estimator $\widehat{f}$ while estimating the true parameter $f$ when the data at hand is $Y$.

A natural choice would be to consider a distance on $\Xi$, say $d: \Xi \times \Xi \rightarrow \R_{+}$ and to define $l : \{\mathds{Y} \rightarrow \Xi\} \times \mathds{Y} \times \Xi \rightarrow \R_{+}; \quad (\widehat{f}, Y, f) \mapsto d(\widehat{f}(Y), f).$

\medskip

Under \nref{INTRO_FREQ_DECISION_PARAMETERSPACE} it is natural to consider an element of the family of $L^{p}$ distances defined for any $p$ in $\R_{+}$ and $x$ and $y$ in $\Xi$ by $\Vert x - y \Vert_{L^{p}} = (\int_{[0, 1]} \vert x(t) - y(t) \vert^{p} \, \text{d}t)^{1/p}$ with the limit cases $\Vert x - y \Vert_{L^{\infty}} = \sup_{t \in [0, 1]} \{ \vert x(t) - y(t) \vert \}$ and $\Vert x - y\Vert_{L^{0}} = \int_{[0, 1]} \mathds{1}_{\{ \vert x(t) - y(t) \vert > 0\}} \, \text{d}t$.

In this thesis we will only consider the quadratic loss function $L^{2}$.
Notice, though, that our results could be easily generalised to the case where given a measurable function $\mathfrak{u}$ in $\Xi$, one considers for any $x$ in $\Xi$ its weighted norm $\Vert x \Vert_{L^{2}_{\mathfrak{u}}} = (\int_{[0, 1]} \vert (x \star \mathfrak{u})(t)\vert^{2} \, \text{d}t)^{1/2} = (\int_{[0, 1]} \vert (\int_{[0, 1]} x(v) \cdot \mathfrak{u}(t - v) \, \text{d}v)\vert^{2} \, \text{d}t)^{1/2}$ where $\star$ stands for the convolution operator on $\Xi$.
In addition, this type of norm will nonetheless play an important role later where we consider minimax optimality over Sobolev's ellipsoids.

\medskip

In order to apply decision theory, we have to assume that the object $f$ we try to estimate belongs to the space where the loss function is finite, for which we give the following notations.
\begin{de}\label{DE_INTRO_FREQ_SPACELXI}
Let $\mathds{L}^{2}$ be the subset of $\Xi$ such that $\mathds{L}^{2} := \{x \in \Xi: \Vert x \Vert_{L^{2}} < \infty\}$ and in addition, for any function $\mathfrak{u}$ in $\Xi$ and any $r$ in $\R_{+}$ let be $\mathds{L}_{\mathfrak{u}}^{2} := \{x \in \Xi: \Vert x \Vert_{L_{\mathfrak{u}}^{2}} < \infty\}$ and $\Xi_{\mathfrak{u}}(r) := \{ x \in \Xi: \Vert x \Vert_{L^{2}_{\mathfrak{u}}} < r\}$.
\assEnd
\end{de}


\bigskip

We have seen that we are interested in estimation methods which are based on the estimation of the Fourier transform of $f$, $\theta^{\circ}$.
In the case of the $L^{2}$-norm, we can see that considering the loss function on $\Theta$ is sufficient to quantify the performance on $\Xi$.
Indeed, let be the $l^{2}$-norm on $\Theta$ defined for any $[x]$ in $\Theta$ by $\Vert [x] \Vert_{l^{2}} = (\sum_{s \in \mathds{Z}} \vert [x](s) \vert^{2})^{1/2}$ and the associated space $\mathcal{L}^{2} = \{[x] \in \Xi: \Vert [x] \Vert_{l^{2}} < \infty\}$.
Given a sequence $[\mathfrak{u}]$ in $\Theta$, we can define the weighted norm which is given, for any $[x]$ in $\Theta$ by $\Vert [x] \Vert_{l^{2}_{[\mathfrak{u}]}} = (\sum_{s \in \mathds{Z}} \vert [x](s) [\mathfrak{u}](s) \vert^{2})^{1/2}$ and for any $r$ in $\R_{+}$ we define the associated space $\Theta({[\mathfrak{u}]}, r) := \{[x] \in \Theta: \Vert [x] \Vert_{l^{2}_{[\mathfrak{u}]}} < r\}$.

The theorem of Plancherel gives us the link between those distances, we have for any $x$ and $\mathfrak{u}$ in $\Xi$ and their Fourier transforms $[x]$ and $[\mathfrak{u}]$ the $\Vert x \Vert^{2}_{L^{2}_{\mathfrak{u}}} = \Vert [x] \Vert_{l_{[\mathfrak{u}]}^{2}}^{2}$.

We hence assume from now on that the parameter of interest has finite norm.

\begin{as}\label{AS_INTRO_FREQ_DECISION_THETAL2}
The parameter of interest $f$ is in $\mathds{L}^{2}$.
\assEnd
\end{as}

This assumption is equivalent to assuming that $\theta^{\circ}$ is in $\mathcal{L}^{2}$.

We shall highlight that this definition has to be adapted under \nref{AS_INTRO_DATA_UNKNOWN} where we obtain $ l : \{\mathds{Y}^{2} \rightarrow \Xi\} \times \mathds{Y}^{2} \times \Xi \rightarrow \R_{+}; \quad (\widehat{f}, Y, \epsilon, f) \mapsto d(\widehat{f}(Y, \epsilon), f)$.

\subsubsection{The risk function $\left(\mathcal{R}_{n} : (\{\mathds{Y} \rightarrow \Theta\} \times \Theta) \rightarrow \R_{+}\right)_{n \in \N}$}\label{INTRO_FREQ_DECISION_RISKFUNCION}
One can notice the the loss function defined previously depends on the observation and, as such, is a random object that cannot be optimised over the choice of estimator.

A way to overcome this limitation is considering a so called risk function such as the expected loss function $\mathcal{R}_{n}(\widehat{f}, f) = \E\left[l(\widehat{f}, Y^{n}, f)\right]$ or $\mathcal{R}_{n, n_{\lambda}}(\widehat{f}, f, h) = \E\left[l(\widehat{f}, Y^{n}, \epsilon^{n_{\lambda}}, f)\right]$ depending on the considered set of assumptions.

The following assumption, which will be verified in every model we consider allows us to obtain interesting upper bounds for the quadratic risk of projection estimators.
\begin{as}\label{as:il}
Assume that there exist constants $V_{1}$ and $V_{2}$ in $\R_{+}^{}\star$ such that, for any $s$ in $\mathds{F}$, we have $V_{1} \leq \V[\langle Y_{0} \vert e_{s} \rangle_{\Xi}] \leq V_{2}$.
In addition assume that there exist constants $V_{3}$, $V_{4}$, and $\cst{4}$ such that $V_{3} \leq \V[\langle \epsilon_{0} \vert e_{s} \rangle_{\Xi}] \leq V_{4}$ and $\ssE^2\FuEx\Vabs{\fedf[(s)]-\hfedf[(s)]}^4\leq\cst{4}$.
\assEnd
\end{as}

This hypothesis allows us to show the following result.

\begin{lm}\label{ge:oSv:re}
If \nref{as:il} holds true, then
\begin{inparaenum} 
\item[\mylabel{ge:oSv:re:ii}{{\dr\upshape(i)}}]
${\FuEx\Vabs{\fedf[(s)]\hfedfmpI[(s)]}^2}\leq 2 V_{4} + 1$;
\item[\mylabel{ge:oSv:re:iii}{{\dr\upshape(ii)}}]
$\P(\Vabs{\hfedfmpI[(s)]}^2<1/\ssE)\leq4V_{4}(1\wedge \iSv[s]/\ssE)$,
\item[\mylabel{ge:oSv:re:iv}{{\dr\upshape(iii)}}] $\FuEx\Vabs{\fedf[(s)]-\hfedf[(s)]}^2\Vabs{\hfedfmpI[(s)]}^2\leq
  2 (\cst{4} + V_{4})(1\wedge \iSv[s]/\ssE)$.
\end{inparaenum}
\end{lm}

\begin{pro}[Proof of \cref{ge:oSv:re}]
  Since $\ssE\E\Vabs{\fedf(s)-\hfedf(s)}^2=\V[\langle \epsilon_{0} \vert e_{s} \rangle_{\Xi}]\leq V_{4}$ we obtain \ref{ge:oSv:re:ii} as follows
\begin{multline*}
  \E\Vabs{\fedf(s)\hfedfmpI}^2\leq 2\E\{\Vabs{\fedf(s)-\hfedf(s)}^2\Vabs{\hfedfmpI}^2+\Ind{\{|\hfedf(s)|^2\geq1/\ssE\}}\}\\
  \leq 2(\ssE\E(\fedf(s)-\hfedf(s))^2+1) \leq 2 V_{4} + 1.
\end{multline*}

Consider \ref{ge:oSv:re:iii}. Trivially, for any $s\in\Nz$ we
have $\P(|\hfedf(s)|^2<1/\ssE)\leq 1$. If $1\leq
4  V_{4} \ssE^{-1}|\fedf(s)|^{-2})=4V_{4} \ssE^{-1}\iSv[s]$, then obviously
$\P(|\hfedf(s)|^2<\ssE^{-1})\leq\min
(1,4V_{4} \ssE^{-1}\iSv[s])$. Otherwise, we have $\ssE^{-1}<
|\fedf(s)|^2/(4V_{4})$
and hence using Tchebychev's inequaltiy,
\begin{multline*}
\P(|\hfedf(s)|^2<\ssE^{-1})\leq
\P(|\hfedf(s)-\fedf(s)|>|\fedf(s)|/(2\sqrt{V_{4}}))\leq 4\iSv[s]\E\Vabs{\fedf(s)-\hfedf(s)}^2\\\leq4 V_{4} \ssE^{-1} \iSv[s] = \min(1,4V_{4} \ssE^{-1} \iSv[s])  
\end{multline*}
 where we have used again that $\ssE\E\Vabs{\fedf(s)-\hfedf(s)}^2\leq1$. Combining both cases we obtain \ref{ge:oSv:re:iii}.
 
 Consider
 \ref{ge:oSv:re:iv}. Due to \nref{as:il} there is a numerical constant $\cst{4}$ such that
$\ssE^2\E|\fedf(s)-\hfedf(s)|^4\leq \cst{4}$ 
, which in turn implies
\begin{multline*}
 \E\Vabs{\fedf(s)-\hfedf(s)}^2\Vabs{\hfedfmpI}^2\leq \E\Big\{\Vabs{\fedf(s)-\hfedf(s)}^2\Vabs{\hfedfmpI}^22\big[\frac{\Vabs{\fedf(s)-\hfedf(s)}^2}{|\fedf(s)|^2}+\frac{|\hfedf(s)|^2}{|\fedf(s)|^2}\big]\Big\}\\
\leq
\frac{2\ssE\E\Vabs{\fedf(s)-\hfedf(s)}^4}{|\fedf(s)|^2}+\frac{2\E\Vabs{\fedf(s)-\hfedf(s)}^2}{|\fedf(s)|^2}\leq
2(\cst{4} + V_{4}) \ssE^{-1} \iSv[s]. 
\end{multline*}
Combining the last bound and $\E\Vabs{\fedf(s)-\hfedf(s)}^2|\hfedfmpI|^2\leq\ssE\E\Vabs{\fedf(s)-\hfedf(s)}^2\leq V_{4}$  implies \ref{ge:oSv:re:iv},
which completes the proof.
\proEnd
\end{pro}

In addition we will use the following notations.

\begin{nota}
For any $m$ in $\mathds{M}$; $s$ in $\mathds{F}$; and $\theta$ in $\Theta$, let be the following quantities:
\begin{multline*}
\b_{m}^{2}(\theta) := \Vert \theta_{\underline{0}}\Vert_{l^{2}}^{-2} \Vert \theta_{\underline{m}} \Vert^{2}; \quad \Lambda(s) = \vert \lambda^{-1}(s) \vert^{2};\\
\Lambda_{\circ}(m) = m^{-1} \sum\nolimits_{0 < s \leq m} \Lambda(s); \quad \Lambda_{+}(m) := \max\nolimits_{s \in \mathds{F}_{m}}\{\Lambda(s)\}.
\end{multline*}
\assEnd
\end{nota}
Notice that, if $\mathds{F} = \Z$, then $\sum_{s \in \mathds{F}_{m}} \Lambda(s) = 2 m \Lambda_{\circ}(m) + \Lambda(0)$ and if $\mathds{F} = \N^{\star}$ then $\sum_{s \in \mathds{F}_{m}} \Lambda(s) = m \Lambda_{\circ}(m)$.
So in both case we will write $\sum_{s \in \mathds{F}_{m}} \Lambda(s) = \cst{} m \Lambda_{\circ}(m)$.

\begin{ex}{\textsc{Projection estimator} \\}\label{EX_INTRO_FREQ_DECISION_RISKFUNCTION_MSEPROJ}
If one considers a projection estimator, as in \nref{INTRO_FREQ_PROJEST}, one can carry the following computations out for any $m$ in $\mathds{M}$,
\[ \mathcal{R}_{n}(\theta_{n, \overline{m}}, \theta, \Lambda) = \E\left[\Vert \theta_{n, \overline{m}} - \theta \Vert_{l^{2}}^{2} \right] = \sum\nolimits_{s \in \mathds{F}} \V\left[ \theta_{n, \overline{m}}(s) \right] + \vert \E\left[ \theta_{n, \overline{m}}(s) \right] - \theta(s) \vert^{2}.\]

Them, under \nref{as:il}, the quadratic risk can be simplified, depending on the set of assumptions we accept:
\begin{itemize}
\item under \nref{AS_INTRO_DATA_KNOWN} and \nref{AS_INTRO_DATA_INDEPENDENT}
\begin{multline*}
\mathcal{R}_{n}(\theta_{n, \overline{m}}, \theta, \Lambda) = n^{-1} \sum\nolimits_{s \in \mathds{F}_{m}} \Lambda(s) \V\left[ \langle Y_{0} \vert e_{s} \rangle_{L^{2}} \right] + \Vert \theta_{\underline{0}}\Vert_{l^{2}}^{2}\mathfrak{b}_{m}^{2}(\theta)\\
\leq n^{-1} V_{2} \cst{} m \Lambda_{\circ}(s)  + \Vert \theta_{\underline{0}}\Vert_{l^{2}}^{2}\mathfrak{b}_{m}^{2}(\theta) \leq (V_{2} \cst{} + \Vert \theta^{\circ}_{\underline{0}} \Vert_{l^{2}}^{2})[n^{-1} m \Lambda_{\circ}(m) \vee \b_{m}^{2}(\theta^{\circ})];
\end{multline*}
but also
\begin{multline*}
\mathcal{R}_{n}(\theta_{n, \overline{m}}, \theta, \Lambda) \geq n^{-1} V_{1} \cst{} m \Lambda_{\circ}(s)  + \Vert \theta_{\underline{0}}\Vert_{l^{2}}^{2}\mathfrak{b}_{m}^{2}(\theta) \geq (V_{1} \cst{} \vee \Vert \theta^{\circ}_{\underline{0}} \Vert_{l^{2}}^{2})[n^{-1} m \Lambda_{\circ}(m) \vee \b_{m}^{2}(\theta^{\circ})];
\end{multline*}
\item under \nref{AS_INTRO_DATA_KNOWN} and \nref{AS_MARGINS_INTRO_DATA_REGULAR}, using \nref{LMI_INTRO_DATA_REGULAR},
\begin{multline*}
\mathcal{R}_{n}(\theta_{n, \overline{m}}, \theta, \Lambda) = n^{-2} \sum\nolimits_{s \in \mathds{F}_{m}} \Lambda(s) \V\left[\sum\nolimits_{p = 1}^{n} \langle Y_{p} \vert e_{s} \rangle_{L^{2}} \right] + \Vert \theta_{\underline{0}}\Vert_{l^{2}}^{2}\mathfrak{b}_{m}^{2}(\theta)\\
 \leq (\cst{} (1 + 4 \sum\nolimits_{p = 1}^{\infty} \beta(Y_{0}, Y_{p})) + \Vert \theta_{\underline{0}}\Vert_{l^{2}}^{2})[n^{-1} m \Lambda_{\circ}(s) \vee \mathfrak{b}_{m}^{2}(\theta)];
\end{multline*}
\item under \nref{AS_INTRO_DATA_UNKNOWN} and \nref{AS_INTRO_DATA_INDEPENDENT}, start by noticing that, as for any $s$ in $\mathds{F}$, we have $(1 \wedge \Lambda(s)) \leq 1$ and that $\theta^{\circ}$ is square summable, we have, $\sum_{s \in \mathds{F}} \vert \theta^{\circ}(s) \vert^{2} (1 \wedge n_{\lambda}^{-1} \Lambda(s)) < \infty$. Hence, using \nref{ge:oSv:re}, we may write,
\begin{multline*}
  \mathcal{R}_{n, \ssE}(\theta_{n, n_{\lambda}, \overline{m}}, \theta, \Lambda) = \sum\nolimits_{s \in \mathds{F}_{m}} \Lambda(s) \left(\V\left[ \phi_{n}(s) \right] \E\left[ \vert \lambda_{n_{\lambda}}^{+}(s) \lambda(s) \vert^{2} \right]\right) + \Vert \theta_{\underline{0}}\Vert_{l^{2}}^{2}\mathfrak{b}_{m}^{2}(\theta)\\
  + \sum\nolimits_{s \in \mathds{F}_{m}} \vert \theta(s) \vert^{2} \E \left[ \left\vert \lambda_{n_{\lambda}}^{+}(s) \right\vert^{2} \left\vert \lambda(s) - \lambda_{n_{\lambda}}(s)\right\vert^{2} \right] + \sum\nolimits_{s \in \mathds{F}_{m}} \vert \theta(s) \vert^{2} \P( \{\vert \lambda_{n_{\lambda}}(s) \vert^{2} < n_{\lambda}^{-1}\})\\
  \leq (V_{2} \cst{} + \Vert \theta_{\underline{0}}\Vert_{l^{2}}^{2}) [n^{-1} m \Lambda_{\circ}(m) \vee \mathfrak{b}_{m}^{2}(\theta)] + 2 \cst{} (\cst{4} + 3 V_{4}) \sum_{s \in \mathds{F}}\vert \theta(s) \vert^{2} (1 \wedge n_{\lambda}^{-1} \Lambda(s)).
\end{multline*}
\end{itemize}
\remEnd
\end{ex}

\begin{nota}\label{rates}
In particular, we denote in the following way the risk for projection estimators:
\begin{multline*}
\mathcal{R}_{n}^{m}(\theta^{\circ}, \Lambda) := [n^{-1} m \Lambda_{\circ}(m) \vee \b_{m}^{2}(\theta^{\circ})]; \quad \mathcal{R}_{n_{\lambda}}^{\dagger}(\theta^{\circ}, \Lambda) := \sum_{s \in \mathds{F}}\vert \theta(s) \vert^{2} (1 \wedge n_{\lambda}^{-1} \Lambda(s)).
\end{multline*}
\end{nota}
The risk function hence allows us to quantify the performance of an estimator independently of the random observation.
Alternatively, one can consider the probability to exceed a certain loss.

\begin{de}\label{DE_INTRO_FREQ_DECISION_RISKFUNCTION_THRESHOLDOVERCOME}
We define the sequence of functions
\[\mathfrak{R}_{n} : (\mathds{Y} \rightarrow \Xi) \times \Xi \times \R_{+} \rightarrow \R_{+}; \quad (\widehat{f}, f, a) \mapsto \P_{f}^{n}\left(l(\widehat{f}, Y, f) \geq a \right).\]
\assEnd
\end{de}

In general, one is interested in the asymptotic behaviour of $\mathcal{R}$ or $\mathfrak{R}$ (and then replacing $a$ by a sequence $(a_{n})_{n \in \N}$) when $n$ tends to infinity.
In particular, for a given estimator $\widehat{f}$ and a fixed value $f$ of the parameter of interest, the sequence $\mathcal{R}_{n}(\widehat{f}, f)$ is called convergence rate of $\widehat{f}$ at $f$ and if $\mathfrak{R}_{n}(\widehat{f}, f, a_{n})$ tends to $0$ as $n$ tends to infinity, $a_{n}$ is called speed of convergence in probability of $\widehat{f}$ at $f$.
If this sequence tends to zero, the estimator is called consistent.

\medskip

While it is technically feasible to minimise the risk function over $\widehat{f}$ for each $f$, the result will be discountenancing as the minimisers will invariably be functions almost surely equal to $f$ itself which brilliantly yields a loss function equal to $0$, independently of the observation and hence a risk function equal to $0$.
Our goal being to estimate $f$, it is obvious that such an estimator is not at hand.

We are interested in this thesis in two formulations of optimality which allow to overcome this limitation.

\subsubsection{Oracle optimality}\label{INTRO_FREQ_DECISION_ORACLEOPT}
Consider $\mathcal{E}$, a family of estimators and a risk function $\mathcal{R}$.

\begin{de}\label{DE_INTRO_FREQ_DECISION_ORACLEOPT_CONVRATE}
A sequence of functions $\left(\mathcal{R}_{\mathcal{E}, n} : \Xi \rightarrow \R_{+}\right)_{n \in \N}$ is called oracle risk for the family of estimators $\mathcal{E}$ if there exist a constant $C$ in $[1, \infty[$ such that, for any $f$ in $\Xi$, and all $n$, we have: $\mathcal{R}_{\mathcal{E}, n}(f) \leq C \cdot \inf\limits_{\widehat{f} \in \mathcal{E}} \mathcal{R}_{n}(\widehat{f}, f)$, or, depending on the considered set of assumptions, $\mathcal{R}_{\mathcal{E}, n, n_{\lambda}}(f, h) \leq C \cdot \inf\limits_{\widehat{f} \in \mathcal{E}} \mathcal{R}_{n, n_{\lambda}}(\widehat{f}, f, h)$.
\assEnd
\end{de}

\begin{de}\label{DE_INTRO_FREQ_DECISION_ORACLEOPT_EXACTCONVRATE}
A sequence of functions $\mathcal{R}^{\circ}_{\mathcal{E}, n} : \Xi \rightarrow \R_{+}$ is called exact oracle convergence rate for the family of estimators $\mathcal{E}$ if, in addition to being an oracle convergence rate, there exists an element $\widehat{f}$ of $\mathcal{E}$ such that for any $f$ in $\Xi$ and $n$ in $\N$ we have: $\mathcal{R}^{\circ}_{\mathcal{E}, n}(f) \geq C^{-1} \cdot \mathcal{R}_{n}(\widehat{f}, f)$ or $\mathcal{R}^{\circ}_{\mathcal{E}, n, n_{\lambda}}(f, h) \geq C^{-1} \cdot \mathcal{R}_{n, n_{\lambda}}(\widehat{f}, f, h)$ depending on the type of data at hand.
An estimator such as $\widehat{f}$ is called oracle optimal.
\assEnd
\end{de}

We see that those definitions are "up to a constant" and we will in general be more interested in the asymptotic rate as $n$ and/or $n_{\lambda}$ tend to infinity and we hence introduce the following notations.

\begin{nota}
Let be $(a_{n})_{n \in \N}$ a sequence of elements of $\mathds{K}$.
We define the sets $\mathfrak{o}_{n}(a) := \{b \in \mathds{K}^{\N}: \lim_{n \rightarrow \infty}\vert b_{n}/a_{n} \vert = 0\}$; and $\mathcal{O}_{n}(a) := \{b \in \mathds{K}^{\N}: \exists C \in \R_{+} \lim_{n \rightarrow \infty}\vert b_{n}/a_{n} \vert \leq C\}$.
If $a \in \mathcal{O}(b)$ and $b \in \mathcal{O}(a)$ then we denote $a \approx b$.

On the other hand we also define the sets $\mathfrak{o}_{\P}(a)$ and $\mathcal{O}_{\P}(a)$ as the sets of sequences of probability distributions $\P_{n}$ on $\mathds{K}$ such that, if $(X_{n})_{n \in \N}$ is a sequence of $\mathds{K}$-valued random variables verifying $X_{n} \sim \P_{n}$, then we have,
\[ \P_{n} \in \mathfrak{o}_{\P}(a) \iff \forall \epsilon \in \R_{+}^{\star}, \quad \lim_{n \rightarrow \infty} \P(\vert X_{n} / a_{n} \vert \geq \epsilon) = 0\]
\[ \P_{n} \in \mathcal{O}_{\P}(a) \iff \forall \epsilon \in \R_{+}^{\star}, \exists M \in \R_{+}, N \in \N: \quad \forall n > N, \P(\vert X_{n} / a_{n} \vert \geq M) \leq \epsilon\]
\assEnd
\end{nota}

In particular, throughout this thesis, we shall distinguish the following two cases for $\fxdf$, respectively called parametric and non-parametric which commonly lead to very different behaviour of the optimal rates:
\begin{Liste}[]
\item[\mylabel{oo:xdf:p}{\dgrau\bfseries{(p)}}] there exist a finite $K$ of $\mathds{F}$ such that, for any $K'$ smaller than $K$, $\b_{K'}^{2}(\theta^{\circ}) > 0$ and $\b_{K}^{2}(\theta^{\circ}) = 0$;
\item[\mylabel{oo:xdf:np}{\dgrau\bfseries{(np)}}] for all finite $K$ in $\mathds{F}$, $\b_{K}(\theta^{\circ})>0$.
\end{Liste}

Note that the Fourier series expansion of the function of interest $f$ is, in case \ref{oo:xdf:p}, \textit{finite}, i.e., $f=\sum_{s \in \mathds{F}_{K}}\theta^{\circ}(s)e_{s}$ for some finite $K$ in $\N$ while in the opposite case
\ref{oo:xdf:np}, it is \textit{infinite}, i.e., not finite.

\begin{il}\label{IL_INTRO_FREQ_DECISION}
The upper bounds we give will be discussed in such "numerical discussions" where we consider the following typical behaviours of $\fxdf$ and $\fedf$ and give an equivalent to the upper bound in terms of an explicit function of $n$.

Regarding the operator eigen-values $\fedf$, we consider the following two cases, respectively called ordinary smooth and super-smooth:
\begin{Liste}[]
\item[\mylabel{il:edf:o}{\dg\bfseries{(o)}}] there exists a strictly positive real number $a$ such that $\Lambda(m) \approx m^{2a}$, then $m \Lambda_{\circ}(m) \approx m^{2a+1}$ and $\Lambda_{+}(m) \approx m^{2a}$;
\item[\mylabel{il:edf:s}{\dg\bfseries{(s)}}] there exists a strictly positive real number $a$ such that $\Lambda(m) \approx \exp(m^{2a})$, then $m \Lambda_{\circ}(m) \approx m^{-(1-2a)_+}\exp(m^{2a})$ and $\Lambda_{+}(m) \approx \exp(m^{2a})$.
\end{Liste}

For the parameter of interest $\theta^{\circ}$, the behaviours of its tails i.e., $\left(\b_{m}^{2}(\theta^{\circ})\right)_{m \in \mathds{F}} = \Vert \theta^{\circ}_{\underline{0}}\Vert_{l^{2}}^{-2}\Vert \Pi_{\underline{m}} \theta^{\circ} \Vert_{l^{2}}^{2}$ will also be of interest.
We distinguish the cases \ref{oo:xdf:p} and \ref{oo:xdf:np}, and with \ref{oo:xdf:np} distinguish the super smooth and ordinary smooth for the parameter of interest.
\begin{Liste}[]
\item[\mylabel{il:xdf:o}{\dg\bfseries{(o)}}] there exists a strictly positive real number $p$ such that $\vert \theta^{\circ}(s) \vert^{2} \approx s^{-2p -1}$, in this case, we have $\b_{m}^2(\theta^{\circ}) \approx m^{-2p}$;
\item[\mylabel{il:xdf:s}{\dg\bfseries{(s)}}] there exists a strictly positive real number $p$ such that $\vert \theta^{\circ}(s) \vert^{2} \approx s^{2p -1} \exp[-s^{2p}]$, and then we have $\b_{m}^2(\theta^{\circ}) \approx \exp(-m^{2p})$.
\end{Liste}

We consider the following situations: in the cases \begin{inparaenum}[i]
\item[\mylabel{il:po}{\dg\bfseries{[p-o]}}] and \item[\mylabel{il:ps}{\dg\bfseries{[p-s]}}] the parameter of interest has a finite representation \ref{oo:xdf:p} and the operator is either ordinary smooth \ref{il:edf:o} or super smooth \ref{il:edf:s}.
In the cases \item[\mylabel{il:oo}{\dg\bfseries{[o-o]}}] and \item[\mylabel{il:os}{\dg\bfseries{[o-s]}}] the parameter of interest is ordinary smooth \ref{il:xdf:o} and the operator is either ordinary smooth \ref{il:edf:o} or super smooth \ref{il:edf:s}.
Case \item[\mylabel{il:so}{\dg\bfseries{[s-o]}}] is the opposite of case \ref{il:os}.
\end{inparaenum}
\ilEnd
\end{il}

While the names given here to the typical cases may seem arbitrary, we shall justify them through the examples treated in this thesis where the decaying rate of $\fxdf$ and $\fedf$ respectively can be interpreted in terms of function smoothness.

The particular interest for these different cases will also appear natural as the behaviour of the optimal rate will be considerably different in our examples; moreover, this phenomenon is observed in many statistical models, also outside of our field of interest.

We carry on with the projection estimators example.

\textbf{Known operator}

The bound we derived in \nref{rates} depends on the
dimension parameter $\Di$ and hence by selecting an optimal value they
will be minimised, which we formulate next.  For a sequence
$\Nsuite[n]{a_n}$ of real numbers with minimal value in a set
$A\subset{\Nz}$ we set
$\argmin\set{a_n,n\in A}:=\min\{m\in A:a_m\leq a_n,\;\forall n\in A
\}$. For all $n\in\Nz$ we define
\begin{multline}\label{oo:de:nra}
  \dRa{\Di}{\xdf,\Lambda}:=[\bias^2(\xdf)\vee\Di \oiSv \ssY^{-1}]
  :=\max\vectB{\bias^2(\xdf), \Di \oiSv \ssY^{-1}},\\
  \hfill
  \onDi:=\onDi(\xdf,\Lambda):=\argmin\Nset[{\Di\in\Nz}]{\dRa{\Di}{\xdf,\Lambda}}
  \quad\text{ and }\hfill\\
  \oRa{\xdf,\Lambda}:=\dRa{\onDi}{\xdf,\Lambda}=\min\Nset[{\Di\in\Nz}]{\dRa{\Di}{\xdf,\Lambda}}.
\end{multline}

\begin{te}
Consequently, the  rate $\Nsuite[\ssY]{\oRa{\xdf,\Lambda}}$, the dimension parameters $\Nsuite[\ssY]{\onDi}$  and  the projection estimators  $\Nsuite[\ssY]{\txdfPr[\onDi]}$, respectively, is an oracle
rate, an oracle dimension and oracle optimal estimator (up to a constant).
\end{te}

\begin{rmk}\label{RMK_INTRO_FREQ_DECISION_ORACLEOPT_OPTPROJ}\label{oo:rem:ora}
We shall emphasise that $\oRa{\xdf,\Lambda}\geq \ssY^{-1}$ for all
  $\ssY\in\Nz$, and
  
  $\lim_{n \rightarrow \infty} \oRa{\xdf,\Lambda}=0$.
  Observe that for all $\delta>0$ there exists $\Di_{\delta}\in\Nz$ and
  $\ssY_\delta\in\Nz$ such that for all $\ssY\geq \ssY_{\delta}$ holds
  $\bias[\Di_\delta]^2(\xdf)\leq \delta$ and
  $\Di_{\delta} \oiSv[\Di_\delta] \ssY^{-1}\leq\delta$, and whence
  $\oRa{\xdf,\Lambda}\leq\dRa{\Di_\delta}{\xdf,\Lambda}\leq \delta$.
  Moreover, we have $\oDi{\ssY}\in\nset{1,\ssY}$. Indeed, by construction
  holds
  $\bias[\ssY]^2(\xdf)\leq 1<(\ssY+1)\ssY^{-1}\leq
  (\ssY+1)\oiSv[\ssY+1]{}\ssY^{-1}$, and hence
  $\dRa{\ssY}{\xdf,\Lambda}<\dRa{\Di}{\xdf,\Lambda}$ for all
  $\Di\in \nsetro{\ssY+1,\infty}$ which in turn implies the claim
  $\oDi{\ssY}\in\nset{1,\ssY}$. Obviously, it follows thus
  $\oRa{\xdf,\Lambda}=\min\set{ \dRa{\Di}{\xdf,\Lambda}
    ,\Di\in\nset{1,\ssY}}$ for all $\ssY\in\Nz$. We shall use those
  elementary findings in the sequel without further reference.
The sequence $\mathcal{R}_{n}^{\circ}(\theta, \lambda)$ is then an exact oracle convergence rate and the projection estimator $\theta_{n, \overline{m_{n}^{\circ}}}$ is an oracle optimal estimator.
\remEnd
\end{rmk}

\begin{rmk}\label{ge:oo:rem:ora2}
In case \ref{oo:xdf:p}, the oracle rate is parametric, that is
$\oRa{\xdf, \Lambda} \approx \ssY^{-1}$. More precisely, if $\xdf=0$ then
for each  $\Di\in\Nz$,
$\E\Vnormlp{\txdfPr-\xdf}^2=\cst{}\Di\oiSv[\Di]\ssY^{-1}$,
and hence $\oDi{\ssY}=1$ and $\oRa{\xdf, \Lambda}=\oiSv[1]\ssY^{-1}\sim\ssY^{-1}$. Otherwise
if there is $K\in\Nz$  with $\bias[K-1](\xdf)>0$ and
$\bias[K](\xdf)=0$, then setting
$\ssY_{\xdf}:=\tfrac{K\oiSv[K]}{\bias[K-1]^2(\xdf)}$, for all
$\ssY\geq \ssY_{\xdf}$ holds
$\bias[K-1]^2(\xdf)>K\oiSv[K]\ssY^{-1}$, and hence  $\oDi{\ssY}=K$ and
$\oRa{\xdf,\Lambda}= K\oiSv[K]\ssY^{-1}\sim \ssY^{-1}$.
On the other hand side, in case \ref{oo:xdf:np} the oracle rate is
non-parametric, more precisely, it holds
$\lim_{\ssY\to\infty}\ssY\oRa{\xdf,\Lambda}=\infty$. Indeed, since
$\bias[\oDi{\ssY}]^2(\xdf)\leq\oRa{\xdf,\Lambda}=\oRa{\xdf,\Lambda}\in\mathfrak{o}_{n}(1)$ follows $\oDi{\ssY}\to\infty$ and hence
$\oDi{\ssY}\oiSv[\oDi{\ssY}]\to\infty$ which implies the claim because
$\ssY\oRa{\xdf,\Lambda}\geq\oDi{\ssY}\oiSv[\oDi{\ssY}]$.
\end{rmk}

\begin{il}\label{ge:il:oo:kn}
Let us illustrate the rates obtained in the case \ref{oo:xdf:np}.
\begin{Liste}[]
\item[\mylabel{il:oo:oo}{\dg\bfseries{[o-o]}}] 
$\oRa{\xdf,\Lambda}\approx(\oDi{\ssY})^{-2p}\approx (\oDi{\ssY})^{2a+1}\ssY^{-1}$, and hence,
    $\oDi{\ssY}\approx \ssY^{1/(2p+2a+1)}$ and $\oRa{\xdf,\Lambda}\approx\ssY^{-2p/(2p+2a+1)}$
\item[\mylabel{il:oo:os}{\dg\bfseries{[o-s]}}]
$\oRa{\xdf,\Lambda}\approx(\oDi{\ssY})^{-2p}\approx (\oDi{\ssY})^{-(1-2a)_+}\exp((\oDi{\ssY})^{2a})\ssY^{-1}$, and hence,\\
    $\oDi{\ssY}\approx (\log\ssY - \tfrac{2p-(1-2a)_+}{2a}\log\log\ssY)^{1/(2a)}$ and $\oRa{\xdf,\Lambda}\approx(\log\ssY)^{-p/a}$.
\item[\mylabel{il:oo:so}{\dg\bfseries{[s-o]}}] 
$\oRa{\xdf,\Lambda}\approx\exp(-(\oDi{\ssY})^{2p})\approx (\oDi{\ssY})^{2a+1}\ssY^{-1}$, and hence,\\
    $\oDi{\ssY}\approx (\log\ssY - \tfrac{2a+1}{2p}\log\log\ssY)^{1/(2p)}$ and $\oRa{\xdf,\Lambda}\approx(\log\ssY)^{(2a+1)/(2p)}\ssY^{-1}$.
\end{Liste}
\ilEnd
\end{il}

\medskip

\textbf{Unknown operator}

Let us remind that we have
\begin{equation*}
\mathcal{R}_{n, n_{\lambda}}(\theta_{n, n_{\lambda}, \overline{m_{n}^{\circ}}}) \leq (V_{2} \cst{} + \Vert \theta^{\circ}_{\underline{0}} \Vert_{l^{2}}^{2})\mathcal{R}_{n}^{\circ}(\theta^{\circ}, \Lambda) + 2 \cst{} \mathcal{R}_{n_{\lambda}}^{\dagger}(\theta^{\circ}, \Lambda)
\end{equation*}
We note that $\Vnormlp{\xdf_{\underline{0}}}^2=0$ implies
  $\mRa{\xdf,\Lambda}=0$, while for $\Vnormlp{\xdf_{\underline{0}}}^2>0$ holds
  $\mRa{\xdf,\Lambda}\geq \sum_{s:\iSv[s]>\ssE} \vert \fxdf[(s)] \vert ^2+\ssE^{-1}\sum_{s:\iSv[s]\leq\ssE} \vert \fxdf[(s)] \vert ^2  \geq\ssE^{-1}\sum_{s\in\Nz} \vert \fxdf[(s)] \vert ^2=\cst{}\Vnormlp{\xdf_{\underline{0}}}^2
  \ssE^{-1}$, thereby whenever $\xdf\ne 0$
  any additional term of order $\ssY^{-1}+\ssE^{-1}$
  is negligible with respect to the rate
  $\oRa{\xdf,\Lambda}+\mRa{\xdf,\Lambda}$, since
  $\oRa{\xdf,\Lambda}\geq \ssY^{-1}$, 
  which we will use below without further reference. We shall
  emphasise that in case $\ssY=\ssE$ it holds
  \begin{multline}\label{ge:oo:e7}
    \mRa[\ssY]{\xdf,\Lambda}=\sum\nolimits_{s\in \mathds{F}_{m_{n}^{\circ}}} \vert \fxdf[(s)] \vert^2 [1 \wedge \ssY^{-1}\iSv[s]]
    + \sum\nolimits_{s\in \mathds{F}_{m_{n}^{\circ}}^{c}} \vert \fxdf[(s)] \vert ^2[1\wedge\ssY^{-1}\iSv[s]]\\
    \leq \cst{}\Vnormlp{\xdf_{\underline{0}}}^2 \ssY^{-1} \oDi{\ssY}
    \oiSv[\oDi{\ssY}] +
    \cst{}\Vnormlp{\xdf_{\underline{0}}}^2\bias[\oDi{\ssY}]^2(\theta^{\circ})\leq
    \Vnormlp{\xdf_{\underline{0}}}^2\dRa{\oDi{\ssY}}{\xdf,\Lambda}
  \end{multline}
  which in turn implies $\mathcal{R}_{n, n_{\lambda}}(\theta_{n, n_{\lambda}, \overline{m_{n}^{\circ}}}) \leq (V_{2} \cst{} + (1 + 2 \cst{})\Vert \theta^{\circ}_{\underline{0}} \Vert_{l^{2}}^{2})\mathcal{R}_{n}^{\circ}(\theta^{\circ}, \Lambda)$.
  In other words, the estimation of the unknown operator $T$ is negligible whenever $\ssY\leq\ssE$.

\begin{rmk}
We note that in case \ref{oo:xdf:p}
$\mRa{\xdf,\Lambda}\leq
\Vnormlp{\xdf_{\underline{0}}}^2\miSv[K]\ssE^{-1}$
and hence
\begin{equation}\label{oo:au:p}
 \nmRi{\hxdfPr[\oDi{\ssY}]}{\xdf,\Lambda}
 % \FuEx\Vnormlp{\hxdfPr[\oDi{\ssY}]-\xdf}^2
 \leq
\cst{}\{[1\vee\Vnormlp{\xdf_{\underline{0}}}^2]\{
K\oiSv[K]\ssY^{-1}+\miSv[K]\ssE^{-1}\}
\end{equation}
for all $\ssE\in\Nz$ and $\ssY\geq\ssY_{\xdf}$ with $\ssY_{\xdf}$ as in \nref{oo:rem:ora}. In other words the
rate is parametric in both the $\rE$-sample size $\ssE$ and the $\rY$-sample size $\ssY$. Thereby, the  additional estimation of the operator is negligible whenever $\ssE\geq\ssY$.
In the opposite case \ref{oo:xdf:np}, it is obviously of interest to characterise the minimal size $\ssE$ of the additional sample from $\rE$ needed to attain the same rate as in case of a known operator.
Thus, in the next illustration we let the $\rE$-sample size depend on the $\rY$-sample size $\ssY$ as well. 
\remEnd
\end{rmk}

 Let us now briefly illustrate the rates we already defined by stating the
 order of $\oDi{\ssY}(\xdf,\Lambda)$ and $\oRa{\xdf,\Lambda}$ for the cases introduced in \nref{IL_INTRO_FREQ_DECISION}.
% ....................................................................
% <<Il upper bound oo>>
% ....................................................................
\begin{il}\label{ge:il:oo:uk}
\begin{Liste}[]
\item[\mylabel{au:il:oo:oo}{\dg\bfseries{[o-o]}}]
For $p>a$ holds
  $\sum_{s\in\Nz}|\fxdf[(s)]|^2\iSv[s]<\infty$, and hence
  $\moRaE\approx \ssE^{-1}$, while for $p=a$ and $p<a$ holds
  $\sum_{s=1}^{\Di}|\fxdf[(s)]|^2\iSv[s]\approx\log(\Di)$ and
  $\sum_{s=1}^{\Di}|\fxdf[(s)]|^2\iSv[s]\approx\Di^{2(a-p)}$, respectively.
  For $p\leq a$ with $\Di_{\ssE}:=\floor{\ssE^{1/2a}}$ it follows
  $\moRaE\approx\ssE^{-1}\sum_{s\in\nset{1,\Di_{\ssE}}}\iSv[(s)]|\fxdf[(s)]|^2+\bias[\Di_{\ssE}](\xdf)^2$,
  and thereby, 
   $\moRaE\approx\log(\ssE)\ssE^{-1}$ for $p=a$, while 
  $\moRaE\approx\ssE^{-p/a}$ for $p<a$.
\item[\mylabel{au:il:oo:os}{\dg\bfseries{[o-s]}}]
Since
  $\sum_{s=1}^{\Di}|\fxdf[(s)]|^2\iSv[s]\approx\Di^{-2p-1}\iSv[\Di]$  
  the decomposition  in \ref{au:il:oo:oo} with $\Di_{\ssE}:=\floor{(\log\ssE)^{1/(2a)}}$ implies $\moRaE\approx(\log\ssE)^{-p/a}$.
\item[\mylabel{au:il:oo:so}{\dg\bfseries{[s-o]}}] Since
  $\sum_{s\in\Nz}|\fxdf[(s)]|^2\iSv[s]<\infty$ it follows $\moRaE\approx \ssE^{-1}$.
\end{Liste}
\ilEnd
\end{il}

We see that, given a family of estimators, oracle optimality defines the best element of this family.
However, this requires to restrict ourselves to a family of estimator.

\subsubsection{Minimax optimality}\label{INTRO_FREQ_DECISION_MINIMAXOPT}

An alternative to oracle optimality is minimax optimality.

\begin{de}\label{DE_INTRO_FREQ_DECISION_MINIMAXOPT_MAXRATE}
Considering a subset $\widetilde{\Xi}$ of $\Xi$, and an estimator $\widetilde{f}$, we call "maximal convergence rate of $\widetilde{f}$ over $\widetilde{\Xi}$" the sequence indexed by n defined by $\mathcal{R}_{n}(\widetilde{f}, \widetilde{\Xi}, \Lambda) := \sup\limits_{f \in \widetilde{\Xi}} \mathcal{R}_{n}(\widetilde{f}, f).$
Alternatively, if the operator is unknown, we denote $\widetilde{\Xi}$ and $\widetilde{\mathcal{L}(\Xi)}$ two subsets, respectively of $\Xi$ and $\mathcal{L}(\Xi)$ and we have $\mathcal{R}_{n, n_{\lambda}}(\widetilde{f}, \widetilde{\Xi}, \widetilde{\mathcal{L}(\Xi)}) := \sup\limits_{T \in \widetilde{\mathcal{L}(\Xi)}} \sup\limits_{f \in \widetilde{\Xi}} \mathcal{R}_{n, n_{\lambda}}(\widetilde{f}, f, T)$.
\assEnd
\end{de}
We see here that the maximal convergence rate of an estimator corresponds to its worst case scenario over a set of true parameters.
The idea will be to find an estimator with the best worst case scenario.

\begin{de}\label{DE_INTRO_FREQ_DECISION_MINIMAXOPT_OPTRATE}
Considering a subset $\widetilde{\Xi}$ of $\Xi$, a sequence $\mathcal{R}_{n}^{\star}(\widetilde{\Xi}, \Lambda)$ is called minimax convergence rate if there exist a constant $C$ greater than $1$ such that, for any $n$ in $\N$ 
$\mathcal{R}_{n}^{\star}(\widetilde{\Xi}, \Lambda) \leq C \cdot \inf\nolimits_{\widetilde{f} \in \left\{\mathds{Y} \rightarrow \Xi \right\}} \mathcal{R}_{n}(\widetilde{f}, \widetilde{\Xi}, \Lambda)$ where the infimum is taken over all possible estimator.

Moreover, $\mathcal{R}_{n}^{\star}(\widetilde{\Xi}, \Lambda)$ is called minimax optimal convergence rate if there exists some estimator $\widehat{f}$ such that $\mathcal{R}_{n}^{\star}(\widetilde{\Xi}, \Lambda) \geq C^{-1} \cdot \mathcal{R}_{n}(\widehat{f}, \widetilde{\Xi}, \Lambda).$
An estimator such as $\widehat{f}$ is called minimax optimal.
\assEnd
\end{de}
In this definition, be aware that the infimum is taken over all possible estimator of $f$.

An example of space which we use in this thesis as $\widetilde{\Theta}$ are Sobolev's ellipsoids which we already introduced informally previously.
\begin{de}\label{DE_INTRO_FREQ_DECISION_MINIMAXOPT_SOBOL}
Given a constant $r$ in $\R_{+}$, and a positive, decreasing sequence of numbers smaller than $1$, $\left(\mathfrak{a}(s)\right)_{s \in \mathds{F}}$, we define the Sobolev's ellipsoid $\Theta(\mathfrak{a}, r)$ by $\Theta(\mathfrak{a}, r) := \left\{\theta \in \Theta: \Vert \theta \Vert_{\mathfrak{a}} \leq r \right\}$.
\assEnd
\end{de}

Those ellipsoid are interesting as they can directly be related to classes of regularity for the counterpart space $\Xi$.

We now carry on with the projection estimator example.

\textbf{Known operator}

While considering projection estimators, in the case where the operator is known, we may emphasise that for all $\Di\in\Nz^{\star}$ and  any $\xdf\in\rwCxdf$, $\Vnormlp{\xdf_{\underline{0}}}^2\bias^2(\xdf)=\Vnormlp{\xdf_{\underline{\Di}}}^2=\sum_{ \vert s \vert >\Di}(\xdfCw[(s)]^2/\xdfCw[(s)]^2)\fxdf[(s)]^2\leq
\xdfCw[(\Di)]^2\Vnorm[1/{\xdfCw[]}]{\xdf_{\underline{\Di}}}^2\leq
\xdfCw[(\Di)]^2\xdfCr^2$ which we use in the sequel
without further reference.
It follows for all $\Di,\ssY\in\Nz$ that 
  \begin{multline}\label{oo:e4}
\nRi{\txdfPr}{\rwCxdf,\Lambda}:=\sup\set{\nRi{\txdfPr}{\xdf,\Lambda},\xdf\in\rwCxdf}\\
\leq (2+\xdfCr^2)\max\big(\xdfCw^2,\Di \oiSv \ssY^{-1}\big).
\end{multline}
The upper bound in the last display depends on the dimension parameter
$\Di$ and hence by choosing an optimal value $\mnDi$ the upper bound
will be minimised which we formulate next. For all $n\in\Nz$ we define
\begin{multline}
  \label{mm:de:nra}
 \dRa{\Di}{\xdfCw[],\Lambda}:=[\xdfCw^2\vee\Di\oiSv \ssY^{-1}]:=\max\big(\xdfCw^2,\Di \oiSv \ssY^{-1}\big),\\
\hfill \mnDi(\xdfCw[]):=\mnDi(\xdfCw[],\Lambda):=\argmin\Nset[{\Di\in\Nz}]{\dRa{\Di}{\xdfCw[],\Lambda}}\quad\text{ and }\hfill\\\mnRa{\xdfCw[],\Lambda}:=\dRa{\mnDi({\xdfCw[]})}{\xdfCw[],\Lambda}=\min\Nset[{\Di\in\Nz}]{\dRa{\Di}{\xdfCw[],\Lambda}}.
\end{multline}
From \eqref{oo:e4} we deduce that
$\nRi{\txdfPr[\mnDi({\xdfCw[]})]}{\rwCxdf,\Lambda}\leq(2+\xdfCr^2)\mnRa{\xdfCw[],\Lambda}$ for
all $n\in\Nz$. On the other
  hand side, for example, \ncite{JohannesSchwarz2013a} have shown  that
  $\inf_{\widetilde{\theta}}\nRi{\widetilde{\theta}}{\rwCxdf,\Lambda}$, where the infimum is taken over all
  possible estimators $\widetilde{\theta}$ of $\xdf$, is up to a constant bounded
  from below by $\mnRa{\xdfCw[],\Lambda}$.  Consequently, the rate
  $\Nsuite[n]{\mnRa{\xdfCw[],\Lambda}}$, the dimension parameters $\Nsuite[n]{\mnDi(\xdfCw[])}$
  and the projection estimators $\Nsuite[n]{\txdfPr[\mnDi({\xdfCw[]})]}$, respectively, is a
  minimax rate, a minimax dimension and minimax optimal (up to a
  constant).

% ....................................................................
% Bemerkung mm
% ....................................................................
\begin{rmk}\label{oo:rem:mra}
By construction it holds 
$\mnRa{\xdfCw[],\Lambda}\geq \ssY^{-1}$ for all $\ssY\in\Nz$.
The following statements can be
shown using the same arguments as in \nref{oo:rem:ora}
by exploiting that the sequence $\xdfCw[]$ is assumed to be
non-increasing, strictly positive with limit zero and $\xdfCw[(1)]=1$. 
Thereby, we conclude that 
$\mnRa{\xdfCw[],\Lambda}=\mathfrak{o}_{n}(1)$ and $\ssY\mnRa{\xdfCw[],\Lambda}\to\infty$ as well as 
$\mnDi(\xdfCw[])\in\nset{1,\ssY}$ for all $\ssY\in\Nz$. It follows also that
$\mnDi(\xdfCw[])=\argmin\Nset[{\Di\in\nset{1,n}}]{\dRa{\Di}{\xdfCw[],\Lambda}}$ and 
$\mnRa{\xdfCw[],\Lambda}=\min\Nset[{\Di\in\nset{1,n}}]{\dRa{\Di}{\xdfCw[],\Lambda}}$ for all
$\ssY\in\Nz$. We shall stress that in this situation the rate
$\mnRa{\xdfCw[],\Lambda}$ is non-parametric. \remEnd
\end{rmk}


Let us now briefly illustrate the last definitions by stating the
order of $\mnDi(\xdfCw[],\Lambda)$ and $\mnRa{\xdfCw[],\Lambda}$ for typical choices of the
sequence $\xdfCw[]$.

% ....................................................................
% <<Il upper bound oo>>
% ....................................................................
\begin{il}\label{il:mm}We will illustrate all our results considering
  the following two configurations for the sequence $\xdfCw[]$. Let 
\begin{Liste}[]
\item[\mylabel{il:mm:o}{\dg\bfseries{(o)}}] $\xdfCw^2\approx
  \Di^{-2p}$ with $p>1$;
\item[\mylabel{il:mm:s}{\dg\bfseries{(s)}}] $\xdfCw^2\approx
  \exp(-\Di^{2p})$ with $p>0$.
\end{Liste}
We consider as in \nref{il:oo} the situations \ref{il:oo}, \ref{il:os}
and \ref{il:so}. 
\begin{Liste}[]
\item[\mylabel{il:mm:oo}{\dg\bfseries{[o-o]}}] 
$\dRa{\mnDi}{\xdfCw[],\Lambda}\approx(\mnDi)^{-2p}\approx (\mnDi)^{2a+1}\ssY^{-1}$, and hence,

    $\mnDi({\xdfCw[]})\approx \ssY^{1/(2p+2a+1)}$ and $\mnRa{\xdfCw[],\Lambda}\approx\ssY^{-2p/(2p+2a+1)}$
\item[\mylabel{il:mm:os}{\dg\bfseries{[o-s]}}]
$\dRa{\mnDi}{\xdfCw[],\Lambda}\approx(\mnDi)^{-2p}\approx (\mnDi)^{-(1-2a)_+}\exp((\mnDi)^{2a})\ssY^{-1}$, and hence,\\
    $\mnDi({\xdfCw[]})\approx (\log\ssY - \tfrac{2p-(1-2a)_+}{2a}\log\log\ssY)^{1/(2a)}$ and $\mnRa{\xdfCw[],\Lambda}\approx(\log\ssY)^{-p/a}$.
\item[\mylabel{il:mm:so}{\dg\bfseries{[s-o]}}] 
$\dRa{\mnDi}{\xdfCw[],\Lambda}\approx\exp(-(\mnDi)^{2p})\approx (\mnDi)^{2a+1}\ssY^{-1}$, and hence,\\
    $\mnDi({\xdfCw[]})\approx (\log\ssY - \tfrac{2a+1}{2p}\log\log\ssY)^{1/(2p)}$ and $\mnRa{\xdfCw[],\Lambda}\approx(\log\ssY)^{(2a+1)/(2p)}\ssY^{-1}$.\ilEnd
\end{Liste}
\end{il}

\medskip

\textbf{Unknown operator}

Consider now the case where the operator is unknown.
For all $\ssE\in\Nz$ we define
\begin{equation}\label{oo:de:mra}
  \mmRa{\xdfCw[],\Lambda}:=\max_{s\in\Nz}\{\xdfCw[(s)]^2[1\wedge \iSv[s]/\ssE]\}.
\end{equation}
then for all $\ssE\in\Nz$ holds
$\sup_{\xdf\in\rwCxdf}\mmRa{\xdf,\Lambda}\leq
\xdfCr^2\mmRa{\xdfCw[],\Lambda}$, since for all
$\xdf\in\rwCxdf$ 
\begin{equation}\label{oo:e8}
  \mmRa{\xdf,\Lambda}=\sum_{s\in\Nz} \vert \fxdf[(s)] \vert ^2[1\wedge \iSv[s]/\ssE]\leq
\max_{s\in\Nz}\{\xdfCw[(s)]^2\min(1,\iSv[s]/\ssE)\}\Vnorm[1/{\xdfCw[]}]{\xdf}^2.
\end{equation}
It follows for all $\Di,\ssY,\ssE\in\Nz$ immediately that 
\begin{equation}\label{oo:e9}
  \nmRi{\hxdfPr}{\rwCxdf,\Lambda}
  \leq (\xdfCr^2+8) \dRa{\Di}{\xdfCw[],\Lambda}+8(\cst{4}+1)\xdfCr^2\mmRa{\xdfCw[],\Lambda}.
\end{equation}
The upper bound in the last display depends on the dimension parameter
$\Di$ and hence by choosing an optimal value $\mnDi$ as in
\eqref{mm:de:nra} the upper bound
will be minimised, that is
\begin{equation}\label{oo:e10}
  \nmRi{\hxdfPr[\mnDi]}{\rwCxdf,\Lambda}
  \leq (\xdfCr^2+8) \mnRa{\xdfCw[],\Lambda}+8(\cst{4}+1)\xdfCr^2\mmRa{\xdfCw[],\Lambda}.
\end{equation}

% ....................................................................
% <<Il upper bound oo>>
% ....................................................................
\begin{il}\label{au:il:mm}
  Consider as in \nref{il:mm} the usual
  behaviours \ref{il:mm:oo}, \ref{il:mm:os} and \ref{il:mm:so} for the
  sequences $\Nsuite[\Di]{\xdfCw}$ and $\Nsuite[\Di]{\iSv[\Di]}$,
  where we have derived in  \nref{il:mm} the corresponding minimax
  rates $\Nsuite[\ssY]{\mnRa{\xdfCw[],\Lambda}}$, while for  the rate
  $\Nsuite[\ssE]{\mmRa{\xdfCw[],\Lambda}}$ we get:
\begin{Liste}[]
\item[\mylabel{au:il:mm:oo}{\dg\bfseries{[o-o]}}]  
$\mmRa{\xdfCw[],\Lambda}\approx\ssE^{-(p\wedge a)/a}$
\item[\mylabel{au:il:mm:os}{\dg\bfseries{[o-s]}}]
 $\mmRa{\xdfCw[],\Lambda}\approx(\log\ssE)^{-p/a}$.
\item[\mylabel{au:il:mm:so}{\dg\bfseries{[s-o]}}] 
 $\mmRa{\xdfCw[],\Lambda}\approx \ssE^{-1}$.\ilEnd
\end{Liste}
\end{il}

\begin{rmk} Since the operator $T$ is not known, it is natural to
  consider a maximal risk also over a class for $\edf$ characterising the behaviour of
  $\Nsuite[s]{\iSv[s]= \vert \fedf[(s)] \vert ^{-2}}$, precisely $\rwCedf:=\{\edf\in\lp[2]:\edfCr^{-2}\leq\edfCw[s] \vert \fedf[ \vert s \vert ] \vert ^{2}=
\edfCw[s]/\iSv[ \vert s \vert ]\leq \edfCr^{2},\;\forall s\in\Nz\}\cap\cD$.
We shall note that for all $\Di\in\Nz$ and any $\edf\in\rwCedf$,
$\edfCr^{-2}\leq\miSv/\edfCwm\leq \edfCr^{2}$,
$\edfCr^{-2}\leq\oiSv/\edfCwo\leq \edfCr^{2}$. Setting
for all $\ssY,\ssE\in\Nz$
\begin{multline}\label{mm:de:mnra}
 \dRa{\Di}{\xdfCw[],\edfCw[]}:=[\xdfCw^2\vee\Di\edfCwo \ssY^{-1}],
\hfill
\mnDi(\xdfCw[],\edfCw[]):=\argmin\Nset[{\Di\in\Nz}]{\dRa{\Di}{\xdfCw[],\edfCw[]}},\hfill\\\mnRa{\xdfCw[],\edfCw[]}:=\dRa{\mnDi({\xdfCw[],\edfCw[]})}{\xdfCw[],\edfCw[]}=\min\Nset[{\Di\in\Nz}]{\dRa{\Di}{\xdfCw[],\edfCw[]}}\quad\text{
  and }\\
\mnRa{\xdfCw[],\edfCw[]}:=\max\{\xdfCw[(s)]\min(1,\edfCw[s]/\ssE),s\in\Nz\}.
\end{multline}
we have 
\begin{multline}\label{oo:e11}
 \mnRa{\xdfCw[],\Lambda}=\min_{\Di\in\Nz}\{[\xdfCw\vee\Di\oiSv\ssY^{-1}]\}\leq
 \edfCr^2\min_{\Di\in\Nz}\{[\xdfCw\vee\Di\edfCwo\ssY^{-1}]\}\leq \edfCr^2\dRa{\Di}{\xdfCw[],\edfCw[]}\\ 
  \mmRa{\xdfCw[],\Lambda}=\max_{s\in\Nz}\{\xdfCw[(s)]^2[1\wedge \iSv[s]/\ssE]\}\leq\edfCr^2\mnRa{\xdfCw[],\edfCw[]}.
\end{multline}
It follows for all $\Di,\ssY\in\Nz$ immediately that 
\begin{equation}\label{oo:e12}
  \nmRi{\hxdfPr}{\rwCxdf,\rwCedf}
  \leq (\xdfCr^2+8\edfCr^2) \mnRa{\xdfCw[],\edfCw[]}
+8(\cst{4}+1)\edfCr^2\xdfCr^2\mmRa{\xdfCw[],\edfCw[]}.
\end{equation}
\ncite{JohannesSchwarz2013a} have shown  that
  $\inf_{\hxdf}\nmRi{\hxdf}{\rwCxdf,\rwCedf}$, where the infimum is taken over all
  possible estimators $\hxdf$ of $\xdf$, is up to a constant bounded
  from below by $\mnRa{\xdfCw[],\edfCw[]}\vee\mmRa{\xdfCw[],\edfCw[]} $.  Consequently, the rate
  $\Nsuite[n]{\mnRa{\xdfCw[],\edfCw[]}\vee\mmRa{\xdfCw[],\edfCw[]}}$, the dimension parameters $\Nsuite[n]{\mnDi(\xdfCw[])}$
  and the projection estimators $\Nsuite[n]{\txdfPr[\mnDi({\xdfCw[]})]}$, respectively, is a
  minimax rate, a minimax dimension and minimax optimal (up to a
  constant).
\remEnd
\end{rmk}

%\subsection{Adaptivity}\label{INTRO_FREQ_ADAPTIVITY}
%We have given in the previous subsection criteria allowing to justify the choice of an estimator.
%However, we have also seen, while considering the example of projection estimators, that satisfying estimators often rely on some knowledge of the true parameter, which is in practice not known, in order to chose a tuning parameter.
%Hence, developing adaptive methods (that is to say, methods which do not rely on such knowledge) is mandatory for practical use and many technics have been considered in the past.
%We briefly present here the penalised contrast model selection technic, presented in more details, for example in \ncite{barron1999risk}, or \ncite{comte2006penalized}, and for which we will give a new interpretation in the following chapters as well as a new technic to obtain optimality results.
%
%Let be a sieve family such as $(\mathds{U}_{\overline{m}})_{m \in \mathds{M}}$, and the naive estimator $\theta_{n}$.
%One can notice that for any $x$ in $\mathds{U}_{\overline{m}}$, we have $\Vert x - \theta^{\circ} \Vert_{l^{2}}^{2} = \Vert x - \theta_{n} + \theta_{n} - \theta^{\circ} \Vert_{l^{2}}^{2} = \Vert x - \theta_{n} \Vert_{l^{2}}^{2}  - 2 \langle x - \theta_{n} \vert \theta^{\circ}_{\overline{m}} - \theta_{n} \rangle_{l^{2}} + \Vert \theta^{\circ} - \theta_{n} \Vert_{l^{2}}^{2} = \Vert x - \theta_{n} \Vert_{l^{2}}^{2}  - 2 (\langle x \vert \theta^{\circ} \rangle_{l^{2}} - \langle x \vert \theta_{n} \rangle_{l^{2}} - \langle \theta_{n} \vert \theta^{\circ} \rangle_{l^{2}} + \langle \theta_{n} \vert \theta_{n} \rangle_{l^{2}}) + \Vert \theta^{\circ} - \theta_{n} \Vert_{l^{2}}^{2}$.
%Hence, one would define the so called contrast function $\gamma$ from $\Theta_{\overline{m}}$ to $\R$ which to $x$ associates $\gamma(x) = \Vert x \Vert_{l^{2}}^{2} - 2 \langle x \vert \theta^{\circ}_{\overline{m}} \rangle_{l^{2}}$ and the associated contrast minimising estimator, if it exists $x_{n, m} = \argmin_{x \in \Theta_{\overline{m}}} \{\gamma(x)\}$.
\section{Bayesian approach}\label{INTRO_BAYES}

\subsection{The Bayesian paradigm}\label{INTRO_BAYES_PARADIGM}
\begin{itemize}
\item Bayes' theorem;
\item prior distribution;
\item posterior distribution (include conditions of existence);
\end{itemize}

\subsection{Typical priors for non-parametric models}\label{INTRO_BAYES_PRIOR}
\begin{itemize}
\item Gaussian process prior
\item Sieve priors (specific case)

\[\mathds{P}^{n}_{\boldsymbol{\theta}}(\theta) = \exp\left[-\frac{1}{2}\sum\limits_{\vert j \vert \leq m} \vert \theta_{j} \vert^{2}\right] \cdot \prod\limits_{\vert j \vert > m}\delta_{0}(\theta_{j})\]

\item Chinese restaurant process
\item Dirichlet process
\end{itemize}

\subsection{The pragmatic Bayesian approach}\label{INTRO_BAYES_PRAGMATIC}
\begin{itemize}
\item Consistence
\item contraction rate
\item exact contraction rate
\item uniform contraction rate
\item oracle optimality
\item minimax optimality
\end{itemize}

\subsection{Existing central results}\label{INTRO_BAYES_BIBLIO}
\begin{itemize}
\item Goshal Van der Vaart
\item Nickl
\end{itemize}

\subsection{Iteration procedure, self informative limit and Bayes carrier}\label{INTRO_BAYES_ITERATIVE}
\section{Examples of inverse problems}\label{1.4}

\subsection{Inverse Gaussian sequence space}\label{1.4.1}
Consider the Gaussian process $Y(x)$, defined on $[0, 1[$ with constant volatility $\frac{1}{n}$ with $n$ in $\mathds{N}^{\star}$ and mean process $f \star g$ where $f$ and $g$ are functions from $[0, 1[$ to $\mathds{R}$.
In short, we have $dY(x) = (f \star g)(x) dx + \frac{1}{n} dW(x)$ where $W$ is the Brownian motion.
We want to estimate $f$ while observing a realisation of $Y$.
We assume that $g$ is known.



We denote $\theta$ and $\lambda$ respectively the Fourier transforms of $f$ and $g$ respectively.

The likelihood with respect to the standard Brownian motion, noted $\P^{\circ}$, for this model can be written as follows (see \ncite{liptser2013statistics})
\[\frac{d \P_{Y^{n} \vert f, g}^{n}}{d \P^{\circ}} \propto \exp\left[\int_{[0, 1[} \frac{1}{\sqrt{n}} (f \star g)(x) dW(x) - \frac{1}{2} \left\Vert \frac{f \star g}{\sqrt{n}} \right\Vert^{2}\right].\]

We use the fact that the volatility of the process is constant and the properties of the Fourier transform to show that there exist a sequence of independent random variables with standard normal distribution such that the likelihood of the Fourier transform of the process is given by:
\[\frac{d\P^{n}_{Y^{n} \vert (\theta, \lambda)}}{d \P^{\circ}} \propto \exp\left[ -\frac{1}{2}\sum\limits_{j \in \mathds{Z}} \frac{\left(\theta_{j} \lambda_{j} - \xi_{j}\right)^{2}}{\sqrt{n}}\right].\]
Therefore, the Fourier transform of the observed process follows a Gaussian process indexed by $\mathds{Z}$, with mean $\theta \cdot \lambda$ and variance $\frac{1}{n}$.

Note that is the volatility was not constant, we would obtain
\[\frac{d\P^{n}_{Y^{n} \vert (\theta, \lambda)}}{d \P^{\circ}} \propto \exp\left[ -\frac{1}{2}\sum\limits_{j \in \mathds{Z}} \left((\sigma \star (\theta \lambda))_{j} - \xi_{j}\right)^{2}\right].\]
The mean process would hence be $\sigma \star (\theta \cdot \lambda)$, which can be rewritten as an inverse problem with a non diagonal operator, more precisely a Toeplitz operator.
We do not consider this case in this thesis.

Another motivation for this model is the heat equation.
\textcolor{red}{Heat equation + oracle rate for projection estimate + minimax rate (so all notations are introduced before moving on)}

\subsection{Circular density deconvolution}\label{1.4.2}

The circular deconvolution model is defined as follows: let $X$ and $\epsilon$ be circular random variables (that is to say, taking values in the unit circle, identified to the interval $[0,1[$) with respective distributions $\mathds{P}^{X}$ and $\mathds{P}^{\epsilon}$ and densities $f^{X}$ and $f^{\epsilon}$ with respect to some common and known dominating measure $\mu$ on $([0, 1[, \mathcal{A})$.
We would hence write for any $x$ in $[0, 1[$, $f^{X}(x) = \frac{d \mathds{P}^{X}}{d \mu} (x)$ for instance.

\begin{de}{\textsc{Modular addition}\\}\label{de1.4.1}
From now on we denote by $\Box$ the modular addition on $[0,1[$. That is to say
\[\forall (x, y) \in [0,1[^{2}, \quad x\Box y = x+y [1] = x + y - \lfloor x + y \rfloor.\]
\end{de}

The object of interest is $f^{X}$ while we only observe identically distributed replications $Y^{n} = \left(Y_{k}\right)_{k \in \llbracket 1, n \rrbracket}$ of the random variable $Y$, defined by $Y := X \Box \epsilon$.
We note $\mathds{P}^{Y}$ the distribution of the random variable $Y$ and $f^{Y}$ its density with respect to $\mu$.
One would notice that $\mathds{P}^{Y}$ and $f^{Y}$ are respectively given, for any $A$ in $\mathcal{A}$ and $y$ in $[0, 1[$, by $\mathds{P}^{Y}(A) = (\mathds{P}^{X} * \mathds{P}^{\epsilon})(A) = \int\limits_{[0,1[}\int\limits_{[0,1[} \mathds{1}_{A}(x \Box s)d\mathds{P}^{X}(x)d\mathds{P}^{\epsilon}(s)$ and $f^{Y}(y) = (f^{X} * f^{\epsilon})(y) = \int\limits_{0}^{1} f^{X}(y \Box (- s))f^{\epsilon}(s)d\mu(s)$.
Indeed, for any $\mu$-measurable and $\mu$-almost surely bounded function $g$, we have
\begin{alignat*}{3}
&\mathds{E}\left[g(Y)\right] &&=&& \mathds{E}\left[g(X \Box \epsilon)\right]\\
& &&=&&\int\limits_{0}^{1}\int\limits_{0}^{1} g(x \Box s) d\mathds{P}^{X}(x)d\mathds{P}^{\epsilon}(s)\\
& &&=&&\int\limits_{0}^{1}\int\limits_{0}^{1} g(y) d\mathds{P}^{X}(y \Box (-s))d\mathds{P}^{\epsilon}(s)\\
& &&=&&\int\limits_{0}^{1} g(y) \int\limits_{0}^{1} d\mathds{P}^{\epsilon}(s) d\mathds{P}^{X}(y \Box (-s))\\
& &&=&&\int\limits_{0}^{1} g(y) \int\limits_{0}^{1}f^{X}(y \Box (- s)) f^{\epsilon}(s)d\mu(s) d\mu(y);
\end{alignat*}
one should note that the integrals above converge, according to the dominated convergence theorem.

We will thus note $\mathcal{D}_{\mu}([0,1[)$ the space of densities on $[0, 1[$ with respect to $\mu$.
Moreover we write indifferently $^{*}\cdot$ the unary operator which associates to a distribution itself convoluted with $\mathds{P}^{\epsilon}$ and the unary operator which associates to a density itself convoluted with $f^{\epsilon}$.
That is to say, given a probability measure $\mathds{P}$ on $\left([0, 1[, \mathcal{A}\right)$, $^{*}\mathds{P}$ is such that, for any $A$ in $\mathcal{A}$, $^{*}\mathds{P}_{f}(A) = (\mathds{P}^{\epsilon}*\mathds{P}_{f})(A)$.
And for any element $f$ of $\mathcal{D}_{\mu}([0, 1[)$, $^{*}f$ is such that, for any $x$ in $[0, 1[$, $^{*}f(x) = (f * f^{\epsilon})(x)$.
The model can therefore at first be written $\left([0,1[^{n}, ^{*}\mathds{P}_{f}, f\in\mathcal{D}_{\mu}([0,1[) \right)$, where $\mathds{P}_{f}$ is the probability distribution with density $f$ with respect to $\mu$.

\textcolor{red}{I think it should be possible to show that $\mathds{P}^{\epsilon}$ does not have to be continuous w.r.t $\mu$ and that $\mathds{P}^{Y}$ would be anyway. Hence we do not need a density for $\mathds{P}^{\epsilon}$ and we can compute the Fourier transform of the distribution anyway.}

\medskip

\begin{rem}{\textsc{Positive (semi-)definitiveness}\\}\label{rem1.4.1}
A sequence/function $[f]$ from $\mathds{Z}$ to $\mathds{C}$ is positive (semi-)definite iff, for any finite subset $\left\{x_{1}, \hdots, x_{n}\right\}$, the Toeplitz matrix $A=(a_{i,j})_{(i,j) \in \llbracket 1, n \rrbracket^{2}}$ with $a_{i,j}$ defined by $[f](x_{i} - x_{j})$ is positive (semi-)definite.

In particular, this requires that $[f](x) = \overline{[f](-x)}$, $[f](0) > 0$ and for all $x$, $[f](x) \leq [f](0).$
\end{rem}

\medskip

Then, by denoting $\mathcal{M}([0, 1[)$ the set of all probability measures on $[0,1[$ and $\mathcal{S}^{+}(\mathds{Z})$ the set of all positive definite, complex valued, functions $[f]$ on $\mathds{Z}$ with $[f](0)=1$, we define the Fourier transform.

\begin{de}{\textsc{Fourier transform of measures}\\}\label{de1.4.2}
We denote by $\mathcal{F}$ the Fourier transform operator on measures :
\begin{alignat*}{4}
&\mathcal{F} : \quad && \mathcal{M}([0, 1[) &&\rightarrow&& \mathcal{S}^{+}(\mathds{Z})\\
& && \nu && \mapsto && \left(j \mapsto \int\limits_{0}^{1} \exp\left[- 2 i \pi j x\right] d\nu(x)\right).
\end{alignat*}
\end{de}

\begin{nota}{\textsc{Fourier basis functions}\\}\label{nota1.4.1}
As we will operate in the frequency domain for most of the remaining note, it is convenient to use the following notation for the orthonormal basis used in Fourier transform :
\[\forall j \in \mathds{Z}, \forall x \in [0, 1[, \quad e_{j}(x) := \exp[- 2 i \pi j x].\]
\end{nota}

\begin{rmk}\label{rmk1.4.1}
It is convenient to note that for any $x$ and $s$ in $[0, 1[$ and $j$ in $\mathds{Z}$, we have $e_{j}[x \Box s] = e_{j}[x]e_{j}[s]$, due to the periodicity of the complex exponential function.
\end{rmk}

As we are interested in densities of probability distributions dominated by a common measure $\mu$ we define the Fourier transform with respect to $\mu$.

\begin{de}{\textsc{Fourier transform of densities}\\}\label{de1.4.3}
We denote by $\mathcal{F}_{\mu}$ the Fourier transform operator of densities with respect to the measure $\mu$ :
\begin{alignat*}{4}
&\mathcal{F}_{\mu} : \quad && \mathcal{D}_{\mu}([0,1[) &&\rightarrow&& \mathcal{S}^{+}(\mathds{Z})\\
& && f && \mapsto && \left(j \mapsto \int\limits_{0}^{1} e_{j}(x) f(x) d\mu(x)\right).
\end{alignat*}
\end{de}

\begin{nota}{\textsc{Fourier transform of useful functions}\\}\label{nota1.4.2}
From now on we adopt the following notations for the functions which will appear regularly :
\begin{alignat*}{4}
&\forall j \in \mathds{Z}, && \theta^{\circ}_{j} := \mathcal{F}_{\mu}(f^{X})(j);\\
& && \lambda_{j} := \mathcal{F}_{\mu}(f^{\epsilon})(j);\\
&\forall f \in \mathcal{D}_{\mu}([0, 1[), \forall j \in \mathds{Z}, \quad && [f](j) := \mathcal{F}_{\mu}(f)(j).\\
\end{alignat*}
\end{nota}

Obviously, we have
\begin{alignat*}{4}
&\forall j \in \mathds{Z}, && \mathcal{F}(f^{Y})(j)&&=&&\int\limits_{0}^{1} e_{j}(y) \mathds{P}^{Y}(dy)\\
& && &&=&&\int\limits_{0}^{1}\int\limits_{0}^{1} e_{j}(x \Box s) \mathds{P}^{X}(dx)\mathds{P}^{\epsilon}(ds)\\
& && &&=&&\int\limits_{0}^{1}e_{j}(s)\int\limits_{0}^{1} e_{j}(x) \mathds{P}^{X}(dx)\mathds{P}^{\epsilon}(ds)\\
& && &&=&&\int\limits_{0}^{1}e_{j}(s)\mathds{P}^{\epsilon}(ds)\int\limits_{0}^{1} e_{j}(x)\mathds{P}^{X}(dx)\\
& && &&=&&\mathcal{F}(\mathds{P}^{\epsilon})(j) \mathcal{F}(\mathds{P}^{X})(j)\\
%& && &&=&&\int\limits_{0}^{1} \exp\left[- 2 i \pi j y\right] f^{Y}(y)\mu(dy)\\
& && &&=&&\int\limits_{0}^{1} f^{\epsilon}(s) e_{j}(s) d\mu(s) \int\limits_{0}^{1} e_{j}(x) f^{X}(x)\mu(dx)\\
& && &&=&&\mathcal{F}_{\mu}(f^{\epsilon})(j) \mathcal{F}_{\mu}(f^{X})(j)\\
& && &&=&& \theta^{\circ}_{j} \lambda_{j}
\end{alignat*}
so the Fourier transform, exchanges convolution with point-wise product.

\medskip

The following theorem, which is a special case of Bochner's theorem, allows us to formulate an inverse for the Fourier transform.

\begin{thm}{\textsc{Herglotz's representation theorem}\\}\label{thm1.4.1}
A function $[f]$ from $\mathds{Z}$ to $\mathds{C}$ with $[f](0) = 1$ is semi-definite positive iff there exist $\mu$ in $\mathcal{M}([0, 1[)$ such that for all $j$ in $\mathds{Z}$, we have
\[[f](j) = \int\limits_{[0, 1[} \exp[- 2 i \pi j x] d\mu(x).\]
\end{thm}

The properties of the set $\mathcal{S}^{+}(\mathds{Z})$ can be interpreted as follow :

\begin{alignat*}{5}
& && \mathcal{F}(f)(j)&&=&& \overline{\mathcal{F}(f^{Y})(-j)}&& \quad f \text{ is real valued;}\\ 
& && \mathcal{F}(f)(0) &&=&& 1&& \quad f \text{ integrates at }1;\\
\end{alignat*}
and $\mathcal{F}(f)$ positive semi-definitive implies the positivity of $f$.

The Fourier transform being bijective, one can safely write its inversion and we have, for any function $[f]$ in $\mathcal{S}^{+}$ :
\begin{alignat*}{4}
&\forall A \in \mathcal{A},&& \quad \mathcal{F}^{-1}[f](A) &&=&& \int\limits_{A}\sum\limits_{j \in \mathds{Z}} [f](j)e_{j}(x)dx;\\
&\forall x \in [0, 1[,&& \quad \mathcal{F}_{\mu}^{-1}[f](x) &&=&& \sum\limits_{j \in \mathds{Z}} [f](j)e_{j}(x).
\end{alignat*}

However, in the most general case, the above mentioned series do not necessarily converge and one would need to consider the densities on our model as Schwartz distributions (see \ncite{Bill86}).
We avoid this difficulty by assuming the considered distributions dominated by the Lebesgue measure.
We hence drop the $\mu$ index from now on (and, for example note $\mathcal{D}([0,1[)$ instead of $\mathcal{D}_{\mu}([0, 1)$).

We will hence consider the model written in these terms : $\left([0, 1[^{n}, \mathds{P}_{[f]}, f \in \mathcal{S}^{+}(\mathds{Z})\right)$; where $\mathds{P}_{[f]}$ is the distribution which admits the density with respect to $\mu$ which Fourier transform is $[f]$.

\medskip

As a concluding note for this section, let us mention the risk we will use to formulate optimality of the different inference methods described there after.
For a given, strictly positive real number, we define the usual scalar product on $\mathcal{D}([0,1[)$ :

\begin{de}{\textsc{Scalar product $\langle \cdot \vert \cdot\rangle_{L^{2}}$ on $\mathcal{D}([0,1[)$}\\}\label{de1.4.4}
We define the scalar product
\begin{alignat*}{4}
& \langle \cdot \vert \cdot \rangle_{L^{2}} : && \quad \mathcal{D}([0,1[) \times \mathcal{D}([0,1[) && \rightarrow && \overline{\mathds{R}}.\\
& && \quad (f, g) && \mapsto && \int\limits_{[0, 1[} f(x) \overline{g(x)} dx
\end{alignat*}
\end{de}

We obtain with this scalar product the natural $L^{2}$ norm :
\begin{de}{\textsc{$L^{2}$-norm $\Vert \cdot \Vert_{L^{2}}$ on $\mathcal{D}([0,1[)$}\\}\label{de1.4.5}
We define the norm
\begin{alignat*}{4}
& \Vert \cdot \Vert_{L^{2}} : && \mathcal{D}([0,1[) && \rightarrow && \overline{\mathds{R}_{+}}.\\
& && f && \mapsto && \langle f \vert f \rangle_{L^{2}}^{1/2} = \left(\int\limits_{[0, 1[} \vert f(x)\vert^{2} dx\right)^{1/2}
\end{alignat*}
\end{de}

For statistical inference it is generally necessary to assume that the objects of interest have finite norm.
We hence define the space $\mathds{L}^{2}$:
\begin{de}{\textsc{Space $\mathds{L}^{2}$ of functions}\\}\label{de1.4.6}
We define the set
\[\mathds{L}^{2} := \left\{f \in \mathcal{D}([0, 1[) : \Vert f \Vert_{L^{2}} < \infty \right\}.\]
\end{de}

It is common to consider the larger family of norms $\Vert \cdot \Vert_{L^{p}}$ for any number $p$ in $[1, \infty]$ which however do not define an inner product space :
\begin{de}{\textsc{$L^{p}$-norm $\Vert \cdot \Vert_{L^{p}}$ on $\mathcal{D}([0,1[)$}\\}\label{de1.4.7}
We define the norm
\begin{alignat*}{4}
& \Vert \cdot \Vert_{L^{p}} : && \mathcal{D}([0,1[) && \rightarrow && \overline{\mathds{R}_{+}}.\\
& && f && \mapsto && \left(\int\limits_{[0, 1[} \vert f(x)\vert^{p} dx\right)^{1/p}
\end{alignat*}
\end{de}

Obviously one can define the associated spaces:
\begin{de}{\textsc{Space $\mathds{L}^{p}$ of functions}\\}\label{de1.4.8}
We define the set
\[\mathds{L}^{p} := \left\{f \in \mathcal{D}([0, 1[) : \Vert f \Vert_{L^{p}} < \infty \right\}.\]
\end{de}

A last kind of norm which is of interest are the weighted norms.
Using a weighted norm as loss function allows to give more interest to some specific features of the functions (high or low frequencies for example).
\begin{de}{\textsc{$L^{p}_{\mathfrak{u}}$-norm $\Vert \cdot \Vert_{L^{p}_{\mathfrak{u}}}$ on $\mathcal{D}([0,1[)$}\\}\label{de1.4.9}
Consider a distribution $\mathfrak{u}$ from $[0,1[$ to $\mathds{R}$.
We define the norm
\begin{alignat*}{4}
& \Vert \cdot \Vert_{L^{r}_{\mathfrak{u}}} : && \mathcal{D}([0,1[) && \rightarrow && \overline{\mathds{R}_{+}}.\\
& && f && \mapsto && \left(\int\limits_{[0, 1[} \vert (f*\mathfrak{u})(x)\vert^{p} dx\right)^{1/p}
\end{alignat*}
In particular, if $\mathfrak{u}$ is the Dirac distribution in $0$, we find the definition of $\Vert \cdot \Vert_{L^{p}}$.
\end{de}

We finally define the associated spaces:
\begin{de}{\textsc{Space $\mathds{L}_{\mathfrak{u}}^{p}$ of functions}\\}\label{de1.4.10}
We define the set
\[\mathds{L}_{\mathfrak{u}}^{p} := \left\{f \in \mathcal{D}([0, 1[) : \Vert f \Vert_{L_{\mathfrak{u}}^{p}} < \infty \right\}.\]
\end{de}

Given that we use $\mathcal{S}^{+}(\mathds{Z})$ as a parameter space, it is interesting to compare the norms defined above to norms on this space.

For this purpose, we introduce the dot product for sequences.
\begin{de}{\textsc{Product $\cdot$ on $\mathcal{S}^{+}(\mathds{Z})$}\\}\label{de1.4.11}
We define the bi-linear operator
\begin{alignat*}{4}
& \cdot : && \quad \mathcal{S}^{+}(\mathds{Z})^{2} &&\rightarrow&& \mathcal{S}^{+}(\mathds{Z})\\
& && ([f], [g]) && \mapsto && [f]\cdot[g] := \left(j \mapsto [f](j)[g](j)\right).
\end{alignat*}
\end{de}

We will also use the inner product of $\mathcal{S}^{+}(\mathds{Z})$
\begin{de}{\textsc{Inner product $\left\langle \cdot \vert \cdot \right\rangle_{l^{2}}$ on $\mathcal{S}^{+}(\mathds{Z})$}\\}\label{de1.4.12}
We define the operator
\begin{alignat*}{4}
& \left\langle \cdot \vert \cdot \right\rangle_{l^{2}} : && \quad \mathcal{S}^{+}(\mathds{Z})^{2} &&\rightarrow&& \overline{\mathds{C}}\\
& && ([f], [g]) && \mapsto && \sum\limits_{j \in \mathds{Z}} ([f]\cdot\overline{[g]})(j).
\end{alignat*}
\end{de}

This leads to the natural $l^{2}$-norm
\begin{de}{\textsc{$l^{2}$-norm $\Vert \cdot \Vert_{l^{2}}$ on $\mathcal{S}^{+}(\mathds{Z})$}\\}\label{de1.4.13}
We define the norm
\begin{alignat*}{4}
& \Vert \cdot \Vert_{l^{2}} : && \quad \mathcal{S}^{+}(\mathds{Z}) &&\rightarrow&& \overline{\mathds{R}_{+}}\\
& &&\quad [f] && \mapsto && \left(\sum\limits_{j \in \mathds{Z}} \vert[f](j)\vert^{2}\right)^{1/2}.
\end{alignat*}
\end{de}

It is common to consider the larger family of norms $\Vert \cdot \Vert_{l^{p}}$ for any number $p$ in $[1, \infty]$ which however do not define an inner product space :
\begin{de}{\textsc{$l^{p}$-norm $\Vert \cdot \Vert_{l^{p}}$ on $\mathcal{S}^{+}(\mathds{Z})$}\\}\label{de1.4.14}
We define the norm
\begin{alignat*}{4}
& \Vert \cdot \Vert_{l^{p}} : && \mathcal{S}^{+}(\mathds{Z}) && \rightarrow && \overline{\mathds{R}_{+}}.\\
& && f && \mapsto && \left(\sum\limits_{j \in \mathds{Z}} \vert [f](j) \vert^{p}\right)^{1/p}
\end{alignat*}
\end{de}

A last kind of norm which is of interest are the weighted norms.
Using a weighted norm as loss function allows to give more interest to some specific features of the functions (high or low frequencies for example).
\begin{de}{\textsc{$l^{p}_{\mathfrak{u}}$-norm $\Vert \cdot \Vert_{l^{p}_{\mathfrak{u}}}$ on $\mathcal{S}^{+}(\mathds{Z})$}\\}\label{de1.4.15}
Consider an element $[\mathfrak{u}]$ of $\mathcal{S}^{+}(\mathds{Z})$.
We define the norm
\begin{alignat*}{4}
& \Vert \cdot \Vert_{l^{p}_{[\mathfrak{u}]}} : && \mathcal{S}^{+}(\mathds{Z}) && \rightarrow && \mathds{R}_{+}.\\
& && [f] && \mapsto && \left(\sum\limits_{j \in \mathds{Z}} \vert ([f]\cdot[\mathfrak{u}])(j)\vert^{p} dx\right)^{1/p}
\end{alignat*}
In particular, if $[\mathfrak{u}]$ is the sequence constantly equal to $1$, we find the definition of $\Vert \cdot \Vert_{l^{p}}$.
\end{de}

As previously, we define the spaces associated with these norms:
\begin{de}{\textsc{Spaces $\mathcal{L}^{2}, \mathcal{L}^{p}, \mathcal{L}_{[\mathfrak{u}]}^{p}$ of functions}\\}\label{de1.4.16}
We define the sets
\begin{alignat*}{3}
&\mathcal{L}^{2} &&:=&& \left\{[f] \in \mathcal{S}^{+}(\mathds{Z}) : \Vert [f] \Vert_{l^{2}} < \infty \right\};\\
&\mathcal{L}_{\mathfrak{u}}^{p} &&:=&& \left\{[f] \in \mathcal{S}^{+}(\mathds{Z}) : \Vert [f] \Vert_{l^{p}} < \infty \right\};\\
&\mathcal{L}_{[\mathfrak{u}]}^{p} &&:=&& \left\{[f] \in \mathcal{S}^{+}(\mathds{Z}) : \Vert [f] \Vert_{l_{[\mathfrak{u}]}^{p}} < \infty \right\}.
\end{alignat*}
\end{de}

We have, for any $p$ in $[1 ,\infty]$ and $f$ in $\mathcal{D}([0, 1[)$.
\begin{alignat*}{3}
&\Vert f \Vert^{r}_{\mathfrak{u}} &&=&& \left(\int\limits_{[0, 1[} \vert (f*\mathfrak{u})(x)\vert^{p} dx\right)^{1/p}\\
& &&=&& \left(\int\limits_{[0, 1[} \left\vert \sum\limits_{j \in \mathds{Z}}([f]\cdot [\mathfrak{u}])(j) \cdot e_{j}(x)\right\vert^{p} dx\right)^{1/p}\\
& &&\leq&& \left(\int\limits_{[0, 1[} \sum\limits_{j \in \mathds{Z}}\left((\vert[f]\cdot [\mathfrak{u}])(j)\vert \cdot \vert e_{j}(x)\vert\right)^{p} dx\right)^{1/p}\\
& &&\leq&& \left(\sum\limits_{j \in \mathds{Z}}\vert ([f]\cdot[\mathfrak{u}])(j)\vert^{p}\int\limits_{[0, 1[} \vert e_{j}(x)\vert^{r} dx\right)^{1/p}\\
& &&\leq&& \left(\sum\limits_{j \in \mathds{Z}}\vert ([f] \cdot [\mathfrak{u}])(j)\vert^{p}\cdot 1\right)^{1/p}\\
& &&\leq&& \Vert[f]\cdot [\mathfrak{u}]\Vert_{l^{p}}\\
& &&\leq&& \Vert[f]\Vert_{l^{p}_{[\mathfrak{u}]}}.
\end{alignat*}

For the specific case of $p=2$, the theorem of Plancherel holds and we have
\begin{alignat*}{3}
&\Vert f \Vert^{2}_{\mathfrak{u}} &&=&& \Vert f * \mathfrak{u} \Vert^{2}\\
& &&=&& \Vert [f] \cdot [\mathfrak{u}]\Vert_{l^{2}}\\
& &&=&& \Vert [f] \Vert_{l_{[\mathfrak{u}]}^{2}}.
\end{alignat*}

We hence assume from now on that the parameter of interest has finite norm.

\begin{as}\label{as1.4.1}
The parameter of interest $\theta^{\circ}$ is in $\mathcal{L}_{[\mathfrak{u}]}^{p}$.
\end{as}
%
\chapter{Bayesian interpretation of penalised contrast model selection}\label{2}
%
In this chapter, we consider the family of Bayesian methods described as "Gaussian sieve priors" in \nref{1.3.2} as well as an adaptive variant of these priors, the hierarchical sieve priors where the threshold parameter is a random variable with a specified prior distribution.
We study their behaviour under two asymptotic, respectively described in \nref{1.3.3} and \nref{1.3.5}.
%
In \nref{2.1} we consider the self informative Bayes carrier of Gaussian sieve priors under continuity assumptions for the likelihood and show that its support is contained in the maximum likelihood set.
Then, in \nref{2.2} we show that the distribution of the hyper-parameter in the hierarchical prior contracts around the set of maximisers of a penalised contrast criterion.
This section highlights a new link between Bayesian adaptive estimation and the frequentist penalised contrast model selection.
%
\medskip
%
In \nref{2.3}, while considering the noise asymptotic, we line out two strategies of proof which allow to obtain contraction rates. The first relies on posterior moment bounding and which, up to our knowledge, is new; the second is specific to the hierarchical sieve prior and is similar to the one used in \ncite{JJASRS}.
In \nref{2.4} we apply this strategies to the specific inverse Gaussian sequence space model.
Doing so, we obtain exact contraction rate for the (iterated) Gaussian sieve prior using the first scheme of proof; and the iterated hierarchical prior using the second.
This yields optimality for sieve priors with properly chosen threshold parameter; as well as for penalised contrast model selection; and for any iterated version of the hierarchical prior we consider.
The most interesting point of this subsection is the novel way to show optimality of the penalised contrast model selection.
%
\medskip
%
In \nref{2.5} we inquire the use of the discussed method to the circular deconvolution model and show that a direct use of those methods is not possible in this context.
We give nonetheless some tracks for a fix.
%
\medskip
%
Finally, we conclude this chapter with a note about the shape of the posterior mean of the hierarchical prior, motivating the shape of the frequentist estimators we use in \nref{3}.
%
\section{Iterated Gaussian sieve prior}\label{2.1}

We consider in this part a statistical model with a functional parameter space as described in \nref{1.1.1}.
We adopt a sieve prior as described in \nref{1.3.2} and first give interest to the asymptotic presented in \nref{1.3.5}.

\medskip

We first remind the following notations.
The parameter space $\Theta$ is a function space $\Theta = \{ \theta : \mathds{F} \rightarrow \mathds{I} \}$; with $\mathds{F}$ a subset of $\R$ and $\mathds{I}$ a subset of $\C$.

To derive the self informative Bayes carrier we formulate the following hypothesis.

\begin{as}{\textsc{Countability assumption}\\}\label{as2.1.1}
We assume that the set $\mathds{F}$ is countable.
\end{as}

We equip $\Theta$ with the usual $\mathds{L}^{2}$ norm that is, $\Vert \theta \Vert^{2} = \sum\limits_{j \in \mathds{J}} \vert \theta_{j} \vert^{2}$ and consider the Borel sigma algebra $\mathcal{B}$ of the topology generated by this $\mathds{L}^{2}$ norm.

On the other hand our observation $Y$ take values in the space $(\mathds{Y}, \mathcal{Y})$ with distribution in the family $(\P_{Y \vert \boldsymbol{\theta}})_{\boldsymbol{\theta} \in \Theta}$.


We assume the existence of a function $l: (\Theta, \mathcal{B}) \times (\mathds{Y}, \mathcal{Y}) \rightarrow \R$ such that the likelihood with respect to some reference measures $\P^{\circ}$ is given by:

\[L(\boldsymbol{\theta}, y) \propto \exp\left[-l(\boldsymbol{\theta}, y)\right].\]

Then, the family of Gaussian sieve priors is indexed by a threshold parameter $m$ in the set of subsets of $\mathds{J}$, denoted $\mathcal{P}(\mathds{J})$, and we denote by $\P_{\boldsymbol{\theta}^{m}}$ the element of this family with index $m$; moreover, we denote $\boldsymbol{\theta}^{m}$ a random variable following this distribution. 
There exists a reference measure $\Q^{\circ}$ such that the sieve prior with threshold parameter $m$ admits a density of the shape

\[\frac{d\P_{\boldsymbol{\theta}^{m}}}{d\Q^{\circ}}(\boldsymbol{\theta}) \propto  \exp\left[-\frac{1}{2}\sum\limits_{j \in m} \vert \boldsymbol{\theta}_{j} \vert^{2}\right] \cdot \prod\limits_{j \notin m} \delta_{0}(\boldsymbol{\theta}_{j}).\]

If we denote by $\Theta_{m}$ the set $\{\theta \in \Theta : \forall j \notin m, \theta_{j} = 0\}$, Bayes' theorem gives the following shape for the iterated posterior distribution:

\begin{alignat*}{3}
& \frac{d\P_{\boldsymbol{\theta}^{m}\vert Y}^{\eta}}{d\Q^{\circ}}(\boldsymbol{\theta}, y)&& = && \frac{\exp\left[-\left(\frac{1}{2}\sum\limits_{j \in m} \vert \boldsymbol{\theta}_{j} \vert^{2} + \eta l(\boldsymbol{\theta}, y)\right)\right] \cdot \prod\limits_{j \notin m} \delta_{0}(\boldsymbol{\theta}_{j})}{\int_{\Theta_{m}} \exp\left[-\left(\frac{1}{2}\sum\limits_{j \in m} \vert \mu_{j} \vert^{2} + \eta l(\mu, y)\right)\right] d\mu}\\
& && = && \frac{\prod\limits_{j \notin m} \delta_{0}(\boldsymbol{\theta}_{j})}{\int_{\Theta_{m}} \exp\left[-\frac{1}{2}\sum\limits_{j \in m} \left(\vert \mu_{j} \vert^{2} - \vert \boldsymbol{\theta}_{j} \vert^{2}\right)\right]\exp\left[-\eta\left(l(\mu, y) - l(\boldsymbol{\theta}, y)\right)\right] d\mu}.
\end{alignat*}

The following assumption is also needed to obtain the self informative Bayes carrier.

\begin{as}{\textsc{Continuous likelihood asumption}\\}\label{as2.1.2}
Assume that for any $m$ in $\mathcal{P}(\mathds{J})$ and $y$, $\Theta_{m} \rightarrow \R_{+}, \theta \mapsto l(\theta, y)$ is continuous.
\end{as}

The use of a threshold parameter brings us back to the study of a parametric model and the results from \textcolor{red}{ref Bunke} can be used to derive the self informative Bayes carrier.

\begin{thm}{\textsc{Self informative Bayes carrier for a sieve prior}\\}\label{thm2.1.1}
Under \nref{as2.1.1} and \nref{as2.1.2} the support of the Bayesian carrier is contained in the set of minimisers of $\theta \mapsto l(\theta, y)$.
\end{thm}

\begin{pro}{\textsc{Proof of \nref{thm2.1.1}}\\}\label{pro2.1.1}
Let's remind that the definition of continuity gives us:
\[\forall \theta \in \Theta_{m}, \forall \epsilon \in \R_{+}^{\star}, \exists \delta \in \R_{+}^{\star} : \forall \mu \in \Theta_{m}, \Vert \mu - \theta \Vert < \delta \Rightarrow \vert l(\mu, y) - l(\theta, y) \vert < \epsilon.\]

\medskip

Then, for any $B$ in $\mathcal{B}$ such that $\inf\limits_{\theta \in B} l(\theta, y) > \inf\limits_{\mu \in \Theta_{m}} l(\mu, y)$, there exist $\delta$ in $\R_{+}^{\star}$ and a ball $\mathcal{E}$ of $\Theta_{m}$ of radius $\delta$ such that, $\sup\limits_{\mu \in \mathcal{E}} l(\mu, y) < \inf\limits_{\theta \in B}l(\theta, y)$ and hence $\sup\limits_{\mu \in \mathcal{E}}l(\mu, y) - \inf\limits_{\theta \in B}l(\theta, y) < 0$.

Hence we can write
\begin{alignat*}{3}
& \P_{\boldsymbol{\theta}^{m}\vert Y}^{\eta}(B) && = && \int_{B} \frac{\prod\limits_{\vert j \vert > m} \delta_{0}(\boldsymbol{\theta}_{j})}{\int_{\Theta_{m}} \exp\left[-\frac{1}{2}\sum\limits_{\vert j \vert \leq m} \left(\vert \mu_{j} \vert^{2} - \vert \boldsymbol{\theta}_{j} \vert^{2}\right)\right]\exp\left[-\eta\left(l(\mu, y) - l(\boldsymbol{\theta}, y)\right)\right] d\mu} d \theta\\
& && \leq && \int_{B} \frac{\prod\limits_{\vert j \vert > m} \delta_{0}(\theta_{j})}{\exp\left[-\eta\left(\sup\limits_{\mu \in \mathcal{E}} l(\mu, y) - \inf\limits_{\theta \in B}l(\theta, y)\right)\right] \int_{\mathcal{E}} \exp\left[-\frac{1}{2}\sum\limits_{\vert j \vert \leq m} \left(\vert \mu_{j} \vert^{2} - \vert \boldsymbol{\theta}_{j} \vert^{2}\right)\right]d\mu} d \theta\\
& && \leq && \frac{1}{\exp\left[-\eta\left(\sup\limits_{\mu \in \mathcal{E}} l(\mu, y) - \inf\limits_{\theta \in B}l(\theta, y)\right)\right]}\int_{B} \frac{\prod\limits_{\vert j \vert > m} \delta_{0}(\theta_{j}) \exp\left[-\frac{1}{2}\sum\limits_{\vert j \vert \leq m} \vert \boldsymbol{\theta}_{j} \vert^{2}\right]}{ \int_{\mathcal{E}} \exp\left[-\frac{1}{2}\sum\limits_{\vert j \vert \leq m} \vert \mu_{j} \vert^{2}\right]d\mu} d \theta\\
& && \rightarrow && 0.
\end{alignat*}
\qedsymbol
\end{pro}

We have hence showed that under the iteration asymptotic, the posterior distribution contracts itself on maximisers of the likelihood, constrained by $\theta_{j} = 0$ for any $\vert j \vert > m$.

\textcolor{red}{Add remark with several maximisers}

There is hence a clear link between this type of prior distribution and projection estimators.
We will see that, while considering the noise asymptotic, the choice of the threshold is determinant for the quality of the estimation.
The choice of the threshold for the projection estimators and for sieve priors should be led in a similar fashion, that is, balancing the bias (small value of the threshold) and the variance (high value of the threshold).
As stated previously, the ideal choice of this parameter is however dependent on the parameter of interest and hence not available.
It is hence important to inquire adaptive methods for the selection of this parameter.
Some methods for the frequentist estimation were outlined in the introduction such as the penalised contrast model selection.
In the next section, we introduce the hierarchical sieve prior which consists in modelling the threshold parameter as a random variable.
We will show that by selecting the prior distribution for this hyper-parameter properly, the iteration asymptotic gives a Bayesian interpretation to the penalised contrast model selection.

\section{Adaptivity using a hierarchical prior}\label{2.2}

We denote $\P_{\boldsymbol{\theta}^{M}}$ a so called hierarchical prior distribution, described hereafter, and $\boldsymbol{\theta}^{M}$ a random variable following this prior.
Define $G$ a finite subset of $\mathds{J}$ and  $\pen: \mathcal{P}(G) \rightarrow \R_{+}$ a so-called penalty function.
The threshold parameter noted $m$ for the sieve prior described in the previous section is now a $\mathcal{P}(G)$-valued random variable denoted $M$. We note $\P_{M}$ the distribution of this parameter.

The density of $\P_{M}$ with respect to the counting measure has the shape
\[\P_{M}(m) \propto \exp[- \pen(m)] \mathds{1}_{m \subset G}.\]

The dependance structure between the different quantities of the model is then the following:

\begin{alignat*}{3}
& \P_{\boldsymbol{\theta}^{M} \vert M=m} && = && \P_{\boldsymbol{\theta}^{m}};\\
& \P_{Y \vert \boldsymbol{\theta}, M} && = && \P_{Y \vert \boldsymbol{\theta}}.
\end{alignat*}

We can then obtain the following form for the posterior distribution of the hyper parameter:

\begin{alignat*}{3}
&\P_{M \vert Y}(m, y) &&\propto&& \frac{d\P_{M, Y}}{d\P^{\circ}}(m, y)\\
& &&\propto&&\int_{\Theta}\frac{d\P_{M, Y, \boldsymbol{\theta}^{M}}}{d\P^{\circ} \, d\Q^{\circ}}(m, y, \theta)d\mathds{Q}^{\circ}(\theta)\\
& &&\propto&&\int_{\Theta}\frac{d\P_{Y \vert M, \boldsymbol{\theta}^{M}}}{d\P^{\circ}}(m, y, \theta) \, \frac{d\P_{M, \boldsymbol{\theta}^{M}}}{d\mathds{Q}^{\circ}}(m, \theta)d\mathds{Q}^{\circ}(\theta)\\
& &&\propto&&\int_{\Theta}\frac{d\P_{Y \vert \boldsymbol{\theta}^{M}}}{d\P^{\circ}}(y, \theta) \, \frac{d\P_{\boldsymbol{\theta}^{M}\vert M}}{d\mathds{Q}^{\circ}}(m, \theta) \P_{M}(m)d\mathds{Q}^{\circ}(\theta)\\
& &&\propto&&\P_{M}(m)\int_{\Theta}\frac{d\P_{Y \vert \boldsymbol{\theta}^{M}}}{d\P^{\circ}}(y, \theta) \, \frac{d\P_{\boldsymbol{\theta}^{m}}}{d\mathds{Q}^{\circ}}(m, \theta) d\mathds{Q}^{\circ}(\theta)\\
& && = && \frac{\P_{M}(m)\int_{\Theta}\frac{d\P_{Y \vert \boldsymbol{\theta}^{M}}}{d\P^{\circ}}(y, \theta) \, \frac{d\P_{\boldsymbol{\theta}^{m}}}{d\mathds{Q}^{\circ}}(m, \theta) d\mathds{Q}^{\circ}(\theta)}{\sum\limits_{j \subset G}\P_{M}(j)\int_{\Theta}\frac{d\P_{Y \vert \boldsymbol{\theta}^{M}}}{d\P^{\circ}}(y, \theta) \, \frac{d\P_{\boldsymbol{\theta}^{m}}}{d\mathds{Q}^{\circ}}(j, \theta) d\mathds{Q}^{\circ}(\theta)}\\
& && = && \frac{\exp[- \pen(m)] \int_{\Theta_{m}}\exp[-\frac{1}{2}(2 l(y, \theta) + \sum\limits_{k \in m} \vert \theta_{k} \vert^{2})] d\mathds{Q}^{\circ}(\theta)}{\sum\limits_{j \subset G}\exp[- \pen(j)] \int_{\Theta_{j}}\exp[-\frac{1}{2}(2 l(y, \theta) + \sum\limits_{k \in j} \vert \theta_{k} \vert^{2})] d\mathds{Q}^{\circ}(\theta)}.
\end{alignat*}

From this, we can deduce the iterated posterior.
Indeed, by defining
\[\exp[\Upsilon(Y, m)] := \int_{\Theta_{m}}\exp[-\frac{1}{2}(2 l(y, \theta) + \sum\limits_{k \in m} \vert \theta_{k} \vert^{2})] d\mathds{Q}^{\circ}(\theta)\]
we have:

\begin{alignat*}{3}
&\P_{M \vert Y}^{\eta}(m, y) && = && \frac{\P_{M}(m)\left(\int_{\Theta_{m}}\exp[-\frac{1}{2}(2 l(y, \theta) + \sum\limits_{k \in m} \vert \theta_{k} \vert^{2})] d\mathds{Q}^{\circ}(\theta)\right)^{\eta}}{\sum\limits_{j \subset J}\P_{M}(j)\left(\int_{\Theta_{j}}\exp[-\frac{1}{2}(2 l(y, \theta) + \sum\limits_{k \in j} \vert \theta_{k} \vert^{2})] d\mathds{Q}^{\circ}(\theta)\right)^{\eta}}\\
& && = && \frac{\exp[-\pen(m) + \eta \Upsilon(Y, m)]}{\sum\limits_{j \subset G}\exp[- \pen(j) + \eta \Upsilon(Y, j)]} \mathds{1}_{m \subset G}\\
& && = && \frac{1}{\sum\limits_{j \subset G}\exp\left[\eta \left(\Upsilon(Y, j) - \Upsilon(Y, m)\right) - \left(\pen(j) - \pen(m)\right)\right]} \mathds{1}_{m \subset G}
\end{alignat*}

and we can deduce the self informative Bayes carrier.

\begin{lm}{\textsc{Self informative Bayes carrier of the hyper-parameter in a hierarchical sieve prior I}\\}\label{lm2.2.1}
The support of the self informative Bayes carrier for the hyper-parameter $M$ is
\[\argmax\limits_{m \subset G} \{\Upsilon(Y, m)\}.\]
\end{lm}

Unfortunately, in many practical cases, the choice led by $\argmax\limits_{m \subset G} \{\Upsilon(Y, m)\}$ is $G$ itself and leads to inconsistent inference (as we will show later).
However, if one allows the prior distribution to depend on $\eta$ and to take the shape $\exp[- \eta \pen(m)] \mathds{1}_{m \subset G}$, we obtain the following theorem.

\begin{thm}{\textsc{Self informative Bayes carrier of the hyper-parameter in a hierarchical sieve prior II}\\}\label{thm2.2.1}
Using the modified prior which depends on $\eta$, the support of the self informative Bayes carrier for the hyper-parameter $M$ is
\[\argmax\limits_{m \subset G} \{\Upsilon(Y, m) - \pen(Y, m)\}.\]
\end{thm}

\begin{pro}{\textsc{Proof of \nref{thm2.2.1}}\\}\label{pro2.2.1}
For any finite set $P$ of subsets of $G$ such that $\max\limits_{m \in P} \Upsilon(Y, m) - \pen(Y, m) < \max\limits_{k \subset G} \Upsilon(Y, k) - \pen(Y, k)$, we can write

\begin{alignat*}{3}
& \P_{M\vert Y}^{\eta}(P) && = && \sum\limits_{m \in P} \frac{1}{\sum\limits_{j \subset G}\exp\left[\eta \left(\Upsilon(Y, j) - \Upsilon(Y, m) - \left(\pen(j) - \pen(m)\right)\right)\right]} \mathds{1}_{m \subset G}\\
& && \leq && \frac{Card(P)}{\exp\left[\eta \left(\max\limits_{j \subset G}\left(\Upsilon(Y, j) - \pen(j)\right) - \max\limits_{m \in P}\left(\Upsilon(Y, m) - \pen(m)\right)\right)\right]} \mathds{1}_{m \subset G}\\
& && \rightarrow && 0.
\end{alignat*}
\qedsymbol
\end{pro}


The posterior distribution for $\boldsymbol{\theta}^{M}$ itself follows:

\begin{alignat*}{3}
&\frac{d\mathds{Q}_{\boldsymbol{\theta}^{M} \vert Y}}{d\P^{\circ}}(\theta, y) && \propto && \frac{d\P_{\boldsymbol{\theta}^{M}, Y}}{d\mathds{Q}^{\circ} \, d\P^{\circ}}(\theta, y)\\
& &&\propto&& \sum\limits_{m \subset J} \frac{d\P_{\boldsymbol{\theta}^{M}, Y, M}}{d\mathds{Q}^{\circ} \, d\P^{\circ} \, d\P^{\circ}}(\theta, y, m)\\
& &&\propto&& \sum\limits_{m \subset J} \frac{d\P_{\boldsymbol{\theta}^{M} \vert Y, M}}{d\mathds{Q}^{\circ}}(\theta, y, m) \frac{d\P_{Y, M}}{d\P^{\circ} \, d\P^{\circ}}\\
& &&\propto&& \sum\limits_{m \subset J} \frac{d\P_{\boldsymbol{\theta}^{m} \vert Y}}{d\mathds{Q}^{\circ}}(\theta, y, m) \frac{d\P_{M \vert Y}}{d\P^{\circ}} \frac{d\P_{Y}}{d\P^{\circ}}(Y)\\
& &&=&& \sum\limits_{m \subset J} \frac{d\P_{\boldsymbol{\theta}^{m} \vert Y}}{d\mathds{Q}^{\circ}}(\theta, y, m) \frac{d\P_{M \vert Y}}{d\P^{\circ}}.
\end{alignat*}

From this, we can deduce the iterated posterior distribution for $\boldsymbol{\theta}^{M}$:
\begin{alignat*}{3}
& \frac{d\mathds{Q}_{\boldsymbol{\theta}^{M} \vert Y}^{\eta}}{d\P^{\circ}}(\theta, y) && = && \sum\limits_{m \subset G} \frac{d\P_{\boldsymbol{\theta}^{m} \vert Y}^{\eta}}{d\mathds{Q}^{\circ}}(\theta, y, m) \frac{d\P_{M \vert Y}^{\eta}}{d\P^{\circ}}(m, y)\\
& && = && \sum\limits_{m \subset G} \frac{\exp\left[-\left(\frac{1}{2}\sum\limits_{j \in m} \vert \boldsymbol{\theta}_{j} \vert^{2} + \eta l(\boldsymbol{\theta}, y)\right)\right] \cdot \prod\limits_{j \notin m} \delta_{0}(\boldsymbol{\theta}_{j})}{\int_{\Theta_{m}} \exp\left[-\left(\frac{1}{2}\sum\limits_{j \in m} \vert \mu_{j} \vert^{2} + \eta l(\mu, y)\right)\right] d\mu} \frac{\exp[-\pen(m) + \eta \Upsilon(Y, m)]}{\sum\limits_{j \subset G}\exp[- \pen(j) + \eta \Upsilon(Y, j)]} \mathds{1}_{m \subset G}\\
\end{alignat*}

And as a consequence, we can deduce the self informative Bayes carrier.

\begin{thm}{\textsc{Self informative carrier using a hierarchical sieve prior}\\}
Denote $\widehat{m} := \argmax\limits_{m \subset G} \{\Upsilon(Y, m) - \pen(m)\}$ then the support of the self informative Bayes carrier is contained in $\argmax\limits_{\theta \in \Theta_{m}, m \in \widehat{m}}\{-l(\theta, Y)\}$.
\end{thm}

We have hence seen in these two first sections investigated the behaviour of the sieve prior and its hierarchical version under the iterative asymptotic and shown that under some mild assumptions, their self informative Bayes carriers correspond to some constrained maximum likelihood estimator and penalised contrast model selection version of it respectively.

We should now investigate the behaviour of these (iterated) priors under the noise asymptotic and define hypotheses under which they behave properly.
\section{Proof strategies for contraction rates}\label{BAYES_STRATEGIES}

In this section, we depict two proof strategies for contraction rates.
They will be used in the next sections to compute contraction rates for sieve and hierarchical sieve priors respectively.

The first proof relies on moment bounding of the random variable $\Vert \boldsymbol{\theta} - \theta^{\circ} \Vert$.
The second proof relies on the use of exponential concentration inequalities.

\subsection{A moment control based method for contraction rate computation}\label{BAYES_STRATEGIES_MOMENT}

In this section we outline a method to prove contraction rates which requires to bound properly some moments of the posterior distribution.
We later use this method in the case of the inverse Gaussian sequence space with a sieve prior.
Provided that bounds are available for the required moments, this method barely needs any other assumption on the model.
Moreover, it appears that, in the example we display here, it leads to the same rate as the frequentist optimal convergence rate without a logarithmic loss as it is often the case with popular methods.

A limitation is that moments of posterior distributions are not always explicitly available, in particular for non conjugate prior.
A consequence is that we were not able to use this method for the deconvolution model nor for computation of contraction rate of the hierarchical prior.

However, we believe that the method could be generalised to wider cases, for example using convergence of distribution in Wasserstein distance implying convergence of moments.

A similar method to obtain lower bounds is described in annex.
Unfortunately, it could not be used in any practical case here.

\bigskip

For all this section, $\Phi_{n}$ is the sequence which we want to prove to be a contraction rate; it is in general a function of $\theta^{\circ}$ but we do not make this dependence appear in this section as it has no influence on the proof.

\begin{lm}\label{LM_BAYES_STRATEGIES_UPPEREXPEC}{\textsc{Upper bound for posterior expectation}\\}
Assume $\max\left\{ \E^{n}_{\theta^{\circ}}\left[\E^{n}_{\boldsymbol{\theta} \vert Y^{n}}\left[\Vert \boldsymbol{\theta} - \theta^{\circ} \Vert\right]\right], \sqrt{\V^{n}_{\theta^{\circ}}\left[ \E^{n}_{\boldsymbol{\theta} \vert Y^{n}}\left[\Vert \boldsymbol{\theta} - \theta^{\circ} \Vert\right]\right]} \right\} \in \mathcal{O}(\Phi_{n})$.
Then, for any increasing unbounded sequence $c_{n}$, we have:
\[\lim\limits_{n \rightarrow \infty} \mathds{P}_{\theta^{\circ}}^{n}\left(\E_{\boldsymbol{\theta} \vert Y^{n}}^{n}\left[\Vert \boldsymbol{\theta} - \theta^{\circ} \Vert \right] \geq c_{n}\Phi_{n}\right) = 0.\]
\end{lm}

\begin{pro}\label{PRO_BAYES_STRATEGIES_UPPEREXPEC}{\textsc{Proof of \nref{LM_BAYES_STRATEGIES_UPPEREXPEC}}\\}
Define the sequence of random variables $\mathcal{S}_{n} := \frac{\E_{\boldsymbol{\theta} \vert Y^{n}}^{n}\left[\Vert \boldsymbol{\theta} - \theta^{\circ} \Vert \right] - \E^{n}_{\theta^{\circ}}\left[\E_{\boldsymbol{\theta} \vert Y^{n}}^{n}\left[\Vert \boldsymbol{\theta} - \theta^{\circ} \Vert \right]\right]}{\sqrt{\V^{n}_{\theta^{\circ}}\left[\E_{\boldsymbol{\theta} \vert Y^{n}}^{n}\left[\Vert \boldsymbol{\theta} - \theta^{\circ} \Vert \right]\right]}}$.
This is a sequence of random variables with common expectation $0$ and variance $1$ and, as such, their distributions form a sequence of tight measures.
Hence, for any increasing unbounded sequence $c_{n}$ and $K_{n} := \E^{n}_{\theta^{\circ}}\left[\E_{\boldsymbol{\theta} \vert Y^{n}}^{n}\left[\Vert \boldsymbol{\theta} - \theta^{\circ} \Vert \right]\right] + c_{n} \sqrt{\V^{n}_{\theta^{\circ}}\left[\E_{\boldsymbol{\theta} \vert Y^{n}}^{n}\left[\Vert \boldsymbol{\theta} - \theta^{\circ} \Vert \right]\right]}$ we can write
\begin{alignat*}{3}
& \mathds{P}_{\theta^{\circ}}^{n}\left(\E_{\boldsymbol{\theta} \vert Y^{n}}^{n}\left[\Vert \boldsymbol{\theta} - \theta^{\circ} \Vert \right] \geq K_{n}\right) && = && \mathds{P}_{\theta^{\circ}}^{n}\left(S_{n} \geq \frac{K_{n} - \E^{n}_{\theta^{\circ}}\left[\E_{\boldsymbol{\theta} \vert Y^{n}}^{n}\left[\Vert \boldsymbol{\theta} - \theta^{\circ} \Vert \right]\right]}{\sqrt{\V^{n}_{\theta^{\circ}}\left[\E_{\boldsymbol{\theta} \vert Y^{n}}^{n}\left[\Vert \boldsymbol{\theta} - \theta^{\circ} \Vert \right]\right]}}\right)\\
& &&=&& \mathds{P}_{\theta^{\circ}}^{n}\left(S_{n} \geq c_{n}\right)
\end{alignat*}
which tends to $0$ as $S_{n}$ is tight.
\qedsymbol
\end{pro}

\begin{lm}\label{LM_BAYES_STRATEGIES_UPPERVAR}{\textsc{Upper bound for posterior variance}\\}
Assume $\max\left\{ \E^{n}_{\theta^{\circ}}\left[\sqrt{\V^{n}_{\boldsymbol{\theta} \vert Y^{n}}\left[\Vert \boldsymbol{\theta} - \theta^{\circ} \Vert\right]}\right], \sqrt{\V^{n}_{\theta^{\circ}}\left[ \sqrt{\V^{n}_{\boldsymbol{\theta} \vert Y^{n}}\left[\Vert \boldsymbol{\theta} - \theta^{\circ} \Vert\right]}\right]} \right\} \in \mathcal{O}(\Phi_{n})$.
Then, for any increasing unbounded sequence $c_{n}$, we have:
\[\lim\limits_{n \rightarrow \infty} \mathds{P}_{\theta^{\circ}}^{n}\left(\sqrt{\V_{\boldsymbol{\theta} \vert Y^{n}}^{n}\left[\Vert \boldsymbol{\theta} - \theta^{\circ} \Vert \right]} \geq c_{n}\Phi_{n}\right) = 0.\]
\end{lm}

\begin{pro}\label{PRO_BAYES_STRATEGIES_UPPERVAR}{\textsc{Proof of \nref{LM_BAYES_STRATEGIES_UPPERVAR}}\\}
Define the sequence of random variables $\mathcal{S}_{n} := \frac{\sqrt{\V_{\boldsymbol{\theta} \vert Y^{n}}^{n}\left[\Vert \boldsymbol{\theta} - \theta^{\circ} \Vert \right]} - \E^{n}_{\theta^{\circ}}\left[\sqrt{\V_{\boldsymbol{\theta} \vert Y^{n}}^{n}\left[\Vert \boldsymbol{\theta} - \theta^{\circ} \Vert \right]}\right]}{\sqrt{\V^{n}_{\theta^{\circ}}\left[\sqrt{\V_{\boldsymbol{\theta} \vert Y^{n}}^{n}\left[\Vert \boldsymbol{\theta} - \theta^{\circ} \Vert \right]}\right]}}$.
This is a sequence of random variables with common expectation $0$ and variance $1$ and, as such, their distributions for a sequence of tight measures.
Hence, for any increasing unbounded sequence $c_{n}$ and $K_{n} := \E^{n}_{\theta^{\circ}}\left[\sqrt{\V_{\boldsymbol{\theta} \vert Y^{n}}^{n}\left[\Vert \boldsymbol{\theta} - \theta^{\circ} \Vert \right]}\right] + c_{n} \sqrt{\V^{n}_{\theta^{\circ}}\left[\sqrt{\V_{\boldsymbol{\theta} \vert Y^{n}}^{n}\left[\Vert \boldsymbol{\theta} - \theta^{\circ} \Vert \right]}\right]}$ we can write
\begin{alignat*}{3}
& \mathds{P}_{\theta^{\circ}}^{n}\left(\sqrt{\V_{\boldsymbol{\theta} \vert Y^{n}}^{n}\left[\Vert \boldsymbol{\theta} - \theta^{\circ} \Vert \right]} \geq K_{n}\right) && = && \mathds{P}_{\theta^{\circ}}^{n}\left(S_{n} \geq \frac{K_{n} - \E^{n}_{\theta^{\circ}}\left[\sqrt{\V_{\boldsymbol{\theta} \vert Y^{n}}^{n}\left[\Vert \boldsymbol{\theta} - \theta^{\circ} \Vert \right]}\right]}{\sqrt{\V^{n}_{\theta^{\circ}}\left[\sqrt{\V_{\boldsymbol{\theta} \vert Y^{n}}^{n}\left[\Vert \boldsymbol{\theta} - \theta^{\circ} \Vert \right]}\right]}}\right)\\
& &&=&& \mathds{P}_{\theta^{\circ}}^{n}\left(S_{n} \geq c_{n}\right)
\end{alignat*}
which tends to $0$ as $S_{n}$ is tight.
\qedsymbol
\end{pro}

\begin{thm}\label{THM_BAYES_STRATEGIES_MOMENT}{\textsc{Upper bound}\\}
Under the hypotheses of \nref{LM_BAYES_STRATEGIES_UPPEREXPEC} and \nref{LM_BAYES_STRATEGIES_UPPERVAR} we have for any increasing unbounded sequence $c_{n}$
\[\lim\limits_{n \rightarrow \infty} \E_{\theta^{\circ}}^{n}\left[\mathds{P}^{n}_{\boldsymbol{\theta} \vert Y^{n}}\left(\Vert \boldsymbol{\theta} - \theta^{\circ} \Vert > c_{n}\Phi_{n} \right)\right] = 0.\]
\end{thm}

\begin{pro}\label{PRO_BAYES_STRATEGIES_MOMENT}{\textsc{Proof of \nref{THM_BAYES_STRATEGIES_MOMENT}}\\}
Define the tight sequence of random variables $S_{n} := \frac{\Vert \boldsymbol{\theta} - \theta^{\circ} \Vert - \E^{n}_{\boldsymbol{\theta} \vert Y^{n}}\left[\Vert \boldsymbol{\theta} - \theta^{\circ} \Vert\right]}{\sqrt{\V^{n}_{\boldsymbol{\theta} \vert Y^{n}}\left[\Vert \boldsymbol{\theta} - \theta^{\circ} \Vert\right]}}$.
We consider the sequence of events $\Omega_{n} := \left\{\left\{\E^{n}_{\boldsymbol{\theta} \vert Y^{n}}\left[\Vert \boldsymbol{\theta} - \theta^{\circ} \Vert\right] \geq c_{n} \Phi_{n}\right\} \cap \left\{\sqrt{\V^{n}_{\boldsymbol{\theta} \vert Y^{n}}\left[\Vert \boldsymbol{\theta} - \theta^{\circ} \Vert\right]} \geq c_{n} \Phi_{n}\right\}\right\}$.
We have $\mathds{P}^{n}_{\theta^{\circ}}(\Omega_{n}) \leq \max\left(\mathds{P}^{n}_{\theta^{\circ}}\left(\left\{\E_{\boldsymbol{\theta} \vert Y^{n}}\left[\Vert \boldsymbol{\theta} - \theta^{\circ} \Vert\right] \geq c_{n} \Phi_{n}\right\}\right), \mathds{P}^{n}_{\theta^{\circ}}\left(\left\{\sqrt{\V^{n}_{\boldsymbol{\theta} \vert Y^{n}}\left[\Vert \boldsymbol{\theta} - \theta^{\circ} \Vert\right]} \geq c_{n} \Phi_{n}\right\}\right)\right)$ which hence tends to $0$.
Hence, for $K_{n} := c_{n} \Phi_{n} (1 + c_{n})$, we can write
\begin{alignat*}{3}
& \E_{\theta^{\circ}}^{n}\left[\mathds{P}^{n}_{\boldsymbol{\theta} \vert Y^{n}}\left(\Vert \boldsymbol{\theta} - \theta^{\circ} \Vert > K_{n} \right)\right] && = && \E_{\theta^{\circ}}^{n}\left[\mathds{P}^{n}_{\boldsymbol{\theta} \vert Y^{n}}\left(S_{n} > \frac{K_{n} - \E^{n}_{\boldsymbol{\theta} \vert Y^{n}}\left[\Vert \boldsymbol{\theta} - \theta^{\circ} \Vert\right]}{\sqrt{\V^{n}_{\boldsymbol{\theta} \vert Y^{n}}\left[\Vert \boldsymbol{\theta} - \theta^{\circ} \Vert\right]}} \right)\right]\\
& && = && \E_{\theta^{\circ}}^{n}\left[\mathds{1}_{\Omega_{n}}\mathds{P}^{n}_{\boldsymbol{\theta} \vert Y^{n}}\left(S_{n} > \frac{K_{n} - \E^{n}_{\boldsymbol{\theta} \vert Y^{n}}\left[\Vert \boldsymbol{\theta} - \theta^{\circ} \Vert\right]}{\sqrt{\V^{n}_{\boldsymbol{\theta} \vert Y^{n}}\left[\Vert \boldsymbol{\theta} - \theta^{\circ} \Vert\right]}} \right)\right]\\
& && && + \E_{\theta^{\circ}}^{n}\left[\mathds{1}_{\Omega_{n}^{c}}\mathds{P}^{n}_{\boldsymbol{\theta} \vert Y^{n}}\left(S_{n} > \frac{K_{n} - \E^{n}_{\boldsymbol{\theta} \vert Y^{n}}\left[\Vert \boldsymbol{\theta} - \theta^{\circ} \Vert\right]}{\sqrt{\V^{n}_{\boldsymbol{\theta} \vert Y^{n}}\left[\Vert \boldsymbol{\theta} - \theta^{\circ} \Vert\right]}} \right)\right]\\
& && \leq && \mathds{P}_{\theta^{\circ}}^{n}\left(\Omega_{n}\right) + \mathds{P}^{n}_{\theta^{\circ}}\left(\Omega_{n}^{c}\right) \cdot \E_{\theta^{\circ}}^{n}\left[\mathds{P}^{n}_{\boldsymbol{\theta} \vert Y^{n}}\left(S_{n} > \frac{K_{n} - c_{n} \Phi_{n}}{c_{n} \Phi_{n}} \right)\right]\\
& && \leq && \mathds{P}_{\theta^{\circ}}^{n}\left(\Omega_{n}\right) + \E_{\theta^{\circ}}^{n}\left[\mathds{P}^{n}_{\boldsymbol{\theta} \vert Y^{n}}\left(S_{n} > c_{n} \right)\right].
\end{alignat*}
We can conclude as $S_{n}$ is a tight sequence, $c_{n}$ tends to infinity and $\mathds{P}_{\theta^{\circ}}^{n}\left(\Omega_{n}\right)$ tends to $0$.
\qedsymbol
\end{pro}

\subsection{An exponential concentration inequality based proof for contraction rates of hierarchical sieve priors}\label{BAYES_STRATEGIES_EXPO}

We give here the structure of the proof we use to prove the optimality of the (finally) iterated hierarchical sieve prior.
This method takes advantage of the structure of the hierarchical prior and the additive form of the $l^{2}$ norm.
It is similar to the one used in \ncite{JJASRS}.

\begin{as}{\textsc{Non asymptotic loading of small sets} \\}\label{AS_BAYES_STRATEGIES_EXPO_SMALLSET}
There exist a sequence of sets $G^{-}_{n} \subset G^{\circ}_{n}$ and a sequence of real numbers $K_{A, n}$ such that the sequence of events $\mathcal{A}_{m, n} := \{\Upsilon^{\eta}(y, G^{\circ}_{n} \setminus m) < K_{A, n}\}$ verifies
\begin{alignat*}{3}
& \sum\limits_{m \subset G^{-}_{n}} \exp\left[\eta\left(K_{A, n} - (\pen(m) - \pen(G^{\circ}_{n}))\right)\right] && \in && \mathfrak{o}_{n}(1)\\
& \sum\limits_{m \subset G^{-}_{n}}\mathds{P}_{\theta^{\circ}}^{n}\left[ \mathcal{A}_{m, n}^{c}\right] && \in && \mathfrak{o}_{n}(1)\\
\end{alignat*}
\end{as}

\begin{as}{\textsc{Non asymptotic loading of large sets} \\}\label{AS_BAYES_STRATEGIES_EXPO_LARGESET}
There exist a sequence of sets $G^{+}_{n} \supset G^{\circ}_{n}$ and a sequence of real numbers $K_{B, n}$ such that the sequence of events $\mathcal{B}_{m, n} := \{\Upsilon^{\eta}(y, m \setminus G^{\circ}_{n}) < K_{B, n}\}$ verifies
\begin{alignat*}{3}
& \sum\limits_{m \subset G^{-}_{n}} \exp\left[\eta\left(K_{B, n} - (\pen(m) - \pen(G^{\circ}_{n}))\right)\right] && \in && \mathfrak{o}_{n}(1)\\
& \sum\limits_{m \subset G^{-}_{n}}\mathds{P}_{\theta^{\circ}}^{n}\left[ \mathcal{B}_{m}^{c}\right] && \in && \mathfrak{o}_{n}(1)\\
\end{alignat*}
\end{as}

\begin{as}{\textsc{Optimal contraction of proper sieves} \\}\label{AS_BAYES_STRATEGIES_EXPO_OPT}
With the notations of \nref{AS_BAYES_STRATEGIES_EXPO_SMALLSET} and \nref{AS_BAYES_STRATEGIES_EXPO_LARGESET} assume
\[\sum\limits_{G^{-}_{n} \subset m \subset G^{+}_{n}} \E_{\theta^{\circ}}^{n}\left[\P_{\boldsymbol{\theta}^{m} \vert Y^{n}}^{n, (\eta)} \left(\left\Vert \boldsymbol{\theta}^{m} - \theta^{\circ}_{j} \right\Vert_{l^{2}}^{2} > \Phi_{n}\right)\right] \in \mathfrak{o}_{n}(1)\]
\end{as}

Note that those assumptions are generally obtained using concentration inequalities such as the one displayed in \nref{USEFULRESULTS}.

\begin{thm}{\textsc{Contraction rate for iterated posterior of hierarchical Gaussian sieve priors} \\}\label{THM_BAYES_STRATEGIES_EXPO}
Under \nref{AS_BAYES_STRATEGIES_EXPO_SMALLSET}, \nref{AS_BAYES_STRATEGIES_EXPO_LARGESET}, and \nref{AS_BAYES_STRATEGIES_EXPO_SMALLSET}, for any $\eta$ in $\llbracket 1, \infty \llbracket$ there exists a constant $K$ such that
\[\lim\limits_{n \rightarrow \infty} \E_{\theta^{\circ}}^{n} \left[\P_{\boldsymbol{\theta}^{M} \vert Y^{n}}^{n, (\eta)}\left(\left\Vert \boldsymbol{\theta}^{M} - \theta^{\circ} \right\Vert_{l^{2}}^{2} \geq K \Phi_{n}\right)\right] = 0.\]
\end{thm}

\begin{pro}{\textsc{Proof of \nref{THM_BAYES_STRATEGIES_EXPO}} \\}\label{PRO_BAYES_STRATEGIES_EXPO}
First, notice that we have the following decomposition:
\begin{alignat*}{3}
& \E_{\theta^{\circ}}^{n}\left[\P^{n, (\eta)}_{\boldsymbol{\theta}^{M} \vert Y^{n}}\left(\left\Vert  \boldsymbol{\theta}^{M} - \theta^{\circ}\right\Vert_{l^{2}}^{2} > \Phi_{n} \right)\right] && = && \E_{\theta^{\circ}}^{n}\left[\sum\limits_{m \subset G_{n}}\P^{n, (\eta)}_{\boldsymbol{\theta}^{M} \vert Y^{n}}\left(\left\{\left\Vert  \boldsymbol{\theta}^{M} - \theta^{\circ}\right\Vert_{l^{2}}^{2} > \Phi_{n}\right\} \cap \left\{ M = m \right\} \right)\right]\\
& && = && \sum\limits_{m \subset G_{n}}\E_{\theta^{\circ}}^{n}\left[\P^{n, (\eta)}_{\boldsymbol{\theta}^{M} \vert Y^{n}, M = m}\left(\left\{\left\Vert  \boldsymbol{\theta}^{M} - \theta^{\circ}\right\Vert_{l^{2}}^{2} > \Phi_{n}\right\}\right) \cdot \mathds{P}_{M \vert Y^{n}}^{n, (\eta)}\left(M = m\right)\right]\\
& && = && \sum\limits_{m \subset G_{n}}\E_{\theta^{\circ}}^{n}\left[\P^{n, (\eta)}_{\boldsymbol{\theta}^{m} \vert Y^{n}}\left(\left\{\left\Vert  \boldsymbol{\theta}^{m} - \theta^{\circ}\right\Vert_{l^{2}}^{2} > \Phi_{n}\right\}\right) \cdot \mathds{P}_{M \vert Y^{n}}^{n, (\eta)}\left(M = m\right)\right]\\.
\end{alignat*}

Then, for any three subsets $G^{\circ}_{n}$, $G^{+}_{n}$ and $G^{-}_{n}$ with $G^{-}_{n} \subset G^{\circ}_{n} \subset G^{+}_{n} \subset G_{n}$, we have

\begin{alignat*}{4}
& \E_{\theta^{\circ}}^{n}\left[\P^{n, (\eta)}_{\boldsymbol{\theta}^{M} \vert Y^{n}}\left(\left\Vert  \boldsymbol{\theta} - \theta^{\circ}\right\Vert_{l^{2}}^{2} > \Phi_{n} \right)\right] && \leq && \sum\limits_{m \subset G^{-}_{n}}\E_{\theta^{\circ}}^{n} &&\left[\mathds{P}_{M \vert Y^{n}}^{n, (\eta)}\left(M = m\right)\right] + \sum\limits_{m \supset G^{+}_{n}}\E_{\theta^{\circ}}^{n}\left[\mathds{P}_{M \vert Y^{n}}^{n, (\eta)}\left(M = m\right)\right]\\
& && && && +  \sum\limits_{G^{-}_{n} \subset m \subset G^{+}_{n}}\E_{\theta^{\circ}}^{n}\left[\P^{n, (\eta)}_{\boldsymbol{\theta}^{m} \vert Y^{n}}\left(\left\{\left\Vert  \boldsymbol{\theta}^{m} - \theta^{\circ}\right\Vert_{l^{2}}^{2} > \Phi_{n}\right\}\right)\right]\\
& && \leq && &&\underbrace{\E_{\theta^{\circ}}^{n}\left[\mathds{P}_{M \vert Y^{n}}^{n, (\eta)}\left(M \subset G^{-}_{n}\right)\right]}_{=: A} + \underbrace{\E_{\theta^{\circ}}^{n}\left[\mathds{P}_{M \vert Y^{n}}^{n, (\eta)}\left(M \supset G^{+}_{n}\right)\right]}_{=: B}\\
& && && && +  \sum\limits_{G^{-}_{n} \subset m \subset G^{+}_{n}}\underbrace{\E_{\theta^{\circ}}^{n}\left[\P^{n, (\eta)}_{\boldsymbol{\theta}^{m} \vert Y^{n}}\left(\left\{\left\Vert  \boldsymbol{\theta}^{m} - \theta^{\circ}\right\Vert_{l^{2}}^{2} > \Phi_{n}\right\}\right)\right]}_{=:C_{m}}
\end{alignat*}

The goal is then to control the three sums using concentration inequalities.

We begin with $A$, where the conclusion is given by \nref{AS_BAYES_STRATEGIES_EXPO_LARGESET}:
\begin{alignat*}{4}
& A && = && \sum\limits_{m \subset G^{-}_{n}} && \mathds{E}_{\theta^{\circ}}^{n}\left[\frac{\exp\left[\eta\left(-\pen(m) + \Upsilon^{\eta}(y, m)\right)\right]}{\sum\limits_{j \subset G}\exp\left[\eta \left(- \pen(j) + \Upsilon^{\eta}(y, j)\right)\right]} \mathds{1}_{\mathcal{A}_{m}}\right] + \\
& && && && \mathds{E}_{\theta^{\circ}}^{n}\left[\frac{\exp\left[\eta\left(-\pen(m) + \Upsilon^{\eta}(y, m)\right)\right]}{\sum\limits_{j \subset G}\exp\left[\eta \left(- \pen(j) + \Upsilon^{\eta}(y, j)\right)\right]} \mathds{1}_{\mathcal{A}_{m}^{c}}\right]\\
& && \leq && \sum\limits_{m \subset G^{-}_{n}} && \mathds{E}_{\theta^{\circ}}^{n}\left[\exp\left[\eta\left(\left(\Upsilon^{\eta}(y, G^{\circ}_{n}) - \Upsilon^{\eta}(y, m)\right) - (\pen(m) - \pen(G^{\circ}_{n}))\right)\right] \mathds{1}_{\mathcal{A}_{m}}\right] + \\
& && && && \mathds{P}_{\theta^{\circ}}^{n}\left[ \mathcal{A}_{m}^{c}\right]\\
& && \leq && \sum\limits_{m \subset G^{-}_{n}} && \mathds{E}_{\theta^{\circ}}^{n}\left[\exp\left[\eta\left(\Upsilon^{\eta}(y, G^{\circ}_{n} \setminus m) - (\pen(m) - \pen(G^{\circ}_{n}))\right)\right] \mathds{1}_{\mathcal{A}_{m}}\right] + \mathds{P}_{\theta^{\circ}}^{n}\left[ \mathcal{A}_{m}^{c}\right]\\
& && \leq && \sum\limits_{m \subset G^{-}_{n}} && \exp\left[\eta\left(K_{A, n} - (\pen(m) - \pen(G^{\circ}_{n}))\right)\right] + \mathds{P}_{\theta^{\circ}}^{n}\left[ \mathcal{A}_{m}^{c}\right]\\
& && \in && \mathfrak{o}_{n}(1)
\end{alignat*}

\medskip

We process similarly for $B$, where the conclusion is given by \nref{AS_BAYES_STRATEGIES_EXPO_LARGESET}:
\begin{alignat*}{4}
& B && = && \sum\limits_{m \supset G^{+}_{n}} && \mathds{E}_{\theta^{\circ}}^{n}\left[\frac{\exp\left[\eta\left(-\pen(m) + \Upsilon^{\eta}(y, m)\right)\right]}{\sum\limits_{j \subset G}\exp\left[\eta \left(- \pen(j) + \Upsilon^{\eta}(y, j)\right)\right]} \mathds{1}_{\mathcal{B}_{m}}\right] + \\
& && && && \mathds{E}_{\theta^{\circ}}^{n}\left[\frac{\exp\left[\eta\left(-\pen(m) + \Upsilon^{\eta}(y, m)\right)\right]}{\sum\limits_{j \subset G}\exp\left[\eta \left(- \pen(j) + \Upsilon^{\eta}(y, j)\right)\right]} \mathds{1}_{\mathcal{B}_{m}^{c}}\right]\\
& && \leq && \sum\limits_{m \supset G^{+}_{n}} && \mathds{E}_{\theta^{\circ}}^{n}\left[\exp\left[\eta\left(\left(\Upsilon^{\eta}(y, G^{\circ}_{n}) - \Upsilon^{\eta}(y, m)\right) - (\pen(m) - \pen(G^{\circ}_{n}))\right)\right] \mathds{1}_{\mathcal{B}_{m}}\right] + \\
& && && && \mathds{P}_{\theta^{\circ}}^{n}\left[ \mathcal{B}_{m}^{c}\right]\\
& && \leq && \sum\limits_{m \supset G^{+}_{n}} && \mathds{E}_{\theta^{\circ}}^{n}\left[\exp\left[\eta\left(- \Upsilon^{\eta}(y, m \setminus G^{\circ}_{n}) - (\pen(m) - \pen(G^{\circ}_{n}))\right)\right] \mathds{1}_{\mathcal{B}_{m}}\right] + \mathds{P}_{\theta^{\circ}}^{n}\left[ \mathcal{B}_{m}^{c}\right]\\
& && \leq && \sum\limits_{m \supset G^{+}_{n}} && \exp\left[\eta\left(K_{B, n} - (\pen(m) - \pen(G^{\circ}_{n}))\right)\right] + \mathds{P}_{\theta^{\circ}}^{n}\left[ \mathcal{B}_{m}^{c}\right]\\
& && \in && \mathfrak{o}_{n}(1)
\end{alignat*}

\medskip

Finally, $C_{m}$ is directly controlled by \nref{AS_BAYES_STRATEGIES_EXPO_OPT}.

\qedsymbol
\end{pro}

\subsection{An exponential concentration inequality based proof for contraction rates of self informative Bayes carrier of hierarchical sieve priors}\label{BAYES_STRATEGIES_EXPOLIM}

In the previous section, we described the kind of proof used in \ncite{JJASRS} and argued that it can also be used with a finitely iterated posterior.
We present here an adaptation of this scheme for the self informative Bayes carrier.
The main subtlety lies in the fact that the hyper-parameter only loads extrema of a penalised contrast function.

\begin{as}{\textsc{Non asymptotic loading of small sets, revisited} \\}\label{AS_BAYES_STRATEGIES_EXPOLIM_SMALLSET}
There exist a sequence of sets $G^{-}_{n} \subset G^{\circ}_{n}$ such that
\[\sum\limits_{m \subset G^{-}_{n}} \P_{\theta^{\circ}}^{n}\left(- \Upsilon(G^{\circ}_{n} \setminus m, Y^{n}) < \pen(G^{\circ}_{n} - \pen(m))\right) \in \mathfrak{o}_{n}(1)\]
\end{as}

\begin{as}{\textsc{Non asymptotic loading of large sets, revisited} \\}\label{AS_BAYES_STRATEGIES_EXPOLIM_LARGESET}
There exist a sequence of sets $G^{+}_{n} \supset G^{\circ}_{n}$ such that
\[\sum\limits_{m \supset G^{+}_{n}} \P_{\theta^{\circ}}^{n}\left(\Upsilon(m \setminus G^{\circ}_{n}, Y^{n}) < \pen(G^{\circ}_{n} - \pen(m))\right) \in \mathfrak{o}_{n}(1)\]
\end{as}

\begin{as}{\textsc{Optimal contraction of proper sieves, revisited} \\}\label{AS_BAYES_STRATEGIES_EXPOLIM_OPT}
With the notations of \nref{AS_BAYES_STRATEGIES_EXPOLIM_SMALLSET} and \nref{AS_BAYES_STRATEGIES_EXPOLIM_LARGESET} assume
\[\sum\limits_{G^{-}_{n} \subset m \subset G^{+}_{n}} \P_{\theta^{\circ}}^{n}\left[\left\Vert \overline{\theta}^{m} - \theta^{\circ}_{j} \right\Vert_{l^{2}}^{2} > \Phi_{n}\right] \in \mathfrak{o}_{n}(1)\]
\end{as}

Note that those assumptions are generally obtained using concentration inequalities such as the one displayed in \nref{USEFULRESULTS}.

\begin{thm}{\textsc{Contraction rate for iterated posterior of hierarchical Gaussian sieve priors} \\}\label{THM_BAYES_STRATEGIES_EXPOLIM}
Under \nref{AS_BAYES_STRATEGIES_EXPOLIM_SMALLSET}, \nref{AS_BAYES_STRATEGIES_EXPOLIM_LARGESET}, and \nref{AS_BAYES_STRATEGIES_EXPOLIM_OPT}, there exists a constant $K$ such that
\[\lim\limits_{n \rightarrow \infty} \E_{\theta^{\circ}}^{n} \left[\P_{\boldsymbol{\theta}^{M} \vert Y^{n}}^{n, (\infty)}\left(\left\Vert \boldsymbol{\theta}^{M} - \theta^{\circ} \right\Vert_{l^{2}}^{2} \geq K \Phi_{n}\right)\right] = 0.\]
\end{thm}

\begin{pro}{\textsc{Proof of \nref{THM_BAYES_STRATEGIES_EXPOLIM}} \\}\label{PRO_BAYES_STRATEGIES_EXPOLIM}
We start the proof in a similar fashion to \nref{THM_BAYES_STRATEGIES_EXPO}:
\begin{alignat*}{3}
& \E_{\theta^{\circ}}^{n}\left[\P^{n, (\infty)}_{\boldsymbol{\theta}^{M} \vert Y^{n}}\left(\left\Vert  \boldsymbol{\theta}^{M} - \theta^{\circ}\right\Vert_{l^{2}}^{2} > \Phi_{n} \right)\right] && = && \E_{\theta^{\circ}}^{n}\left[\sum\limits_{m \subset G_{n}}\P^{n, (\infty)}_{\boldsymbol{\theta}^{M} \vert Y^{n}}\left(\left\{\left\Vert  \boldsymbol{\theta}^{M} - \theta^{\circ}\right\Vert_{l^{2}}^{2} > \Phi_{n}\right\} \cap \left\{ M = m \right\} \right)\right]\\
& && = && \sum\limits_{m \subset G_{n}}\E_{\theta^{\circ}}^{n}\left[\P^{n, (\infty)}_{\boldsymbol{\theta}^{M} \vert Y^{n}, M = m}\left(\left\{\left\Vert  \boldsymbol{\theta}^{M} - \theta^{\circ}\right\Vert_{l^{2}}^{2} > \Phi_{n}\right\}\right) \cdot \mathds{P}_{M \vert Y^{n}}^{n, (\infty)}\left(M = m\right)\right]\\
& && = && \sum\limits_{m \subset G_{n}}\E_{\theta^{\circ}}^{n}\left[\P^{n, (\infty)}_{\boldsymbol{\theta}^{m} \vert Y^{n}}\left(\left\{\left\Vert  \boldsymbol{\theta}^{m} - \theta^{\circ}\right\Vert_{l^{2}}^{2} > \Phi_{n}\right\}\right) \cdot \mathds{P}_{M \vert Y^{n}}^{n, (\infty)}\left(M = m\right)\right]\\.
\end{alignat*}

Then, for any three subsets $G^{\circ}_{n}$, $G^{+}_{n}$ and $G^{-}_{n}$ with $G^{-}_{n} \subset G^{\circ}_{n} \subset G^{+}_{n} \subset G_{n}$, we have

\begin{alignat*}{4}
& \E_{\theta^{\circ}}^{n}\left[\P^{n, (\infty)}_{\boldsymbol{\theta}^{M} \vert Y^{n}}\left(\left\Vert  \boldsymbol{\theta} - \theta^{\circ}\right\Vert_{l^{2}}^{2} > \Phi_{n} \right)\right] && \leq && \sum\limits_{m \subset G^{-}_{n}}\E_{\theta^{\circ}}^{n} &&\left[\mathds{P}_{M \vert Y^{n}}^{n, (\infty)}\left(M = m\right)\right] + \sum\limits_{m \supset G^{+}_{n}}\E_{\theta^{\circ}}^{n}\left[\mathds{P}_{M \vert Y^{n}}^{n, (\infty)}\left(M = m\right)\right]\\
& && && && +  \sum\limits_{G^{-}_{n} \subset m \subset G^{+}_{n}}\E_{\theta^{\circ}}^{n}\left[\P^{n, (\infty)}_{\boldsymbol{\theta}^{m} \vert Y^{n}}\left(\left\{\left\Vert  \boldsymbol{\theta}^{m} - \theta^{\circ}\right\Vert_{l^{2}}^{2} > \Phi_{n}\right\}\right)\right]\\
& && \leq && &&\underbrace{\E_{\theta^{\circ}}^{n}\left[\mathds{P}_{M \vert Y^{n}}^{n, (\eta)}\left(M \subset G^{-}_{n}\right)\right]}_{=: A} + \underbrace{\E_{\theta^{\circ}}^{n}\left[\mathds{P}_{M \vert Y^{n}}^{n, (\eta)}\left(M \supset G^{+}_{n}\right)\right]}_{=: B}\\
& && && && +  \sum\limits_{G^{-}_{n} \subset m \subset G^{+}_{n}}\underbrace{\E_{\theta^{\circ}}^{n}\left[\P^{n, (\eta)}_{\boldsymbol{\theta}^{m} \vert Y^{n}}\left(\left\{\left\Vert  \boldsymbol{\theta}^{m} - \theta^{\circ}\right\Vert_{l^{2}}^{2} > \Phi_{n}\right\}\right)\right]}_{=:C_{m}}
\end{alignat*}

The goal is then to control the three sums using concentration inequalities.

We begin with $A$, the conclusion is given by \nref{AS_BAYES_STRATEGIES_EXPOLIM_SMALLSET}:

\begin{alignat*}{3}
& A && = && \P_{\theta^{\circ}}^{n}\left[ \forall l \supset G^{-}_{n}, \pen(\widehat{m}) + \Upsilon(\widehat{m}, Y) < \pen(l) + \Upsilon(l, Y) \right]\\
& && \leq && \P_{\theta^{\circ}}^{n}\left[ \exists m \subset G^{-}_{n}, \pen(m) + \Upsilon(m, Y) < \pen(G^{\circ}_{n}) + \Upsilon(G^{\circ}_{n}, Y) \right]\\
& && \leq && \sum\limits_{m \subset G^{-}_{n}}\P_{\theta^{\circ}}^{n}\left[\pen(m) + \Upsilon(m, Y) < \pen(G^{\circ}_{n}) + \Upsilon(G^{\circ}_{n}, Y) \right]\\
& && \leq && \sum\limits_{m \subset G^{-}_{n}}\P_{\theta^{\circ}}^{n}\left[- \Upsilon(G^{\circ}_{n} \setminus m, Y) < \pen(G^{\circ}_{n}) - \pen(m)\right]\\
& && \in && \mathfrak{o}_{n}(1).
\end{alignat*}

\medskip

We process similarly for $B$, the conclusion is given by \nref{AS_BAYES_STRATEGIES_EXPOLIM_LARGESET}:

\begin{alignat*}{4}
& B && = && \P_{\theta^{\circ}}^{n}\left[ \forall l \subset G^{+}_{n}, \pen(\widehat{m}) + \Upsilon(\widehat{m}, Y) < \pen(l) + \Upsilon(l, Y) \right]\\
& && \leq && \P_{\theta^{\circ}}^{n}\left[ \exists m \supset G^{+}_{n}, \pen(m) + \Upsilon(m, Y) < \pen(G^{\circ}_{n}) + \Upsilon(G^{\circ}_{n}, Y) \right]\\
& && \leq && \sum\limits_{m \supset G^{+}_{n}}\P_{\theta^{\circ}}^{n}\left[\pen(m) + \Upsilon(m, Y) < \pen(G^{\circ}_{n}) + \Upsilon(G^{\circ}_{n}, Y) \right]\\
& && \leq && \sum\limits_{m \supset G^{+}_{n}}\P_{\theta^{\circ}}^{n}\left[\Upsilon(m \setminus G^{\circ}_{n}, Y) < \pen(G^{\circ}_{n}) - \pen(m)\right]\\
& && \in && \mathfrak{o}_{n}(1).
\end{alignat*}

\medskip

Finally, $C_{m}$ is directly controlled by \nref{AS_BAYES_STRATEGIES_EXPOLIM_OPT}.

\qedsymbol
\end{pro}
\section{Application to the inverse Gaussian sequence space model}\label{BAYES_GAUSS}

In this section, we consider the inverse Gaussian sequence space model as introduced in \nref{INTRO_IGSSM_DE}.
First, we investigate about the self informative Bayes limit/carrier of (hierarchical) Gaussian sieve priors using the technics presented in \nref{THM_BAYES_SIEVE_SELF_INFORMATIVE}, \nref{THM_BAYES_HIERARCHICAL_LIMITTHRESHOLD} and \nref{THM_BAYES_HIERARCHICAL_LIMIT}.
Then, we use the methodology described in \nref{BAYES_STRATEGIES_MOMENT} to compute upper bounds of the Gaussian sieve priors described in \nref{BAYES_SIEVE} when applied to this specific model.
Doing so, we will notice that it gives us, for a general case, the same speed as the convergence rate of projection estimators and that, by choosing properly the threshold parameter, we reach the oracle rate of convergence as well as the minimax optimal rate, \textbf{without a} $\boldsymbol{\log}$\textbf{-loss}.
However, we also see that this strategy cannot be applied to the hierarchical priors we are interested in.
Hence, we then use the strategy exposed in \nref{BAYES_STRATEGIES_EXPO} and show that under some regularity conditions, the iterated hierarchical prior leads to optimal posterior contraction rate.
As a consequence, we can conclude about the oracle and minimax speed of convergence of the penalised contrast model selection estimator with a new strategy of proof.

\subsection{Self informative Bayes carrier for Gaussian sieve in iGSSM}\label{BAYES_GAUSS_SELFINFORM}
We first consider the asymptotic $\eta \rightarrow \infty$ for the Gaussian sieve prior.

\begin{thm}\label{THM_BAYES_GAUSS_SELFINFORM}
For a Gaussian sieve prior with threshold parameter $m$, the self informative Bayes carrier is the singleton given by:
$\theta_{n, \overline{m}} = \left(\theta_{n, \overline{m}}(s)\right)_{s \in \N} = \left(\phi_{n}(s)\lambda^{-1}(s) \mathds{1}_{\vert s \vert \leq m}\right)_{s \in \N}$.
\reEnd
\end{thm}

\begin{pro}{\textsc{Proof of \nref{THM_BAYES_GAUSS_SELFINFORM}} \\}\label{PRO_BAYES_GAUSS_SELFINFORM}
In this model, we explicitly have that $\mathds{M} = \N$; in addition, for any $\theta$ in $\Theta_{\overline{m}}$, and $\phi_{n}$ in $\Theta$, there exists $C$ only depending on $\phi_{n}$ and $n$ such that,
\[l(\theta, \phi_{n}) = -n^{-1/2}\left(\sum\nolimits_{s \leq m} \phi_{n}(s) \lambda(s) \theta(s) - \sum\nolimits_{s \leq m} \Lambda(s)^{-1} \theta(s)^{2}/2 \right) + C;\]
which is continuous with respect to $\theta$; therefore, \nref{AS_BAYES_SIEVE_CONTINUOUS} is verified.

We can hence apply \nref{THM_BAYES_SIEVE_SELF_INFORMATIVE} which proves that the support of the self informative Bayes carrier is contained in the set of maximisers of $l(\theta, \phi_{n})$ which is obviously the singleton $\{\left(\theta_{n}(s)\lambda^{-1}(s) \mathds{1}_{\vert s \vert \leq m}\right)_{s \in \N}\}$.
\proEnd
\end{pro}

As an alternative, one could have noticed that the prior and likelihood are conjugated.

Define for any $s$ in $\N$ and $\eta$ in $\N^{\star}$ the quantities
\[\widetilde{\theta}^{(\eta)}(s) := (n \eta \phi_{n}(s) \lambda(s))/(1 + n \eta \lambda(s)^{2}); \quad \sigma^{(\eta)}(s) := (1 + n \eta \lambda(s)^{2})^{-1}.\]
Then, for any $s$ in $\N$, the posterior distribution of $\boldsymbol{\theta}(s)$ after $\eta$ iterations is given by
\[\P_{\boldsymbol{\theta}(s) \vert \phi_{n}}^{(\eta)} = \mathcal{N}(\widetilde{\theta}^{(\eta)}(s), \sigma^{(\eta)}(s)) \mathds{1}_{\vert s \vert \leq m} + \delta_{0}(\boldsymbol{\theta}(s)) \mathds{1}_{\vert s \vert > m}.\]
Considering the respective limits of $\widetilde{\theta}^{(\eta)}(s)$ and $\sigma^{(\eta)}(s)$ as $\eta$ tends to $\infty$ for any $s$ in $\N$ coincides with our previous statement.

\subsection{Contraction rate for Gaussian sieve in iGSSM}\label{BAYES_GAUSS_CONTRACT}
We now investigate the behaviour of the Gaussian sieve prior applied to iGSSM as $n$ tends to $\infty$.
In this context, it is interesting to let $\eta$ and $m$ depend on $n$; we hence note $\eta_{n}$ and $m_{n}$.

\medskip

First consider the strategy exposed in \nref{BAYES_STRATEGIES_MOMENT}.
To apply it, we will place ourselves under the following hypothesis that apparently limits the possible choices for the threshold parameter.
In practice, the thresholds which are left aside would be too large and are known to lead to a poor estimation performance.

\begin{as}\label{AS_BAYES_GAUSS_CONTRACT_THRESHOLD}
Assume that $m_{n}$ and $\eta_{n}$ are chosen in such a way that either of
\begin{itemize}
\item $\sum\nolimits_{s \leq m_{n}} \Lambda(s)n^{-1} \in \mathcal{O}_{n}(1)$
\item $\sum\nolimits_{ s \leq m_{n}} (\Lambda(s) \vert\theta^{\circ}(s)\vert)^{2}(n \eta_{n})^{-2} \in \mathcal{O}_{n}\left(\sum\nolimits_{ s \leq m_{n}} \Lambda(s) n^{-1}\right)$ and

$\sum\nolimits_{ s \leq m_{n}} (\Lambda(s)^{3/2} \left\vert\theta^{\circ}(s)\right\vert)(n^{3/2} \eta_{n})^{-1} \in \mathcal{O}_{n}\left(\sum\nolimits_{ s \leq m_{n}} \Lambda(s)n^{-1}\right)$
\end{itemize}
stand true.
\assEnd
\end{as}

We illustrate this hypothesis under the typical behaviours of $\theta$ and $\lambda$

\begin{il}
Consider the first inclusion $\sum\nolimits_{s \leq m_{n}} \Lambda(s)/n \in \mathcal{O}_{n}(1)$.

Notice that \ref{oo:xdf:p} and \ref{oo:xdf:np} have no influence here.
\begin{Liste}[]
\item[\ref{il:po}, \ref{il:oo}, and \ref{il:so}] we have $\sum\nolimits_{s \leq m_{n}} \Lambda(s)/n = n^{-1} m \Lambda_{\circ}(m_{n}) \approx n^{-1} m_{n}^{2 a + 1}$ and hence the first inclusion is equivalent to $m_{n} \in \mathcal{O}_{n}(n^{1/(2 a + 1)})$.
\item[\ref{il:ps}, and \ref{il:os}] we have $\sum\nolimits_{s \leq m_{n}} \Lambda(s)/n = n^{-1} m \Lambda_{\circ}(m_{n}) \approx n^{-1} m_{n}^{-(1 -2a)_{+}} \exp[m_{n}^{2a}]$ and hence the first inclusion is equivalent to $m_{n} \in \mathcal{O}_{n}(\log(n)^{1/(2 a)})$.
\end{Liste}
In the second inclusion, $\sum\nolimits_{s \leq m_{n}} (\Lambda(s) \vert\theta^{\circ}(s)\vert)^{2}/(n \eta_{n})^{2} \in \mathcal{O}\left(\sum\nolimits_{ s \leq m_{n}} \Lambda(s)/n\right)$, the regularity of $\theta$ also intervenes.
Notice that, under \ref{il:oo} and \ref{il:os}, $\sum\nolimits_{s \leq m_{n}} \Lambda(s)/n \approx n^{-1} m_{n}^{2 a + 1}$ while under \ref{il:so} we have $\sum\nolimits_{s \leq m_{n}} \Lambda(s)/n \approx n^{-1} m_{n}^{-(1 -2a)_{+}} \exp[m_{n}^{2a}]$.
\begin{Liste}[]
\item[\ref{oo:xdf:p}] $\sum\nolimits_{s \leq m_{n}} (\Lambda(s) \vert\theta^{\circ}(s)\vert)^{2}/(n \eta_{n})^{2} \leq \sum\nolimits_{s \leq K} (\Lambda(s) \vert\theta^{\circ}(s)\vert)^{2}/(n \eta_{n})^{2} \in \mathfrak{o}_{n}(n^{-1})$ and hence the inclusion is always verified.
\item[\ref{oo:xdf:np}] We now have to distinguish the different regularities of $\theta$ and $\lambda$.
In any case, notice that $\sum\nolimits_{s \leq m_{n}} (\Lambda(s) \vert\theta^{\circ}(s)\vert)^{2}/(n \eta_{n})^{2} \leq (n \eta_{n})^{-2} \sum\nolimits_{s \leq m_{n}} \Lambda(s)^{2} \cdot (\Vert \theta^{\circ} \Vert_{l^{2}}^{2} - \b_{m}^{2}(\theta^{\circ}))$
\begin{Liste}[]
\item[\ref{il:oo}] $(n \eta_{n})^{-2} \sum\limits_{s \leq m_{n}} \Lambda(s)^{2} \cdot \sum\limits_{s \leq m_{n}} \vert\theta^{\circ}(s)\vert^{2} \approx (n \eta_{n})^{-2} \cdot m^{4a + 1}$ implies $m_{n} \in \mathcal{O}_{n}(n^{1/(2a)} \eta_{n}^{1/a})$;
\item[\ref{il:os}] $(n \eta_{n})^{-2} \sum\limits_{s \leq m_{n}} \Lambda(s)^{2} \cdot \sum\limits_{s \leq m_{n}} \vert\theta^{\circ}(s)\vert^{2} \approx (n \eta_{n})^{-2} m^{-(1 - 4a)_{+}} \exp[m^{4 a}]$ implies $m_{n} \in \mathcal{O}_{n}(\log(n \eta_{n}^{2})^{1/(4a)})$;
\item[\ref{il:so}] $(n \eta_{n})^{-2} \sum\limits_{s \leq m_{n}} \Lambda(s)^{2} \cdot \sum\limits_{s \leq m_{n}} \vert\theta^{\circ}(s)\vert^{2} \approx (n \eta_{n})^{-2} \cdot m^{4a + 1}$ implies $m_{n} \in \mathcal{O}_{n}(n^{1/(2a)} \eta_{n}^{1/a})$.
\end{Liste}
\end{Liste}
Finally, for the third inclusion $\sum\nolimits_{ s \leq m_{n}} (\Lambda(s)^{3/2} \left\vert\theta^{\circ}(s)\right\vert)/(n^{3/2} \eta_{n}) \in \mathcal{O}\left(\sum\nolimits_{ s \leq m_{n}} \Lambda(s)/n\right)$ notice that we have $\sum\nolimits_{ s \leq m_{n}} (\Lambda(s)^{3/2} \left\vert\theta^{\circ}(s)\right\vert)/(n^{3/2} \eta_{n}) \leq (n^{3/2} \eta_{n})^{-1} \cdot \sum\nolimits_{ s \leq m_{n}} \Lambda(s)^{3} \cdot (\Vert \theta^{\circ} \Vert_{l^{2}}^{2} - \b_{m_{n}}^{2}(\theta^{\circ}))$.
Under \ref{il:oo} and \ref{il:os}, $\sum\nolimits_{s \leq m_{n}} \Lambda(s)/n \approx n^{-1} m_{n}^{2 a + 1}$ while under \ref{il:so} we have $\sum\nolimits_{s \leq m_{n}} \Lambda(s)/n \approx n^{-1} m_{n}^{-(1 -2a)_{+}} \exp[m_{n}^{2a}]$.
\begin{Liste}[]
\item[\ref{oo:xdf:p}] $\sum\nolimits_{ s \leq m_{n}} (\Lambda(s)^{3/2} \left\vert\theta^{\circ}(s)\right\vert)/(n^{3/2} \eta_{n}) \leq (n^{3/2} \eta_{n})^{-1}\sum\nolimits_{ s \leq K} (\Lambda(s)^{3/2} \left\vert\theta^{\circ}(s)\right\vert) \in \mathfrak{o}_{n}(n^{-1})$ and hence the inclusion is always verified.
\item[\ref{oo:xdf:np}] We now have to distinguish the different regularities of $\theta$ and $\lambda$.
\begin{Liste}[]
\item[\ref{il:oo}] $(n^{3/2} \eta_{n})^{-1} \cdot \sum\nolimits_{ s \leq m_{n}} \Lambda(s)^{3} \cdot (\Vert \theta^{\circ} \Vert_{l^{2}}^{2} - \b_{m_{n}}^{2}(\theta^{\circ})) \approx (n^{3/2} \eta_{n})^{-1} \cdot m^{6 a + 1}$ implies $m_{n} \in \mathcal{O}_{n}((\eta_{n} \sqrt{n})^{1/(4a)})$;
\item[\ref{il:os}] $(n^{3/2} \eta_{n})^{-1} \cdot \sum\nolimits_{ s \leq m_{n}} \Lambda(s)^{3} \cdot (\Vert \theta^{\circ} \Vert_{l^{2}}^{2} - \b_{m_{n}}^{2}(\theta^{\circ})) \approx (n^{3/2} \eta_{n})^{-1} \cdot m^{-(1 - 6a)_{+}} \exp[m^{6a}]$ implies $m_{n} \in \mathcal{O}_{n}(\log(\sqrt{n} \eta_{n})1{1/(6a)})$;
\item[\ref{il:so}] $(n^{3/2} \eta_{n})^{-1} \cdot \sum\nolimits_{ s \leq m_{n}} \Lambda(s)^{3} \cdot (\Vert \theta^{\circ} \Vert_{l^{2}}^{2} - \b_{m_{n}}^{2}(\theta^{\circ})) \approx (n^{3/2} \eta_{n})^{-1} \cdot m^{6 a + 1}$ implies $m_{n} \in \mathcal{O}_{n}((\eta_{n} \sqrt{n})^{1/(4a)})$.
\end{Liste}
\end{Liste}
\ilEnd
\end{il}
We see that in any case, one can chose the sequence $(\eta_{n})_{n \in \N}$ in such a way that the condition is weaker that $m_{n} \in \mathcal{O}_{n}(n)$; unfortunately, this choice generally depends on the ill-posedness parameter $a$ and adaptively chosing $\eta$ is not considered here.

Under this hypothesis we can obtain the contraction rate we hoped for.

\begin{cor}\label{COR_BAYES_GAUSS_CONTRACT_SIEVE}
Under \nref{AS_BAYES_GAUSS_CONTRACT_THRESHOLD}, for any $\theta^{\circ}$ in $\Theta$ and increasing, unbounded sequence $c_{n}$, we have
\begin{alignat*}{3}
& && \lim\nolimits_{n \rightarrow \infty} \E&&\left[\P_{\boldsymbol{\theta}_{\overline{m_{n}}}\vert \phi_{n}}^{(\eta)}\left(\left\Vert \theta^{\circ} - \boldsymbol{\theta}_{\overline{m_{n}}}\right\Vert_{l^{2}}^{2} \leq c_{n} \Phi^{m_{n}}_{n} \right)\right] = 1.
\end{alignat*}
\reEnd
\end{cor}

\begin{pro}{\textsc{Proof of \nref{COR_BAYES_GAUSS_CONTRACT_SIEVE}}\\}\label{PRO_BAYES_GAUSS_CONTRACT_SIEVE}
Remind that, for any $s$ in $\N$, $\phi_{n}(s) = \phi(s) + n^{-1/2} \xi(s)$, where $(\xi(s))_{s \in \N}$ is an \iid Gaussian white noise process.
We will apply \nref{THM_BAYES_STRATEGIES_MOMENT} and hence need to show:
\begin{alignat*}{3}
& \E\left[\E_{\boldsymbol{\theta} \vert \phi_{n}}\left[\Vert \boldsymbol{\theta} - \theta^{\circ} \Vert_{l^{2}}^{2}\right]\right] && \in && \mathcal{O}_{n}(\Phi_{n}^{m_{n}}); \quad \V\left[\E_{\boldsymbol{\theta}\vert \phi_{n}}\left[\Vert \boldsymbol{\theta} - \theta^{\circ} \Vert_{l^{2}}^{2}\right]\right]^{1/2} \in \mathcal{O}_{n}(\Phi_{n}^{m_{n}});\\
&\E\left[\V_{\boldsymbol{\theta}\vert \phi_{n}}\left[\Vert \boldsymbol{\theta} - \theta^{\circ} \Vert_{l^{2}}^{2}\right]^{1/2}\right] && \in && \mathcal{O}_{n}(\Phi_{n}^{m_{n}}); \quad \V\left[\V_{\boldsymbol{\theta}\vert \phi_{n}}\left[\Vert \boldsymbol{\theta} - \theta^{\circ} \Vert_{l^{2}}^{2}\right]^{1/2}\right]^{1/2} \in \mathcal{O}_{n}(\Phi_{n}^{m_{n}}).
\end{alignat*}
We use the fact that $\Vert \boldsymbol{\theta} - \theta^{\circ} \Vert_{l^{2}}^{2} = \sum\nolimits_{\vert s \vert \leq m_{n}} \left( \boldsymbol{\theta}(s) - \theta^{\circ}(s)\right)^{2} + \mathfrak{b}_{m_{n}}^{2}(\theta^{\circ})$ and that we know the distribution of $\boldsymbol{\theta}(s)$.
This gives us the expectation and variance of the posterior distribution of $\Vert \boldsymbol{\theta} - \theta^{\circ} \Vert_{l^{2}}^{2}$.
We use in addition $(1 + \Lambda(s)/(n \eta_{n}))^{-1} \leq 1$ to obtain upper bounds for these quantities.
\begin{alignat*}{2}
&\E &&_{\boldsymbol{\theta}_{\overline{m_{n}}} \vert \phi_{n}}\left[\Vert \boldsymbol{\theta} - \theta^{\circ} \Vert_{l^{2}}^{2}\right] = \sum\limits_{\vert s \vert \leq m_{n}} \left(\frac{\Lambda(s)/(n \eta_{n})}{\Lambda(s)/(n \eta_{n}) + 1}\right)\left(1 + \frac{\left(- \theta^{\circ}(s) + \eta_{n} \sqrt{n} \xi(s) \lambda(s)\right)^{2}}{(\eta_{n} n)/\Lambda(s) + 1}\right) + \mathfrak{b}_{m_{n}}^{2}\\
& && \leq \sum\limits_{\vert s \vert \leq m_{n}} (\Lambda(s)/n \eta_{n}) + \sum\nolimits_{\vert s \vert \leq m_{n}} (\Lambda(s)/(n \eta_{n}))^{2}\left(- \theta^{\circ}(s) + \eta_{n} \sqrt{n} \xi(s) \lambda(s)\right)^{2} + \mathfrak{b}_{m_{n}}^{2}(\theta^{\circ});
\end{alignat*}
\begin{alignat*}{2}
&\V&&_{\boldsymbol{\theta}_{\overline{m_{n}}} \vert \phi_{n}}\left[\Vert \boldsymbol{\theta} - \theta^{\circ} \Vert_{l^{2}}^{2}\right] = 2 \sum\limits_{\vert s \vert \leq m_{n}} \left(\frac{\Lambda(s)/(n \eta_{n})}{\Lambda(s)/(n \eta_{n}) + 1}\right)^{2}\left(1 + 2 \frac{\left(- \theta^{\circ}(s) + \eta_{n} \sqrt{n} \xi(s) \lambda(s)\right)^{2}}{(\eta_{n} n)/\Lambda(s) + 1}\right)\\
& &&\leq 2 \sum\nolimits_{\vert s \vert \leq m_{n}} (\Lambda(s)/(n \eta_{n}))^{2} + 4 \sum\nolimits_{\vert s \vert \leq m_{n}} (\Lambda(s)/(n \eta_{n}))^{3} \left(- \theta^{\circ}(s) + \eta_{n} \sqrt{n} \xi(s) \lambda(s)\right)^{2}.
\end{alignat*}
In addition, we use the sub-additivity of the square root to obtain this upper bound:
\begin{alignat*}{1}
& \V_{\boldsymbol{\theta}_{\overline{m_{n}}} \vert \phi_{n}}\left[\Vert \boldsymbol{\theta} - \theta^{\circ} \Vert_{l^{2}}^{2}\right]^{1/2}\\
& \leq \sqrt{2} \sum\nolimits_{\vert s \vert \leq m_{n}} \Lambda(s)/(n \eta_{n}) + 2 \sum\nolimits_{\vert s \vert \leq m_{n}} (\Lambda(s)/(n \eta_{n}))^{3/2} \left\vert- \theta^{\circ}(s) + \eta_{n} \sqrt{n} \xi(s) \lambda(s)\right\vert.
\end{alignat*}
Using linearity of the expectation and the standard Gaussian distribution of $\xi_{j}$ we have:
\begin{alignat*}{2}
&\E&&\left[\E_{\boldsymbol{\theta}_{\overline{m_{n}}} \vert \phi_{n}}\left[\Vert \boldsymbol{\theta} - \theta^{\circ} \Vert_{l^{2}}^{2}\right]\right]\\
& &&\leq \sum\nolimits_{\vert s \vert \leq m_{n}} \Lambda(s)/(n \eta_{n}) + \sum\nolimits_{\vert s \vert \leq m_{n}} \Lambda(s)/n + \sum\nolimits_{\vert s \vert \leq m_{n}}(\Lambda(s)/(n \eta_{n}))^{2} \left\vert \theta^{\circ}(s) \right\vert^{2} + \mathfrak{b}_{m_{n}}^{2}(\theta^{\circ}).
\end{alignat*}
The same properties give us this bound:
\begin{alignat*}{3}
&\V\left[\E_{\boldsymbol{\theta}_{\overline{m_{n}}} \vert \phi_{n}}\left[\Vert \boldsymbol{\theta} - \theta^{\circ} \Vert_{l^{2}}^{2}\right]\right] &&\leq&& 2 \sum\nolimits_{\vert s \vert \leq m_{n}} (\Lambda(s)/n)^{2} + 4 \sum\nolimits_{\vert s \vert \leq m_{n}}(\Lambda(s)^{3}/(\eta_{n}^{2} n^{3})) \left\vert\theta^{\circ}(s)\right\vert^{2};
\end{alignat*}
and we use the sub-additivity of the square root in addition:
\begin{alignat*}{3}
&\V\left[\E_{\boldsymbol{\theta}_{\overline{m_{n}}} \vert \phi_{n}}\left[\Vert \boldsymbol{\theta} - \theta^{\circ} \Vert_{l^{2}}^{2}\right]\right]^{1/2} &&\leq&& \sqrt{2}\sum\nolimits_{\vert s \vert \leq m_{n}} (\Lambda(s)/n) + 2 \sum\nolimits_{\vert s \vert \leq m_{n}}(\Lambda(s)^{3/2}/(\eta_{n} n^{3/2})) \left\vert\theta^{\circ}(s)\right\vert.\\
\end{alignat*}
To control the moments of the posterior variance, we use the properties of the folded Gaussian random variables:
\begin{alignat*}{2}
&\E&&\left[\V_{\boldsymbol{\theta}_{\overline{m_{n}}} \vert \phi_{n}}\left[\Vert \boldsymbol{\theta} - \theta^{\circ} \Vert_{l^{2}}^{2}\right]^{1/2}\right]\\
%& && \leq \sqrt{2} \sum\nolimits_{\vert s \vert \leq m_{n}} (\Lambda(s)/(n \eta_{n})) + 2 \sum\nolimits_{\vert s \vert \leq m_{n}} (\Lambda(s)/(n \eta_{n}))^{3/2} \E\left[\left\vert- \theta^{\circ}(s) + \eta_{n} \sqrt{n} \xi(s) \lambda(s)\right\vert\right]\\
%& && \leq \sqrt{2} \sum\nolimits_{\vert s \vert \leq m_{n}} (\Lambda(s)/(n \eta_{n})) + 2 \sum\nolimits_{\vert s \vert \leq m_{n}} (\Lambda(s)/(n \eta_{n}))^{3/2} \left(\eta_{n} \vert \lambda(s) \vert (2/\pi)^{1/2} \exp\left[-(\left\vert\theta(s)^{\circ}\right\vert^{2} \Lambda(s))/(2 \eta_{n}^{2})\right]\right.\\
%& && \left.+ \vert \theta(s)^{\circ} \vert \erf\left\{\vert \theta^{\circ}(s)\vert / (\sqrt{2} \eta_{n} \vert \lambda(s)\vert )\right\}\right)\\
& && \leq \sqrt{2} \sum\limits_{\vert s \vert \leq m_{n}} \Lambda(s)/(n \eta_{n}) + 2 \sum\limits_{\vert s \vert \leq m_{n}} (2/(\pi \cdot n^{3} \eta_{n}))^{1/2}\Lambda(s) \exp\left[-(\left(\theta^{\circ}(s)\right)^{2} \Lambda(s))/(2 \eta_{n}^{2})\right]\\
& && + \sum\nolimits_{\vert s \vert \leq m_{n}}(\Lambda(s)/(n \eta_{n}))^{3/2}\vert\theta^{\circ}(s)\vert;
\end{alignat*}
\[\V\left[\V_{\boldsymbol{\theta}_{\overline{m_{n}}} \vert \phi_{n}}\left[\Vert \boldsymbol{\theta}_{\overline{m_{n}}} - \theta^{\circ} \Vert_{l^{2}}^{2}\right]^{1/2}\right] \leq 2 \sum\nolimits_{\vert s \vert \leq m_{n}} (\Lambda(s)/(n \eta_{n}))^{3} \cdot \left[ \left\vert\theta^{\circ}(s)\right\vert^{2} + \eta_{n}^{2}/\Lambda(s)\right];\]
\begin{alignat*}{2}
&\V&&\left[\V_{\boldsymbol{\theta}_{\overline{m_{n}}} \vert \phi_{n}}\left[\Vert \boldsymbol{\theta}_{\overline{m_{n}}} - \theta^{\circ} \Vert_{l^{2}}^{2}\right]^{1/2}\right]^{1/2}\\
& && \leq \sqrt{2} \sum\nolimits_{\vert s \vert \leq m_{n}} ((\Lambda(s)^{3}\left\vert\theta^{\circ}(s)\right\vert^{2})(n \eta_{n})^{-3})^{2} + \sum\nolimits_{\vert s \vert \leq m_{n}} \Lambda(s)/(n^{3} \eta_{n})^{1/2}.
\end{alignat*}
Using \nref{AS_BAYES_GAUSS_CONTRACT_THRESHOLD}, the leading term in each of these bounds is for the most of order $\Phi_{n}^{m_{n}}$ and hence, we can apply \nref{THM_BAYES_STRATEGIES_MOMENT} which proves the statement.
\proEnd
\end{pro}

Notice that if one selects $m_{n} = m_{n}^{\circ}$ we obtain the oracle rate of convergence of projection estimators.

\begin{cor}\label{COR_BAYES_GAUSS_CONTRACT_ORACLESIEVE}
For any $\theta^{\circ}$ in $\Theta$ and increasing, unbounded sequence $c_{n}$, we have
\begin{alignat*}{3}
& && \lim\nolimits_{n \rightarrow \infty} \E&&\left[\P_{\boldsymbol{\theta}_{\overline{m_{n}^{\circ}}}\vert \phi_{n}}^{(\eta)}\left(\left\Vert \theta^{\circ} - \boldsymbol{\theta}_{\overline{m_{n}^{\circ}}}\right\Vert_{l^{2}}^{2} \leq c_{n} \Phi^{\circ}_{n} \right)\right] = 1.
\end{alignat*}
\reEnd
\end{cor}

We have hence seen that Gaussian sieve priors contract around the true parameter at the same rate as the projection estimator with identical threshold parameter contract and that, in particular, the best Gaussian sieve prior contracts at the oracle convergence rate of the projection estimators.

\subsection{Self informative Bayes carrier for the hierarchical prior}\label{BAYES_GAUSS_HIERARCHICAL}
In this subsection, we propose an analytical shape for a hierarchical Gaussian sieve prior to use in the context of an inverse Gaussian sequence space model.

We doubly justify this choice, first by showing that the self informative limit is a penalised contrast maximiser projection estimator and, in the next subsection, that this choice yields good contraction properties.

\medskip

First remind that for any $s$ in $\N$, we have:
\[\widetilde{\theta}^{(\eta)}(s) = (n \eta \phi_{n}(s) \lambda(s)) \cdot (1 + n \eta \lambda(s)^{2})^{-1}; \text{ and }\quad \sigma^{(\eta)}(s) = (1 + n \eta \vert \lambda(s)\vert^{2})^{-1};\]
and define for any $m$ in $\N$ the notations
\[\sigma^{(\eta)}_{\overline{m}} := (\sigma^{(\eta)}(s) \mathds{1}_{\{s \leq m\}})_{s \in \N}; \text{ and }\quad \widetilde{\theta}^{(\eta)}_{\overline{m}} := (\widetilde{\theta}^{(\eta)}(s) \mathds{1}_{\{ s \leq m\}})_{s \in \N}.\]

Then, we define, for any $m$ in $\N$, the quantity $\Lambda_{+}(m) := \max_{ s \leq m} \{\Lambda(s)\}$.
We then take $G_{n} := \max\left\{m \in \llbracket 1, n \rrbracket : \Lambda_{+}(m) / n \leq \Lambda(0)\right\}$.

For any $m$ in $\N$, we make the following choice for the prior distribution of $M$
\[\P_{M}(\{m\}) \propto \exp\left[-\eta/2 \left(3 m + \sum\nolimits_{s = 0}^{m} \log(\sigma^{\eta}(s))\right) \right].\]

Using the notations of \nref{BAYES_HIERARCHICAL} (and keeping in mind the notation for weighted norms given in \nref{INTRO_FREQ_DECISION_LOSSFUNCION} in the context of Sobolev's ellipsoid, and the convention "$0/0 = 0$"), we have 
\begin{alignat*}{3}
& \pen(m) && = && (\eta/2)\left(3 m + \sum\nolimits_{s = 0}^{m} \log(\sigma^{(\eta)}(s))\right);\\
& \Upsilon^{\eta}(Y, m) && = && \sum\nolimits_{s = 0}^{m} n \vert \phi_{n}(s) \vert^{2}\left(\Lambda(s)(n \eta)^{-1} + 1\right)^{-1} + (1/2) \sum\nolimits_{s = 0}^{m} \log(\sigma^{(\eta)}(s)).
\end{alignat*}

Which leads us to the iterated prior of the hyper-parameter:

\[\P_{M \vert \phi_{n}}^{(\eta)}(m) \propto \exp\!\!\left[- (\eta/2)\left( 3 m - n \sum\nolimits_{s \leq m} \vert 
\phi_{n}(s)\vert^{2}(\Lambda(s)(n \eta)^{-1} + 1)^{-1} \right) \right].\]

We can hence simplify our notation in the following way: $\pen(m) = 3 m$ and $\Upsilon^{\eta}(Y, m) = \sum\nolimits_{s \leq m} n \vert \phi_{n}(s) \vert^{2}(\Lambda(s)(n \eta)^{-1} + 1)^{-1}$.
Let us remind that the iterated distribution for $\boldsymbol{\theta}_{\overline{M}}\vert \phi_{n}$ is given by $\P_{\boldsymbol{\theta}_{\overline{M}} \vert \phi_{n}}^{(\eta)} = \sum\nolimits_{m \in \mathds{N}}\P_{\boldsymbol{\theta}_{\overline{m}} \vert \phi_{n}}^{(\eta)} \cdot \P_{M \vert \phi_{n}}^{(\eta)}(m)$.
Hence, according to \nref{THM_BAYES_HIERARCHICAL_LIMIT}, the self informative limit for the hyper-parameter is $\widehat{m} := \argmin_{m \leq G_{n}} \{3 m - n \sum\nolimits_{s \leq m}\vert \phi_{n}(s) \vert^{2}\}$; and the self informative Bayes limit for $\boldsymbol{\theta}_{\overline{M}}$ is the associated projection estimator $\theta_{n, \overline{m}}$.

\medskip

Note that, defining for any $m$ in $\llbracket 1, G_{n} \rrbracket$ the quantity $E(m) = 3 m - n \sum\nolimits_{s \leq m}\vert \phi_{n}(s) \vert^{2}$; for all distinct $k$ and $m$ in $\llbracket 1, G_{n} \rrbracket$, we almost surely have $E(k) - E(m) \neq 0$ since $\Upsilon(k) - \Upsilon(m)$ is a random variable with absolutely continuous distribution with respect to Lebesgue measure and hence, $\P_{\theta^{\circ}}\!\!\left[\{\Upsilon(k) - \Upsilon(m) = \pen(k) - \pen(m)\}\right] = 0$.

\subsection{Contraction rate for the hierarchical prior}\label{BAYES_GAUSS_CONTRACT_HIERARCHICAL}

In this subsection, we discuss the contraction rate of the hierarchical Gaussian iterated posterior distribution by applying the methodology described in \nref{BAYES_STRATEGIES_EXPO}.

The results are similar to the ones obtained in \ncite{JJASRS} but extended to the iterated posterior distribution, included in the case of "$\eta = \infty$", in such a way that it offers a novel proof for optimality of the penalised contrast maximiser projection estimator.

Remind that we defined for any $m$ in $\N$ the quantities $\Lambda_{+}(m) = \max_{\vert s \vert \leq m}\{\Lambda(s)\}$ and $\Lambda_{\circ}(m) = m^{-1} \sum_{\vert s \vert \leq m}\Lambda(s)$.

The results are obtained using the following contraction inequalities, which can be found in this form in \ncite{JJASRS} as a result adapted from \ncite{Birge2001} and \ncite{LaurentLM2012}.
\begin{lm}\label{BAYES_GAUSS_CONTRACT_HIERARCHICAL_LM_PROB}\label{lmA.1.1}
Let $\{X(s)\}_{s \geq 1}$ be independent and normally distributed random variables with real mean $\alpha(s)$ and standard deviation $\beta(s) \geq 0$. For $m \in \mathds{N}$, set $S_{m} := \sum \limits_{s = 1}^{m} X(s)^{2}$ and consider $v_{m} \geq \sum\limits_{s = 1}^{m} \beta(s)^{2}, t_{m} \geq \max \limits_{1 \leq s \leq m} \beta(s)^{2}$ and $r_{m} \geq \sum\limits_{s = 1}^{m} \alpha(s)^{2}$.
Then for all $c \geq 0$, we have
\begin{alignat*}{3}
&\sup\limits_{m \geq 1} \exp\left[\frac{c (c \wedge 1) (v_{m} + 2 r_{m})}{4 t_{m}}\right]\mathds{P}\left(S_{m} - \mathds{E}[S_{m}] \leq - c (v_{m} + 2 r_{m})\right) &&\leq&& 1; \\
&\sup\limits_{m \geq 1} \exp\left[\frac{c (c \wedge 1) (v_{m} + 2 r_{m})}{4 t_{m}}\right]\mathds{P}\left(S_{m} - \mathds{E}[S_{m}] \geq \frac{3 c}{2} (v_{m} + 2 r_{m})\right) &&\leq&& 1.
\end{alignat*}
\reEnd
\end{lm}

\begin{lm}\label{BAYES_GAUSS_CONTRACT_HIERARCHICAL_LM_ESP}\label{lmA.1.2}
Let $\{X(s)\}_{s \geq 1}$ be independent and normally distributed random variables with real mean $\alpha(s)$ and standard deviation $\beta(s) \geq 0$. For $m \in \mathds{N}$, set $S_{m} := \sum \limits_{s = 1}^{m} X(s)^{2}$ and consider $v_{m} \geq \sum\limits_{s = 1}^{m} \beta(s)^{2}, t_{m} \geq \max \limits_{1 \leq s \leq m} \beta(s)^{2}$ and $r_{m} \geq \sum\limits_{s = 1}^{m} \alpha(s)^{2}$.
Then for all $c \geq 0$, we have
\[\sup\limits_{m \geq 1}(6 t_{m})^{-1} \exp\left[\frac{c (v_{m} + 2 r_{m})}{4 t_{m}}\right] \mathds{E}\left[S_{m} - \mathds{E}[S_{m}] - \frac{3}{2} c (v_{m} + 2 r_{m})\right]_{+} \leq 1\]
with $(a)_{+} := (a \vee 0).$
\reEnd
\end{lm}
We will use them to obtain concentration of sums of the shape $\sum\nolimits_{s = m_{1}}^{m_{2}}(\phi_{n}(s) \lambda(s)^{-1} - \theta^{\circ}(s))^{2}$ and $\sum\nolimits_{s = m_{1}}^{m_{2}} \phi_{n}(s)^{2}$.

\medskip

We start by stating the set of assumptions which allow us to obtain our results.

\begin{as}\label{AS_BAYES_GAUSS_CONTRACT_HIERARCHICAL_LAMBDA}
Suppose that $\lambda$ is monotonically and polynomially decreasing, that is, there exist $c$ in $[1, \infty[$ and $a$ in $\mathds{R}_{+}$ such that $\Lambda(m) \approx m^{-2a}$.
\end{as}

This assumption assures that $\Lambda_{+}(m) = \Lambda(m)$ for any $m$ and that there exist a constant $L := L(\lambda)$ in $[1, \infty[$, independent of $\theta^{\circ}$ such that for any sequence $\left(m_{n}\right)_{n \in \mathds{N}^{\star}}$ 
\[\sup\nolimits_{n \in \mathds{N}^{\star}} m_{n} \Lambda(m_{n})(n \Phi_{n}^{m_{n}})^{-1} \leq \sup\nolimits_{n \in \mathds{N}^{\star}} \Lambda(m_{n})/\Lambda_{\circ}(m_{n}) \leq L.\]
It basically requires that we are in the situation \ref{il:oo} or \ref{il:so} and is not valid under \ref{il:os}.

\begin{as}\label{AS_BAYES_GAUSS_CONTRACT_HIERARCHICAL_ORACLE}
Let $\theta^{\circ}$ and $\lambda$ be such that there exists $n^{\circ}$ in $\mathds{N}^{\star}$
\[0 < \kappa^{\circ} := \kappa^{\circ}(\theta^{\circ}, \lambda) := \inf\nolimits_{n \geq n^{\circ}} \left\{\left(\Phi_{n}^{\circ}(\theta^{\circ})\right)^{-1} \left[\mathfrak{b}_{m_{n}^{\circ}} \wedge n^{-1} m_{n}^{\circ} \Lambda_{\circ}(m_{n}^{\circ})\right]\right\} \leq 1\]
\end{as}

\begin{as}\label{AS_BAYES_GAUSS_CONTRACT_HIERARCHICAL_MINIMAX}
Let $\mathfrak{a}$ and $\lambda$ be sequences such that there exists $n^{\star}$ in $\mathds{N}^{\star}$
\[0 < \kappa^{\star} := \kappa^{\star}(\mathfrak{a}, \lambda) := \inf\nolimits_{n > n^{\star}} \left\{\left(\Phi_{n}^{\star}\right)^{-1}\left[\mathfrak{a}_{m_{n}^{\star}} \wedge n^{-1} m_{n}^{\star} \Lambda_{\circ}(m_{n}^{\star}) \right]\right\} \leq 1.\]
\end{as}

The corollaries hereafter generalise the results obtained in \ncite{JJASRS} to finitely iterated posterior distributions.
The proofs are sensibly similar to the original ones and we hence skip them.

\begin{cor}\label{COR_BAYES_GAUSS_CONTRACT_HIERARCHICAL_ORACLE}
Under \nref{AS_BAYES_GAUSS_CONTRACT_HIERARCHICAL_LAMBDA} and \textsc{\cref{AS_BAYES_GAUSS_CONTRACT_HIERARCHICAL_ORACLE}}, if, in addition $\log(G_{n})/m_{n}^{\circ} \rightarrow 0$ as $n \rightarrow \infty$ then with $D^{\circ} := D^{\circ}(\theta^{\circ}, \lambda) = \lceil 5 L/\kappa^{\circ} \rceil$ and $K^{\circ} := 10(2 \vee \Vert \theta^{\circ} \Vert_{l^{2}}^{2})L^{2}(16 \vee D^{\circ} \Lambda_{D^{\circ}})$ we have, for any $\eta$ ($1 \leq \eta < \infty$):
\[\lim\nolimits_{n \rightarrow \infty} \E\left[\P_{\boldsymbol{\theta}_{\overline{M}} \vert \phi_{n}}^{n, (\eta)} \left(\left(K^{\circ}\right)^{-1} \Phi_{n}^{\circ}(\theta^{\circ}) \leq \Vert \theta^{\circ} - \boldsymbol{\theta}_{\overline{M}} \Vert_{l^{2}}^{2} \leq K^{\circ} \Phi_{n}^{\circ}(\theta^{\circ})\right)\right] = 1.\]
\reEnd
\end{cor}

\begin{cor}\label{COR_BAYES_GAUSS_CONTRACT_HIERARCHICAL_MINIMAX}
Under \nref{AS_BAYES_GAUSS_CONTRACT_HIERARCHICAL_LAMBDA} and \nref{AS_BAYES_GAUSS_CONTRACT_HIERARCHICAL_MINIMAX}, if, in addition, $\log(G_{n})/m_{n}^{\star} \rightarrow 0$ as $n \rightarrow \infty$ then, for any $\eta$ ($1 \leq \eta < \infty$)
\begin{itemize}
\item for all $\theta^{\circ}$ in $\Theta_{\mathfrak{a}}(r)$, with $D^{\star} := D^{\star}(\mathfrak{a}, \lambda) = \lceil 5 L/\kappa^{\star} \rceil$ and $K^{\star} := 16 L^{2} (2 \vee r)(16 \vee D^{\star} \Lambda_{D^{\star}})$, we have
\[\lim\nolimits_{n \rightarrow \infty} \E\left[\P_{\boldsymbol{\theta}_{\overline{M}} \vert \phi_{n}}^{n, (\eta)}\left(\Vert \theta^{\circ} - \boldsymbol{\theta}_{\overline{M}} \Vert_{l^{2}}^{2} \leq K^{\star} \Phi_{n}^{\star}\right)\right] =1;\]
\item for any monotonically increasing and unbounded sequence $K_{n}$ holds
\[\lim\nolimits_{n \rightarrow \infty} \inf\nolimits_{\theta^{\circ} \in \Theta_{\mathfrak{a}}(r)} \E\left[\P_{\boldsymbol{\theta}_{\overline{M}} \vert \phi_{n}}^{n, (\eta)}\left(\Vert \theta^{\circ} - \boldsymbol{\theta}_{\overline{M}} \Vert_{l^{2}}^{2} \leq K_{n} \Phi_{n}^{\star}\right)\right] =1.\]
\end{itemize}
\reEnd
\end{cor}

However, the following theorem assert that the results hold true in the asymptotic case where $\eta$ tends to $\infty$.
The proofs are displayed in \nref{PROOF_BAYES_IGSSM_KNOWN_IID_ORACLE_NP} and \nref{PROOF_BAYES_IGSSM_KNOWN_IID_MINIMAX_NP} respectively.

\begin{thm}\label{THM_BAYES_IGSSM_KNOWN_IID_ORACLE_NP}
Under \nref{AS_BAYES_GAUSS_CONTRACT_HIERARCHICAL_LAMBDA}, \textsc{\cref{AS_BAYES_GAUSS_CONTRACT_HIERARCHICAL_ORACLE}} and the condition that $\limsup\nolimits_{n \rightarrow \infty} \log\left(G_{n}\right) (m_{n}^{\circ})^{-1}$, define $D^{\circ} := \left\lceil 3 (\kappa^{\circ})^{-1} + 1 \right\rceil$ and $K^{\circ} := 16 L \cdot \left[9 \vee D^{\circ} \Lambda_{D^{\circ}}\right]$; then, we have for all $\theta^{\circ}$ in $\Theta$,
\[\lim\nolimits_{n \rightarrow \infty} \E\left[\P_{\boldsymbol{\theta}_{\overline{M}} \vert \phi_{n}}^{n, (\infty)}\left(\left(K^{\circ}\right)^{-1} \Phi_{n}^{\circ}(\theta^{\circ}) \leq \Vert \boldsymbol{\theta}_{\overline{M}} - \theta^{\circ} \Vert_{l^{2}}^{2} \leq K^{\circ} \Phi_{n}^{\circ}(\theta^{\circ}) \right)\right] = 1.\]
\end{thm}

\begin{thm}\label{THM_BAYES_IGSSM_KNOWN_IID_MINIMAX_NP}
Under \textsc{\cref{AS_BAYES_GAUSS_CONTRACT_HIERARCHICAL_LAMBDA}}, \textsc{\cref{AS_BAYES_GAUSS_CONTRACT_HIERARCHICAL_MINIMAX}} and the condition that $\limsup\nolimits_{n \rightarrow \infty} \frac{\log\left(G_{n}\right)}{m_{n}^{\star}},$ define $D^{\star} := \left\lceil \frac{3 \left(1 \vee r\right)}{\kappa^{\star} L} + 1 \right\rceil$ and $K^{\star} := 6 (1 \vee r) (9L \vee D^{\star} \Lambda_{D^{\star}})$; then, we have for all $\theta^{\circ}$ in $\Theta^{\mathfrak{a}}(r)$,
\[\lim\nolimits_{n \rightarrow \infty} \E\left[\P_{\boldsymbol{\theta}_{\overline{M}} \vert \phi_{n}}^{n, (\infty)}\left(\Vert \boldsymbol{\theta}_{\overline{M}} - \theta^{\circ} \Vert_{l^{2}}^{2} \leq K^{\star} \Phi_{n}^{\star} \right)\right] = 1,\]
and, for any increasing sequence $K_{n}$ such that $\lim\nolimits_{n \rightarrow \infty} K_{n} = \infty,$
\[\lim\nolimits_{n \rightarrow \infty} \inf\nolimits_{\theta^{\circ} \in \Theta^{\mathfrak{a}}(r)} \E\left[\P_{\boldsymbol{\theta}_{\overline{M}} \vert \phi_{n}}^{n, (\infty)}\left(\Vert \boldsymbol{\theta}_{\overline{M}} - \theta^{\circ} \Vert_{l^{2}}^{2} \leq K_{n} \Phi_{n}^{\star} \right)\right] = 1.\]
\end{thm}


We have hence showed that the self informative Bayes carrier contracts around the true parameter with the oracle optimal rate of sieve priors and with minimax optimal rate over Sobolev's ellipsoids.
We will see in \nref{FREQ_IGSSM} that the self informative limit also converges with optimal rates.
\section{Second application example: the circular density deconvolution model}\label{2.5}
\section{On the shape of the posterior mean}\label{BAYES_POSTMEAN}

We have hence seen that in a general case, considering the asymptotic iteration , the posterior distribution using a sieve prior contracts around the projection estimator and while using a hierarchical prior, the posterior contracts around some penalised contrast maximiser projection estimator.

It is also interesting to note that for any number of iteration $\eta$, the posterior mean can be written both as a shrinkage and as an aggregation estimator.
Indeed, we have

\begin{alignat*}{3}
& \E_{\boldsymbol{\theta}^{M}\vert Y^{n}}^{\eta}\left[\boldsymbol{\theta}^{M}\right] && = && \E_{\boldsymbol{\theta}^{M}\vert Y^{n}}^{\eta}\left[\sum\limits_{m \in G} \boldsymbol{\theta}^{M} \mathds{1}_{M = m}\right]\\
& && = && \sum\limits_{m \in G} \E_{\boldsymbol{\theta}^{M}\vert Y^{n}}^{\eta}\left[ \boldsymbol{\theta}^{M} \mathds{1}_{M = m}\right]\\
& && = && \sum\limits_{m \in G} \mathds{P}_{M \vert Y^{n}}^{\eta}(m = M) \E_{\boldsymbol{\theta}^{m}\vert Y^{n}}^{\eta}\left[ \boldsymbol{\theta}^{m} \right];
\end{alignat*}

and we see here the aggregation form of this estimator.

On the other hand, if we write the expectation of the components individually, we obtain:

\begin{alignat*}{3}
& \E_{\boldsymbol{\theta}^{M}\vert Y^{n}}^{\eta}\left[\boldsymbol{\theta}_{j}^{M}\right] && = && \E_{\boldsymbol{\theta}^{M}\vert Y^{n}}^{\eta}\left[ \boldsymbol{\theta}_{j}^{m} \mathds{1}_{M \geq j}\right]\\
& && = && \mathds{P}_{M \vert Y^{n}}^{\eta}(M \geq j) \E_{\boldsymbol{\theta}^{m}\vert Y^{n}}^{\eta}\left[ \boldsymbol{\theta}^{m}_{j} \right];
\end{alignat*}

where we see the shrinkage property.

\medskip

Aggregation estimates gathered a lot of interest, see for example \ncite{rigollet2007linear}.
While considering such estimators, the goal is to reach the convergence rate of the best estimator contributing to the aggregation.

\medskip

In the next chapter, we hence investigate the properties of this estimator both in inverse Gaussian sequence space model and circular density deconvolution.

%
\chapter{Minimax and oracle optimal adaptive aggregation}\label{3}
We inquire in this chapter the properties of aggregation estimators as introduced in \nref{2.6}.
We introduce first a skim of proof for oracle and minimax optimality of this kind of estimator before applying it to the inverse Gaussian sequence space and the circular deconvolution models respectively introduced in \nref{1.4.1} and \nref{1.4.2}, including in presence of dependance and partially known operator.
%
\subsection{Oracle optimality}\label{FREQ_GAUSS_ORACLE}

A direct application of \nref{lmA.1.1} and \nref{lmA.1.2} gives us the following result.
\begin{cor*}
For any $m$ and $n$ in $\N$, we have
\[\P(\Vert \theta_{n, \overline{m}} - \theta^{\circ}_{\overline{m}} \Vert_{l^{2}}^{2} \geq \penSv / 7) \leq \exp[- \tfrac{\tfrac{2 \kappa}{21} (\tfrac{2 \kappa}{21} \wedge 1)}{4} m \cmiSv];\]
\[\E[(\Vert \theta_{n, \overline{m}} - \theta^{\circ}_{\overline{m}} \Vert_{l^{2}}^{2} - \penSv / 7 )_{+}] \leq 6 n^{-1}\Lambda_{+}(m) \exp[- \tfrac{\kappa}{42} m \cmiSv].\]
\reEnd
\end{cor*}



\begin{thm}\label{THM_FREQ_IGSSM_KNOWN_IID_ORACLE_NP}
\reEnd
\end{thm}

\subsection{Minimax optimality}\label{FREQ_GAUSS_MINIMAX}
  By applying the same strategy, we derive bounds for the maximal risk over
  ellipsoids  $\rwCxdf$ of the aggregated estimator $\txdfAg[{\erWe[]}]$.
  Therefore, we aim next to control the second and
  third right hand side term in \eqref{freq:ge:strat:kn:co:agg:e1} uniformly over
  $\rwCxdf$.
  Keeping the definition \nref{gauss:pen} of
$\daRa{\Di}{\xdfCw[],\Lambda}$  in mind it holds
$\xdfCr^2\dRa{\Di}{\xdfCw[],\Lambda}\geq\Vnormlp{\xdf_{\underline{0}}}^2\bias^2(\xdf)$
uniformly for all $\xdf\in\rwCxdf$ and for all
$\Di\in\Nz$.
\begin{de}
We use $\DipenSv$ and $\penSv$ defined in \nref{gau:pen:oo} and define
  $\daRa{\Di}{(\xdfCw[])}:=\daRa{\Di}{(\xdfCw[],\Lambda)}:=[\xdfCw^2\vee \DipenSv\,\ssY^{-1}]$.
  Then, it holds
  \begin{equation*}
    [\xdfCr^2+\cpen]\daRa{\Di}{(\xdfCw[])}\geq\big[\Vnormlp{\xdf_{\underline{0}}}^2\bias^2(\xdf)\vee\penSv\big]\quad\text{for
      all $\Di\in\nset{1,\ssY}$ and $\xdf\in\rwCxdf$}.
\end{equation*}
And we define the specific choice:
    $\aDi{\ssY}(\xdfCw[]):=\argmin\Nset[\Di\in\Nz]{\daRa{\Di}{(\xdfCw[],\Lambda)}}\in\nset{1,\ssY}$;\\
    $\naRa{(\xdfCw[])}:=\naRa{(\xdfCw[],\Lambda)}:=\min\Nset[\Di\in\Nz]{\daRa{\Di}{(\xdfCw[],\Lambda)}}$
    with $\daRa{\aDi{\ssY}(\xdfCw[])}{(\xdfCw[],\Lambda)}=\naRa{(\xdfCw[],\Lambda)}$.
\assEnd
\end{de}
For any $\pdDi,\mdDi\in\nset{1,\ssY}$ let us define 
\begin{multline*}
\mDi:=\min\set{\Di\in\nset{1,\mdDi}: \Vnormlp{\xdf_{\underline{0}}}^2\bias[\Di]^2(\xdf)\leq
  [\xdfCr^2+4\cpen]\dRa{\mdDi}{\xdfCw[]}}\quad\text{and}\\\pDi:=\max\set{\Di\in\nset{\pdDi,\ssY}:
   \penSv \leq 2[3\xdfCr^2+ 2\cpen] \dRa{\pdDi}{\xdfCw[]}}
\end{multline*}
where  the defining sets obviously contains $\mdDi$ and $\pdDi$, respectively, and hence, they are
not empty.
Hence we have, for any $\theta^{\circ}$ in $\rwCxdf$
 \begin{multline*}
    \E\Vnormlp{\txdfAg-\xdf}^2 \leq 3c([\xdfCr^2+\cpen]\daRa{\pDi}{(\xdfCw[])}) +2r^{2} [r^{2}+28 (\tfrac{3c}{2} \vee 1)][\xdfCr^2+\cpen]\daRa{\mdDi}{(\xdfCw[])} \\
    + \tfrac{2}{7}\sum\nolimits_{\Di\in\nsetlo{\pDi,\ssY}}\tfrac{21c}{2}[\xdfCr^2+\cpen]\daRa{\Di}{(\xdfCw[])}\exp[\eta n (\tfrac{21c}{2}[\xdfCr^2+\cpen]\daRa{\Di}{(\xdfCw[])}/2 - 14 (\tfrac{3c}{2} \vee 1) [\xdfCr^2+\cpen]\daRa{\pdDi}{(\xdfCw[])})]\\\hfill
    +2r^{2} (\sum\nolimits_{m = 1}^{m_{-}} \exp[-14 \eta n (\tfrac{3c}{2} \vee 1) [\xdfCr^2+\cpen]\daRa{\mdDi}{(\xdfCw[])}] + \exp[- \tfrac{c (c \wedge 1)}{4} m_{-}^{\dagger} \cmiSv[\mdDi]])\\
+2\sum_{\Di\in\nset{\pDi,\ssY}}6 n^{-1}\Lambda_{+}(m) \exp[- \tfrac{c}{4} m \cmiSv]  
+\tfrac{2}{7}\sum_{\Di\in\nsetlo{\pDi,\ssY}}\tfrac{21c}{2}[\xdfCr^2+\cpen]\daRa{\Di}{(\xdfCw[])}\exp[- \tfrac{c (c \wedge 1)}{4} m \cmiSv].
\end{multline*}

\begin{thm}\label{THM_FREQ_IGSSM_KNOWN_IID_MINIMAX_NP}
\reEnd
\end{thm}



\section{Independent observations with unknown noise distribution}
%
\subsection{Circular deconvolution with independent data and known noise density}\label{FREQ_CIRCDECONV_KNOWN_IID}

As stated in \nref{BAYES}, the non-conjugated nature of the hierarchical Gaussian sieve in the context of circular deconvolution does not allow to compute the posterior mean analytically.
However, in this part we mimic the form of this posterior mean and construct an estimator from this idea.

We start by reminding the definition of the projection estimators, which we will use to surrogate the posterior mean of sieve priors, which appear in the structure of the posterior mean of hierarchical sieves.

\begin{rem}{\textsc{Projection estimators} \\}\label{REM_FREQ_CIRCDECONV_KNOWN_IID_PROJEST}
We recall the notation for the projection estimators, for any $m$ in $\mathds{Z}$, we have
\begin{alignat*}{3}
& \overline{\theta}_{m} && := && \frac{1}{n}\sum\limits_{p = 1}^{n} \frac{e_{m}(Y_{p})}{\lambda_{m}};\\
& \left( \overline{\theta}^{m}_{j} \right)_{j \in \mathds{Z}} && := &&\left(\mathds{1}_{\vert j \vert \leq m} \overline{\theta}_{j}\right)_{j \in \mathds{Z}}.
\end{alignat*}
\end{rem}

As in the posterior mean of hierarchical sieves, we define a weight sequence, corresponding to the posterior distribution of the threshold parameter.

\begin{de}{\textsc{Weight sequence} \\}\label{DE_FREQ_CIRCDECONV_KNOWN_IID_WEIGHT}
Let be the following quantities:
\begin{alignat*}{3}
& \kappa && := && \frac{23}{2};\\
& \cmiSv && := && \frac{\log\left(\Di \iSv[\Di] \vee (\Di + 2)\right)^{2}}{\log\left(\Di + 2\right)^{2}};\\
& \DipenSv && := && \Di \iSv[\Di] \cmiSv; \\
& \pen(\Di) && := && \frac{9}{2} \cdot 12 \cdot \cpen \cdot \DipenSv;\\
& \Upsilon(Y, \Di) && := && n \left\Vert \overline{\theta}^{m} \right\Vert_{l^{2}}^{2}.
\end{alignat*}
Then, for any couple of natural integers $n$ and $\eta$, we define the distribution $\P_{M \vert Y^{n}}^{n, (\eta)}$, dominated by the counting measure on $\N^{\star}$ such that, for any $m$ in $\llbracket 1, n \rrbracket$
\[\P_{M \vert Y^{n}}^{n, (\eta)}(m) := \frac{\exp\left[\eta\left(- \pen(m) + \Upsilon(Y^{n}, m)\right)\right]}{\sum\limits_{k = 1}^{n} \exp\left[\eta\left(- \pen(k) + \Upsilon(Y^{n}, k)\right)\right]}.\]
\end{de}

With those definitions at hand, we are able to define an estimator that reproduces the structure of the posterior mean of iterated hierarchical sieves.

\begin{de}{\textsc{Aggregation/shrinkage estimator} \\}\label{DEFREQ_CIRCDECONV_KNOWN_IID_AGGREGEST}
Using the notations we just introduced, we define, for any strictly positive integer $\eta$ the shrinkage/aggregation estimator $\widehat{\theta}^{(\eta)}$ such that, for any $j$ in $\mathds{Z}$
\begin{alignat*}{3}
& \widehat{\theta}^{(\eta)}_{j} && := && \P_{M \vert Y^{n}}^{n, (\eta)}(\llbracket \vert j \vert, n \rrbracket) \overline{\theta}_{j};\\
& \widehat{\theta}^{\eta} && := && \sum\limits_{j = 1}^{n} \P_{M \vert Y^{n}}^{n, (\eta)}(j) \overline{\theta}^{j}.
\end{alignat*}
\end{de}

As previously, one can notice that, as $\eta$ tends to infinity, this estimator converges to the penalised contrast maximiser projection estimator with penalty function $\pen$ and contrast $\Upsilon$.

Using the method described in \nref{FREQ_STRATEGY}, we are able to show that, for any $\theta^{\circ}$, the sequence defined hereafter is a convergence rate.

\begin{de}{\textsc{Convergence rate} \\}\label{DE_FREQ_CIRCDECONV_KNOWN_IID_CONVRATE}
Let be the sequences:
\[m^{\dagger}_{n} := \argmin_{m \in \N}\left\{\left[\mathfrak{b}_{m}^{2}(\theta^{\circ})\mathfrak{b}_{0}^{-2}(\theta^{\circ}) \vee 2 \frac{m \Lambda_{(m)}}{n} \psi_{n}\right]\right\};\]
and
\[\Phi^{\dagger}_{n} := \left[\mathfrak{b}_{m^{\dagger}_{n}}^{2}(\theta^{\circ})\mathfrak{b}_{0}^{-2}(\theta^{\circ}) \vee 2 \frac{m^{\dagger}_{n} \Lambda_{(m^{\dagger}_{n})}}{n} \psi_{n}\right].\]
\end{de}

\bigskip

% ....................................................................
% <<Ass upper bound p>>
% ....................................................................
\begin{as}\label{ass:ub:p}
Let $\fxdf$ have a finite series expansion as definied in \ref{oo:xdf:p}, that is, either
\begin{inparaenum}[i]\renewcommand{\theenumi}{\dgrau\rm(\alph{enumi})}
\item\label{ass:ub:p:c1}
	$\fxdf=\left(\mathds{1}_{j = 0}\right)_{j \in \mathds{Z}}$, i.e., $\bias[0](\fxdf)=\Vnormlp{\Proj[{\mHiH[0]}]^\perp\fxdf}^2=0$ or
\item\label{ass:ub:p:c2}
	there is $K\in\N$ with $1\geq \bias[{K-1}](\fxdf)>0$ and $\bias[K](\fxdf)=0$.
\end{inparaenum}
In case  \ref{ass:ub:p:c1} set $\dr\ssY_{\fxdf,\iSv}:=\ceil{15(\tfrac{300}{\sqrt{\cpen}})^4}$ while in case \ref{ass:ub:p:c2} given $K_{\fydf}:=K\dr\vee 3(\tfrac{800\Vnormlp[1]{\fydf}}{\cpen})^2$ and $c_{\fxdf}:=\tfrac{2\Vnormlp{\Proj[{\mHiH[0]}]^\perp\fxdf}^2+484\cpen}{\Vnormlp{\Proj[{\mHiH[0]}]^\perp\fxdf}^2\bias[{K-1}]^2(\fxdf)}$ let there $\ssY_{\fxdf,\iSv}\in\N$ be with $\ssY_{\fxdf,\iSv}>\ceil{c_{\fxdf}\DipenSv[K_{\fydf}]\dr\vee15(\tfrac{300}{\sqrt{\cpen}})^4}$ such that $\sDi{\ssY}:=\max\{\Di\in\nset{K,\ssY}:c_{\fxdf}\,\DipenSv<\ssY\}$ where the defining set contains $K_{\fydf}$ and thus it is not empty, satisfies $\cmiSv[\sDi{\ssY}]\sDi{\ssY}\geq K_{\fydf}(\log\ssY)$ for all $\ssY\geq \ssY_{\fxdf,\iSv}$.
\end{as}
% ....................................................................
% <<Rem Ass upper bound p>>
% ....................................................................
\begin{rmk}\label{rem:ass:ub:p}
 Keep in mind that $\Nsuite[\Di]{\bias(\fxdf)}\subset[0,1]$ is monotonically non increasing with $\bias[1](\fxdf)\leq1$ and $\lim_{\Di\to\infty}\bias(\fxdf)=0$.
Thereby, in case \ref{ass:ub:p:c2} of \nref{ass:ub:p} holds $\Vnormlp{\Proj[{\mHiH[0]}]^\perp\fxdf}^2>0$ and $\tfrac{2\Vnormlp{\Proj[{\mHiH[0]}]^\perp\fxdf}^2+484\cpen}{\Vnormlp{\Proj[{\mHiH[0]}]^\perp\fxdf}^2\bias[{K-1}]^2(\fxdf)}\geq1$.
We shall stress that in \nref{re:ub:co1} and \nref{re:ub:co2} in the \nref{PRO_FREQ_CIRCDECONV_KNOWN_IID_ORACLE_P} we derive upper bounds for the partially data-driven aggregated OSE featuring a deterioration of the upper bound which, due to \nref{ass:ub:p} is avoided in the next assertion.
\end{rmk}
% ....................................................................
% <<Il Ass upper bound p>> \ref{il:ass:ub:p}
% ....................................................................
\begin{il}\label{il:ass:ub:p}
Let us illustrate \nref{ass:ub:p} considering as in \nref{il:oo} the commonly studied behaviours \ref{il:edf:o} and \ref{il:edf:s} for the sequence  $\Nsuite[j]{\iSv[j]}$.
\begin{Liste}[]
\item[\mylabel{il:ass:ub:p:o}{\dg\bfseries{(o)}}]
Let $\iSv[\Di]\sim \Di^{2a}$, $a>0$, then  we have $\cmSv\sim1$, $\miSv\sim \oiSv\sim\Di^{2a}$, $\DipenSv=\cmSv \Di \miSv\sim \Di^{2a+1}$ and hence $1\sim\DipenSv[\sDi{\ssY}]\ssY^{-1}\sim(\sDi{\ssY})^{2a+1}\ssY^{-1}$ implies $\sDi{\ssY}\sim\ssY^{1/(2a+1)}$ and $\sDi{\ssY}\cmSv[\sDi{\ssY}]\sim\ssY^{1/(2a+1)}$.
\item[\mylabel{il:ass:ub:p:s}{\dg\bfseries{(s)}}]
Let $\iSv[\Di]\sim \exp(\Di^{2a})$, $a>0$, then  we have $\cmSv\sim(\Di^{2a})^2$, $\DipenSv=\Di \cmSv \miSv\sim \Di^{1+4a}\exp(\Di^{2a})$ and hence $\ssY\sim\DipenSv[\sDi{\ssY}]\sim (\sDi{\ssY})^{1+4a}\exp((\sDi{\ssY})^{2a})$ implies $\sDi{\ssY}\sim(\log\ssY-\tfrac{1+4a}{2a}\log\log\ssY)^{1/(2a)}$ and $\sDi{\ssY}\cmSv[\sDi{\ssY}]\sim (\log \ssY)^{2+1/(2a)}$.
\end{Liste}
Clearly, in both cases \ref{il:ass:ub:p:o} and \ref{il:ass:ub:p:s}, there is ${\ssY}_{\fxdf,\iSv}\in\Nz$ such that $\cmiSv[\sDi{\ssY}]\sDi{\ssY}\geq K_{\fydf}(\log\ssY)$  for all $\ssY\geq{\ssY}_{\fxdf,\iSv}$ holds true.
\end{il}

More precisely, we obtain the following theorem, for which the proof is given in \nref{PRO_FREQ_CIRCDECONV_KNOWN_IID_ORACLE_P}.
% ....................................................................
% <<Re upper bound p>>
% ....................................................................
\begin{thm}\label{THM_FREQ_CIRCDECONV_KNOWN_IID_ORACLE_P}
Let $\fxdf$ have a finite series expansion as defined in \ref{oo:xdf:p}.
Under \nref{ass:ub:p} there is a finite numerical constant $\cst{}$ such that for all $\dr\ssY\in\N$,
\begin{equation}\label{re:ub:p:e1}
\FuEx[\ssY]{\fxdf}\left[\Vnormlp{\txdf-\xdf}^2\right]
\leq\cst{}\{\DipenSv[{\ssY_{\fxdf,\iSv}}]+\VnormLp{\Proj[{\mHiH[0]^\perp}]\xdf}^2\ssY_{\xdf,\iSv}+ \Vnormlp[1]{\fydf}^2\}\ssY^{-1}.
\end{equation}
\end{thm}
% ....................................................................
% <<Il upper bound co2>>
% ....................................................................
\begin{il}\label{il:ub:p}
Let us illustrate \nref{THM_FREQ_CIRCDECONV_KNOWN_IID_ORACLE_P}
  considering as in \nref{il:ass:ub:p} the behaviours
  \ref{il:edf:o} and \ref{il:edf:s} for the sequence
  $\Nsuite[j]{\iSv[j]}$.  Keeping in mind that as shown in \nref{il:ass:ub:p} there is ${\ssY}_{\xdf,\iSv}\in\Nz$ such that
    $\cmiSv[\sDi{\ssY}]\sDi{\ssY}\geq K_{\ydf}(\log\ssY)$  for all $\ssY\geq{\ssY}_{\xdf,\iSv}$ 
 holds true, due to \nref{re:ub:co2} there is a constant $\cst{\xdf,\edf}$
 depending only on the densities $\xdf$ and $\edf$ such that 
 $\FuEx[\ssY]{\rY}\VnormLp{\txdf-\xdf}^2\leq
 \cst{\xdf,\edf}\ssY^{-1}$ for all $\ssY\in\Nz$. Comparing the last result
 with the oracle rate derived in \ref{il:oo:po} and  \ref{il:oo:so}
 in \nref{il:oo} we conclude, that $\txdf$ is optimal  in an oracle sense in both cases \ref{il:oo:po} and  \ref{il:oo:so}.
\end{il}

\medskip

% ....................................................................
% <<Ass upper bound np>> \ref{ass:ub:np}
% ....................................................................
\begin{as}\label{ass:ub:np} Let $\xdf$  have an infinite series expansion
  as definied in \ref{oo:xdf:np}, that is, $1\geq \bias(\xdf)>0$ for all $\Di\in\Nz$.
Given   $\Di_{\ydf}:=\dr3(\tfrac{800\Vnormlp[1]{\fydf}}{\cpen})^2$ and
$\tDi_{\ydf}=\min\{\Di\in\Nz:\bias[\Di_{\ydf}](\xdf)>\bias[\Di](\xdf)\}$
 there is $\ssY_{\xdf,\iSv}\in\Nz$ with
$\ssY_{\xdf,\iSv}\geq\ceil{\tfrac{\DipenSv[\tDi_{\ydf}]}{\bias[\tDi_{\ydf}]^2(\xdf)}\vee\dr15(\tfrac{300}{\sqrt{\cpen}})^4}$
  such that either \begin{inparaenum}[i]\renewcommand{\theenumi}{\dgrau\rm(\alph{enumi})}\item\label{ass:ub:np:c1}
$\cmiSv[\aDi{\ssY}]\aDi{\ssY}\geq \Di_{\ydf}|\log\hRa{\xdf,\iSv}|$ 
for all
$\ssY\geq{\ssY}_{\xdf,\iSv}$ or \item\label{ass:ub:np:c2}  
$\aDi{\ssY}\leq  \Di_{\ydf}|\log\hRa{\xdf,\iSv}|$ for all
$\ssY\geq{\ssY}_{\xdf,\iSv}$.
\end{inparaenum}
We set in case \ref{ass:ub:np:c1}  $\sDi{\ssY}:=\aDi{\ssY}$ and  in case \ref{ass:ub:np:c2}  $\sDi{\ssY}:= \Di_{\ydf}|\log\hRa{\xdf,\iSv}|$.
\end{as}
% ....................................................................
% <<Rem Ass upper bound np>>
% ....................................................................
\begin{rmk}\label{rem:ass:ub:np}
Considering $\Di_{\ydf}:=\dr3(\tfrac{800\Vnormlp[1]{\fydf}}{\cpen})^2$ and
$\tDi_{\ydf}=\min\{\Di\in\Nz:\bias[\Di_{\ydf}](\xdf)>\bias[\Di](\xdf)\}$
as defined in \nref{ass:ub:np} the defining set is not empty since $\bias[\Di](\xdf)>0$ for all
$\Di\in\Nz$ and $\lim_{\Di\to\infty}\bias[\Di](\xdf)=0$. Moreover, it
holds $\tDi_{\ydf}>\Di_{\ydf}$ due to the the monotonicity of
$\bias(\xdf)$. Noting that 
$\ceil{\tfrac{\DipenSv[\tDi_{\ydf}]}{\bias[\tDi_{\ydf}]^2(\xdf)}\vee\dr15(\tfrac{300}{\sqrt{\cpen}})^4}\geq \DipenSv[\tDi_{\ydf}]\geq\tDi_{\ydf}$ by construction,
for all $\ssY\geq\ssY_{\xdf,\iSv}$ as in \nref{ass:ub:np} holds 
$\oRaDi{\Di_{\ydf},\xdf,\iSv}\geq\bias[\Di_{\ydf}]^2(\xdf)>\bias[\tDi_{\ydf}]^2(\xdf)% =\bias[\tDi_{\Sv}]^2(\So)[1\vee
% \tfrac{\tDi_{\Sv}\cmiSv[\tDi_{\Sv}]\miSv[\tDi_{\Sv}]/\nlIm}{\bias[\tDi_{\Sv}]^2(\So)}]
=\oRaDi{\tDi_{\ydf},\xdf,\iSv}$
and hence, for all
$\ssY\geq\ssY_{\xdf,\iSv}$ we have $\aDi{\ssY}>
\Di_{\ydf}$.
We use these preliminary findings in the proof of \nref{THM_FREQ_CIRCDECONV_KNOWN_IID_ORACLE_NP}.  
\end{rmk}
% ....................................................................
% <<Il Ass upper bound p>> \ref{il:ass:ub:np}
% ....................................................................
\begin{il}\label{il:ass:ub:np}
Let us illustrate \nref{ass:ub:np}
  considering as in \nref{il:oo} usual
  behaviour \ref{il:oo:oo}, \ref{il:oo:so} and \ref{il:oo:os}
 for the sequences $\Nsuite[\Di]{\bias[\Di](\xdf)}$ and
  $\Nsuite[\Di]{\iSv[\Di]}$:
 \begin{Liste}[]
\item[\mylabel{il:ass:ub:np:oo}{\dg\bfseries{[o-o]}}] Since
  $\bias^2(\xdf)\sim\Di^{-2p}$ and  $\DipenSv\sim\Di^{2a+1}$
    (cf.  \nref{il:ass:ub:p} \ref{il:ass:ub:p:o}) follows
    $\hRa{\xdf,\iSv}\sim(\aDi{\ssY})^{-2p}\sim\DipenSv[\aDi{\ssY}]\ssY^{-1}\sim(\aDi{\ssY})^{2a+1}\ssY^{-1}$ which
    implies $\aDi{\ssY}\sim\ssY^{1/(2p+2a+1)}$,
    $\cmiSv[\aDi{\ssY}]\aDi{\ssY}\sim\ssY^{1/(2p+2a+1)}$,
    $\hRa{\xdf,\iSv}\sim\ssY^{-2p/(2p+2a+1)}$ and $|\log\hRa{\xdf,\iSv}|\sim(\log\ssY)$.
 \item[\mylabel{il:ass:ub:np:os}{\dg\bfseries{[o-s]}}]
    Since
  $\bias^2(\xdf)\sim\Di^{-2p}$ and $\DipenSv\sim\Di^{1+4a}\exp(\Di^{2a})$ (cf. \nref{il:ass:ub:p} \ref{il:ass:ub:p:s}) follows
    $\hRa{\xdf,\iSv}\sim(\aDi{\ssY})^{-2p}\sim\DipenSv[\aDi{\ssY}]\ssY^{-1}\sim(\aDi{\ssY})^{1+4a}\exp((\aDi{\ssY})^{2a})$
    which implies  $\aDi{\ssY}\sim(\log\ssY)^{1/(2a)}$, $\cmiSv[\aDi{\ssY}]\aDi{\ssY}\sim(\log\ssY)^{2+1/(2a)}$, 
    $\hRa{\xdf,\iSv}\sim(\log\ssY)^{-p/a}$ and $|\log\hRa{\xdf,\iSv}|\sim(\log\log\ssY)$.
 \item[\mylabel{il:ass:ub:np:so}{\dg\bfseries{[s-o]}}]  Since
  $\bias^2(\xdf)\sim\exp(-\Di^{2p})$ and    $\DipenSv\sim\Di^{2a+1}$
    (cf.  \nref{il:ass:ub:p} \ref{il:ass:ub:p:o}) follows
    $\hRa{\xdf,\iSv}\sim\exp(-(\aDi{\ssY})^{2p})\sim\DipenSv[\aDi{\ssY}]\ssY^{-1}\sim
(\aDi{\ssY})^{2a+1}\ssY^{-1}$
    which implies  $\aDi{\ssY}\sim(\log\ssY)^{1/(2p)}$, $\cmiSv[\aDi{\ssY}]\aDi{\ssY}\sim(\log\ssY)^{1/(2p)}$,
    $\hRa{\xdf,\iSv}\sim(\log\ssY)^{(2a+1)/(2p)}\ssY^{-1}$
    and     $|\log\hRa{\xdf,\iSv}|\sim(\log\ssY)$.
  \end{Liste}
Clearly,  there is ${\ssY}_{\xdf,\iSv}\in\Nz$ such that for all
$\ssY\geq{\ssY}_{\xdf,\iSv}$ in the cases \ref{il:ass:ub:np:oo} and
\ref{il:ass:ub:np:os}   $\cmiSv[\aDi{\ssY}]\aDi{\ssY}\geq
\Di_{\ydf}|\log\hRa{\xdf,\iSv}|$, i.e., \nref{ass:ub:np}
\ref{ass:ub:np:c1} holds, while in case \ref{il:ass:ub:np:so}
$\aDi{\ssY}\leq \Di_{\ydf}|\log\hRa{\xdf,\iSv}|$ for $p\geq1/2$, i.e., \nref{ass:ub:np}
\ref{ass:ub:np:c2} holds, and $\cmiSv[\aDi{\ssY}]\aDi{\ssY}\geq
\Di_{\ydf}|\log\hRa{\xdf,\iSv}|$ for $p<1/2$, i.e., \nref{ass:ub:np}
\ref{ass:ub:np:c1} holds.
\end{il}

\begin{thm}\label{THM_FREQ_CIRCDECONV_KNOWN_IID_ORACLE_NP}
Let be the constants $K := \frac{\sqrt{2} - 1}{21 \sqrt{2}}$, and $C_{\lambda, \theta^{\circ}} \geq \sum\limits_{j = 1}^{\infty} \exp\left[- \eta \frac{\psi_{n} m \overline{\Lambda}_{m}}{2}\right]$ then, for any $n$ and $\eta$ integers greater than $1$, we have
\begin{alignat*}{3}
& \E_{\theta^{\circ}}^{n}\left[\Vert \widehat{\theta}^{(\eta)} - \theta^{\circ} \Vert_{l^{2}}^{2}\right] && \leq && 174 \mathfrak{b}_{0}^{2} \Phi^{\dagger}_{n} + \\
& && && \frac{1}{n} 32 C_{\lambda, \theta} \exp\left[- K\left(\frac{\psi_{n} 2 m^{\circ}_{n}}{\left\Vert \theta^{\circ} \right\Vert_{l^{2}}^{2} \left\Vert \lambda \right\Vert_{l^{2}}^{2}}\wedge \sqrt{n \psi_{n}}\right) + \log(\psi_{n}) + 2 \log\left(\Lambda_{n} \vee n\right)\right];
\end{alignat*}
assuming $m^{\dagger}_{n}$ tends to $\infty$, this gives
\[\E_{\theta^{\circ}}^{n}\left[\Vert \widehat{\theta}^{(\eta)} - \theta^{\circ} \Vert_{l^{2}}^{2}\right] \in \mathcal{O}_{n}\left(\Phi^{\dagger}_{n}\right).\]
\end{thm}

Comparison with the oracle rate of projection estimators reveals that in many cases, we obtain an oracle optimal estimator.

\begin{il}\label{IL_FREQ_CIRCDECONV_KNOWN_IID_ORACLE_NP}
Assume $m^{\dagger}_{n}$ tends to infinity and let be two positive real numbers $p$ and $a$.

If $\mathfrak{b}_{m}^{2} \asymp_{m \rightarrow \infty} m^{-2p}$ and $\Lambda_{m} \asymp_{m \rightarrow \infty} m^{2a}$, then we have $\psi_{n} \asymp_{n \rightarrow \infty} 4a$; $m^{\dagger}_{n} \asymp_{n \rightarrow \infty} n^{\frac{1}{2a + 2p + 1}}$; and $\Phi^{\dagger}_{n} \asymp_{n \rightarrow \infty} n^{-\frac{2p}{2a + 2p + 1}}$ and we have
\begin{alignat*}{3}
& \E_{\theta^{\circ}}^{n}\left[\left\Vert \widehat{\theta}^{(\eta)} - \theta^{\circ} \right\Vert_{l^{2}}^{2}\right] && \in && \mathcal{O}_{n}\left(n^{-\frac{2p}{2a + 2p + 1}} +\right.\\
& && && \left. \frac{1}{n} C_{\lambda, \theta^{\circ}} \exp\left[- K \left(\frac{8 a n^{\frac{1}{2a + 2p + 1}}}{\Vert \lambda \Vert_{l^{2}} \Vert \theta^{\circ} \Vert_{l_{2}}} \wedge \sqrt{4 a n}\right) + 4 a \log(n) + 2 n^{2a} \right]\right)\\
& && \in && \mathcal{O}_{n}(n^{-\frac{2p}{2a + 2p + 1}})
\end{alignat*}

\medskip

On the other hand, if $\Lambda_{m} \asymp_{m \rightarrow \infty} \exp\left[m^{2a}\right]$, then we have $\psi_{n} \asymp_{n \rightarrow \infty} \frac{n^{4a}}{\log(n)^{2}}$; $m^{\dagger}_{n} \asymp_{n \rightarrow \infty} \log(n)^{\frac{1}{2a}}$; and $\Phi^{\dagger}_{n} \asymp_{n \rightarrow \infty} \log(n)^{-\frac{p}{a}}$.
Which leads to
\begin{alignat*}{3}
& \E_{\theta^{\circ}}^{n}\left[\left\Vert \widehat{\theta}^{(\eta)} - \theta^{\circ} \right\Vert_{l^{2}}^{2}\right] && \in && \mathcal{O}_{n}\left(\log(n)^{-\frac{p}{a}} + \right. \\
& && && \left. \frac{1}{n} C_{\lambda, \theta^{\circ}} \exp\left[- K \left(\frac{n^{4a} 2 \log(n)^{\frac{1 - 4a}{2a}}}{\Vert \lambda \Vert_{l^{2}} \Vert \theta^{\circ} \Vert_{l_{2}}} \wedge \frac{n^{2a + \frac{1}{2}}}{\log(n)}\right) + 4 a \log(n) + 2 n^{2a} \right]\right)\\
& && \in && \mathcal{O}_{n}(\log(n)^{-\frac{p}{a}}).
\end{alignat*}
\end{il}
\section{Circular deconvolution with beta mixing data and known noise density}\label{FREQ_CIRCDECONV_KNOWN_BETA}

Considering the positive results obtained in the previous section, we are now interested in generalising those results to the situation where our observations are not a sequence of independent identically distributed variables anymore but may suffer from dependence.

\begin{as}{\textsc{Strictly stationary, absolutely regular process} \\}\label{AS_FREQ_CIRCDECONV_KNOWN_BETA_STRICTLYSTA}
We assume in this section that the process of observations $(Y_{p})_{p \in \mathds{Z}}$ is strictly stationary and absolutely regular as described in \nref{DEPENDENTDATA}.
\end{as}

\begin{as}{\textsc{Rich space} \\}\label{AS_FREQ_CIRCDECONV_KNOWN_BETA_RICHSPACE}
We assume in this section that the process of observations $(Y_{p})_{p \in \mathds{Z}}$ is strictly stationary and absolutely regular as described in \nref{DEPENDENTDATA}.
\end{as}

\begin{as}\label{AS_FREQ_CIRCDECONV_KNOWN_BETA_JOINT}
Assume that, for any integer $p$, the joint distribution $\P_{Y_{0}, Y_{p} \vert \theta^{\circ}}$ of $Y_{0}$ and $Y_{p}$ admits a density $f_{Y_{0}, Y_{p}}$ which is square integrable.

Let $\Vert f_{Y_{0}, Y_{p}} \Vert_{L^{2}}^{2} := \int_{0}^{1} \int_{0}^{1} \vert f_{Y_{0}, Y_{p}}(x, y)\vert^{2}dx \, dy < \infty$ with a slight abuse of notations.
If we denote further by $h \otimes g : [0, 1]^{2} \rightarrow \R$ the bivariate function $[h \otimes g](x, y) := h(x) g(y)$ then let assume $\gamma_{f} := \sup\limits_{p \geq 1} \Vert f_{Y_{0}^{n}, Y_{p}^{n}\vert \theta^{\circ}} - f_{Y_{0}^{n}\vert \theta^{\circ}} \otimes f_{Y_{0}^{n}\vert \theta^{\circ}} \Vert_{L^{2}} < \infty$.

Assume in addition $\sum\limits_{p = 1}^{\infty} \beta(Y_{0}, Y_{p}) < \infty$ and $\gamma := \sup\limits_{\theta^{\circ} \in \Theta(\mathfrak{a}, r)} \gamma_{\theta} < \infty$
\end{as}

As in the posterior mean of hierarchical sieves, we define a weight sequence, corresponding to the posterior distribution of the threshold parameter.

\begin{de}{\textsc{Weight sequence} \\}\label{DE_FREQ_CIRCDECONV_KNOWN_BETA_WEIGHT}
\end{de}

With those definitions at hand, we are able to define an estimator that reproduces the structure of the posterior mean of iterated hierarchical sieves.

\begin{de}{\textsc{Aggregation/shrinkage estimator} \\}\label{DE_FREQ_CIRCDECONV_KNOWN_BETA_AGGREGEST}
Using the notations we just introduced, we define, for any strictly positive integer $\eta$ the shrinkage/aggregation estimator $\widehat{\theta}^{(\eta)}$ such that, for any $j$ in $\mathds{Z}$
\begin{alignat*}{3}
& \widehat{\theta}^{(\eta)}_{j} && := && \P_{M \vert Y^{n}}^{n, (\eta)}(\llbracket \vert j \vert, n \rrbracket) \overline{\theta}_{j};\\
& \widehat{\theta}^{\eta} && := && \sum\limits_{j = 1}^{n} \P_{M \vert Y^{n}}^{n, (\eta)}(j) \overline{\theta}^{j}.
\end{alignat*}
\end{de}

As previously, one can notice that, as $\eta$ tends to infinity, this estimator converges to the penalised contrast maximiser projection estimator with penalty function $\pen$ and contrast $\Upsilon$.

\medskip

Using the method described in \nref{FREQ_STRATEGY}, we are able to show that, for any $\theta^{\circ}$, the sequence defined hereafter is a convergence rate.

\begin{de}{\textsc{Convergence rate} \\}\label{DE_FREQ_CIRCDECONV_KNOWN_BETA_CONVRATE}
\end{de}

More precisely, we obtain the following theorem, for which the proof is given in \nref{PRO_FREQ_CIRCDECONV_KNOWN_BETA_ORACLE_NP}.

\begin{thm}\label{THM_FREQ_CIRCDECONV_KNOWN_BETA_ORACLE_NP}
\end{thm}

Comparison with the oracle rate of projection estimators reveals that in many cases, we obtain an oracle optimal estimator.

\begin{il}\label{IL_FREQ_CIRCDECONV_KNOWN_BETA_RATE}
\end{il}
\subsection{Circular deconvolution with independent data and partially known noise density}\label{FREQ_CIRCDECONV_UNKNOWN_IID}

Finally, in this section, we consider the case of a partially known operator as described in \nref{INTRO_INVERSE_UNKNOWN}.

As a consequence, we need here to simultaneously estimate the distribution $\P_{\epsilon}$ of the noise random variable $\epsilon$ and the density of interest $f^{X}$.
We now assume that we have at hand two independent samples.
The first is an i.i.d. sample from $\P^{\epsilon}$, denoted $\epsilon^{q} = \left(\epsilon_{r}\right)_{r \in \llbracket 1, q \rrbracket}$; the second is, as in the known operator case, a sample from the convolved distribution, assumed here to be i.i.d., denoted $Y^{n} = \left(Y_{p}\right)_{p \in \llbracket 1, n \rrbracket}$.

\medskip

We want to adapt our aggregation estimator shape to this case. In this perspective let us define an estimators for $\lambda^{-1}$.

\begin{de}{\textsc{Thresholded estimator} \\}\label{DE_FREQ_CIRCDECONV_UNKNOWN_IID_THRESHOLDEDEST}
For any $m$ in $\mathds{Z}$ define, with the convention "$0/0 = 0$"
\begin{alignat*}{3}
& \widehat{\lambda}_{m} && := && \frac{1}{q} \sum\limits_{r = 1}^{q} e_{m}(\epsilon_{r});\\
& \widehat{\lambda}^{+}_{m} && := && \frac{1}{\widehat{\lambda}_{m}} \mathds{1}_{\vert \widehat{\lambda}_{m}\vert^{2} > \frac{1}{q}}.
\end{alignat*}
Mimicking notations we used until here we also note, for any $m$ in $\N$
\begin{alignat*}{3}
& \widehat{\Lambda}_{m} && := && \vert \widehat{\lambda}^{+}_{m} \vert^{2}\\
& \widehat{\Lambda}_{(m)} && := && \max\left\{\widehat{\Lambda}_{k}, k \in \llbracket 1, m \rrbracket \right\};
\end{alignat*}
\end{de}

With this definition, we define an alternative form for projection estimators which we aggregated in the two previous sections.

\begin{de}{\textsc{Thresholded projection estimators} \\}\label{DE_FREQ_CIRCDECONV_UNKNOWN_IID_THRESHOLDEDPROJEST}
For any $m$ in $\mathds{Z}$, let be
\begin{alignat*}{3}
& \overline{\theta}_{m}^{+} && := && \mathds{1}_{m = 0} + \mathds{1}_{m \neq 0} \frac{1}{n}\sum\limits_{p = 1}^{n} e_{m}(Y_{p}) \widehat{\lambda}_{m}^{+};\\
& \left( \overline{\theta}^{m, +}_{j} \right)_{j \in \mathds{Z}} && := &&\left(\mathds{1}_{\vert j \vert \leq m} \overline{\theta}^{+}_{j}\right)_{j \in \mathds{Z}}.
\end{alignat*}
\end{de}

We give the following shape to the weight sequence.

\begin{de}{\textsc{Weight sequence} \\}\label{DE_FREQ_CIRCDECONV_UNKNOWN_IID_WEIGHT}
Let be the following quantities:
\begin{alignat*}{3}
& \kappa && \geq && 1;\\
& \sqrt{\delta^{\widehat{\Lambda}}_{m}} && := && \frac{\log\left(k \widehat{\Lambda}_{(m)} \vee \left(m + 2\right)\right)}{\log\left(m + 2\right)}\\
& \Delta^{\widehat{\Lambda}}_{m} && := && \delta^{\widehat{\Lambda}}_{m} m \widehat{\Lambda}_{(m)}\\
& \pen(m) && := && \frac{9}{2} 12 \kappa \Delta^{\widehat{\Lambda}}_{m};\\
& \Upsilon(Y, \epsilon, m) && := && n \left\Vert \overline{\theta}^{m, +} \right\Vert_{l^{2}}^{2}.
\end{alignat*}
Then, for any couple of natural integers $n$ and $\eta$, we define the distribution $\P_{M \vert Y^{n}, \epsilon^{q}}^{n, (\eta)}$, dominated by the counting measure on $\N^{\star}$ such that, for any $m$ in $\llbracket 1, n \rrbracket$
\[\P_{M \vert Y^{n}, \epsilon^{q}}^{n, (\eta)}(m) := \frac{\exp\left[\eta\left(- \pen(m) + \Upsilon(Y^{n}, \epsilon^{q}, m)\right)\right]}{\sum\limits_{k = 1}^{n} \exp\left[\eta\left(- \pen(k) + \Upsilon(Y^{n}, \epsilon^{q}, k)\right)\right]}.\]
\end{de}

With those definitions at hand, we are able to define an estimator that reproduces the structure of the posterior mean of iterated hierarchical sieves.

\begin{de}{\textsc{Aggregation/shrinkage estimator} \\}\label{DE_FREQ_CIRCDECONV_UNKNOWN_IID_AGGREGEST}
Using the notations we just introduced, we define, for any strictly positive integer $\eta$ the shrinkage/aggregation estimator $\widehat{\theta}^{(\eta)}$ such that, for any $j$ in $\mathds{Z}$
\begin{alignat*}{3}
& \widehat{\theta}^{(\eta)}_{j} && := && \P_{M \vert Y^{n}, \epsilon^{q}}^{n, (\eta)}(\llbracket \vert j \vert, n \rrbracket) \overline{\theta}^{+}_{j};\\
& \widehat{\theta}^{\eta} && := && \sum\limits_{j = 1}^{n} \P_{M \vert Y^{n}, \epsilon^{q}}^{n, (\eta)}(j) \overline{\theta}^{j, +}.
\end{alignat*}
\end{de}

As previously, one can notice that, as $\eta$ tends to infinity, this estimator converges to the penalised contrast maximiser projection estimator with penalty function $\pen$ and contrast $\Upsilon$.

In addition, this time, it is important to note that this estimator does not depend on characteristics of $\lambda$ nor $\theta^{\circ}$ and is hence fully data driven and well designed for the context of a partially unknown operator problem.

\medskip

Using the method described in \nref{FREQ_STRATEGY}, we are able to show that, for any $\theta^{\circ}$, the sequence defined hereafter is a convergence rate.

\begin{de}{\textsc{Convergence rate} \\}\label{DE_FREQ_CIRCDECONV_UNKNOWN_IID_CONVRATE}
Let be the sequences:
\[m^{\dagger}_{n} := \argmin_{m \in \N}\left\{\left[\mathfrak{b}_{m}^{2}(\theta^{\circ})\mathfrak{b}_{0}^{-2}(\theta^{\circ}) \vee 2 \frac{m \Lambda_{(m)}}{n} \psi_{n}\right]\right\};\]
and
\[\Phi^{\dagger}_{n} := \left[\mathfrak{b}_{m^{\dagger}_{n}}^{2}(\theta^{\circ})\mathfrak{b}_{0}^{-2}(\theta^{\circ}) \vee 2 \frac{m^{\dagger}_{n} \Lambda_{(m^{\dagger}_{n})}}{n} \psi_{n}\right].\]
\end{de}

\begin{te}
For each $\Di\in\Nz$ keep in mind that
$\VnormLp{\ProjC[{\mHiH}]\xdf}^2=\VnormLp{\ProjC[{\mHiH[0]}]\xdf}^2\bias[k]^2(\xdf)$,  
$\hRaDi{\Di,\xdf,\iSv}:=[\bias^2(\xdf)\vee \DipenSv\,\ssY^{-1}]$
and
introduce in addition
$\dxdfPr=\sum_{j\in\nset{-\Di,\Di}}\hfedfmpI[j]\fydf[j]\bas_j$. Note
that  $\dxdfPr=\Proj[{\mHiH[\Di]}]\dxdfPr[\ssY]$
and $\VnormLp{\ProjC[{\mHiH[\Di]}]\dxdfPr[\ssY]}^2=2\sum_{j\in\nsetlo{\Di,\ssY}}\eiSv[j]|\fydf[j]|^2$. For any $\pdDi,\mdDi\in\nset{1,\ssY}$ let us define 
\begin{multline}\label{de:au:*Di}
\mDi:=\min\set{k\in\nset{1,\mdDi}: \VnormLp{\ProjC[{\mHiH[0]}]\xdf}^2\bias[k]^2(\xdf)\leq
  [2\VnormLp{\ProjC[{\mHiH[0]}]\xdf}^2+7576\cpen]\hRaDi{\mdDi,\xdf,\iSv}}\quad\text{and}\\\pDi:=\max\set{k\in\nset{\pdDi,\ssY}:
   \epenSv[k] \leq [3\VnormLp{\ProjC[{\mHiH[\pdDi]}]\dxdfPr[\ssY]}^2+9*\epenSv[\pdDi]]}
\end{multline}
where  the defining set obviously contains $\mdDi$ and $\pdDi$, respectively, 
and hence, they are
not empty. Keep in mind that $\pDi:=\pDi(\rE_1,\dotsc,\rE_{\ssE})$ is
random but does not depend on the sample $\rY_1,\dotsc,\rY_{\ssY}$.
\end{te}

\begin{as}\label{ass:au:ub:p}
Let $\xdf$ have a finite series expansion as defined in \ref{oo:xdf:p}, that is, either
\begin{inparaenum}[i]
\renewcommand{\theenumi}{\dgrau\rm(\alph{enumi})}
\item\label{ass:au:ub:p:c1} $\xdf=\bas_0$, i.e., $\bias[0](\xdf)=\VnormLp{\Proj[{\mHiH[0]}]^\perp\xdf}^2=0$ or
\item\label{ass:au:ub:p:c2} there is $K\in\Nz$ with $1\geq \bias[{[K-1] }](\xdf)>0$ and $\bias[K](\xdf)=0$.
\end{inparaenum}
In case \ref{ass:au:ub:p:c1} set $\dr K_{\ydf}:=\ceil{15\tfrac{300^4}{\cpen^2}\vee3\tfrac{800^2}{\cpen^2}}$ while in case \ref{ass:au:ub:p:c2} given $K_{\ydf}:=K\dr\vee
3\tfrac{800^2\Vnormlp[1]{\fydf}^2}{\cpen^2}$ and $c_{\xdf}:=\tfrac{2\VnormLp{\Proj[{\mHiH[0]}]^\perp\xdf}^2+7576\cpen}{\VnormLp{\Proj[{\mHiH[0]}]^\perp\xdf}^2\bias[{[K-1]}]^2(\xdf)}$ let there $\ssY_{\xdf,\iSv},\ssE_{\xdf,\iSv}\in\Nz$ be with $\ssY_{\xdf,\iSv}>\ceil{c_{\xdf}\DipenSv[K_{\ydf}]\dr\vee15\tfrac{300^4}{\cpen^2}}$ and $\ssE_{\xdf,\iSv}>\ceil{289\log(K_{\ydf}+2)\cmiSv[K_{\ydf}]\miSv[K_{\ydf}]}$ such that $\sDi{\ssY}:=\max\{\Di\in\nset{K,\ssY}:c_{\xdf}\,\DipenSv<\ssY\}$ and $\sDi{\ssE}:=\max\{\Di\in\nset{K_{\ydf},\ssE}:289\log(\Di+2)\cmiSv\miSv\leq\ssE\}$ where the defining sets contain $K_{\ydf}$ and thus they are not empty, satisfies $\cmiSv[\sDi{\ssY}]\sDi{\ssY}\geq K_{\ydf}(\log\ssY)$ for all $\ssY\geq \ssY_{\xdf,\iSv}$ and $\cmiSv[\sDi{\ssE}]\sDi{\ssE}\geq K_{\ydf}(\log\ssE)$ for all $\ssE\geq \ssE_{\xdf,\iSv}$, respectively.
\end{as}

\begin{il}\label{il:ass:au:ub:p}
Let us illustrate \nref{ass:au:ub:p} considering as in \nref{il:oo} the commonly studied behaviours \ref{il:edf:o} and \ref{il:edf:s} for the sequence  $\Nsuite[j]{\iSv[j]}$.
\begin{Liste}[]
\item[\mylabel{il:ass:au:ub:p:o}{\dg\bfseries{(o)}}]
	Let $\iSv[\Di]\sim \Di^{2a}$, $a>0$, then $\sDi{\ssY}\cmSv[\sDi{\ssY}]\sim\ssY^{1/(2a+1)}$ (cf. \nref{il:ass:ub:p} \ref{il:ass:ub:p:o}), while $\ssE\sim(\log\sDi{\ssE})\cmSv[\sDi{\ssY}]\miSv[\sDi{\ssE}]\sim(\log\sDi{\ssE})(\sDi{\ssE})^{2a}$ implies $\sDi{\ssE}\sim(\ssE/\log\ssE)^{1/(2a)}$ and $\sDi{\ssE}\cmSv[\sDi{\ssE}]\sim (\ssE/\log\ssE)^{1/(2a)}$.
\item[\mylabel{il:ass:au:ub:p:s}{\dg\bfseries{(s)}}]
	Let $\iSv[\Di]\sim \exp(\Di^{2a})$, $a>0$, then $\sDi{\ssY}\cmSv[\sDi{\ssY}]\sim (\log \ssY)^{2+1/(2a)}$ (cf. \nref{il:ass:ub:p} \ref{il:ass:ub:p:s}), while $\ssE\sim(\log\sDi{\ssE})\cmSv[\sDi{\ssE}]\miSv[\sDi{\ssE}]\sim (\log\sDi{\ssE})(\sDi{\ssE})^{4a}\exp((\sDi{\ssE})^{2a})$ implies $\sDi{\ssE}\sim(\log \ssE-\tfrac{1+4a}{2a}\log\log\ssE-\tfrac{1}{2a}\log\log\log\ssE)^{1/(2a)}$ and $\sDi{\ssE}\cmSv[\sDi{\ssE}]\sim (\log \ssE)^{2+1/(2a)}$.
\end{Liste}
Clearly, in both cases \ref{il:ass:au:ub:p:o} and \ref{il:ass:au:ub:p:s}, there are ${\ssY}_{\xdf,\iSv},{\ssE}_{\xdf,\iSv}\in\Nz$ such that $\cmiSv[\sDi{\ssY}]\sDi{\ssY}\geq K_{\ydf}(\log\ssY)$ for all $\ssY\geq{\ssY}_{\xdf,\iSv}$ and  $\cmiSv[\sDi{\ssE}]\sDi{\ssE}\geq K_{\ydf}(\log\ssE)$ for all $\ssE\geq{\ssE}_{\xdf,\iSv}$  hold true.
\end{il}


\begin{thm}\label{THM_FREQ_CIRCDECONV_UNKNOWN_IID_ORACLE_P}
Let $\xdf$ have a finite series expansion as defined in \ref{oo:xdf:p}.
Under \nref{ass:au:ub:p} there is a finite numerical constant $\cst{}$ such that for all $\dr\ssY,\ssE\in\Nz$
\begin{equation}\label{re:au:ub:p:e1}
\FuEx[\ssY,\ssE]{\rY,\rE}\VnormLp{\hxdf-\xdf}^2
\leq\cst{}(1\vee\VnormLp{\Proj[{\mHiH[0]^\perp}]\xdf}^2)(\DipenSv[\ssY_{\xdf,\iSv}]\ssY^{-1}+K_{\ydf}\miSv[K_{\ydf}]^2\ssE^{-1}+\ssE_{\xdf,\iSv}\ssE^{-1})
\end{equation}
\end{thm}

Comparison with the oracle rate of projection estimators reveals that in many cases, we obtain an oracle optimal estimator.

\begin{il}\label{IL_FREQ_CIRCDECONV_UNKNOWN_IID_ORACLE_P}
Let us illustrate \nref{THM_FREQ_CIRCDECONV_UNKNOWN_IID_ORACLE_P} considering as in \nref{il:ass:au:ub:p} the behaviours \ref{il:edf:o} and \ref{il:edf:s} for the sequence $\Nsuite[j]{\iSv[j]}$.  
Keeping in mind that as shown in \nref{il:ass:au:ub:p} there are ${\ssY}_{\xdf,\iSv},{\ssE}_{\xdf,\iSv}\in\Nz$ such that $\cmiSv[\sDi{\ssY}]\sDi{\ssY}\geq K_{\ydf}(\log\ssY)$  for all $\ssY\geq{\ssY}_{\xdf,\iSv}$ and $\cmiSv[\sDi{\ssE}]\sDi{\ssE}\geq K_{\ydf}(\log\ssE)$  for all $\ssE\geq{\ssE}_{\xdf,\iSv}$ hold true, due to \nref{THM_FREQ_CIRCDECONV_UNKNOWN_IID_ORACLE_P} there is a constant $\cst{\xdf,\edf}$ depending only on the densities $\xdf$ and $\edf$ such that $\FuEx[\ssY,\ssE]{\rY,\rE}\VnormLp{\hxdf-\xdf}^2\leq \cst{\xdf,\edf}(\ssY^{-1}+\ssE^{-1})$ for all $\ssY,\ssE\in\Nz$.
Comparing the last result with the oracle rates derived in \nref{il:ee} we conclude, that $\hxdf$ is optimal in an oracle sense in both, the case \ref{il:oo:po} and  \ref{il:oo:so}.
\end{il}

\begin{as}\label{ass:au:ub:np}
Let $\xdf$  have an infinite series expansion as definied in \ref{oo:xdf:np}, that is, $1\geq \bias(\xdf)>0$ for all $\Di\in\Nz$.
Given   $\Di_{\ydf}:=\dr3*800^2\Vnormlp[1]{\fydf}^2\cpen^{-2}$ and $\tDi_{\ydf}=\min\{\Di\in\Nz:\bias[\Di_{\ydf}](\xdf)>\bias[\Di](\xdf)\}$ there are $\ssY_{\xdf,\iSv},\ssE_{\xdf,\iSv}\in\Nz$ with $\ssY_{\xdf,\iSv}\geq\DipenSv[\tDi_{\ydf}]\bias[\tDi_{\ydf}]^{-2}(\xdf)\vee\dr15*300^4\cpen^{-2}$ and $\ssE_{\xdf,\iSv}\geq289\log(\Di_{\ydf}+2)\cmiSv[\Di_{\ydf}]\miSv[\Di_{\ydf}]$ such that \begin{inparaenum}[i]\renewcommand{\theenumi}{\dgrau\rm(\alph{enumi})} \item\label{ass:au:ub:np:c0} for all $\ssE\geq\ssE_{\xdf,\iSv}$, $\sDi{\ssE}:=\max\{\Di\in\nset{\Di_{\ydf},\ssE}:289\log(\Di+2)\cmiSv[\Di]\miSv[\Di]\leq\ssE\}$, where the defining set containing $\Di_{\ydf}$ is not empty, satisfies $\cmiSv[\sDi{\ssE}]\sDi{\ssE}\geq \Di_{\ydf}|\log\mRa{\xdf,\iSv}|$ and  either 
\item\label{ass:au:ub:np:c1}
$\cmiSv[\aDi{\ssY}]\aDi{\ssY}\geq \Di_{\ydf}|\log\hRa{\xdf,\iSv}|$ 
for all
$\ssY\geq{\ssY}_{\xdf,\iSv}$ or \item\label{ass:au:ub:np:c2}  
$\aDi{\ssY}\leq  \Di_{\ydf}|\log\hRa{\xdf,\iSv}|$ for all
$\ssY\geq{\ssY}_{\xdf,\iSv}$. \end{inparaenum} We set 
$\sDi{\ssY}:= \ceil{\Di_{\ydf}|\log\hRa{\xdf,\iSv}|}\wedge\ssY$ for
$\ssY<\ssY_{\xdf,\iSv}$ and $\sDi{\ssY}:= \ceil{\Di_{\ydf}|\log\hRa{\xdf,\iSv}|}\vee\aDi{\ssY}$ for
$\ssY\geq\ssY_{\xdf,\iSv}$, and in addition $\sDi{\ssE}:=\sDi{\ssY}$
for $\ssE<\ssE_{\xdf,\iSv}$, where consequently in case
\ref{ass:au:ub:np:c1}  $\sDi{\ssY}= \Di_{\ydf}|\log\hRa{\xdf,\iSv}|$ for
$\ssY<\ssY_{\xdf,\iSv}$, $\sDi{\ssY}=\aDi{\ssY}$ for
$\ssY\geq\ssY_{\xdf,\iSv}$ and in case \ref{ass:au:ub:np:c2}
$\sDi{\ssY}= \Di_{\ydf}|\log\hRa{\xdf,\iSv}|$ for all $\ssY\in\Nz$.
\end{as}

\begin{il}\label{il:ass:au:ub:np}
Let us illustrate  \nref{ass:au:ub:np}
  considering as in \nref{il:oo} usual
  behaviour \ref{il:oo:oo}, \ref{il:oo:so} and \ref{il:oo:os}
 for the sequences $\Nsuite[\Di]{\bias[\Di](\xdf)}$ and
  $\Nsuite[\Di]{\iSv[\Di]}$:
 \begin{Liste}[]
\item[\mylabel{il:ass:au:ub:np:oo}{\dg\bfseries{[o-o]}}]
$\cmiSv[\aDi{\ssY}]\aDi{\ssY}\sim\ssY^{1/(2p+2a+1)}$ and
$|\log\hRa{\xdf,\iSv}|\sim(\log\ssY)$ (cf.  \nref{il:ass:ub:np}
\ref{il:ass:ub:np:oo}) while
$\sDi{\ssE}\cmSv[\sDi{\ssE}]\sim (\ssE/\log\ssE)^{1/(2a)}$ (cf.  \nref{il:ass:au:ub:p}
\ref{il:ass:au:ub:p:o})  and $|\log\mRa{\xdf,\iSv}|\sim(\log\ssE)$ (cf.  \nref{il:ee}
\ref{il:ee:oo})
 \item[\mylabel{il:ass:au:ub:np:os}{\dg\bfseries{[o-s]}}]
$\cmiSv[\aDi{\ssY}]\aDi{\ssY}\sim(\log \ssY)^{2+1/(2a)}$ and
$|\log\hRa{\xdf,\iSv}|\sim(\log\log\ssY)$ (cf.  \nref{il:ass:ub:np}
\ref{il:ass:ub:np:os}) while
$\sDi{\ssE}\cmSv[\sDi{\ssE}]\sim (\log\ssE)^{2+1/(2a)}$ (cf.  \nref{il:ass:au:ub:p}
\ref{il:ass:ub:p:s})  and $|\log\mRa{\xdf,\iSv}|\sim(\log\log\ssE)$ (cf.  \nref{il:ee}
\ref{il:ee:os})
 \item[\mylabel{il:ass:au:ub:np:so}{\dg\bfseries{[s-o]}}]
$\cmiSv[\aDi{\ssY}]\aDi{\ssY}\sim(\log \ssY)^{1/(2p)}$ and
$|\log\hRa{\xdf,\iSv}|\sim(\log\ssY)$ (cf.  \nref{il:ass:ub:np}
\ref{il:ass:ub:np:so}) while
$\sDi{\ssE}\cmSv[\sDi{\ssE}]\sim (\ssE/\log\ssE)^{1/(2a)}$ (cf.  \nref{il:ass:au:ub:p}
\ref{il:ass:au:ub:p:o}) and $|\log\mRa{\xdf,\iSv}|\sim(\log\ssE)$ (cf.  \nref{il:ee}
\ref{il:ee:so})
\end{Liste}
Clearly, there is ${\ssE}_{\xdf,\iSv}\in\Nz$ such that for all
$\ssE\geq{\ssE}_{\xdf,\iSv}$ in the three cases
\ref{il:ass:au:ub:np:oo}, \ref{il:ass:ub:np:os} and
\ref{il:ass:ub:np:so}   $\cmiSv[\sDi{\ssE}]\sDi{\ssE}\geq
\Di_{\ydf}|\log\mRa{\xdf,\iSv}|$, i.e., \nref{ass:au:ub:np}
\ref{ass:au:ub:np:c0} holds.
On the other hand side,  there is ${\ssY}_{\xdf,\iSv}\in\Nz$ such that for all
$\ssY\geq{\ssY}_{\xdf,\iSv}$ in the cases \ref{il:ass:ub:np:oo} and
\ref{il:ass:ub:np:os}   $\cmiSv[\aDi{\ssY}]\aDi{\ssY}\geq
\Di_{\ydf}|\log\hRa{\xdf,\iSv}|$, i.e., \nref{ass:au:ub:np}
\ref{ass:au:ub:np:c1} holds,   while in case \ref{il:ass:au:ub:np:so}
$\aDi{\ssY}\leq \Di_{\ydf}|\log\hRa{\xdf,\iSv}|$ for $p\geq1/2$, i.e., \nref{ass:au:ub:np}
\ref{ass:au:ub:np:c2} holds, and $\cmiSv[\aDi{\ssY}]\aDi{\ssY}\geq
\Di_{\ydf}|\log\hRa{\xdf,\iSv}|$ for $p<1/2$, i.e., \nref{ass:au:ub:np}
\ref{ass:au:ub:np:c1} holds.
\end{il}

\begin{thm}\label{THM_FREQ_CIRCDECONV_UNKNOWN_IID_ORACLE_NP}
\label{re:au:ub:np} Let $\xdf$ have an infinite series expansion
  as definied in \ref{oo:xdf:np}. Under \nref{ass:au:ub:np} there is a finite numerical
  constant $\cst{}$ such that for all $\dr\ssY,\ssE\in\Nz$
\begin{multline}\label{re:au:ub:np:e1}
\FuEx[\ssY,\ssE]{\rY,\rE}\VnormLp{\hxdfPr[]-\xdf}^2\leq
\cst{}\big\{[1\vee\VnormLp{\Proj[{\mHiH[0]^\perp}]\xdf}^2]\,\hRaDi{\sDi{\ssY}\wedge\sDi{\ssE},\xdf,\iSv}+\mRa{\xdf,\iSv}\\
\hfill+[1\vee\VnormLp{\Proj[{\mHiH[0]^\perp}]\xdf}^2](\DipenSv[\ssY_{\xdf,\iSv}]\ssY^{-1}+\ssE_{\xdf,\iSv}\ssE^{-1})+\Vnormlp[1]{\fydf}^2\ssY^{-1}+\ssE_{\xdf,\iSv}^2\ssE^{-1}\big\}.
\end{multline}
\end{thm}

Comparison with the oracle rate of projection estimators reveals that in many cases, we obtain an oracle optimal estimator.

\begin{cor}\label{COR_FREQ_CIRCDECONV_UNKNOWN_IID_ORACLE_NP}
Under the assumptions of
  \nref{re:au:ub:np}  for  $\ssY\in\Nz$ let
  $\ssE_{\ssY}:=\ssE(\ssY)\in\Nz$ such that
  $\aDi{\ssY}\leq\sDi{\ssE_{\ssY}}$. If in addition
  $\lim_{\ssY\to\infty}\cmiSv[\aDi{\ssY}]\aDi{\ssY}|\log\hRa{\xdf,\iSv}|^{-1}=\infty$, then there is a finite constant $\cst{\xdf,\edf}$ depending on the densities 
$\xdf$ and $\edf$ such that
\begin{equation*}
\FuEx[\ssY,\ssE]{\rY,\rE}\VnormLp{\hxdf-\xdf}^2\leq\cst{\xdf,\edf}(\hRa{\xdf,\iSv}+\mRa[\ssE_{\ssY}]{\xdf,\iSv})\text{
  for all } \ssY\in\Nz .
\end{equation*}
\end{cor}

\begin{il}\label{IL_FREQ_CIRCDECONV_UNKNOWN_IID_ORACLE_NP}
Let us illustrate  \nref{re:au:ub:np} considering as in \nref{il:ass:au:ub:np} usual
  behaviour  \ref{il:ass:au:ub:np:oo}, \ref{il:ass:au:ub:np:os} and \ref{il:ass:au:ub:np:so}
 for the sequences $\Nsuite[\Di]{\bias[\Di](\xdf)}$ and
  $\Nsuite[\Di]{\iSv[\Di]}$. In light of \nref{il:ass:au:ub:np}, we
  apply \nref{re:au:ub:np}, where  we need only check \nref{ass:au:ub:np}. The rates then follow by an evaluation of the upper bound. Let
  $\Nsuite[\ssY]{\ssE_{\ssY}}$ be a sequence of positive integers and
  suppose that the limits  $q_{\text{o-o}}$, $q_{\text{o-s}}$, and
$q_{\text{s-o}}$  defined in \nref{il:ee} exists in the respective cases.
\begin{Liste}[]
\item[\mylabel{il:au:ub:np:oo}{\dg\bfseries{[o-o]}}] 
Since \nref{ass:au:ub:np} \ref{ass:au:ub:np:c0} with $\sDi{\ssE}\sim
(\ssE/\log\ssE)^{1/(2a)}$ and \ref{ass:au:ub:np:c1} with
$\oDi{\ssY}\sim \ssY^{1/(2p+2a+1)}$ hold true
(cf., respectively, \nref{il:ass:au:ub:p} \ref{il:ass:au:ub:p:o} and
\nref{il:ass:ub:np} \ref{il:ass:ub:np:oo}), due to
\nref{re:au:ub:np} and \nref{il:ee} \ref{il:ee:oo}
there is a constant $\cst{\xdf,\edf}$ depending on $\xdf$ and $\edf$
such that
\begin{equation}\label{il:au:ub:np:oo:e1}\FuEx[\ssY,\ssE]{\rY,\rE}\VnormLp{\hxdf-\xdf}^2\leq\cst{\xdf,\edf}\{(\aDi{\ssY}\wedge\sDi{\ssE})^{-2p}+\ssE^{-(p\wedge
    a)/a}\},\quad\forall\;\ssY,\ssE\in\Nz.\end{equation} 
We consider two cases. Firstly, let $p> a$. If
$q_{\text{o-o}}=\lim_{\ssY\to\infty}\ssY^{2p/(2p+2a+1)}\ssE_{\ssY}^{-1}<\infty$,
then
\begin{equation*}
\frac{\aDi{\ssY}}{\sDi{\ssE}}\sim\frac{\ssY^{1/(2p+2a+1)}}{(\ssE_{\ssY}/\log\ssE_{\ssY})^{1/(2a)}}=\frac{\ssY^{1/(2p+2a+1)}}{(\ssE_{\ssY})^{1/(2p)}}\frac{(\log\ssE_{\ssY})^{1/(2a)}}{(\ssE_{\ssY})^{1/(2a)-1/(2p)}}=o(1).
\end{equation*}
This means $\aDi{\ssY}\lesssim\sDi{\ssE}$ so the resulting upper bound
is of order
$(\aDi{\ssY})^{-2p}+\ssE_{\ssY}^{-1}\lesssim(\aDi{\ssY})^{-2p}$. Suppose
now that $q_{\text{o-o}}=\infty$. If in addition
$q^b_{\text{o-o}}=\lim_{\ssY\to\infty}\aDi{\ssY}(\sDi{\ssE})^{-1}<\infty$
then the first summand in the upper bound in \eqref{il:au:ub:np:oo:e1}
reduces to $(\aDi{\ssY})^{-2p}$ and thus (keep $q_{\text{o-o}}=\infty$
in mind) the resulting upper bound
is of order $\ssE_{\ssY}^{-1}$. Now consider
$q^b_{\text{o-o}}=\infty$, then   the upper bound in
\eqref{il:au:ub:np:oo:e1} is of order $(\sDi{\ssE})^{-2p}+\ssE_{\ssY}^{-1}\lesssim\ssE_{\ssY}^{-1}$ because $p>a$. Combining both
  cases, we obtain in case $p>a$ that as
 $\ssY\to\infty$
\begin{equation*}
\FuEx[\ssY,\ssE]{\rY,\rE}\VnormLp{\hxdf-\xdf}^2=\left\{\begin{array}{ll}
O(\ssY^{-2p/(2p+2a+1)}),& \text{if }q_{\text{o-o}}<\infty,\\
O(\ssE_{\ssY}^{-1}),& \text{otherwise }.
\end{array}\right.
\end{equation*}
Now assume $p\leq a$. First, suppose 
$q^b_{\text{o-o}}=\lim_{\ssY\to\infty}\aDi{\ssY}(\sDi{\ssE})^{-1}<\infty$
then the first summand in the upper bound in \eqref{il:au:ub:np:oo:e1}
reduces to $(\aDi{\ssY})^{-2p}$ and moreover, it follows that
$q_{\text{o-o}}<\infty$. Therefore, the resulting upper bound
is of order $(\aDi{\ssY})^{-2p}$. Now consider
$q^b_{\text{o-o}}=\infty$, then   the upper bound in
\eqref{il:au:ub:np:oo:e1} is of order $(\ssE_{\ssY}/\log\ssE_{\ssY})^{-p/a}+\ssE_{\ssY}^{-p/a}\lesssim(\ssE_{\ssY}/\log\ssE_{\ssY})^{-p/a}$. Combining both
  cases, we obtain in case $p\leq a$ that as
 $\ssY\to\infty$
\begin{equation*}
\FuEx[\ssY,\ssE]{\rY,\rE}\VnormLp{\hxdf-\xdf}^2=\left\{\begin{array}{ll}
O(\ssY^{-2p/(2p+2a+1)}),& \text{if }q^b_{\text{o-o}}<\infty,\\
O((\ssE_{\ssY}/\log\ssE_{\ssY})^{-p/a}),& \text{otherwise }.
\end{array}\right.
\end{equation*}
 \item[\mylabel{il:au:ub:np:os}{\dg\bfseries{[o-s]}}]
Since \nref{ass:au:ub:np} \ref{ass:au:ub:np:c0} with $\sDi{\ssE}\sim
(\log\ssE)^{1/(2a)}$ and \ref{ass:au:ub:np:c1} with
$\oDi{\ssY}\sim (\log\ssY)^{1/(2a)}$ hold true
(cf., respectively, \nref{il:ass:au:ub:p} \ref{il:ass:au:ub:p:s} and
\nref{il:ass:ub:np} \ref{il:ass:ub:np:os}), due to
\nref{re:au:ub:np} and \nref{il:ee} \ref{il:ee:os}
there is a constant $\cst{\xdf,\edf}$ depending on $\xdf$ and $\edf$
such that
\begin{equation}\label{il:au:ub:np:os:e2}\FuEx[\ssY,\ssE]{\rY,\rE}\VnormLp{\hxdf-\xdf}^2\leq\cst{\xdf,\edf}\{(\log\ssY)^{-p/a}+(\log\ssE)^{-p/a}\},\quad\forall\;\ssY,\ssE\in\Nz.\end{equation} 
Considering $q_{\text{o-s}}=\lim_{\ssY\to\infty}(\log \ssY)(\log \ssE_{\ssY})^{-1}$  it follows that as
 $\ssY\to\infty$
\begin{equation*}
\FuEx[\ssY,\ssE]{\rY,\rE}\VnormLp{\hxdf-\xdf}^2=\left\{\begin{array}{ll}
O\big((\log\ssY)^{-p/a}\big),& \text{if }q_{\text{o-s}}<\infty,\\
O\big((\log \ssE_{\ssY})^{-p/a}\big),& \text{otherwise }.
\end{array}\right.
\end{equation*}
\item[\mylabel{il:au:ub:np:so}{\dg\bfseries{[s-o]}}] 
Since \nref{ass:au:ub:np} \ref{ass:au:ub:np:c0} with $\sDi{\ssE}\sim
(\ssE/\log\ssE)^{1/(2a)}$, \ref{ass:au:ub:np:c2} with
$\sDi{\ssY}\sim (\log\ssY)$ for $p\geq1/2$ and \ref{ass:au:ub:np:c1} with
$\sDi{\ssY}\sim (\log\ssY)^{1/(2p)}$ for $p<1/2$  hold true
(cf., respectively, \nref{il:ass:au:ub:p} \ref{il:ass:au:ub:p:o} and
\nref{il:ass:ub:np} \ref{il:ass:ub:np:so}), 
 due to
\nref{re:au:ub:np} and \nref{il:ee} \ref{il:ee:os}
there is a constant $\cst{\xdf,\edf}$ depending on $\xdf$ and $\edf$
such that
\begin{equation}\label{il:au:ub:np:os:e3}\FuEx[\ssY,\ssE]{\rY,\rE}\VnormLp{\hxdf-\xdf}^2\leq\cst{\xdf,\edf}\{\hRaDi{\sDi{\ssY}\wedge(\ssE/\log\ssE)^{1/(2a)},\xdf,\iSv}+\ssE^{-1}\},\quad\forall\;\ssY,\ssE\in\Nz.\end{equation} 
Clearly, if
$q^b_{\text{s-o}}=\lim_{\ssY\to\infty}\ssY(\sDi{\ssY})^{-(2a+1)}\ssE_{\ssY}^{-1}<\infty$
then holds $\sDi{\ssY}=(\log\ssY)^{1\vee1/(2p)}\lesssim(\ssE_{\ssY}/\log\ssE_{\ssY})^{1/(2a)}$ and hence 
$\hRaDi{\sDi{\ssY}\wedge(\ssE_{\ssY}/\log\ssE_{\ssY})^{1/(2a)},\xdf,\iSv}+\ssE_{\ssY}^{-1}\lesssim(\sDi{\ssY})^{2a+1}\ssY^{-1}$. Suppose
now that $q^b_{\text{s-o}}=\infty$, then 
\begin{multline*}
\hRaDi{\sDi{\ssY}\wedge(\ssE_{\ssY}/\log\ssE_{\ssY})^{1/(2a)},\xdf,\iSv}+\ssE_{\ssY}^{-1}\\
\hfill\lesssim(\sDi{\ssY})^{2a+1}\ssY^{-1}\vee\hRaDi{(\ssE_{\ssY}/\log\ssE_{\ssY})^{1/(2a)},\xdf,\iSv}+\ssE_{\ssY}^{-1}\\
\lesssim\exp(-(\ssE_{\ssY}/\log\ssE_{\ssY})^{p/a})\vee\ssY^{-1}(\ssE_{\ssY}/\log\ssE_{\ssY})^{-(2a+1)/(2a)}+\ssE_{\ssY}^{-1}\lesssim \ssE_{\ssY}^{-1}.
\end{multline*}
Consequently, 
it follows that as
 $\ssY\to\infty$
\begin{equation*}
\FuEx[\ssY,\ssE]{\rY,\rE}\VnormLp{\hxdf-\xdf}^2=\left\{\begin{array}{ll}
O\big(\ssY^{-1}(\log\ssY)^{(2a+1)[1\vee1/(2p)]}\big),& \text{if }q^b_{\text{o-s}}<\infty,\\
O\big(\ssE_{\ssY}^{-1}),& \text{otherwise }.
\end{array}\right.
\end{equation*}
  \end{Liste}
Comparing the last rates with the oracle rates derived in
 \nref{il:oo} \ref{il:oo:oo}, \ref{il:oo:os} and \ref{il:oo:so} we see
 that in case \ref{il:au:ub:np:oo} with $p>a$, \ref{il:au:ub:np:os} and
 \ref{il:au:ub:np:so} with $p<1/2$ $\hxdf$ attains
 the oracle rate, while in case \ref{il:au:ub:np:oo} with $p\leq a$
 and  \ref{il:au:ub:np:so} with $p\geq1/2$ the rate of the fully data-driven estimator $\hxdf$ features a detoriation  by a logarithmic factor compared to the
 oracle rate.
\end{il}
\section{Circular deconvolution with beta mixing data and partially known noise density}
%
\appendix
%
\chapter{Simulation skim}\label{A}
\chapter{R Code}\label{B}
%
\bibliography{biblio.bib}{}
\nocite{*}
\bibliographystyle{apalike}
%

\end{document}
%%% Local Variables:
%%% mode: latex
%%% TeX-master: t
%%% End:
