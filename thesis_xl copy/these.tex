\documentclass[a4paper,11pt]{book}
\synctex=1

\usepackage{graphicx}
% put all the other packages here:


\usepackage{packages}
\usepackage{ACDE-abrev-package}
\usepackage{command}
\usepackage{style}
\usepackage{personal}
\usepackage{theorem}
\usepackage{hyperref} 

\begin{document}
\frontmatter
%%%%%%%%%%%%%%%%%%%%%%%% Page 1 %%%%%%%%%%%%%%%%%%%%%%%%%
\pagestyle{empty}

\begin{center}
\huge	\textbf{INAUGURAL-DISSERTATION}
\end{center} 

\bigskip

\bigskip

\begin{center}
	\Large zur\\
	Erlangung der Doktorw\"urde\\
	der\\
	Naturwissenschaftlich-Mathematischen Gesamtfakult\"at\\
	der Ruprecht-Karls-Universit\"at\\
	Heidelberg
\end{center}

\vspace*{\fill}

\begin{center}
	\large vorgelegt von 
\end{center}

\bigskip

\begin{center}
\Large \myDegree \myName

\bigskip

aus \myBirthPlace
\end{center}

\vspace*{\fill}

\begin{flushleft}
\Large Tag der m\"undlichen Pr\"ufung: % do not print the date !
\end{flushleft}

\newpage

%%%%%%%%%%%%%%%%%%%%%%%% Page 2 %%%%%%%%%%%%%%%%%%%%%%%%%
~

\bigskip

\begin{center}
\huge	\textbf{\myTitle}
\end{center} 

\vspace*{\fill}

\Large
\begin{center}
\noindent\begin{tabular}{@{}l@{\enskip}l}
Betreuer:& \mySupervisor\\
& \myCoSupervisor
\end{tabular}
\end{center}

\bigskip

~

\newpage 

%%%%%%%%%%%%%%%%%%%%%%%% Page 3 %%%%%%%%%%%%%%%%%%%%%%%%%

~

\bigskip

\huge \textbf{Acknowledgments}

\vspace*{\fill}

\large 
\myAcknowledgements


\bigskip

~

\newpage 

%%%%%%%%%%%%%%%%%%%%%%%% Page 4 %%%%%%%%%%%%%%%%%%%%%%%%%

\huge \textbf{Zusammenfassung}

\vspace*{\fill}

\normalsize
\myZusammenfassung



\newpage

%%%%%%%%%%%%%%%%%%%%%%%% Page 5 %%%%%%%%%%%%%%%%%%%%%%%%%


\huge \textbf{Abstract}

\vspace*{\fill}

\normalsize
\myAbstract


\bigskip

~

\newpage 

%%%%%%%%%%%%%%%%%%%%%%%% Page 6 %%%%%%%%%%%%%%%%%%%%%%%%%


\thispagestyle{empty}  % one empty page before printing a dedication (you may do that) or contents
\quad 
\newpage
%
\tableofcontents
%
%
\mainmatter

\pagestyle{fancy}
\chapter{Introduction}\label{INTRO}
\section{Inverse problems}\label{INTRO_INVERSEPROBLEMS}
\textit{We introduce here some fundamentals of inverse problem theory. This section builds upon results which can be found, for example, in \ncite{engle1996regularization}}.

Consider the situation when one wishes to estimate an object, say $f$ belonging to a space $\Xi$.
The object $f$ will be referred to as "parameter of interest" and the space $\Xi$ as "parameter space".
We assume that this parameter has some influence on a system which we are able to observe.
Hence, recording observation of this system allows us to learn about this parameter.
These observations will be referred to as "data" and denoted by $Y$.
Our ability to learn in such a way is central as it underpins our ability to understand the behaviour of a system, to predict it and to influence it.
This is a wide family of problems and we shall give more precision about the specific subfamily we consider.

We will give particular interest to inverse problems, a family of models where one wants to infer on $f$ but the data we observe comes from a system led by a different parameter $g$ which can be written $g := T(f)$ where $T$ is an mapping from $\Xi$ to itself.

These models gathered interest for a long time due to their numerous applications, theoretical physics, astrophysics, medical imaging, econometrics, or acoustics are just a few of the countless examples of such applications.
Many of those models have the particularity to be ill-posed in the sense of \citet{cite:hadamard}.
That is to say, if we build an estimator $\widehat{g}$ of $g = T(f)$ from the data $Y$ and try to apply the inverse $T^{-1}$ of $T$ to this estimator in order to estimate $f$, one of the following problems might arise:
\begin{itemize}
\item non existence (the equation $T(x) = \widehat{g}$ does not have a solution);
\item non unicity (the equation $T(x) = \widehat{g}$ has multiple solutions);
\item non stability (the solutions to the equations $T(x) = \widehat{g}$ does not depend continuously on $\widehat{g}$).
\end{itemize}

Though Hadamard thought that inverse problems do not arise in practical situations and that problems of our realm only are of the well-posed kind.
Evolution of science proved him wrong and ill-posed problems now have many applications.
The specific challenges they represent has since gathered ever increasing interest.
We will use two examples throughout this thesis, respectively introduced in \nref{INTRO_IGSSM} and \nref{INTRO_CIRCULARDECONVOLUTION}.

\bigskip

From now on, we will assume that $\Xi$ is an infinite dimensional vector space on $\mathds{K}$ (standing for either $\R$ or $\mathds{C}$), equipped with a norm $\Vert \cdot \Vert_{\Xi}$ which is derived from an inner product $\langle \cdot \vert \cdot \rangle_{\Xi}$ and $\Xi$ is hence an infinite dimensional Hilbert space.
We denote by $\mathcal{L}(\Xi)$ the set of bounded endomorphisms on $\Xi$, that is to say linear operators $S$ from $\Xi$ onto itself such that there exists $M$ in $\R_{+}$ verifying, for any $x$ in $\Xi$, the following inequality $\Vert S(x) \Vert_{\Xi} \leq M \Vert x \Vert_{\Xi}$.
In addition, we denote, for any $S$ in $\mathcal{L}(\Xi)$, $\mathcal{D}(S)$ its definition domain, $\mathcal{R}(S)$ its range, and $\mathcal{N}(S)$ its kernel.
Assume, from now on, that $T$ is an element of $\mathcal{L}(\Xi)$.

In this case, the following property gives us sufficient and necessary conditions under which the two first forms of ill-posedness do not happen.

\begin{pr*}
For any $S$ in $\mathcal{L}(\Xi)$, and any element $x$ of $\Xi$, there exists an unique solution to the equation $S(y) = \widehat{S(x)}$ for any estimate $\widehat{S(x)}$ of $S(x)$ in $\Xi$ if and only if
\item[\mylabel{BACKGROUND_INVERSEPROBLEMS_EXISTENCE}{\dgrau{\bfseries{(existence): }}}] $\widehat{S(x)}$ belongs to the range $\mathcal{R}(S)$ of $S$;
\item[\mylabel{BACKGROUND_INVERSEPROBLEMS_UNIQUENESS}{\dgrau\bfseries{(uniqueness): }}] the operator $S$ is injective, i.e. $\mathcal{N}(S) = \{0\}$.
\reEnd
\end{pr*}

In the case where the existence condition is not fulfilled, one would look for an approximate solution $\widetilde{f}$ minimising an objective function which could be the distance with respect to $\Vert \cdot \Vert_{\Xi}$, that is to say, if it exists, $\widetilde{f} \in \argmin_{x \in \mathcal{D}(T)}\Vert T(x) - \widehat{g} \Vert_{\Xi}$.
If the uniqueness condition is not fulfilled then we can look for the solution with minimal norm, once again, assuming that it exists.

We will see that the orthogonal projection operators, with respect to $\langle \cdot \vert \cdot \rangle_{\Xi}$, plays an important role.
Indeed, one can show how the last property relates to the  orthogonal projection onto the closure of the range of $T$, $\overline{\mathcal{R}}(T)$.
First introduce the following notations.

\begin{de}
For any $S$ in $\mathcal{L}(\Xi)$, denote by $S^{\star}$ its adjoint operator with respect to $\langle \cdot \vert \cdot \rangle_{\Xi}$, that is to say the unique operator such that for any $x$ and $y$ in $\Xi$ we have $ \langle S(x) \vert y \rangle_{\Xi} = \langle x \vert S^{\star}(y) \rangle_{\Xi}$.
For any subspace $\mathds{U}$ of $\Xi$, denote by $\Pi_{\mathds{U}}$ the orthogonal projection onto $\mathds{U}$ with respect to $\langle \cdot \vert \cdot \rangle_{\Xi}$.
\assEnd
\end{de}

We can now formulate the following property linking the distance minimising criteria with the orthogonal projection onto the closure of the range of $T$.

\begin{pr*}
For any $S$ in $\mathcal{L}(\Xi)$; any element $x$ of $\Xi$; any estimate $\widehat{S(x)}$ of $S(x)$ in $\Xi$; and any estimate $\widetilde{x}$ of $x$ which lies within $\mathcal{D}(S)$, the following assertions are equivalent:
\item[\mylabel{BACKGROUND_INVERSEPROBLEMS_PROJECTION_i}{\dgrau{\bfseries{i (distance to the target minimisation)}}}]: $\widetilde{x}$ minimises the function $y \mapsto \Vert \widehat{S(x)} - S(y) \Vert_{\Xi}$;
\item[\mylabel{BACKGROUND_INVERSEPROBLEMS_PROJECTION_ii}{\dgrau\bfseries{ii}}]: $\Pi_{\overline{R}(S)}(\widehat{S(x)}) = S(\widetilde{x})$;
\item[\mylabel{BACKGROUND_INVERSEPROBLEMS_PROJECTION_iii}{\dgrau\bfseries{iii (normal equation)}}]: $S^{\star}(\widehat{S(x)}) = S^{\star}(S(\widetilde{x}))$.
\reEnd
\end{pr*}

Given those considerations, it is naturally that one defines the generalised inverse (also called pseudo inverse or Moore-Penrose inverse).

\begin{de}
For any linear subspace $\mathds{U}$ of $\Xi$, denote $\mathds{U}^{\perp}$ its orthogonal complement with respect to $\langle \cdot \vert \cdot \rangle_{\Xi}$ that is $\mathds{U}^{\perp} := \{x \in \Xi: \forall u \in \mathds{U}, \langle x \vert u \rangle_{\Xi} = 0\}$.
Moreover, denote $\oplus$ the direct sum binary operator.
Then, for any linear operator $S$, define its generalised inverse $S^{+}$ as the unique linear extension of $S^{-1}: \mathcal{R}(S) \rightarrow \mathcal{N}(S)^{\perp}$ to the domain $\mathcal{D}(S^{+}) := \mathcal{R}(S) \oplus \mathcal{R}(S)^{\perp}$ with $\mathcal{N}(S^{+}) = \mathcal{R}(S)^{\perp}$ satisfying for any $x$ in $\mathcal{D}(S^{+})$ the equality $S^{+}(x) := S^{-1}(\Pi_{\overline{\mathcal{R}}(S)}(x))$.
\assEnd
\end{de}

One should note that the generalised inverse has the following important properties.

\begin{rmk}
For any $S$ in $\mathcal{L}(\Xi)$, the following equalities stand: $S S^{+} S = S$, $S^{+} S S^{+} = S^{+}$, $S^{+} S = \Pi_{\mathcal{N}(S)^{\perp}}$ and for any $x$ in $\mathcal{D}(S^{+})$, $S S^{+}(x) = \Pi_{\overline{\mathcal{R}}(S)}(x)$.
In addition, one should notice that if $S$ is injective, so is $S^{\star}S$ and as a consequence, $S^{\star}S : \Xi \rightarrow \mathcal{R}(S^{\star}S)$ is invertible which implies that for any $x$ in $\mathcal{R}(S) \oplus \mathcal{R}(S)^{\perp}$ we have that $(S^{\star} S)^{+} S^{\star} x$ is the unique solution of \ref{BACKGROUND_INVERSEPROBLEMS_PROJECTION_iii} which implies that $S^{-1}(\Pi_{\overline{\mathcal{R}}(S)}x) = \{S^{+} x\} = \{(S^{\star} S)^{+} S^{\star} x\}$.
Moreover, if $S$ is invertible, $S^{+}$ and $S^{-1}$ coincide.
\remEnd
\end{rmk}

We hence see that the Moore-Penrose inverse offers a solution to the two first sources of ill-posedness.

\begin{pr*}
For any linear operator $S$ from $\Xi$ onto itself and $x$ in $\mathcal{D}(S^{+})$, $S^{+}(x)$ is an element of $S^{-1}(\Pi_{\overline{R}(S)} x)$ and, hence fulfils \ref{BACKGROUND_INVERSEPROBLEMS_PROJECTION_i}.
Moreover, $S^{+}(x)$ is the unique element fulfilling this condition with minimal $\Vert \cdot \Vert_{\Xi}$-norm, that is $\Vert S^{+} x \Vert_{\Xi} = \inf \{\Vert h \Vert_{\Xi}: h \in S^{-1}(\Pi_{\overline{\mathcal{R}}(S)} x)\}$.
\reEnd
\end{pr*}

We will work under a set of assumptions where the two first kinds of ill-posedness do not happen.
However, we give more attention to the third source of ill-posedness.
The next property gives a general condition under which it occurs.

\begin{pr*}
Let $\Xi$ be infinite dimensional and $S$ be an injective compact linear operator from $\Xi$ onto itself.
Then $\inf_{h \in \Xi} \{\Vert S(h) \Vert_{\Xi}: \Vert h \Vert_{\Xi} = 1 \} = 0$ which implies that $S^{-1}$ (and hence $S^{+}$) are not continuous.
\reEnd
\end{pr*}

This discontinuity property highlights the need to define a so called regularised version of the Moore-Penrose inverse.
Indeed, it implies that there exists $\epsilon$ in $\R_{+}^{\star}$ such that for any $\delta$ in $\R_{+}^{\star}$, there exists a couple $(x, y)$ of elements of $\Xi$ with $\Vert x - y \Vert_{\Xi} \leq \delta$, such that $\Vert S^{+}(x) - S^{+}(y) \Vert_{\Xi} \geq \epsilon$.
Taking $x = g$ and $(y_{n})_{n \in \N} = (\widehat{g}_{n})_{n \in \N}$ a sequence of estimators, it means that even if $(\widehat{g}_{n})_{n \in \N}$ is a consistent sequence of estimations for $g$, $S^{+}(\widehat{g})$ would still not be a consistent estimator of $f$.

\medskip

We will see later in this overview that depending on the approach one uses, the strategy to overcome this difficulty will not be the same.
Namely, in the frequentist paradigm, one introduces the notion of regularisation in order to define a continuous approximation of $T^{+}$ whereas in the Bayesian paradigm, this regularisation occurs naturally in this derivation of the posterior distribution.

To make this clearer, we will first introduce the shape that our data will take.
\section{Frequentist approach}\label{1.2}

\subsection{Estimation}\label{1.2.1}
\begin{itemize}
\item (M/Z-estimation);
\item projection;
\item kernel smoother...
\end{itemize}

\subsection{Decision theory}\label{1.2.2}
\begin{itemize}
\item Loss function;
\item risk;
\item oracle optimality;
\item minimax optimality;
\end{itemize}

\subsection{Adaptivity}\label{1.2.3}
\begin{itemize}
\item penalised contrast
\item Lepski
\item ...
\end{itemize}
\section{Bayesian approach}\label{1.3}

\subsection{The Bayesian paradigm}\label{1.3.1}
\begin{itemize}
\item Bayes' theorem;
\item prior distribution;
\item posterior distribution (include conditions of existence);
\end{itemize}

\subsection{Typical priors for non-parametric models}\label{1.3.2}
\begin{itemize}
\item Gaussian process prior
\item Sieve priors (specific case)

\[\mathds{P}^{n}_{\boldsymbol{\theta}}(\theta) = \exp\left[-\frac{1}{2}\sum\limits_{\vert j \vert \leq m} \vert \theta_{j} \vert^{2}\right] \cdot \prod\limits_{\vert j \vert > m}\delta_{0}(\theta_{j})\]

\item Chinese restaurant process
\item Dirichlet process
\end{itemize}

\subsection{The pragmatic Bayesian approach}\label{1.3.3}
\begin{itemize}
\item Consistence
\item contraction rate
\item exact contraction rate
\item uniform contraction rate
\item oracle optimality
\item minimax optimality
\end{itemize}

\subsection{Existing central results}\label{1.3.4}
\begin{itemize}
\item Goshal Van der Vaart
\item Nickl
\end{itemize}

\subsection{Iteration procedure, self informative limit and Bayes carrier}\label{1.3.5}
\section{First examples of inverse problem: the inverse Gaussian sequence space model}\label{INTRO_IGSSM}

Consider the Gaussian process $Y(x)$, defined on $[0, 1[$ with constant volatility $\frac{1}{n}$ with $n$ in $\mathds{N}^{\star}$ and mean process $f \star g$ where $f$ and $g$ are functions from $[0, 1[$ to $\mathds{R}$.
In short, we have $dY(x) = (f \star g)(x) dx + \frac{1}{n} dW(x)$ where $W$ is the Brownian motion.
We want to estimate $f$ while observing a realisation of $Y$.
We assume that $g$ is known.



We denote $\theta$ and $\lambda$ respectively the Fourier transforms of $f$ and $g$ respectively.

The likelihood with respect to the standard Brownian motion, noted $\P^{\circ}$, for this model can be written as follows (see \ncite{liptser2013statistics})
\[\frac{d \P_{Y^{n} \vert f, g}^{n}}{d \P^{\circ}} \propto \exp\left[\int_{[0, 1[} \frac{1}{\sqrt{n}} (f \star g)(x) dW(x) - \frac{1}{2} \left\Vert \frac{f \star g}{\sqrt{n}} \right\Vert^{2}\right].\]

We use the fact that the volatility of the process is constant and the properties of the Fourier transform to show that there exist a sequence of independent random variables with standard normal distribution such that the likelihood of the Fourier transform of the process is given by:
\[\frac{d\P^{n}_{Y^{n} \vert (\theta, \lambda)}}{d \P^{\circ}} \propto \exp\left[ -\frac{1}{2}\sum\limits_{j \in \mathds{Z}} \frac{\left(\theta_{j} \lambda_{j} - \xi_{j}\right)^{2}}{\sqrt{n}}\right].\]
Therefore, the Fourier transform of the observed process follows a Gaussian process indexed by $\mathds{Z}$, with mean $\theta \cdot \lambda$ and variance $\frac{1}{n}$.

Note that if the volatility was not constant, we would obtain
\[\frac{d\P^{n}_{Y^{n} \vert (\theta, \lambda)}}{d \P^{\circ}} \propto \exp\left[ -\frac{1}{2}\sum\limits_{j \in \mathds{Z}} \left((\sigma \star (\theta \lambda))_{j} - \xi_{j}\right)^{2}\right].\]
The mean process would hence be $\sigma \star (\theta \cdot \lambda)$, which can be rewritten as an inverse problem with a non diagonal operator, more precisely a Toeplitz operator.
We do not consider this case in this thesis.

Another motivation for this model is the heat equation.
\textcolor{red}{Heat equation + oracle rate for projection estimate + minimax rate (so all notations are introduced before moving on)}

\begin{il}\label{IL_INTRO_IGSSM_KNOWN_ORACLE}
Let us now illustrate the oracle rate $\Nsuite[n]{\oRa{\xdf,\iSv}}$  in those
cases. Firstly, recall that for any error density $\edf$ and associated $\Nsuite[\Di]{\iSv[\Di]}$ in case \ref{oo:xdf:p} the oracle rate is parametric, that is
$\oRa{\xdf,\iSv}\sim\ssY^{-1}$. Thereby, in both, the case \ref{il:po}
and \ref{il:ps}, the oracle rate is parametric, i.e. setting
$\ssY_{\xdf}:=\tfrac{K\oiSv[K]}{\bias[K-1]^2(\xdf)}$, for all
$\ssY\geq \ssY_{\xdf}$ holds
$\bias[K-1]^2(\xdf)>K\oiSv[K]\ssY^{-1}$, and hence  $\oDi{\ssY}=K$ and
$\oRa{\xdf,\iSv}= K\oiSv[K]\ssY^{-1}\sim \ssY^{-1}$, where
  \begin{Liste}
  \item[\mylabel{IL_INTRO_IGSSM_KNOWN_ORACLE_PO}{\dg\bfseries{[p-o]}}] $\ssY_{\xdf}\sim  K^{2a+1}/\bias[K-1]^2(\xdf)$, 
  \item[\mylabel{IL_INTRO_IGSSM_KNOWN_ORACLE_PS}{\dg\bfseries{[p-s]}}] $\ssY_{\xdf}\sim  K^{-(1-2a)_+}\exp(K^{2a})/\bias[K-1]^2(\xdf)$. 
\end{Liste}
On the other hand side, to illustrate \ref{oo:xdf:np}, where the oracle rate is
non-parametric, more precisely,
$\lim_{\ssY\to\infty}\ssY\oRa{\xdf,\iSv}=\infty$ we consider the  cases
\begin{Liste}[]
\item[\mylabel{IL_INTRO_IGSSM_KNOWN_ORACLE_OO}{\dg\bfseries{[o-o]}}] 
$\oRa{\xdf,\iSv}\sim(\oDi{\ssY})^{-2p}\sim (\oDi{\ssY})^{2a+1}\ssY^{-1}$, and hence,
    $\oDi{\ssY}\sim \ssY^{1/(2p+2a+1)}$ and $\oRa{\xdf,\iSv}\sim\ssY^{-2p/(2p+2a+1)}$
\item[\mylabel{IL_INTRO_IGSSM_KNOWN_ORACLE_OS}{\dg\bfseries{[o-s]}}]
$\oRa{\xdf,\iSv}\sim(\oDi{\ssY})^{-2p}\sim (\oDi{\ssY})^{-(1-2a)_+}\exp((\oDi{\ssY})^{2a})\ssY^{-1}$, and hence,\\
    $\oDi{\ssY}\sim (\log\ssY - \tfrac{2p-(1-2a)_+}{2a}\log\log\ssY)^{1/(2a)}$ and $\oRa{\xdf,\iSv}\sim(\log\ssY)^{-p/a}$.
\item[\mylabel{IL_INTRO_IGSSM_KNOWN_ORACLE_SO}{\dg\bfseries{[s-o]}}] 
$\oRa{\xdf,\iSv}\sim\exp(-(\oDi{\ssY})^{2p})\sim (\oDi{\ssY})^{2a+1}\ssY^{-1}$, and hence,\\
    $\oDi{\ssY}\sim (\log\ssY - \tfrac{2a+1}{2p}\log\log\ssY)^{1/(2p)}$ and $\oRa{\xdf,\iSv}\sim(\log\ssY)^{(2a+1)/(2p)}\ssY^{-1}$.
\end{Liste}
\end{il}
\section{Circular density deconvolution}\label{INTRO_CIRCULARDECONVOLUTION}\label{bm:ak}

Let $X$ and $\epsilon$ be circular random variables (that is to say, taking values in the unit circle), and we describe their position by a measure of angle taking values in $[0,1[$; we denote $\mathds{P}_{X}$ and $\mathds{P}_{\epsilon}$ the respective distributions of these measures of angle.
Assume that $\P_{X}$ and $\P_{\epsilon}$ admit respective densities $f$ and $h$ with respect to the Lebesgue measure on $[0, 1]$, denoted $\mu$ and we denote $\mathcal{B}$ the Borel $\sigma$-algebra on $[0, 1]$.

\begin{de}{\textsc{Modular addition}\\}\label{DE_INTRO_CIRCULARDECONVOLUTION_MODADD}
From now on we denote by $\Box$ the modular addition on $[0,1[$. That is to say, for any $x$ and $y$ in $[0, 1[$, $x\Box y = x + y - \lfloor x + y \rfloor$.
\assEnd
\end{de}

We want to estimate $f$ while observing replications of the random variable $Y = X \Box \epsilon$ which distribution we denote $\P_{Y}$.
One would notice that $\mathds{P}_{Y}$ is given, for any $A$ in $\mathcal{B}$, by $\mathds{P}_{Y}(A) = (\mathds{P}_{X} \star \mathds{P}_{\epsilon})(A) = \int_{[0,1[}\int_{[0,1[} \mathds{1}_{A}(x \Box s)\d\mathds{P}_{X}(x)\d\mathds{P}_{\epsilon}(s)$.
Moreover, $\P_{Y}$ also admits a density with respect to the Lebesgue measure, denoted $g$ and for any $y$ in $[0, 1]$, it is given by $g(y) = (f \star h)(y) = \int\nolimits_{0}^{1} f(y \Box (- s)) h(s)\d\mu(s)$.
Indeed, for any $\mu$-measurable and $\mu$-almost surely bounded function $t$, we have
\begin{alignat*}{3}
&\mathds{E}\left[t(Y)\right] &&=&& \mathds{E}\left[t(X \Box \epsilon)\right] = \int_{0}^{1}\int_{0}^{1} t(x \Box s) \d\mathds{P}_{X}(x)\d\mathds{P}_{\epsilon}(s)\\
& &&=&&\int_{0}^{1}\int_{0}^{1} t(y) \d\mathds{P}_{X}(y \Box (-s))\d\mathds{P}_{\epsilon}(s) = \int_{0}^{1} t(y) \int_{0}^{1} \d\mathds{P}_{\epsilon}(s) \d\mathds{P}_{X}(y \Box (-s))\\
& &&=&&\int_{0}^{1} t(y) \int_{0}^{1}f(y \Box (- s)) h(s)\d\mu(s) \d\mu(y).
\end{alignat*}

Keeping in mind the notations introduced previously, we have $\Xi$ is the space of square integrable complex valued functions defined on $[0, 1[$, and $T$ is the operator which associated to any such function $t$ the function given by $t \star h$.
We equip $\Xi$ with the usual internal addition; outer product; and scalar product.
That is to say, for any two functions $x$ and $y$ from $[0, 1]$ to $\C$, we have $x + y : t \mapsto x(t) + y(t)$; with any $a$ in $\C$, $a \cdot x: t \mapsto a \cdot x(t)$; and finally $\langle x \vert y \rangle_{L^{2}} = \int_{[0, 1]} x(t) \overline{y(t)} \d t$, keeping in mind that for any complex number $z$, we denote by $\overline{z}$ its conjugated complex number.
We hence will use the complex trigonometric orthonormal basis and the associated Fourier transform.

\begin{nota}\label{NOTA_INTRO_CIRCULARDECONVOLUTION_FOURIERBASIS}
Let be the orthonormal complex trigonometric basis of $\Xi$, for any $s$ in $\Z$ we have:
\[e_{s} : [0, 1] \rightarrow \C, \quad t \mapsto \exp[- 2 \imath \pi s x].\]

Denoting $\mathcal{M}([0, 1[)$ the space of measures on $[0, 1]$, we define $\fourier$, the Fourier transform operator on this set:
\[\fourier : \mathcal{M}([0, 1[) \rightarrow \C(\mathds{Z}), \quad \nu \mapsto [\mu] := \fourier(\mu) = \left(s \mapsto \int\nolimits_{0}^{1} e_{s}(t) \d\nu(t)\right).\]
In particular, if $\P$ is a probability measure and $Z$ is a random variable with distribution $\P$, we have for any $s$ in $\Z$, that $[\P](s) = \E[e_{s}(Z)]$.

We use the same notation for the Fourier transform on $\Xi$:
\[\fourier : \Xi \rightarrow \C^{\mathds{Z}}, \quad x \mapsto [x] := \fourier(x) = \left(s \mapsto \int\nolimits_{0}^{1} e_{s}(t) x(t) \d t\right).\]
In particular, if $x$ is a density associated with a probability distribution $\P$, their Fourier transforms coincide.
\assEnd
\end{nota}

As a consequence, $\Theta$ will be the space of $\Z$-indexed, $\C$-valued, square summable sequences.
It is equipped with the usual operation, for any $[x]$ in $\Theta$, we have $\overline{[x]}: s \mapsto \overline{[x](s)}$; with $[y]$ in $\Theta$, we have $[x] + [y] : s \mapsto [x](s) + [y](s)$, as well as, $[x] \cdot [y] : s \mapsto [x](s) \cdot [y](s)$; in addition, with $a$ in $\C$, $a\cdot[x]: s \mapsto a \cdot [x](s)$; and finally we have $\langle [x] \vert [y] \rangle_{l^{2}} = \sum_{s\in \Z} [x](s) \overline{[y]}(s)$.

\begin{rmk}\label{RMK_INTRO_CIRCULARDECONVOLUTION_PERIOD}
It is convenient to note that for any $t_{1}$ and $t_{2}$ in $[0, 1[$ and $s$ in $\mathds{Z}$, we have $e_{s}(t_{1} \Box t_{2}) = e_{s}(t_{1})e_{s}(t_{2})$, due to the periodicity of the complex exponential function.
Hence, for two functions $x$ and $y$ of $\Xi$, we hace $\fourier(x \star y) = \fourier(x) \cdot \fourier(y)$.
\end{rmk}

Let us recall the following notations.
\begin{nota}\label{NOTA_INTRO_CIRCULARDECONVOLUTION_FOURIERTRANSFORM}
We denote $\theta^{\circ} := \fourier(f); \lambda := \fourier(h); \phi := \fourier(g) = \theta^{\circ} \cdot \lambda$.
\end{nota}


Notice that, due to the fact that $f$, $g$ and $h$ are densities associated with some probability distributions, we know that they belong to a specific subspace of $\Xi$, more precisely, the subspace of positive-valued functions with integral $1$.
It is interesting to wonder what is the image set of this subset of $\Xi$ with respect to $\fourier$.
Let's introduce the subspace of $\C^{\Z}$ of so-called positive (semi-)definitive sequences.

\begin{de}\label{DE_INTRO_CIRCULARDECONVOLUTION_SEMIDEF}
A $\C$-valued sequence $[x]$ indexed by $\mathds{Z}$ is positive (semi-)definite iff, for any natural integer $q$ and vector $\left\{s_{1}, \hdots, s_{q}\right\}$ with entries in $\Z$, the Toeplitz matrix $A=(a_{i,j})_{(i,j) \in \llbracket 1, q \rrbracket^{2}}$ with $a_{i,j}$ defined by $[x](s_{i} - s_{j})$ is positive (semi-)definite.

In particular, this requires that $[x](s) = \overline{[x]}(-s)$, $[x](0) > 0$, and for all $s$, $[x](s) \leq [x](0)$.
\assEnd
\end{de}
Then, by denoting $\mathcal{S}^{+}(\mathds{Z})$ the set of all positive definite, complex valued, sequences $[x]$ indexed by $\Z$ with $[x](0)=1$, we formulate Herglotz's representation theorem, which is a special case of Bochner's theorem.
\begin{thm}\label{THM_INTRO_CIRCULARDECONVOLUTION_HERGLOTZ}
A function $[x]$ from $\Z$ to $\C$ with $[x](0) = 1$ is semi-definite positive iff there exist $\mu$ in $\mathcal{M}([0, 1[)$ such that for all $s$ in $\mathds{Z}$, we have $[x](s) = [\mu](s)$.
\reEnd
\end{thm}
However, notice that a semi-definite positive sequence needs not to be square summable, and hence the associated measure does not always admit a density (and in particular not a square summable one) with respect to the Lebesgue measure.
Nonetheless, by Plancherel theorem, any sequence $[x]$ is square summable if and only if its inverse Fourier transform also is.

To sum up, given a probability measure $\P$ on $[0, 1]$ admitting a square integrable density $p$ with respect to the Lebesgue measure; their Fourier transforms $[\P]$ and $[p]$ have the following properties:
\begin{alignat*}{4}
& [p] && = && [\P]; \quad &&\sum_{s \in \Z}\vert[p](s)\vert^{2} < \infty \quad p \text{ is square summable};\\ 
& [p](s) && = && \overline{[p]}(-s) \quad p \text{ is real valued;} \quad && [p](0) = 1 \quad p \text{ integrates to } 1;
\end{alignat*}
and $[p]$ positive semi-definitive implies the positivity of $p$.

\medskip

We now consider the implications of \nref{AS_INTRO_DATA_KNOWN}, \nref{AS_INTRO_DATA_UNKNOWN} in this model.

Under \nref{AS_INTRO_DATA_KNOWN}, we assume that $\P_{\epsilon}$, and hence $h$ and $\lambda$, are known.
The $\Z$-indexed, $[0, 1]$-valued stochastic process $Y = (Y_{p})_{p \in \Z}$ is strictly stationary with $Y_{0} \sim \P_{Y}$ and hence, for any $s$ in $\Z$, we have $\E[e_{s}(Y_{0})] = \phi(s) = \theta^{\circ}(s) \lambda(s)$.
As we observe $Y^{n} = (Y_{p})_{p \in \llbracket 1, n \rrbracket}$, we define, for any $s$ in $\Z$, $\phi_{n}(s) = \sum_{p = 1}^{n} e_{s}(Y_{p}) / n$ and $\theta_{n}(s) = \phi_{n}(s) / \lambda(s)$.

Under \nref{AS_INTRO_DATA_UNKNOWN}, $\P_{\epsilon}$ is not known, and hence neither are $h$ and $\lambda$.
We hence consider two $\Z$-indexed, $[0, 1]$-valued stochastic processes $Y = (Y_{p})_{p \in \Z}$ and $\epsilon = (\epsilon_{p})_{p \in \Z}$ which are strictly stationary with $Y_{0} \sim \P_{Y}$ and $\epsilon_{0} \sim \P_{\epsilon}$ and hence, for any $s$ in $\Z$, we have $\E[e_{s}(Y_{0})] = \phi(s) = \theta^{\circ}(s) \lambda(s)$ and $\E[e_{s}(\epsilon_{0})] = \lambda(s)$.
We observe the sub-vectors $Y^{n} = (Y_{p})_{p \in \llbracket 1, n \rrbracket}$ and $\epsilon^{n_{\lambda}} = (\epsilon_{p})_{p \in \llbracket 1, n_{\lambda} \rrbracket}$ and we define $\phi_{n}(s) = n^{-1} \sum_{p = 1}^{n} e_{s}(Y_{p})$, $\lambda_{n_{\lambda}}(s) = n_{\lambda}^{-1}\sum_{p = 1}^{n_{\lambda}} e_{s}(\epsilon_{p})$, $\lambda_{n_{\lambda}}^{+}(s) = \mathds{1}_{\{\vert \lambda_{n_{\lambda}}(s) \vert^{2} > (n_{\lambda})^{-1}\}} \lambda_{n_{\lambda}}(s)^{-1}$ and $\theta_{n, n_{\lambda}}(s) = \phi_{n}(s) \lambda_{n_{\lambda}}(s)^{+}$.

We now separate the study of the convergence rates of projection estimators and of minimax rates depending on the assumptions.
In this perspective it is useful to remind the following notations.
\begin{nota}
For any $s$ in $\Z$ we define, $\Lambda(s) = \vert \lambda(s) \vert^{-2}$, we obviously have $\Lambda(s) = \Lambda(-s)$ and $\Lambda(0) = 1$.
In addition, for any $m$ in $\N$, we defined $\Lambda_{+}(m) = \max_{\vert s \vert \leq m} \{\Lambda(s)\}$, $\Lambda_{\circ}(m) = m^{-1} \sum_{0 < s \leq m} \Lambda(s)$, and $\b_{m}^{2}(\theta^{\circ}) = \sum_{\vert s \vert < m} \vert \theta^{\circ} \vert^{2}$.
\end{nota}

Before moving on to the concrete study of the convergence rates for this model, let us illustrate in \ref{fig:myfig} the impact of the noise density on the observation density.
We see that a faster decay of the Fourier coefficients (top left of each panel) translate to a smoother density for the noise (top right panel) and how it influences the observation density (bottom right panel), while the density of interest (bottom left panel) remains unchanged.
It is obvious that the convolution operator as a neutral element (the Dirac distribution $\delta_{0}$), which corresponds to the direct problem case, and an absorbing element (the uniform distribution) where the problem cannot be solved.

The practical implications of this phenomenon can be seen when comparing \nref{circ:proj:direct} and \nref{circ:proj:expo}.
In \nref{circ:proj:direct} we can compare the projection estimator with threshold values $1$, $8$, $16$, and $24$ to the true parameter while observing a sample from the direct problem (the noise density is the Dirac distribution).
We can see there that while the estimate with threshold parameter $8$ is the closest to the truth, the degradation with values $16$ and $24$ does not seem too bad.
In \nref{circ:proj:expo}, the same objects are represented, however, the sample, which as the same size as in \nref{circ:proj:direct}, is from the inverse problem where the noise density is super-smooth.
We see that, the estimation with threshold $8$ is not as good as in the direct case but also the degradation when the parameter value is larger is way worth.

\begin{figure}
  \centering
  \begin{tabular}{@{}c@{}}
    \includegraphics[width=.4\linewidth]{density/illu-deconv2/950.png} \\[\abovecaptionskip]
  \end{tabular}
  \begin{tabular}{@{}c@{}}
    \includegraphics[width=.4\linewidth]{density/illu-deconv2/965.png} \\[\abovecaptionskip]
  \end{tabular}

  \vspace{\floatsep}

  \begin{tabular}{@{}c@{}}
    \includegraphics[width=.4\linewidth]{density/illu-deconv2/980.png} \\[\abovecaptionskip]
  \end{tabular}
  \begin{tabular}{@{}c@{}}
    \includegraphics[width=.4\linewidth]{density/illu-deconv2/997.png} \\[\abovecaptionskip]
  \end{tabular}

  \caption{Influence of the noise density smoothness on the observation density}\label{fig:myfig}
\end{figure}

\begin{figure}
  \centering
  \begin{tabular}{@{}c@{}}
    \includegraphics[width=.4\linewidth]{density/anim-proj/1.png} \\[\abovecaptionskip]
  \end{tabular}
  \begin{tabular}{@{}c@{}}
    \includegraphics[width=.4\linewidth]{density/anim-proj/8.png} \\[\abovecaptionskip]
  \end{tabular}

  \vspace{\floatsep}

  \begin{tabular}{@{}c@{}}
    \includegraphics[width=.4\linewidth]{density/anim-proj/16.png} \\[\abovecaptionskip]
  \end{tabular}
  \begin{tabular}{@{}c@{}}
    \includegraphics[width=.4\linewidth]{density/anim-proj/24.png} \\[\abovecaptionskip]
  \end{tabular}

  \caption{Influence of the threshold parameter choice on the estimation in the direct problem case}\label{circ:proj:direct}
\end{figure}

\begin{figure}
  \centering
  \begin{tabular}{@{}c@{}}
    \includegraphics[width=.4\linewidth]{density/anim-proj-expo/1.png} \\[\abovecaptionskip]
  \end{tabular}
  \begin{tabular}{@{}c@{}}
    \includegraphics[width=.4\linewidth]{density/anim-proj-expo/8.png} \\[\abovecaptionskip]
  \end{tabular}

  \vspace{\floatsep}

  \begin{tabular}{@{}c@{}}
    \includegraphics[width=.4\linewidth]{density/anim-proj-expo/16.png} \\[\abovecaptionskip]
  \end{tabular}
  \begin{tabular}{@{}c@{}}
    \includegraphics[width=.4\linewidth]{density/anim-proj-expo/24.png} \\[\abovecaptionskip]
  \end{tabular}

  \caption{Influence of the threshold parameter choice on the estimation in the severely ill-posed problem case}\label{circ:proj:expo}
\end{figure}

%\begin{figure}
%\centering
%\begin{subfigure}{.4\textwidth}
%  \centering
%  \includegraphics[width=1.\linewidth]{density/illu-deconv2/800.png}
%  \caption{$\theta^{\circ}$ polynomial and $\lambda$ polynomial}
%  \label{fig3:sub1}
%\end{subfigure}%
%\begin{subfigure}{.4\textwidth}
%  \centering
%  \includegraphics[width=1.\linewidth]{density/illu-deconv2/850.png}
%  \caption{$\theta^{\circ}$ polynomial and $\lambda$ exponential}
%  \label{fig3:sub2}
%\end{subfigure}
%\begin{subfigure}{.4\textwidth}
%  \centering
%  \includegraphics[width=1.\linewidth]{density/illu-deconv2/900.png}
%  \caption{$\theta^{\circ}$ polynomial and $\lambda$ exponential}
%  \label{fig3:sub3}
%\end{subfigure}
%\begin{subfigure}{.4\textwidth}
%  \centering
%  \includegraphics[width=1.\linewidth]{density/illu-deconv2/950.png}
%  \caption{$\theta^{\circ}$ polynomial and $\lambda$ exponential}
%  \label{fig3:sub4}
%\end{subfigure}
%\begin{subfigure}{.4\textwidth}
%  \centering
%  \includegraphics[width=1.\linewidth]{density/illu-deconv2/997.png}
%  \caption{$\theta^{\circ}$ polynomial and $\lambda$ exponential}
%  \label{fig3:sub5}
%\end{subfigure}
%\caption{Estimated median of the quadratic error of the estimator given by the posterior mean for different classes of $\theta^{\circ}$ and $\lambda$ with $\theta^{\times} \equiv 0$ and $s \equiv 1$.}
%\label{EQM}
%\end{figure}

\subsection{Known noise density, independent observations process}
% ....................................................................
% Oracle optimality known error density
% ....................................................................
\begin{te}
We place ourselves under \nref{AS_INTRO_DATA_KNOWN} and \nref{AS_INTRO_DATA_INDEPENDENT}.
We hence observe an \iid $\ssY$-sample $\rY_1,\dotsc,\rY_{\ssY}$ from $g = f \star h$.
Given an estimator $\hxdf[]$ of $\xdf\in\lp[2]$ based on the observations we measure its accuracy by a quadratic risk, that is, $\E\Vnormlp{\hxdf[]-\xdf}^2$.
Keep in mind that throughout the thesis we assume that $ \vert \fedf[(s)] \vert >0$ holds for all $s\in\Zz$.
Considering $\Lambda=\Nsuite{\iSv[s]}$ with $\iSv[s]:= \vert \fedf[(s)] \vert ^{-2}$ for $s\in\Nz$, we set $\miSv=\max\set{\iSv[s],s\in\nset{1,\Di}}$ and $\oiSv=\tfrac{1}{\Di}\sum_{s=1}^{\Di}\iSv[s]$.

Notice that due to $\E[\Vert \theta_{n} - \theta^{\circ} \Vert_{l^{2}}^{2}] + n^{-1} \Vert \theta^{\circ}_{\underline{0}} \Vert_{l_{2}}^{2} = n^{-1} \sum_{-m \leq s \leq m} \Lambda(s) + \Vert \theta^{\circ}_{\underline{0}} \Vert_{l^{2}}^{2}(1 + n^{-1})\b_{m}^{2}(\theta^{\circ})$
together with, for any $s$ in $\Z$, $\Lambda(s) = \Lambda(-s)$, $\vert \phi(0) \vert = 1$ and $\vert \phi(s) \vert < 1$ for $s \neq 0$ we indeed have
\[\E[\Vert \theta_{n} - \theta^{\circ} \Vert_{l^{2}}^{2}] + n^{-1} \Vert \theta^{\circ}_{\underline{0}} \Vert_{l_{2}}^{2} \leq 2 n^{-1} m \Lambda_{\circ}(s) + \Vert \theta^{\circ}_{\underline{0}} \Vert_{l^{2}}^{2}(1 + n^{-1})\b_{m}^{2}(\theta^{\circ}).\]

Finally, since
$\Zsuite[s]{\fedf[(s)]}$ is a $\lp^2$ sequence having only non-zero
components bounded by one, i.e., $0< \vert \fedf[(s)] \vert \leq1$, for all
$s\in\Zz$, it follows $\lim_{\Di\to\infty}\miSv=\infty$ and for any
diverging sequence $\Nsuite[\ssY]{\Di_{\ssY}}$ of positive integers,
i.e., $\lim_{\ssY\to\infty}\Di_{\ssY}=\infty:\Leftrightarrow\forall
K>0:\exists n_o\in\Nz:\forall n\geq n_o:\Di_n\geq K$, holds $\lim_{\Di\to\infty}\Di_{\ssY}\oiSv[\Di_{\ssY}]=\infty$.
Notice that, as in \nref{as:il}, we have $\V[e_{s}(Y)] = \E\vert e_{s}(Y) \vert^{2} - \vert \E[e_{s}(Y)] \vert^{2} = 1 - \vert \phi(s) \vert^{2}$ which is bounded from above by $1$.
The analysis we carried out previously hence is still valid and we remind the following definitions.
\end{te}

\subsubsection{Quadratic risk bounds}
The bound we derived in \nref{rates} depends on the
dimension parameter $\Di$ and hence by selecting an optimal value they
will be minimised, which we formulate next.  For a sequence
$\Nsuite[n]{a_n}$ of real numbers with minimal value in a set
$A\subset{\Nz}$ we set
$\argmin\set{a_n,n\in A}:=\min\{m\in A:a_m\leq a_n,\;\forall n\in A
\}$. For all $n\in\Nz$ we define
\begin{multline*}
  \dRa{\Di}{\xdf,\Lambda}:=[\bias^2(\xdf)\vee\Di \oiSv \ssY^{-1}]
  :=\max\vectB{\bias^2(\xdf), \Di \oiSv \ssY^{-1}},\\
  \hfill
  \onDi:=\onDi(\xdf,\Lambda):=\argmin\Nset[{\Di\in\Nz}]{\dRa{\Di}{\xdf,\Lambda}}
  \quad\text{ and }\hfill\\
  \oRa{\xdf,\Lambda}:=\dRa{\onDi}{\xdf,\Lambda}=\min\Nset[{\Di\in\Nz}]{\dRa{\Di}{\xdf,\Lambda}}.
\end{multline*}

\begin{te}
Consequently, the  rate $\Nsuite[\ssY]{\oRa{\xdf,\Lambda}}$, the dimension parameters $\Nsuite[\ssY]{\onDi}$  and  the projection estimators  $\Nsuite[\ssY]{\txdfPr[\onDi]}$, respectively, is an oracle
rate, an oracle dimension and oracle optimal (up to a constant).
\end{te}

\begin{rmk}
We shall emphasise that $\oRa{\xdf,\Lambda}\geq \ssY^{-1}$ for all
  $\ssY\in\Nz$, and
  
  $\lim_{n \rightarrow \infty} \oRa{\xdf,\Lambda}=0$.
  Observe that for all $\delta>0$ there exists $\Di_{\delta}\in\Nz$ and
  $\ssY_\delta\in\Nz$ such that for all $\ssY\geq \ssY_{\delta}$ holds
  $\bias[\Di_\delta]^2(\xdf)\leq \delta$ and
  $\Di_{\delta} \oiSv[\Di_\delta] \ssY^{-1}\leq\delta$, and whence
  $\oRa{\xdf,\Lambda}\leq\dRa{\Di_\delta}{\xdf,\Lambda}\leq \delta$.
  Moreover, we have $\oDi{\ssY}\in\nset{1,\ssY}$. Indeed, by construction
  holds
  $\bias[\ssY]^2(\xdf)\leq 1<(\ssY+1)\ssY^{-1}\leq
  (\ssY+1)\oiSv[\ssY+1]{}\ssY^{-1}$, and hence
  $\dRa{\ssY}{\xdf,\Lambda}<\dRa{\Di}{\xdf,\Lambda}$ for all
  $\Di\in \nsetro{\ssY+1,\infty}$ which in turn implies the claim
  $\oDi{\ssY}\in\nset{1,\ssY}$. Obviously, it follows thus
  $\oRa{\xdf,\Lambda}=\min\set{ \dRa{\Di}{\xdf,\Lambda}
    ,\Di\in\nset{1,\ssY}}$ for all $\ssY\in\Nz$. We shall use those
  elementary findings in the sequel without further reference.
The sequence $\mathcal{R}_{n}^{\circ}(\theta, \lambda)$ is then an exact oracle convergence rate and the projection estimator $\theta_{n, \overline{m_{n}^{\circ}}}$ is an oracle optimal estimator.
\remEnd
\end{rmk}

\begin{rmk}
In case \ref{oo:xdf:p}, the oracle rate is parametric, that is
$\oRa{\xdf, \Lambda} \approx \ssY^{-1}$. More precisely, if $\xdf=0$ then
for each  $\Di\in\Nz$,
$\E\Vnormlp{\txdfPr-\xdf}^2=2\Di\oiSv[\Di]\ssY^{-1}$,
and hence $\oDi{\ssY}=1$ and $\oRa{\xdf, \Lambda}=2\oiSv[1]\ssY^{-1}\sim\ssY^{-1}$. Otherwise
if there is $K\in\Nz$  with $\bias[K-1](\xdf)>0$ and
$\bias[K](\xdf)=0$, then setting
$\ssY_{\xdf}:=\tfrac{K\oiSv[K]}{\bias[K-1]^2(\xdf)}$, for all
$\ssY\geq \ssY_{\xdf}$ holds
$\bias[K-1]^2(\xdf)>K\oiSv[K]\ssY^{-1}$, and hence  $\oDi{\ssY}=K$ and
$\oRa{\xdf,\Lambda}= K\oiSv[K]\ssY^{-1}\sim \ssY^{-1}$.
On the other hand side, in case \ref{oo:xdf:np} the oracle rate is
non-parametric, more precisely, it holds
$\lim_{\ssY\to\infty}\ssY\oRa{\xdf,\Lambda}=\infty$. Indeed, since
$\bias[\oDi{\ssY}]^2(\xdf)\leq\oRa{\xdf,\Lambda}=\oRa{\xdf,\Lambda}\in\mathfrak{o}_{n}(1)$ follows $\oDi{\ssY}\to\infty$ and hence
$\oDi{\ssY}\oiSv[\oDi{\ssY}]\to\infty$ which implies the claim because
$\ssY\oRa{\xdf,\Lambda}\geq\oDi{\ssY}\oiSv[\oDi{\ssY}]$.
\end{rmk}

\subsubsection{Maximal risk bounds}
We may emphasise that for all $\Di\in\Nz^{\star}$ and  any $\xdf\in\rwCxdf$, $\Vnormlp{\xdf_{\underline{0}}}^2\bias^2(\xdf)=\Vnormlp{\xdf_{\underline{\Di}}}^2=\sum_{ \vert s \vert >\Di}(\xdfCw[(s)]^2/\xdfCw[(s)]^2)\fxdf[(s)]^2\leq
\xdfCw[(\Di)]^2\Vnorm[1/{\xdfCw[]}]{\xdf_{\underline{\Di}}}^2\leq
\xdfCw[(\Di)]^2\xdfCr^2$ which we use in the sequel
without further reference.
It follows for all $\Di,\ssY\in\Nz$ that 
  \begin{multline}
\nRi{\txdfPr}{\rwCxdf,\Lambda}:=\sup\set{\nRi{\txdfPr}{\xdf,\Lambda},\xdf\in\rwCxdf}\\
\leq (2+\xdfCr^2)\max\big(\xdfCw^2,\Di \oiSv \ssY^{-1}\big).
\end{multline}
The upper bound in the last display depends on the dimension parameter
$\Di$ and hence by choosing an optimal value $\mnDi$ the upper bound
will be minimised which we formulate next. For all $n\in\Nz$ we define
\begin{multline}
 \dRa{\Di}{\xdfCw[],\Lambda}:=[\xdfCw^2\vee\Di\oiSv \ssY^{-1}]:=\max\big(\xdfCw^2,\Di \oiSv \ssY^{-1}\big),\\
\hfill \mnDi(\xdfCw[]):=\mnDi(\xdfCw[],\Lambda):=\argmin\Nset[{\Di\in\Nz}]{\dRa{\Di}{\xdfCw[],\Lambda}}\quad\text{ and }\hfill\\\mnRa{\xdfCw[],\Lambda}:=\dRa{\mnDi({\xdfCw[]})}{\xdfCw[],\Lambda}=\min\Nset[{\Di\in\Nz}]{\dRa{\Di}{\xdfCw[],\Lambda}}.
\end{multline}
From \eqref{oo:e4} we deduce that
$\nRi{\txdfPr[\mnDi({\xdfCw[]})]}{\rwCxdf,\Lambda}\leq(2+\xdfCr^2)\mnRa{\xdfCw[],\Lambda}$ for
all $n\in\Nz$. On the other
  hand side, for example, \ncite{JohannesSchwarz2013a} have shown  that
  $\inf_{\widetilde{\theta}}\nRi{\widetilde{\theta}}{\rwCxdf,\Lambda}$, where the infimum is taken over all
  possible estimators $\widetilde{\theta}$ of $\xdf$, is up to a constant bounded
  from below by $\mnRa{\xdfCw[],\Lambda}$.  Consequently, the rate
  $\Nsuite[n]{\mnRa{\xdfCw[],\Lambda}}$, the dimension parameters $\Nsuite[n]{\mnDi(\xdfCw[])}$
  and the projection estimators $\Nsuite[n]{\txdfPr[\mnDi({\xdfCw[]})]}$, respectively, is a
  minimax rate, a minimax dimension and minimax optimal (up to a
  constant).

% ....................................................................
% Bemerkung mm
% ....................................................................
\begin{rmk}
By construction it holds 
$\mnRa{\xdfCw[],\Lambda}\geq \ssY^{-1}$ for all $\ssY\in\Nz$.
The following statements can be
shown using the same arguments as in \nref{oo:rem:ora}
by exploiting that the sequence $\xdfCw[]$ is assumed to be
non-increasing, strictly positive with limit zero and $\xdfCw[(1)]=1$. 
Thereby, we conclude that 
$\mnRa{\xdfCw[],\Lambda}=\mathfrak{o}_{n}(1)$ and $\ssY\mnRa{\xdfCw[],\Lambda}\to\infty$ as well as 
$\mnDi(\xdfCw[])\in\nset{1,\ssY}$ for all $\ssY\in\Nz$. It follows also that
$\mnDi(\xdfCw[])=\argmin\Nset[{\Di\in\nset{1,n}}]{\dRa{\Di}{\xdfCw[],\Lambda}}$ and 
$\mnRa{\xdfCw[],\Lambda}=\min\Nset[{\Di\in\nset{1,n}}]{\dRa{\Di}{\xdfCw[],\Lambda}}$ for all
$\ssY\in\Nz$. We shall stress that in this situation the rate
$\mnRa{\xdfCw[],\Lambda}$ is non-parametric. \remEnd
\end{rmk}

\subsection{Unknown noise density, independent observations process}
\begin{te}
We place ourselves under \nref{AS_INTRO_DATA_UNKNOWN} and \nref{AS_INTRO_DATA_INDEPENDENT}.
We hence observe independent \iid $\ssY$-sample
  $\rY_1,\dotsc,\rY_{\ssY}$ from $g$ and \iid $\ssE$-sample
  $\rE_1,\dotsc,\rE_{\ssE}$ from $h$.
  Note that we define the projection estimators in the following way
$\hxdfPr:=\mathds{1}_{\{\vert s \vert \leq m\}}\hfedfmpI[(s)]\hfydf[(s)]$ with
$\hfedfmpI[(s)]:=\hfedfI[(s)]\Ind{\{ \vert \hfedf[(s)] \vert ^2\geq1/\ssE\}}$.
\end{te}
% ....................................................................
% <<Re Sv Moore Penrose Inverse>> 
% ....................................................................
Note that the following result is given in Theorem 2.10 of \ncite{Petrov1995}.

\begin{lm}\label{circ:oSv:re}
There is a finite numerical constant $\cst{4}>0$ such that
for all $s\in\Zz$ hold
\begin{equation}\label{circ:oSv:re:i}
\ssE^2\FuEx\Vabs{\fedf[(s)]-\hfedf[(s)]}^4\leq\cst{4}.
\end{equation}
\end{lm}

Hence, \nref{as:il} is also valid in this model and so is the analysis we carried out later.
We hence remind here the following definitions.
% ....................................................................
% Upper bound hat
% ....................................................................
\subsubsection{Quadratic risk bounds}
Let us remind that we have
\begin{equation*}
\mathcal{R}_{n, n_{\lambda}}(\theta_{n, n_{\lambda}, \overline{m_{n}^{\circ}}}) \leq (V_{2} \cst{} + \Vert \theta^{\circ}_{\underline{0}} \Vert_{l^{2}}^{2})\mathcal{R}_{n}^{\circ}(\theta^{\circ}, \Lambda) + 2 \cst{} \mathcal{R}_{n_{\lambda}}^{\dagger}(\theta^{\circ}, \Lambda)
\end{equation*}
We note that $\Vnormlp{\xdf_{\underline{0}}}^2=0$ implies
  $\mRa{\xdf,\Lambda}=0$, while for $\Vnormlp{\xdf_{\underline{0}}}^2>0$ holds
  $\mRa{\xdf,\Lambda}\geq \sum_{s:\iSv[s]>\ssE} \vert \fxdf[(s)] \vert ^2+\ssE^{-1}\sum_{s:\iSv[s]\leq\ssE} \vert \fxdf[(s)] \vert ^2  \geq\ssE^{-1}\sum_{s\in\Nz} \vert \fxdf[(s)] \vert ^2=\cst{}\Vnormlp{\xdf_{\underline{0}}}^2
  \ssE^{-1}$, thereby whenever $\xdf\ne 0$
  any additional term of order $\ssY^{-1}+\ssE^{-1}$
  is negligible with respect to the rate
  $\oRa{\xdf,\Lambda}+\mRa{\xdf,\Lambda}$, since
  $\oRa{\xdf,\Lambda}\geq \ssY^{-1}$, 
  which we will use below without further reference. We shall
  emphasise that in case $\ssY=\ssE$ it holds
  \begin{multline}
    \mRa[\ssY]{\xdf,\Lambda}=\sum_{s\in \mathds{F}_{m_{n}^{\circ}}} \vert \fxdf[(s)] \vert^2 [1 \wedge \ssY^{-1}\iSv[s]]
    + \sum_{s\in \mathds{F}_{m_{n}^{\circ}}^{c}} \vert \fxdf[(s)] \vert ^2[1\wedge\ssY^{-1}\iSv[s]]\\
    \leq \cst{}\Vnormlp{\xdf_{\underline{0}}}^2 \ssY^{-1} \oDi{\ssY}
    \oiSv[\oDi{\ssY}] +
    \cst{}\Vnormlp{\xdf_{\underline{0}}}^2\bias[\oDi{\ssY}]^2(\theta^{\circ})\leq
    \Vnormlp{\xdf_{\underline{0}}}^2\dRa{\oDi{\ssY}}{\xdf,\Lambda}
  \end{multline}
  which in turn implies $\mathcal{R}_{n, n_{\lambda}}(\theta_{n, n_{\lambda}, \overline{m_{n}^{\circ}}}) \leq (V_{2} \cst{} + (1 + 2 \cst{})\Vert \theta^{\circ}_{\underline{0}} \Vert_{l^{2}}^{2})\mathcal{R}_{n}^{\circ}(\theta^{\circ}, \Lambda)$.
  In other words, the estimation of the unknown operator $T$ is negligible whenever $\ssY\leq\ssE$.

\begin{rmk}
We note that in case \ref{oo:xdf:p}
$\mRa{\xdf,\Lambda}\leq
\Vnormlp{\xdf_{\underline{0}}}^2\miSv[K]\ssE^{-1}$
and hence
\begin{equation}
 \nmRi{\hxdfPr[\oDi{\ssY}]}{\xdf,\Lambda}
 % \FuEx\Vnormlp{\hxdfPr[\oDi{\ssY}]-\xdf}^2
 \leq
\cst{}\{[1\vee\Vnormlp{\xdf_{\underline{0}}}^2]\{
K\oiSv[K]\ssY^{-1}+\miSv[K]\ssE^{-1}\}
\end{equation}
for all $\ssE\in\Nz$ and $\ssY\geq\ssY_{\xdf}$ with $\ssY_{\xdf}$ as in \nref{oo:rem:ora}. In other words the
rate is parametric in both the $\rE$-sample size $\ssE$ and the $\rY$-sample size $\ssY$. Thereby, the  additional estimation of the error
density is negligible whenever $\ssE\geq\ssY$.  In the
opposite case \ref{oo:xdf:np}, it is obviously of interest to characterise the minimal size $\ssE$ of the additional
sample from $\rE$ needed to attain the same rate as in case of a known
error density. Thus, in the next illustration we let the $\rE$-sample size 
depend on the $\rY$-sample size $\ssY$ as well. 
\remEnd
\end{rmk}

\subsubsection{Maximal risk bounds}
In the minimax case, for all $\ssE\in\Nz$ we define
\begin{equation}
  \mmRa{\xdfCw[],\Lambda}:=\max_{s\in\Nz}\{\xdfCw[(s)]^2[1\wedge \iSv[s]/\ssE]\}\Vnorm[1/{\xdfCw[]}]{\xdf}^2.
\end{equation}
then for all $\ssE\in\Nz$ holds
$\sup_{\xdf\in\rwCxdf}\mmRa{\xdf,\Lambda}\leq
\xdfCr^2\mmRa{\xdfCw[],\Lambda}$, since for all
$\xdf\in\rwCxdf$ 
\begin{equation}
  \mmRa{\xdf,\Lambda}=\sum_{s\in\Nz} \vert \fxdf[(s)] \vert ^2[1\wedge \iSv[s]/\ssE]\leq
\max_{s\in\Nz}\{\xdfCw[(s)]^2\min(1,\iSv[s]/\ssE)\}\Vnorm[1/{\xdfCw[]}]{\xdf}^2.
\end{equation}
It follows for all $\Di,\ssY,\ssE\in\Nz$ immediately that 
\begin{equation}
  \nmRi{\hxdfPr}{\rwCxdf,\Lambda}
  \leq (\xdfCr^2+8) \dRa{\Di}{\xdfCw[],\Lambda}+8(\cst{4}+1)\xdfCr^2\mmRa{\xdfCw[],\Lambda}.
\end{equation}
The upper bound in the last display depends on the dimension parameter
$\Di$ and hence by choosing an optimal value $\mnDi$ as in
\eqref{mm:de:nra} the upper bound
will be minimised, that is
\begin{equation}
  \nmRi{\hxdfPr[\mnDi]}{\rwCxdf,\Lambda}
  \leq (\xdfCr^2+8) \mnRa{\xdfCw[],\Lambda}+8(\cst{4}+1)\xdfCr^2\mmRa{\xdfCw[],\Lambda}.
\end{equation}

\begin{rmk} Since the operator $T$ is not known, it is natural to
  consider a maximal risk also over a class for $\edf$ characterising the behaviour of
  $\Nsuite[s]{\iSv[s]= \vert \fedf[(s)] \vert ^{-2}}$, precisely $\rwCedf:=\{\edf\in\lp[2]:\edfCr^{-2}\leq\edfCw[s] \vert \fedf[ \vert s \vert ] \vert ^{2}=
\edfCw[s]/\iSv[ \vert s \vert ]\leq \edfCr^{2},\;\forall s\in\Nz\}\cap\cD$.
We shall note that for all $\Di\in\Nz$ and any $\edf\in\rwCedf$,
$\edfCr^{-2}\leq\miSv/\edfCwm\leq \edfCr^{2}$,
$\edfCr^{-2}\leq\oiSv/\edfCwo\leq \edfCr^{2}$. Setting
for all $\ssY,\ssE\in\Nz$
\begin{multline}
 \dRa{\Di}{\xdfCw[],\edfCw[]}:=[\xdfCw^2\vee\Di\edfCwo \ssY^{-1}],
\hfill
\mnDi(\xdfCw[],\edfCw[]):=\argmin\Nset[{\Di\in\Nz}]{\dRa{\Di}{\xdfCw[],\edfCw[]}},\hfill\\\mnRa{\xdfCw[],\edfCw[]}:=\dRa{\mnDi({\xdfCw[],\edfCw[]})}{\xdfCw[],\edfCw[]}=\min\Nset[{\Di\in\Nz}]{\dRa{\Di}{\xdfCw[],\edfCw[]}}\quad\text{
  and }\\
\mnRa{\xdfCw[],\edfCw[]}:=\max\{\xdfCw[(s)]\min(1,\edfCw[s]/\ssE),s\in\Nz\}.
\end{multline}
we have 
\begin{multline}
 \mnRa{\xdfCw[],\Lambda}=\min_{\Di\in\Nz}\{[\xdfCw\vee\Di\oiSv\ssY^{-1}]\}\leq
 \edfCr^2\min_{\Di\in\Nz}\{[\xdfCw\vee\Di\edfCwo\ssY^{-1}]\}\leq \edfCr^2\dRa{\Di}{\xdfCw[],\edfCw[]}\\ 
  \mmRa{\xdfCw[],\Lambda}=\max_{s\in\Nz}\{\xdfCw[(s)]^2[1\wedge \iSv[s]/\ssE]\}\leq\edfCr^2\mnRa{\xdfCw[],\edfCw[]}.
\end{multline}
It follows for all $\Di,\ssY\in\Nz$ immediately that 
\begin{equation}
  \nmRi{\hxdfPr}{\rwCxdf,\rwCedf}
  \leq (\xdfCr^2+8\edfCr^2) \mnRa{\xdfCw[],\edfCw[]}
+8(\cst{4}+1)\edfCr^2\xdfCr^2\mmRa{\xdfCw[],\edfCw[]}.
\end{equation}
\ncite{JohannesSchwarz2013a} have shown  that
  $\inf_{\hxdf}\nmRi{\hxdf}{\rwCxdf,\rwCedf}$, where the infimum is taken over all
  possible estimators $\hxdf$ of $\xdf$, is up to a constant bounded
  from below by $\mnRa{\xdfCw[],\edfCw[]}\vee\mmRa{\xdfCw[],\edfCw[]} $.  Consequently, the rate
  $\Nsuite[n]{\mnRa{\xdfCw[],\edfCw[]}\vee\mmRa{\xdfCw[],\edfCw[]}}$, the dimension parameters $\Nsuite[n]{\mnDi(\xdfCw[])}$
  and the projection estimators $\Nsuite[n]{\txdfPr[\mnDi({\xdfCw[]})]}$, respectively, is a
  minimax rate, a minimax dimension and minimax optimal (up to a
  constant).
\remEnd
\end{rmk}


%%% Local Variables:
%%% mode: latex
%%% TeX-master: "_0DACD"
%%% End:

\chapter{Bayesian interpretation of penalised contrast model selection}\label{BAYES}

In this chapter, we consider the family of Bayesian methods described as "Gaussian sieve priors" in \nref{INTRO_BAYES_PRIOR} as well as an adaptive variant of these priors, the hierarchical sieve priors where the threshold parameter is a random variable with a specified prior distribution.
We study their behaviour under two asymptotic, respectively described in \nref{INTRO_BAYES_PRAGMATIC} and \nref{INTRO_BAYES_ITERATIVE}.

In \nref{BAYES_SIEVE} we consider the self informative Bayes carrier of Gaussian sieve priors under continuity assumptions for the likelihood and show that its support is contained in the maximum likelihood set.
Then, in \nref{BAYES_HIERARCHICAL} we show that the distribution of the hyper-parameter in the hierarchical prior contracts around the set of maximisers of a penalised contrast criterion.
This section highlights a new link between Bayesian adaptive estimation and the frequentist penalised contrast model selection.

\medskip

In \nref{BAYES_STRATEGIES}, while considering the noise asymptotic, we line out two strategies of proof which allow to obtain contraction rates. The first relies on posterior moment bounding and which, up to our knowledge, is new; the second is specific to the hierarchical sieve prior and is similar to the one used in \ncite{JJASRS}.
In \nref{BAYES_GAUSS} we apply this strategies to the specific inverse Gaussian sequence space model.
Doing so, we obtain exact contraction rate for the (iterated) Gaussian sieve prior using the first scheme of proof; and the iterated hierarchical prior using the second.
This yields optimality for sieve priors with properly chosen threshold parameter; as well as for penalised contrast model selection; and for any iterated version of the hierarchical prior we consider.
The most interesting point of this subsection is the novel way to show optimality of the penalised contrast model selection.

 \medskip
%
% In \nref{BAYES_CIRCULARDECON} we inquire the use of the discussed method to the circular deconvolution model and show that a direct use of those methods is not possible in this context.
% We give nonetheless some tracks for a fix.

\medskip

Finally, we conclude this chapter in \nref{BAYES_POSTMEAN} with a note about the shape of the posterior mean of the hierarchical prior, motivating the shape of the frequentist estimators we use in \nref{FREQ}.

\section{Iterated Gaussian sieve prior}\label{2.1}

We consider in this part a statistical model with a functional parameter space as described in \nref{1.1.1}.
We adopt a sieve prior as described in \nref{1.3.2} and first give interest to the asymptotic presented in \nref{1.3.5}.

\medskip

We first remind the following notations.
The parameter space $\Theta$ is a function space $\Theta = \{ \theta : \mathds{F} \rightarrow \mathds{I} \}$; with $\mathds{F}$ a subset of $\R$ and $\mathds{I}$ a subset of $\C$.

To derive the self informative Bayes carrier we formulate the following hypothesis.

\begin{as}{\textsc{Countability assumption}\\}\label{as2.1.1}
We assume that the set $\mathds{F}$ is countable.
\end{as}

We equip $\Theta$ with the usual $\mathds{L}^{2}$ norm that is, $\Vert \theta \Vert^{2} = \sum\limits_{j \in \mathds{J}} \vert \theta_{j} \vert^{2}$ and consider the Borel sigma algebra $\mathcal{B}$ of the topology generated by this $\mathds{L}^{2}$ norm.

On the other hand our observation $Y$ take values in the space $(\mathds{Y}, \mathcal{Y})$ with distribution in the family $(\P_{Y \vert \boldsymbol{\theta}})_{\boldsymbol{\theta} \in \Theta}$.


We assume the existence of a function $l: (\Theta, \mathcal{B}) \times (\mathds{Y}, \mathcal{Y}) \rightarrow \R$ such that the likelihood with respect to some reference measures $\P^{\circ}$ is given by:

\[L(\boldsymbol{\theta}, y) \propto \exp\left[-l(\boldsymbol{\theta}, y)\right].\]

Then, the family of Gaussian sieve priors is indexed by a threshold parameter $m$ in the set of subsets of $\mathds{J}$, denoted $\mathcal{P}(\mathds{J})$, and we denote by $\P_{\boldsymbol{\theta}^{m}}$ the element of this family with index $m$; moreover, we denote $\boldsymbol{\theta}^{m}$ a random variable following this distribution. 
There exists a reference measure $\Q^{\circ}$ such that the sieve prior with threshold parameter $m$ admits a density of the shape

\[\frac{d\P_{\boldsymbol{\theta}^{m}}}{d\Q^{\circ}}(\boldsymbol{\theta}) \propto  \exp\left[-\frac{1}{2}\sum\limits_{j \in m} \vert \boldsymbol{\theta}_{j} \vert^{2}\right] \cdot \prod\limits_{j \notin m} \delta_{0}(\boldsymbol{\theta}_{j}).\]

If we denote by $\Theta_{m}$ the set $\{\theta \in \Theta : \forall j \notin m, \theta_{j} = 0\}$, Bayes' theorem gives the following shape for the iterated posterior distribution:

\begin{alignat*}{3}
& \frac{d\P_{\boldsymbol{\theta}^{m}\vert Y}^{\eta}}{d\Q^{\circ}}(\boldsymbol{\theta}, y)&& = && \frac{\exp\left[-\left(\frac{1}{2}\sum\limits_{j \in m} \vert \boldsymbol{\theta}_{j} \vert^{2} + \eta l(\boldsymbol{\theta}, y)\right)\right] \cdot \prod\limits_{j \notin m} \delta_{0}(\boldsymbol{\theta}_{j})}{\int_{\Theta_{m}} \exp\left[-\left(\frac{1}{2}\sum\limits_{j \in m} \vert \mu_{j} \vert^{2} + \eta l(\mu, y)\right)\right] d\mu}\\
& && = && \frac{\prod\limits_{j \notin m} \delta_{0}(\boldsymbol{\theta}_{j})}{\int_{\Theta_{m}} \exp\left[-\frac{1}{2}\sum\limits_{j \in m} \left(\vert \mu_{j} \vert^{2} - \vert \boldsymbol{\theta}_{j} \vert^{2}\right)\right]\exp\left[-\eta\left(l(\mu, y) - l(\boldsymbol{\theta}, y)\right)\right] d\mu}.
\end{alignat*}

The following assumption is also needed to obtain the self informative Bayes carrier.

\begin{as}{\textsc{Continuous likelihood asumption}\\}\label{as2.1.2}
Assume that for any $m$ in $\mathcal{P}(\mathds{J})$ and $y$, $\Theta_{m} \rightarrow \R_{+}, \theta \mapsto l(\theta, y)$ is continuous.
\end{as}

The use of a threshold parameter brings us back to the study of a parametric model and the results from \textcolor{red}{ref Bunke} can be used to derive the self informative Bayes carrier.

\begin{thm}{\textsc{Self informative Bayes carrier for a sieve prior}\\}\label{thm2.1.1}
Under \nref{as2.1.1} and \nref{as2.1.2} the support of the Bayesian carrier is contained in the set of minimisers of $\theta \mapsto l(\theta, y)$.
\end{thm}

\begin{pro}{\textsc{Proof of \nref{thm2.1.1}}\\}\label{pro2.1.1}
Let's remind that the definition of continuity gives us:
\[\forall \theta \in \Theta_{m}, \forall \epsilon \in \R_{+}^{\star}, \exists \delta \in \R_{+}^{\star} : \forall \mu \in \Theta_{m}, \Vert \mu - \theta \Vert < \delta \Rightarrow \vert l(\mu, y) - l(\theta, y) \vert < \epsilon.\]

\medskip

Then, for any $B$ in $\mathcal{B}$ such that $\inf\limits_{\theta \in B} l(\theta, y) > \inf\limits_{\mu \in \Theta_{m}} l(\mu, y)$, there exist $\delta$ in $\R_{+}^{\star}$ and a ball $\mathcal{E}$ of $\Theta_{m}$ of radius $\delta$ such that, $\sup\limits_{\mu \in \mathcal{E}} l(\mu, y) < \inf\limits_{\theta \in B}l(\theta, y)$ and hence $\sup\limits_{\mu \in \mathcal{E}}l(\mu, y) - \inf\limits_{\theta \in B}l(\theta, y) < 0$.

Hence we can write
\begin{alignat*}{3}
& \P_{\boldsymbol{\theta}^{m}\vert Y}^{\eta}(B) && = && \int_{B} \frac{\prod\limits_{\vert j \vert > m} \delta_{0}(\boldsymbol{\theta}_{j})}{\int_{\Theta_{m}} \exp\left[-\frac{1}{2}\sum\limits_{\vert j \vert \leq m} \left(\vert \mu_{j} \vert^{2} - \vert \boldsymbol{\theta}_{j} \vert^{2}\right)\right]\exp\left[-\eta\left(l(\mu, y) - l(\boldsymbol{\theta}, y)\right)\right] d\mu} d \theta\\
& && \leq && \int_{B} \frac{\prod\limits_{\vert j \vert > m} \delta_{0}(\theta_{j})}{\exp\left[-\eta\left(\sup\limits_{\mu \in \mathcal{E}} l(\mu, y) - \inf\limits_{\theta \in B}l(\theta, y)\right)\right] \int_{\mathcal{E}} \exp\left[-\frac{1}{2}\sum\limits_{\vert j \vert \leq m} \left(\vert \mu_{j} \vert^{2} - \vert \boldsymbol{\theta}_{j} \vert^{2}\right)\right]d\mu} d \theta\\
& && \leq && \frac{1}{\exp\left[-\eta\left(\sup\limits_{\mu \in \mathcal{E}} l(\mu, y) - \inf\limits_{\theta \in B}l(\theta, y)\right)\right]}\int_{B} \frac{\prod\limits_{\vert j \vert > m} \delta_{0}(\theta_{j}) \exp\left[-\frac{1}{2}\sum\limits_{\vert j \vert \leq m} \vert \boldsymbol{\theta}_{j} \vert^{2}\right]}{ \int_{\mathcal{E}} \exp\left[-\frac{1}{2}\sum\limits_{\vert j \vert \leq m} \vert \mu_{j} \vert^{2}\right]d\mu} d \theta\\
& && \rightarrow && 0.
\end{alignat*}
\qedsymbol
\end{pro}

We have hence showed that under the iteration asymptotic, the posterior distribution contracts itself on maximisers of the likelihood, constrained by $\theta_{j} = 0$ for any $\vert j \vert > m$.

\textcolor{red}{Add remark with several maximisers}

There is hence a clear link between this type of prior distribution and projection estimators.
We will see that, while considering the noise asymptotic, the choice of the threshold is determinant for the quality of the estimation.
The choice of the threshold for the projection estimators and for sieve priors should be led in a similar fashion, that is, balancing the bias (small value of the threshold) and the variance (high value of the threshold).
As stated previously, the ideal choice of this parameter is however dependent on the parameter of interest and hence not available.
It is hence important to inquire adaptive methods for the selection of this parameter.
Some methods for the frequentist estimation were outlined in the introduction such as the penalised contrast model selection.
In the next section, we introduce the hierarchical sieve prior which consists in modelling the threshold parameter as a random variable.
We will show that by selecting the prior distribution for this hyper-parameter properly, the iteration asymptotic gives a Bayesian interpretation to the penalised contrast model selection.

\section{Adaptivity using a hierarchical prior}\label{2.2}

We denote $\P_{\boldsymbol{\theta}^{M}}$ a so called hierarchical prior distribution, described hereafter, and $\boldsymbol{\theta}^{M}$ a random variable following this prior.
Define $G$ a finite subset of $\mathds{J}$ and  $\pen: \mathcal{P}(G) \rightarrow \R_{+}$ a so-called penalty function.
The threshold parameter noted $m$ for the sieve prior described in the previous section is now a $\mathcal{P}(G)$-valued random variable denoted $M$. We note $\P_{M}$ the distribution of this parameter.

The density of $\P_{M}$ with respect to the counting measure has the shape
\[\P_{M}(m) \propto \exp[- \pen(m)] \mathds{1}_{m \subset G}.\]

The dependance structure between the different quantities of the model is then the following:

\begin{alignat*}{3}
& \P_{\boldsymbol{\theta}^{M} \vert M=m} && = && \P_{\boldsymbol{\theta}^{m}};\\
& \P_{Y \vert \boldsymbol{\theta}, M} && = && \P_{Y \vert \boldsymbol{\theta}}.
\end{alignat*}

We can then obtain the following form for the posterior distribution of the hyper parameter:

\begin{alignat*}{3}
&\P_{M \vert Y}(m, y) &&\propto&& \frac{d\P_{M, Y}}{d\P^{\circ}}(m, y)\\
& &&\propto&&\int_{\Theta}\frac{d\P_{M, Y, \boldsymbol{\theta}^{M}}}{d\P^{\circ} \, d\Q^{\circ}}(m, y, \theta)d\mathds{Q}^{\circ}(\theta)\\
& &&\propto&&\int_{\Theta}\frac{d\P_{Y \vert M, \boldsymbol{\theta}^{M}}}{d\P^{\circ}}(m, y, \theta) \, \frac{d\P_{M, \boldsymbol{\theta}^{M}}}{d\mathds{Q}^{\circ}}(m, \theta)d\mathds{Q}^{\circ}(\theta)\\
& &&\propto&&\int_{\Theta}\frac{d\P_{Y \vert \boldsymbol{\theta}^{M}}}{d\P^{\circ}}(y, \theta) \, \frac{d\P_{\boldsymbol{\theta}^{M}\vert M}}{d\mathds{Q}^{\circ}}(m, \theta) \P_{M}(m)d\mathds{Q}^{\circ}(\theta)\\
& &&\propto&&\P_{M}(m)\int_{\Theta}\frac{d\P_{Y \vert \boldsymbol{\theta}^{M}}}{d\P^{\circ}}(y, \theta) \, \frac{d\P_{\boldsymbol{\theta}^{m}}}{d\mathds{Q}^{\circ}}(m, \theta) d\mathds{Q}^{\circ}(\theta)\\
& && = && \frac{\P_{M}(m)\int_{\Theta}\frac{d\P_{Y \vert \boldsymbol{\theta}^{M}}}{d\P^{\circ}}(y, \theta) \, \frac{d\P_{\boldsymbol{\theta}^{m}}}{d\mathds{Q}^{\circ}}(m, \theta) d\mathds{Q}^{\circ}(\theta)}{\sum\limits_{j \subset G}\P_{M}(j)\int_{\Theta}\frac{d\P_{Y \vert \boldsymbol{\theta}^{M}}}{d\P^{\circ}}(y, \theta) \, \frac{d\P_{\boldsymbol{\theta}^{m}}}{d\mathds{Q}^{\circ}}(j, \theta) d\mathds{Q}^{\circ}(\theta)}\\
& && = && \frac{\exp[- \pen(m)] \int_{\Theta_{m}}\exp[-\frac{1}{2}(2 l(y, \theta) + \sum\limits_{k \in m} \vert \theta_{k} \vert^{2})] d\mathds{Q}^{\circ}(\theta)}{\sum\limits_{j \subset G}\exp[- \pen(j)] \int_{\Theta_{j}}\exp[-\frac{1}{2}(2 l(y, \theta) + \sum\limits_{k \in j} \vert \theta_{k} \vert^{2})] d\mathds{Q}^{\circ}(\theta)}.
\end{alignat*}

From this, we can deduce the iterated posterior.
Indeed, by defining
\[\exp[\Upsilon(Y, m)] := \int_{\Theta_{m}}\exp[-\frac{1}{2}(2 l(y, \theta) + \sum\limits_{k \in m} \vert \theta_{k} \vert^{2})] d\mathds{Q}^{\circ}(\theta)\]
we have:

\begin{alignat*}{3}
&\P_{M \vert Y}^{\eta}(m, y) && = && \frac{\P_{M}(m)\left(\int_{\Theta_{m}}\exp[-\frac{1}{2}(2 l(y, \theta) + \sum\limits_{k \in m} \vert \theta_{k} \vert^{2})] d\mathds{Q}^{\circ}(\theta)\right)^{\eta}}{\sum\limits_{j \subset J}\P_{M}(j)\left(\int_{\Theta_{j}}\exp[-\frac{1}{2}(2 l(y, \theta) + \sum\limits_{k \in j} \vert \theta_{k} \vert^{2})] d\mathds{Q}^{\circ}(\theta)\right)^{\eta}}\\
& && = && \frac{\exp[-\pen(m) + \eta \Upsilon(Y, m)]}{\sum\limits_{j \subset G}\exp[- \pen(j) + \eta \Upsilon(Y, j)]} \mathds{1}_{m \subset G}\\
& && = && \frac{1}{\sum\limits_{j \subset G}\exp\left[\eta \left(\Upsilon(Y, j) - \Upsilon(Y, m)\right) - \left(\pen(j) - \pen(m)\right)\right]} \mathds{1}_{m \subset G}
\end{alignat*}

and we can deduce the self informative Bayes carrier.

\begin{lm}{\textsc{Self informative Bayes carrier of the hyper-parameter in a hierarchical sieve prior I}\\}\label{lm2.2.1}
The support of the self informative Bayes carrier for the hyper-parameter $M$ is
\[\argmax\limits_{m \subset G} \{\Upsilon(Y, m)\}.\]
\end{lm}

Unfortunately, in many practical cases, the choice led by $\argmax\limits_{m \subset G} \{\Upsilon(Y, m)\}$ is $G$ itself and leads to inconsistent inference (as we will show later).
However, if one allows the prior distribution to depend on $\eta$ and to take the shape $\exp[- \eta \pen(m)] \mathds{1}_{m \subset G}$, we obtain the following theorem.

\begin{thm}{\textsc{Self informative Bayes carrier of the hyper-parameter in a hierarchical sieve prior II}\\}\label{thm2.2.1}
Using the modified prior which depends on $\eta$, the support of the self informative Bayes carrier for the hyper-parameter $M$ is
\[\argmax\limits_{m \subset G} \{\Upsilon(Y, m) - \pen(Y, m)\}.\]
\end{thm}

\begin{pro}{\textsc{Proof of \nref{thm2.2.1}}\\}\label{pro2.2.1}
For any finite set $P$ of subsets of $G$ such that $\max\limits_{m \in P} \Upsilon(Y, m) - \pen(Y, m) < \max\limits_{k \subset G} \Upsilon(Y, k) - \pen(Y, k)$, we can write

\begin{alignat*}{3}
& \P_{M\vert Y}^{\eta}(P) && = && \sum\limits_{m \in P} \frac{1}{\sum\limits_{j \subset G}\exp\left[\eta \left(\Upsilon(Y, j) - \Upsilon(Y, m) - \left(\pen(j) - \pen(m)\right)\right)\right]} \mathds{1}_{m \subset G}\\
& && \leq && \frac{Card(P)}{\exp\left[\eta \left(\max\limits_{j \subset G}\left(\Upsilon(Y, j) - \pen(j)\right) - \max\limits_{m \in P}\left(\Upsilon(Y, m) - \pen(m)\right)\right)\right]} \mathds{1}_{m \subset G}\\
& && \rightarrow && 0.
\end{alignat*}
\qedsymbol
\end{pro}


The posterior distribution for $\boldsymbol{\theta}^{M}$ itself follows:

\begin{alignat*}{3}
&\frac{d\mathds{Q}_{\boldsymbol{\theta}^{M} \vert Y}}{d\P^{\circ}}(\theta, y) && \propto && \frac{d\P_{\boldsymbol{\theta}^{M}, Y}}{d\mathds{Q}^{\circ} \, d\P^{\circ}}(\theta, y)\\
& &&\propto&& \sum\limits_{m \subset J} \frac{d\P_{\boldsymbol{\theta}^{M}, Y, M}}{d\mathds{Q}^{\circ} \, d\P^{\circ} \, d\P^{\circ}}(\theta, y, m)\\
& &&\propto&& \sum\limits_{m \subset J} \frac{d\P_{\boldsymbol{\theta}^{M} \vert Y, M}}{d\mathds{Q}^{\circ}}(\theta, y, m) \frac{d\P_{Y, M}}{d\P^{\circ} \, d\P^{\circ}}\\
& &&\propto&& \sum\limits_{m \subset J} \frac{d\P_{\boldsymbol{\theta}^{m} \vert Y}}{d\mathds{Q}^{\circ}}(\theta, y, m) \frac{d\P_{M \vert Y}}{d\P^{\circ}} \frac{d\P_{Y}}{d\P^{\circ}}(Y)\\
& &&=&& \sum\limits_{m \subset J} \frac{d\P_{\boldsymbol{\theta}^{m} \vert Y}}{d\mathds{Q}^{\circ}}(\theta, y, m) \frac{d\P_{M \vert Y}}{d\P^{\circ}}.
\end{alignat*}

From this, we can deduce the iterated posterior distribution for $\boldsymbol{\theta}^{M}$:
\begin{alignat*}{3}
& \frac{d\mathds{Q}_{\boldsymbol{\theta}^{M} \vert Y}^{\eta}}{d\P^{\circ}}(\theta, y) && = && \sum\limits_{m \subset G} \frac{d\P_{\boldsymbol{\theta}^{m} \vert Y}^{\eta}}{d\mathds{Q}^{\circ}}(\theta, y, m) \frac{d\P_{M \vert Y}^{\eta}}{d\P^{\circ}}(m, y)\\
& && = && \sum\limits_{m \subset G} \frac{\exp\left[-\left(\frac{1}{2}\sum\limits_{j \in m} \vert \boldsymbol{\theta}_{j} \vert^{2} + \eta l(\boldsymbol{\theta}, y)\right)\right] \cdot \prod\limits_{j \notin m} \delta_{0}(\boldsymbol{\theta}_{j})}{\int_{\Theta_{m}} \exp\left[-\left(\frac{1}{2}\sum\limits_{j \in m} \vert \mu_{j} \vert^{2} + \eta l(\mu, y)\right)\right] d\mu} \frac{\exp[-\pen(m) + \eta \Upsilon(Y, m)]}{\sum\limits_{j \subset G}\exp[- \pen(j) + \eta \Upsilon(Y, j)]} \mathds{1}_{m \subset G}\\
\end{alignat*}

And as a consequence, we can deduce the self informative Bayes carrier.

\begin{thm}{\textsc{Self informative carrier using a hierarchical sieve prior}\\}
Denote $\widehat{m} := \argmax\limits_{m \subset G} \{\Upsilon(Y, m) - \pen(m)\}$ then the support of the self informative Bayes carrier is contained in $\argmax\limits_{\theta \in \Theta_{m}, m \in \widehat{m}}\{-l(\theta, Y)\}$.
\end{thm}

We have hence seen in these two first sections investigated the behaviour of the sieve prior and its hierarchical version under the iterative asymptotic and shown that under some mild assumptions, their self informative Bayes carriers correspond to some constrained maximum likelihood estimator and penalised contrast model selection version of it respectively.

We should now investigate the behaviour of these (iterated) priors under the noise asymptotic and define hypotheses under which they behave properly.
\section{Shape of the aggregation estimators}\label{FREQ_GENERAL_SHAPE}\label{freq:ge:shape}

We want to define a family of estimators $(\widehat{\theta}^{(\eta)})_{\eta \in ]0, \infty]}$ of $\theta^{\circ}$ and estimators for $f$, $\widehat{f}^{(\eta)} := \fourier^{\star}(\widehat{\theta}^{(\eta)})$ such that, for any $\eta$, $\widehat{\theta}^{(\eta)}$ has the shape
\begin{equation}\label{freq:ge:shape:kn}
(\widehat{\theta}^{(\eta)}(s))_{s \in \mathds{F}} = \sum\nolimits_{m \in \N} \P_{M}^{(\eta)}(m) \cdot (\theta_{n, \overline{m}}(s))_{s \in \mathds{F}} = \sum\nolimits_{m \geq \vert s \vert} \P_{M}^{(\eta)}(m) \cdot (\theta_{n}(s))_{s \in \mathds{F}}
\end{equation}
under \nref{AS_INTRO_DATA_KNOWN}, and
\begin{equation}\label{freq:ge:shape:uk}
(\widehat{\theta}^{(\eta)}(s))_{s \in \mathds{F}} = \sum\nolimits_{m \in \N} \widehat{\P}_{M}^{(\eta)}(m) \cdot (\theta_{n, n_{\lambda}, \overline{m}}(s))_{s \in \mathds{F}} = \sum\nolimits_{m \geq \vert s \vert} \widehat{\P}_{M}^{(\eta)}(m) \cdot (\theta_{n, n_{\lambda}}(s))_{s \in \mathds{F}}
\end{equation}
under \nref{AS_INTRO_DATA_UNKNOWN}.
The sequence $(\P_{M}^{(\eta)}(m))_{m \in \N}$ is the aggregation sequence.
Under \nref{AS_INTRO_DATA_KNOWN} it depends on the observations $Y^{n}$ as well on the known operator $T$ through its eigen values $(\lambda(s))_{s \in \mathds{F}}$ whereas under \nref{AS_INTRO_DATA_UNKNOWN}, it only depends on the observed data $Y^{n}$ and $\epsilon^{n_{\lambda}}$.
This notation is motivated by the Bayesian inspiration of the method.

We give in \nref{fig:ge:adaptive:aggregation} an illustration of the aggregation estimator used in a Gaussian sequence space model in the direct problem case, that is to say $\lambda(s) = 1$ for any $s$ in $\N$.

\begin{figure}
  \centering
  \begin{tabular}{@{}c@{}}
    \includegraphics[width=.49\linewidth]{gauss/adaptive/aggregation_surv.png} \\[\abovecaptionskip]
  \end{tabular}
  \begin{tabular}{@{}c@{}}
    \includegraphics[width=.49\linewidth]{gauss/adaptive/aggregation.png} \\[\abovecaptionskip]
  \end{tabular}
  \caption{Aggregation estimator on an Gaussian sequence space model, direct problem case}
  \label{fig:ge:adaptive:aggregation}
\end{figure}

Taking inspiration in the posterior distributions obtained with a hierarchical prior in the previous chapter, we will give the following shape to the aggregation weights.
In the case \nref{AS_INTRO_DATA_KNOWN} let be the following functions
\begin{multline}\label{freq:ge:shape:kn:we}
\Upsilon : \mathds{N} \to \R_{+}, \quad m \mapsto \Upsilon(m); \qquad \pen^{\Lambda} : \N \to \R_{+}, \quad m \mapsto \pen^{\Lambda}(m);\\
\P_{M}^{(\eta)} : \N \to \R_{+}, \quad m \mapsto \tfrac{\exp[-\eta n (-\Upsilon(m) + \pen^{\Lambda}(m))]}{\sum\nolimits_{k = 0}^{n} \exp[-\eta n (-\Upsilon(k) + \pen^{\Lambda}(k))]} \mathds{1}_{m \leq n};
\end{multline}
where $\Upsilon$ depends on the observations $Y^{n}$ as well as the known operator $T$ through the sequence $\lambda$ of its eigen values; and $\pen^{\Lambda}$ depends only on the parameter $T$ through the sequence $\lambda$ of its eigen values.
Under \nref{AS_INTRO_DATA_UNKNOWN}, we define
\begin{multline}\label{freq:ge:shape:uk:we}
\Upsilon : \N \to \R_{+}, \quad m \mapsto \Upsilon(m); \qquad \pen^{\widehat{\Lambda}} : \N \to \R_{+}, \quad m \mapsto \pen^{\widehat{\Lambda}}(m);\\
\widehat{\P}_{M}^{(\eta)} : \N \to \R_{+}; \quad m \mapsto \tfrac{\exp[\eta n (\Upsilon(m) - \pen^{\widehat{\Lambda}}(m))]}{\sum\nolimits_{k = 0}^{n} \exp[\eta n (\Upsilon(k) - \pen^{\widehat{\Lambda}}(k))]} \mathds{1}_{m \leq n};
\end{multline}
where $\Upsilon$ depends solely the observations $Y^{n}$ and $\epsilon^{n_{\lambda}}$; and $\pen^{\widehat{\Lambda}}$ depends only on the observations $\epsilon^{n_{\lambda}}$.
The functions $\Upsilon$, and $\pen^{\Lambda}$ will respectively be called contrast and penalty.
For any subset $S$ of $\N$, we denote $\P_{M}^{(\eta)}(S) = \sum_{k \in S} \P_{M}^{(\eta)}(k)$.
One would expect that as the amount of data increases, the number of coefficients estimated increases too, as our observations allows us to recover more information about the system of interest as illustrated in \nref{fig:ge:adaptive:M} by representing $\P_{M}^{(\eta)}(\llbracket m, n \rrbracket)$ for increasnig values of $n$.

\begin{figure}
  \centering
  \begin{tabular}{@{}c@{}}
    \includegraphics[width=.49\linewidth]{density/distM/1.png} \\[\abovecaptionskip]
  \end{tabular}
  \begin{tabular}{@{}c@{}}
    \includegraphics[width=.49\linewidth]{density/distM/20.png} \\[\abovecaptionskip]
  \end{tabular}
  
    \begin{tabular}{@{}c@{}}
    \includegraphics[width=.49\linewidth]{density/distM/40.png} \\[\abovecaptionskip]
  \end{tabular}
  \begin{tabular}{@{}c@{}}
    \includegraphics[width=.49\linewidth]{density/distM/50.png} \\[\abovecaptionskip]
  \end{tabular}
  \caption{Evolution of the aggregation weights }
  \label{fig:ge:adaptive:M}
\end{figure}

\medskip

Consider first the asymptotic when one lets $\eta$ tend to infinity.
Under \nref{AS_INTRO_DATA_KNOWN}, following a model selection approach (c.f. \ncite{barron1999risk} and \ncite{Massart2007} for an extensive description), a dimension parameter $\hDi$ is determined among a collection of admissible values $\nset{1,\ssY}$ by minimising the penalised contrast function $-\Vnormlp{\txdfPr}+\penSv$, that is
\begin{equation}\label{freq:ge:shape:kn:de:ms}
  \tDi:=\argmin\nolimits_{\Di\in\nset{1,\ssY}} \big\{-\Upsilon(m) + \penSv\big\}.
\end{equation}
If $\tDi$ minimises uniquely the penalised contrast function, then it is easily seen that the discrete probability measure $\rWe[]$ on the set $\nset{1,\ssY}$ given by the weights $\rWe[](\{\Di\})=\rWe$ as in \eqref{freq:ge:shape:kn:we} degenerates to a Dirac measure $\dirac[\tDi]$ on the point $\tDi$ as $\rWc\to\infty$.
Precisely, for any $\Di\in\nset{1,n}$ holds
\begin{equation}\label{freq:ge:shape:kn:de:msWe}
  \lim\nolimits_{\rWc\to\infty}\rWe=\dirac[\tDi](\{\Di\})=:\msWe
\end{equation}
Thereby, in the sequel we consider the model selected estimator
\[\txdfPr[\tDi]=\widehat{\theta}^{(\infty)} =\sum\nolimits_{\Di\in\nset{1,\ssY}}\msWe\txdfPr\]
as an aggregation with respect to the discrete measure $\msWe[]=\dirac[\tDi]$ on the set $\nset{1,\ssY}$.
We give in \nref{fig:ge:adaptive:selection} an illustration of the model selection estimator used on a Gaussian sequence space model, in the direct problem case, that is to say $\lambda(s) = 1$ for all $s$.

\begin{figure}
\centering
  \begin{tabular}{@{}c@{}}
    \includegraphics[width=.49\linewidth]{gauss/adaptive/model_selection.png} \\[\abovecaptionskip]
  \end{tabular}
  \begin{tabular}{@{}c@{}}
    \includegraphics[width=.49\linewidth]{gauss/adaptive/penalised_contrast.png} \\[\abovecaptionskip]
  \end{tabular}
  \caption{Model selection estimator on an Gaussian sequence space model, direct problem case}
  \label{fig:ge:adaptive:selection}
\end{figure}

Under \nref{AS_INTRO_DATA_UNKNOWN} consider again a  model selection approach by
minimising now the penalised contrast function $\Upsilon(m)+\peneSv$, that is
\begin{equation}\label{freq:ge:shape:uk:de:ms}
  \hDi:=\argmin_{\Di\in\nset{1,\ssY}} \big\{-\Upsilon(m)+\peneSv\big\}.
\end{equation}
If $\hDi$ minimises uniquely the penalised contrast function, then for any $\Di\in\nset{1,n}$ holds
\begin{equation}\label{freq:ge:shape:uk:de:msWe}
  \lim_{\rWc\to\infty}\erWe=\dirac[\hDi](\{\Di\})=:\widehat{\P}_{M}^{(\infty)}.
\end{equation}
Thereby, we consider again the model selected estimator

$\txdfPr[\tDi] = \widehat{\theta}^{(\infty)} = \sum_{\Di\in\nset{1,\ssY}}\widehat{\P}_{M}^{(\infty)}(m)\hxdfPr$ as an aggregation with respect to the discrete measure $\msWe[]=\dirac[\hDi]$ on the set $\nset{1,\ssY}$.

We will consider two examples in this chapter, namely the inverse Gaussian sequence space model as well as the circular deconvolution model.
In both cases the functions $\Upsilon$, $\pen^{\Lambda}$, and $\pen^{\widehat{\Lambda}}$ take the same shape which we hence give here.
\begin{de}\label{freq:ge:shape:kn:de:pen:oo}
  Under \nref{AS_INTRO_DATA_KNOWN}, let be a universal constant $\cpen$ to be fixed depending on the considered model.
  For any $\Di$ in $\nset{1,\ssY}$, remind that $\Lambda(m) = \vert \lambda(m) \vert^{-2}$, and $\Lambda_{+}(m) = \max\{\Lambda(s), s \in \mathds{F}_{m} \}$ and define
  \begin{alignat*}{4}
  & \Upsilon(m) && := && \Vert \theta_{n, \overline{m}} \Vert_{l^{2}}^{2};  \quad && \cmiSv := \tfrac{\log^{2}(\Di\miSv \vee(\Di+2))}{\log^{2}(\Di+2)}\geq1;\\
  & \DipenSv && := && \cmiSv \Di \miSv; \quad && \penSv:= \penD.
  \end{alignat*}
  \assEnd
\end{de}
\begin{de}\label{freq:ge:shape:uk:de:pen:oo}
  Under \nref{AS_INTRO_DATA_UNKNOWN}, let be a universal constant $\cpen$ to be fixed depending on the considered model.
  Then, for any $m$ in $\N$, we define
  \begin{alignat*}{4}
  & \Upsilon(m) && := && \Vert \theta_{n, n_{\lambda}, \overline{m}}\Vert_{l^{2}}^{2}; && \quad \eiSv[(s)]:= \vert \hfedfmpI[(s)] \vert ^2\\
    & \meiSv && := && \max\{\eiSv[(l)],l\in\nset{1,\Di}\}; && \quad \cmeiSv:=\tfrac{\log^{2}(\Di\meiSv\vee(\Di+2))}{\log^{2}(\Di+2)}\geq1;\\
    &\DipeneSv && := && \cmeiSv \Di \meiSv;&& \quad \peneSv:= \peneD.
  \end{alignat*}
  \assEnd
\end{de}
Notice that, with the exception of the constant $\kappa$, our estimator is now fully determined, in both cases \nref{AS_INTRO_DATA_KNOWN} and \nref{AS_INTRO_DATA_UNKNOWN}.

\section{Strategy of proof for optimality of aggregation estimator}\label{freq:ge:strat}\label{FREQ_STRATEGY}
As we have now given a precise shape to our aggregation estimator, we propose a strategy to compute upper bounds for its convergence rate in $l^{2}$-norm.
Our method is inspired by the strategy to compute upper bounds for the contraction rate of hierarchical sieves we presented in the previous chapter.
We will hence highlight a decomposition of the risk which separates the risk obtained by taking values of the threshold which are respectively "too small", "too large", or "optimal".
Those terms should be understood with respect to the quadratic risk of the projection estimator associated with this choice of threshold.
One would then prove that the values of the threshold which are too small or too large do not receive an important weight under $\P_{M}^{(\eta)}$ or $\widehat{P}_{M}^{(\eta)}$.
Before going any further, notice that, for any $\Di$ and $m^{\bullet}$ in $\N$, the aggregation weights can be bounded in the following way:
\begin{multline*}
\tfrac{\exp[-\eta n (-\Vert \theta_{n, \overline{m}} \Vert_{l^{2}} + \pen^{\Lambda}(m))]}{\sum\nolimits_{k = 0}^{n} \exp[-\eta n (-\Vert \theta_{n, \overline{k}} \Vert_{l^{2}} + \pen^{\Lambda}(k))]} \mathds{1}_{m \leq n}\\
\leq \exp[-\eta n (\Vert \theta_{n, \overline{m}} \Vert_{l^{2}} - \Vert \theta_{n, \overline{m^{\bullet}}} \Vert_{l^{2}} + \pen^{\Lambda}(m^{\bullet}) - \pen^{\Lambda}(m))] \mathds{1}_{m \leq n}\text{; and}
\end{multline*}
\begin{multline*}
\tfrac{\exp[-\eta n (-\Vert \theta_{n, n_{\lambda}, \overline{m}} \Vert_{l^{2}} + \pen^{\Lambda}(m))]}{\sum\nolimits_{k = 0}^{n} \exp[-\eta n (-\Vert \theta_{n, n_{\lambda}, \overline{k}} \Vert_{l^{2}} + \pen^{\Lambda}(k))]} \mathds{1}_{m \leq n}\\
\leq \exp[-\eta n (\Vert \theta_{n, n_{\lambda}, \overline{m}} \Vert_{l^{2}} - \Vert \theta_{n, n_{\lambda}, \overline{m^{\bullet}}} \Vert_{l^{2}} + \pen^{\Lambda}(m^{\bullet}) - \pen^{\Lambda}(m))] \mathds{1}_{m \leq n}.
\end{multline*}

Then, the following lemma, which proof is given in \nref{pro:re:contr} allows to derive an upper bound which is easier to control.

\begin{lm}\label{re:contr}
Given $\ssY\in\Nz$ and $\xdfPr[],\dxdfPr[]\in\lp[2]$ consider the
  families of  orthogonal projections
  
  $\setB{\dxdfPr=\dProj{\Di}{}\dxdfPr[],\Di\in\nset{1,n}}$ and $\setB{\xdfPr=\dProj{\Di}{}\xdf,\Di\in\nset{1,n}}$.
  
  If $\Vnormlp{\dProj{\Di}{}^\perp\xdf}^2=\Vnormlp{\xdf_{\underline{0}}}^2\bias^2(\xdf)$ for all
  $\Di\in\nset{1,\ssY}$, then for any $l\in\nset{1,n}$ holds
 \begin{resListeN}[]
\item\label{re:contr:e1}
$\Vnormlp{\dxdfPr[k]}^2-\Vnormlp{\dxdfPr[l]}^2\leq
\tfrac{11}{2}\Vnormlp{\dxdfPr[l]-\xdfPr[l]}^2-\tfrac{1}{2}\Vnormlp{\xdf_{\underline{0}}}^2\{\bias[k]^2(\xdf)-\bias[l]^2(\xdf)\}$,
for all $k\in\nsetro{1,l}$;
\item\label{re:contr:e2}
$\Vnormlp{\dxdfPr[k]}^2-\Vnormlp{\dxdfPr[l]}^2\leq \tfrac{7}{2}\Vnormlp{\dxdfPr[k]-\xdfPr[k]}^2+\tfrac{3}{2}\Vnormlp{\xdf_{\underline{0}}}^2
\{\bias[l]^2(\xdf)-\bias[k]^2(\xdf)\}$, for all $k\in\nsetlo{l,n}$.
\end{resListeN}
\reEnd
\end{lm}

\subsection{Known operator}\label{freq:ge:strat:kn}\label{FREQ:GE:STRAT:KN}
Consider first the case \nref{AS_INTRO_DATA_KNOWN}.
We shall hence keep in mind \ref{freq:ge:shape:kn}, \ref{freq:ge:shape:kn:we}, \nref{freq:ge:shape:kn:de:pen:oo} as well as \ref{freq:ge:shape:kn:de:ms} and \ref{freq:ge:shape:kn:de:msWe}.
\textbf{Note that the detailed proofs for all results given here can be found in \nref{pro:freq:ge:strat:kn}}.

Both for the quadratic and the maximal risk, our strategy is based on the decomposition of the quadratic loss function displayed in \nref{freq:ge:strat:kn:co:agg}.
This decomposition is independent of the model and only relies on the fact that the parameter space is equipped with a nested sieve and the fact that our estimator aggregation structure takes advantage of it.
\begin{lm}\label{freq:ge:strat:kn:co:agg}
First writing the $l^{2}$-distance between $\theta^{\circ}$ and $\widehat{\theta}^{\eta}$ we obtain, for any $\mDi$ and $\pDi$ in $\llbracket 1, n \rrbracket$ such that $\mDi \leq \pDi$, and sequence $(\pen(m))_{m \in \N}$ of compensating terms,
\begin{multline}\label{freq:ge:strat:kn:co:agg:e1}
    \Vnormlp{\txdfAg-\xdf}^2\leq \tfrac{2}{7}\pen(\pDi) +2\Vnormlp{\xdf_{\underline{0}}}^2\bias[\mDi]^2(\xdf)\\\hfill
    +2\Vnormlp{\xdf_{\underline{0}}}^2\We[](\nsetro{1,\mDi})+\tfrac{2}{7}\sum\nolimits_{\Di\in\nsetlo{\pDi,\ssY}}\pen(m)\We\Ind{\{\Vnormlp{\txdfPr-\xdf_{\overline{m}}}^2<\pen(\Di)/7\}}\\
+2\sum_{\Di\in\nset{\pDi,\ssY}}\vectp{\Vnormlp{\txdfPr-\xdf_{\overline{m}}}^2-\pen(\Di)/7}  
+\tfrac{2}{7}\sum_{\Di\in\nsetlo{\pDi,\ssY}}\pen(\Di)\Ind{\{\Vnormlp{\txdfPr-\xdf_{\overline{m}}}^2\geq \pen(m)/7\}}.
\end{multline}
\reEnd
\end{lm}
The proof strategy will be articulated around the search for sequences $\pDi$, $\mDi$ and $\pen(m)$ such that each term is properly controlled.
In practice, the terms $\tfrac{2}{7}\pen(\pDi)$ and $2\Vnormlp{\xdf_{\underline{0}}}^2\bias[\mDi]^2(\xdf)$ will be the leading terms in the sum.

\subsubsection{Quadratic risk bounds}\label{freq:ge:strat:kn:qu}
We propose a strategy which allows to prove that the sequence defined hereafter is an upper bound for the quadratic risk of the aggregation estimator we just defined.
\begin{de}\label{freq:ge:strat:kn:qu:de:rate}
Remind that we defined for any $\theta$ in $\Theta$ and $\Di$ in $\N$ the following term $\b_{m}^{2}(\theta) = \Vert \theta_{\underline{m}} \Vert_{l^{2}}^{2}\Vert \theta_{\underline{0}} \Vert_{l^{2}}^{-2} \leq 1$.
We then define a family of sequences $(\daRa{\Di}{(\xdf)})_{\Di \in \N} := (\daRa{\Di}{(\xdf,\Lambda)})_{\Di \in \N} = ([\b_{m}^{2}(\theta^{\circ}) \vee \penSv/\cpen])_{\Di \in \N}$ and hence it holds for all $\Di$ in $\nset{1,\ssY}$
    \begin{equation}\label{freq:ge:strat:kn:qu:de:rate:e1}
      [\Vnormlp{\xdf_{\underline{0}}}^2+\cpen]\daRa{\Di}{(\xdf)}\geq\Vnormlp{\xdf_{\underline{0}}}^2\bias^2(\xdf)\vee\penSv.
      \end{equation}
We intend to prove that the specific choice
\begin{multline*}
\aDi{\ssY}(\xdf):=\argmin\Nset[\Di\in\Nz]{\daRa{\Di}{(\xdf)}}\in\nset{1,\ssY}; \\
\naRa{(\xdf)}:=\naRa{(\xdf,\Lambda)}:=\min\Nset[\Di\in\Nz]{\daRa{\Di}{(\xdf)}}
\end{multline*}
with $\daRa{\aDi{\ssY}}{(\xdf,\Lambda)}=\naRa{(\xdf,\Lambda)}$ defines an upper bound for the convergence rate of the aggregation estimators.
\assEnd
\end{de}
Note that the proofs for the results displayed here can be found in \nref{pro:freq:ge:strat:kn:qu}
\begin{rmk}\label{freq:ge:strat:kn:qu:rmk:rate} The following statements can be
shown using the same   arguments as in \nref{oo:rem:ora}
by exploiting that the sequence $\bias^2(\xdf)$ is non-increasing with limit zero and $\bias[0]^2(\xdf)\leq1$. 
By construction  for all $\ssY\in\Nz$ it hold 
$\naRa{(\xdf)}\geq \ssY^{-1}$ and $\naRa{(\xdf)}=\mathfrak{o}_{n}(1)$.
Moreover, for all $\ssY\in\Nz$ we have $\aDi{\ssY}(\xdf)\in\nset{1,\ssY}$,
$\aDi{\ssY}(\xdf)=\argmin\Nset[{\Di\in\nset{1,\ssY}}]{\daRa{\Di}{(\xdf)}}$ and 
$\naRa{(\xdf)}=\min\Nset[{\Di\in\nset{1,\ssY}}]{\daRa{\Di}{(\xdf)}}$. 
Thereby, in case \ref{oo:xdf:p} we conclude that $\aDi{\ssY}(\xdf)=K$ and
the rate $\naRa{(\xdf)}$ is parametric, that is
$\naRa{(\xdf)}=\DipenSv[K]\ssY^{-1}\approx\ssY^{-1}$, and hence
equals the oracle rate $\oRa{\xdf}$, i.e. $\oRa{\xdf}\approx\naRa{(\xdf)}$. On the other
hand side, in case \ref{oo:xdf:np}  the rate
$\naRa{(\xdf)}$ is nonparametric, that is,
$\ssY\naRa{(\xdf)}\to\infty$ and $\aDi{\ssY}(\xdf)\to\infty$  as
$\ssY\to\infty$. Moreover, by construction holds $\naRa{(\xdf)}\geq\oRa{\xdf}$.
 \remEnd
\end{rmk}

\begin{te}
 Let us first briefly illustrate the last definitions by stating the
 order of $\aDi{\ssY}(\xdf)$ and $\naRa{(\xdf)}$ in the cases considered
 in \nref{il:oo}
\end{te}
% ....................................................................
% <<Il upper bound oo>>
% ....................................................................
\begin{il}\label{freq:ge:strat:kn:qu:il:rate}Let us illustrate  \nref{freq:ge:strat:kn:qu:de:rate}
  considering as in \nref{il:oo} usual behaviour \ref{il:oo:oo},
  \ref{il:oo:so} and \ref{il:oo:os} for the sequences
  $\Nsuite[\Di]{\bias[\Di](\xdf)}$ and $\Nsuite[\Di]{\iSv[\Di]}$:
  \begin{Liste}[]
  \item[\mylabel{freq:ge:strat:kn:qu:il:rate:np:oo}{\dg\bfseries{[o-o]}}] Since
    $\bias^2(\xdf)\approx\Di^{-2p}$ and $\DipenSv\approx\Di^{2a+1}$ follows
    $\daRa{\aDi{\ssY}}{(\xdf,\Lambda)}\approx(\aDi{\ssY})^{-2p}\approx\DipenSv[\aDi{\ssY}]\ssY^{-1}\approx(\aDi{\ssY})^{2a+1}\ssY^{-1}$
    which implies $\aDi{\ssY}\approx\ssY^{1/(2p+2a+1)}$,
    $\cmiSv[\aDi{\ssY}]\aDi{\ssY}\approx\ssY^{1/(2p+2a+1)}$,
    $\naRa{(\xdf)}\approx\ssY^{-2p/(2p+2a+1)}$ and
    $ \vert \log\naRa{(\xdf)} \vert \approx(\log\ssY)$.
  \item[\mylabel{freq:ge:strat:kn:qu:il:rate:np:os}{\dg\bfseries{[o-s]}}] Since
    $\bias^2(\xdf)\approx\Di^{-2p}$ and
    $\DipenSv\approx\Di^{1+4a}\exp(\Di^{2a})$ follows
    $\daRa{\aDi{\ssY}}{(\xdf,\Lambda)}\approx(\aDi{\ssY})^{-2p}\approx\DipenSv[\aDi{\ssY}]\ssY^{-1}\approx(\aDi{\ssY})^{1+4a}\exp((\aDi{\ssY})^{2a})$
    which implies $\aDi{\ssY}\approx(\log\ssY)^{1/(2a)}$,
    $\cmiSv[\aDi{\ssY}]\aDi{\ssY}\approx(\log\ssY)^{2+1/(2a)}$,
    $\naRa{(\xdf)}\approx(\log\ssY)^{-p/a}$ and
    $ \vert \log\naRa{(\xdf)} \vert \approx(\log\log\ssY)$.
  \item[\mylabel{freq:ge:strat:kn:qu:il:rate:np:so}{\dg\bfseries{[s-o]}}] Since
    $\bias^2(\xdf)\approx\exp(-\Di^{2p})$ and $\DipenSv\approx\Di^{2a+1}$
    follows
    $\daRa{\aDi{\ssY}}{(\xdf,\Lambda)}\approx\exp(-(\aDi{\ssY})^{2p})\approx\DipenSv[\aDi{\ssY}]\ssY^{-1}\approx
    (\aDi{\ssY})^{2a+1}\ssY^{-1}$ which implies
    $\aDi{\ssY}\approx(\log\ssY)^{1/(2p)}$,
    $\cmiSv[\aDi{\ssY}]\aDi{\ssY}\approx(\log\ssY)^{1/(2p)}$,
    $\naRa{(\xdf)}\approx(\log\ssY)^{(2a+1)/(2p)}\ssY^{-1}$ and
    $ \vert \log\naRa{(\xdf)} \vert \approx(\log\ssY)$.
  \end{Liste}
  We note that  in the three cases \ref{freq:ge:strat:kn:qu:il:rate:np:oo},
  \ref{freq:ge:strat:kn:qu:il:rate:np:os} and \ref{freq:ge:strat:kn:qu:il:rate:np:so} the rate
  $\naRa{(\xdf)}$ coincide with the
  oracle rate $\oRa{\xdf}$ derived in \nref{il:oo} \ref{il:oo:oo},
  \ref{il:oo:os} and \ref{il:oo:so}, respectively.\ilEnd 
\end{il}
% ....................................................................
% Define *Di
% ....................................................................
\begin{te}
  Under \nref{freq:ge:shape:kn:de:pen:oo} for arbitrary $\pdDi,\mdDi \in \nset{1,\ssY}$ let us define
  \begin{multline}\label{freq:ge:strat:kn:qu:de:mDipDi}
    \mDi:=\min\set{\Di\in\nset{1,\mdDi}: \Vnormlp{\xdf_{\underline{0}}}^2\bias^2(\xdf) \leq [\Vnormlp{\xdf_{\underline{0}}}^2+4\cpen]\daRa{\mdDi}{(\xdf)}}\quad\text{and}\\
    \pDi:=\max\set{\Di\in\nset{\pdDi,\ssY} : \penSv \leq 2[3\Vnormlp{\xdf_{\underline{0}}}^2 + 2\cpen] \daRa{\pdDi}{(\xdf)}}
  \end{multline}
  where the defining set obviously contains $\mdDi$ and $\pdDi$, respectively, and hence, it is not empty.
\end{te}
\begin{te}
Considering the third and fourth terms on the right hand side of \eqref{freq:ge:strat:kn:co:agg:e1}, we will use the following lemma to control them.
\end{te}
% ....................................................................
% <<Re Sum Random weights>>
% ....................................................................
\begin{lm}\label{freq:ge:strat:kn:qu:re:SrWe:ag}
Consider the data-driven aggregation weights $\rWe[]$
  as in \eqref{freq:ge:shape:kn:we}.  Under \nref{freq:ge:shape:kn:de:pen:oo} with
  $\cpen\geq8\log(3e)$ for any
  $\mdDi,\pdDi\in\nset{1,\ssY}$ and associated $\pDi,\mDi\in\nset{1,\ssY}$
  as in \eqref{freq:ge:strat:kn:qu:de:mDipDi} hold
  \begin{resListeN}
  \item\label{freq:ge:strat:kn:qu:re:SrWe:ag:i}
    $\rWe[](\nsetro{1,\mDi})
    \Ind{\setB{\Vnormlp{\txdfPr[\mdDi]-\xdfPr[\mdDi]}^2<\cpen\daRa{\mdDi}{(\xdf)}/7}}\leq
    \tfrac{1}{\rWc\cpen}\Ind{\{\mDi>1\}}
    \exp\big(-\tfrac{3\rWc\cpen}{14} \ssY\daRa{\mdDi}{(\xdf)}\big)$;
  \item\label{freq:ge:strat:kn:qu:re:SrWe:ag:ii}
    $\sum_{\Di\in\nsetlo{\pDi,\ssY}}\penSv\rWe
    \Ind{\{\Vnormlp{\txdfPr[\Di]-\xdfPr[\Di]}^2<\penSv/7\}}
    \leq \ssY^{-1}\{\tfrac{16}{\cpen\rWc^{2}}+ \tfrac{8}{\rWc}\}$.
  \end{resListeN}\reEnd
\end{lm}
% ....................................................................
% <<Te Sum MS Random weights>>
% ....................................................................
\begin{te}
  We combine the upper bound in \nref{freq:ge:strat:kn:co:agg} and the bounds given in \nref{freq:ge:strat:kn:qu:re:SrWe:ag}.
  Clearly, due to \nref{freq:ge:strat:kn:qu:re:SrWe:ag} we have
  \begin{displaymath}
    \E\rWe[](\nsetro{1,\mDi})\leq\Ind{\{\mDi>1\}} \{\tfrac{1}{\rWc\cpen}\exp\big(-\tfrac{3\rWc\cpen}{14}n\daRa{\mdDi}{(\xdf)}\big) + \P\big(\Vnormlp{\txdfPr[\mdDi]-\xdfPr[\mdDi]}^2 \geq \tfrac{\cpen}{7}\daRa{\mdDi}{(\xdf)}\big)\}
  \end{displaymath}
  and, hence from \eqref{freq:ge:strat:kn:co:agg} follows immediately
  \begin{multline}\label{freq:ge:strat:kn:qu:e1}
    \E\Vnormlp{\txdfAg-\xdf}^2\leq \ssY^{-1} \{\tfrac{32}{7\cpen\rWc^{2}} + \tfrac{16}{7\rWc}\} + \tfrac{2}{\rWc\cpen}\Vnormlp{\xdf_{\underline{0}}}^2\Ind{\{\mDi>1\}} \exp\big(-\tfrac{3\rWc\cpen}{14}n\daRa{\mdDi}{(\xdf)}\big)\\
    \hfill + 2\Vnormlp{\xdf_{\underline{0}}}^2\Ind{\{\mDi>1\}} \P\big(\Vnormlp{\txdfPr[\mdDi] - \xdfPr[\mdDi]}^2 \geq \tfrac{\cpen}{7} \daRa{\mdDi}{(\xdf)} \big) + \tfrac{2}{7} \penSv[\pDi]  + 2 \Vnormlp{\xdf_{\underline{0}}}^2 \bias[\mDi]^2(\xdf) \\
     +2\sum_{\Di\in\nset{\pdDi,n}}\E\vectp{\Vnormlp{\txdfPr-\xdfPr}^2-\tfrac{1}{7}\penSv}\\
    +\tfrac{2}{7}\sum_{\Di\in\nset{\pdDi,\ssY}}\penSv
    \P\big(\Vnormlp{\txdfPr-\xdfPr}^2\geq\tfrac{1}{7}\penSv\big)
  \end{multline}
\end{te}
% ....................................................................
% Outline Oracle optimality
% ....................................................................
\begin{te}
 The next result can be directly deduced from \nref{freq:ge:strat:kn:qu:re:SrWe:ag} by letting $\rWc\to\infty$.
 However, we think the direct proof given in \nref{pro:freq:ge:strat:kn} provides an interesting illustration of the values $\pDi,\mDi\in\nset{1,\ssY}$ as defined in \eqref{freq:ge:strat:kn:qu:de:mDipDi}.
\end{te}
% ....................................................................
% <<Upper bound random weights>>
% ....................................................................
\begin{lm}\label{freq:ge:strat:kn:qu:re:SrWe:ms}
Consider the data-driven model selection weights $\msWe[]$ as in \eqref{freq:ge:shape:kn:de:msWe}.
Under definition \nref{freq:ge:shape:kn:de:pen:oo} for any $\mdDi,\pdDi\in\nset{1,\ssY}$ and associated $\pDi,\mDi\in\nset{1,n}$ as in \eqref{freq:ge:strat:kn:qu:de:mDipDi} hold
\begin{resListeN}[]
  \item\label{freq:ge:strat:kn:qu:re:SrWe:ms:i}
    $\msWe[](\nsetro{1,\mDi})\Ind{\{\Vnormlp{\theta_{n, \overline{\mdDi}} - \xdfPr[\mdDi]}^2
      <\cpen\daRa{\mdDi}{(\xdf)}/7\}}=0$;
  \item\label{freq:ge:strat:kn:qu:re:SrWe:ms:ii}
    $\sum_{\Di\in\nsetlo{\pDi,\ssY}}\penSv\msWe\Ind{\{\Vnormlp{\txdfPr-\xdfPr}^2<\penSv/7\}}=0$.
  \end{resListeN}
  \reEnd
\end{lm}
% ....................................................................
% <<Te Sum MS Random weights>>
% ....................................................................
\begin{te}
We combine again the upper bound in \nref{freq:ge:strat:kn:co:agg} and the bounds given in \nref{freq:ge:strat:kn:qu:re:SrWe:ms}.
Clearly, due to \nref{freq:ge:strat:kn:qu:re:SrWe:ms} we have
$\E\msWe[](\nsetro{1,\mDi})=\P\big(\Vnormlp{\txdfPr[\mdDi]-\xdfPr[\mdDi]}^2\geq\cpen\daRa{\mdDi}{(\xdf)}/7\big)$
and, hence from \eqref{freq:ge:strat:kn:co:agg} follows immediately
  \begin{multline}\label{freq:ge:strat:kn:qu:e2}
\E\Vnormlp{\txdfPr[\hDi]-\xdf}^2\leq 2\sum\nolimits_{\Di\in\nset{\pdDi,\ssY}}\E\vectp{\Vnormlp{\txdfPr-\xdfPr}^2-\tfrac{1}{7}\penSv}\\
\hfill \tfrac{2}{7}\penSv[\pDi] + 2\Vnormlp{\xdf_{\underline{0}}}^2\bias[\mDi]^2(\xdf) + 2\Vnormlp{\xdf_{\underline{0}}}^2\Ind{\{\mDi>1\}}\P\big(\Vnormlp{\txdfPr[\mdDi]-\xdfPr[\mdDi]}^2\geq\tfrac{\cpen}{7}\daRa{\mdDi}{(\xdf)}\big)\\
+\tfrac{2}{7}\sum_{\Di\in\nset{\pdDi,\ssY}}\penSv\P\big(\Vnormlp{\txdfPr-\xdfPr}^2\geq\tfrac{1}{7}\penSv\big)
\end{multline}
The deviations of the last three terms in the last display \eqref{freq:ge:strat:kn:qu:e2} and also in
\eqref{freq:ge:strat:kn:qu:e1} we bound by exploiting usual concentration inequalities which depend on the model considered.
We hence formulate this property as the assumption to be verified in order to use this strategy.
\end{te}

\begin{as}\label{freq:ge:strat:kn:qu:as}
Remind that we defined $\oiSv=\tfrac{1}{\Di}\sum_{s\in\nset{1,\Di}}\iSv[s]$,

$\miSv= \max\{\iSv[s],s\in\nset{1,\Di}\}$, $\cmSv\geq1$ and $\DipenSv=\cmSv\Di \miSv$.
Assume that there are numerical constants $(\cst{i})_{1 \in \llbracket 1, 11 \rrbracket}$, such that for all $\ssY\in\Nz$ and $\Di\in\nset{1,\ssY}$ holds
  \begin{resListeN}[]
  \item\label{freq:ge:strat:kn:qu:as:i}
    $\E \vectp{\Vnormlp{\txdfPr-\xdfPr}^2 - 12\tfrac{\DipenSv}{\ssY}}  \leq 
    \cst{1} \bigg[\tfrac{\cst{2} \, \miSv}{\ssY}\exp\big(-\cmSv\Di\cst{3} \big)+\tfrac{\cst{4}\Di\miSv}{n^2}\exp\big(- \cst{5} \sqrt{n\cmSv}\big) \bigg]$
  \item\label{freq:ge:strat:kn:qu:as:ii}
    $\P\big(\Vnormlp{\txdfPr-\xdfPr}^2 \geq 12\DipenSv\ssY^{-1}\big)\leq 
    \cst{6} \bigg[\exp\big(-\cst{7} \cmSv\Di\big)
    +\exp\big(-\cst{8} \sqrt{\ssY\cmSv}\big)\bigg]$
  \item\label{freq:ge:strat:kn:qu:as:iii}
    $\P\big(\Vnormlp{\txdfPr-\xdfPr}^2 \geq 12\daRa{\Di}{(\xdf,\Lambda)}\big)\leq 
    \cst{9} \bigg[\exp\big(\tfrac{-\cst{10} \ssY\daRa{\Di}{(\xdf,\Lambda)}}{\miSv}\big)
    +\exp\big(\tfrac{-\cst{11} \ssY\sqrt{\daRa{\Di}{(\xdf,\Lambda)}}}{\sqrt{\Di\miSv}}\big)\bigg]$
  \end{resListeN}
\end{as}

\begin{te} Consider now the partially data-driven aggregation of the
  orthogonal series estimators using either  aggregation weights $\rWe[]$
  as in \eqref{freq:ge:shape:kn:we} or model selection weights $\msWe[]$ as in \eqref{freq:ge:shape:kn:de:msWe}
  combining \nref{freq:ge:strat:kn:qu:as} and the upper bound given
  in \eqref{freq:ge:strat:kn:qu:e1}  or \eqref{freq:ge:strat:kn:qu:e2} we obtain the next result, which proof is immediate and we omit it.
\end{te}

\begin{lm}\label{ak:re:nd:rest}
Assume that \nref{freq:ge:strat:kn:qu:as} holds true and use the penalty described in \nref{freq:ge:shape:kn:de:pen:oo} with $\kappa \geq 84$ so that $\penSv/7 \geq 12 n^{-1} \Delta_{\Lambda}(m)$for any $m$ in $\llbracket 1, n \rrbracket$.
Then, for all $\ssY\in\Nz$ and $\Di\in\nset{1,\ssY}$ hold
  \begin{resListeN}[]
  \item\label{ak:re:nd:rest1} let $\Di_{\cst{3}}:=\floor{  3(2/\cst{3})^2}$ and $\ssY_{\cst{5}}:={15(\cst{5})^{-4}}$ then\\ 
    $\sum_{\Di=1}^{\ssY}\E
    \vectp{\Vnormlp{\txdfPr-\xdfPr}^2-\penSv/7}
    \leq \cst{1}\ssY^{-1}\big[\tfrac{2 \cst{2}}{\cst{3}}\miSv[\Di_{\cst{3}}] + \cst{4} \ssY_{\cst{5}} \miSv[\ssY_{\cst{5}}]\big]$
  \item\label{ak:re:nd:rest2} let
    $\Di_{\cst{7}}:=\floor{3(2 / \cst{7})^2}$ and
    $\ssY_{\cst{8}}:=15(3/\cst{8})^4$ then
    \begin{multline*}
    \sum_{\Di=1}^{\ssY}\penSv/7\P\big(\Vnormlp{\txdfPr-\xdfPr}^2\geq\penSv/7\big)\\
    \leq\cst{6} n^{-1} \big[\miSv[\Di_{\cst{7}}]^2\Di_{\cst{7}}^2+\miSv[\ssY_{\cst{8}}]^2 \ssY_{\cst{8}}^{2}\big]
    \end{multline*}
  \item\label{ak:re:nd:rest3} 
  $\P\big(\Vnormlp{\theta_{n, \overline{m^{\dagger}_{-}}}-\theta^{\circ}_{\overline{m^{\dagger}_{n}}}}^2 \geq 12\daRaS{\mdDi}{\xdf,\Lambda}\big)\leq 
    \cst{9} \big[\exp\big(\tfrac{-\cst{10}\ssY\daRaS{\mdDi}{\xdf,\Lambda}}{\Lambda_{+}(\mdDi)}\big)+(\cst{8})^{-2}\ssY^{-1}\big]$
  \end{resListeN}
  \reEnd
\end{lm}


Injecting \nref{ak:re:nd:rest} in either \nref{freq:ge:strat:kn:qu:e1} or \nref{freq:ge:strat:kn:qu:e2} we directly obtain the following result.

\begin{lm}\label{freq:ge:strat:kn:qu:ub}
Assume that \nref{freq:ge:strat:kn:qu:as} holds true.
Consider the penalty sequence $\penSv$ as in \nref{freq:ge:shape:kn:de:pen:oo} with numerical constant $\cpen \geq 84$.
Let $\widehat{\theta}^{(\eta)}$ be an aggregation estimator using either the aggregation weights \nref{freq:ge:shape:kn:we} or the model selection weights \nref{freq:ge:shape:kn:de:msWe}.
Let $\ssY_{\cst{5}}$, $n_{\cst{8}}$, $m_{\cst{3}}$, and $m_{\cst{7}}$ be as in \nref{{ak:re:nd:rest}}.
Then, there is a finite numerical constant $\cst{}$ such that for any $\mdDi$, $\pdDi$ and associated $\mDi$ and $\pDi$ as in \nref{freq:ge:strat:kn:qu:de:mDipDi} holds
	\begin{multline}\label{freq:ge:strat:kn:qu:ub:e1}
\E\Vnormlp{\widehat{\theta}^{(\eta)}-\xdf}^2\leq \tfrac{2}{7}\penSv[\pDi] + 2\Vnormlp{\xdf_{\underline{0}}}^2\bias[\mDi]^2(\xdf) + \cst{}\Vnormlp{\xdf_{\underline{0}}}^2\Ind{\{\mDi>1\}} \big[\exp\big(-\cst{10}\delta_{\Lambda}(\mdDi) \mdDi\big)\big]\\
 + \cst{}\big[\Vnormlp{\xdf_{\underline{0}}}^2\Ind{\{\mDi>1\}} + \miSv[\Di_{\cst{7}}]^2\Di_{\cst{7}}^2+\miSv[\ssY_{o}]^2\big]\ssY^{-1}.
\end{multline} 
\reEnd
\end{lm}
% ....................................................................
% <<Re upper bound ag>>
% ....................................................................
%\begin{lm}\label{freq:ge:strat:kn:qu:ub}
%Under \nref{freq:ge:strat:kn:qu:as}, consider the penalty sequence $\penSv:=\penD$, $\Di\in\nset{1,n}$, as in \nref{freq:ge:shape:kn:de:pen:oo}.
%Let $\widehat{\theta}^{(\eta)}=\sum_{\Di=1}^{\ssY}\We\txdfPr$ be an aggregation of the orthogonal series estimators, using either aggregation weights $\rWe[]$ as in \eqref{freq:ge:shape:kn:we}, or model selection weights $\msWe[]$ as in \eqref{freq:ge:shape:kn:de:msWe}.
%  
%There is a finite numerical constant $\cst{}>0$ such that for any $\mdDi,\pdDi\in\nset{1,n}$ and associated $\pDi,\mDi\in\nset{1,n}$ as defined in \eqref{freq:ge:strat:kn:qu:de:mDipDi} hold
%\begin{equation}\label{freq:ge:strat:kn:qu:ub:e1}
%\E\Vnormlp{\txdfPr[\hDi]-\xdf}^2\leq \cst{} \daRa{\dDi}{(\xdf)} + \tfrac{2}{7}\penSv[\pDi] + 2\Vnormlp{\xdf_{\underline{0}}}^2\bias[\mDi]^2(\xdf)
%\end{equation}
%\reEnd
%\end{lm}
% ....................................................................
% <<Te upper bound ag p np>>
% ....................................................................
\begin{te} The last bound allows us to derive an upper bound of the
  risk for  data-driven aggregated estimator  in the two cases
  \ref{oo:xdf:p} and \ref{oo:xdf:np} introduced in \nref{bm:ak}.
\end{te}
% ....................................................................
% <<Re upper bound ag p np>>
% ....................................................................
\begin{thm}\label{freq:ge:strat:kn:qu:pnp}
Under \nref{freq:ge:strat:kn:qu:as}, consider the penalty sequence $\penSv:=\penD$, $\Di\in\nset{1,n}$, as in \nref{freq:ge:shape:kn:de:pen:oo} with numerical constant $\cpen \geq 84$.
Let $\txdfAg[{\erWe[]}]=\sum_{\Di=1}^{\ssY}\We\txdfPr$ be an aggregation of the orthogonal series estimators, using either aggregation weights $\rWe[]$ as in \eqref{freq:ge:shape:kn:we}, or model selection weights $\msWe[]$ as in \eqref{freq:ge:shape:kn:de:msWe}.
\begin{Liste}[]
\item[\mylabel{ak:ag:ub:pnp:p}{\dgrau\bfseries{(p)}}]Assume there is $K\in\Nz$
  with   $1\geq \bias[{[K-1] }](\xdf)>0$ and $\bias[\Di](\xdf)=0$. For
  $K>0$ let
  $ c_{\xdf}:=\tfrac{\Vnormlp{\xdf_{\underline{0}}}^2+4\cpen}{\Vnormlp{\xdf_{\underline{0}}}^2\bias[{[K-1]}]^2(\xdf)}>1$ and
  $\ssY_{\xdf}:=\gauss{c_{\xdf}\DipenSv[K]}\in\Nz$. If
  $\ssY\in\nset{1,\ssY_{\xdf}}$ then set $\sDi{\ssY}:=\Di_{\cst{3}}\log(\ssY)$, and otherwise if
  $\ssY>\ssY_{\xdf}$ then set
  $\sDi{\ssY}:=\max\{\Di\in\nset{K,\ssY}:\ssY>c_{\xdf}\DipenSv\}$
  where the defining set contains $K$ and thus is not empty.
There is a finite constant $\cst{\xdf,\Lambda}$
given in \eqref{ak:ag:ub:pnp:p7} depending only on $\xdf$ and $\Lambda$ such that for all $n\in\Nz$ holds
\begin{equation}\label{ak:ag:ub:pnp:e1}
  \nRi{\txdfAg[{\erWe[]}]}{\xdf,\Lambda}
  % \E\Vnormlp{\txdfAg[{\erWe[]}]-\xdf}^2
  \leq
  \cst{}\Vnormlp{\xdf_{\underline{0}}}^2\big[
  \ssY^{-1}\vee\exp\big(-\cst{10}\cmiSv[\sDi{\ssY}]\sDi{\ssY}\big)\big]
  + \cst{\xdf,\Lambda}\ssY^{-1}.
\end{equation}
\item[\mylabel{ak:ag:ub:pnp:np}{\dgrau\bfseries{(np)}}] Assume that
  $\bias(\xdf)>0$ for all  $\Di\in\Nz$.
There is a finite finite constant $\cst{\xdf,\Lambda}$ given in
\eqref{ak:ag:ub:pnp:p8} depending only $\xdf$ and $\Lambda$ such that for all
$\ssY\in\Nz$  holds 
 \begin{equation}\label{ak:ag:ub:pnp:e2}
   \nRi{\txdfAg[{\erWe[]}]}{\xdf,\Lambda}
   % \E\Vnormlp{\txdfAg[{\erWe[]}]-\xdf}^2
    \leq 
   \cst{}(\Vnormlp{\xdf_{\underline{0}}}^2\vee1)\min_{\Di\in\nset{1,\ssY}}\big[\dRa{\Di}{\xdf,\Lambda}\vee\exp\big(-\cst{10}\cmiSv\Di\big)\big]\\
   +\cst{\xdf,\Lambda}\ssY^{-1}.
\end{equation}
\end{Liste}  
\end{thm}

Hence, using \nref{freq:ge:strat:kn:qu:pnp} gives us the following result.
% ....................................................................
% <<Re upper bound ag p np>>
% ....................................................................
\begin{cor}\label{ge:ak:ag:ub2:pnp}
  Let the assumptions of \nref{freq:ge:strat:kn:qu:pnp} be satisfied.
  \begin{Liste}[]
  \item[\mylabel{ge:ak:ag:ub2:pnp:p}{\dgrau\bfseries{(p)}}]
    If in addition
    \begin{inparaenum}\item[\mylabel{ge:ak:ag:ub2:pnp:pc}{{\dgrau\bfseries(A1)}}]
      there is $\ssY_{\xdf,\Lambda}\in\Nz$ such that
      $\cmiSv[\sDi{\ssY}]\sDi{\ssY}\geq (\cst{10})^{-1}(\log\ssY)$ for all
      $\ssY\geq \ssY_{\xdf,\Lambda}$
    \end{inparaenum}
    holds true, then there is a constant $\cst{\xdf,\Lambda}$ depending
    only on $\xdf$ and $\Lambda$ such that for all $n\in\Nz$ holds
    $\nRi{\txdfAg[{\erWe[]}]}{\xdf,\Lambda} \leq
    \cst{\xdf,\Lambda}\ssY^{-1}$.
  \item[\mylabel{ge:ak:ag:ub2:pnp:np}{\dgrau\bfseries{(np)}}]
    If in addition
    \begin{inparaenum}\item[\mylabel{ge:ak:ag:ub2:pnp:npc}{{\dgrau\bfseries(A2)}}]
      there is  $\ssY_{\xdf,\Lambda}\in\Nz$ such that
      
      $\aDi{\ssY}(\xdf)\cmSv[\aDi{\ssY}(\xdf)]\geq (\cst{10})^{-1} \vert \log\naRa{(\xdf,\Lambda)} \vert $
      for all $\ssY\geq \ssY_{\xdf,\Lambda}$
    \end{inparaenum}
    holds true, then there is a constant $\cst{\xdf,\Lambda}$ depending
    only on $\xdf$ and $\Lambda$ such that $\nRi{\txdfAg[{\erWe[]}]}{\xdf,\Lambda}
    \leq \cst{\xdf,\Lambda}\naRa{(\xdf,\Lambda)}$ for all $n\in\Nz$ holds true.
  \end{Liste}  
\end{cor}

% ....................................................................
% <<Il upper bound ag np>>
% ....................................................................
\begin{il}\label{ak:ag:ub:pnp:il}Let us briefly illustrate the last
  results. In case \ref{ak:ag:ub:pnp:p} the partially data-driven aggregation leads
  to an estimator attaining the parametric oracle rate (see
  \nref{oo:rem:ora}), if the additional assumption
  \ref{ge:ak:ag:ub2:pnp:pc} is satisfied.  Consider  the two cases \ref{il:edf:o} and
  \ref{il:edf:s} for $\edf$ as in \nref{il:oo}:
\begin{Liste}[]
\item[\mylabel{ak:il:edf:o}{\dg\bfseries{(o)}}]   $1\approx\DipenSv[\sDi{\ssY}]\ssY^{-1}\approx(\sDi{\ssY})^{2a+1}\ssY^{-1}$   implies $\sDi{\ssY}\approx\ssY^{1/(2a+1)}$ and
    $\sDi{\ssY}\cmSv[\sDi{\ssY}]\approx\ssY^{1/(2a+1)}$
\item[\mylabel{ak:il:edf:s}{\dg\bfseries{(s)}}]  $\ssY\approx\DipenSv[\sDi{\ssY}]\approx
    (\sDi{\ssY})^{1+4a}\exp((\sDi{\ssY})^{2a})$ implies
    $\sDi{\ssY}\approx(\log
    \ssY-\tfrac{1+4a}{2a}\log\log\ssY)^{1/(2a)}$ and 
    $\sDi{\ssY}\cmSv[\sDi{\ssY}]\approx (\log \ssY)^{2+1/(2a)}$.
  \end{Liste}
  Clearly in both cases \ref{ak:il:edf:o} and \ref{ak:il:edf:s}, the
  additional condition \ref{ge:ak:ag:ub2:pnp:pc} of \nref{ge:ak:ag:ub2:pnp}
  holds true. Therefore, in this situation the aggregated estimator
  attains the oracle rate.  On the other hand side, in case
  \ref{ge:ak:ag:ub2:pnp:np} the partially data-driven aggregation leads
  to an estimator attaining the rate $\naRa{(\xdf,\Lambda)}$ (see
  \nref{oo:rem:ora}), if the additional assumption
  \ref{ge:ak:ag:ub2:pnp:npc} is satisfied. Otherwise, the upper bound
  faces a deterioration of the rate, which we illustrate considering as
  in \nref{freq:ge:strat:kn:qu:il:rate} usual behaviour \ref{freq:ge:strat:kn:qu:il:rate:np:oo},
  \ref{freq:ge:strat:kn:qu:il:rate:np:os} and \ref{freq:ge:strat:kn:qu:il:rate:np:so} for the
  sequences $\Nsuite[\Di]{\bias[\Di](\xdf)}$ and
  $\Nsuite[\Di]{\iSv[\Di]}$. In case \ref{freq:ge:strat:kn:qu:il:rate:np:oo},
  \ref{freq:ge:strat:kn:qu:il:rate:np:os} and \ref{freq:ge:strat:kn:qu:il:rate:np:so} only with
  $p<1/2$ the assumption \ref{ge:ak:ag:ub2:pnp:npc} is satisfied, and
  $\naRa{(\xdf,\Lambda)}$ equals the oracle rate $\oRa{\xdf,\Lambda}$ (cf.
  \ref{freq:ge:strat:kn:qu:il:rate:np:oo} \ref{freq:ge:strat:kn:qu:il:rate:np:oo},
  \ref{freq:ge:strat:kn:qu:il:rate:np:os} and \ref{freq:ge:strat:kn:qu:il:rate:np:so}). Thereby, the
  partially data-driven aggregation leads to an estimator attaining
  the oracle rate $\oRa{\xdf,\Lambda}$. In case \ref{freq:ge:strat:kn:qu:il:rate:np:os}
  with $p\geq1/2$ the assumption \ref{ge:ak:ag:ub2:pnp:npc} is not
  satisfied. However, with
  $\sDi{\ssY}:=\Di_{\cst{3}} \vert \log\naRaS{\xdf,\Lambda} \vert \approx(\log\ssY)$ holds
  $\min_{\Di\in\nset{1,\ssY}}\big[\dRa{\Di}{\xdf,\Lambda}\vee\exp\big(\tfrac{-\cmiSv\Di}{\Di_{\cst{3}}}\big)\leq\daRa{\sDi{\ssY}}{\xdf,\Lambda}\approx(\log\ssY)^{2a+1}\ssY^{-1}$.
  In this situation the rate of the partially data-driven estimator
  $\txdfAg[{\erWe[]}]$ features a deterioration by a logarithmic factor
  $(\log\ssY)^{(2a+1)(1-1/(2p))}$ compared to the oracle rate, i.e.
  $\daRaS{\sDi{\ssY}}{\xdf,\Lambda}\approx(\log\ssY)^{2a+1}\ssY^{-1}$ versus
  $\oRaS{\xdf,\Lambda}\approx(\log\ssY)^{(2a+1)/(2p)}\ssY^{-1}$.\ilEnd
\end{il}

\subsubsection{Maximal risk bounds}\label{freq:ge:strat:kn:ma}
\begin{te}
  By applying \nref{freq:ge:strat:kn:co:agg}, we derive bounds for the maximal risk over ellipsoids $\rwCxdf$ of the aggregated estimator $\txdfAg[{\erWe[]}]$ using either aggregation weights $\rWe[]$ as in \eqref{freq:ge:shape:kn:we} or model selection weights $\msWe[]$ as in \eqref{freq:ge:shape:uk:we}.
  Therefore, we aim next to control the second and third right hand side term in \eqref{freq:ge:strat:kn:co:agg:e1} uniformly over $\rwCxdf$.
  Keeping the definition \eqref{oo:de:mra} of $\dRa{\Di}{\xdfCw[],\Lambda}$ in mind it holds $\xdfCr^2\dRa{\Di}{\xdfCw[],\Lambda} \geq \Vnormlp{\xdf_{\underline{0}}}^2 \bias^2(\xdf)$ uniformly for all $\xdf \in \rwCxdf$ and for all $\Di\in\Nz$.
  The proofs for the results displayed here can be found in \nref{pro:freq:ge:strat:kn:ma}.
  We then gives the following definition for the sequence which we want to prove to be an upper bound for the maximal risk of the aggregation estimator.
  Note that in this case we use $\DipenSv$ and $\penSv$ as defined in \nref{freq:ge:shape:kn:de:pen:oo} and hence the rates for the quadratic as well as the maximal risk are obtained for the same estimator.
  
\begin{de}\label{freq:ge:strat:kn:ma:de:rate}
  Let be the following family of sequences,
  $\daRa{\Di}{(\xdfCw[])}:=\daRa{\Di}{(\xdfCw[],\Lambda)}:=[\xdfCw^2\vee \DipenSv\,\ssY^{-1}]$.
  Then it holds for all $\Di$ in $\nset{1,\ssY}$ and $\xdf$ in $\rwCxdf$
  \begin{equation}\label{freq:ge:strat:kn:ma:de:rate:e1}
    [\xdfCr^2+\cpen]\daRa{\Di}{(\xdfCw[])}\geq\big[\Vnormlp{\xdf_{\underline{0}}}^2\bias^2(\xdf)\vee\penSv\big]
\end{equation}
Considering the following specific case, we aim to show that it describes an upper bound for the maximal risk over $\rwCxdf$ for our aggregation estimator,
\begin{multline*}
\aDi{\ssY}(\xdfCw[]):=\argmin\Nset[\Di\in\Nz]{\daRa{\Di}{(\xdfCw[],\Lambda)}}\in\nset{1,\ssY}\\
    \naRa{(\xdfCw[])}:=\naRa{(\xdfCw[],\Lambda)}:=\min\Nset[\Di\in\Nz]{\daRa{\Di}{(\xdfCw[],\Lambda)}}; \text{ with } \daRa{\aDi{\ssY}(\xdfCw[])}{(\xdfCw[],\Lambda)}=\naRa{(\xdfCw[],\Lambda)}
    \end{multline*}
\assEnd
\end{de}

\end{te}
% ....................................................................
% <<Il upper bound oo>>
% ....................................................................
\begin{il}\label{ak:mrb:ass:il}
Let us illustrate \nref{freq:ge:strat:kn:ma:de:rate} considering as in \nref{il:mm} usual behaviour \ref{il:mm:oo}, \ref{il:mm:so} and \ref{il:mm:os} for the sequences $\Nsuite[\Di]{\xdfCw[(\Di)]}$ and $\Nsuite[\Di]{\iSv[\Di]}$:
  \begin{Liste}[]
  \item[\mylabel{ak:mrb:ass:il:oo}{\dg\bfseries{[o-o]}}] Since
   $\DipenSv\approx\Di^{2a+1}$
    follows $\aDi{\ssY}(\xdfCw[])\approx\ssY^{1/(2p+2a+1)}$,
    $\cmiSv[\aDi{\ssY}({\xdfCw[]})]\aDi{\ssY}(\xdfCw[])\approx\ssY^{1/(2p+2a+1)}$,
    $\naRa{(\xdfCw[],\Lambda)}\approx\ssY^{-2p/(2p+2a+1)}$ and $ \vert \log\naRa{(\xdfCw[],\Lambda)} \vert \approx(\log\ssY)$.
 \item[\mylabel{ak:mrb:ass:il:os}{\dg\bfseries{[o-s]}}]
    Since $\DipenSv\approx\Di^{1+4a}\exp(\Di^{2a})$ follows  $\aDi{\ssY}(\xdfCw[])\approx(\log\ssY)^{1/(2a)}$, $\cmiSv[\aDi{\ssY}({\xdfCw[]})]\aDi{\ssY}(\xdfCw[])\approx(\log\ssY)^{2+1/(2a)}$, 
    $\naRa{(\xdfCw[],\Lambda)}\approx(\log\ssY)^{-p/a}$ and $ \vert \log\naRa{(\xdfCw[],\Lambda)} \vert \approx(\log\log\ssY)$.
 \item[\mylabel{ak:mrb:ass:il:so}{\dg\bfseries{[s-o]}}]  Since
   $\DipenSv\approx\Di^{2a+1}$
    follows  $\aDi{\ssY}(\xdfCw[])\approx(\log\ssY)^{1/(2p)}$, $\cmiSv[\aDi{\ssY}({\xdfCw[]})]\aDi{\ssY}(\xdfCw[])\approx(\log\ssY)^{1/(2p)}$,
    $\naRa{(\xdfCw[],\Lambda)}\approx(\log\ssY)^{(2a+1)/(2p)}\ssY^{-1}$
    and     $ \vert \log\naRa{(\xdfCw[],\Lambda)} \vert \approx(\log\ssY)$.\ilEnd
  \end{Liste}
    We note that  in the three cases \ref{ak:mrb:ass:il:oo},
  \ref{ak:mrb:ass:il:os} and \ref{ak:mrb:ass:il:so} the rate
  $\naRa{(\xdfCw[],\Lambda)}$ coincide with the
  minimax rate $\mnRa{\xdfCw[],\Lambda}$ derived in \nref{il:mm} \ref{il:mm:oo},
  \ref{il:mm:os} and \ref{il:mm:so}, respectively.\ilEnd 
\end{il}
\begin{te}  
Keeping in mind \eqref{freq:ge:strat:kn:ma:de:rate:e1} for any $\pdDi,\mdDi\in\nset{1,\ssY}$ let us define 
\begin{multline}\label{freq:ge:strat:kn:ma:de:mDipDi}
\mDi:=\min\set{\Di\in\nset{1,\mdDi}: \Vnormlp{\xdf_{\underline{0}}}^2\bias[\Di]^2(\xdf)\leq
  [\xdfCr^2+4\cpen]\dRa{\mdDi}{\xdfCw[]}}\quad\text{and}\\\pDi:=\max\set{\Di\in\nset{\pdDi,\ssY}:
   \penSv \leq 2[3\xdfCr^2+ 2\cpen] \dRa{\pdDi}{\xdfCw[]}}
\end{multline}
where  the defining sets obviously contains $\mdDi$ and $\pdDi$, repsectively, and hence, they are
not empty.
\end{te}
% ....................................................................
% <<Re maximal Sum Random weights>>
% ....................................................................
\begin{lm}\label{ak:re:SrWe:ag:mm}
Consider the data-driven aggregation weights $\rWe[]$
  as in \eqref{freq:ge:shape:kn:we} and the rates described in \nref{freq:ge:strat:kn:ma:de:rate} with
  $\cpen\geq8\log(3e)$  for any
  $\mdDi,\pdDi\in\nset{1,\ssY}$ and associated $\pDi,\mDi\in\nset{1,\ssY}$
  as in \eqref{freq:ge:strat:kn:ma:de:mDipDi} hold
  \begin{resListeN}
  \item\label{ak:re:SrWe:ag:mm:i}
    $\rWe[](\nsetro{1,\mDi})
    \Ind{\setB{\Vnormlp{\txdfPr[\mdDi]-\xdfPr[\mdDi]}^2<\cpen\daRa{\mdDi}{(\xdfCw[])}/7}}\leq
    \tfrac{1}{\rWc\cpen}\Ind{\{\mDi>1\}}
    \exp\big(-\tfrac{3\rWc\cpen}{14} \ssY\daRa{\mdDi}{(\xdfCw[])}\big)$;
  \item\label{ak:re:SrWe:ag:mm:ii}
    $\sum_{\Di\in\nsetlo{\pDi,\ssY}}\penSv\rWe
    \Ind{\{\Vnormlp{\txdfPr[\Di]-\xdfPr[\Di]}^2<\penSv/7\}}
    \leq \ssY^{-1}\{\tfrac{16}{\cpen\rWc^{2}}+ \tfrac{8}{\rWc}\}$.
  \end{resListeN}
  \reEnd
\end{lm}
% ....................................................................
% <<Te maximal Sum Random weights>>
% ....................................................................
\begin{te}
  The next result can also immediately be deduced from
  \nref{ak:re:SrWe:ag:mm} letting $\rWc\to\infty$. On the other hand
  side, a direct proof follows line by line the proof of
  \nref{freq:ge:strat:kn:qu:re:SrWe:ms} using \eqref{freq:ge:strat:kn:ma:de:rate:e1} rather than
  \eqref{freq:ge:strat:kn:qu:de:rate:e1}, and we omit the details.
\end{te}
% ....................................................................
% <<Maximal Upper bound random weights ms>>
% ....................................................................
\begin{lm}\label{ak:re:SrWe:ms:mm}
Consider the data-driven model selection weights $\msWe[]$
  as in \eqref{freq:ge:shape:uk:we}.  Under definition \nref{freq:ge:strat:kn:ma:de:rate} 
  for any $\mdDi,\pdDi\in\nset{1,\ssY}$ and associated
  $\pDi,\mDi\in\nset{1,n}$ as in \eqref{freq:ge:strat:kn:ma:de:mDipDi} hold
  \begin{resListeN}[]
  \item\label{ak:re:SrWe:ms:mm:i}
    $\msWe[](\nsetro{1,\mDi})\Ind{\{\Vnormlp{\txdfPr[\mdDi]-\xdfPr[\mdDi]}^2
      <\cpen\daRa{\mdDi}{(\xdfCw[])}/7\}}=0$;
  \item\label{ak:re:SrWe:ms:mm:ii}
    $\sum_{\Di\in\nsetlo{\pDi,\ssY}}\penSv\msWe\Ind{\{\Vnormlp{\txdfPr-\xdfPr}^2<\penSv/7\}}=0$.
  \end{resListeN}
  \reEnd
\end{lm}
% --------------------------------------------------------------------
% <<Text unifom bounds over ellipsoid>>
% --------------------------------------------------------------------
%\begin{te}
%Keep in mind that $\ydf=\xdf\cdot\edf$.
%We note that uniformly for all $\xdf\in\rwCxdf$ by applying the
%  Cauchy-Schwarz inequality holds   $\Vnormlp[1]{\fydf}\leq
%\Vnorm[{\xdfCw[]}]{\edf}\Vnorm[1/{\xdfCw[]}]{\xdf}\leq \xdfCr\Vnorm[{\xdfCw[]}]{\edf}$. Moreover,
%\nref{re:conc} \ref{re:conc:iii} still holds with
%$\daRa{\mdDi}{(\xdf,\Lambda)}$ replaced by $\daRa{\mdDi}{(\xdfCw[],\Lambda)}$, since
%$\daRa{\mdDi}{(\xdfCw[],\Lambda)}\geq \DipenSv\ssY^{-1}$ for all
%$\ssY,\Di\in\Nz$. Thereby, we obtain the next assertion, and we omit its elementary proof.
%\end{te}


\begin{lm}\label{ak:re:nd:rest:ma}
Assume that \nref{freq:ge:strat:kn:qu:as} holds true and use the penalty described in \nref{freq:ge:shape:kn:de:pen:oo} with $\kappa \geq 84$ so that $\penSv/7 \geq 12 n^{-1} \Delta_{\Lambda}(m)$for any $m$ in $\llbracket 1, n \rrbracket$.
Then, for all $\ssY\in\Nz$ and $\Di\in\nset{1,\ssY}$ hold
  \begin{resListeN}[]
  \item\label{ak:re:nd:rest1:ma} let $\Di_{\cst{3}}:=\floor{  3(2/\cst{3})^2}$ and $\ssY_{\cst{5}}:={15(\cst{5})^{-4}}$ then\\ 
    $\sup\limits_{\xdf \in \Theta(\mathfrak{a}, r)}\sum_{\Di=1}^{\ssY}\E
    \vectp{\Vnormlp{\txdfPr-\xdfPr}^2-\penSv/7}
    \leq \cst{1}\ssY^{-1}\big[\miSv[\Di_{\cst{3}}] + \miSv[\ssY_{\cst{5}}]\big]$
  \item\label{ak:re:nd:rest2:ma} let
    $\Di_{\cst{7}}:=\floor{3(2 / \cst{7})^2}$ and
    $\ssY_{\cst{8}}:=15(3/\cst{8})^4$ then\\
    $\sup\limits_{\xdf \in \Theta(\mathfrak{a}, r)}\sum_{\Di=1}^{\ssY}\tfrac{\penSv}{7}\P\big(\Vnormlp{\txdfPr-\xdfPr}^2
    \geq\tfrac{\penSv}{7}\big)\leq\cst{6} n^{-1} \big[\miSv[\Di_{\cst{7}}]^2\Di_{\cst{7}}^2+\miSv[\ssY_{\cst{8}}]^2\big]$
  \item\label{ak:re:nd:rest3:ma} 
  $\sup\limits_{\xdf \in \Theta(\mathfrak{a}, r)}\P\big(\Vnormlp{\theta_{n, \overline{m^{\dagger}_{-}}}-\theta^{\circ}_{\overline{m^{\dagger}_{n}}}}^2 \geq 12\daRaS{\mdDi}{\xdf,\Lambda}\big)\leq 
    \cst{9} \big[\exp\big(\tfrac{-\cst{10}\ssY\daRaS{\mdDi}{\xdf,\Lambda}}{\Lambda_{+}(\mdDi)}\big)+\ssY^{-1}\big]$
  \end{resListeN}
  \reEnd
\end{lm}
% --------------------------------------------------------------------
% <<Te penalty>>
% --------------------------------------------------------------------
\begin{te}
  Consider now the partially data-driven aggregation of the orthogonal series estimators using either aggregation weights $\rWe[]$ as in \eqref{freq:ge:shape:kn:we} or model selection weights $\msWe[]$ as in \eqref{freq:ge:shape:uk:we}.
  From \eqref{freq:ge:strat:kn:co:agg:e1}, combining \nref{ak:re:SrWe:ag:mm} and \nref{ak:re:SrWe:ms:mm} we obtain by replacing $\daRa{\mdDi}{(\xdf)}$ by $\daRa{\mdDi}{(\xdfCw[])}$ upper bounds similar to \eqref{freq:ge:strat:kn:qu:e1} and  \eqref{freq:ge:strat:kn:qu:e2}, respectively.
  Those upper bounds together with \nref{ak:re:nd:rest:ma} allow us to show the next assertion \nref{ak:ag:ub:mm}.
\end{te}
% ....................................................................
% <<Re upper bound ag>>
% ....................................................................
\begin{lm}\label{ak:ag:ub:mm}
Assume that \nref{freq:ge:strat:kn:qu:as} holds true.

Consider the penalty sequence $\penSv:=\penD$,
  $\Di\in\nset{1,n}$, as in \nref{freq:ge:shape:kn:de:pen:oo} with numerical
  constant $\cpen$ to be specified depending on the model.
  
  Let $\widehat{\theta}^{(\eta)}=\sum_{\Di=1}^{\ssY}
  \We\txdfPr$ be an aggregation of the orthogonal series estimators using either
  aggregation weights $\rWe[]$
  as in \eqref{freq:ge:shape:kn:we} or model selection weights $\msWe[]$
  as in \eqref{freq:ge:shape:uk:we}.
  There is a finite numerical constant $\cst{}>0$ such that for any
  $\xdf\in\rwCxdf$, $\mdDi,\pdDi\in\nset{1,n}$ and associated $\pDi,\mDi\in\nset{1,n}$ as defined in \eqref{freq:ge:strat:kn:ma:de:mDipDi} hold
    \begin{equation}\label{ak:ag:ub:mm:e1}
    \E\Vnormlp{\txdfAg[{\erWe[]}]-\xdf}^2\leq \tfrac{2}{7}\penSv[\pDi]
    +2\Vnormlp{\xdf_{\underline{0}}}^2\bias[\mDi]^2(\xdf)
    %\\\hfill
 + \cst{} \naRa{(\xdfCw[])}.
  \end{equation}
  \reEnd
\end{lm}
% ....................................................................
% <<Te upper bound ag p np>>
% ....................................................................
\begin{te}
The last bound allows us to derive an upper bound of the maximal risk over the ellipsoid $\rwCxdf$ for the partially data-driven aggregated estimator in the case \ref{oo:xdf:np} introduced in \nref{bm:ak}.
\end{te}
% ....................................................................
% <<Re upper bound ag p np>>
% ....................................................................
\begin{thm}\label{ak:ag:ub:pnp:mm}
Assume that \nref{freq:ge:strat:kn:qu:as} holds true and consider the penalty sequence $\penSv:=\penD$, $\Di\in\nset{1,n}$, as in \nref{freq:ge:shape:kn:de:pen:oo}.
Let $\txdfAg[{\erWe[]}]=\sum_{\Di=1}^{\ssY} \We\txdfPr$ be an aggregation of the orthogonal series estimators using either aggregation weights $\rWe[]$ as in \eqref{freq:ge:shape:kn:we} or model selection weights $\msWe[]$ as in \eqref{freq:ge:shape:uk:we}. % Let $\Di_{\edf,\xdfCr}:=\floor{3(800\Vnorm[{\xdfCw[]}]{\edf}\xdfCr)^2}$ and
    % $ \ssY_{o}:=15({300})^4$.
There is a finite constant $\cst{\xdfCw[],\xdfCr,\Lambda}$ given in
\eqref{ak:ag:ub:pnp:p8} depending only on $\xdfCw[]$, $\xdfCr$ and $\Lambda$ such that for all
$\ssY\in\Nz$ and for all $\sDi{\ssY}\in\nset{\aDi{\ssY}(\xdfCw[]),\ssY}$  with
 $\aDi{\ssY}(\xdfCw[])\in\nset{1,n}$ as in \nref{freq:ge:strat:kn:ma:de:rate} holds 
 \begin{equation}\label{ak:ag:ub:pnp:mm:e1}
 \nRi{\txdfAg[{\erWe[]}]}{\rwCxdf,\Lambda}
   %\sup_{\xdf\in\rwCxdf}\E\Vnormlp{\txdfAg[{\erWe[]}]-\xdf}^2% \leq
    \leq 
   \cst{}(\xdfCr^2\vee1)\min_{\Di\in\nset{1,\ssY}}\big[\daRa{\Di}{(\xdfCw[],\Lambda)}\vee\exp\big(-\cst{10}\cmiSv\Di\big)]\big)\big]
   +\cst{\xdfCw[],\xdfCr,\Lambda}\ssY^{-1}.
\end{equation}
\reEnd
\end{thm}

\begin{cor}\label{ge:ak:ag:ub2:pnp:mm}
  Let the assumptions of \nref{ak:ag:ub:pnp:mm} be satisfied.  If in
  addition
  \begin{inparaenum}\item[\mylabel{ge:ak:ag:ub2:pnp:mm:c}{{\dgrau\bfseries(A)}}]
    there is $\ssY_{\xdfCw[],\xdfCr,\Lambda}\in\Nz$  such that
    $\aDi{\ssY}({\xdfCw[]})\cmSv[\aDi{\ssY}({\xdfCw[]})]\geq(\cst{10})^{-1} \vert \log\naRa{(\xdfCw[])} \vert $
    for all $\ssY\geq \ssY_{\xdfCw[],\xdfCr,\Lambda}$
  \end{inparaenum}
  holds true, then there is a constant $\cst{\xdfCw[],\xdfCr,\Lambda}$
  depending only on $\rwCxdf$ and $\Lambda$ such that
  $ \nRi{\txdfAg[{\erWe[]}]}{\rwCxdf,\Lambda} \leq
  \cst{\xdfCw[],\xdfCr,\Lambda}\naRa{(\xdfCw[],\Lambda)}$ for all $n\in\Nz$
  holds true.
\end{cor}
% % ....................................................................
% % <<Rem upper bound ag np>>
% % ....................................................................
% \begin{rmk}\label{ak:ag:ub:pnp:mm:rem} Keeping in mind that $\aDi{\ssY}({\xdfCw[]})\cmSv[\aDi{\ssY}({\xdfCw[]})]\to\infty$ as $\ssY\to\infty$ if there
%   is $\ssY_{\xdfCw[],\xdfCr,\Lambda}\in\Nz$ such that for all $\ssY\geq \ssY_{\xdfCw[],\xdfCr,\Lambda}$ in addition
%   $\aDi{\ssY}({\xdfCw[]})\cmSv[\aDi{\ssY}({\xdfCw[]})]\geq\Di_{\xdfCw[],\xdfCr,\Lambda} \vert \log\naRa{(\xdfCw[])} \vert $
%    holds true, then we have trivially
%   $\exp\big(\tfrac{-\cmiSv[\aDi{\ssY}({\xdfCw[]})]\aDi{\ssY}({\xdfCw[]})}{200\Vnorm[{\xdfCw[]}]{\edf}\xdfCr}\big)\leq
%   \naRa{(\xdfCw[])}$ while for $\ssY< \ssY_{\xdfCw[],\xdfCr,\Lambda}$ we have
%   $\exp\big(\tfrac{-\cmiSv[\aDi{\ssY}({\xdfCw[]})]\aDi{\ssY}({\xdfCw[]})}{200\Vnorm[{\xdfCw[]}]{\edf}\xdfCr}\big)\leq1\leq
%   \ssY\naRa{(\xdfCw[])}\leq \ssY_{\xdfCw[],\xdfCr,\Lambda} \naRa{{\xdfCw[]}}$. Thereby, from
%   \eqref{ak:ag:ub:pnp:e2} with $\sDi{\ssY}:=\aDi{\ssY}({\xdfCw[]})$ follows immediately
%   \begin{equation}\label{ak:ag:ub:pnp:rem:mm:e1}
%     \nRi{\txdfAg[{\erWe[]}]}{\rwCxdf,\Lambda}
%     % \E\Vnormlp{\txdfAg[{\erWe[]}]-\xdf}^2% \leq
%     \leq \cst{\xdfCw[],\xdfCr,\Lambda} \naRa{(\xdfCw[],\Lambda)}
% \end{equation}
% Consequently, if $ \vert \log\naRa{(\xdfCw[],\Lambda)} \vert =o(\aDi{\ssY}({\xdfCw[]})\cmSv[\aDi{\ssY}({\xdfCw[]})])$ then the data-driven
% aggregated estimator attains the rate $\naRa{(\xdfCw[],\Lambda)}$. Otherwise,
% the upper bound faces a deterioration of the rate, which we illustrate next.\remEnd
% \end{rmk}
% ....................................................................
% <<Il upper bound 3>>
% ....................................................................
\begin{il}\label{ak:il:ub:np:mm}
  Let us illustrate  \nref{ak:ag:ub:pnp:mm} and
  \nref{ge:ak:ag:ub2:pnp:mm}. Under \nref{ge:ak:ag:ub2:pnp:mm:c} the
  partially data-driven
aggregated estimator attains the rate $\naRa{(\xdfCw[],\Lambda)}$. Otherwise,
the upper bound faces a deterioration of the rate, which  we illustrate considering as in \nref{il:mm} usual
  behaviour  \ref{il:mm:oo}, \ref{il:mm:os} and \ref{il:mm:so}
 for the sequences $\Nsuite[\Di]{\xdfCw[(\Di)]}$ and
 $\Nsuite[\Di]{\iSv[\Di]}$.
 In case \ref{ak:mrb:ass:il:oo},
  \ref{ak:mrb:ass:il:os} and \ref{ak:mrb:ass:il:so} only with
  $p<1/2$ the assumption \ref{ge:ak:ag:ub2:pnp:mm:c} is satisfied, and
  $\naRa{(\xdfCw[],\Lambda)}$ equals the oracle rate $\oRa{\xdfCw[],\Lambda}$ (cf.
  \nref{ak:mrb:ass:il} \ref{ak:mrb:ass:il:oo},
  \ref{ak:mrb:ass:il:os} and \ref{ak:mrb:ass:il:so}). Thereby, the
  partially data-driven aggregation leads to an estimator attaining
  the oracle rate $\oRa{\xdfCw[],\Lambda}$. In case \ref{ak:mrb:ass:il:os}
  with $p\geq1/2$ the assumption \ref{ge:ak:ag:ub2:pnp:mm:c} is not
  satisfied. However, with
  $\sDi{\ssY}:=\Di_{\cst{3}} \vert \log\naRa{(\xdfCw[],\Lambda)} \vert \approx(\log\ssY)$ holds
  $\min_{\Di\in\nset{1,\ssY}}\big[\dRa{\Di}{\xdfCw[],\Lambda}\vee\exp\big(\tfrac{-\cmiSv\Di}{\Di_{\cst{3}}}\big)\leq\daRa{\sDi{\ssY}}{(\xdfCw[],\Lambda)}\approx(\log\ssY)^{2a+1}\ssY^{-1}$.
  In this situation the rate of the partially data-driven estimator
  $\txdfAg[{\erWe[]}]$ features a deterioration by a logarithmic factor
  $(\log\ssY)^{(2a+1)(1-1/(2p))}$ compared to the oracle rate, i.e.
  $\daRa{\sDi{\ssY}}{(\xdfCw[],\Lambda)}\approx(\log\ssY)^{2a+1}\ssY^{-1}$ versus
  $\oRa{\xdfCw[],\Lambda}\approx(\log\ssY)^{(2a+1)/(2p)}\ssY^{-1}$.\ilEnd
\end{il}


\subsection{Unknown operator}\label{freq:ge:strat:uk}\label{FREQ:GE:STRAT:UK}
Consider now the case \nref{AS_INTRO_DATA_UNKNOWN}.
We shall hence keep in mind \ref{freq:ge:shape:uk}, \ref{freq:ge:shape:uk:we}, \nref{freq:ge:shape:uk:de:pen:oo} as well as \ref{freq:ge:shape:uk:de:ms} and \ref{freq:ge:shape:uk:de:msWe}.
\textbf{Note that the detailed proofs for all results given here can be found in \nref{pro:freq:ge:strat:uk}}.

We will assume, from now on, that \nref{as:il} holds true.

Both for the quadratic and the maximal risk, our strategy is based on the decomposition of the quadratic loss function displayed in \nref{freq:ge:strat:kn:co:agg}.
This decomposition is independent of the model an only relies on the fact that the parameter space is equipped with a nested sieve and the fact that our estimator aggregation structure takes advantage of it.

\begin{lm}\label{co:agg:au}\label{freq:ge:strat:kn:co:agg:au:e1}
  Consider the aggregated OSE
  $\widehat{\theta}^{(\eta)}=\sum_{\Di=1}^n\We\hxdfPr$ with weights
  $\We[(\Di)]\in[0,1]$, $\Di\in\nset{1,\ssY}$, satisfying
  $ \sum_{\Di=1}^n\We[(\Di)]=1$ and a sequence
  $(\pen(m))_{\Di\in\nset{1,\ssY}}$ of non-negative compensation terms.
  Given $\Di\in\Nz$ let
  $\dxdfPr:=\sum_{s=-\Di}^{\Di}\hfedfmpI[(s)]\fydf[(s)]$. For any
  $\mDi\in\nset{1,\ssY}$, $\pDi\in\nset{1,\ssY}$, and the sequence of events $(\xEv)_{s \in \Z} = (\{\vert \hfedfmpI[(s)] \vert ^2 \geq \ssE^{-1} \})_{s \in \Z}$ holds
  \begin{multline}\label{co:agg:au:e1}
    \Vnormlp{\widehat{\theta}^{(\eta)}-\xdf}^2\leq 
    3\Vnormlp{\hxdfPr[\pDi]-\dxdfPr[\pDi]}^2
    +3 \Vnormlp{\xdf_{\underline{0}}}^2\bias[\mDi]^2(\xdf)
    \\\hfill
    +3 \Vnormlp{\xdf_{\underline{0}}}^2\We[](\nsetro{1,\mDi})
    +\tfrac{3}{7}\sum_{l\in\nsetlo{\pDi,\ssY}}\pen(l)\We[(l)]
    \Ind{\{\Vnormlp{\hxdfPr[l]-\dxdfPr[l]}^2<\pen(l)\}}\\\hfill
    +3\sum_{l\in\nsetlo{\pDi,\ssY}}\vectp{\Vnormlp{\hxdfPr[l]-\dxdfPr[l]}^2-\pen(l)/7}
    +\tfrac{3}{7}\sum_{l\in\nsetlo{\pDi,\ssY}}\pen(l)
    \Ind{\{\Vnormlp{\hxdfPr[l]-\dxdfPr[l]}^2\geq\pen(l)/7\}}\\
    +6\sum_{s\in\nset{1,\ssY}} \vert \hfedfmpI[(s)] \vert ^2 \vert \fedf[(s)]-\hfedf[(s)] \vert ^2 \vert \fxdf[(s)] \vert ^2
    +2\sum_{s\in\nset{1,\ssY}}\Ind{\xEv^c} \vert \fxdf[(s)] \vert ^2
 \end{multline}
\end{lm}
% ....................................................................
% <<Te weighrts>>
% ....................................................................
\begin{te}
Keep in mind the shape of the estimator given in \ref{freq:ge:shape:uk}, \nref{freq:ge:shape:uk:we}, \nref{freq:ge:shape:uk:de:ms}, \nref{freq:ge:shape:uk:de:msWe}, and \nref{freq:ge:shape:uk:de:pen:oo}.
\end{te}

\subsubsection{Quadratic risk bounds}\label{freq:ge:strat:uk:qu}
\begin{te}
  We derive bounds for the risk of the aggregated estimator $\hxdfAg$
  and the model selected estimator $\theta_{n, \overline{\hDi}}$ by applying
  \nref{co:agg:au}. Therefore, we aim next to control the third and
  fourth right hand side term in \eqref{co:agg:au:e1}.
  The proofs for the results stated here can be found in \nref{pro:freq:ge:strat:uk:qu}.
\end{te}
% --------------------------------------------------------------------
% <<Text Definition {p \vert m}Di>>
% --------------------------------------------------------------------
\begin{te}
For each $\Di\in\Nz$ keep in mind that
$\Vnormlp{\xdf_{\underline{\Di}}}^2=\Vnormlp{\xdf_{\underline{0}}}^2\bias[\Di]^2(\xdf)$,  
  $\daRa{\Di}{(\theta^{\circ},\Lambda)}:=[\b_{m}^{2}(\theta^{\circ})\vee \DipenSv\,\ssY^{-1}]$ as in
  \nref{freq:ge:strat:kn:ma:de:rate} and
introduce in addition
$\dxdfPr=\mathds{1}_{\{\vert s \vert \leq m\}}\hfedfmpI[(s)]\fydf[(s)]$. Note
that  $\dxdfPr=\Proj[\Di]\dxdfPr[\ssY]$
and $\Vnormlp{\ProjC[\Di]\dxdfPr[\ssY]}^2=2\sum_{s\in\nsetlo{\Di,\ssY}}\eiSv[(s)] \vert \fydf[(s)] \vert ^2$. For any $\pdDi,\mdDi\in\nset{1,\ssY}$ let us define 
\begin{multline}\label{au:de:*Di:ag}
\mDi:=\min\set{\Di\in\nset{1,\mdDi}: \Vnormlp{\xdf_{\underline{0}}}^2\bias^2(\xdf)\leq
  [\Vnormlp{\xdf_{\underline{0}}}^2+104\cpen]\daRa{\mdDi}{(\xdf,\Lambda)}}\quad\text{and}\\\pDi:=\max\set{\Di\in\nset{\pdDi,\ssY}:
   \peneSv \leq 2[3\Vnormlp{\ProjC[\pdDi]\dxdfPr[\ssY]}^2+2\peneSv[(\pdDi)]]}
\end{multline}
where  the defining set obviously contains $\mdDi$ and $\pdDi$, respectively, 
and hence, they are
not empty. Keep in mind that $\pDi:=\pDi(\rE_1,\dotsc,\rE_{\ssE})$ is
random but does not depend on the sample $\rY_1,\dotsc,\rY_{\ssY}$.
\end{te}
% ....................................................................
% <<Re Sum Random weights>>
% ....................................................................
\begin{lm}\label{au:re:SrWe:ag}
Consider the data-driven aggregation weights $\erWe[]$ as in \eqref{freq:ge:shape:uk:we}.  Using the penalty as in \nref{freq:ge:shape:uk:de:pen:oo} with $\aixEv[l] := \setB{1/4\leq\iSv[s]^{-1}\eiSv[(s)] \leq9/4,\;\forall\;s\in\nset{1,l}}$, $l\in\nset{1,\ssY}$,
    for any
  $\mdDi,\pdDi\in\nset{1,\ssY}$ and associated $\pDi,\mDi\in\nset{1,\ssY}$
  as in \eqref{au:de:*Di:ag} hold
  \begin{resListeN}
  \item\label{au:re:SrWe:ag:i}$\rWe[](\nsetro{1,\mDi})\\
  \leq \tfrac{50}{\rWc\cpen}\Ind{\{\mDi>1\}} \exp\big(-\tfrac{\rWc\cpen}{2} \ssY\daRa{\mdDi}{(\xdf,\Lambda)}\big) + \Ind{\{\Vnormlp{\hxdfPr[\mdDi]-\dxdfPr[\mdDi]}^2\geq\peneSv[(\mdDi)]/7\}\cup\aixEv[\mdDi]^c}$;
  \item\label{au:re:SrWe:ag:ii}
    $\sum_{\Di\in\nsetlo{\pDi,n}}\peneSv\erWe\Ind{\{\Vnormlp{\hxdfPr-\dxdfPr}^2<\peneSv/7\}}
    \leq \ssY^{-1}\{\tfrac{16}{\cpen\rWc^{2}}+ \tfrac{8}{\rWc}\}$.
  \end{resListeN}
  \reEnd
\end{lm}
% ....................................................................
% <<Te Sum MS Random weights>>
% ....................................................................
\begin{te}
  We combine the upper bound in \eqref{co:agg:au:e1} and the bounds given
  in \nref{au:re:SrWe:ag}. Conditionally on $\rE_1,\dotsc,\rE_{\ssE}$ the r.v.'s
$\rY_1,\dotsc,\rY_{\ssY}$ are \iid and we  denote by $\P_{\rY \vert \rE}$ and $\E_{\rY \vert \rE}$ their conditional
 distribution and expectation, respectively. Clearly, due to \nref{au:re:SrWe:ag} we have
  \begin{multline*}
    \E_{\rY \vert \rE}\erWe[](\nsetro{1,\mDi})\leq\Ind{\{\mDi>1\}}
    \big[\tfrac{50}{\rWc\cpen}\exp\big(-\tfrac{3\rWc\cpen}{14}n\daRa{\mdDi}{(\xdf,\Lambda)}\big)\\+
    \P_{\rY \vert \rE}\big(\Vnormlp{\hxdfPr[\mdDi]-\dxdfPr[\mdDi]}^2
    \geq\peneSv[(\mdDi)]/7\big) \Ind{\aixEv[\mdDi]} + \Ind{\aixEv[\mdDi]^c}\big]
  \end{multline*}
  and, hence from \eqref{co:agg:au:e1} follows immediately
  \begin{multline}\label{co:agg:au:ag}
   \E_{\rY \vert \rE}\Vnormlp{\widehat{\theta}^{(\eta)}-\xdf}^2\leq 3\E_{\rY \vert \rE}\Vnormlp{\hxdfPr[\pDi]-\dxdfPr[\pDi]}^2 + 3 \Vnormlp{\xdf_{\underline{0}}}^2\bias[\mDi]^2(\xdf)\\\hfill
    +\tfrac{150}{\rWc\cpen}\Vnormlp{\xdf_{\underline{0}}}^2\Ind{\{\mDi>1\}} \exp\big(-\tfrac{3\rWc\cpen}{14}n\daRa{\mdDi}{(\xdf,\Lambda)}\big) +\tfrac{3}{7}\ssY^{-1}\{\tfrac{16}{\cpen\rWc^{2}}+ \tfrac{8}{\rWc}\}\\\hfill
    +3 \Vnormlp{\xdf_{\underline{0}}}^2\Ind{\{\mDi>1\}} \big[\P_{\rY \vert \rE}\big(\Vnormlp{\hxdfPr[\mdDi]-\dxdfPr[\mdDi]}^2 \geq\peneSv[(\mdDi)]/7\big) \Ind{\aixEv[\mdDi]} + \Ind{\aixEv[\mdDi]^c}\big]\\\hfill
    +3\sum_{l\in\nsetlo{\pDi,\ssY}}\E_{\rY \vert \rE}\vectp{\Vnormlp{\hxdfPr[l]-\dxdfPr[l]}^2-\pen(l)/7}\\
     + \tfrac{3}{7}\sum_{l\in\nsetlo{\pDi,\ssY}}\peneSv[(l)]\P_{\rY \vert \rE}\big(\Vnormlp{\hxdfPr[l]-\dxdfPr[l]}^2\geq\pen(l)/7\big)\\
    +6\sum_{s\in\nset{1,\ssY}} \vert \hfedfmpI[(s)] \vert ^2 \vert \fedf[(s)]-\hfedf[(s)] \vert ^2 \vert \fxdf[(s)] \vert ^2
    +2\sum_{s\in\nset{1,\ssY}}\Ind{\xEv^c} \vert \fxdf[(s)] \vert ^2
  \end{multline}
\end{te}
% ....................................................................
% Outline Oracle optimality
% ....................................................................
\begin{te}
 The next result can be directly deduced from \nref{au:re:SrWe:ag} by letting
  $\rWc\to\infty$. However, we think the direct proof given in annex 
  provides  an interesting illustration  of the values
  $\pDi,\mDi\in\nset{1,\ssY}$ as defined in \eqref{au:de:*Di:ag}.
\end{te}
% ....................................................................
% <<Upper bound random weights>>
% ....................................................................
\begin{lm}\label{au:re:SrWe:ms}
Consider the data-driven model selection weights $\msWe[]$
  as in \nref{freq:ge:shape:uk:de:msWe}.  Under definition \nref{freq:ge:shape:uk:de:pen:oo} for any $\mdDi,\pdDi\in\nset{1,\ssY}$ and associated
  $\pDi,\mDi\in\nset{1,n}$ as in \eqref{au:de:*Di:ag} hold
  \begin{resListeN}[]
  \item\label{au:re:SrWe:ms:i}
    $\msWe[](\nsetro{1,\mDi})\Ind{\{\Vnormlp{\hxdfPr[\mdDi]-\dxdfPr[\mdDi]}^2<\peneSv[(\mdDi)]/7\}\cap\aixEv[\mdDi]}=0$;
  \item\label{au:re:SrWe:ms:ii}
    $\sum_{\Di\in\nsetlo{\pDi,\ssY}}\peneSv\msWe\Ind{\{\Vnormlp{\hxdfPr-\dxdfPr}^2<\peneSv/7\}}=0$.
  \end{resListeN}
  \reEnd
\end{lm}
% ....................................................................
% <<Te Sum MS Random weights>>
% ....................................................................
\begin{te}
  We combine the upper bound in \eqref{co:agg:au:e1} and the bounds given
  in \nref{au:re:SrWe:ms}.  Clearly, due to \nref{au:re:SrWe:ms} we have
  \begin{equation*}
    \E_{\rY \vert \rE}\msWe[](\nsetro{1,\mDi})\leq\Ind{\{\mDi>1\}}
    \big[ \P_{\rY \vert \rE}\big(\Vnormlp{\hxdfPr[\mdDi]-\dxdfPr[\mdDi]}^2
    \geq\peneSv[(\mdDi)]/7\big) \Ind{\aixEv[\mdDi]} + \Ind{\aixEv[\mdDi]^c}\big]
  \end{equation*}
  and, hence from \eqref{co:agg:au:e1} follows immediately
  \begin{multline}\label{co:agg:au:ms}
   \E_{\rY \vert \rE}\Vnormlp{\widehat{\theta}^{(\eta)}-\xdf}^2\leq 
    3\E_{\rY \vert \rE}\Vnormlp{\hxdfPr[\pDi]-\dxdfPr[\pDi]}^2
    +3 \Vnormlp{\xdf_{\underline{0}}}^2\bias[\mDi]^2(\xdf)\\\hfill
    +3 \Vnormlp{\xdf_{\underline{0}}}^2\Ind{\{\mDi>1\}}
    \big[\P_{\rY \vert \rE}\big(\Vnormlp{\hxdfPr[\mdDi]-\dxdfPr[\mdDi]}^2
    \geq\peneSv[(\mdDi)]/7\big) \Ind{\aixEv[\mdDi]} + \Ind{\aixEv[\mdDi]^c}\big]
    \\\hfill
    +3\sum_{l\in\nsetlo{\pDi,\ssY}}\E_{\rY \vert \rE}\vectp{\Vnormlp{\hxdfPr[l]-\dxdfPr[l]}^2-\pen(l)/7}\\
    +\tfrac{3}{7}\sum_{l\in\nsetlo{\pDi,\ssY}}\peneSv[(l)]\P_{\rY \vert \rE}\big(
    \Vnormlp{\hxdfPr[l]-\dxdfPr[l]}^2\geq\pen(l)/7\big)\\
    +6\sum_{s\in\nset{1,\ssY}} \vert \hfedfmpI[(s)] \vert ^2 \vert \fedf[(s)]-\hfedf[(s)] \vert ^2 \vert \fxdf[(s)] \vert ^2
    +2\sum_{s\in\nset{1,\ssY}}\Ind{\xEv^c} \vert \fxdf[(s)] \vert ^2
  \end{multline}
\end{te}



\begin{te}
The deviations of the last three terms in the last display
\eqref{co:agg:au:ms} and also in \eqref{co:agg:au:ag} need to be bounded using concentration inequalities which depend on the considered model.
We hence formulate it as the central hypothesis to be verified in order to apply this method.
\end{te}

% --------------------------------------------------------------------
% <<Re ND rest>>
% --------------------------------------------------------------------
\begin{as}\label{freq:ge:strat:uk:qu:as}
  Consider
  $\hxdfPr-\dxdfPr=\sum_{|s|\in\nset{1,\Di}}\hfedfmpI[(s)](\hfydf[(s)]-\fydf[(s)])\bas_s$.
  Conditionally on $\{\rE_1,\dotsc,\rE_{\ssE}\}$ the r.v.'s
  $\{\rY_1,\dotsc,\rY_{\ssY}\}$ are \iid and we denote by
  $\P_{Y\vert \epsilon}$ and $\E_{\rY\vert\rE}$ their conditional
  distribution and expectation, respectively.  Let
  $\eiSv[(s)]=|\hfedfmpI[(s)]|^2$,
  $\oeiSv=\tfrac{1}{\Di}\sum_{s\in\nset{1,\Di}}\eiSv[(s)]$,
  $\meiSv= \max\{\eiSv[(s)],s\in\nset{1,\Di}\}$,
  $\DiepenSv=\cmeiSv\Di \meiSv$ and $\cmeiSv\geq1$.  Then there is a
  numerical constant $\cst{}$ such that for all $\ssY\in\Nz$ and
  $\Di\in\nset{1,\ssY}$ holds
  \begin{resListeN}[]
  \item\label{freq:ge:strat:uk:qu:as:i}
    $\E_{\rY\vert\rE} \vectp{\Vnormlp{\hxdfPr-\dxdfPr}^2 - 12\DiepenSv\ssY^{-1}}   \\
    \quad\quad\leq\cst{1} \bigg[\tfrac{\cst{2}\,\meiSv}{\ssY}
    \exp\big(-\cst{3}\cmeiSv\Di\big)
    +\tfrac{\cst{4}\Di\meiSv}{n^2}\exp\big(-\cst{5}\sqrt{\ssY\cmeiSv}\big) \bigg]$
  \item\label{freq:ge:strat:uk:qu:as:ii}
$\P_{Y\vert \epsilon}\big(\Vnormlp{\hxdfPr-\dxdfPr}^2 \geq 12\DiepenSv\ssY^{-1}\big)\\
\leq 
\cst{6} \bigg[\exp\big(-\cst{7}\cmeiSv\Di\big)
+\exp\big(-\cst{8}\sqrt{\ssY\cmeiSv}\big)\bigg]$
   \item\label{freq:ge:strat:uk:qu:as:iii}
     $\P_{Y\vert \epsilon}\big(\Vnormlp{\hxdfPr-\dxdfPr}^2 \geq 12\DiepenSv\ssY^{-1}\big)\\
     \leq 
     \cst{9} \bigg[\exp\big(-\cst{10}\cmeiSv\Di\big)
     +\exp\big(\tfrac{-\cst{11}\ssY\sqrt{\daRa{\Di}{\xdf,\eiSv}}}{\sqrt{\Di\meiSv}}\big)\bigg]$
   \item\label{freq:ge:strat:uk:qu:as:iv}
   $\P\big(|\hfedf[(s)]/\fedf[(s)]-1|>1/3\big)\leq \cst{12}\exp\big(-\cst{13}\ssE|\fedf[(s)]|^2\big)\leq \cst{14}\exp\big(-\tfrac{\cst{15}\ssE}{\miSv}\big).$
\end{resListeN}
\assEnd
\end{as}

%\begin{lm}\label{re:evrest}
%Assume that \nref{freq:ge:strat:uk:qu:as} holds true.
%For $\Di\in\Nz$ consider
%  $\aixEv[\Di]:=\{1/2\leq|\fedf[(s)]\hfedfmpI[(s)]|\leq3/2:\;\forall\;s\in\nset{1,\Di}\}$. 
%\begin{resListeN}[]
%\item\label{re:evrest:i}
%For all $\Di,\ssE\in\Nz$ with  $\miSv[k]\leq
%(4/9)\ssE$ holds $\P(\aixEv^c)\leq 2\Di\exp\big(-\tfrac{\ssE}{72\miSv}\big)$.
%\item\label{re:evrest:ii}
%Given  $\Di\in\Nz$ let $\ssE(\Di):=\ceil{9\miSv/4}$  then 
%$\P(\aixEv^c)\leq(555\Di\ssE(k)^2\ssE^{-2})\wedge(12\Di\ssE(\Di)\ssE^{-1})$
%holds true for all
%$\ssE\in\Nz$.
%\item\label{re:evrest:iii}
%For all $\Di,\ssE\in\Nz$ with $\ssE\geq289\log(\Di+2)\cmiSv\miSv$ holds $\P(\aixEv^c)\leq(11226\ssE^{-2})\wedge(53\ssE^{-1})$.
%\end{resListeN}
%\end{lm}
This hypothesis allows us to control the remaining random elements in our bound.

\begin{lm}\label{au:re:nd:rest}
Consider $\peneSv=\peneD$,
  $\Di\in\nset{1,n}$, as in \nref{freq:ge:shape:uk:de:pen:oo} with $\cpen\geq84$.
Let $\Di_{\cst{3}}:=[\floor{3(\tfrac{2}{\cst{3}})^2}\vee \cst{2}]$ and $\ssY_{\cst{5}}:=15(\tfrac{1}{\cst{5}})^4$; as well as $\Di_{\cst{7}}:=\floor{3(\tfrac{2}{\cst{7}})^2}$ and
    $\ssY_{\cst{8}}:=\floor{15({3/\cst{8}})^4}$.
    There exists a finite numerical constant  $\cst{}>0$ such that for all $n\in\Nz$ and all $\mdDi\in\nset{1,\ssY}$  hold
\begin{resListeN}
\item\label{au:re:nd:rest1}
$\sum_{\Di=1}^{\ssY}\E_{\rY \vert \rE}\vectp{\Vnormlp{\hxdfPr-\dxdfPr}^2-\peneSv/7}\\
\leq\cst{}\ssY^{-1}\big[(1\vee\meiSv[\Di_{\cst{3}}])\Di_{\cst{3}}+(1\vee\meiSv[\ssY_{\cst{5}}]\ssY_{\cst{5}})\big]$;
\item\label{au:re:nd:rest2}
  $\sum_{\Di=1}^{\ssY}\peneSv\P_{\rY \vert \rE}\big(\Vnormlp{\hxdfPr-\dxdfPr}^2\geq\peneSv/7\big)\leq\cst{}\ssY^{-1}\big[(1\vee\meiSv[\Di_{\cst{7}}]^2)\Di_{\cst{7}}^2+(1\vee\meiSv[\ssY_{\cst{8}}]^2\ssY_{\cst{8}}^2)\big]$;
\item\label{au:re:nd:rest3}
  $\P_{\rY \vert \rE}\big(\Vnormlp{\hxdfPr[\mdDi]-\dxdfPr[\mdDi]}^2\geq\peneSv[(\mdDi)]/7\big)\leq    \cst{}  \big[\exp\big(-\cst{11}\cmeiSv[\mdDi]\mdDi\big)
    +\ssY^{-1}\big]$.
\end{resListeN}
\reEnd
\end{lm}
% --------------------------------------------------------------------
% <<Te penalty>>
% --------------------------------------------------------------------
\begin{te}Consider now the fully data-driven aggregation of the
  orthogonal series estimators using either  aggregation weights $\erWe[]$
  as in \eqref{freq:ge:shape:uk:we} or model selection weights $\widehat{\P}_{M}^{(\infty)}$ as in \nref{freq:ge:shape:uk:de:msWe}
  combining \nref{freq:ge:strat:uk:qu:as} and the upper bound given
  in \eqref{co:agg:au:ag}  or \eqref{co:agg:au:ms} we obtain the next result. 
\end{te}
% ....................................................................
% <<Re upper bound ag>>
% ....................................................................
\begin{lm}\label{au:ag:ub}
Let \nref{freq:ge:strat:uk:qu:as} hold true.
Consider the penalty sequence $\peneSv:=\peneD$, $\Di\in\nset{1,n}$, as in \nref{freq:ge:shape:uk:de:pen:oo}.
  Let $\widehat{\theta}^{(\eta)}=\sum_{\Di=1}^{\ssY} \erWe\hxdfPr$ be an aggregation of the orthogonal series estimators using either aggregation weights $\erWe[]$ as in \eqref{freq:ge:shape:uk:we} or model selection weights $\widehat{P}_{M}^{(\infty)}$ as in \nref{freq:ge:shape:uk:de:msWe}.
  Then, there is a finite numerical constant $\cst{}>0$ such that for all $\ssY,\ssE\in\Nz$, for any $\mdDi,\pdDi\in\nset{1,n}$ and associated $\mDi\in\nset{1,n}$
  as defined in \eqref{au:de:*Di:ag} hold
\begin{multline}\label{au:ag:ub:e1}
  \E\Vnormlp{\widehat{\theta}^{(\eta)}-\xdf}^2 \leq 2\penSv[\pdDi] +\tfrac{12}{7}\Vnormlp{\xdf_{\underline{0}}}^2\bias[\pdDi]^2(\xdf)+3 \Vnormlp{\xdf_{\underline{0}}}^2\bias[\mDi]^2(\xdf)\\\hfill
  + \cst{}\big[
    \Vnormlp{\xdf_{\underline{0}}}^2\Ind{\{\mDi>1\}} \FuVg[\ssE]{\rE}(\aixEv[\mdDi]^c)+\ssE\FuVg[\ssE]{\rE}(\aixEv[\pdDi]^c) \big] + \cst{}\mRa{\xdf,\Lambda}
  \end{multline}
  \reEnd
\end{lm}
% ....................................................................
% <<Te upper bound ag p np>>
% ....................................................................
\begin{te} The last bound allows us to derive an upper bound of the
  risk for the fully data-driven aggregated estimator in the two cases
  \ref{oo:xdf:p} and \ref{oo:xdf:np} introduced in \nref{bm:ak}.
\end{te}
% ....................................................................
% <<Re upper bound ag p np>>
% ....................................................................
\begin{thm}\label{au:ag:ub:pnp}
Let \nref{freq:ge:strat:uk:qu:as} hold true.
Consider the   penalty sequence $\peneSv:=\peneD$,
  $\Di\in\nset{1,\ssY}$, as in \nref{freq:ge:shape:uk:de:pen:oo}.
  Let $\hxdfAg[{\erWe[]}]=\sum_{\Di=1}^{\ssY} \erWe\hxdfPr$ be an aggregation of the orthogonal series estimators using either aggregation weights $\erWe[]$ as in \eqref{freq:ge:shape:uk:we} or model selection weights $\msWe[]$ as in \nref{freq:ge:shape:uk:de:msWe}.
  \begin{Liste}[]
  \item[\mylabel{au:ag:ub:pnp:p}{\dgrau\bfseries{(p)}}]Assume there is
    $K\in\Nz_0$ with $1\geq \bias[{[K-1] }](\xdf)>0$ and
    $\bias[\Di](\xdf)=0$. For $K>0$ let
    $c_{\xdf}:=\tfrac{\Vnormlp{\xdf_{\underline{0}}}^2+104\cpen}{\Vnormlp{\xdf_{\underline{0}}}^2\bias[{[K-1]}]^2(\xdf)}>1$,
    $\ssY_{\xdf,\Lambda}:=\floor{c_{\xdf}\DipenSv[K]}\in\Nz$ and
    $\ssE(\xdf,\Lambda):=\floor{289\log(K+2)\cmiSv[K]\miSv[K]}\in\Nz$. If
    $\ssY>\ssY_{\xdf,\Lambda}$ and $\ssE>\ssE(\xdf,\Lambda)$ then set
    $\sDi{\ssY}:=\max\{\Di\in\nset{K,\ssY}:\ssY>c_{\xdf}\DipenSv\}$
    and
    $\sDi{\ssE}:=\max\{\Di\in\nset{K,\ssE}:289\log(\Di+2)\cmiSv[\Di]\miSv[\Di]\leq\ssE\}$
    where the defining set, respectively, contains $K$ and thus is not
    empty, and otherwise $\sDi{\ssY}\wedge\sDi{\ssE}:=\Di_{\cst{3}}\log(\ssY\wedge\ssE)$.
    There is a numerical constant $\cst{}$ and a  constant $\cst{\xdf,\Lambda}$ given in
    \eqref{au:ag:ub:pnp:p9} depending only on $\xdf$ and $\Lambda$ such
    that for all $\ssY,\ssE\in\Nz$ holds
    \begin{equation}\label{au:ag:ub:pnp:e1}
       \nmRi{\hxdfAg[{\erWe[]}]}{\xdf,\Lambda}
       %\E\Vnormlp{\widehat{\theta}^{(\eta)}-\xdf}^2
       \leq
      \cst{}\Vnormlp{\xdf_{\underline{0}}}^2\big[\ssY^{-1}\vee \ssE^{-1} \vee
      \exp\big(\tfrac{-\cmiSv[\sDi{\ssY}\wedge\sDi{\ssE}]\sDi{\ssY}\wedge\sDi{\ssE}}{\Di_{\cst{3}}}\big)\big]\\
      + \cst{\xdf,\Lambda}\{\ssY^{-1}\vee\ssE^{-1}\}.
    \end{equation}
  \item[\mylabel{au:ag:ub:pnp:np}{\dgrau\bfseries{(np)}}] Assume that
    $\bias(\xdf)>0$ for all $\Di\in\Nz$. Let    
    $\ssE(\Lambda):=\floor{289\log(3)\cmiSv[1]\miSv[1]}\in\Nz$. If
    $\ssE>\ssE(\Lambda)$ then set
    $\sDi{\ssE}:=\max\{\Di\in\nset{1,\ssE}:289\log(\Di+2)\cmiSv[\Di]\miSv[\Di]\leq\ssE\}$
    where the defining set, respectively, contains $1$ and thus is not
    empty.  There is a numerical constant $\cst{}$  such that for all $\ssY\in\Nz$
    with $\aDi{\ssY}:=\aDi{\ssY}(\xdf)\in\nset{1,n}$ as in \nref{freq:ge:strat:kn:ma:de:rate} and
    for all $\ssE>\ssE(\Lambda)$ holds
    \begin{multline}\label{au:ag:ub:pnp:e2}
     \nmRi{\hxdfAg[{\erWe[]}]}{\xdf,\Lambda}  
     %\E\Vnormlp{\widehat{\theta}^{(\eta)}-\xdf}^2
     \leq\cst{}(1\vee\Vnormlp{\xdf_{\underline{0}}}^2)\min_{\Di\in\nset{1,\ssY}}\{\daRa{\Di}{(\xdf,\Lambda)}\vee\exp\big(\tfrac{-\cmiSv[\Di]\Di}{\Di_{\cst{3}}}\big)\}\Ind{\{\ssE>\ssE(\Lambda)\}}\\\hfill
+\cst{}(1\vee\Vnormlp{\xdf_{\underline{0}}}^2)\{\bias[\aDi{\ssY}\wedge\sDi{\ssE}]^2(\xdf)\vee\exp\big(\tfrac{-\cmiSv[\sDi{\ssE}]\sDi{\ssE}}{\Di_{\cst{3}}}\big)\}\Ind{\{\ssE>\ssE(\Lambda)\}} \\\hfill
 +\cst{}\mRa{\xdf,\Lambda}   + \cst{}(1\vee\Vnormlp{\xdf_{\underline{0}}}^2)\miSv[1]^2\ssE^{-1}  
    +\cst{}\{\miSv[\Di_{\cst{3}}]^2\Di_{\cst{3}}^3+\miSv[\ssY_{o}]^2\}\ssY^{-1}
  \end{multline}
  while for $\ssE\in\nset{1,\ssE(\Lambda)}$ we have
  
  $\cst{}\mRa{\xdf,\Lambda}
    + \cst{}(1\vee\Vnormlp{\xdf_{\underline{0}}}^2)\miSv[1]^2\ssE^{-1}  
    +\cst{}\{\miSv[\Di_{\cst{3}}]^2\Di_{\cst{3}}^3+\miSv[\ssY_{o}]^2\}\ssY^{-1}$. 
\end{Liste}  
\reEnd
\end{thm}

\begin{cor}\label{ge:au:ag:ub2:pnp}
  Let the assumptions of \nref{au:ag:ub:pnp} be satisfied.
  \begin{Liste}[]
  \item[\mylabel{ge:au:ag:ub2:pnp:p}{\dgrau\bfseries{(p)}}]
    If \ref{ge:ak:ag:ub2:pnp:pc} as in \nref{ge:ak:ag:ub2:pnp} and 
    in addition
    \begin{inparaenum}% \item[\mylabel{ge:au:ag:ub2:pnp:pc:a}{{\dgrau\bfseries(A1)}}]
      % there is $\ssY_{\xdf,\Lambda}\in\Nz$ such that
      % $\cmiSv[\sDi{\ssY}]\sDi{\ssY}\geq \Di_{\cst{3}}(\log\ssY)$ for all
      % $\ssY\geq \ssY_{\xdf,\Lambda}$ and
    \item[\mylabel{ge:au:ag:ub2:pnp:pc:b}{{\dgrau\bfseries(A4)}}]
            there is $\ssE(\xdf,\Lambda)\in\Nz$ such that
      $\cmiSv[\sDi{\ssE}]\sDi{\ssE}\geq \Di_{\cst{3}}(\log\ssE)$ for all
      $\ssE\geq \ssE(\xdf,\Lambda)$ 
    \end{inparaenum}
    hold true, then there is a constant $\cst{\xdf,\Lambda}$ depending
    only on $\xdf$ and $\Lambda$ such that for all $\ssY,\ssE\in\Nz$ holds
    $\nmRi{\hxdfAg[{\erWe[]}]}{\xdf,\Lambda} \leq
    \cst{\xdf,\Lambda}[\ssY^{-1}\vee\ssE^{-1}]$.
  \item[\mylabel{ge:au:ag:ub2:pnp:np}{\dgrau\bfseries{(np)}}]
    If  \ref{ge:ak:ag:ub2:pnp:npc} as in \nref{ge:ak:ag:ub2:pnp} and \ref{ge:au:ag:ub2:pnp:pc:b}
    hold true, then there is a constant $\cst{\xdf,\Lambda}$ depending
    only on $\xdf$ and $\Lambda$ such that $\nmRi{\hxdfAg[{\erWe[]}]}{\xdf,\Lambda}
    \leq \cst{\xdf,\Lambda}\{\naRa{(\xdf,\Lambda)}+\mRa{\xdf,\Lambda}+\bias[\sDi{\ssE}\wedge\aDi{\ssY}]^2(\xdf)\}$ for all $\ssY,\ssE\in\Nz$ holds true.
  \end{Liste}  
\end{cor}
% ....................................................................
% <<Rem upper bound ag p np>>
% ....................................................................
\begin{il}\label{au:ag:ub:pnp:il}
  Let us briefly illustrate the last results. In case
  \ref{au:ag:ub:pnp:p} the fully data-driven aggregation leads to an
  estimator attaining the parametric oracle rate (see
  \nref{oo:rem:ora}), if the additional assumptions
  \ref{ge:ak:ag:ub2:pnp:pc} and \ref{ge:au:ag:ub2:pnp:pc:b} are satisfied.
  Consider the two cases \ref{il:edf:o} and \ref{il:edf:s} for the
  operator Fourier sequence $\edf$ as in \nref{il:oo}, where in both cases 
 \ref{ge:ak:ag:ub2:pnp:pc} holds true (cf. \nref{ak:ag:ub:pnp:il}
 \ref{ak:il:edf:o} and \ref{ak:il:edf:s}), while
  \begin{Liste}[]
  \item[\mylabel{au:il:edf:o}{\dg\bfseries{(o)}}]
    $\ssE\sim(\log\sDi{\ssE})\cmSv[\sDi{\ssY}]\miSv[\sDi{\ssE}]
    \sim(\log\sDi{\ssE})(\sDi{\ssE})^{2a}$
    implies $\sDi{\ssE}\sim(\ssE/\log\ssE)^{1/(2a)}$ and
    $\sDi{\ssE}\cmSv[\sDi{\ssE}]\sim (\ssE/\log\ssE)^{1/(2a)}$.
  \item[\mylabel{au:il:edf:s}{\dg\bfseries{(s)}}]
    $\ssE\sim(\log\sDi{\ssE})\cmSv[\sDi{\ssE}]\miSv[\sDi{\ssE}]\sim
    (\log\sDi{\ssE})(\sDi{\ssE})^{4a}\exp((\sDi{\ssE})^{2a})$ implies
    $\sDi{\ssE}\sim(\log
    \ssE-\tfrac{1+4a}{2a}\log\log\ssE-\tfrac{1}{2a}\log\log\log\ssE)^{1/(2a)}$
    and $\sDi{\ssE}\cmSv[\sDi{\ssE}]\sim (\log \ssE)^{2+1/(2a)}$.
  \end{Liste}
  Clearly in both cases \ref{au:il:edf:o} and \ref{au:il:edf:s} also
  \ref{ge:au:ag:ub2:pnp:pc:b} is satisfied. Therefore, in this situation
  the fully data-driven aggregated estimator attains the parametric
  oracle rate. On the other hand side, in case \ref{ge:au:ag:ub2:pnp:np}
  the fully data-driven aggregation leads to an estimator attaining
  the rate $\naRa{(\xdf,\Lambda)}+\mRa{\xdf,\Lambda}$ (\nref{ge:au:ag:ub2:pnp}),
  if \ref{ge:ak:ag:ub2:pnp:npc} and
  \ref{ge:au:ag:ub2:pnp:pc:b} are satisfied and
  $\bias[\sDi{\ssE}\wedge\aDi{\ssY}]^2(\xdf)$ is negligible with
  respect to $\naRa{(\xdf,\Lambda)}+\mRa{\xdf,\Lambda}$, otherwise the upper
  bound faces a deterioration of the rate, which we illustrate
  considering as in \nref{freq:ge:strat:kn:qu:il:rate} the usual behaviour
  \ref{freq:ge:strat:kn:qu:il:rate:np:oo}, \ref{freq:ge:strat:kn:qu:il:rate:np:os} and
  \ref{freq:ge:strat:kn:qu:il:rate:np:so} for the sequences
  $\Nsuite[\Di]{\bias[\Di](\xdf)}$ and $\Nsuite[\Di]{\iSv[\Di]}$.
  In all three cases \ref{freq:ge:strat:kn:qu:il:rate:np:oo}, \ref{freq:ge:strat:kn:qu:il:rate:np:os} and
  \ref{freq:ge:strat:kn:qu:il:rate:np:so} the assumption \ref{ge:au:ag:ub2:pnp:pc:b}
  holds true. Moreover, in case \ref{freq:ge:strat:kn:qu:il:rate:np:oo},
  \ref{freq:ge:strat:kn:qu:il:rate:np:os} and \ref{freq:ge:strat:kn:qu:il:rate:np:so} only with
  $p<1/2$ the assumption \ref{ge:ak:ag:ub2:pnp:npc} is satisfied, and
  $\naRa{(\xdf,\Lambda)}$ equals the oracle rate $\oRa{\xdf,\Lambda}$ (cf.
  \nref{freq:ge:strat:kn:qu:il:rate:np:oo} \ref{freq:ge:strat:kn:qu:il:rate:np:oo},
  \ref{freq:ge:strat:kn:qu:il:rate:np:os} and \ref{freq:ge:strat:kn:qu:il:rate:np:so}). In case
  \ref{freq:ge:strat:kn:qu:il:rate:np:os} and \ref{freq:ge:strat:kn:qu:il:rate:np:so}
  $\bias[\sDi{\ssE}]^2(\xdf)\leq \cst{\xdf,\Lambda} \mRa{\xdf,\Lambda}$ while
  in case \ref{freq:ge:strat:kn:qu:il:rate:np:oo}
  $\bias[\sDi{\ssE}]^2(\xdf)\sim(\ssE/\log\ssE)^{-p/a}$, hence 
  \begin{Liste}[]
  \item[\mylabel{au:ag:ub:pnp:il:oo}{\dg\bfseries{[o-o]}}]
    $\nmRi{\hxdfAg[{\erWe[]}]}{\xdf,\Lambda}
    \leq
    \cst{\xdf,\Lambda}\{\ssY^{-2p/(2p+2a+1)}+\ssE^{-(p\wedge a)/a}+
    (\ssE/\log\ssE)^{-p/a}\}$
      \item[\mylabel{au:ag:ub:pnp:il:os}{\dg\bfseries{[o-s]}}]
    $\nmRi{\hxdfAg[{\erWe[]}]}{\xdf,\Lambda}
    \leq
    \cst{\xdf,\Lambda}\{(\log \ssY)^{-p/a}+(\log \ssE)^{-p/a}\}$
  \item[\mylabel{au:ag:ub:pnp:il:so}{\dg\bfseries{[s-o]}}]
    $\nmRi{\hxdfAg[{\erWe[]}]}{\xdf,\Lambda}
    \leq
    \cst{\xdf,\Lambda}\{(\log\ssY)^{(2a+1)/(2p\wedge1)}\ssY^{-1}+\ssE^{-1}\}$
  \end{Liste}
  Consequently, the fully data-driven estimator attains the oracle
  rate in case \ref{freq:ge:strat:kn:qu:il:rate:np:oo} with $p>a$,
  \ref{freq:ge:strat:kn:qu:il:rate:np:os} and \ref{freq:ge:strat:kn:qu:il:rate:np:so} with
  $p\leq1/2$, while in  case \ref{freq:ge:strat:kn:qu:il:rate:np:oo} with $p\leq a$ and
  \ref{freq:ge:strat:kn:qu:il:rate:np:so} with $p>1/2$ the rate of the fully data-driven estimator
  $\hxdfAg[{\erWe[]}]$ features a deterioration by a logarithmic factor
  $(\log\ssE)^{p/a}$  and $(\log\ssY)^{(2a+1)(1-1/(2p))}$, respectively, compared to the oracle rate.\ilEnd
\end{il}


\subsubsection{Maximal risk bounds}\label{freq:ge:strat:uk:ma}
% --------------------------------------------------------------------
% <<Text Definition AG {p \vert m}Di>>
% --------------------------------------------------------------------
\begin{te}
    By applying \nref{co:agg:au} we derive bounds for the maximal risk defined in \eqref{oo:e4} over ellipsoids  $\rwCxdf$ of the fully data-driven aggregated estimator $\hxdfAg[{\erWe[]}]$ using either aggregation weights $\erWe[]$
  as in \eqref{freq:ge:shape:uk:we} or model selection weights $\widehat{\P}_{M}^{(\eta)}$ as in \nref{freq:ge:shape:uk:de:msWe}.
  Therefore, we aim next to control the second and third right hand side term in \eqref{co:agg:au:e1} uniformly over $\rwCxdf$.
  Results stated here are proven in \nref{pro:freq:ge:strat:uk:ma}.
\end{te}
% --------------------------------------------------------------------
% <<Text Definition {p \vert m}Di>>
% --------------------------------------------------------------------
\begin{te}
  For each $\Di\in\Nz$ keeping 
the definition \ref{freq:ge:strat:kn:ma:de:rate} of
  $\daRa{\Di}{(\xdfCw[],\Lambda)}:=[\xdfCw\vee \DipenSv\,\ssY^{-1}]$ in
  mind it holds
$\xdfCr^2\daRa{\Di}{(\xdfCw[],\Lambda)}\geq\Vnormlp{\xdf_{\underline{0}}}^2\bias^2(\xdf)$
uniformely for all $\xdf\in\rwCxdf$ and for all
$\Di\in\Nz$.  Introduce in addition
$\dxdfPr=\sum_{s\in\nset{-\Di,\Di}}\hfedfmpI[(s)]\fydf[(s)]$. Note
that  $\dxdfPr=\Proj[\Di]\dxdfPr[\ssY]$
and $\Vnormlp{\ProjC[\Di]\dxdfPr[\ssY]}^2=2\sum_{s\in\nsetlo{\Di,\ssY}}\eiSv[(s)] \vert \fydf[(s)] \vert ^2$. For any $\pdDi,\mdDi\in\nset{1,\ssY}$ let us define 
\begin{multline}\label{au:mrb:de:*Di}
\mDi:=\min\set{\Di\in\nset{1,\mdDi}: \Vnormlp{\xdf_{\underline{0}}}^2\bias^2(\xdf)\leq
  [\xdfCr^2+104\cpen]\daRa{\mdDi}{(\xdfCw[],\Lambda)}}\quad\text{and}\\\pDi:=\max\set{\Di\in\nset{\pdDi,\ssY}:
   \peneSv \leq 2[3\Vnormlp{\ProjC[\pdDi]\dxdfPr[\ssY]}^2+2\peneSv[(\pdDi)]]}
\end{multline}
where  the defining set obviously contains $\mdDi$ and $\pdDi$, respectively, 
and hence, they are
not empty. Keep in mind that $\pDi:=\pDi(\rE_1,\dotsc,\rE_{\ssE})$ is
random but does not depend on the sample $\rY_1,\dotsc,\rY_{\ssY}$.
\end{te}
% ....................................................................
% <<Re Sum Random weights>>
% ....................................................................
\begin{lm}\label{au:mrb:re:SrWe:ag}
Consider the data-driven aggregation weights $\erWe[]$ as in \eqref{freq:ge:shape:uk:we}.
Using the aggregation weights as in \nref{freq:ge:shape:uk:de:pen:oo} with
  $\cpen\geq8\log(3e)$ and
  
    $\aixEv[l]:=\setB{1/4\leq\iSv[s]^{-1}\eiSv[(s)]\leq9/4,\;\forall\;s\in\nset{1,l}}$, $l\in\nset{1,\ssY}$,
    for any
  $\mdDi,\pdDi\in\nset{1,\ssY}$ and associated $\pDi,\mDi\in\nset{1,\ssY}$
  as in \eqref{au:mrb:de:*Di} hold
  \begin{resListeN}
  \item\label{au:mrb:re:SrWe:ag:i}
    $\rWe[](\nsetro{1,\mDi})\leq \tfrac{50}{\rWc\cpen}\Ind{\{\mDi>1\}}
    \exp\big(-\tfrac{\rWc\cpen}{2} \ssY\daRa{\mdDi}{(\xdfCw[],\Lambda)}\big)\\
    \quad+\Ind{\{\Vnormlp{\hxdfPr[\mdDi]-\dxdfPr[\mdDi]}^2\geq\peneSv[(\mdDi)]/7\}\cup\aixEv[\mdDi]^c}$;
  \item\label{au:mrb:re:SrWe:ag:ii}
    $\sum_{\Di\in\nsetlo{\pDi,n}}\peneSv\erWe\Ind{\{\Vnormlp{\hxdfPr-\dxdfPr}^2<\peneSv/7\}}
    \leq \ssY^{-1}\{\tfrac{16}{\cpen\rWc^{2}}+ \tfrac{8}{\rWc}\}$.
  \end{resListeN}
\end{lm}
% --------------------------------------------------------------------
% <<Text unifom bounds over ellipsoid>>
% --------------------------------------------------------------------
\begin{te}Keeping in mind that $\ydf=\xdf\cdot\edf$. We note that
  uniformly for all $\xdf\in\rwCxdf$ by applying the
  Cauchy-Schwarz inequality holds   $\Vnormlp[1]{\fydf}\leq
\Vnorm[{\xdfCw[]}]{\edf}\Vnorm[1/{\xdfCw[]}]{\xdf}\leq
\Vnorm[{\xdfCw[]}]{\edf}\xdfCr$. Thereby, we obtain the next assertion
immediately from \nref{freq:ge:strat:uk:qu:as}, and we omit its elementary proof.
\end{te}
% --------------------------------------------------------------------
% <<Re ND rest>>
% --------------------------------------------------------------------
\begin{lm}\label{freq:ge:strat:uk:ma:as}
Assume that \nref{freq:ge:strat:uk:qu:as} holds true.
Then, we have
\begin{resListeN}
\item\label{freq:ge:strat:uk:ma:as1}
$\sup_{\xdf\in\rwCxdf}\E\sum\limits_{\Di=1}^{\ssY}\E_{\rY \vert \rE}\vectp{\Vnormlp{\hxdfPr-\dxdfPr}^2-\tfrac{1}{7}\peneSv} \in \naRa{(\xdfCw[],\Lambda)}$;
\item\label{freq:ge:strat:uk:ma:as2}
  $\sup\limits_{\xdf\in\rwCxdf}\E\sum\limits_{\Di=1}^{\ssY}\peneSv\P_{\rY \vert \rE}\big(\Vnormlp{\hxdfPr-\dxdfPr}^2\geq\tfrac{1}{7}\peneSv\big) \in \naRa{(\xdfCw[],\Lambda)}$;
\item\label{freq:ge:strat:uk:ma:as3}
  $\sup\limits_{\xdf\in\rwCxdf}\E\P_{\rY \vert \rE}\big(\Vnormlp{\hxdfPr[\mdDi]-\dxdfPr[\mdDi]}^2\geq\tfrac{1}{7}\peneSv[(\mdDi)]\big) \in \naRa{(\xdfCw[],\Lambda)}$.
\end{resListeN}
%\begin{resListeN}
%\item\label{freq:ge:strat:uk:ma:as1}
%$\sup_{\xdf\in\rwCxdf}\E\sum\limits_{\Di=1}^{\ssY}\E_{\rY \vert \rE}\vectp{\Vnormlp{\hxdfPr-\dxdfPr}^2-\tfrac{1}{7}\peneSv}\leq
%\cst{}\ssY^{-1}\big[(1\vee\meiSv[\Di_{\edf,\xdfCr}])\Di_{\edf,\xdfCr}+(1\vee\meiSv[\ssY_{o}]\ssY_{o})\big]$;
%\item\label{freq:ge:strat:uk:ma:as2}
%  $\sup\limits_{\xdf\in\rwCxdf}\E\sum\limits_{\Di=1}^{\ssY}\peneSv\P_{\rY \vert \rE}\big(\Vnormlp{\hxdfPr-\dxdfPr}^2\geq\tfrac{1}{7}\peneSv\big)\leq\cst{}\ssY^{-1}\big[(1\vee\meiSv[\Di_{\edf,\xdfCr}]^2)\Di_{\edf,\xdfCr}^2+(1\vee\meiSv[\ssY_{o}]^2\ssY_{o}^2)\big]$;
%\item\label{freq:ge:strat:uk:ma:as3}
%  $\sup\limits_{\xdf\in\rwCxdf}\E\P_{\rY \vert \rE}\big(\Vnormlp{\hxdfPr[\mdDi]-\dxdfPr[\mdDi]}^2\geq\tfrac{1}{7}\peneSv[(\mdDi)]\big)\leq    \cst{}  \big[\exp\big(\tfrac{-\cmeiSv[\mdDi]\mdDi}{200\Vnorm[{\xdfCw[]}]{\edf}\xdfCr}\big)
%    +\ssY^{-1}\big]$.
%\end{resListeN}  
\reEnd
\end{lm}
% --------------------------------------------------------------------
% <<Te penalty>>
% --------------------------------------------------------------------
\begin{te}Consider now the fully data-driven aggregation of the
  orthogonal series estimators using either  aggregation weights $\erWe[]$
  as in \eqref{freq:ge:shape:uk:we} or model selection weights $\widehat{\P}_{M}^{(\infty)}$ as in \nref{freq:ge:shape:uk:de:msWe}
  combining \nref{freq:ge:strat:uk:ma:as} and the upper bound given
  in \eqref{co:agg:au:ag}  or \eqref{co:agg:au:ms} we obtain the next result. 
\end{te}
% ....................................................................
% <<Re upper bound ag>>
% ....................................................................
\begin{lm}\label{au:mrb:ag:ub}
Assume that \nref{freq:ge:strat:uk:qu:as} holds true and consider the penalty sequence $\peneSv:=\peneD$, $\Di\in\nset{1,n}$, as in \nref{freq:ge:shape:uk:de:pen:oo}.
  Let$\widehat{\theta}^{(\eta)}=\sum_{\Di=1}^{\ssY} \erWe\hxdfPr$ be an aggregation of the orthogonal series estimators using either aggregation weights $\erWe[]$ as in \eqref{freq:ge:shape:uk:we} or model selection weights $\msWe[]$ as in \nref{freq:ge:shape:uk:de:msWe}.
  There is a finite numerical constant $\cst{}>0$ such that for all $\ssY,\ssE\in\Nz$, for any $\xdf\in\rwCxdf$, any $\mdDi,\pdDi\in\nset{1,n}$ and associated $\mDi\in\nset{1,n}$ as  defined in \eqref{au:de:*Di:ag} hold
    \begin{multline}\label{au:mrb:ag:ub:e1}
  \E\Vnormlp{\widehat{\theta}^{(\eta)}-\xdf}^2\leq
  2\penSv[\pdDi] +\tfrac{12}{7}\Vnormlp{\xdf_{\underline{0}}}^2\bias[\pdDi]^2(\xdf)+3 \Vnormlp{\xdf_{\underline{0}}}^2\bias[\mDi]^2(\xdf)\\\hfill
    + \cst{}\big[
    \Vnormlp{\xdf_{\underline{0}}}^2\Ind{\{\mDi>1\}} \FuVg[\ssE]{\rE}(\aixEv[\mdDi]^c)+\ssE\FuVg[\ssE]{\rE}(\aixEv[\pdDi]^c) \big]
    \\\hfill
    +\cst{}\mRa{\xdf,\Lambda}
    +\cst{}\ssY^{-1}\{\miSv[\Di_{\edf,\xdfCr}]^2\Di_{\edf,\xdfCr}^3+\miSv[\ssY_{o}]^2+\Vnormlp{\xdf_{\underline{0}}}^2\Ind{\{\mDi>1\}}\}
  \end{multline}
\end{lm}
% ....................................................................
% <<Te upper bound ag p np>>
% ....................................................................
\begin{te}The last bound allows us to derive an upper bound of the
  maximal risk over the ellipsoid $\rwCxdf$ for the 
  fully data-driven aggregated estimator.
\end{te}
% ....................................................................
% <<Re upper bound ag p np>>
% ....................................................................
\begin{thm}\label{au:mrb:ag:ub:pnp}
  Consider the   penalty sequence $\peneSv:=\peneD$,
  $\Di\in\nset{1,\ssY}$, as in \nref{freq:ge:shape:uk:de:pen:oo} with numerical
  constant $\cpen\geq84$. Let
  $\hxdfAg[{\erWe[]}]=\sum_{\Di=1}^{\ssY} \erWe\hxdfPr$ be an
  aggregation of the orthogonal series estimators using either
  aggregation weights $\erWe[]$ as in \eqref{freq:ge:shape:uk:we} or model
  selection weights $\msWe[]$ as in \nref{freq:ge:shape:uk:de:msWe}. Let
  $\dr\Di_{\edf,\xdfCr}:=\floor{3(400\Vnorm[{\xdfCw[]}]{\edf}\xdfCr)^2}$
  and $\dr \ssY_{o}:=15({600})^4$. Let
  $\ssE(\Lambda):=\floor{289\log(3)\cmiSv[1]\miSv[1]}\in\Nz$. If
  $\ssE>\ssE(\Lambda)$ then set
  $\sDi{\ssE}:=\max\{\Di\in\nset{1,\ssE}:289\log(\Di+2)\cmiSv[\Di]\miSv[\Di]\leq\ssE\}$
  where the defining set, respectively, contains $1$ and thus is not
  empty.  There is a numerical constant $\cst{}$ such that for all
  $\ssY\in\Nz$ with $\aDi{\ssY}:=\aDi{\ssY}(\xdf)\in\nset{1,n}$ as in
  \nref{freq:ge:shape:uk:de:pen:oo} and for all $\ssE>\ssE(\Lambda)$ holds
  \begin{multline}\label{au:mrb:ag:ub:pnp:e1}
    \nmRi{\hxdfAg[{\erWe[]}]}{\rwCxdf,\Lambda}  
    \leq\cst{}(1\vee\xdfCr^2)\min_{\Di\in\nset{1,\ssY}}
    \{\daRa{\Di}{(\xdfCw[],\Lambda)}\vee
    \exp\big(\tfrac{-\cmiSv[\Di]\Di}{\Di_{\edf,\xdfCr}}\big)\} \\\hfill
    +\cst{}(1\vee\xdfCr^2)\{\xdfCw[(\aDi{\ssY}\wedge\sDi{\ssE})]^2\vee
    \exp\big(\tfrac{-\cmiSv[\sDi{\ssE}]\sDi{\ssE}}{\Di_{\edf,\xdfCr}}\big)\}\\\hfill
    +\cst{}\xdfCr^2\mmRa{\xdfCw[],\Lambda}   + \cst{}(1\vee\xdfCr^2)\miSv[1]^2\ssE^{-1}  
    +\cst{}\{\miSv[\Di_{\edf,\xdfCr}]^2\Di_{\edf,\xdfCr}^3+\miSv[\ssY_{o}]^2\}\ssY^{-1}
  \end{multline}
  while for $\ssE\in\nset{1,\ssE(\Lambda)}$ we have
  \begin{multline}\label{au:mrb:ag:ub:pnp:e2}
    \nmRi{\hxdfAg[{\erWe[]}]}{\rwCxdf,\Lambda}  
    \leq \cst{}\xdfCr^2\mmRa{\xdfCw[],\Lambda}\\
    + \cst{}(1\vee\xdfCr^2)\miSv[1]^2\ssE^{-1}  
    +\cst{}\{\miSv[\Di_{\edf,\xdfCr}]^2\Di_{\edf,\xdfCr}^3+\miSv[\ssY_{o}]^2\}\ssY^{-1}.
  \end{multline}
\end{thm}


\begin{cor}\label{au:mrb:ag:ub2:pnp}
  Let the assumptions of \nref{au:mrb:ag:ub:pnp} be satisfied.
    If  \ref{ge:ak:ag:ub2:pnp:npc} as in \nref{ge:ak:ag:ub2:pnp} and \ref{ge:au:ag:ub2:pnp:pc:b}
    as in \nref{ge:au:ag:ub2:pnp} hold true, then there is a constant $\cst{\xdfCw[],\xdfCr,\Lambda}$ depending
    only on $\xdfCw[]$, $\xdfCr$ and $\Lambda$ such that $\nmRi{\hxdfAg[{\erWe[]}]}{\rwCxdf,\Lambda}
    \leq \cst{\xdfCw[],\xdfCr,\Lambda}\{\naRa{(\xdfCw[],\Lambda)}+\mmRa{\xdfCw[],\Lambda}+\xdfCw[(\sDi{\ssE}\wedge\aDi{\ssY})]^2\}$ for all $\ssY,\ssE\in\Nz$ holds true.
\end{cor}
% ....................................................................
% <<Rem upper bound ag p np>>
% ....................................................................
\begin{il}\label{au:mrb:ag:ub:pnp:il} As in \nref{au:ag:ub:pnp:il}
  shown in both cases \ref{au:il:edf:o} and \ref{au:il:edf:s} is
  \ref{ge:au:ag:ub2:pnp:pc:b} satisfied.  The fully data-driven
  aggregation
  leads to an estimator attaining
  the rate $\naRa{(\xdf,\Lambda)}+\mmRa{\xdfCw[],\Lambda}$ (\nref{au:mrb:ag:ub2:pnp}),
  if also  \ref{ge:au:ag:ub2:pnp:pc:b} is satisfied and
  $\xdfCw[(\sDi{\ssE})]^2$ is negligible with
  respect to $\mmRa{\xdfCw[],\Lambda}$, otherwise the upper
  bound faces a deterioration of the rate, which we illustrate
  considering as in \nref{freq:ge:strat:kn:qu:il:rate} the usual behaviour
  \ref{freq:ge:strat:kn:qu:il:rate:np:oo}, \ref{freq:ge:strat:kn:qu:il:rate:np:os} and
  \ref{freq:ge:strat:kn:qu:il:rate:np:so} for the sequences
  $\Nsuite[\Di]{\xdfCw[(\Di)]^2}$ and $\Nsuite[\Di]{\iSv[\Di]}$.
  In all three cases \ref{freq:ge:strat:kn:qu:il:rate:np:oo}, \ref{freq:ge:strat:kn:qu:il:rate:np:os} and
  \ref{freq:ge:strat:kn:qu:il:rate:np:so} the assumption \ref{ge:au:ag:ub2:pnp:pc:b}
  holds true. Moreover, in case \ref{freq:ge:strat:kn:qu:il:rate:np:oo},
  \ref{freq:ge:strat:kn:qu:il:rate:np:os} and \ref{freq:ge:strat:kn:qu:il:rate:np:so} only with
  $p<1/2$ the assumption \ref{ge:ak:ag:ub2:pnp:npc} is satisfied, and
  $\naRa{(\xdfCw[],\Lambda)}$ equals the oracle rate $\mnRa{\xdfCw[],\Lambda}$ (cf.
  \nref{freq:ge:strat:kn:qu:il:rate:np:oo} \ref{freq:ge:strat:kn:qu:il:rate:np:oo},
  \ref{freq:ge:strat:kn:qu:il:rate:np:os} and \ref{freq:ge:strat:kn:qu:il:rate:np:so}). In case
  \ref{freq:ge:strat:kn:qu:il:rate:np:os} and \ref{freq:ge:strat:kn:qu:il:rate:np:so}
  $\xdfCw[(\sDi{\ssE})]^2\leq \cst{\xdfCw[],\xdfCr,\Lambda} \mmRa{\xdfCw[],\Lambda}$ while
  in case \ref{freq:ge:strat:kn:qu:il:rate:np:oo}
  $\xdfCw[(\sDi{\ssE})]^2\sim(\ssE/\log\ssE)^{-p/a}$, hence 
  \begin{Liste}[]
  \item[\mylabel{au:mrb:ag:ub:pnp:il:oo}{\dg\bfseries{[o-o]}}]
    $\nmRi{\hxdfAg[{\erWe[]}]}{\rwCxdf,\Lambda}
    \leq
    \cst{\xdfCw[],\xdfCr,\Lambda}\{\ssY^{-2p/(2p+2a+1)}+\ssE^{-(p\wedge a)/a}+
    (\ssE/\log\ssE)^{-p/a}\}$
      \item[\mylabel{au:mrb:ag:ub:pnp:il:os}{\dg\bfseries{[o-s]}}]
    $\nmRi{\hxdfAg[{\erWe[]}]}{\rwCxdf,\Lambda}
    \leq
    \cst{\xdfCw[],\xdfCr,\Lambda}\{(\log \ssY)^{-p/a}+(\log \ssE)^{-p/a}\}$
  \item[\mylabel{au:mrb:ag:ub:pnp:il:so}{\dg\bfseries{[s-o]}}]
    $\nmRi{\hxdfAg[{\erWe[]}]}{\rwCxdf,\Lambda}
    \leq
    \cst{\xdfCw[],\xdfCr,\Lambda}\{(\log\ssY)^{(2a+1)/(2p\wedge1)}\ssY^{-1}+\ssE^{-1}\}$
  \end{Liste}
  Consequently, the fully data-driven estimator attains the minimax
  rate in case \ref{freq:ge:strat:kn:qu:il:rate:np:oo} with $p>a$,
  \ref{freq:ge:strat:kn:qu:il:rate:np:os} and \ref{freq:ge:strat:kn:qu:il:rate:np:so} with
  $p\leq1/2$, while in  case \ref{freq:ge:strat:kn:qu:il:rate:np:oo} with $p\leq a$ and
  \ref{freq:ge:strat:kn:qu:il:rate:np:so} with $p>1/2$ the rate of the fully data-driven estimator
  $\hxdfAg[{\erWe[]}]$ features a deterioration by a logarithmic factor
  $(\log\ssE)^{p/a}$  and $(\log\ssY)^{(2a+1)(1-1/(2p))}$,
  respectively, compared to the minimax  rate.\ilEnd
\end{il}
\section{First application example: the inverse Gaussian sequence space model}\label{2.4}

In this section, we consider the inverse Gaussian sequence space model and use the methodology described in \nref{2.3} to compute upper bounds of the Gaussian sieve priors described in \nref{2.1} when applied to this specific model.
Doing so, we will notice that it gives us, for a very general case, the same speed as the convergence rate of projection estimators and that, by choosing properly the threshold parameter, we reach the oracle rate of convergence as well as the minimax optimal rate, \textbf{without a $\boldsymbol{\log}$-loss}.

Then, using a methodology similar to \ncite{JJASRS} we show that under some regularity conditions, the iterated hierarchical prior leads to optimal posterior contraction rate.
As a consequence, we can conclude about the oracle and minimax optimality of the penalised contrast model selection estimator with a new strategy of proof.

\subsection{Contraction rate for sieve priors}\label{2.4.2}
Considering this model, we use a Gaussian sieve prior for $\theta$ as described in \nref{2.1} and inquire the behaviour of the posterior distribution under the asymptotic $n \rightarrow \infty$.
To sum up our setting we have:
\begin{alignat*}{4}
& \text{noise level:} \quad && n && \in && \N; \\
& \text{parameter of interest:} \quad && \theta^{\circ} = (\theta^{\circ}_{j})_{j \in \N} && \in && \mathds{L}^{2}(\R^{\N}); \\
& \text{convolution operator:} \quad && \lambda = (\lambda_{j})_{j \in \N} && \in && \R^{\N}; \\
& \text{noise sequence:} \quad && \xi = (\xi_{j})_{j \in \N} && \sim_{i.i.d.} && \mathcal{N}(0, 1); \\
& \text{observation:} \quad && Y^{n} = (Y_{j}^{n})_{j \in \N} && = && \theta^{\circ}_{j} \lambda_{j} + \frac{1}{\sqrt{n}} \xi_{j}; \\
& \text{threshold sequence:} \quad && m_{n} && \in && \N^{\N}; \\
& \text{prior guess} \quad && \boldsymbol{\theta}^{m_{n}} = (\boldsymbol{\theta}^{m_{n}}_{j})_{j \in \N} && \sim && \mathcal{N}(0, 1) \mathds{1}_{j \leq m_{n}} + \delta_{0} \mathds{1}_{j < m_{n}}. \\
\end{alignat*}

We are in a conjugated case and the iterated posterior is easily derived.
Define for any $j$ in $\N$ and $\eta$ in $\N^{\star}$ the quantities

\[\widehat{\theta}^{(\eta)}_{j} := \frac{n \eta Y^{n}_{j} \lambda_{j}}{1 + n \eta \lambda_{j}^{2}}; \quad \sigma^{(\eta)}_{j} := \frac{1}{1 + n \eta \lambda_{j}^{2}}.\]
Then, for any $j$ in $\N$, the posterior distribution of $\boldsymbol{\theta}_{j}$ after $\eta$ iterations is given by
\[\boldsymbol{\theta}_{j} \vert Y^{n, \eta} \sim \mathcal{N}(\widehat{\theta}^{(\eta)}_{j}, \sigma^{(\eta)}_{j}) \mathds{1}_{j \leq m_{n}} + \delta_{0}(\boldsymbol{\theta}_{j}) \mathds{1}_{j > m_{n}}.\]

According to \nref{2.1}, for $n$ fixed, if $\eta$ tends to infinity, the posterior distribution contracts around the maximiser of the constrained likelihood.
Considering the limits of $\widehat{\theta}^{(\eta)}_{j}$ and $\sigma^{(\eta)}_{j}$ as $\eta$ tells us that this maximiser is the projection estimator $\overline{\theta}^{m_{n}} = (\overline{\theta}^{m_{n}}_{j})_{j \in \N} = (\frac{Y_{j}}{\lambda_{j}} \mathds{1}_{j \leq m_{n}})_{j \in \N}$.

\medskip

We can then compute the quantities appearing in \nref{2.3} which gives the following results.

\begin{cor}\label{cor2.4.1}
For any $\theta^{\circ}$ in $\Theta$ and increasing, unbounded sequence $c_{n}$, we have
\begin{alignat*}{3}
& && \lim\limits_{n \rightarrow \infty} \E_{\theta^{\circ}}^{n}&&\left[\P_{\boldsymbol{\theta}^{m_{n}}\vert Y^{n}}^{n, (\eta)}\left(d^{2}\left(\theta^{\circ}, \boldsymbol{\theta}^{m_{n}}\right) \leq c_{n} \Phi^{m_{n}}_{n} \right)\right] = 1.
\end{alignat*}
\end{cor}

\begin{pro}\label{pro2.4.1}
We want to find a sequence $\left(K_{n}\right)_{n \in \mathds{N}}$ (for short, $K_{n}$) converging to $0$ such that
\[\lim\limits_{n \rightarrow \infty} \E_{\theta^{\circ}}^{n}\left[\P_{\boldsymbol{\theta}^{m_{n}}\vert Y^{n}}^{n}\left(\Vert \boldsymbol{\theta}^{m_{n}} - \theta^{\circ} \Vert^{2} \geq K_{n}\right)\right] = 0.\]

For any $n$, we define $S^{m_{n}} := \sum_{j=1}^{m_{n}} \left(\boldsymbol{\theta}^{m_{n}} - \theta^{\circ}\right)^{2}$. Therefore we have : 
\[\E_{\theta^{\circ}}^{n}\left[\P_{\boldsymbol{\theta}^{m_{n}}\vert Y^{n}}^{n}\left(\Vert \boldsymbol{\theta}^{m_{n}} - \theta^{\circ} \Vert^{2} \geq K_{n}\right)\right] = \E_{\theta^{\circ}}^{n}\left[\P_{\boldsymbol{\theta}^{m_{n}}\vert Y^{n}}^{n}\left(S^{m_{n}} \geq K_{n} - \mathfrak{b}_{m_{n}}\right)\right].\]

By definition, $S^{m_{n}}$ has finite expectation and strictly positive, finite variance. We define $\mathcal{S}^{m_{n}} := \frac{S^{m_{n}} - \E_{\boldsymbol{\theta}^{m_{n}} \vert Y^{n}}^{n}\left[S^{m_{n}}\right]}{\sqrt{\V_{\boldsymbol{\theta}^{m_{n}}\vert Y^{n}}^{n}[S^{m_{n}}]}}.$ The sequence of random variables defined this way is tight as their expectations are all equal to $0$ and their variances to $1$. We now have to look for a contraction rate for this new family of random variables as we have :

\[\E_{\theta^{\circ}}^{n}\left[\P_{\boldsymbol{\theta}^{m_{n}}\vert Y^{n}}^{n}\left(\Vert \boldsymbol{\theta}^{m_{n}} - \theta^{\circ} \Vert^{2} \geq K_{n}\right)\right] = \E_{\theta^{\circ}}^{n}\left[\P_{\boldsymbol{\theta}^{m_{n}}\vert Y^{n}}^{n}\left(\mathcal{S}^{m_{n}} \geq \frac{K_{n} - \mathfrak{b}_{m_{n}}^{2} - \E_{\boldsymbol{\theta}^{m_{n}} \vert Y^{n}}^{n}\left[S^{m_{n}}\right]}{\sqrt{\V_{\boldsymbol{\theta}^{m_{n}} \vert Y^{n}}^{n}\left[S^{m_{n}}\right]}}\right)\right].\]

We now control the convergence in probability of $\E_{\boldsymbol{\theta}^{m_{n}} \vert Y^{n}}^{n}\left[\mathcal{S}^{m_{n}}\right]$ and $\V_{\boldsymbol{\theta}^{m_{n}} \vert Y^{n}}^{n}\left[\mathcal{S}^{m_{n}}\right]$ which are given by
\begin{alignat*}{3}
&\E_{\boldsymbol{\theta}^{m_{n}} \vert Y^{n}}^{n}\left[S^{m_{n}}\right] &&=&& \sum\limits_{j = 1}^{m_{n}} \frac{\Lambda_{j}}{n \eta}\cdot \left(\frac{1}{\frac{\Lambda_{j}}{n \eta} + 1}\right)\left(1 + \frac{\left(- \theta^{\circ}_{j} + \eta \sqrt{n} \xi_{j} \lambda_{j}\right)^{2}}{\frac{\eta n}{\Lambda_{j}}\left(\frac{\Lambda_{j}}{\eta n} + 1\right)}\right)\\
& && \leq && \sum\limits_{j = 1}^{m_{n}} \frac{\Lambda_{j}}{n \eta} + \sum\limits_{j = 1}^{m_{n}} \frac{\Lambda_{j}^{2}}{n^{2} \eta^{2}}\left(- \theta^{\circ}_{j} + \eta \sqrt{n} \xi_{j} \lambda_{j}\right)^{2};\\
&\V_{\boldsymbol{\theta}^{m_{n}} \vert Y^{n}}^{n}\left[S^{m_{n}}\right] &&=&& 2 \sum\limits_{j = 1}^{m_{n}} \left(\frac{\Lambda_{j}}{n \eta}\cdot \frac{1}{\frac{\Lambda_{j}}{n \eta} + 1}\right)^{2}\left(1 + 2 \frac{\left(- \theta^{\circ}_{j} + \eta \sqrt{n} \xi_{j} \lambda_{j}\right)^{2}}{\frac{\eta n}{\Lambda_{j}}\left(\frac{\Lambda_{j}}{\eta n} + 1\right)}\right)\\
& &&\leq && 2 \sum\limits_{j = 1}^{m_{n}} \frac{\Lambda_{j}^{2}}{n^{2} \eta^{2}} + 4 \sum\limits_{j = 1}^{m_{n}} \frac{\Lambda_{j}^{3}}{n^{3} \eta^{3}} \left(- \theta^{\circ}_{j} + \eta \sqrt{n} \xi_{j} \lambda_{j}\right)^{2}.
\end{alignat*}

We now control the stochastic parts of those moments.

\textcolor{blue}{
Define for some sequence $(u_{n})_{n \in \mathds{N}}$ , tending to $0$ and any deterministic sequence $(a_{j})$ the event $\Omega_{m_{n}} := \left\{\sum\limits_{j=1}^{m_{n}}a_{j}\left(- \theta^{\circ}_{j} + \eta \sqrt{n} \xi_{j} \lambda_{j}\right)^{2} \leq u_{n}\right\}.$\\
Obviously, $\Omega_{m_{n}}^{c} = \left\{\sum\limits_{j=1}^{m_{n}} a_{j} \left( - \theta^{\circ}_{j} + \eta \sqrt{n} \xi_{j} \lambda_{j}\right)^{2} \geq u_{n}\right\}$ has probability
\[\P_{\theta^{\circ}}^{n}\left(\Omega_{m_{n}}^{c}\right) = \P_{\theta^{\circ}}^{n}\left(\sum\limits_{j=1}^{m_{n}}\left(- \theta^{\circ}_{j} + \eta \sqrt{n} \xi_{j} \lambda_{j}\right)^{2} \geq u_{n}\right).\]
In the same spirit as previously, we define the sequence of random variables $T^{m_{n}} := \sum\limits_{j=1}^{m_{n}}a_{j}\left(- \theta^{\circ}_{j} + \eta \sqrt{n} \xi_{j} \lambda_{j}\right)^{2}.$
We have
\begin{alignat*}{3}
&\E_{\boldsymbol{\theta}^{m_{n}} \vert Y^{n}}^{n}\left[T^{m_{n}}\right] &&=&& \sum\limits_{j = 1}^{m_{n}} a_{j} \frac{\eta^{2} n}{\Lambda_{j}} \left[1 + \frac{\Lambda_{j}}{n \eta^{2}} \left(\theta^{\circ}_{j} \right)^{2}\right]\\
& &&\leq&& \sum\limits_{j = 1}^{m_{n}} a_{j} \frac{\eta^{2} n}{\Lambda_{j}} \left[1 + \frac{\Lambda_{1}}{\eta^{2}} \left(\theta^{\circ}_{j}\right)^{2}\right];\\
&\V_{\boldsymbol{\theta}^{m_{n}} \vert Y^{n}}^{n}\left[T^{m_{n}}\right] &&=&& 2 \sum\limits_{j = 1}^{m_{n}} a_{j}^{2} \left(\frac{\eta^{2} n}{\Lambda_{j}}\right)^{2} \left[1 + 4 \frac{\Lambda_{j}}{\eta^{2} n} \left(\theta^{\circ}_{j}\right)^{2}\right]\\
& &&\leq&& 2 \sum\limits_{j = 1}^{m_{n}} a_{j}^{2} \left(\frac{\eta^{2} n}{\Lambda_{j}}\right)^{2} \left[1 + 4 \frac{\Lambda_{1}}{\eta^{2}} \left(\theta^{\circ}_{j}\right)^{2}\right];
\end{alignat*}
and the sequence of random variables $\mathcal{T}^{m_{n}} := \frac{T^{m_{n}} - \E_{\boldsymbol{\theta}^{m_{n}} \vert Y^{n}}^{n}\left[T^{m_{n}}\right]}{\sqrt{\V_{\boldsymbol{\theta}^{m_{n}} \vert Y^{n}}^{n}\left[T^{m_{n}}\right]}}$ is tight.
Therefore, $\P_{\theta^{\circ}}^{n}\left(\sum\limits_{j=1}^{m_{n}}\left(- \theta^{\circ}_{j} + \eta \sqrt{n} \xi_{j} \lambda_{j}\right)^{2} \geq u_{n}\right) = \P_{\theta^{\circ}}^{n}\left(\mathcal{T}^{m_{n}} \geq \frac{u_{n} - \E_{\boldsymbol{\theta}^{m_{n}} \vert Y^{n}}^{n}\left[T^{m_{n}}\right]}{\sqrt{\V_{\boldsymbol{\theta}^{m_{n}} \vert Y^{n}}^{n}\left[T^{m_{n}}\right]}}\right)$.
Consider any sequence $(c_{n})$ diverging to infinity.
Then if
\begin{alignat*}{3}
&u_{n} &&=&& \sqrt{\V_{\boldsymbol{\theta}^{m_{n}} \vert Y^{n}}^{n}\left[T^{m_{n}}\right]} c_{n} + \E_{\boldsymbol{\theta}^{m_{n}} \vert Y^{n}}^{n}\left[T^{m_{n}}\right]\\
& &&=&& c_{n} \cdot \sqrt{2 \sum\limits_{j = 1}^{m_{n}} a_{j}^{2} \left(\frac{\eta^{2} n}{\Lambda_{j}}\right)^{2} \left[1 + 4 \frac{\Lambda_{j}}{\eta^{2} n} \left(\theta^{\circ}_{j} \right)^{2}\right]} + \sum\limits_{j = 1}^{m_{n}} a_{j} \frac{\eta^{2} n}{\Lambda_{j}} \left[1 + \frac{\Lambda_{j}}{n \eta^{2}} \left(\theta^{\circ}_{j} \right)^{2}\right].
\end{alignat*}
Then $\P_{\theta^{\circ}}^{n}(\Omega_{m_{n}}^{c}) \leq  \P_{\theta^{\circ}}^{n}\left(\mathcal{T}^{m_{n}} \geq c_{n}\right) \rightarrow 0$ as $\mathcal{T}^{m_{n}}$ is tight.
}

\medskip

We can now conclude about the posterior contraction by defining
\begin{alignat*}{3}
&K_{n} &&:=&& \mathfrak{b}_{m_{n}}^{2} +  \sum\limits_{j = 1}^{m_{n}} \frac{\Lambda_{j}}{n \eta}\cdot \left(\frac{1}{\frac{\Lambda_{j}}{n \eta} + 1}\right)\left(1 + \frac{u_{n}}{\frac{\eta n}{\Lambda_{j}}\left(\frac{\Lambda_{j}}{\eta n} + 1\right)}\right)\\
& && && + c_{n} \cdot \sqrt{2 \sum\limits_{j = 1}^{m_{n}} \left(\frac{\Lambda_{j}}{n \eta}\cdot \frac{1}{\frac{\Lambda_{j}}{n \eta} + 1}\right)^{2}\left\{1 + 2 \frac{u_{n}}{\frac{\eta n}{\Lambda_{j}}\left(\frac{\Lambda_{j}}{\eta n} + 1\right)}\right\}}\\
& &&=&& \mathcal{O}\left(c_{n} \cdot \frac{m_{n} \overline{\Lambda}_{m_{n}}}{n \eta} \vee \mathfrak{b}_{m_{n}}^{2}\right)
\end{alignat*}


Indeed :
\begin{alignat*}{3}
&\E_{\theta^{\circ}}^{n}\left[\P_{\boldsymbol{\theta}^{m_{n}}\vert Y^{n}}^{n}\left(\Vert \boldsymbol{\theta}^{m_{n}} - \theta^{\circ} \Vert^{2} \geq K_{n}\right)\right] && \leq &&\E_{\theta^{\circ}}^{n}\left[\mathds{1}_{\Omega{n}}\P_{\theta^{m_{n}} \vert Y^{n}}^{n}\left(\Vert \boldsymbol{\theta}^{m_{n}} - \theta^{\circ} \Vert^{2} \geq K_{n}\right)\right] + \P_{\theta^{\circ}}(\Omega{n}^{c})\\
& &&\leq &&\E_{\theta^{\circ}}^{n}\left[\P_{\theta^{m_{n}}\vert Y^{n}}^{n}\left(\mathcal{S}^{m_{n}} \geq c_{n} \right)\right] + \P_{\theta^{\circ}}(\Omega_{n}^{c})\\
\end{alignat*}
Which tends to $0$ as $\mathcal{S}^{m_{n}}$ is a tight sequence of random variables.
One could notice that if $\eta$ diverges to infinity, the sequence $c_{n}$ cancels and we recover the frequentist $\mathds{L}_{2}$ rate of convergence for projection estimators.
\end{pro}

Notice that if one selects $m_{n} = m_{n}^{\circ}$ we obtain the oracle rate of convergence of projection estimators.




\begin{cor}\label{cor2}
For any increasing and unbounded sequence, we have
\begin{alignat*}{5}
& \lim_{n \rightarrow \infty}&& && \inf\limits_{\theta^{\circ} \in \Theta_{\mathfrak{a}}(r)} && \E_{\theta^{\circ}}^{n}\left[\P_{\boldsymbol{\theta}^{m_{n}^{\star}}\vert Y^{n}}^{n, (\eta)}\left(d^{2}\left(\theta^{\circ}, \boldsymbol{\theta}\right) \leq c_{n} \cdot \Psi^{\star}_{n}(\Theta_{\mathfrak{a}}(r)) \right)\right]&& = 1;
\end{alignat*}
\end{cor}

Moreover, if one lets the number of iterations tend to infinity, we observe that the distribution degenerates around the projection estimator as defined in \textsc{\cref{2.1}}:
\[\lim\limits_{\eta \rightarrow \infty} \P_{\boldsymbol{\theta}^{m} \vert Y^{n}}^{n, (\eta)} = \delta_{\overline{\theta}^{m}}.\]


\subsection{Contraction rate for the hierarchical prior}\label{2.4.3}

Let be $G_{n} := \max\left\{m \in \llbracket 1, n \rrbracket : \Lambda_{m} / n \leq \Lambda_{1}\right\}$.
We give the following specific shape to the prior for the threshold parameter $\P_{M}^{n}(M = m) = \frac{\exp\left(-3 \cdot \eta \cdot \frac{m}{2} \right) \cdot \prod\limits_{j = 1}^{m} \left(\frac{1}{\sigma_{j}^{(\eta)}}\right)^{2}}{\sum\limits_{k =1}^{G_{n}} \exp\left(-3 \cdot \eta \cdot \frac{k}{2} \right) \cdot \prod\limits_{j = 1}^{k} \left(\frac{1}{\sigma_{j}^{(\eta)}}\right)^{2}}$ with $\sigma_{j}^{(\eta)}$ as defined in \textsc{\cref{2.4.2}}.

Hence, for all $m$ in $\llbracket 1, G_{n} \rrbracket$, the posterior distribution are characterised by :
\[\P_{M \vert Y^{n}}^{n, (\eta)}(m) = \frac{\exp\!\!\left[- \frac{1}{2} \left( 3 m \eta - \Vert \widehat{\theta}^{m, (\eta)} \Vert_{\sigma^{m, (\eta)}}^{2} \right)\right] }{\sum\limits_{k = 1}^{G_{n}} \exp\!\!\left[ - \frac{1}{2} \left( 3 k \eta - \Vert \widehat{\theta}^{k, (\eta)} \Vert_{\sigma^{k, (\eta)}}^{2}\right) \right]},\]
and
\[\P_{\boldsymbol{\theta}^{M} \vert Y^{n}}^{n, (\eta)} = \sum\limits_{m \in \mathds{N}^{\star}}\P_{\boldsymbol{\theta}^{m} \vert Y^{n}}^{n, (\eta)} \cdot \P_{M \vert Y^{n}}^{n, (\eta)}(m);\]
and the posterior mean is then $\widehat{\theta}^{M, (\eta)} :=  \sum\limits_{m \in \mathds{N}^{\star}} \widehat{\theta}^{m, (\eta)} \P_{M \vert Y^{n}}^{n, (\eta)}(m) = \left(\widehat{\theta}_{j}^{(\eta)} \cdot \P_{M \vert Y^{n}}^{n, (\eta)} \left(M \geq j\right)\right)_{j \in \mathds{N}^{\star}}.$

As we have seen previously with the sieve priors, the iteration procedure conserves the contraction rate.

\begin{cor}\label{cor3}
Under \textsc{\cref{as1}} and \textsc{\cref{as2}}, if, in addition $\log(G_{n})/m_{n}^{\circ} \rightarrow 0$ as $n \rightarrow \infty$ then with $D^{\circ} := D^{\circ}(\theta^{\circ}, \lambda) = \lceil 5 L/\kappa^{\circ} \rceil$ and $K^{\circ} := 10(2 \vee \Vert \theta^{\circ} \Vert^{2})L^{2}(16 \vee D^{\circ} \Lambda_{D^{\circ}})$ we have, for any $\eta$ ($1 \leq \eta < \infty$):
\[\lim\limits_{n \rightarrow \infty} \E_{\theta^{\circ}}^{n}\left[\P_{\boldsymbol{\theta}^{M} \vert Y^{n}}^{n, (\eta)} \left(\left(K^{\circ}\right)^{-1} \Phi_{n}^{\circ} \leq \Vert \theta^{\circ} - \boldsymbol{\theta}^{M} \Vert_{l^{2}}^{2} \leq K^{\circ} \Phi_{n}^{\circ}\right)\right] = 1.\]
\end{cor}

\begin{cor}\label{cor4}
Under \textsc{\cref{as1}} and \textsc{\cref{as3}}, if, in addition, $\log(G_{n})/m_{n}^{\star} \rightarrow 0$ as $n \rightarrow \infty$ then, for any $\eta$ ($1 \leq \eta < \infty$)
\begin{itemize}
\item for all $\theta^{\circ}$ in $\Theta_{\mathfrak{a}}(r)$, with $D^{\star} := D^{\star}(\mathfrak{a}, \lambda) = \lceil 5 L/\kappa^{\star} \rceil$ and $K^{\star} := 16\left(2 \vee r\right)L^{2}\left(16 \vee D^{\star} \Lambda_{D^{\star}}\right)\left(1 \vee r \right)$, we have
\[\lim\limits_{n \rightarrow \infty} \E_{\theta^{\circ}}^{n}\left[\P_{\boldsymbol{\theta}^{M} \vert Y^{n}}^{n, (\eta)}\left(\Vert \theta^{\circ} - \boldsymbol{\theta}^{M} \Vert^{2} \leq K^{\star} \Phi_{n}^{\star}\right)\right] =1;\]
\item for any monotonically increasing and unbounded sequence $K_{n}$ holds
\[\lim\limits_{n \rightarrow \infty} \inf\limits_{\theta^{\circ} \in \Theta_{\mathfrak{a}}(r)} \E_{\theta^{\circ}}^{n}\left[\P_{\boldsymbol{\theta}^{M} \vert Y^{n}}^{n, (\eta)}\left(\Vert \theta^{\circ} - \boldsymbol{\theta}^{M} \Vert^{2} \leq K_{n} \Phi_{n}^{\star}\right)\right] =1.\]
\end{itemize}
\end{cor}


Now, in this adaptive case, we consider the eventuality of letting $\eta$ tend to infinity.
In the spirit of the frequentist model selection method presented in \textsc{\cref{2.3}}, define $\Upsilon_{\eta}(m) = - \sum\limits_{j = 1}^{m} \frac{1}{1 + \frac{\Lambda_{j}}{\eta n}} Y_{j}^{2}$ and $E_{\eta}(m) = \pen(m) + \Upsilon_{\eta}(m)$.

We see that for all $m$ in $\llbracket 1, G_{n} \rrbracket$,
\[\P_{M\vert Y^{n}}^{n, (\eta)}(m) = \frac{1}{\sum\limits_{k = 1}^{G_{n}}\exp\!\!\left[- \frac{\eta n}{2}\left(E_{\eta}(k) - E_{\eta}(m)\right)\right]}.\]

If $\eta$ tends to $+\infty$, for all $m$, $\Upsilon_{\eta}(m)$ tends to $\Upsilon(m) := -\sum\limits_{j = 1}^{m} \left(Y_{j}\right)^{2}$ and we define for all $m$, $E(m) := \pen(m) + \Upsilon(m)$.

\medskip

Interestingly, if we define the contrast $\Gamma$ for any sequence $\theta^{\star}$ in $\Theta$ as
\[\Gamma\left(\theta^{\star}\right) := \sum\limits_{j = 1}^{G_{n}} \left(\theta^{\star}_{j}\right)^{2}\lambda_{j}^{2} - 2 \sum\limits_{j = 1}^{G_{n}} \theta^{\star}_{j}\lambda_{j} Y_{j},\]
we see, by differentiating $\Gamma$ summand-wise, that $\overline{\theta}^{G_{n}}$ minimises this contrast and that $\Gamma\left(\overline{\theta}^{G_{n}}\right) = \Upsilon\left(G_{n}\right)$.

\medskip

If for all $k$ different from $m$, $E(k) - E(m) > 0$, then $\P_{M\vert Y^{(n)}}^{n,(\eta)}(m)$ trivially tends to $1$ as $\eta$ tends to $\infty$. On the other hand, if there exists $k$ such that $E(k) - E(m) < 0$,  then $\P_{M\vert Y^{n}}^{n, (\eta)}(m)$ obviously tends to $0$ as $\eta$ tends to $\infty$. So we see that, similarly to the model selection, this method only selects threshold parameters that minimise a penalised contrast.

\medskip

Note that for all distinct $k$ and $m$ in $\llbracket 1, G_{n} \rrbracket$, we almost surely have $E(k) - E(m) \neq 0$ since $\Upsilon(k) - \Upsilon(m)$ is a random variable with absolutely continuous distribution with respect to Lebesgue measure and hence, $\P_{\theta^{\circ}}\!\!\left[\{\Upsilon(k) - \Upsilon(m) = \pen(k) - \pen(m)\}\right] = 0$.

We hence define $\widehat{m} := \argmin\limits_{m \in \llbracket 1, G_{n} \rrbracket}\{E(m)\}$ and $\overline{\theta}^{\widehat{m}}$ the associated projection estimator. Hence, the self informative Bayes limit is $\overline{\theta}^{\widehat{m}}$ and the self informative Bayes carrier is degenerated on it: $\P_{\boldsymbol{\theta}^{M} \vert Y^{n}}^{n, (\infty)} = \delta_{\overline{\theta}^{\widehat{m}}}$.

\medskip

We obtain here optimality results both for the self informative limit and self informative Bayes carrier.

\begin{thm}\label{thm3}
Consider $\overline{\theta}^{\widehat{m}}$ the frequentist estimator given by the self-informative limit.
Under \textsc{\cref{as1}}, \textsc{\cref{as2}} and the condition that $\limsup\limits_{n \rightarrow \infty}\frac{\log\left(\frac{G_{n}^{2}}{\Phi_{n}^{\circ}}\right)}{m_{n}^{\circ}} \leq \frac{5}{9 L}$, we have

\[\exists C^{\circ} \in \mathds{R}_{+}^{\star} : \forall \theta^{\circ} \in \Theta, \quad \E_{\theta^{\circ}}^{n}\left[\Vert \overline{\theta}^{\widehat{m}} - \theta^{\circ} \Vert^{2}\right] \leq C^{\circ} \Phi_{n}^{\circ}.\]
\end{thm}

This first theorem states that, under our set of assumptions, the self-informative limit reaches the oracle rate of the projection estimators.

\begin{thm}\label{thm4}
Under \textsc{\cref{as1}}, \textsc{\cref{as2}} and the condition that $\limsup\limits_{\epsilon \rightarrow 0} \frac{\log\left(G_{n}\right)}{m_{n}^{\circ}},$ define $D^{\circ} := \left\lceil \frac{3}{\kappa^{\circ}} + 1 \right\rceil$ and $K^{\circ} := 16 L \cdot \left[9 \vee D^{\circ} \Lambda_{D^{\circ}}\right]$; then, we have for all $\theta^{\circ}$ in $\Theta$,
\[\lim\limits_{n \rightarrow \infty} \E_{\theta^{\circ}}^{n}\left[\P_{\boldsymbol{\theta}^{M} \vert Y^{n}}^{n, (\infty)}\left(\left(K^{\circ}\right)^{-1} \Phi_{n}^{\circ} \leq \Vert \boldsymbol{\theta}^{M} - \theta^{\circ} \Vert^{2} \leq K^{\circ} \Phi_{n}^{\circ} \right)\right] = 1.\]
\end{thm}

This result states that the self informative Bayes carrier contracts with oracle optimal rate of the sieve priors under our set of assumptions.


\begin{thm}\label{thm5}
Consider $\overline{\theta}^{\widehat{m}}$ the frequentist estimator given by the self-informative limit.
Then, under \textsc{\cref{as1}}, \textsc{\cref{as3}} and the condition that $\limsup\limits_{n \rightarrow \infty}\frac{\log\left(\frac{G_{n}^{2}}{\Phi_{n}^{\star}}\right)}{m_{n}^{\star}} < \frac{5}{9 L}$, we have

\[\exists C^{\star} \in \mathds{R}_{+}^{\star} : \quad \sup\limits_{\theta^{\circ}\in \Theta}\E_{\theta^{\circ}}^{n}\left[\Vert \overline{\theta}^{\widehat{m}} - \theta^{\circ} \Vert^{2}\right] \leq C^{\star} \Psi_{n}^{\star}.\]
\end{thm}

This result shows that the self-informative limit converges with minimax optimal rate over Sobolev's ellipsoids under our set of assumptions.

\begin{thm}\label{thm6}
Under \textsc{\cref{as1}}, \textsc{\cref{as3}} and the condition that $\limsup\limits_{n \rightarrow \infty} \frac{\log\left(G_{n}\right)}{m_{n}^{\star}},$ define $D^{\star} := \left\lceil \frac{3 \left(1 \vee L^{\circ}\right)}{\kappa^{\star}} + 1 \right\rceil$ and $K^{\star} := 9 L \left(1 \vee L^{\circ} \right) D^{\star} \Lambda_{D^{\star}}$; then, we have for all $\theta^{\circ}$ in $\Theta^{\mathfrak{a}}(r)$,
\[\lim\limits_{n \rightarrow \infty} \E_{\theta^{\circ}}^{n}\left[\P_{\boldsymbol{\theta}^{M} \vert Y^{n}}^{n, (\infty)}\left(\Vert \boldsymbol{\theta}^{M} - \theta^{\circ} \Vert^{2} \leq K^{\star} \Psi_{n}^{\star} \right)\right] = 1,\]
and, for any increasing function $K_{n}$ such that $\lim\limits_{n \rightarrow \infty} K_{n} = \infty,$
\[\lim\limits_{n \rightarrow \infty} \sup\limits_{\theta^{\circ} \in \Theta^{\mathfrak{a}}(r)} \E_{\theta^{\circ}}^{n}\left[\P_{\boldsymbol{\theta}^{M} \vert Y^{n}}^{n, (\infty)}\left(\Vert \boldsymbol{\theta}^{M} - \theta^{\circ} \Vert^{2} \leq K_{n} \Psi_{n}^{\star} \right)\right] = 1.\]
\end{thm}
%\section{Second application example: the circular density deconvolution model}\label{BAYES_CIRCULARDECON}


We present here the tracks for application of our Bayesian methodologies in the context of circular density deconvolution.
Facing the challenge of non conjugation, the strategies of proof presented in \nref{2.3} are unfortunately not applicable, however, we suggest some investigation pathways.

\subsection{Pseudo-Gaussian sieve priors}\label{2.5.1}
\underline{Definition of the prior}

Let be $\left(\theta^{\times}_{j}\right)_{j \in \mathds{Z}}$ in $\mathcal{S}^{+}(\mathds{Z})$ a prior mean sequence, considered as fixed.
The Bayesian inference is then made by selecting a sequence of priors, indexed by $n$.
In particular, the sequence of prior is determined either by the selection of a dimension parameter sequence $\left(m_{n}\right)_{n \in \mathds{N}}$ or the selection of a prior variance sequence $\left(s_{j}^{n}\right)_{j \in \mathds{N}^{*}, n \in \mathds{N}}$ in $\mathds{R}_{+}^{\mathds{N}^{*} \times \mathds{N}}$.
The prior variance sequence is indexed by $\mathds{N}^{*}$ only as the nature of the object of interest constrains its shape (see \textsc{\autoref{hergoltz}}).
We will hence only be interested in the distribution of the Fourier coefficients positively indexed as they entirely determine the Fourier transform.


\begin{de}{\textsc{Sieve prior}\\}\label{de2.5.2}
Let be $\left(\frak{s}_{j}\right)_{j \in \mathds{N}^{*}}$ a prior variance sequence independant of $n$ as well as $\left(m_{n}\right)_{n \in \mathds{N}}$ in $\mathds{N}^{\mathds{N}}$ a threshold parameter sequence.
We then note $\left(\mathds{Q}_{\boldsymbol{f}^{m_{n}}}\right)_{n \in \mathds{N}}$ the sequence of pseudo-Gaussian process prior distributions such that, for any $n$ in $\mathds{N}$, the prior variance parameter is given by $\left(\frak{s}_{j} \mathds{1}_{\{j \leq m_{n}\}}\right)_{j \in \mathds{N}}$.
The density is then given by
\[\frac{d \mathds{Q}_{\boldsymbol{f}^{m_{n}}}}{d \mathds{Q}^{\circ}}(f) = \exp\left[-\frac{1}{2}\sum\limits_{j = 1}^{m_{n}} \frac{\vert \theta^{\times}_{j}\vert^{2}}{\frak{s}_{j}} + \sum\limits_{j = 1}^{m_{n}} \frac{\theta^{\times}_{j}}{\frak{s}_{j}}[f]_{j} +\Delta_{2}^{n}(f)\right].\]

We note $\mathcal{G}$ the family of all sieve prior sequences.
\end{de}


\underline{Shape of the posterior distribution}

Define the following sequences.
\begin{de}{\textsc{Pseudo-posterior mean/variance sequences} \\}\label{de2.5.3}
We define the two following sequences which can be seen as the posterior mean and variance of the Gaussian part of the posterior distribution:
\begin{alignat*}{3}
& \widehat{\theta}_{\eta, j}^{n} && := && \frac{\frac{\theta^{\times}_{j}}{\eta n} + s^{n}_{j} \overline{e}_{j}^{n} \lambda_{j}}{\frac{1}{\eta n} + s^{n}_{j} \lambda_{j}^{2}};\\
& \sigma_{\eta, j}^{n} &&:=&& \frac{s^{n}_{j}}{1 + \eta n s^{n}_{j} \lambda_{j}^{2}}.
\end{alignat*}
\end{de}

For any prior $\mathds{Q}_{\boldsymbol{f}}$ dominated by $\mathds{Q}^{\circ}$, Bayes' theorem tells us that the posterior $\mathds{Q}_{\boldsymbol{f}\vert Y^{n}}$ has density :
\begin{alignat*}{3}
&\frac{d\mathds{Q}_{\boldsymbol{f}\vert Y^{n}}}{d\mathds{Q}^{\circ}}(f, y^{n}) &&=&&\frac{\frac{d \mathds{Q}_{\boldsymbol{f}, Y^{n}}}{d \mathds{Q}^{\circ}d \mathds{P}^{\circ}}}{\frac{d \mathds{P}_{Y^{n}}}{d \mathds{P}^{\circ}}}(f, y^{n})\\
& &&=&& \frac{\frac{d \mathds{P}_{Y^{n} \vert \boldsymbol{f}}}{d \mathds{P}^{\circ}} \cdot \frac{d \mathds{Q}_{\boldsymbol{f}}}{d \mathds{Q}^{\circ}}}{\frac{d\mathds{P}_{Y^{n}}}{d\mathds{P}^{\circ}}}(f, y^{n}).
\end{alignat*}

Calculations show that, for $\mathcal{G}$, the posterior distribution is given by
\[\frac{d \mathds{Q}_{\boldsymbol{f}^{m^{n}}\vert Y^{n}}}{d \mathds{Q}^{\circ}}(f, y^{n}) \propto \exp\left[-\frac{1}{2}\sum\limits_{j = 1}^{m_{n}} \frac{\vert \widehat{\theta}_{\eta, j}^{n}\vert^{2}}{\sigma_{\eta, j}^{n}} + \sum\limits_{j = 1}^{m_{n}} \frac{\widehat{\theta}_{\eta, j}^{n}}{\sigma_{\eta, j}^{n}}[f]_{j} - \frac{n}{2} \Delta_{1}^{n}(f, y^{n}) +\Delta_{2}^{n}(f)\right];\]

and for $\mathcal{H}$
\[\frac{d \mathds{Q}_{\boldsymbol{f}^{s^{n}}\vert Y^{n}}}{d \mathds{Q}^{\circ}}(f, y^{n})\propto\exp\left[-\frac{1}{2}\sum\limits_{j = 1}^{\infty} \frac{\vert \widehat{\theta}_{\eta, j}^{n}\vert^{2}}{\sigma_{\eta, j}^{n}} + \sum\limits_{j = 1}^{\infty} \frac{\widehat{\theta}_{\eta, j}^{n}}{\sigma_{\eta, j}^{n}}[f]_{j} - \frac{n}{2} \Delta_{1}^{n}(f, y^{n}) +\Delta_{2}^{n}(f)\right].\]

\underline{Contraction rate for non adaptive priors}

From now on, we note $\widetilde{Y}^{n, s^{n}} = \left(\widetilde{Y}^{n, s^{n}}_{j}\right)_{j \in \mathds{N}^{*}}=\left(\widehat{\theta}_{\eta, j}^{n} + \sigma_{\eta, j}^{n}\xi_{j}\right)_{j \in \mathds{N}^{*}}$, the Gaussian process we try to approach; $\widetilde{\mathds{Q}}_{\boldsymbol{f}^{s^{n}} \vert Y^{n}}^{n}$ its distribution and $\mathds{E}_{\widetilde{\mathds{Q}}_{\boldsymbol{f}^{s{n}} \vert Y^{n}}^{n}}$ the expectation under this distribution.

Results on the contraction of the posterior distribution could be obtained by relating the moments of the posterior distribution with those of $\widetilde{\mathds{Q}}_{\boldsymbol{f}^{s^{n}} \vert Y^{n}}^{n}$.

We shall consider the following notations.

\begin{nota}\label{nota2.5.1}
For any $j$, $m$ and $n$ in $\mathds{N}^{*}$; and $[\mathfrak{u}]$ in $\mathds{R}_{+}^{\mathds{Z}}$ define
\begin{alignat*}{12}
& \Lambda^{[\mathfrak{u}]}_{j} &&:=&& \frac{\vert [\mathfrak{u}](j) \vert^{2}}{\vert \lambda_{j} \vert^{2}}; \quad\quad && \overline{\Lambda}^{[\mathfrak{u}]}_{m} &&:=&& \frac{1}{2m}\sum\limits_{\vert j \vert = 1}^{m} \Lambda_{j}; \quad && \Lambda^{[\mathfrak{u}]}_{(m)} &&:=&& \max\limits_{0 < j \leq m}\left\{ \Lambda_{j}\right\}; \quad && \mathfrak{b}_{m} &&:=&& \sum\limits_{j > m} \vert \theta^{\circ}_{j} \vert^{2}\\
& && && && && && && && && && && &&\\
& && && && && && \phi_{n}^{m_{n}}(f^{X}, [\mathfrak{u}]) && := && && \left(\frac{m_{n} \overline{\Lambda}^{[\mathfrak{u}]}_{m_{n}}}{n}\vee \mathfrak{b}_{m_{n}}^{2}\right). && && &&
\end{alignat*}
\end{nota}

One would note that the expected error and the error variance under $\widetilde{\mathds{Q}}_{\boldsymbol{f}^{s^{n}} \vert Y^{n}}^{n}$ are then respectively given by
\begin{alignat*}{3}
&\mathds{E}_{\widetilde{\mathds{Q}}_{\boldsymbol{f}^{s{n}} \vert Y^{n}}^{n}}\left[\Vert [f] - \theta^{\circ} \Vert^{2}\right] &&=&& 2 \mathds{E}_{\widetilde{\mathds{Q}}_{\boldsymbol{f}^{s{n}} \vert Y^{n}}^{n}}\left[\sum\limits_{j \in \mathds{N^{*}}} \left\vert[f]_{j} - \theta^{\circ}_{j} \right\vert^{2}\right]\\
& &&=&& 2 \sum\limits_{j \in \mathds{N^{*}}} \mathds{E}_{\widetilde{\mathds{Q}}_{\boldsymbol{f}^{s{n}} \vert Y^{n}}^{n}}\left[\left\vert[f]_{j} - \theta^{\circ}_{j} \right\vert^{2}\right]\\
& &&=&& 2 \sum\limits_{j \in \mathds{N^{*}}} \left\{\mathds{V}_{\widetilde{\mathds{Q}}_{\boldsymbol{f}^{s{n}} \vert Y^{n}}^{n}}\left[[f]_{j} - \theta^{\circ}_{j}\right] + \vert\mathds{E}_{\widetilde{\mathds{Q}}_{\boldsymbol{f}^{s{n}} \vert Y^{n}}^{n}}\left[[f]_{j} - \theta^{\circ}_{j}\right]\vert^{2}\right\}\\
& &&=&& 2 \sum\limits_{j \in \mathds{N^{*}}} \left\{\sigma_{\eta, j}^{n} + \left\vert\widehat{\theta}_{\eta, j}^{n} - \theta^{\circ}_{j}\right\vert^{2}\right\}\\
& &&=&& 2 \sum\limits_{j \in \mathds{N^{*}}} \left\{\frac{s_{j}^{n}}{1 + \eta n s_{j}^{n} \lambda_{j}^{2}} + \left\vert\frac{\frac{\Lambda_{j}\left(\theta^{\times}_{j} - \theta^{\circ}_{j}\right)}{\eta n s_{j}^{n}} + \frac{\overline{e}_{j}^{n}}{\lambda_{j}}- \theta^{\circ}_{j}}{\frac{\Lambda_{j}}{\eta n s_{j}^{n}} + 1}\right\vert^{2}\right\};
\end{alignat*}
and
\begin{alignat*}{3}
&\mathds{V}_{\widetilde{\mathds{Q}}_{\boldsymbol{f}^{s{n}} \vert Y^{n}}^{n}}\left[\Vert [f] - \theta^{\circ} \Vert^{2}\right] &&=&& \mathds{V}_{\widetilde{\mathds{Q}}_{\boldsymbol{f}^{s{n}} \vert Y^{n}}^{n}}\left[\sum\limits_{j \in \mathds{N^{*}}} \left\vert[f]_{j} - \theta^{\circ}_{j} \right\vert^{2}\right]\\
& &&=&& \sum\limits_{j \in \mathds{N^{*}}}\mathds{V}_{\widetilde{\mathds{Q}}_{\boldsymbol{f}^{s{n}} \vert Y^{n}}^{n}}\left[\left\vert[f]_{j} - \theta^{\circ}_{j} \right\vert^{2}\right]\\
& &&=&& \sum\limits_{j \in \mathds{N^{*}}}\left\{\mathds{E}_{\widetilde{\mathds{Q}}_{\boldsymbol{f}^{s{n}} \vert Y^{n}}^{n}}\left[\left\vert[f]_{j} - \theta^{\circ}_{j} \right\vert^{4}\right] - \mathds{E}_{\widetilde{\mathds{Q}}_{\boldsymbol{f}^{s{n}} \vert Y^{n}}^{n}}\left[\left\vert[f]_{j} - \theta^{\circ}_{j} \right\vert^{2}\right]^{2}\right\}\\
& &&=&& \sum\limits_{j \in \mathds{N^{*}}}\left\{\mathds{E}_{\widetilde{\mathds{Q}}_{\boldsymbol{f}^{s{n}} \vert Y^{n}}^{n}}\left[\left\vert[f]_{j} - \theta^{\circ}_{j} \right\vert^{4}\right] - \left(\mathds{V}_{\widetilde{\mathds{Q}}_{\boldsymbol{f}^{s{n}} \vert Y^{n}}^{n}}\left[[f]_{j} - \theta^{\circ}_{j}\right] + \vert\mathds{E}_{\widetilde{\mathds{Q}}_{\boldsymbol{f}^{s{n}} \vert Y^{n}}^{n}}\left[[f]_{j} - \theta^{\circ}_{j}\right]\vert^{2}\right)^{2}\right\}\\
& &&=&& \sum\limits_{j \in \mathds{N^{*}}}\left\{3 \left(\sigma_{\eta, j}^{n}\right)^{2} + 6 \sigma_{\eta, j}^{n} \left\vert\widehat{\theta}_{\eta, j}^{n} - \theta^{\circ}_{j}\right\vert^{2} + \left\vert\widehat{\theta}_{\eta, j}^{n} - \theta^{\circ}_{j}\right\vert^{4} - \left(\sigma_{\eta, j}^{n} + \left\vert\widehat{\theta}_{\eta, j}^{n} - \theta^{\circ}_{j}\right\vert^{2}\right)^{2}\right\}\\
& &&=&& 2 \sum\limits_{j \in \mathds{N^{*}}}\left\{ \left(\sigma_{\eta, j}^{n}\right)^{2} + 2 \sigma_{\eta, j}^{n} \left\vert\widehat{\theta}_{\eta, j}^{n} - \theta^{\circ}_{j}\right\vert^{2}\right\}\\
& &&=&& 2 \sum\limits_{j \in \mathds{N^{*}}}\frac{s_{j}^{n}}{1 + \eta n s_{j}^{n} \lambda_{j}^{2}} \cdot \left\{ \left(\frac{s_{j}^{n}}{1 + \eta n s_{j}^{n} \lambda_{j}^{2}}\right) + 2 \left\vert\frac{\frac{\Lambda_{j}\left(\theta^{\times}_{j} - \theta^{\circ}_{j}\right)}{\eta n s_{j}^{n}} + \frac{\overline{e}_{j}^{n}}{\lambda_{j}}- \theta^{\circ}_{j}}{\frac{\Lambda_{j}}{\eta n s_{j}^{n}} + 1}\right\vert^{2}\right\}.
\end{alignat*}

We can also derive the expected value and the variance of those quantities under $\mathds{P}^{Y}$ :

\begin{alignat*}{4}
& \mathds{E}_{\theta^{\circ}}\left[\mathds{E}_{\widetilde{\mathds{Q}}_{\boldsymbol{f}^{s{n}} \vert Y^{n}}^{n}}\left[\Vert [f] - \theta^{\circ} \Vert^{2}\right]\right] &&=&& \mathds{E}_{\theta^{\circ}}&&\left[2 \sum\limits_{j \in \mathds{N^{*}}} \left\{\frac{s_{j}^{n}}{1 + \eta n s_{j}^{n} \lambda_{j}^{2}} + \left[\frac{\frac{\Lambda_{j}\left(\theta^{\times}_{j} - \theta^{\circ}_{j}\right)}{\eta n s_{j}^{n}} + \frac{\overline{e}_{j}^{n}}{\lambda_{j}}- \theta^{\circ}_{j}}{\frac{\Lambda_{j}}{\eta n s_{j}^{n}} + 1}\right]^{2}\right\}\right]\\
& &&=&& 2 \sum\limits_{j \in \mathds{N^{*}}}&&\left\{ \frac{s_{j}^{n}}{1 + \eta n s_{j}^{n} \lambda_{j}^{2}} + \mathds{E}_{\theta^{\circ}}\left[ \left[\frac{\frac{\Lambda_{j}\left(\theta^{\times}_{j} - \theta^{\circ}_{j}\right)}{\eta n s_{j}^{n}} + \frac{\overline{e}_{j}^{n}}{\lambda_{j}}- \theta^{\circ}_{j}}{\frac{\Lambda_{j}}{\eta n s_{j}^{n}} + 1}\right]^{2}\right]\right\}\\
& &&=&& 2 \sum\limits_{j \in \mathds{N^{*}}}&&\left\{ \frac{s_{j}^{n}}{1 + \eta n s_{j}^{n} \lambda_{j}^{2}} + \mathds{V}_{\theta^{\circ}}\left[\frac{\frac{\Lambda_{j}\left(\theta^{\times}_{j} - \theta^{\circ}_{j}\right)}{\eta n s_{j}^{n}} + \frac{\overline{e}_{j}^{n}}{\lambda_{j}}- \theta^{\circ}_{j}}{\frac{\Lambda_{j}}{\eta n s_{j}^{n}} + 1}\right] \right.\\
& && && && \left.+ \mathds{E}_{\theta^{\circ}}\left[\frac{\frac{\Lambda_{j}\left(\theta^{\times}_{j} - \theta^{\circ}_{j}\right)}{\eta n s_{j}^{n}} + \frac{\overline{e}_{j}^{n}}{\lambda_{j}}- \theta^{\circ}_{j}}{\frac{\Lambda_{j}}{\eta n s_{j}^{n}} + 1}\right]^{2}\right\}\\
& &&=&& 2 \sum\limits_{j \in \mathds{N^{*}}}&&\left\{ \frac{s_{j}^{n}}{1 + \eta n s_{j}^{n} \lambda_{j}^{2}} \right.\\
& && && && \left.+ \frac{1}{\left(\frac{\Lambda_{j}}{\eta n s_{j}^{n}} + 1\right)^{2}}\left(\mathds{V}_{\theta^{\circ}}\left[\frac{\overline{e}_{j}^{n}}{\lambda_{j}}\right] + \left(\frac{\Lambda_{j}\left(\theta^{\times}_{j} - \theta^{\circ}_{j}\right)}{\eta n s_{j}^{n}} - \theta^{\circ}_{j} + \mathds{E}_{\theta^{\circ}}\left[\frac{\overline{e}_{j}^{n}}{\lambda_{j}}\right]\right)^{2}\right)\right\}\\
& &&=&& 2 \sum\limits_{j \in \mathds{N^{*}}}&&\left\{ \frac{s_{j}^{n}}{1 + \eta n s_{j}^{n} \lambda_{j}^{2}} \right.\\
& && && && \left.+ \frac{1}{\left(\frac{\Lambda_{j}}{\eta n s_{j}^{n}} + 1\right)^{2}}\left(\frac{\Lambda_{j}}{n}\left(1 - \frac{\vert \theta^{\circ}_{j} \vert^{2}}{\Lambda_{j}}\right) + \left(\frac{\Lambda_{j}\left(\theta^{\times}_{j} - \theta^{\circ}_{j}\right)}{\eta n s_{j}^{n}}\right)^{2}\right)\right\}\\
& &&=&& 2 \sum\limits_{j \in \mathds{N^{*}}}&&\left\{ \frac{\frac{\Lambda_{j}}{\eta n}}{\frac{\Lambda_{j}}{\eta n s_{j}^{n}} + 1} \right.\\
& && && && \left.+ \frac{1}{\left(\frac{\Lambda_{j}}{\eta n s_{j}^{n}} + 1\right)^{2}}\left(\frac{\Lambda_{j}}{n}\left(1 - \frac{\vert \theta^{\circ}_{j} \vert^{2}}{\Lambda_{j}}\right) + \left(\frac{\Lambda_{j}\left(\theta^{\times}_{j} - \theta^{\circ}_{j}\right)}{\eta n s_{j}^{n}}\right)^{2}\right)\right\}\\
\end{alignat*}

\begin{as}{\textsc{Small moment perturbations assumption}\\}\label{as2.5.1}
Assume that for any $n$ in $\mathds{N}$, any two indexes $j_{1}, j_{2}$ in $\llbracket 1, m_{n} \rrbracket$ and natural numbers $p_{1}, p_{2} \leq 2$, we have
\[\mathds{E}_{\boldsymbol{f}^{m_{n}}\vert Y^{n}}^{n}\left[ [\boldsymbol{f}]_{j_{1}}^{p_{1}} [\boldsymbol{f}]_{j_{2}}^{p_{2}}\right] = \mathds{E}_{\widetilde{\mathds{P}}\vert Y^{n}}^{n}\left[[\boldsymbol{f}]_{j_{1}}^{p_{1}} [\boldsymbol{f}]_{j_{2}}^{p_{2}}\right] + \mathcal{O}_{\mathds{P}^{Y}}\left(\phi_{n}^{m_{n}}(f^{X})\right).\]
\end{as}

\textcolor{red}{Goal : transforming this assumption in assumption on regularity of $\Delta_{1}^{n} + \Delta_{2}^{n}$}

\begin{thm}{\textsc{Oracle contraction rate}\\}\label{thm2.5.1}
Under \textsc{\autoref{as_moment}}, $\phi_{n}^{m_{n}}(f^{X})$ is a contraction rate for $p_{\boldsymbol{f}^{m_{n}}\vert Y^{n}}$, that is to say, for any increasing and unbounded sequence $c_{n}$
\[\lim\limits_{n \rightarrow \infty} \mathds{E}_{f^{X}}\left[\mathds{P}_{\boldsymbol{f}^{m_{n}} \vert Y^{n}}^{n}\left(\Vert [f] - [f^{X}] \Vert^{2} \leq c_{n}\phi_{n}^{m_{n}}(f^{X}) \right)\right] = 1.\]
\end{thm}


\subsection{Hierarchical priors}\label{2.5.2}
Consider a family of sequences of (non adaptive) priors, indexed by a sequence of tuning parameters $\left(m_{n}\right)_{n \in \mathds{N}}$ in some sequence of measurable spaces $\left(J^{n}, \mathcal{J}^{n}\right)$.
Each element of this family is noted $\left(\mathds{Q}_{\boldsymbol{f}^{m}}^{n}\right)_{n \in \mathds{N}}$.
A way to build an adaptive prior from this family is to consider $m$ as a random hyper-parameter $M$ and put a prior $\mathds{R}_{M}$ on it.
For example, one could define $\mathds{R}_{M}$ with a density (w.r.t. some measure $\mathds{R}^{\circ}$ depending on the specific case) of the shape
\[r_{M}(m) = \frac{\exp[- \pen(m)]}{\int\limits_{\mathcal{J}}\exp[- \pen(k)] dk};\]
where $\pen$ is a function of the parameter $m$.
We denote $\mathds{Q}_{\boldsymbol{f}^{M}}$ the prior distribution such that, while conditioning on the hyper-parameter, one recovers the non adaptive prior :
\[\mathds{Q}_{\boldsymbol{f}^{M} \vert M = m}^{n} = \mathds{Q}_{\boldsymbol{f}^{m}}^{n};\]
and conditionally on $f$, $Y$ is independent of $M$:
\[\mathds{P}_{Y^{n}\vert f, M}^{n} = \mathds{P}_{Y^{n}\vert f}^{n}.\]
Based on that, one can compute the posterior distribution:
\begin{alignat*}{3}
&\frac{d\mathds{R}_{M \vert Y^{n}}^{n}}{d\mathds{R}^{\circ}}(m, y^{n}) &&=&& \frac{\frac{d\mathds{P}_{M, Y^{n}}^{n}}{d\mathds{R}^{\circ} \, d\mathds{P}^{\circ}}(m, y^{n})}{\frac{d\mathds{P}_{Y^{n}}^{n}}{d \mathds{P}^{\circ}}(y^{n})}\\
& &&=&&\frac{\int\limits_{\mathcal{D}([0,1[)}\frac{d\mathds{P}_{M, Y^{n}, \boldsymbol{f}^{M}}^{n}}{d\mathds{R}^{\circ} \, d\mathds{P}^{\circ} \, d\mathds{Q}^{\circ}}(m, y^{n}, f)d\mathds{Q}^{\circ}(f)}{\frac{d\mathds{P}_{Y^{n}}^{n}}{d \mathds{P}^{\circ}}(y^{n})}\\
& &&=&&\frac{\int\limits_{\mathcal{D}([0,1[)}\frac{d\mathds{P}_{Y^{n} \vert M, \boldsymbol{f}^{M}}^{n}}{d\mathds{P}^{\circ}}(m, y^{n}, f) \, \frac{d\mathds{P}_{M, \boldsymbol{f}^{M}}^{n}}{d\mathds{R}^{\circ} \, d\mathds{Q}^{\circ}}(m, f)d\mathds{Q}^{\circ}(f)}{\frac{d\mathds{P}_{Y^{n}}^{n}}{d \mathds{P}^{\circ}}(y^{n})}\\
& &&=&&\frac{\int\limits_{\mathcal{D}([0,1[)}\frac{d\mathds{P}_{Y^{n} \vert \boldsymbol{f}^{M}}^{n}}{d\mathds{P}^{\circ}}(y^{n}, f) \, \frac{d\mathds{P}_{\boldsymbol{f}^{M}\vert M}^{n}}{d\mathds{Q}^{\circ}}(m, f) \frac{d \mathds{R}_{M}}{d\mathds{R}^{\circ}}(m)d\mathds{Q}^{\circ}(f)}{\frac{d\mathds{P}_{Y^{n}}^{n}}{d \mathds{P}^{\circ}}(y^{n})}\\
& &&=&&\frac{\frac{d \mathds{R}_{M}}{d\mathds{R}^{\circ}}(m) \int\limits_{\mathcal{D}([0,1[)}\frac{d\mathds{P}_{Y^{n} \vert \boldsymbol{f}}^{n}}{d\mathds{P}^{\circ}}(y^{n}, f) \, \frac{d\mathds{P}_{\boldsymbol{f}^{m}}^{n}}{d\mathds{Q}^{\circ}}(m, f) d\mathds{Q}^{\circ}(f)}{\frac{d\mathds{P}_{Y^{n}}^{n}}{d \mathds{P}^{\circ}}(y^{n})}\\
& &&=&&\frac{d \mathds{R}_{M}}{d\mathds{R}^{\circ}}(m) \int\limits_{\mathcal{D}([0,1[)}\frac{d\mathds{P}_{\boldsymbol{f}^{m}\vert Y^{n}}^{n}}{d\mathds{Q}^{\circ}}(y^{n}, f, m) d\mathds{Q}^{\circ}(f)\\
& &&=&&\frac{\frac{d \mathds{R}_{M}}{d\mathds{R}^{\circ}}(m) \frac{d\mathds{P}_{Y^{n} \vert M}^{n}}{d\mathds{P}^{\circ}}(y^{n}, m)}{\int\limits_{J}\frac{d \mathds{R}_{M}}{d\mathds{R}^{\circ}}(m) \frac{d\mathds{P}_{Y^{n} \vert M}^{n}}{d\mathds{P}^{\circ}}(y^{n}, m) dm};
\end{alignat*}

\begin{alignat*}{3}
&\frac{d\mathds{Q}_{\boldsymbol{f}^{M} \vert Y^{n}}^{n}}{d\mathds{P}^{\circ}}(f, y^{n}) &&=&& \frac{\frac{d\mathds{P}_{\boldsymbol{f}^{M}, Y^{n}}^{n}}{d\mathds{Q}^{\circ} \, d\mathds{P}^{\circ}}(f, y^{n})}{\frac{d\mathds{P}_{Y^{n}}^{n}}{d \mathds{P}^{\circ}}(y^{n})}\\
& &&=&& \frac{\int\limits_{J} \frac{d\mathds{P}_{\boldsymbol{f}^{M}, Y^{n}, M}^{n}}{d\mathds{Q}^{\circ} \, d\mathds{P}^{\circ} \, d\mathds{R}^{\circ}}(f, y^{n}, m) dm}{\frac{d\mathds{P}_{Y^{n}}^{n}}{d \mathds{P}^{\circ}}(y^{n})}\\
& &&=&& \frac{\int\limits_{J} \frac{d\mathds{P}_{\boldsymbol{f}^{M} \vert Y^{n}, M}^{n}}{d\mathds{Q}^{\circ}}(f, y^{n}, m) \frac{d\mathds{P}_{Y^{n}, M}^{n}}{d\mathds{P}^{\circ} \, d\mathds{R}^{\circ}} dm}{\frac{d\mathds{P}_{Y^{n}}^{n}}{d \mathds{P}^{\circ}}(y^{n})}\\
& &&=&& \frac{\int\limits_{J} \frac{d\mathds{P}_{\boldsymbol{f}^{m} \vert Y^{n}}^{n}}{d\mathds{Q}^{\circ}}(f, y^{n}, m) \frac{d\mathds{P}_{M \vert Y^{n}}^{n}}{d\mathds{R}^{\circ}} \frac{d\mathds{P}_{Y^{n}}}{d\mathds{P}^{\circ}}(Y^{n}) dm}{\frac{d\mathds{P}_{Y^{n}}^{n}}{d \mathds{P}^{\circ}}(y^{n})}\\
& &&=&& \int\limits_{J} \frac{d\mathds{P}_{\boldsymbol{f}^{m} \vert Y^{n}}^{n}}{d\mathds{Q}^{\circ}}(f, y^{n}, m) \frac{d\mathds{P}_{M \vert Y^{n}}^{n}}{d\mathds{R}^{\circ}} dm.
\end{alignat*}

Interestingly, the posterior mean is then given by
\begin{alignat*}{3}
&d\mathds{E}_{\boldsymbol{f}^{M} \vert Y^{n}}^{n}[\boldsymbol{f}] &&=&& \int\limits_{J} \mathds{E}_{\boldsymbol{f}^{m} \vert Y^{n}}[\boldsymbol{f}] \frac{d\mathds{P}_{M \vert Y^{n}}^{n}}{d\mathds{R}^{\circ}}(m) dm;
\end{alignat*}
which is the aggregation of the posterior means obtained with the non adaptive priors where the weights are given by the posterior density of $M$.

\begin{ex}{\textsc{Hierarchical sieve\\}}\label{ex2.6.1}
One could apply this method to a subclass of sieve priors.
Consider the sequence $\left(G_{n}\right)_{n \in \mathds{N}}$ such that for all $n$, $G_{n} = \max\left\{m \in \llbracket 1, n \rrbracket : \frac{\Lambda_{(m)}}{n} \leq \Lambda_{1}\right\}$.
We would then consider $J^{n} = \llbracket 1, G_{n} \rrbracket$ and $\mathcal{J}^{n} = \mathcal{P}(J^{n})$ the set of all subsets of $J^{n}$ as it is a finite space.
The considered sequences of priors are elements of $\mathcal{G}$ as described earlier.
Hence the posterior distributions are
\begin{alignat*}{3}
& \frac{d\mathds{R}_{M \vert Y^{n}}^{n}}{d\mathds{R}^{\circ}}(m, y^{n}) &&=&& \exp[-\pen(m)] \int\limits_{\mathcal{D}([0,1[)}\exp\left[-\frac{1}{2}\sum\limits_{j = 1}^{m_{n}} \frac{\vert \widehat{\theta}_{\eta, j}^{n}\vert^{2}}{\sigma_{\eta, j}^{n}} + \sum\limits_{j = 1}^{m_{n}} \frac{\widehat{\theta}_{\eta, j}^{n}}{\sigma_{\eta, j}^{n}}[f]_{j} - \frac{n}{2} \Delta_{1}^{n}(f, y^{n}) +\Delta_{2}^{n}(f)\right] d\mathds{Q}^{\circ}(f);\\
&\frac{d\mathds{Q}_{\boldsymbol{f}^{M} \vert Y^{n}}^{n}}{d\mathds{P}^{\circ}}(f, y^{n}) &&=&& \int\limits_{J} \frac{d\mathds{P}_{M \vert Y^{n}}^{n}}{d\mathds{R}^{\circ}} \exp\left[-\frac{1}{2}\sum\limits_{j = 1}^{m_{n}} \frac{\vert \widehat{\theta}_{\eta, j}^{n}\vert^{2}}{\sigma_{\eta, j}^{n}} + \sum\limits_{j = 1}^{m_{n}} \frac{\widehat{\theta}_{\eta, j}^{n}}{\sigma_{\eta, j}^{n}}[f]_{j} - \frac{n}{2} \Delta_{1}^{n}(f, y^{n}) +\Delta_{2}^{n}(f)\right] dm.
\end{alignat*}
\end{ex}
\section{On the shape of the posterior mean}\label{BAYES_POSTMEAN}

We have hence seen that in a general case, considering the asymptotic iteration , the posterior distribution using a sieve prior contracts around the projection estimator and while using a hierarchical prior, the posterior contracts around some penalised contrast maximiser projection estimator.

It is also interesting to note that for any number of iteration $\eta$, the posterior mean can be written both as a shrinkage and as an aggregation estimator.
Indeed, we have

\begin{alignat*}{3}
& \E_{\boldsymbol{\theta}^{M}\vert Y^{n}}^{\eta}\left[\boldsymbol{\theta}^{M}\right] && = && \E_{\boldsymbol{\theta}^{M}\vert Y^{n}}^{\eta}\left[\sum\limits_{m \in G} \boldsymbol{\theta}^{M} \mathds{1}_{M = m}\right]\\
& && = && \sum\limits_{m \in G} \E_{\boldsymbol{\theta}^{M}\vert Y^{n}}^{\eta}\left[ \boldsymbol{\theta}^{M} \mathds{1}_{M = m}\right]\\
& && = && \sum\limits_{m \in G} \mathds{P}_{M \vert Y^{n}}^{\eta}(m = M) \E_{\boldsymbol{\theta}^{m}\vert Y^{n}}^{\eta}\left[ \boldsymbol{\theta}^{m} \right];
\end{alignat*}

and we see here the aggregation form of this estimator.

On the other hand, if we write the expectation of the components individually, we obtain:

\begin{alignat*}{3}
& \E_{\boldsymbol{\theta}^{M}\vert Y^{n}}^{\eta}\left[\boldsymbol{\theta}_{j}^{M}\right] && = && \E_{\boldsymbol{\theta}^{M}\vert Y^{n}}^{\eta}\left[ \boldsymbol{\theta}_{j}^{m} \mathds{1}_{M \geq j}\right]\\
& && = && \mathds{P}_{M \vert Y^{n}}^{\eta}(M \geq j) \E_{\boldsymbol{\theta}^{m}\vert Y^{n}}^{\eta}\left[ \boldsymbol{\theta}^{m}_{j} \right];
\end{alignat*}

where we see the shrinkage property.

\medskip

Aggregation estimates gathered a lot of interest, see for example \ncite{rigollet2007linear}.
While considering such estimators, the goal is to reach the convergence rate of the best estimator contributing to the aggregation.

\medskip

In the next chapter, we hence investigate the properties of this estimator both in inverse Gaussian sequence space model and circular density deconvolution.

%%
\chapter{Minimax and oracle optimal adaptive aggregation}\label{FREQ}
We inquire in this chapter the properties of aggregation estimators as introduced in \nref{BAYES_POSTMEAN}.
We introduce first a skim of proof for oracle and minimax optimality of this kind of estimator before applying it to the inverse Gaussian sequence space and the circular deconvolution models respectively introduced in \nref{INTRO_IGSSM} and \nref{INTRO_CIRCULARDECONVOLUTION}, including in presence of dependance and partially known operator.

\section{Shape of the aggregation estimators}\label{FREQ_GENERAL_SHAPE}\label{freq:ge:shape}

We want to define a family of estimators $(\widehat{\theta}^{(\eta)})_{\eta \in ]0, \infty]}$ of $\theta^{\circ}$ and estimators for $f$, $\widehat{f}^{(\eta)} := \fourier^{\star}(\widehat{\theta}^{(\eta)})$ such that, for any $\eta$, $\widehat{\theta}^{(\eta)}$ has the shape
\begin{equation}\label{freq:ge:shape:kn}
(\widehat{\theta}^{(\eta)}(s))_{s \in \mathds{F}} = \sum\nolimits_{m \in \N} \P_{M}^{(\eta)}(m) \cdot (\theta_{n, \overline{m}}(s))_{s \in \mathds{F}} = \sum\nolimits_{m \geq \vert s \vert} \P_{M}^{(\eta)}(m) \cdot (\theta_{n}(s))_{s \in \mathds{F}}
\end{equation}
under \nref{AS_INTRO_DATA_KNOWN}, and
\begin{equation}\label{freq:ge:shape:uk}
(\widehat{\theta}^{(\eta)}(s))_{s \in \mathds{F}} = \sum\nolimits_{m \in \N} \widehat{\P}_{M}^{(\eta)}(m) \cdot (\theta_{n, n_{\lambda}, \overline{m}}(s))_{s \in \mathds{F}} = \sum\nolimits_{m \geq \vert s \vert} \widehat{\P}_{M}^{(\eta)}(m) \cdot (\theta_{n, n_{\lambda}}(s))_{s \in \mathds{F}}
\end{equation}
under \nref{AS_INTRO_DATA_UNKNOWN}.
The sequence $(\P_{M}^{(\eta)}(m))_{m \in \N}$ is the aggregation sequence.
Under \nref{AS_INTRO_DATA_KNOWN} it depends on the observations $Y^{n}$ as well on the known operator $T$ through its eigen values $(\lambda(s))_{s \in \mathds{F}}$ whereas under \nref{AS_INTRO_DATA_UNKNOWN}, it only depends on the observed data $Y^{n}$ and $\epsilon^{n_{\lambda}}$.
This notation is motivated by the Bayesian inspiration of the method.

We give in \nref{fig:ge:adaptive:aggregation} an illustration of the aggregation estimator used in a Gaussian sequence space model in the direct problem case, that is to say $\lambda(s) = 1$ for any $s$ in $\N$.

\begin{figure}
  \centering
  \begin{tabular}{@{}c@{}}
    \includegraphics[width=.49\linewidth]{gauss/adaptive/aggregation_surv.png} \\[\abovecaptionskip]
  \end{tabular}
  \begin{tabular}{@{}c@{}}
    \includegraphics[width=.49\linewidth]{gauss/adaptive/aggregation.png} \\[\abovecaptionskip]
  \end{tabular}
  \caption{Aggregation estimator on an Gaussian sequence space model, direct problem case}
  \label{fig:ge:adaptive:aggregation}
\end{figure}

Taking inspiration in the posterior distributions obtained with a hierarchical prior in the previous chapter, we will give the following shape to the aggregation weights.
In the case \nref{AS_INTRO_DATA_KNOWN} let be the following functions
\begin{multline}\label{freq:ge:shape:kn:we}
\Upsilon : \mathds{N} \to \R_{+}, \quad m \mapsto \Upsilon(m); \qquad \pen^{\Lambda} : \N \to \R_{+}, \quad m \mapsto \pen^{\Lambda}(m);\\
\P_{M}^{(\eta)} : \N \to \R_{+}, \quad m \mapsto \tfrac{\exp[-\eta n (-\Upsilon(m) + \pen^{\Lambda}(m))]}{\sum\nolimits_{k = 0}^{n} \exp[-\eta n (-\Upsilon(k) + \pen^{\Lambda}(k))]} \mathds{1}_{m \leq n};
\end{multline}
where $\Upsilon$ depends on the observations $Y^{n}$ as well as the known operator $T$ through the sequence $\lambda$ of its eigen values; and $\pen^{\Lambda}$ depends only on the parameter $T$ through the sequence $\lambda$ of its eigen values.
Under \nref{AS_INTRO_DATA_UNKNOWN}, we define
\begin{multline}\label{freq:ge:shape:uk:we}
\Upsilon : \N \to \R_{+}, \quad m \mapsto \Upsilon(m); \qquad \pen^{\widehat{\Lambda}} : \N \to \R_{+}, \quad m \mapsto \pen^{\widehat{\Lambda}}(m);\\
\widehat{\P}_{M}^{(\eta)} : \N \to \R_{+}; \quad m \mapsto \tfrac{\exp[\eta n (\Upsilon(m) - \pen^{\widehat{\Lambda}}(m))]}{\sum\nolimits_{k = 0}^{n} \exp[\eta n (\Upsilon(k) - \pen^{\widehat{\Lambda}}(k))]} \mathds{1}_{m \leq n};
\end{multline}
where $\Upsilon$ depends solely the observations $Y^{n}$ and $\epsilon^{n_{\lambda}}$; and $\pen^{\widehat{\Lambda}}$ depends only on the observations $\epsilon^{n_{\lambda}}$.
The functions $\Upsilon$, and $\pen^{\Lambda}$ will respectively be called contrast and penalty.
For any subset $S$ of $\N$, we denote $\P_{M}^{(\eta)}(S) = \sum_{k \in S} \P_{M}^{(\eta)}(k)$.
One would expect that as the amount of data increases, the number of coefficients estimated increases too, as our observations allows us to recover more information about the system of interest as illustrated in \nref{fig:ge:adaptive:M} by representing $\P_{M}^{(\eta)}(\llbracket m, n \rrbracket)$ for increasnig values of $n$.

\begin{figure}
  \centering
  \begin{tabular}{@{}c@{}}
    \includegraphics[width=.49\linewidth]{density/distM/1.png} \\[\abovecaptionskip]
  \end{tabular}
  \begin{tabular}{@{}c@{}}
    \includegraphics[width=.49\linewidth]{density/distM/20.png} \\[\abovecaptionskip]
  \end{tabular}
  
    \begin{tabular}{@{}c@{}}
    \includegraphics[width=.49\linewidth]{density/distM/40.png} \\[\abovecaptionskip]
  \end{tabular}
  \begin{tabular}{@{}c@{}}
    \includegraphics[width=.49\linewidth]{density/distM/50.png} \\[\abovecaptionskip]
  \end{tabular}
  \caption{Evolution of the aggregation weights }
  \label{fig:ge:adaptive:M}
\end{figure}

\medskip

Consider first the asymptotic when one lets $\eta$ tend to infinity.
Under \nref{AS_INTRO_DATA_KNOWN}, following a model selection approach (c.f. \ncite{barron1999risk} and \ncite{Massart2007} for an extensive description), a dimension parameter $\hDi$ is determined among a collection of admissible values $\nset{1,\ssY}$ by minimising the penalised contrast function $-\Vnormlp{\txdfPr}+\penSv$, that is
\begin{equation}\label{freq:ge:shape:kn:de:ms}
  \tDi:=\argmin\nolimits_{\Di\in\nset{1,\ssY}} \big\{-\Upsilon(m) + \penSv\big\}.
\end{equation}
If $\tDi$ minimises uniquely the penalised contrast function, then it is easily seen that the discrete probability measure $\rWe[]$ on the set $\nset{1,\ssY}$ given by the weights $\rWe[](\{\Di\})=\rWe$ as in \eqref{freq:ge:shape:kn:we} degenerates to a Dirac measure $\dirac[\tDi]$ on the point $\tDi$ as $\rWc\to\infty$.
Precisely, for any $\Di\in\nset{1,n}$ holds
\begin{equation}\label{freq:ge:shape:kn:de:msWe}
  \lim\nolimits_{\rWc\to\infty}\rWe=\dirac[\tDi](\{\Di\})=:\msWe
\end{equation}
Thereby, in the sequel we consider the model selected estimator
\[\txdfPr[\tDi]=\widehat{\theta}^{(\infty)} =\sum\nolimits_{\Di\in\nset{1,\ssY}}\msWe\txdfPr\]
as an aggregation with respect to the discrete measure $\msWe[]=\dirac[\tDi]$ on the set $\nset{1,\ssY}$.
We give in \nref{fig:ge:adaptive:selection} an illustration of the model selection estimator used on a Gaussian sequence space model, in the direct problem case, that is to say $\lambda(s) = 1$ for all $s$.

\begin{figure}
\centering
  \begin{tabular}{@{}c@{}}
    \includegraphics[width=.49\linewidth]{gauss/adaptive/model_selection.png} \\[\abovecaptionskip]
  \end{tabular}
  \begin{tabular}{@{}c@{}}
    \includegraphics[width=.49\linewidth]{gauss/adaptive/penalised_contrast.png} \\[\abovecaptionskip]
  \end{tabular}
  \caption{Model selection estimator on an Gaussian sequence space model, direct problem case}
  \label{fig:ge:adaptive:selection}
\end{figure}

Under \nref{AS_INTRO_DATA_UNKNOWN} consider again a  model selection approach by
minimising now the penalised contrast function $\Upsilon(m)+\peneSv$, that is
\begin{equation}\label{freq:ge:shape:uk:de:ms}
  \hDi:=\argmin_{\Di\in\nset{1,\ssY}} \big\{-\Upsilon(m)+\peneSv\big\}.
\end{equation}
If $\hDi$ minimises uniquely the penalised contrast function, then for any $\Di\in\nset{1,n}$ holds
\begin{equation}\label{freq:ge:shape:uk:de:msWe}
  \lim_{\rWc\to\infty}\erWe=\dirac[\hDi](\{\Di\})=:\widehat{\P}_{M}^{(\infty)}.
\end{equation}
Thereby, we consider again the model selected estimator

$\txdfPr[\tDi] = \widehat{\theta}^{(\infty)} = \sum_{\Di\in\nset{1,\ssY}}\widehat{\P}_{M}^{(\infty)}(m)\hxdfPr$ as an aggregation with respect to the discrete measure $\msWe[]=\dirac[\hDi]$ on the set $\nset{1,\ssY}$.

We will consider two examples in this chapter, namely the inverse Gaussian sequence space model as well as the circular deconvolution model.
In both cases the functions $\Upsilon$, $\pen^{\Lambda}$, and $\pen^{\widehat{\Lambda}}$ take the same shape which we hence give here.
\begin{de}\label{freq:ge:shape:kn:de:pen:oo}
  Under \nref{AS_INTRO_DATA_KNOWN}, let be a universal constant $\cpen$ to be fixed depending on the considered model.
  For any $\Di$ in $\nset{1,\ssY}$, remind that $\Lambda(m) = \vert \lambda(m) \vert^{-2}$, and $\Lambda_{+}(m) = \max\{\Lambda(s), s \in \mathds{F}_{m} \}$ and define
  \begin{alignat*}{4}
  & \Upsilon(m) && := && \Vert \theta_{n, \overline{m}} \Vert_{l^{2}}^{2};  \quad && \cmiSv := \tfrac{\log^{2}(\Di\miSv \vee(\Di+2))}{\log^{2}(\Di+2)}\geq1;\\
  & \DipenSv && := && \cmiSv \Di \miSv; \quad && \penSv:= \penD.
  \end{alignat*}
  \assEnd
\end{de}
\begin{de}\label{freq:ge:shape:uk:de:pen:oo}
  Under \nref{AS_INTRO_DATA_UNKNOWN}, let be a universal constant $\cpen$ to be fixed depending on the considered model.
  Then, for any $m$ in $\N$, we define
  \begin{alignat*}{4}
  & \Upsilon(m) && := && \Vert \theta_{n, n_{\lambda}, \overline{m}}\Vert_{l^{2}}^{2}; && \quad \eiSv[(s)]:= \vert \hfedfmpI[(s)] \vert ^2\\
    & \meiSv && := && \max\{\eiSv[(l)],l\in\nset{1,\Di}\}; && \quad \cmeiSv:=\tfrac{\log^{2}(\Di\meiSv\vee(\Di+2))}{\log^{2}(\Di+2)}\geq1;\\
    &\DipeneSv && := && \cmeiSv \Di \meiSv;&& \quad \peneSv:= \peneD.
  \end{alignat*}
  \assEnd
\end{de}
Notice that, with the exception of the constant $\kappa$, our estimator is now fully determined, in both cases \nref{AS_INTRO_DATA_KNOWN} and \nref{AS_INTRO_DATA_UNKNOWN}.

\section{Strategy of proof for optimality of aggregation estimator}\label{freq:ge:strat}\label{FREQ_STRATEGY}
As we have now given a precise shape to our aggregation estimator, we propose a strategy to compute upper bounds for its convergence rate in $l^{2}$-norm.
Our method is inspired by the strategy to compute upper bounds for the contraction rate of hierarchical sieves we presented in the previous chapter.
We will hence highlight a decomposition of the risk which separates the risk obtained by taking values of the threshold which are respectively "too small", "too large", or "optimal".
Those terms should be understood with respect to the quadratic risk of the projection estimator associated with this choice of threshold.
One would then prove that the values of the threshold which are too small or too large do not receive an important weight under $\P_{M}^{(\eta)}$ or $\widehat{P}_{M}^{(\eta)}$.
Before going any further, notice that, for any $\Di$ and $m^{\bullet}$ in $\N$, the aggregation weights can be bounded in the following way:
\begin{multline*}
\tfrac{\exp[-\eta n (-\Vert \theta_{n, \overline{m}} \Vert_{l^{2}} + \pen^{\Lambda}(m))]}{\sum\nolimits_{k = 0}^{n} \exp[-\eta n (-\Vert \theta_{n, \overline{k}} \Vert_{l^{2}} + \pen^{\Lambda}(k))]} \mathds{1}_{m \leq n}\\
\leq \exp[-\eta n (\Vert \theta_{n, \overline{m}} \Vert_{l^{2}} - \Vert \theta_{n, \overline{m^{\bullet}}} \Vert_{l^{2}} + \pen^{\Lambda}(m^{\bullet}) - \pen^{\Lambda}(m))] \mathds{1}_{m \leq n}\text{; and}
\end{multline*}
\begin{multline*}
\tfrac{\exp[-\eta n (-\Vert \theta_{n, n_{\lambda}, \overline{m}} \Vert_{l^{2}} + \pen^{\Lambda}(m))]}{\sum\nolimits_{k = 0}^{n} \exp[-\eta n (-\Vert \theta_{n, n_{\lambda}, \overline{k}} \Vert_{l^{2}} + \pen^{\Lambda}(k))]} \mathds{1}_{m \leq n}\\
\leq \exp[-\eta n (\Vert \theta_{n, n_{\lambda}, \overline{m}} \Vert_{l^{2}} - \Vert \theta_{n, n_{\lambda}, \overline{m^{\bullet}}} \Vert_{l^{2}} + \pen^{\Lambda}(m^{\bullet}) - \pen^{\Lambda}(m))] \mathds{1}_{m \leq n}.
\end{multline*}

Then, the following lemma, which proof is given in \nref{pro:re:contr} allows to derive an upper bound which is easier to control.

\begin{lm}\label{re:contr}
Given $\ssY\in\Nz$ and $\xdfPr[],\dxdfPr[]\in\lp[2]$ consider the
  families of  orthogonal projections
  
  $\setB{\dxdfPr=\dProj{\Di}{}\dxdfPr[],\Di\in\nset{1,n}}$ and $\setB{\xdfPr=\dProj{\Di}{}\xdf,\Di\in\nset{1,n}}$.
  
  If $\Vnormlp{\dProj{\Di}{}^\perp\xdf}^2=\Vnormlp{\xdf_{\underline{0}}}^2\bias^2(\xdf)$ for all
  $\Di\in\nset{1,\ssY}$, then for any $l\in\nset{1,n}$ holds
 \begin{resListeN}[]
\item\label{re:contr:e1}
$\Vnormlp{\dxdfPr[k]}^2-\Vnormlp{\dxdfPr[l]}^2\leq
\tfrac{11}{2}\Vnormlp{\dxdfPr[l]-\xdfPr[l]}^2-\tfrac{1}{2}\Vnormlp{\xdf_{\underline{0}}}^2\{\bias[k]^2(\xdf)-\bias[l]^2(\xdf)\}$,
for all $k\in\nsetro{1,l}$;
\item\label{re:contr:e2}
$\Vnormlp{\dxdfPr[k]}^2-\Vnormlp{\dxdfPr[l]}^2\leq \tfrac{7}{2}\Vnormlp{\dxdfPr[k]-\xdfPr[k]}^2+\tfrac{3}{2}\Vnormlp{\xdf_{\underline{0}}}^2
\{\bias[l]^2(\xdf)-\bias[k]^2(\xdf)\}$, for all $k\in\nsetlo{l,n}$.
\end{resListeN}
\reEnd
\end{lm}

\subsection{Known operator}\label{freq:ge:strat:kn}\label{FREQ:GE:STRAT:KN}
Consider first the case \nref{AS_INTRO_DATA_KNOWN}.
We shall hence keep in mind \ref{freq:ge:shape:kn}, \ref{freq:ge:shape:kn:we}, \nref{freq:ge:shape:kn:de:pen:oo} as well as \ref{freq:ge:shape:kn:de:ms} and \ref{freq:ge:shape:kn:de:msWe}.
\textbf{Note that the detailed proofs for all results given here can be found in \nref{pro:freq:ge:strat:kn}}.

Both for the quadratic and the maximal risk, our strategy is based on the decomposition of the quadratic loss function displayed in \nref{freq:ge:strat:kn:co:agg}.
This decomposition is independent of the model and only relies on the fact that the parameter space is equipped with a nested sieve and the fact that our estimator aggregation structure takes advantage of it.
\begin{lm}\label{freq:ge:strat:kn:co:agg}
First writing the $l^{2}$-distance between $\theta^{\circ}$ and $\widehat{\theta}^{\eta}$ we obtain, for any $\mDi$ and $\pDi$ in $\llbracket 1, n \rrbracket$ such that $\mDi \leq \pDi$, and sequence $(\pen(m))_{m \in \N}$ of compensating terms,
\begin{multline}\label{freq:ge:strat:kn:co:agg:e1}
    \Vnormlp{\txdfAg-\xdf}^2\leq \tfrac{2}{7}\pen(\pDi) +2\Vnormlp{\xdf_{\underline{0}}}^2\bias[\mDi]^2(\xdf)\\\hfill
    +2\Vnormlp{\xdf_{\underline{0}}}^2\We[](\nsetro{1,\mDi})+\tfrac{2}{7}\sum\nolimits_{\Di\in\nsetlo{\pDi,\ssY}}\pen(m)\We\Ind{\{\Vnormlp{\txdfPr-\xdf_{\overline{m}}}^2<\pen(\Di)/7\}}\\
+2\sum_{\Di\in\nset{\pDi,\ssY}}\vectp{\Vnormlp{\txdfPr-\xdf_{\overline{m}}}^2-\pen(\Di)/7}  
+\tfrac{2}{7}\sum_{\Di\in\nsetlo{\pDi,\ssY}}\pen(\Di)\Ind{\{\Vnormlp{\txdfPr-\xdf_{\overline{m}}}^2\geq \pen(m)/7\}}.
\end{multline}
\reEnd
\end{lm}
The proof strategy will be articulated around the search for sequences $\pDi$, $\mDi$ and $\pen(m)$ such that each term is properly controlled.
In practice, the terms $\tfrac{2}{7}\pen(\pDi)$ and $2\Vnormlp{\xdf_{\underline{0}}}^2\bias[\mDi]^2(\xdf)$ will be the leading terms in the sum.

\subsubsection{Quadratic risk bounds}\label{freq:ge:strat:kn:qu}
We propose a strategy which allows to prove that the sequence defined hereafter is an upper bound for the quadratic risk of the aggregation estimator we just defined.
\begin{de}\label{freq:ge:strat:kn:qu:de:rate}
Remind that we defined for any $\theta$ in $\Theta$ and $\Di$ in $\N$ the following term $\b_{m}^{2}(\theta) = \Vert \theta_{\underline{m}} \Vert_{l^{2}}^{2}\Vert \theta_{\underline{0}} \Vert_{l^{2}}^{-2} \leq 1$.
We then define a family of sequences $(\daRa{\Di}{(\xdf)})_{\Di \in \N} := (\daRa{\Di}{(\xdf,\Lambda)})_{\Di \in \N} = ([\b_{m}^{2}(\theta^{\circ}) \vee \penSv/\cpen])_{\Di \in \N}$ and hence it holds for all $\Di$ in $\nset{1,\ssY}$
    \begin{equation}\label{freq:ge:strat:kn:qu:de:rate:e1}
      [\Vnormlp{\xdf_{\underline{0}}}^2+\cpen]\daRa{\Di}{(\xdf)}\geq\Vnormlp{\xdf_{\underline{0}}}^2\bias^2(\xdf)\vee\penSv.
      \end{equation}
We intend to prove that the specific choice
\begin{multline*}
\aDi{\ssY}(\xdf):=\argmin\Nset[\Di\in\Nz]{\daRa{\Di}{(\xdf)}}\in\nset{1,\ssY}; \\
\naRa{(\xdf)}:=\naRa{(\xdf,\Lambda)}:=\min\Nset[\Di\in\Nz]{\daRa{\Di}{(\xdf)}}
\end{multline*}
with $\daRa{\aDi{\ssY}}{(\xdf,\Lambda)}=\naRa{(\xdf,\Lambda)}$ defines an upper bound for the convergence rate of the aggregation estimators.
\assEnd
\end{de}
Note that the proofs for the results displayed here can be found in \nref{pro:freq:ge:strat:kn:qu}
\begin{rmk}\label{freq:ge:strat:kn:qu:rmk:rate} The following statements can be
shown using the same   arguments as in \nref{oo:rem:ora}
by exploiting that the sequence $\bias^2(\xdf)$ is non-increasing with limit zero and $\bias[0]^2(\xdf)\leq1$. 
By construction  for all $\ssY\in\Nz$ it hold 
$\naRa{(\xdf)}\geq \ssY^{-1}$ and $\naRa{(\xdf)}=\mathfrak{o}_{n}(1)$.
Moreover, for all $\ssY\in\Nz$ we have $\aDi{\ssY}(\xdf)\in\nset{1,\ssY}$,
$\aDi{\ssY}(\xdf)=\argmin\Nset[{\Di\in\nset{1,\ssY}}]{\daRa{\Di}{(\xdf)}}$ and 
$\naRa{(\xdf)}=\min\Nset[{\Di\in\nset{1,\ssY}}]{\daRa{\Di}{(\xdf)}}$. 
Thereby, in case \ref{oo:xdf:p} we conclude that $\aDi{\ssY}(\xdf)=K$ and
the rate $\naRa{(\xdf)}$ is parametric, that is
$\naRa{(\xdf)}=\DipenSv[K]\ssY^{-1}\approx\ssY^{-1}$, and hence
equals the oracle rate $\oRa{\xdf}$, i.e. $\oRa{\xdf}\approx\naRa{(\xdf)}$. On the other
hand side, in case \ref{oo:xdf:np}  the rate
$\naRa{(\xdf)}$ is nonparametric, that is,
$\ssY\naRa{(\xdf)}\to\infty$ and $\aDi{\ssY}(\xdf)\to\infty$  as
$\ssY\to\infty$. Moreover, by construction holds $\naRa{(\xdf)}\geq\oRa{\xdf}$.
 \remEnd
\end{rmk}

\begin{te}
 Let us first briefly illustrate the last definitions by stating the
 order of $\aDi{\ssY}(\xdf)$ and $\naRa{(\xdf)}$ in the cases considered
 in \nref{il:oo}
\end{te}
% ....................................................................
% <<Il upper bound oo>>
% ....................................................................
\begin{il}\label{freq:ge:strat:kn:qu:il:rate}Let us illustrate  \nref{freq:ge:strat:kn:qu:de:rate}
  considering as in \nref{il:oo} usual behaviour \ref{il:oo:oo},
  \ref{il:oo:so} and \ref{il:oo:os} for the sequences
  $\Nsuite[\Di]{\bias[\Di](\xdf)}$ and $\Nsuite[\Di]{\iSv[\Di]}$:
  \begin{Liste}[]
  \item[\mylabel{freq:ge:strat:kn:qu:il:rate:np:oo}{\dg\bfseries{[o-o]}}] Since
    $\bias^2(\xdf)\approx\Di^{-2p}$ and $\DipenSv\approx\Di^{2a+1}$ follows
    $\daRa{\aDi{\ssY}}{(\xdf,\Lambda)}\approx(\aDi{\ssY})^{-2p}\approx\DipenSv[\aDi{\ssY}]\ssY^{-1}\approx(\aDi{\ssY})^{2a+1}\ssY^{-1}$
    which implies $\aDi{\ssY}\approx\ssY^{1/(2p+2a+1)}$,
    $\cmiSv[\aDi{\ssY}]\aDi{\ssY}\approx\ssY^{1/(2p+2a+1)}$,
    $\naRa{(\xdf)}\approx\ssY^{-2p/(2p+2a+1)}$ and
    $ \vert \log\naRa{(\xdf)} \vert \approx(\log\ssY)$.
  \item[\mylabel{freq:ge:strat:kn:qu:il:rate:np:os}{\dg\bfseries{[o-s]}}] Since
    $\bias^2(\xdf)\approx\Di^{-2p}$ and
    $\DipenSv\approx\Di^{1+4a}\exp(\Di^{2a})$ follows
    $\daRa{\aDi{\ssY}}{(\xdf,\Lambda)}\approx(\aDi{\ssY})^{-2p}\approx\DipenSv[\aDi{\ssY}]\ssY^{-1}\approx(\aDi{\ssY})^{1+4a}\exp((\aDi{\ssY})^{2a})$
    which implies $\aDi{\ssY}\approx(\log\ssY)^{1/(2a)}$,
    $\cmiSv[\aDi{\ssY}]\aDi{\ssY}\approx(\log\ssY)^{2+1/(2a)}$,
    $\naRa{(\xdf)}\approx(\log\ssY)^{-p/a}$ and
    $ \vert \log\naRa{(\xdf)} \vert \approx(\log\log\ssY)$.
  \item[\mylabel{freq:ge:strat:kn:qu:il:rate:np:so}{\dg\bfseries{[s-o]}}] Since
    $\bias^2(\xdf)\approx\exp(-\Di^{2p})$ and $\DipenSv\approx\Di^{2a+1}$
    follows
    $\daRa{\aDi{\ssY}}{(\xdf,\Lambda)}\approx\exp(-(\aDi{\ssY})^{2p})\approx\DipenSv[\aDi{\ssY}]\ssY^{-1}\approx
    (\aDi{\ssY})^{2a+1}\ssY^{-1}$ which implies
    $\aDi{\ssY}\approx(\log\ssY)^{1/(2p)}$,
    $\cmiSv[\aDi{\ssY}]\aDi{\ssY}\approx(\log\ssY)^{1/(2p)}$,
    $\naRa{(\xdf)}\approx(\log\ssY)^{(2a+1)/(2p)}\ssY^{-1}$ and
    $ \vert \log\naRa{(\xdf)} \vert \approx(\log\ssY)$.
  \end{Liste}
  We note that  in the three cases \ref{freq:ge:strat:kn:qu:il:rate:np:oo},
  \ref{freq:ge:strat:kn:qu:il:rate:np:os} and \ref{freq:ge:strat:kn:qu:il:rate:np:so} the rate
  $\naRa{(\xdf)}$ coincide with the
  oracle rate $\oRa{\xdf}$ derived in \nref{il:oo} \ref{il:oo:oo},
  \ref{il:oo:os} and \ref{il:oo:so}, respectively.\ilEnd 
\end{il}
% ....................................................................
% Define *Di
% ....................................................................
\begin{te}
  Under \nref{freq:ge:shape:kn:de:pen:oo} for arbitrary $\pdDi,\mdDi \in \nset{1,\ssY}$ let us define
  \begin{multline}\label{freq:ge:strat:kn:qu:de:mDipDi}
    \mDi:=\min\set{\Di\in\nset{1,\mdDi}: \Vnormlp{\xdf_{\underline{0}}}^2\bias^2(\xdf) \leq [\Vnormlp{\xdf_{\underline{0}}}^2+4\cpen]\daRa{\mdDi}{(\xdf)}}\quad\text{and}\\
    \pDi:=\max\set{\Di\in\nset{\pdDi,\ssY} : \penSv \leq 2[3\Vnormlp{\xdf_{\underline{0}}}^2 + 2\cpen] \daRa{\pdDi}{(\xdf)}}
  \end{multline}
  where the defining set obviously contains $\mdDi$ and $\pdDi$, respectively, and hence, it is not empty.
\end{te}
\begin{te}
Considering the third and fourth terms on the right hand side of \eqref{freq:ge:strat:kn:co:agg:e1}, we will use the following lemma to control them.
\end{te}
% ....................................................................
% <<Re Sum Random weights>>
% ....................................................................
\begin{lm}\label{freq:ge:strat:kn:qu:re:SrWe:ag}
Consider the data-driven aggregation weights $\rWe[]$
  as in \eqref{freq:ge:shape:kn:we}.  Under \nref{freq:ge:shape:kn:de:pen:oo} with
  $\cpen\geq8\log(3e)$ for any
  $\mdDi,\pdDi\in\nset{1,\ssY}$ and associated $\pDi,\mDi\in\nset{1,\ssY}$
  as in \eqref{freq:ge:strat:kn:qu:de:mDipDi} hold
  \begin{resListeN}
  \item\label{freq:ge:strat:kn:qu:re:SrWe:ag:i}
    $\rWe[](\nsetro{1,\mDi})
    \Ind{\setB{\Vnormlp{\txdfPr[\mdDi]-\xdfPr[\mdDi]}^2<\cpen\daRa{\mdDi}{(\xdf)}/7}}\leq
    \tfrac{1}{\rWc\cpen}\Ind{\{\mDi>1\}}
    \exp\big(-\tfrac{3\rWc\cpen}{14} \ssY\daRa{\mdDi}{(\xdf)}\big)$;
  \item\label{freq:ge:strat:kn:qu:re:SrWe:ag:ii}
    $\sum_{\Di\in\nsetlo{\pDi,\ssY}}\penSv\rWe
    \Ind{\{\Vnormlp{\txdfPr[\Di]-\xdfPr[\Di]}^2<\penSv/7\}}
    \leq \ssY^{-1}\{\tfrac{16}{\cpen\rWc^{2}}+ \tfrac{8}{\rWc}\}$.
  \end{resListeN}\reEnd
\end{lm}
% ....................................................................
% <<Te Sum MS Random weights>>
% ....................................................................
\begin{te}
  We combine the upper bound in \nref{freq:ge:strat:kn:co:agg} and the bounds given in \nref{freq:ge:strat:kn:qu:re:SrWe:ag}.
  Clearly, due to \nref{freq:ge:strat:kn:qu:re:SrWe:ag} we have
  \begin{displaymath}
    \E\rWe[](\nsetro{1,\mDi})\leq\Ind{\{\mDi>1\}} \{\tfrac{1}{\rWc\cpen}\exp\big(-\tfrac{3\rWc\cpen}{14}n\daRa{\mdDi}{(\xdf)}\big) + \P\big(\Vnormlp{\txdfPr[\mdDi]-\xdfPr[\mdDi]}^2 \geq \tfrac{\cpen}{7}\daRa{\mdDi}{(\xdf)}\big)\}
  \end{displaymath}
  and, hence from \eqref{freq:ge:strat:kn:co:agg} follows immediately
  \begin{multline}\label{freq:ge:strat:kn:qu:e1}
    \E\Vnormlp{\txdfAg-\xdf}^2\leq \ssY^{-1} \{\tfrac{32}{7\cpen\rWc^{2}} + \tfrac{16}{7\rWc}\} + \tfrac{2}{\rWc\cpen}\Vnormlp{\xdf_{\underline{0}}}^2\Ind{\{\mDi>1\}} \exp\big(-\tfrac{3\rWc\cpen}{14}n\daRa{\mdDi}{(\xdf)}\big)\\
    \hfill + 2\Vnormlp{\xdf_{\underline{0}}}^2\Ind{\{\mDi>1\}} \P\big(\Vnormlp{\txdfPr[\mdDi] - \xdfPr[\mdDi]}^2 \geq \tfrac{\cpen}{7} \daRa{\mdDi}{(\xdf)} \big) + \tfrac{2}{7} \penSv[\pDi]  + 2 \Vnormlp{\xdf_{\underline{0}}}^2 \bias[\mDi]^2(\xdf) \\
     +2\sum_{\Di\in\nset{\pdDi,n}}\E\vectp{\Vnormlp{\txdfPr-\xdfPr}^2-\tfrac{1}{7}\penSv}\\
    +\tfrac{2}{7}\sum_{\Di\in\nset{\pdDi,\ssY}}\penSv
    \P\big(\Vnormlp{\txdfPr-\xdfPr}^2\geq\tfrac{1}{7}\penSv\big)
  \end{multline}
\end{te}
% ....................................................................
% Outline Oracle optimality
% ....................................................................
\begin{te}
 The next result can be directly deduced from \nref{freq:ge:strat:kn:qu:re:SrWe:ag} by letting $\rWc\to\infty$.
 However, we think the direct proof given in \nref{pro:freq:ge:strat:kn} provides an interesting illustration of the values $\pDi,\mDi\in\nset{1,\ssY}$ as defined in \eqref{freq:ge:strat:kn:qu:de:mDipDi}.
\end{te}
% ....................................................................
% <<Upper bound random weights>>
% ....................................................................
\begin{lm}\label{freq:ge:strat:kn:qu:re:SrWe:ms}
Consider the data-driven model selection weights $\msWe[]$ as in \eqref{freq:ge:shape:kn:de:msWe}.
Under definition \nref{freq:ge:shape:kn:de:pen:oo} for any $\mdDi,\pdDi\in\nset{1,\ssY}$ and associated $\pDi,\mDi\in\nset{1,n}$ as in \eqref{freq:ge:strat:kn:qu:de:mDipDi} hold
\begin{resListeN}[]
  \item\label{freq:ge:strat:kn:qu:re:SrWe:ms:i}
    $\msWe[](\nsetro{1,\mDi})\Ind{\{\Vnormlp{\theta_{n, \overline{\mdDi}} - \xdfPr[\mdDi]}^2
      <\cpen\daRa{\mdDi}{(\xdf)}/7\}}=0$;
  \item\label{freq:ge:strat:kn:qu:re:SrWe:ms:ii}
    $\sum_{\Di\in\nsetlo{\pDi,\ssY}}\penSv\msWe\Ind{\{\Vnormlp{\txdfPr-\xdfPr}^2<\penSv/7\}}=0$.
  \end{resListeN}
  \reEnd
\end{lm}
% ....................................................................
% <<Te Sum MS Random weights>>
% ....................................................................
\begin{te}
We combine again the upper bound in \nref{freq:ge:strat:kn:co:agg} and the bounds given in \nref{freq:ge:strat:kn:qu:re:SrWe:ms}.
Clearly, due to \nref{freq:ge:strat:kn:qu:re:SrWe:ms} we have
$\E\msWe[](\nsetro{1,\mDi})=\P\big(\Vnormlp{\txdfPr[\mdDi]-\xdfPr[\mdDi]}^2\geq\cpen\daRa{\mdDi}{(\xdf)}/7\big)$
and, hence from \eqref{freq:ge:strat:kn:co:agg} follows immediately
  \begin{multline}\label{freq:ge:strat:kn:qu:e2}
\E\Vnormlp{\txdfPr[\hDi]-\xdf}^2\leq 2\sum\nolimits_{\Di\in\nset{\pdDi,\ssY}}\E\vectp{\Vnormlp{\txdfPr-\xdfPr}^2-\tfrac{1}{7}\penSv}\\
\hfill \tfrac{2}{7}\penSv[\pDi] + 2\Vnormlp{\xdf_{\underline{0}}}^2\bias[\mDi]^2(\xdf) + 2\Vnormlp{\xdf_{\underline{0}}}^2\Ind{\{\mDi>1\}}\P\big(\Vnormlp{\txdfPr[\mdDi]-\xdfPr[\mdDi]}^2\geq\tfrac{\cpen}{7}\daRa{\mdDi}{(\xdf)}\big)\\
+\tfrac{2}{7}\sum_{\Di\in\nset{\pdDi,\ssY}}\penSv\P\big(\Vnormlp{\txdfPr-\xdfPr}^2\geq\tfrac{1}{7}\penSv\big)
\end{multline}
The deviations of the last three terms in the last display \eqref{freq:ge:strat:kn:qu:e2} and also in
\eqref{freq:ge:strat:kn:qu:e1} we bound by exploiting usual concentration inequalities which depend on the model considered.
We hence formulate this property as the assumption to be verified in order to use this strategy.
\end{te}

\begin{as}\label{freq:ge:strat:kn:qu:as}
Remind that we defined $\oiSv=\tfrac{1}{\Di}\sum_{s\in\nset{1,\Di}}\iSv[s]$,

$\miSv= \max\{\iSv[s],s\in\nset{1,\Di}\}$, $\cmSv\geq1$ and $\DipenSv=\cmSv\Di \miSv$.
Assume that there are numerical constants $(\cst{i})_{1 \in \llbracket 1, 11 \rrbracket}$, such that for all $\ssY\in\Nz$ and $\Di\in\nset{1,\ssY}$ holds
  \begin{resListeN}[]
  \item\label{freq:ge:strat:kn:qu:as:i}
    $\E \vectp{\Vnormlp{\txdfPr-\xdfPr}^2 - 12\tfrac{\DipenSv}{\ssY}}  \leq 
    \cst{1} \bigg[\tfrac{\cst{2} \, \miSv}{\ssY}\exp\big(-\cmSv\Di\cst{3} \big)+\tfrac{\cst{4}\Di\miSv}{n^2}\exp\big(- \cst{5} \sqrt{n\cmSv}\big) \bigg]$
  \item\label{freq:ge:strat:kn:qu:as:ii}
    $\P\big(\Vnormlp{\txdfPr-\xdfPr}^2 \geq 12\DipenSv\ssY^{-1}\big)\leq 
    \cst{6} \bigg[\exp\big(-\cst{7} \cmSv\Di\big)
    +\exp\big(-\cst{8} \sqrt{\ssY\cmSv}\big)\bigg]$
  \item\label{freq:ge:strat:kn:qu:as:iii}
    $\P\big(\Vnormlp{\txdfPr-\xdfPr}^2 \geq 12\daRa{\Di}{(\xdf,\Lambda)}\big)\leq 
    \cst{9} \bigg[\exp\big(\tfrac{-\cst{10} \ssY\daRa{\Di}{(\xdf,\Lambda)}}{\miSv}\big)
    +\exp\big(\tfrac{-\cst{11} \ssY\sqrt{\daRa{\Di}{(\xdf,\Lambda)}}}{\sqrt{\Di\miSv}}\big)\bigg]$
  \end{resListeN}
\end{as}

\begin{te} Consider now the partially data-driven aggregation of the
  orthogonal series estimators using either  aggregation weights $\rWe[]$
  as in \eqref{freq:ge:shape:kn:we} or model selection weights $\msWe[]$ as in \eqref{freq:ge:shape:kn:de:msWe}
  combining \nref{freq:ge:strat:kn:qu:as} and the upper bound given
  in \eqref{freq:ge:strat:kn:qu:e1}  or \eqref{freq:ge:strat:kn:qu:e2} we obtain the next result, which proof is immediate and we omit it.
\end{te}

\begin{lm}\label{ak:re:nd:rest}
Assume that \nref{freq:ge:strat:kn:qu:as} holds true and use the penalty described in \nref{freq:ge:shape:kn:de:pen:oo} with $\kappa \geq 84$ so that $\penSv/7 \geq 12 n^{-1} \Delta_{\Lambda}(m)$for any $m$ in $\llbracket 1, n \rrbracket$.
Then, for all $\ssY\in\Nz$ and $\Di\in\nset{1,\ssY}$ hold
  \begin{resListeN}[]
  \item\label{ak:re:nd:rest1} let $\Di_{\cst{3}}:=\floor{  3(2/\cst{3})^2}$ and $\ssY_{\cst{5}}:={15(\cst{5})^{-4}}$ then\\ 
    $\sum_{\Di=1}^{\ssY}\E
    \vectp{\Vnormlp{\txdfPr-\xdfPr}^2-\penSv/7}
    \leq \cst{1}\ssY^{-1}\big[\tfrac{2 \cst{2}}{\cst{3}}\miSv[\Di_{\cst{3}}] + \cst{4} \ssY_{\cst{5}} \miSv[\ssY_{\cst{5}}]\big]$
  \item\label{ak:re:nd:rest2} let
    $\Di_{\cst{7}}:=\floor{3(2 / \cst{7})^2}$ and
    $\ssY_{\cst{8}}:=15(3/\cst{8})^4$ then
    \begin{multline*}
    \sum_{\Di=1}^{\ssY}\penSv/7\P\big(\Vnormlp{\txdfPr-\xdfPr}^2\geq\penSv/7\big)\\
    \leq\cst{6} n^{-1} \big[\miSv[\Di_{\cst{7}}]^2\Di_{\cst{7}}^2+\miSv[\ssY_{\cst{8}}]^2 \ssY_{\cst{8}}^{2}\big]
    \end{multline*}
  \item\label{ak:re:nd:rest3} 
  $\P\big(\Vnormlp{\theta_{n, \overline{m^{\dagger}_{-}}}-\theta^{\circ}_{\overline{m^{\dagger}_{n}}}}^2 \geq 12\daRaS{\mdDi}{\xdf,\Lambda}\big)\leq 
    \cst{9} \big[\exp\big(\tfrac{-\cst{10}\ssY\daRaS{\mdDi}{\xdf,\Lambda}}{\Lambda_{+}(\mdDi)}\big)+(\cst{8})^{-2}\ssY^{-1}\big]$
  \end{resListeN}
  \reEnd
\end{lm}


Injecting \nref{ak:re:nd:rest} in either \nref{freq:ge:strat:kn:qu:e1} or \nref{freq:ge:strat:kn:qu:e2} we directly obtain the following result.

\begin{lm}\label{freq:ge:strat:kn:qu:ub}
Assume that \nref{freq:ge:strat:kn:qu:as} holds true.
Consider the penalty sequence $\penSv$ as in \nref{freq:ge:shape:kn:de:pen:oo} with numerical constant $\cpen \geq 84$.
Let $\widehat{\theta}^{(\eta)}$ be an aggregation estimator using either the aggregation weights \nref{freq:ge:shape:kn:we} or the model selection weights \nref{freq:ge:shape:kn:de:msWe}.
Let $\ssY_{\cst{5}}$, $n_{\cst{8}}$, $m_{\cst{3}}$, and $m_{\cst{7}}$ be as in \nref{{ak:re:nd:rest}}.
Then, there is a finite numerical constant $\cst{}$ such that for any $\mdDi$, $\pdDi$ and associated $\mDi$ and $\pDi$ as in \nref{freq:ge:strat:kn:qu:de:mDipDi} holds
	\begin{multline}\label{freq:ge:strat:kn:qu:ub:e1}
\E\Vnormlp{\widehat{\theta}^{(\eta)}-\xdf}^2\leq \tfrac{2}{7}\penSv[\pDi] + 2\Vnormlp{\xdf_{\underline{0}}}^2\bias[\mDi]^2(\xdf) + \cst{}\Vnormlp{\xdf_{\underline{0}}}^2\Ind{\{\mDi>1\}} \big[\exp\big(-\cst{10}\delta_{\Lambda}(\mdDi) \mdDi\big)\big]\\
 + \cst{}\big[\Vnormlp{\xdf_{\underline{0}}}^2\Ind{\{\mDi>1\}} + \miSv[\Di_{\cst{7}}]^2\Di_{\cst{7}}^2+\miSv[\ssY_{o}]^2\big]\ssY^{-1}.
\end{multline} 
\reEnd
\end{lm}
% ....................................................................
% <<Re upper bound ag>>
% ....................................................................
%\begin{lm}\label{freq:ge:strat:kn:qu:ub}
%Under \nref{freq:ge:strat:kn:qu:as}, consider the penalty sequence $\penSv:=\penD$, $\Di\in\nset{1,n}$, as in \nref{freq:ge:shape:kn:de:pen:oo}.
%Let $\widehat{\theta}^{(\eta)}=\sum_{\Di=1}^{\ssY}\We\txdfPr$ be an aggregation of the orthogonal series estimators, using either aggregation weights $\rWe[]$ as in \eqref{freq:ge:shape:kn:we}, or model selection weights $\msWe[]$ as in \eqref{freq:ge:shape:kn:de:msWe}.
%  
%There is a finite numerical constant $\cst{}>0$ such that for any $\mdDi,\pdDi\in\nset{1,n}$ and associated $\pDi,\mDi\in\nset{1,n}$ as defined in \eqref{freq:ge:strat:kn:qu:de:mDipDi} hold
%\begin{equation}\label{freq:ge:strat:kn:qu:ub:e1}
%\E\Vnormlp{\txdfPr[\hDi]-\xdf}^2\leq \cst{} \daRa{\dDi}{(\xdf)} + \tfrac{2}{7}\penSv[\pDi] + 2\Vnormlp{\xdf_{\underline{0}}}^2\bias[\mDi]^2(\xdf)
%\end{equation}
%\reEnd
%\end{lm}
% ....................................................................
% <<Te upper bound ag p np>>
% ....................................................................
\begin{te} The last bound allows us to derive an upper bound of the
  risk for  data-driven aggregated estimator  in the two cases
  \ref{oo:xdf:p} and \ref{oo:xdf:np} introduced in \nref{bm:ak}.
\end{te}
% ....................................................................
% <<Re upper bound ag p np>>
% ....................................................................
\begin{thm}\label{freq:ge:strat:kn:qu:pnp}
Under \nref{freq:ge:strat:kn:qu:as}, consider the penalty sequence $\penSv:=\penD$, $\Di\in\nset{1,n}$, as in \nref{freq:ge:shape:kn:de:pen:oo} with numerical constant $\cpen \geq 84$.
Let $\txdfAg[{\erWe[]}]=\sum_{\Di=1}^{\ssY}\We\txdfPr$ be an aggregation of the orthogonal series estimators, using either aggregation weights $\rWe[]$ as in \eqref{freq:ge:shape:kn:we}, or model selection weights $\msWe[]$ as in \eqref{freq:ge:shape:kn:de:msWe}.
\begin{Liste}[]
\item[\mylabel{ak:ag:ub:pnp:p}{\dgrau\bfseries{(p)}}]Assume there is $K\in\Nz$
  with   $1\geq \bias[{[K-1] }](\xdf)>0$ and $\bias[\Di](\xdf)=0$. For
  $K>0$ let
  $ c_{\xdf}:=\tfrac{\Vnormlp{\xdf_{\underline{0}}}^2+4\cpen}{\Vnormlp{\xdf_{\underline{0}}}^2\bias[{[K-1]}]^2(\xdf)}>1$ and
  $\ssY_{\xdf}:=\gauss{c_{\xdf}\DipenSv[K]}\in\Nz$. If
  $\ssY\in\nset{1,\ssY_{\xdf}}$ then set $\sDi{\ssY}:=\Di_{\cst{3}}\log(\ssY)$, and otherwise if
  $\ssY>\ssY_{\xdf}$ then set
  $\sDi{\ssY}:=\max\{\Di\in\nset{K,\ssY}:\ssY>c_{\xdf}\DipenSv\}$
  where the defining set contains $K$ and thus is not empty.
There is a finite constant $\cst{\xdf,\Lambda}$
given in \eqref{ak:ag:ub:pnp:p7} depending only on $\xdf$ and $\Lambda$ such that for all $n\in\Nz$ holds
\begin{equation}\label{ak:ag:ub:pnp:e1}
  \nRi{\txdfAg[{\erWe[]}]}{\xdf,\Lambda}
  % \E\Vnormlp{\txdfAg[{\erWe[]}]-\xdf}^2
  \leq
  \cst{}\Vnormlp{\xdf_{\underline{0}}}^2\big[
  \ssY^{-1}\vee\exp\big(-\cst{10}\cmiSv[\sDi{\ssY}]\sDi{\ssY}\big)\big]
  + \cst{\xdf,\Lambda}\ssY^{-1}.
\end{equation}
\item[\mylabel{ak:ag:ub:pnp:np}{\dgrau\bfseries{(np)}}] Assume that
  $\bias(\xdf)>0$ for all  $\Di\in\Nz$.
There is a finite finite constant $\cst{\xdf,\Lambda}$ given in
\eqref{ak:ag:ub:pnp:p8} depending only $\xdf$ and $\Lambda$ such that for all
$\ssY\in\Nz$  holds 
 \begin{equation}\label{ak:ag:ub:pnp:e2}
   \nRi{\txdfAg[{\erWe[]}]}{\xdf,\Lambda}
   % \E\Vnormlp{\txdfAg[{\erWe[]}]-\xdf}^2
    \leq 
   \cst{}(\Vnormlp{\xdf_{\underline{0}}}^2\vee1)\min_{\Di\in\nset{1,\ssY}}\big[\dRa{\Di}{\xdf,\Lambda}\vee\exp\big(-\cst{10}\cmiSv\Di\big)\big]\\
   +\cst{\xdf,\Lambda}\ssY^{-1}.
\end{equation}
\end{Liste}  
\end{thm}

Hence, using \nref{freq:ge:strat:kn:qu:pnp} gives us the following result.
% ....................................................................
% <<Re upper bound ag p np>>
% ....................................................................
\begin{cor}\label{ge:ak:ag:ub2:pnp}
  Let the assumptions of \nref{freq:ge:strat:kn:qu:pnp} be satisfied.
  \begin{Liste}[]
  \item[\mylabel{ge:ak:ag:ub2:pnp:p}{\dgrau\bfseries{(p)}}]
    If in addition
    \begin{inparaenum}\item[\mylabel{ge:ak:ag:ub2:pnp:pc}{{\dgrau\bfseries(A1)}}]
      there is $\ssY_{\xdf,\Lambda}\in\Nz$ such that
      $\cmiSv[\sDi{\ssY}]\sDi{\ssY}\geq (\cst{10})^{-1}(\log\ssY)$ for all
      $\ssY\geq \ssY_{\xdf,\Lambda}$
    \end{inparaenum}
    holds true, then there is a constant $\cst{\xdf,\Lambda}$ depending
    only on $\xdf$ and $\Lambda$ such that for all $n\in\Nz$ holds
    $\nRi{\txdfAg[{\erWe[]}]}{\xdf,\Lambda} \leq
    \cst{\xdf,\Lambda}\ssY^{-1}$.
  \item[\mylabel{ge:ak:ag:ub2:pnp:np}{\dgrau\bfseries{(np)}}]
    If in addition
    \begin{inparaenum}\item[\mylabel{ge:ak:ag:ub2:pnp:npc}{{\dgrau\bfseries(A2)}}]
      there is  $\ssY_{\xdf,\Lambda}\in\Nz$ such that
      
      $\aDi{\ssY}(\xdf)\cmSv[\aDi{\ssY}(\xdf)]\geq (\cst{10})^{-1} \vert \log\naRa{(\xdf,\Lambda)} \vert $
      for all $\ssY\geq \ssY_{\xdf,\Lambda}$
    \end{inparaenum}
    holds true, then there is a constant $\cst{\xdf,\Lambda}$ depending
    only on $\xdf$ and $\Lambda$ such that $\nRi{\txdfAg[{\erWe[]}]}{\xdf,\Lambda}
    \leq \cst{\xdf,\Lambda}\naRa{(\xdf,\Lambda)}$ for all $n\in\Nz$ holds true.
  \end{Liste}  
\end{cor}

% ....................................................................
% <<Il upper bound ag np>>
% ....................................................................
\begin{il}\label{ak:ag:ub:pnp:il}Let us briefly illustrate the last
  results. In case \ref{ak:ag:ub:pnp:p} the partially data-driven aggregation leads
  to an estimator attaining the parametric oracle rate (see
  \nref{oo:rem:ora}), if the additional assumption
  \ref{ge:ak:ag:ub2:pnp:pc} is satisfied.  Consider  the two cases \ref{il:edf:o} and
  \ref{il:edf:s} for $\edf$ as in \nref{il:oo}:
\begin{Liste}[]
\item[\mylabel{ak:il:edf:o}{\dg\bfseries{(o)}}]   $1\approx\DipenSv[\sDi{\ssY}]\ssY^{-1}\approx(\sDi{\ssY})^{2a+1}\ssY^{-1}$   implies $\sDi{\ssY}\approx\ssY^{1/(2a+1)}$ and
    $\sDi{\ssY}\cmSv[\sDi{\ssY}]\approx\ssY^{1/(2a+1)}$
\item[\mylabel{ak:il:edf:s}{\dg\bfseries{(s)}}]  $\ssY\approx\DipenSv[\sDi{\ssY}]\approx
    (\sDi{\ssY})^{1+4a}\exp((\sDi{\ssY})^{2a})$ implies
    $\sDi{\ssY}\approx(\log
    \ssY-\tfrac{1+4a}{2a}\log\log\ssY)^{1/(2a)}$ and 
    $\sDi{\ssY}\cmSv[\sDi{\ssY}]\approx (\log \ssY)^{2+1/(2a)}$.
  \end{Liste}
  Clearly in both cases \ref{ak:il:edf:o} and \ref{ak:il:edf:s}, the
  additional condition \ref{ge:ak:ag:ub2:pnp:pc} of \nref{ge:ak:ag:ub2:pnp}
  holds true. Therefore, in this situation the aggregated estimator
  attains the oracle rate.  On the other hand side, in case
  \ref{ge:ak:ag:ub2:pnp:np} the partially data-driven aggregation leads
  to an estimator attaining the rate $\naRa{(\xdf,\Lambda)}$ (see
  \nref{oo:rem:ora}), if the additional assumption
  \ref{ge:ak:ag:ub2:pnp:npc} is satisfied. Otherwise, the upper bound
  faces a deterioration of the rate, which we illustrate considering as
  in \nref{freq:ge:strat:kn:qu:il:rate} usual behaviour \ref{freq:ge:strat:kn:qu:il:rate:np:oo},
  \ref{freq:ge:strat:kn:qu:il:rate:np:os} and \ref{freq:ge:strat:kn:qu:il:rate:np:so} for the
  sequences $\Nsuite[\Di]{\bias[\Di](\xdf)}$ and
  $\Nsuite[\Di]{\iSv[\Di]}$. In case \ref{freq:ge:strat:kn:qu:il:rate:np:oo},
  \ref{freq:ge:strat:kn:qu:il:rate:np:os} and \ref{freq:ge:strat:kn:qu:il:rate:np:so} only with
  $p<1/2$ the assumption \ref{ge:ak:ag:ub2:pnp:npc} is satisfied, and
  $\naRa{(\xdf,\Lambda)}$ equals the oracle rate $\oRa{\xdf,\Lambda}$ (cf.
  \ref{freq:ge:strat:kn:qu:il:rate:np:oo} \ref{freq:ge:strat:kn:qu:il:rate:np:oo},
  \ref{freq:ge:strat:kn:qu:il:rate:np:os} and \ref{freq:ge:strat:kn:qu:il:rate:np:so}). Thereby, the
  partially data-driven aggregation leads to an estimator attaining
  the oracle rate $\oRa{\xdf,\Lambda}$. In case \ref{freq:ge:strat:kn:qu:il:rate:np:os}
  with $p\geq1/2$ the assumption \ref{ge:ak:ag:ub2:pnp:npc} is not
  satisfied. However, with
  $\sDi{\ssY}:=\Di_{\cst{3}} \vert \log\naRaS{\xdf,\Lambda} \vert \approx(\log\ssY)$ holds
  $\min_{\Di\in\nset{1,\ssY}}\big[\dRa{\Di}{\xdf,\Lambda}\vee\exp\big(\tfrac{-\cmiSv\Di}{\Di_{\cst{3}}}\big)\leq\daRa{\sDi{\ssY}}{\xdf,\Lambda}\approx(\log\ssY)^{2a+1}\ssY^{-1}$.
  In this situation the rate of the partially data-driven estimator
  $\txdfAg[{\erWe[]}]$ features a deterioration by a logarithmic factor
  $(\log\ssY)^{(2a+1)(1-1/(2p))}$ compared to the oracle rate, i.e.
  $\daRaS{\sDi{\ssY}}{\xdf,\Lambda}\approx(\log\ssY)^{2a+1}\ssY^{-1}$ versus
  $\oRaS{\xdf,\Lambda}\approx(\log\ssY)^{(2a+1)/(2p)}\ssY^{-1}$.\ilEnd
\end{il}

\subsubsection{Maximal risk bounds}\label{freq:ge:strat:kn:ma}
\begin{te}
  By applying \nref{freq:ge:strat:kn:co:agg}, we derive bounds for the maximal risk over ellipsoids $\rwCxdf$ of the aggregated estimator $\txdfAg[{\erWe[]}]$ using either aggregation weights $\rWe[]$ as in \eqref{freq:ge:shape:kn:we} or model selection weights $\msWe[]$ as in \eqref{freq:ge:shape:uk:we}.
  Therefore, we aim next to control the second and third right hand side term in \eqref{freq:ge:strat:kn:co:agg:e1} uniformly over $\rwCxdf$.
  Keeping the definition \eqref{oo:de:mra} of $\dRa{\Di}{\xdfCw[],\Lambda}$ in mind it holds $\xdfCr^2\dRa{\Di}{\xdfCw[],\Lambda} \geq \Vnormlp{\xdf_{\underline{0}}}^2 \bias^2(\xdf)$ uniformly for all $\xdf \in \rwCxdf$ and for all $\Di\in\Nz$.
  The proofs for the results displayed here can be found in \nref{pro:freq:ge:strat:kn:ma}.
  We then gives the following definition for the sequence which we want to prove to be an upper bound for the maximal risk of the aggregation estimator.
  Note that in this case we use $\DipenSv$ and $\penSv$ as defined in \nref{freq:ge:shape:kn:de:pen:oo} and hence the rates for the quadratic as well as the maximal risk are obtained for the same estimator.
  
\begin{de}\label{freq:ge:strat:kn:ma:de:rate}
  Let be the following family of sequences,
  $\daRa{\Di}{(\xdfCw[])}:=\daRa{\Di}{(\xdfCw[],\Lambda)}:=[\xdfCw^2\vee \DipenSv\,\ssY^{-1}]$.
  Then it holds for all $\Di$ in $\nset{1,\ssY}$ and $\xdf$ in $\rwCxdf$
  \begin{equation}\label{freq:ge:strat:kn:ma:de:rate:e1}
    [\xdfCr^2+\cpen]\daRa{\Di}{(\xdfCw[])}\geq\big[\Vnormlp{\xdf_{\underline{0}}}^2\bias^2(\xdf)\vee\penSv\big]
\end{equation}
Considering the following specific case, we aim to show that it describes an upper bound for the maximal risk over $\rwCxdf$ for our aggregation estimator,
\begin{multline*}
\aDi{\ssY}(\xdfCw[]):=\argmin\Nset[\Di\in\Nz]{\daRa{\Di}{(\xdfCw[],\Lambda)}}\in\nset{1,\ssY}\\
    \naRa{(\xdfCw[])}:=\naRa{(\xdfCw[],\Lambda)}:=\min\Nset[\Di\in\Nz]{\daRa{\Di}{(\xdfCw[],\Lambda)}}; \text{ with } \daRa{\aDi{\ssY}(\xdfCw[])}{(\xdfCw[],\Lambda)}=\naRa{(\xdfCw[],\Lambda)}
    \end{multline*}
\assEnd
\end{de}

\end{te}
% ....................................................................
% <<Il upper bound oo>>
% ....................................................................
\begin{il}\label{ak:mrb:ass:il}
Let us illustrate \nref{freq:ge:strat:kn:ma:de:rate} considering as in \nref{il:mm} usual behaviour \ref{il:mm:oo}, \ref{il:mm:so} and \ref{il:mm:os} for the sequences $\Nsuite[\Di]{\xdfCw[(\Di)]}$ and $\Nsuite[\Di]{\iSv[\Di]}$:
  \begin{Liste}[]
  \item[\mylabel{ak:mrb:ass:il:oo}{\dg\bfseries{[o-o]}}] Since
   $\DipenSv\approx\Di^{2a+1}$
    follows $\aDi{\ssY}(\xdfCw[])\approx\ssY^{1/(2p+2a+1)}$,
    $\cmiSv[\aDi{\ssY}({\xdfCw[]})]\aDi{\ssY}(\xdfCw[])\approx\ssY^{1/(2p+2a+1)}$,
    $\naRa{(\xdfCw[],\Lambda)}\approx\ssY^{-2p/(2p+2a+1)}$ and $ \vert \log\naRa{(\xdfCw[],\Lambda)} \vert \approx(\log\ssY)$.
 \item[\mylabel{ak:mrb:ass:il:os}{\dg\bfseries{[o-s]}}]
    Since $\DipenSv\approx\Di^{1+4a}\exp(\Di^{2a})$ follows  $\aDi{\ssY}(\xdfCw[])\approx(\log\ssY)^{1/(2a)}$, $\cmiSv[\aDi{\ssY}({\xdfCw[]})]\aDi{\ssY}(\xdfCw[])\approx(\log\ssY)^{2+1/(2a)}$, 
    $\naRa{(\xdfCw[],\Lambda)}\approx(\log\ssY)^{-p/a}$ and $ \vert \log\naRa{(\xdfCw[],\Lambda)} \vert \approx(\log\log\ssY)$.
 \item[\mylabel{ak:mrb:ass:il:so}{\dg\bfseries{[s-o]}}]  Since
   $\DipenSv\approx\Di^{2a+1}$
    follows  $\aDi{\ssY}(\xdfCw[])\approx(\log\ssY)^{1/(2p)}$, $\cmiSv[\aDi{\ssY}({\xdfCw[]})]\aDi{\ssY}(\xdfCw[])\approx(\log\ssY)^{1/(2p)}$,
    $\naRa{(\xdfCw[],\Lambda)}\approx(\log\ssY)^{(2a+1)/(2p)}\ssY^{-1}$
    and     $ \vert \log\naRa{(\xdfCw[],\Lambda)} \vert \approx(\log\ssY)$.\ilEnd
  \end{Liste}
    We note that  in the three cases \ref{ak:mrb:ass:il:oo},
  \ref{ak:mrb:ass:il:os} and \ref{ak:mrb:ass:il:so} the rate
  $\naRa{(\xdfCw[],\Lambda)}$ coincide with the
  minimax rate $\mnRa{\xdfCw[],\Lambda}$ derived in \nref{il:mm} \ref{il:mm:oo},
  \ref{il:mm:os} and \ref{il:mm:so}, respectively.\ilEnd 
\end{il}
\begin{te}  
Keeping in mind \eqref{freq:ge:strat:kn:ma:de:rate:e1} for any $\pdDi,\mdDi\in\nset{1,\ssY}$ let us define 
\begin{multline}\label{freq:ge:strat:kn:ma:de:mDipDi}
\mDi:=\min\set{\Di\in\nset{1,\mdDi}: \Vnormlp{\xdf_{\underline{0}}}^2\bias[\Di]^2(\xdf)\leq
  [\xdfCr^2+4\cpen]\dRa{\mdDi}{\xdfCw[]}}\quad\text{and}\\\pDi:=\max\set{\Di\in\nset{\pdDi,\ssY}:
   \penSv \leq 2[3\xdfCr^2+ 2\cpen] \dRa{\pdDi}{\xdfCw[]}}
\end{multline}
where  the defining sets obviously contains $\mdDi$ and $\pdDi$, repsectively, and hence, they are
not empty.
\end{te}
% ....................................................................
% <<Re maximal Sum Random weights>>
% ....................................................................
\begin{lm}\label{ak:re:SrWe:ag:mm}
Consider the data-driven aggregation weights $\rWe[]$
  as in \eqref{freq:ge:shape:kn:we} and the rates described in \nref{freq:ge:strat:kn:ma:de:rate} with
  $\cpen\geq8\log(3e)$  for any
  $\mdDi,\pdDi\in\nset{1,\ssY}$ and associated $\pDi,\mDi\in\nset{1,\ssY}$
  as in \eqref{freq:ge:strat:kn:ma:de:mDipDi} hold
  \begin{resListeN}
  \item\label{ak:re:SrWe:ag:mm:i}
    $\rWe[](\nsetro{1,\mDi})
    \Ind{\setB{\Vnormlp{\txdfPr[\mdDi]-\xdfPr[\mdDi]}^2<\cpen\daRa{\mdDi}{(\xdfCw[])}/7}}\leq
    \tfrac{1}{\rWc\cpen}\Ind{\{\mDi>1\}}
    \exp\big(-\tfrac{3\rWc\cpen}{14} \ssY\daRa{\mdDi}{(\xdfCw[])}\big)$;
  \item\label{ak:re:SrWe:ag:mm:ii}
    $\sum_{\Di\in\nsetlo{\pDi,\ssY}}\penSv\rWe
    \Ind{\{\Vnormlp{\txdfPr[\Di]-\xdfPr[\Di]}^2<\penSv/7\}}
    \leq \ssY^{-1}\{\tfrac{16}{\cpen\rWc^{2}}+ \tfrac{8}{\rWc}\}$.
  \end{resListeN}
  \reEnd
\end{lm}
% ....................................................................
% <<Te maximal Sum Random weights>>
% ....................................................................
\begin{te}
  The next result can also immediately be deduced from
  \nref{ak:re:SrWe:ag:mm} letting $\rWc\to\infty$. On the other hand
  side, a direct proof follows line by line the proof of
  \nref{freq:ge:strat:kn:qu:re:SrWe:ms} using \eqref{freq:ge:strat:kn:ma:de:rate:e1} rather than
  \eqref{freq:ge:strat:kn:qu:de:rate:e1}, and we omit the details.
\end{te}
% ....................................................................
% <<Maximal Upper bound random weights ms>>
% ....................................................................
\begin{lm}\label{ak:re:SrWe:ms:mm}
Consider the data-driven model selection weights $\msWe[]$
  as in \eqref{freq:ge:shape:uk:we}.  Under definition \nref{freq:ge:strat:kn:ma:de:rate} 
  for any $\mdDi,\pdDi\in\nset{1,\ssY}$ and associated
  $\pDi,\mDi\in\nset{1,n}$ as in \eqref{freq:ge:strat:kn:ma:de:mDipDi} hold
  \begin{resListeN}[]
  \item\label{ak:re:SrWe:ms:mm:i}
    $\msWe[](\nsetro{1,\mDi})\Ind{\{\Vnormlp{\txdfPr[\mdDi]-\xdfPr[\mdDi]}^2
      <\cpen\daRa{\mdDi}{(\xdfCw[])}/7\}}=0$;
  \item\label{ak:re:SrWe:ms:mm:ii}
    $\sum_{\Di\in\nsetlo{\pDi,\ssY}}\penSv\msWe\Ind{\{\Vnormlp{\txdfPr-\xdfPr}^2<\penSv/7\}}=0$.
  \end{resListeN}
  \reEnd
\end{lm}
% --------------------------------------------------------------------
% <<Text unifom bounds over ellipsoid>>
% --------------------------------------------------------------------
%\begin{te}
%Keep in mind that $\ydf=\xdf\cdot\edf$.
%We note that uniformly for all $\xdf\in\rwCxdf$ by applying the
%  Cauchy-Schwarz inequality holds   $\Vnormlp[1]{\fydf}\leq
%\Vnorm[{\xdfCw[]}]{\edf}\Vnorm[1/{\xdfCw[]}]{\xdf}\leq \xdfCr\Vnorm[{\xdfCw[]}]{\edf}$. Moreover,
%\nref{re:conc} \ref{re:conc:iii} still holds with
%$\daRa{\mdDi}{(\xdf,\Lambda)}$ replaced by $\daRa{\mdDi}{(\xdfCw[],\Lambda)}$, since
%$\daRa{\mdDi}{(\xdfCw[],\Lambda)}\geq \DipenSv\ssY^{-1}$ for all
%$\ssY,\Di\in\Nz$. Thereby, we obtain the next assertion, and we omit its elementary proof.
%\end{te}


\begin{lm}\label{ak:re:nd:rest:ma}
Assume that \nref{freq:ge:strat:kn:qu:as} holds true and use the penalty described in \nref{freq:ge:shape:kn:de:pen:oo} with $\kappa \geq 84$ so that $\penSv/7 \geq 12 n^{-1} \Delta_{\Lambda}(m)$for any $m$ in $\llbracket 1, n \rrbracket$.
Then, for all $\ssY\in\Nz$ and $\Di\in\nset{1,\ssY}$ hold
  \begin{resListeN}[]
  \item\label{ak:re:nd:rest1:ma} let $\Di_{\cst{3}}:=\floor{  3(2/\cst{3})^2}$ and $\ssY_{\cst{5}}:={15(\cst{5})^{-4}}$ then\\ 
    $\sup\limits_{\xdf \in \Theta(\mathfrak{a}, r)}\sum_{\Di=1}^{\ssY}\E
    \vectp{\Vnormlp{\txdfPr-\xdfPr}^2-\penSv/7}
    \leq \cst{1}\ssY^{-1}\big[\miSv[\Di_{\cst{3}}] + \miSv[\ssY_{\cst{5}}]\big]$
  \item\label{ak:re:nd:rest2:ma} let
    $\Di_{\cst{7}}:=\floor{3(2 / \cst{7})^2}$ and
    $\ssY_{\cst{8}}:=15(3/\cst{8})^4$ then\\
    $\sup\limits_{\xdf \in \Theta(\mathfrak{a}, r)}\sum_{\Di=1}^{\ssY}\tfrac{\penSv}{7}\P\big(\Vnormlp{\txdfPr-\xdfPr}^2
    \geq\tfrac{\penSv}{7}\big)\leq\cst{6} n^{-1} \big[\miSv[\Di_{\cst{7}}]^2\Di_{\cst{7}}^2+\miSv[\ssY_{\cst{8}}]^2\big]$
  \item\label{ak:re:nd:rest3:ma} 
  $\sup\limits_{\xdf \in \Theta(\mathfrak{a}, r)}\P\big(\Vnormlp{\theta_{n, \overline{m^{\dagger}_{-}}}-\theta^{\circ}_{\overline{m^{\dagger}_{n}}}}^2 \geq 12\daRaS{\mdDi}{\xdf,\Lambda}\big)\leq 
    \cst{9} \big[\exp\big(\tfrac{-\cst{10}\ssY\daRaS{\mdDi}{\xdf,\Lambda}}{\Lambda_{+}(\mdDi)}\big)+\ssY^{-1}\big]$
  \end{resListeN}
  \reEnd
\end{lm}
% --------------------------------------------------------------------
% <<Te penalty>>
% --------------------------------------------------------------------
\begin{te}
  Consider now the partially data-driven aggregation of the orthogonal series estimators using either aggregation weights $\rWe[]$ as in \eqref{freq:ge:shape:kn:we} or model selection weights $\msWe[]$ as in \eqref{freq:ge:shape:uk:we}.
  From \eqref{freq:ge:strat:kn:co:agg:e1}, combining \nref{ak:re:SrWe:ag:mm} and \nref{ak:re:SrWe:ms:mm} we obtain by replacing $\daRa{\mdDi}{(\xdf)}$ by $\daRa{\mdDi}{(\xdfCw[])}$ upper bounds similar to \eqref{freq:ge:strat:kn:qu:e1} and  \eqref{freq:ge:strat:kn:qu:e2}, respectively.
  Those upper bounds together with \nref{ak:re:nd:rest:ma} allow us to show the next assertion \nref{ak:ag:ub:mm}.
\end{te}
% ....................................................................
% <<Re upper bound ag>>
% ....................................................................
\begin{lm}\label{ak:ag:ub:mm}
Assume that \nref{freq:ge:strat:kn:qu:as} holds true.

Consider the penalty sequence $\penSv:=\penD$,
  $\Di\in\nset{1,n}$, as in \nref{freq:ge:shape:kn:de:pen:oo} with numerical
  constant $\cpen$ to be specified depending on the model.
  
  Let $\widehat{\theta}^{(\eta)}=\sum_{\Di=1}^{\ssY}
  \We\txdfPr$ be an aggregation of the orthogonal series estimators using either
  aggregation weights $\rWe[]$
  as in \eqref{freq:ge:shape:kn:we} or model selection weights $\msWe[]$
  as in \eqref{freq:ge:shape:uk:we}.
  There is a finite numerical constant $\cst{}>0$ such that for any
  $\xdf\in\rwCxdf$, $\mdDi,\pdDi\in\nset{1,n}$ and associated $\pDi,\mDi\in\nset{1,n}$ as defined in \eqref{freq:ge:strat:kn:ma:de:mDipDi} hold
    \begin{equation}\label{ak:ag:ub:mm:e1}
    \E\Vnormlp{\txdfAg[{\erWe[]}]-\xdf}^2\leq \tfrac{2}{7}\penSv[\pDi]
    +2\Vnormlp{\xdf_{\underline{0}}}^2\bias[\mDi]^2(\xdf)
    %\\\hfill
 + \cst{} \naRa{(\xdfCw[])}.
  \end{equation}
  \reEnd
\end{lm}
% ....................................................................
% <<Te upper bound ag p np>>
% ....................................................................
\begin{te}
The last bound allows us to derive an upper bound of the maximal risk over the ellipsoid $\rwCxdf$ for the partially data-driven aggregated estimator in the case \ref{oo:xdf:np} introduced in \nref{bm:ak}.
\end{te}
% ....................................................................
% <<Re upper bound ag p np>>
% ....................................................................
\begin{thm}\label{ak:ag:ub:pnp:mm}
Assume that \nref{freq:ge:strat:kn:qu:as} holds true and consider the penalty sequence $\penSv:=\penD$, $\Di\in\nset{1,n}$, as in \nref{freq:ge:shape:kn:de:pen:oo}.
Let $\txdfAg[{\erWe[]}]=\sum_{\Di=1}^{\ssY} \We\txdfPr$ be an aggregation of the orthogonal series estimators using either aggregation weights $\rWe[]$ as in \eqref{freq:ge:shape:kn:we} or model selection weights $\msWe[]$ as in \eqref{freq:ge:shape:uk:we}. % Let $\Di_{\edf,\xdfCr}:=\floor{3(800\Vnorm[{\xdfCw[]}]{\edf}\xdfCr)^2}$ and
    % $ \ssY_{o}:=15({300})^4$.
There is a finite constant $\cst{\xdfCw[],\xdfCr,\Lambda}$ given in
\eqref{ak:ag:ub:pnp:p8} depending only on $\xdfCw[]$, $\xdfCr$ and $\Lambda$ such that for all
$\ssY\in\Nz$ and for all $\sDi{\ssY}\in\nset{\aDi{\ssY}(\xdfCw[]),\ssY}$  with
 $\aDi{\ssY}(\xdfCw[])\in\nset{1,n}$ as in \nref{freq:ge:strat:kn:ma:de:rate} holds 
 \begin{equation}\label{ak:ag:ub:pnp:mm:e1}
 \nRi{\txdfAg[{\erWe[]}]}{\rwCxdf,\Lambda}
   %\sup_{\xdf\in\rwCxdf}\E\Vnormlp{\txdfAg[{\erWe[]}]-\xdf}^2% \leq
    \leq 
   \cst{}(\xdfCr^2\vee1)\min_{\Di\in\nset{1,\ssY}}\big[\daRa{\Di}{(\xdfCw[],\Lambda)}\vee\exp\big(-\cst{10}\cmiSv\Di\big)]\big)\big]
   +\cst{\xdfCw[],\xdfCr,\Lambda}\ssY^{-1}.
\end{equation}
\reEnd
\end{thm}

\begin{cor}\label{ge:ak:ag:ub2:pnp:mm}
  Let the assumptions of \nref{ak:ag:ub:pnp:mm} be satisfied.  If in
  addition
  \begin{inparaenum}\item[\mylabel{ge:ak:ag:ub2:pnp:mm:c}{{\dgrau\bfseries(A)}}]
    there is $\ssY_{\xdfCw[],\xdfCr,\Lambda}\in\Nz$  such that
    $\aDi{\ssY}({\xdfCw[]})\cmSv[\aDi{\ssY}({\xdfCw[]})]\geq(\cst{10})^{-1} \vert \log\naRa{(\xdfCw[])} \vert $
    for all $\ssY\geq \ssY_{\xdfCw[],\xdfCr,\Lambda}$
  \end{inparaenum}
  holds true, then there is a constant $\cst{\xdfCw[],\xdfCr,\Lambda}$
  depending only on $\rwCxdf$ and $\Lambda$ such that
  $ \nRi{\txdfAg[{\erWe[]}]}{\rwCxdf,\Lambda} \leq
  \cst{\xdfCw[],\xdfCr,\Lambda}\naRa{(\xdfCw[],\Lambda)}$ for all $n\in\Nz$
  holds true.
\end{cor}
% % ....................................................................
% % <<Rem upper bound ag np>>
% % ....................................................................
% \begin{rmk}\label{ak:ag:ub:pnp:mm:rem} Keeping in mind that $\aDi{\ssY}({\xdfCw[]})\cmSv[\aDi{\ssY}({\xdfCw[]})]\to\infty$ as $\ssY\to\infty$ if there
%   is $\ssY_{\xdfCw[],\xdfCr,\Lambda}\in\Nz$ such that for all $\ssY\geq \ssY_{\xdfCw[],\xdfCr,\Lambda}$ in addition
%   $\aDi{\ssY}({\xdfCw[]})\cmSv[\aDi{\ssY}({\xdfCw[]})]\geq\Di_{\xdfCw[],\xdfCr,\Lambda} \vert \log\naRa{(\xdfCw[])} \vert $
%    holds true, then we have trivially
%   $\exp\big(\tfrac{-\cmiSv[\aDi{\ssY}({\xdfCw[]})]\aDi{\ssY}({\xdfCw[]})}{200\Vnorm[{\xdfCw[]}]{\edf}\xdfCr}\big)\leq
%   \naRa{(\xdfCw[])}$ while for $\ssY< \ssY_{\xdfCw[],\xdfCr,\Lambda}$ we have
%   $\exp\big(\tfrac{-\cmiSv[\aDi{\ssY}({\xdfCw[]})]\aDi{\ssY}({\xdfCw[]})}{200\Vnorm[{\xdfCw[]}]{\edf}\xdfCr}\big)\leq1\leq
%   \ssY\naRa{(\xdfCw[])}\leq \ssY_{\xdfCw[],\xdfCr,\Lambda} \naRa{{\xdfCw[]}}$. Thereby, from
%   \eqref{ak:ag:ub:pnp:e2} with $\sDi{\ssY}:=\aDi{\ssY}({\xdfCw[]})$ follows immediately
%   \begin{equation}\label{ak:ag:ub:pnp:rem:mm:e1}
%     \nRi{\txdfAg[{\erWe[]}]}{\rwCxdf,\Lambda}
%     % \E\Vnormlp{\txdfAg[{\erWe[]}]-\xdf}^2% \leq
%     \leq \cst{\xdfCw[],\xdfCr,\Lambda} \naRa{(\xdfCw[],\Lambda)}
% \end{equation}
% Consequently, if $ \vert \log\naRa{(\xdfCw[],\Lambda)} \vert =o(\aDi{\ssY}({\xdfCw[]})\cmSv[\aDi{\ssY}({\xdfCw[]})])$ then the data-driven
% aggregated estimator attains the rate $\naRa{(\xdfCw[],\Lambda)}$. Otherwise,
% the upper bound faces a deterioration of the rate, which we illustrate next.\remEnd
% \end{rmk}
% ....................................................................
% <<Il upper bound 3>>
% ....................................................................
\begin{il}\label{ak:il:ub:np:mm}
  Let us illustrate  \nref{ak:ag:ub:pnp:mm} and
  \nref{ge:ak:ag:ub2:pnp:mm}. Under \nref{ge:ak:ag:ub2:pnp:mm:c} the
  partially data-driven
aggregated estimator attains the rate $\naRa{(\xdfCw[],\Lambda)}$. Otherwise,
the upper bound faces a deterioration of the rate, which  we illustrate considering as in \nref{il:mm} usual
  behaviour  \ref{il:mm:oo}, \ref{il:mm:os} and \ref{il:mm:so}
 for the sequences $\Nsuite[\Di]{\xdfCw[(\Di)]}$ and
 $\Nsuite[\Di]{\iSv[\Di]}$.
 In case \ref{ak:mrb:ass:il:oo},
  \ref{ak:mrb:ass:il:os} and \ref{ak:mrb:ass:il:so} only with
  $p<1/2$ the assumption \ref{ge:ak:ag:ub2:pnp:mm:c} is satisfied, and
  $\naRa{(\xdfCw[],\Lambda)}$ equals the oracle rate $\oRa{\xdfCw[],\Lambda}$ (cf.
  \nref{ak:mrb:ass:il} \ref{ak:mrb:ass:il:oo},
  \ref{ak:mrb:ass:il:os} and \ref{ak:mrb:ass:il:so}). Thereby, the
  partially data-driven aggregation leads to an estimator attaining
  the oracle rate $\oRa{\xdfCw[],\Lambda}$. In case \ref{ak:mrb:ass:il:os}
  with $p\geq1/2$ the assumption \ref{ge:ak:ag:ub2:pnp:mm:c} is not
  satisfied. However, with
  $\sDi{\ssY}:=\Di_{\cst{3}} \vert \log\naRa{(\xdfCw[],\Lambda)} \vert \approx(\log\ssY)$ holds
  $\min_{\Di\in\nset{1,\ssY}}\big[\dRa{\Di}{\xdfCw[],\Lambda}\vee\exp\big(\tfrac{-\cmiSv\Di}{\Di_{\cst{3}}}\big)\leq\daRa{\sDi{\ssY}}{(\xdfCw[],\Lambda)}\approx(\log\ssY)^{2a+1}\ssY^{-1}$.
  In this situation the rate of the partially data-driven estimator
  $\txdfAg[{\erWe[]}]$ features a deterioration by a logarithmic factor
  $(\log\ssY)^{(2a+1)(1-1/(2p))}$ compared to the oracle rate, i.e.
  $\daRa{\sDi{\ssY}}{(\xdfCw[],\Lambda)}\approx(\log\ssY)^{2a+1}\ssY^{-1}$ versus
  $\oRa{\xdfCw[],\Lambda}\approx(\log\ssY)^{(2a+1)/(2p)}\ssY^{-1}$.\ilEnd
\end{il}


\subsection{Unknown operator}\label{freq:ge:strat:uk}\label{FREQ:GE:STRAT:UK}
Consider now the case \nref{AS_INTRO_DATA_UNKNOWN}.
We shall hence keep in mind \ref{freq:ge:shape:uk}, \ref{freq:ge:shape:uk:we}, \nref{freq:ge:shape:uk:de:pen:oo} as well as \ref{freq:ge:shape:uk:de:ms} and \ref{freq:ge:shape:uk:de:msWe}.
\textbf{Note that the detailed proofs for all results given here can be found in \nref{pro:freq:ge:strat:uk}}.

We will assume, from now on, that \nref{as:il} holds true.

Both for the quadratic and the maximal risk, our strategy is based on the decomposition of the quadratic loss function displayed in \nref{freq:ge:strat:kn:co:agg}.
This decomposition is independent of the model an only relies on the fact that the parameter space is equipped with a nested sieve and the fact that our estimator aggregation structure takes advantage of it.

\begin{lm}\label{co:agg:au}\label{freq:ge:strat:kn:co:agg:au:e1}
  Consider the aggregated OSE
  $\widehat{\theta}^{(\eta)}=\sum_{\Di=1}^n\We\hxdfPr$ with weights
  $\We[(\Di)]\in[0,1]$, $\Di\in\nset{1,\ssY}$, satisfying
  $ \sum_{\Di=1}^n\We[(\Di)]=1$ and a sequence
  $(\pen(m))_{\Di\in\nset{1,\ssY}}$ of non-negative compensation terms.
  Given $\Di\in\Nz$ let
  $\dxdfPr:=\sum_{s=-\Di}^{\Di}\hfedfmpI[(s)]\fydf[(s)]$. For any
  $\mDi\in\nset{1,\ssY}$, $\pDi\in\nset{1,\ssY}$, and the sequence of events $(\xEv)_{s \in \Z} = (\{\vert \hfedfmpI[(s)] \vert ^2 \geq \ssE^{-1} \})_{s \in \Z}$ holds
  \begin{multline}\label{co:agg:au:e1}
    \Vnormlp{\widehat{\theta}^{(\eta)}-\xdf}^2\leq 
    3\Vnormlp{\hxdfPr[\pDi]-\dxdfPr[\pDi]}^2
    +3 \Vnormlp{\xdf_{\underline{0}}}^2\bias[\mDi]^2(\xdf)
    \\\hfill
    +3 \Vnormlp{\xdf_{\underline{0}}}^2\We[](\nsetro{1,\mDi})
    +\tfrac{3}{7}\sum_{l\in\nsetlo{\pDi,\ssY}}\pen(l)\We[(l)]
    \Ind{\{\Vnormlp{\hxdfPr[l]-\dxdfPr[l]}^2<\pen(l)\}}\\\hfill
    +3\sum_{l\in\nsetlo{\pDi,\ssY}}\vectp{\Vnormlp{\hxdfPr[l]-\dxdfPr[l]}^2-\pen(l)/7}
    +\tfrac{3}{7}\sum_{l\in\nsetlo{\pDi,\ssY}}\pen(l)
    \Ind{\{\Vnormlp{\hxdfPr[l]-\dxdfPr[l]}^2\geq\pen(l)/7\}}\\
    +6\sum_{s\in\nset{1,\ssY}} \vert \hfedfmpI[(s)] \vert ^2 \vert \fedf[(s)]-\hfedf[(s)] \vert ^2 \vert \fxdf[(s)] \vert ^2
    +2\sum_{s\in\nset{1,\ssY}}\Ind{\xEv^c} \vert \fxdf[(s)] \vert ^2
 \end{multline}
\end{lm}
% ....................................................................
% <<Te weighrts>>
% ....................................................................
\begin{te}
Keep in mind the shape of the estimator given in \ref{freq:ge:shape:uk}, \nref{freq:ge:shape:uk:we}, \nref{freq:ge:shape:uk:de:ms}, \nref{freq:ge:shape:uk:de:msWe}, and \nref{freq:ge:shape:uk:de:pen:oo}.
\end{te}

\subsubsection{Quadratic risk bounds}\label{freq:ge:strat:uk:qu}
\begin{te}
  We derive bounds for the risk of the aggregated estimator $\hxdfAg$
  and the model selected estimator $\theta_{n, \overline{\hDi}}$ by applying
  \nref{co:agg:au}. Therefore, we aim next to control the third and
  fourth right hand side term in \eqref{co:agg:au:e1}.
  The proofs for the results stated here can be found in \nref{pro:freq:ge:strat:uk:qu}.
\end{te}
% --------------------------------------------------------------------
% <<Text Definition {p \vert m}Di>>
% --------------------------------------------------------------------
\begin{te}
For each $\Di\in\Nz$ keep in mind that
$\Vnormlp{\xdf_{\underline{\Di}}}^2=\Vnormlp{\xdf_{\underline{0}}}^2\bias[\Di]^2(\xdf)$,  
  $\daRa{\Di}{(\theta^{\circ},\Lambda)}:=[\b_{m}^{2}(\theta^{\circ})\vee \DipenSv\,\ssY^{-1}]$ as in
  \nref{freq:ge:strat:kn:ma:de:rate} and
introduce in addition
$\dxdfPr=\mathds{1}_{\{\vert s \vert \leq m\}}\hfedfmpI[(s)]\fydf[(s)]$. Note
that  $\dxdfPr=\Proj[\Di]\dxdfPr[\ssY]$
and $\Vnormlp{\ProjC[\Di]\dxdfPr[\ssY]}^2=2\sum_{s\in\nsetlo{\Di,\ssY}}\eiSv[(s)] \vert \fydf[(s)] \vert ^2$. For any $\pdDi,\mdDi\in\nset{1,\ssY}$ let us define 
\begin{multline}\label{au:de:*Di:ag}
\mDi:=\min\set{\Di\in\nset{1,\mdDi}: \Vnormlp{\xdf_{\underline{0}}}^2\bias^2(\xdf)\leq
  [\Vnormlp{\xdf_{\underline{0}}}^2+104\cpen]\daRa{\mdDi}{(\xdf,\Lambda)}}\quad\text{and}\\\pDi:=\max\set{\Di\in\nset{\pdDi,\ssY}:
   \peneSv \leq 2[3\Vnormlp{\ProjC[\pdDi]\dxdfPr[\ssY]}^2+2\peneSv[(\pdDi)]]}
\end{multline}
where  the defining set obviously contains $\mdDi$ and $\pdDi$, respectively, 
and hence, they are
not empty. Keep in mind that $\pDi:=\pDi(\rE_1,\dotsc,\rE_{\ssE})$ is
random but does not depend on the sample $\rY_1,\dotsc,\rY_{\ssY}$.
\end{te}
% ....................................................................
% <<Re Sum Random weights>>
% ....................................................................
\begin{lm}\label{au:re:SrWe:ag}
Consider the data-driven aggregation weights $\erWe[]$ as in \eqref{freq:ge:shape:uk:we}.  Using the penalty as in \nref{freq:ge:shape:uk:de:pen:oo} with $\aixEv[l] := \setB{1/4\leq\iSv[s]^{-1}\eiSv[(s)] \leq9/4,\;\forall\;s\in\nset{1,l}}$, $l\in\nset{1,\ssY}$,
    for any
  $\mdDi,\pdDi\in\nset{1,\ssY}$ and associated $\pDi,\mDi\in\nset{1,\ssY}$
  as in \eqref{au:de:*Di:ag} hold
  \begin{resListeN}
  \item\label{au:re:SrWe:ag:i}$\rWe[](\nsetro{1,\mDi})\\
  \leq \tfrac{50}{\rWc\cpen}\Ind{\{\mDi>1\}} \exp\big(-\tfrac{\rWc\cpen}{2} \ssY\daRa{\mdDi}{(\xdf,\Lambda)}\big) + \Ind{\{\Vnormlp{\hxdfPr[\mdDi]-\dxdfPr[\mdDi]}^2\geq\peneSv[(\mdDi)]/7\}\cup\aixEv[\mdDi]^c}$;
  \item\label{au:re:SrWe:ag:ii}
    $\sum_{\Di\in\nsetlo{\pDi,n}}\peneSv\erWe\Ind{\{\Vnormlp{\hxdfPr-\dxdfPr}^2<\peneSv/7\}}
    \leq \ssY^{-1}\{\tfrac{16}{\cpen\rWc^{2}}+ \tfrac{8}{\rWc}\}$.
  \end{resListeN}
  \reEnd
\end{lm}
% ....................................................................
% <<Te Sum MS Random weights>>
% ....................................................................
\begin{te}
  We combine the upper bound in \eqref{co:agg:au:e1} and the bounds given
  in \nref{au:re:SrWe:ag}. Conditionally on $\rE_1,\dotsc,\rE_{\ssE}$ the r.v.'s
$\rY_1,\dotsc,\rY_{\ssY}$ are \iid and we  denote by $\P_{\rY \vert \rE}$ and $\E_{\rY \vert \rE}$ their conditional
 distribution and expectation, respectively. Clearly, due to \nref{au:re:SrWe:ag} we have
  \begin{multline*}
    \E_{\rY \vert \rE}\erWe[](\nsetro{1,\mDi})\leq\Ind{\{\mDi>1\}}
    \big[\tfrac{50}{\rWc\cpen}\exp\big(-\tfrac{3\rWc\cpen}{14}n\daRa{\mdDi}{(\xdf,\Lambda)}\big)\\+
    \P_{\rY \vert \rE}\big(\Vnormlp{\hxdfPr[\mdDi]-\dxdfPr[\mdDi]}^2
    \geq\peneSv[(\mdDi)]/7\big) \Ind{\aixEv[\mdDi]} + \Ind{\aixEv[\mdDi]^c}\big]
  \end{multline*}
  and, hence from \eqref{co:agg:au:e1} follows immediately
  \begin{multline}\label{co:agg:au:ag}
   \E_{\rY \vert \rE}\Vnormlp{\widehat{\theta}^{(\eta)}-\xdf}^2\leq 3\E_{\rY \vert \rE}\Vnormlp{\hxdfPr[\pDi]-\dxdfPr[\pDi]}^2 + 3 \Vnormlp{\xdf_{\underline{0}}}^2\bias[\mDi]^2(\xdf)\\\hfill
    +\tfrac{150}{\rWc\cpen}\Vnormlp{\xdf_{\underline{0}}}^2\Ind{\{\mDi>1\}} \exp\big(-\tfrac{3\rWc\cpen}{14}n\daRa{\mdDi}{(\xdf,\Lambda)}\big) +\tfrac{3}{7}\ssY^{-1}\{\tfrac{16}{\cpen\rWc^{2}}+ \tfrac{8}{\rWc}\}\\\hfill
    +3 \Vnormlp{\xdf_{\underline{0}}}^2\Ind{\{\mDi>1\}} \big[\P_{\rY \vert \rE}\big(\Vnormlp{\hxdfPr[\mdDi]-\dxdfPr[\mdDi]}^2 \geq\peneSv[(\mdDi)]/7\big) \Ind{\aixEv[\mdDi]} + \Ind{\aixEv[\mdDi]^c}\big]\\\hfill
    +3\sum_{l\in\nsetlo{\pDi,\ssY}}\E_{\rY \vert \rE}\vectp{\Vnormlp{\hxdfPr[l]-\dxdfPr[l]}^2-\pen(l)/7}\\
     + \tfrac{3}{7}\sum_{l\in\nsetlo{\pDi,\ssY}}\peneSv[(l)]\P_{\rY \vert \rE}\big(\Vnormlp{\hxdfPr[l]-\dxdfPr[l]}^2\geq\pen(l)/7\big)\\
    +6\sum_{s\in\nset{1,\ssY}} \vert \hfedfmpI[(s)] \vert ^2 \vert \fedf[(s)]-\hfedf[(s)] \vert ^2 \vert \fxdf[(s)] \vert ^2
    +2\sum_{s\in\nset{1,\ssY}}\Ind{\xEv^c} \vert \fxdf[(s)] \vert ^2
  \end{multline}
\end{te}
% ....................................................................
% Outline Oracle optimality
% ....................................................................
\begin{te}
 The next result can be directly deduced from \nref{au:re:SrWe:ag} by letting
  $\rWc\to\infty$. However, we think the direct proof given in annex 
  provides  an interesting illustration  of the values
  $\pDi,\mDi\in\nset{1,\ssY}$ as defined in \eqref{au:de:*Di:ag}.
\end{te}
% ....................................................................
% <<Upper bound random weights>>
% ....................................................................
\begin{lm}\label{au:re:SrWe:ms}
Consider the data-driven model selection weights $\msWe[]$
  as in \nref{freq:ge:shape:uk:de:msWe}.  Under definition \nref{freq:ge:shape:uk:de:pen:oo} for any $\mdDi,\pdDi\in\nset{1,\ssY}$ and associated
  $\pDi,\mDi\in\nset{1,n}$ as in \eqref{au:de:*Di:ag} hold
  \begin{resListeN}[]
  \item\label{au:re:SrWe:ms:i}
    $\msWe[](\nsetro{1,\mDi})\Ind{\{\Vnormlp{\hxdfPr[\mdDi]-\dxdfPr[\mdDi]}^2<\peneSv[(\mdDi)]/7\}\cap\aixEv[\mdDi]}=0$;
  \item\label{au:re:SrWe:ms:ii}
    $\sum_{\Di\in\nsetlo{\pDi,\ssY}}\peneSv\msWe\Ind{\{\Vnormlp{\hxdfPr-\dxdfPr}^2<\peneSv/7\}}=0$.
  \end{resListeN}
  \reEnd
\end{lm}
% ....................................................................
% <<Te Sum MS Random weights>>
% ....................................................................
\begin{te}
  We combine the upper bound in \eqref{co:agg:au:e1} and the bounds given
  in \nref{au:re:SrWe:ms}.  Clearly, due to \nref{au:re:SrWe:ms} we have
  \begin{equation*}
    \E_{\rY \vert \rE}\msWe[](\nsetro{1,\mDi})\leq\Ind{\{\mDi>1\}}
    \big[ \P_{\rY \vert \rE}\big(\Vnormlp{\hxdfPr[\mdDi]-\dxdfPr[\mdDi]}^2
    \geq\peneSv[(\mdDi)]/7\big) \Ind{\aixEv[\mdDi]} + \Ind{\aixEv[\mdDi]^c}\big]
  \end{equation*}
  and, hence from \eqref{co:agg:au:e1} follows immediately
  \begin{multline}\label{co:agg:au:ms}
   \E_{\rY \vert \rE}\Vnormlp{\widehat{\theta}^{(\eta)}-\xdf}^2\leq 
    3\E_{\rY \vert \rE}\Vnormlp{\hxdfPr[\pDi]-\dxdfPr[\pDi]}^2
    +3 \Vnormlp{\xdf_{\underline{0}}}^2\bias[\mDi]^2(\xdf)\\\hfill
    +3 \Vnormlp{\xdf_{\underline{0}}}^2\Ind{\{\mDi>1\}}
    \big[\P_{\rY \vert \rE}\big(\Vnormlp{\hxdfPr[\mdDi]-\dxdfPr[\mdDi]}^2
    \geq\peneSv[(\mdDi)]/7\big) \Ind{\aixEv[\mdDi]} + \Ind{\aixEv[\mdDi]^c}\big]
    \\\hfill
    +3\sum_{l\in\nsetlo{\pDi,\ssY}}\E_{\rY \vert \rE}\vectp{\Vnormlp{\hxdfPr[l]-\dxdfPr[l]}^2-\pen(l)/7}\\
    +\tfrac{3}{7}\sum_{l\in\nsetlo{\pDi,\ssY}}\peneSv[(l)]\P_{\rY \vert \rE}\big(
    \Vnormlp{\hxdfPr[l]-\dxdfPr[l]}^2\geq\pen(l)/7\big)\\
    +6\sum_{s\in\nset{1,\ssY}} \vert \hfedfmpI[(s)] \vert ^2 \vert \fedf[(s)]-\hfedf[(s)] \vert ^2 \vert \fxdf[(s)] \vert ^2
    +2\sum_{s\in\nset{1,\ssY}}\Ind{\xEv^c} \vert \fxdf[(s)] \vert ^2
  \end{multline}
\end{te}



\begin{te}
The deviations of the last three terms in the last display
\eqref{co:agg:au:ms} and also in \eqref{co:agg:au:ag} need to be bounded using concentration inequalities which depend on the considered model.
We hence formulate it as the central hypothesis to be verified in order to apply this method.
\end{te}

% --------------------------------------------------------------------
% <<Re ND rest>>
% --------------------------------------------------------------------
\begin{as}\label{freq:ge:strat:uk:qu:as}
  Consider
  $\hxdfPr-\dxdfPr=\sum_{|s|\in\nset{1,\Di}}\hfedfmpI[(s)](\hfydf[(s)]-\fydf[(s)])\bas_s$.
  Conditionally on $\{\rE_1,\dotsc,\rE_{\ssE}\}$ the r.v.'s
  $\{\rY_1,\dotsc,\rY_{\ssY}\}$ are \iid and we denote by
  $\P_{Y\vert \epsilon}$ and $\E_{\rY\vert\rE}$ their conditional
  distribution and expectation, respectively.  Let
  $\eiSv[(s)]=|\hfedfmpI[(s)]|^2$,
  $\oeiSv=\tfrac{1}{\Di}\sum_{s\in\nset{1,\Di}}\eiSv[(s)]$,
  $\meiSv= \max\{\eiSv[(s)],s\in\nset{1,\Di}\}$,
  $\DiepenSv=\cmeiSv\Di \meiSv$ and $\cmeiSv\geq1$.  Then there is a
  numerical constant $\cst{}$ such that for all $\ssY\in\Nz$ and
  $\Di\in\nset{1,\ssY}$ holds
  \begin{resListeN}[]
  \item\label{freq:ge:strat:uk:qu:as:i}
    $\E_{\rY\vert\rE} \vectp{\Vnormlp{\hxdfPr-\dxdfPr}^2 - 12\DiepenSv\ssY^{-1}}   \\
    \quad\quad\leq\cst{1} \bigg[\tfrac{\cst{2}\,\meiSv}{\ssY}
    \exp\big(-\cst{3}\cmeiSv\Di\big)
    +\tfrac{\cst{4}\Di\meiSv}{n^2}\exp\big(-\cst{5}\sqrt{\ssY\cmeiSv}\big) \bigg]$
  \item\label{freq:ge:strat:uk:qu:as:ii}
$\P_{Y\vert \epsilon}\big(\Vnormlp{\hxdfPr-\dxdfPr}^2 \geq 12\DiepenSv\ssY^{-1}\big)\\
\leq 
\cst{6} \bigg[\exp\big(-\cst{7}\cmeiSv\Di\big)
+\exp\big(-\cst{8}\sqrt{\ssY\cmeiSv}\big)\bigg]$
   \item\label{freq:ge:strat:uk:qu:as:iii}
     $\P_{Y\vert \epsilon}\big(\Vnormlp{\hxdfPr-\dxdfPr}^2 \geq 12\DiepenSv\ssY^{-1}\big)\\
     \leq 
     \cst{9} \bigg[\exp\big(-\cst{10}\cmeiSv\Di\big)
     +\exp\big(\tfrac{-\cst{11}\ssY\sqrt{\daRa{\Di}{\xdf,\eiSv}}}{\sqrt{\Di\meiSv}}\big)\bigg]$
   \item\label{freq:ge:strat:uk:qu:as:iv}
   $\P\big(|\hfedf[(s)]/\fedf[(s)]-1|>1/3\big)\leq \cst{12}\exp\big(-\cst{13}\ssE|\fedf[(s)]|^2\big)\leq \cst{14}\exp\big(-\tfrac{\cst{15}\ssE}{\miSv}\big).$
\end{resListeN}
\assEnd
\end{as}

%\begin{lm}\label{re:evrest}
%Assume that \nref{freq:ge:strat:uk:qu:as} holds true.
%For $\Di\in\Nz$ consider
%  $\aixEv[\Di]:=\{1/2\leq|\fedf[(s)]\hfedfmpI[(s)]|\leq3/2:\;\forall\;s\in\nset{1,\Di}\}$. 
%\begin{resListeN}[]
%\item\label{re:evrest:i}
%For all $\Di,\ssE\in\Nz$ with  $\miSv[k]\leq
%(4/9)\ssE$ holds $\P(\aixEv^c)\leq 2\Di\exp\big(-\tfrac{\ssE}{72\miSv}\big)$.
%\item\label{re:evrest:ii}
%Given  $\Di\in\Nz$ let $\ssE(\Di):=\ceil{9\miSv/4}$  then 
%$\P(\aixEv^c)\leq(555\Di\ssE(k)^2\ssE^{-2})\wedge(12\Di\ssE(\Di)\ssE^{-1})$
%holds true for all
%$\ssE\in\Nz$.
%\item\label{re:evrest:iii}
%For all $\Di,\ssE\in\Nz$ with $\ssE\geq289\log(\Di+2)\cmiSv\miSv$ holds $\P(\aixEv^c)\leq(11226\ssE^{-2})\wedge(53\ssE^{-1})$.
%\end{resListeN}
%\end{lm}
This hypothesis allows us to control the remaining random elements in our bound.

\begin{lm}\label{au:re:nd:rest}
Consider $\peneSv=\peneD$,
  $\Di\in\nset{1,n}$, as in \nref{freq:ge:shape:uk:de:pen:oo} with $\cpen\geq84$.
Let $\Di_{\cst{3}}:=[\floor{3(\tfrac{2}{\cst{3}})^2}\vee \cst{2}]$ and $\ssY_{\cst{5}}:=15(\tfrac{1}{\cst{5}})^4$; as well as $\Di_{\cst{7}}:=\floor{3(\tfrac{2}{\cst{7}})^2}$ and
    $\ssY_{\cst{8}}:=\floor{15({3/\cst{8}})^4}$.
    There exists a finite numerical constant  $\cst{}>0$ such that for all $n\in\Nz$ and all $\mdDi\in\nset{1,\ssY}$  hold
\begin{resListeN}
\item\label{au:re:nd:rest1}
$\sum_{\Di=1}^{\ssY}\E_{\rY \vert \rE}\vectp{\Vnormlp{\hxdfPr-\dxdfPr}^2-\peneSv/7}\\
\leq\cst{}\ssY^{-1}\big[(1\vee\meiSv[\Di_{\cst{3}}])\Di_{\cst{3}}+(1\vee\meiSv[\ssY_{\cst{5}}]\ssY_{\cst{5}})\big]$;
\item\label{au:re:nd:rest2}
  $\sum_{\Di=1}^{\ssY}\peneSv\P_{\rY \vert \rE}\big(\Vnormlp{\hxdfPr-\dxdfPr}^2\geq\peneSv/7\big)\leq\cst{}\ssY^{-1}\big[(1\vee\meiSv[\Di_{\cst{7}}]^2)\Di_{\cst{7}}^2+(1\vee\meiSv[\ssY_{\cst{8}}]^2\ssY_{\cst{8}}^2)\big]$;
\item\label{au:re:nd:rest3}
  $\P_{\rY \vert \rE}\big(\Vnormlp{\hxdfPr[\mdDi]-\dxdfPr[\mdDi]}^2\geq\peneSv[(\mdDi)]/7\big)\leq    \cst{}  \big[\exp\big(-\cst{11}\cmeiSv[\mdDi]\mdDi\big)
    +\ssY^{-1}\big]$.
\end{resListeN}
\reEnd
\end{lm}
% --------------------------------------------------------------------
% <<Te penalty>>
% --------------------------------------------------------------------
\begin{te}Consider now the fully data-driven aggregation of the
  orthogonal series estimators using either  aggregation weights $\erWe[]$
  as in \eqref{freq:ge:shape:uk:we} or model selection weights $\widehat{\P}_{M}^{(\infty)}$ as in \nref{freq:ge:shape:uk:de:msWe}
  combining \nref{freq:ge:strat:uk:qu:as} and the upper bound given
  in \eqref{co:agg:au:ag}  or \eqref{co:agg:au:ms} we obtain the next result. 
\end{te}
% ....................................................................
% <<Re upper bound ag>>
% ....................................................................
\begin{lm}\label{au:ag:ub}
Let \nref{freq:ge:strat:uk:qu:as} hold true.
Consider the penalty sequence $\peneSv:=\peneD$, $\Di\in\nset{1,n}$, as in \nref{freq:ge:shape:uk:de:pen:oo}.
  Let $\widehat{\theta}^{(\eta)}=\sum_{\Di=1}^{\ssY} \erWe\hxdfPr$ be an aggregation of the orthogonal series estimators using either aggregation weights $\erWe[]$ as in \eqref{freq:ge:shape:uk:we} or model selection weights $\widehat{P}_{M}^{(\infty)}$ as in \nref{freq:ge:shape:uk:de:msWe}.
  Then, there is a finite numerical constant $\cst{}>0$ such that for all $\ssY,\ssE\in\Nz$, for any $\mdDi,\pdDi\in\nset{1,n}$ and associated $\mDi\in\nset{1,n}$
  as defined in \eqref{au:de:*Di:ag} hold
\begin{multline}\label{au:ag:ub:e1}
  \E\Vnormlp{\widehat{\theta}^{(\eta)}-\xdf}^2 \leq 2\penSv[\pdDi] +\tfrac{12}{7}\Vnormlp{\xdf_{\underline{0}}}^2\bias[\pdDi]^2(\xdf)+3 \Vnormlp{\xdf_{\underline{0}}}^2\bias[\mDi]^2(\xdf)\\\hfill
  + \cst{}\big[
    \Vnormlp{\xdf_{\underline{0}}}^2\Ind{\{\mDi>1\}} \FuVg[\ssE]{\rE}(\aixEv[\mdDi]^c)+\ssE\FuVg[\ssE]{\rE}(\aixEv[\pdDi]^c) \big] + \cst{}\mRa{\xdf,\Lambda}
  \end{multline}
  \reEnd
\end{lm}
% ....................................................................
% <<Te upper bound ag p np>>
% ....................................................................
\begin{te} The last bound allows us to derive an upper bound of the
  risk for the fully data-driven aggregated estimator in the two cases
  \ref{oo:xdf:p} and \ref{oo:xdf:np} introduced in \nref{bm:ak}.
\end{te}
% ....................................................................
% <<Re upper bound ag p np>>
% ....................................................................
\begin{thm}\label{au:ag:ub:pnp}
Let \nref{freq:ge:strat:uk:qu:as} hold true.
Consider the   penalty sequence $\peneSv:=\peneD$,
  $\Di\in\nset{1,\ssY}$, as in \nref{freq:ge:shape:uk:de:pen:oo}.
  Let $\hxdfAg[{\erWe[]}]=\sum_{\Di=1}^{\ssY} \erWe\hxdfPr$ be an aggregation of the orthogonal series estimators using either aggregation weights $\erWe[]$ as in \eqref{freq:ge:shape:uk:we} or model selection weights $\msWe[]$ as in \nref{freq:ge:shape:uk:de:msWe}.
  \begin{Liste}[]
  \item[\mylabel{au:ag:ub:pnp:p}{\dgrau\bfseries{(p)}}]Assume there is
    $K\in\Nz_0$ with $1\geq \bias[{[K-1] }](\xdf)>0$ and
    $\bias[\Di](\xdf)=0$. For $K>0$ let
    $c_{\xdf}:=\tfrac{\Vnormlp{\xdf_{\underline{0}}}^2+104\cpen}{\Vnormlp{\xdf_{\underline{0}}}^2\bias[{[K-1]}]^2(\xdf)}>1$,
    $\ssY_{\xdf,\Lambda}:=\floor{c_{\xdf}\DipenSv[K]}\in\Nz$ and
    $\ssE(\xdf,\Lambda):=\floor{289\log(K+2)\cmiSv[K]\miSv[K]}\in\Nz$. If
    $\ssY>\ssY_{\xdf,\Lambda}$ and $\ssE>\ssE(\xdf,\Lambda)$ then set
    $\sDi{\ssY}:=\max\{\Di\in\nset{K,\ssY}:\ssY>c_{\xdf}\DipenSv\}$
    and
    $\sDi{\ssE}:=\max\{\Di\in\nset{K,\ssE}:289\log(\Di+2)\cmiSv[\Di]\miSv[\Di]\leq\ssE\}$
    where the defining set, respectively, contains $K$ and thus is not
    empty, and otherwise $\sDi{\ssY}\wedge\sDi{\ssE}:=\Di_{\cst{3}}\log(\ssY\wedge\ssE)$.
    There is a numerical constant $\cst{}$ and a  constant $\cst{\xdf,\Lambda}$ given in
    \eqref{au:ag:ub:pnp:p9} depending only on $\xdf$ and $\Lambda$ such
    that for all $\ssY,\ssE\in\Nz$ holds
    \begin{equation}\label{au:ag:ub:pnp:e1}
       \nmRi{\hxdfAg[{\erWe[]}]}{\xdf,\Lambda}
       %\E\Vnormlp{\widehat{\theta}^{(\eta)}-\xdf}^2
       \leq
      \cst{}\Vnormlp{\xdf_{\underline{0}}}^2\big[\ssY^{-1}\vee \ssE^{-1} \vee
      \exp\big(\tfrac{-\cmiSv[\sDi{\ssY}\wedge\sDi{\ssE}]\sDi{\ssY}\wedge\sDi{\ssE}}{\Di_{\cst{3}}}\big)\big]\\
      + \cst{\xdf,\Lambda}\{\ssY^{-1}\vee\ssE^{-1}\}.
    \end{equation}
  \item[\mylabel{au:ag:ub:pnp:np}{\dgrau\bfseries{(np)}}] Assume that
    $\bias(\xdf)>0$ for all $\Di\in\Nz$. Let    
    $\ssE(\Lambda):=\floor{289\log(3)\cmiSv[1]\miSv[1]}\in\Nz$. If
    $\ssE>\ssE(\Lambda)$ then set
    $\sDi{\ssE}:=\max\{\Di\in\nset{1,\ssE}:289\log(\Di+2)\cmiSv[\Di]\miSv[\Di]\leq\ssE\}$
    where the defining set, respectively, contains $1$ and thus is not
    empty.  There is a numerical constant $\cst{}$  such that for all $\ssY\in\Nz$
    with $\aDi{\ssY}:=\aDi{\ssY}(\xdf)\in\nset{1,n}$ as in \nref{freq:ge:strat:kn:ma:de:rate} and
    for all $\ssE>\ssE(\Lambda)$ holds
    \begin{multline}\label{au:ag:ub:pnp:e2}
     \nmRi{\hxdfAg[{\erWe[]}]}{\xdf,\Lambda}  
     %\E\Vnormlp{\widehat{\theta}^{(\eta)}-\xdf}^2
     \leq\cst{}(1\vee\Vnormlp{\xdf_{\underline{0}}}^2)\min_{\Di\in\nset{1,\ssY}}\{\daRa{\Di}{(\xdf,\Lambda)}\vee\exp\big(\tfrac{-\cmiSv[\Di]\Di}{\Di_{\cst{3}}}\big)\}\Ind{\{\ssE>\ssE(\Lambda)\}}\\\hfill
+\cst{}(1\vee\Vnormlp{\xdf_{\underline{0}}}^2)\{\bias[\aDi{\ssY}\wedge\sDi{\ssE}]^2(\xdf)\vee\exp\big(\tfrac{-\cmiSv[\sDi{\ssE}]\sDi{\ssE}}{\Di_{\cst{3}}}\big)\}\Ind{\{\ssE>\ssE(\Lambda)\}} \\\hfill
 +\cst{}\mRa{\xdf,\Lambda}   + \cst{}(1\vee\Vnormlp{\xdf_{\underline{0}}}^2)\miSv[1]^2\ssE^{-1}  
    +\cst{}\{\miSv[\Di_{\cst{3}}]^2\Di_{\cst{3}}^3+\miSv[\ssY_{o}]^2\}\ssY^{-1}
  \end{multline}
  while for $\ssE\in\nset{1,\ssE(\Lambda)}$ we have
  
  $\cst{}\mRa{\xdf,\Lambda}
    + \cst{}(1\vee\Vnormlp{\xdf_{\underline{0}}}^2)\miSv[1]^2\ssE^{-1}  
    +\cst{}\{\miSv[\Di_{\cst{3}}]^2\Di_{\cst{3}}^3+\miSv[\ssY_{o}]^2\}\ssY^{-1}$. 
\end{Liste}  
\reEnd
\end{thm}

\begin{cor}\label{ge:au:ag:ub2:pnp}
  Let the assumptions of \nref{au:ag:ub:pnp} be satisfied.
  \begin{Liste}[]
  \item[\mylabel{ge:au:ag:ub2:pnp:p}{\dgrau\bfseries{(p)}}]
    If \ref{ge:ak:ag:ub2:pnp:pc} as in \nref{ge:ak:ag:ub2:pnp} and 
    in addition
    \begin{inparaenum}% \item[\mylabel{ge:au:ag:ub2:pnp:pc:a}{{\dgrau\bfseries(A1)}}]
      % there is $\ssY_{\xdf,\Lambda}\in\Nz$ such that
      % $\cmiSv[\sDi{\ssY}]\sDi{\ssY}\geq \Di_{\cst{3}}(\log\ssY)$ for all
      % $\ssY\geq \ssY_{\xdf,\Lambda}$ and
    \item[\mylabel{ge:au:ag:ub2:pnp:pc:b}{{\dgrau\bfseries(A4)}}]
            there is $\ssE(\xdf,\Lambda)\in\Nz$ such that
      $\cmiSv[\sDi{\ssE}]\sDi{\ssE}\geq \Di_{\cst{3}}(\log\ssE)$ for all
      $\ssE\geq \ssE(\xdf,\Lambda)$ 
    \end{inparaenum}
    hold true, then there is a constant $\cst{\xdf,\Lambda}$ depending
    only on $\xdf$ and $\Lambda$ such that for all $\ssY,\ssE\in\Nz$ holds
    $\nmRi{\hxdfAg[{\erWe[]}]}{\xdf,\Lambda} \leq
    \cst{\xdf,\Lambda}[\ssY^{-1}\vee\ssE^{-1}]$.
  \item[\mylabel{ge:au:ag:ub2:pnp:np}{\dgrau\bfseries{(np)}}]
    If  \ref{ge:ak:ag:ub2:pnp:npc} as in \nref{ge:ak:ag:ub2:pnp} and \ref{ge:au:ag:ub2:pnp:pc:b}
    hold true, then there is a constant $\cst{\xdf,\Lambda}$ depending
    only on $\xdf$ and $\Lambda$ such that $\nmRi{\hxdfAg[{\erWe[]}]}{\xdf,\Lambda}
    \leq \cst{\xdf,\Lambda}\{\naRa{(\xdf,\Lambda)}+\mRa{\xdf,\Lambda}+\bias[\sDi{\ssE}\wedge\aDi{\ssY}]^2(\xdf)\}$ for all $\ssY,\ssE\in\Nz$ holds true.
  \end{Liste}  
\end{cor}
% ....................................................................
% <<Rem upper bound ag p np>>
% ....................................................................
\begin{il}\label{au:ag:ub:pnp:il}
  Let us briefly illustrate the last results. In case
  \ref{au:ag:ub:pnp:p} the fully data-driven aggregation leads to an
  estimator attaining the parametric oracle rate (see
  \nref{oo:rem:ora}), if the additional assumptions
  \ref{ge:ak:ag:ub2:pnp:pc} and \ref{ge:au:ag:ub2:pnp:pc:b} are satisfied.
  Consider the two cases \ref{il:edf:o} and \ref{il:edf:s} for the
  operator Fourier sequence $\edf$ as in \nref{il:oo}, where in both cases 
 \ref{ge:ak:ag:ub2:pnp:pc} holds true (cf. \nref{ak:ag:ub:pnp:il}
 \ref{ak:il:edf:o} and \ref{ak:il:edf:s}), while
  \begin{Liste}[]
  \item[\mylabel{au:il:edf:o}{\dg\bfseries{(o)}}]
    $\ssE\sim(\log\sDi{\ssE})\cmSv[\sDi{\ssY}]\miSv[\sDi{\ssE}]
    \sim(\log\sDi{\ssE})(\sDi{\ssE})^{2a}$
    implies $\sDi{\ssE}\sim(\ssE/\log\ssE)^{1/(2a)}$ and
    $\sDi{\ssE}\cmSv[\sDi{\ssE}]\sim (\ssE/\log\ssE)^{1/(2a)}$.
  \item[\mylabel{au:il:edf:s}{\dg\bfseries{(s)}}]
    $\ssE\sim(\log\sDi{\ssE})\cmSv[\sDi{\ssE}]\miSv[\sDi{\ssE}]\sim
    (\log\sDi{\ssE})(\sDi{\ssE})^{4a}\exp((\sDi{\ssE})^{2a})$ implies
    $\sDi{\ssE}\sim(\log
    \ssE-\tfrac{1+4a}{2a}\log\log\ssE-\tfrac{1}{2a}\log\log\log\ssE)^{1/(2a)}$
    and $\sDi{\ssE}\cmSv[\sDi{\ssE}]\sim (\log \ssE)^{2+1/(2a)}$.
  \end{Liste}
  Clearly in both cases \ref{au:il:edf:o} and \ref{au:il:edf:s} also
  \ref{ge:au:ag:ub2:pnp:pc:b} is satisfied. Therefore, in this situation
  the fully data-driven aggregated estimator attains the parametric
  oracle rate. On the other hand side, in case \ref{ge:au:ag:ub2:pnp:np}
  the fully data-driven aggregation leads to an estimator attaining
  the rate $\naRa{(\xdf,\Lambda)}+\mRa{\xdf,\Lambda}$ (\nref{ge:au:ag:ub2:pnp}),
  if \ref{ge:ak:ag:ub2:pnp:npc} and
  \ref{ge:au:ag:ub2:pnp:pc:b} are satisfied and
  $\bias[\sDi{\ssE}\wedge\aDi{\ssY}]^2(\xdf)$ is negligible with
  respect to $\naRa{(\xdf,\Lambda)}+\mRa{\xdf,\Lambda}$, otherwise the upper
  bound faces a deterioration of the rate, which we illustrate
  considering as in \nref{freq:ge:strat:kn:qu:il:rate} the usual behaviour
  \ref{freq:ge:strat:kn:qu:il:rate:np:oo}, \ref{freq:ge:strat:kn:qu:il:rate:np:os} and
  \ref{freq:ge:strat:kn:qu:il:rate:np:so} for the sequences
  $\Nsuite[\Di]{\bias[\Di](\xdf)}$ and $\Nsuite[\Di]{\iSv[\Di]}$.
  In all three cases \ref{freq:ge:strat:kn:qu:il:rate:np:oo}, \ref{freq:ge:strat:kn:qu:il:rate:np:os} and
  \ref{freq:ge:strat:kn:qu:il:rate:np:so} the assumption \ref{ge:au:ag:ub2:pnp:pc:b}
  holds true. Moreover, in case \ref{freq:ge:strat:kn:qu:il:rate:np:oo},
  \ref{freq:ge:strat:kn:qu:il:rate:np:os} and \ref{freq:ge:strat:kn:qu:il:rate:np:so} only with
  $p<1/2$ the assumption \ref{ge:ak:ag:ub2:pnp:npc} is satisfied, and
  $\naRa{(\xdf,\Lambda)}$ equals the oracle rate $\oRa{\xdf,\Lambda}$ (cf.
  \nref{freq:ge:strat:kn:qu:il:rate:np:oo} \ref{freq:ge:strat:kn:qu:il:rate:np:oo},
  \ref{freq:ge:strat:kn:qu:il:rate:np:os} and \ref{freq:ge:strat:kn:qu:il:rate:np:so}). In case
  \ref{freq:ge:strat:kn:qu:il:rate:np:os} and \ref{freq:ge:strat:kn:qu:il:rate:np:so}
  $\bias[\sDi{\ssE}]^2(\xdf)\leq \cst{\xdf,\Lambda} \mRa{\xdf,\Lambda}$ while
  in case \ref{freq:ge:strat:kn:qu:il:rate:np:oo}
  $\bias[\sDi{\ssE}]^2(\xdf)\sim(\ssE/\log\ssE)^{-p/a}$, hence 
  \begin{Liste}[]
  \item[\mylabel{au:ag:ub:pnp:il:oo}{\dg\bfseries{[o-o]}}]
    $\nmRi{\hxdfAg[{\erWe[]}]}{\xdf,\Lambda}
    \leq
    \cst{\xdf,\Lambda}\{\ssY^{-2p/(2p+2a+1)}+\ssE^{-(p\wedge a)/a}+
    (\ssE/\log\ssE)^{-p/a}\}$
      \item[\mylabel{au:ag:ub:pnp:il:os}{\dg\bfseries{[o-s]}}]
    $\nmRi{\hxdfAg[{\erWe[]}]}{\xdf,\Lambda}
    \leq
    \cst{\xdf,\Lambda}\{(\log \ssY)^{-p/a}+(\log \ssE)^{-p/a}\}$
  \item[\mylabel{au:ag:ub:pnp:il:so}{\dg\bfseries{[s-o]}}]
    $\nmRi{\hxdfAg[{\erWe[]}]}{\xdf,\Lambda}
    \leq
    \cst{\xdf,\Lambda}\{(\log\ssY)^{(2a+1)/(2p\wedge1)}\ssY^{-1}+\ssE^{-1}\}$
  \end{Liste}
  Consequently, the fully data-driven estimator attains the oracle
  rate in case \ref{freq:ge:strat:kn:qu:il:rate:np:oo} with $p>a$,
  \ref{freq:ge:strat:kn:qu:il:rate:np:os} and \ref{freq:ge:strat:kn:qu:il:rate:np:so} with
  $p\leq1/2$, while in  case \ref{freq:ge:strat:kn:qu:il:rate:np:oo} with $p\leq a$ and
  \ref{freq:ge:strat:kn:qu:il:rate:np:so} with $p>1/2$ the rate of the fully data-driven estimator
  $\hxdfAg[{\erWe[]}]$ features a deterioration by a logarithmic factor
  $(\log\ssE)^{p/a}$  and $(\log\ssY)^{(2a+1)(1-1/(2p))}$, respectively, compared to the oracle rate.\ilEnd
\end{il}


\subsubsection{Maximal risk bounds}\label{freq:ge:strat:uk:ma}
% --------------------------------------------------------------------
% <<Text Definition AG {p \vert m}Di>>
% --------------------------------------------------------------------
\begin{te}
    By applying \nref{co:agg:au} we derive bounds for the maximal risk defined in \eqref{oo:e4} over ellipsoids  $\rwCxdf$ of the fully data-driven aggregated estimator $\hxdfAg[{\erWe[]}]$ using either aggregation weights $\erWe[]$
  as in \eqref{freq:ge:shape:uk:we} or model selection weights $\widehat{\P}_{M}^{(\eta)}$ as in \nref{freq:ge:shape:uk:de:msWe}.
  Therefore, we aim next to control the second and third right hand side term in \eqref{co:agg:au:e1} uniformly over $\rwCxdf$.
  Results stated here are proven in \nref{pro:freq:ge:strat:uk:ma}.
\end{te}
% --------------------------------------------------------------------
% <<Text Definition {p \vert m}Di>>
% --------------------------------------------------------------------
\begin{te}
  For each $\Di\in\Nz$ keeping 
the definition \ref{freq:ge:strat:kn:ma:de:rate} of
  $\daRa{\Di}{(\xdfCw[],\Lambda)}:=[\xdfCw\vee \DipenSv\,\ssY^{-1}]$ in
  mind it holds
$\xdfCr^2\daRa{\Di}{(\xdfCw[],\Lambda)}\geq\Vnormlp{\xdf_{\underline{0}}}^2\bias^2(\xdf)$
uniformely for all $\xdf\in\rwCxdf$ and for all
$\Di\in\Nz$.  Introduce in addition
$\dxdfPr=\sum_{s\in\nset{-\Di,\Di}}\hfedfmpI[(s)]\fydf[(s)]$. Note
that  $\dxdfPr=\Proj[\Di]\dxdfPr[\ssY]$
and $\Vnormlp{\ProjC[\Di]\dxdfPr[\ssY]}^2=2\sum_{s\in\nsetlo{\Di,\ssY}}\eiSv[(s)] \vert \fydf[(s)] \vert ^2$. For any $\pdDi,\mdDi\in\nset{1,\ssY}$ let us define 
\begin{multline}\label{au:mrb:de:*Di}
\mDi:=\min\set{\Di\in\nset{1,\mdDi}: \Vnormlp{\xdf_{\underline{0}}}^2\bias^2(\xdf)\leq
  [\xdfCr^2+104\cpen]\daRa{\mdDi}{(\xdfCw[],\Lambda)}}\quad\text{and}\\\pDi:=\max\set{\Di\in\nset{\pdDi,\ssY}:
   \peneSv \leq 2[3\Vnormlp{\ProjC[\pdDi]\dxdfPr[\ssY]}^2+2\peneSv[(\pdDi)]]}
\end{multline}
where  the defining set obviously contains $\mdDi$ and $\pdDi$, respectively, 
and hence, they are
not empty. Keep in mind that $\pDi:=\pDi(\rE_1,\dotsc,\rE_{\ssE})$ is
random but does not depend on the sample $\rY_1,\dotsc,\rY_{\ssY}$.
\end{te}
% ....................................................................
% <<Re Sum Random weights>>
% ....................................................................
\begin{lm}\label{au:mrb:re:SrWe:ag}
Consider the data-driven aggregation weights $\erWe[]$ as in \eqref{freq:ge:shape:uk:we}.
Using the aggregation weights as in \nref{freq:ge:shape:uk:de:pen:oo} with
  $\cpen\geq8\log(3e)$ and
  
    $\aixEv[l]:=\setB{1/4\leq\iSv[s]^{-1}\eiSv[(s)]\leq9/4,\;\forall\;s\in\nset{1,l}}$, $l\in\nset{1,\ssY}$,
    for any
  $\mdDi,\pdDi\in\nset{1,\ssY}$ and associated $\pDi,\mDi\in\nset{1,\ssY}$
  as in \eqref{au:mrb:de:*Di} hold
  \begin{resListeN}
  \item\label{au:mrb:re:SrWe:ag:i}
    $\rWe[](\nsetro{1,\mDi})\leq \tfrac{50}{\rWc\cpen}\Ind{\{\mDi>1\}}
    \exp\big(-\tfrac{\rWc\cpen}{2} \ssY\daRa{\mdDi}{(\xdfCw[],\Lambda)}\big)\\
    \quad+\Ind{\{\Vnormlp{\hxdfPr[\mdDi]-\dxdfPr[\mdDi]}^2\geq\peneSv[(\mdDi)]/7\}\cup\aixEv[\mdDi]^c}$;
  \item\label{au:mrb:re:SrWe:ag:ii}
    $\sum_{\Di\in\nsetlo{\pDi,n}}\peneSv\erWe\Ind{\{\Vnormlp{\hxdfPr-\dxdfPr}^2<\peneSv/7\}}
    \leq \ssY^{-1}\{\tfrac{16}{\cpen\rWc^{2}}+ \tfrac{8}{\rWc}\}$.
  \end{resListeN}
\end{lm}
% --------------------------------------------------------------------
% <<Text unifom bounds over ellipsoid>>
% --------------------------------------------------------------------
\begin{te}Keeping in mind that $\ydf=\xdf\cdot\edf$. We note that
  uniformly for all $\xdf\in\rwCxdf$ by applying the
  Cauchy-Schwarz inequality holds   $\Vnormlp[1]{\fydf}\leq
\Vnorm[{\xdfCw[]}]{\edf}\Vnorm[1/{\xdfCw[]}]{\xdf}\leq
\Vnorm[{\xdfCw[]}]{\edf}\xdfCr$. Thereby, we obtain the next assertion
immediately from \nref{freq:ge:strat:uk:qu:as}, and we omit its elementary proof.
\end{te}
% --------------------------------------------------------------------
% <<Re ND rest>>
% --------------------------------------------------------------------
\begin{lm}\label{freq:ge:strat:uk:ma:as}
Assume that \nref{freq:ge:strat:uk:qu:as} holds true.
Then, we have
\begin{resListeN}
\item\label{freq:ge:strat:uk:ma:as1}
$\sup_{\xdf\in\rwCxdf}\E\sum\limits_{\Di=1}^{\ssY}\E_{\rY \vert \rE}\vectp{\Vnormlp{\hxdfPr-\dxdfPr}^2-\tfrac{1}{7}\peneSv} \in \naRa{(\xdfCw[],\Lambda)}$;
\item\label{freq:ge:strat:uk:ma:as2}
  $\sup\limits_{\xdf\in\rwCxdf}\E\sum\limits_{\Di=1}^{\ssY}\peneSv\P_{\rY \vert \rE}\big(\Vnormlp{\hxdfPr-\dxdfPr}^2\geq\tfrac{1}{7}\peneSv\big) \in \naRa{(\xdfCw[],\Lambda)}$;
\item\label{freq:ge:strat:uk:ma:as3}
  $\sup\limits_{\xdf\in\rwCxdf}\E\P_{\rY \vert \rE}\big(\Vnormlp{\hxdfPr[\mdDi]-\dxdfPr[\mdDi]}^2\geq\tfrac{1}{7}\peneSv[(\mdDi)]\big) \in \naRa{(\xdfCw[],\Lambda)}$.
\end{resListeN}
%\begin{resListeN}
%\item\label{freq:ge:strat:uk:ma:as1}
%$\sup_{\xdf\in\rwCxdf}\E\sum\limits_{\Di=1}^{\ssY}\E_{\rY \vert \rE}\vectp{\Vnormlp{\hxdfPr-\dxdfPr}^2-\tfrac{1}{7}\peneSv}\leq
%\cst{}\ssY^{-1}\big[(1\vee\meiSv[\Di_{\edf,\xdfCr}])\Di_{\edf,\xdfCr}+(1\vee\meiSv[\ssY_{o}]\ssY_{o})\big]$;
%\item\label{freq:ge:strat:uk:ma:as2}
%  $\sup\limits_{\xdf\in\rwCxdf}\E\sum\limits_{\Di=1}^{\ssY}\peneSv\P_{\rY \vert \rE}\big(\Vnormlp{\hxdfPr-\dxdfPr}^2\geq\tfrac{1}{7}\peneSv\big)\leq\cst{}\ssY^{-1}\big[(1\vee\meiSv[\Di_{\edf,\xdfCr}]^2)\Di_{\edf,\xdfCr}^2+(1\vee\meiSv[\ssY_{o}]^2\ssY_{o}^2)\big]$;
%\item\label{freq:ge:strat:uk:ma:as3}
%  $\sup\limits_{\xdf\in\rwCxdf}\E\P_{\rY \vert \rE}\big(\Vnormlp{\hxdfPr[\mdDi]-\dxdfPr[\mdDi]}^2\geq\tfrac{1}{7}\peneSv[(\mdDi)]\big)\leq    \cst{}  \big[\exp\big(\tfrac{-\cmeiSv[\mdDi]\mdDi}{200\Vnorm[{\xdfCw[]}]{\edf}\xdfCr}\big)
%    +\ssY^{-1}\big]$.
%\end{resListeN}  
\reEnd
\end{lm}
% --------------------------------------------------------------------
% <<Te penalty>>
% --------------------------------------------------------------------
\begin{te}Consider now the fully data-driven aggregation of the
  orthogonal series estimators using either  aggregation weights $\erWe[]$
  as in \eqref{freq:ge:shape:uk:we} or model selection weights $\widehat{\P}_{M}^{(\infty)}$ as in \nref{freq:ge:shape:uk:de:msWe}
  combining \nref{freq:ge:strat:uk:ma:as} and the upper bound given
  in \eqref{co:agg:au:ag}  or \eqref{co:agg:au:ms} we obtain the next result. 
\end{te}
% ....................................................................
% <<Re upper bound ag>>
% ....................................................................
\begin{lm}\label{au:mrb:ag:ub}
Assume that \nref{freq:ge:strat:uk:qu:as} holds true and consider the penalty sequence $\peneSv:=\peneD$, $\Di\in\nset{1,n}$, as in \nref{freq:ge:shape:uk:de:pen:oo}.
  Let$\widehat{\theta}^{(\eta)}=\sum_{\Di=1}^{\ssY} \erWe\hxdfPr$ be an aggregation of the orthogonal series estimators using either aggregation weights $\erWe[]$ as in \eqref{freq:ge:shape:uk:we} or model selection weights $\msWe[]$ as in \nref{freq:ge:shape:uk:de:msWe}.
  There is a finite numerical constant $\cst{}>0$ such that for all $\ssY,\ssE\in\Nz$, for any $\xdf\in\rwCxdf$, any $\mdDi,\pdDi\in\nset{1,n}$ and associated $\mDi\in\nset{1,n}$ as  defined in \eqref{au:de:*Di:ag} hold
    \begin{multline}\label{au:mrb:ag:ub:e1}
  \E\Vnormlp{\widehat{\theta}^{(\eta)}-\xdf}^2\leq
  2\penSv[\pdDi] +\tfrac{12}{7}\Vnormlp{\xdf_{\underline{0}}}^2\bias[\pdDi]^2(\xdf)+3 \Vnormlp{\xdf_{\underline{0}}}^2\bias[\mDi]^2(\xdf)\\\hfill
    + \cst{}\big[
    \Vnormlp{\xdf_{\underline{0}}}^2\Ind{\{\mDi>1\}} \FuVg[\ssE]{\rE}(\aixEv[\mdDi]^c)+\ssE\FuVg[\ssE]{\rE}(\aixEv[\pdDi]^c) \big]
    \\\hfill
    +\cst{}\mRa{\xdf,\Lambda}
    +\cst{}\ssY^{-1}\{\miSv[\Di_{\edf,\xdfCr}]^2\Di_{\edf,\xdfCr}^3+\miSv[\ssY_{o}]^2+\Vnormlp{\xdf_{\underline{0}}}^2\Ind{\{\mDi>1\}}\}
  \end{multline}
\end{lm}
% ....................................................................
% <<Te upper bound ag p np>>
% ....................................................................
\begin{te}The last bound allows us to derive an upper bound of the
  maximal risk over the ellipsoid $\rwCxdf$ for the 
  fully data-driven aggregated estimator.
\end{te}
% ....................................................................
% <<Re upper bound ag p np>>
% ....................................................................
\begin{thm}\label{au:mrb:ag:ub:pnp}
  Consider the   penalty sequence $\peneSv:=\peneD$,
  $\Di\in\nset{1,\ssY}$, as in \nref{freq:ge:shape:uk:de:pen:oo} with numerical
  constant $\cpen\geq84$. Let
  $\hxdfAg[{\erWe[]}]=\sum_{\Di=1}^{\ssY} \erWe\hxdfPr$ be an
  aggregation of the orthogonal series estimators using either
  aggregation weights $\erWe[]$ as in \eqref{freq:ge:shape:uk:we} or model
  selection weights $\msWe[]$ as in \nref{freq:ge:shape:uk:de:msWe}. Let
  $\dr\Di_{\edf,\xdfCr}:=\floor{3(400\Vnorm[{\xdfCw[]}]{\edf}\xdfCr)^2}$
  and $\dr \ssY_{o}:=15({600})^4$. Let
  $\ssE(\Lambda):=\floor{289\log(3)\cmiSv[1]\miSv[1]}\in\Nz$. If
  $\ssE>\ssE(\Lambda)$ then set
  $\sDi{\ssE}:=\max\{\Di\in\nset{1,\ssE}:289\log(\Di+2)\cmiSv[\Di]\miSv[\Di]\leq\ssE\}$
  where the defining set, respectively, contains $1$ and thus is not
  empty.  There is a numerical constant $\cst{}$ such that for all
  $\ssY\in\Nz$ with $\aDi{\ssY}:=\aDi{\ssY}(\xdf)\in\nset{1,n}$ as in
  \nref{freq:ge:shape:uk:de:pen:oo} and for all $\ssE>\ssE(\Lambda)$ holds
  \begin{multline}\label{au:mrb:ag:ub:pnp:e1}
    \nmRi{\hxdfAg[{\erWe[]}]}{\rwCxdf,\Lambda}  
    \leq\cst{}(1\vee\xdfCr^2)\min_{\Di\in\nset{1,\ssY}}
    \{\daRa{\Di}{(\xdfCw[],\Lambda)}\vee
    \exp\big(\tfrac{-\cmiSv[\Di]\Di}{\Di_{\edf,\xdfCr}}\big)\} \\\hfill
    +\cst{}(1\vee\xdfCr^2)\{\xdfCw[(\aDi{\ssY}\wedge\sDi{\ssE})]^2\vee
    \exp\big(\tfrac{-\cmiSv[\sDi{\ssE}]\sDi{\ssE}}{\Di_{\edf,\xdfCr}}\big)\}\\\hfill
    +\cst{}\xdfCr^2\mmRa{\xdfCw[],\Lambda}   + \cst{}(1\vee\xdfCr^2)\miSv[1]^2\ssE^{-1}  
    +\cst{}\{\miSv[\Di_{\edf,\xdfCr}]^2\Di_{\edf,\xdfCr}^3+\miSv[\ssY_{o}]^2\}\ssY^{-1}
  \end{multline}
  while for $\ssE\in\nset{1,\ssE(\Lambda)}$ we have
  \begin{multline}\label{au:mrb:ag:ub:pnp:e2}
    \nmRi{\hxdfAg[{\erWe[]}]}{\rwCxdf,\Lambda}  
    \leq \cst{}\xdfCr^2\mmRa{\xdfCw[],\Lambda}\\
    + \cst{}(1\vee\xdfCr^2)\miSv[1]^2\ssE^{-1}  
    +\cst{}\{\miSv[\Di_{\edf,\xdfCr}]^2\Di_{\edf,\xdfCr}^3+\miSv[\ssY_{o}]^2\}\ssY^{-1}.
  \end{multline}
\end{thm}


\begin{cor}\label{au:mrb:ag:ub2:pnp}
  Let the assumptions of \nref{au:mrb:ag:ub:pnp} be satisfied.
    If  \ref{ge:ak:ag:ub2:pnp:npc} as in \nref{ge:ak:ag:ub2:pnp} and \ref{ge:au:ag:ub2:pnp:pc:b}
    as in \nref{ge:au:ag:ub2:pnp} hold true, then there is a constant $\cst{\xdfCw[],\xdfCr,\Lambda}$ depending
    only on $\xdfCw[]$, $\xdfCr$ and $\Lambda$ such that $\nmRi{\hxdfAg[{\erWe[]}]}{\rwCxdf,\Lambda}
    \leq \cst{\xdfCw[],\xdfCr,\Lambda}\{\naRa{(\xdfCw[],\Lambda)}+\mmRa{\xdfCw[],\Lambda}+\xdfCw[(\sDi{\ssE}\wedge\aDi{\ssY})]^2\}$ for all $\ssY,\ssE\in\Nz$ holds true.
\end{cor}
% ....................................................................
% <<Rem upper bound ag p np>>
% ....................................................................
\begin{il}\label{au:mrb:ag:ub:pnp:il} As in \nref{au:ag:ub:pnp:il}
  shown in both cases \ref{au:il:edf:o} and \ref{au:il:edf:s} is
  \ref{ge:au:ag:ub2:pnp:pc:b} satisfied.  The fully data-driven
  aggregation
  leads to an estimator attaining
  the rate $\naRa{(\xdf,\Lambda)}+\mmRa{\xdfCw[],\Lambda}$ (\nref{au:mrb:ag:ub2:pnp}),
  if also  \ref{ge:au:ag:ub2:pnp:pc:b} is satisfied and
  $\xdfCw[(\sDi{\ssE})]^2$ is negligible with
  respect to $\mmRa{\xdfCw[],\Lambda}$, otherwise the upper
  bound faces a deterioration of the rate, which we illustrate
  considering as in \nref{freq:ge:strat:kn:qu:il:rate} the usual behaviour
  \ref{freq:ge:strat:kn:qu:il:rate:np:oo}, \ref{freq:ge:strat:kn:qu:il:rate:np:os} and
  \ref{freq:ge:strat:kn:qu:il:rate:np:so} for the sequences
  $\Nsuite[\Di]{\xdfCw[(\Di)]^2}$ and $\Nsuite[\Di]{\iSv[\Di]}$.
  In all three cases \ref{freq:ge:strat:kn:qu:il:rate:np:oo}, \ref{freq:ge:strat:kn:qu:il:rate:np:os} and
  \ref{freq:ge:strat:kn:qu:il:rate:np:so} the assumption \ref{ge:au:ag:ub2:pnp:pc:b}
  holds true. Moreover, in case \ref{freq:ge:strat:kn:qu:il:rate:np:oo},
  \ref{freq:ge:strat:kn:qu:il:rate:np:os} and \ref{freq:ge:strat:kn:qu:il:rate:np:so} only with
  $p<1/2$ the assumption \ref{ge:ak:ag:ub2:pnp:npc} is satisfied, and
  $\naRa{(\xdfCw[],\Lambda)}$ equals the oracle rate $\mnRa{\xdfCw[],\Lambda}$ (cf.
  \nref{freq:ge:strat:kn:qu:il:rate:np:oo} \ref{freq:ge:strat:kn:qu:il:rate:np:oo},
  \ref{freq:ge:strat:kn:qu:il:rate:np:os} and \ref{freq:ge:strat:kn:qu:il:rate:np:so}). In case
  \ref{freq:ge:strat:kn:qu:il:rate:np:os} and \ref{freq:ge:strat:kn:qu:il:rate:np:so}
  $\xdfCw[(\sDi{\ssE})]^2\leq \cst{\xdfCw[],\xdfCr,\Lambda} \mmRa{\xdfCw[],\Lambda}$ while
  in case \ref{freq:ge:strat:kn:qu:il:rate:np:oo}
  $\xdfCw[(\sDi{\ssE})]^2\sim(\ssE/\log\ssE)^{-p/a}$, hence 
  \begin{Liste}[]
  \item[\mylabel{au:mrb:ag:ub:pnp:il:oo}{\dg\bfseries{[o-o]}}]
    $\nmRi{\hxdfAg[{\erWe[]}]}{\rwCxdf,\Lambda}
    \leq
    \cst{\xdfCw[],\xdfCr,\Lambda}\{\ssY^{-2p/(2p+2a+1)}+\ssE^{-(p\wedge a)/a}+
    (\ssE/\log\ssE)^{-p/a}\}$
      \item[\mylabel{au:mrb:ag:ub:pnp:il:os}{\dg\bfseries{[o-s]}}]
    $\nmRi{\hxdfAg[{\erWe[]}]}{\rwCxdf,\Lambda}
    \leq
    \cst{\xdfCw[],\xdfCr,\Lambda}\{(\log \ssY)^{-p/a}+(\log \ssE)^{-p/a}\}$
  \item[\mylabel{au:mrb:ag:ub:pnp:il:so}{\dg\bfseries{[s-o]}}]
    $\nmRi{\hxdfAg[{\erWe[]}]}{\rwCxdf,\Lambda}
    \leq
    \cst{\xdfCw[],\xdfCr,\Lambda}\{(\log\ssY)^{(2a+1)/(2p\wedge1)}\ssY^{-1}+\ssE^{-1}\}$
  \end{Liste}
  Consequently, the fully data-driven estimator attains the minimax
  rate in case \ref{freq:ge:strat:kn:qu:il:rate:np:oo} with $p>a$,
  \ref{freq:ge:strat:kn:qu:il:rate:np:os} and \ref{freq:ge:strat:kn:qu:il:rate:np:so} with
  $p\leq1/2$, while in  case \ref{freq:ge:strat:kn:qu:il:rate:np:oo} with $p\leq a$ and
  \ref{freq:ge:strat:kn:qu:il:rate:np:so} with $p>1/2$ the rate of the fully data-driven estimator
  $\hxdfAg[{\erWe[]}]$ features a deterioration by a logarithmic factor
  $(\log\ssE)^{p/a}$  and $(\log\ssY)^{(2a+1)(1-1/(2p))}$,
  respectively, compared to the minimax  rate.\ilEnd
\end{il}
%%
\subsection{Oracle optimality}\label{FREQ_GAUSS_ORACLE}

A direct application of \nref{lmA.1.1} and \nref{lmA.1.2} gives us the following result.
\begin{cor*}
For any $m$ and $n$ in $\N$, we have
\[\P(\Vert \theta_{n, \overline{m}} - \theta^{\circ}_{\overline{m}} \Vert_{l^{2}}^{2} \geq \penSv / 7) \leq \exp[- \tfrac{\tfrac{2 \kappa}{21} (\tfrac{2 \kappa}{21} \wedge 1)}{4} m \cmiSv];\]
\[\E[(\Vert \theta_{n, \overline{m}} - \theta^{\circ}_{\overline{m}} \Vert_{l^{2}}^{2} - \penSv / 7 )_{+}] \leq 6 n^{-1}\Lambda_{+}(m) \exp[- \tfrac{\kappa}{42} m \cmiSv].\]
\reEnd
\end{cor*}



\begin{thm}\label{THM_FREQ_IGSSM_KNOWN_IID_ORACLE_NP}
\reEnd
\end{thm}

\subsection{Minimax optimality}\label{FREQ_GAUSS_MINIMAX}
  By applying the same strategy, we derive bounds for the maximal risk over
  ellipsoids  $\rwCxdf$ of the aggregated estimator $\txdfAg[{\erWe[]}]$.
  Therefore, we aim next to control the second and
  third right hand side term in \eqref{freq:ge:strat:kn:co:agg:e1} uniformly over
  $\rwCxdf$.
  Keeping the definition \nref{gauss:pen} of
$\daRa{\Di}{\xdfCw[],\Lambda}$  in mind it holds
$\xdfCr^2\dRa{\Di}{\xdfCw[],\Lambda}\geq\Vnormlp{\xdf_{\underline{0}}}^2\bias^2(\xdf)$
uniformly for all $\xdf\in\rwCxdf$ and for all
$\Di\in\Nz$.
\begin{de}
We use $\DipenSv$ and $\penSv$ defined in \nref{gau:pen:oo} and define
  $\daRa{\Di}{(\xdfCw[])}:=\daRa{\Di}{(\xdfCw[],\Lambda)}:=[\xdfCw^2\vee \DipenSv\,\ssY^{-1}]$.
  Then, it holds
  \begin{equation*}
    [\xdfCr^2+\cpen]\daRa{\Di}{(\xdfCw[])}\geq\big[\Vnormlp{\xdf_{\underline{0}}}^2\bias^2(\xdf)\vee\penSv\big]\quad\text{for
      all $\Di\in\nset{1,\ssY}$ and $\xdf\in\rwCxdf$}.
\end{equation*}
And we define the specific choice:
    $\aDi{\ssY}(\xdfCw[]):=\argmin\Nset[\Di\in\Nz]{\daRa{\Di}{(\xdfCw[],\Lambda)}}\in\nset{1,\ssY}$;\\
    $\naRa{(\xdfCw[])}:=\naRa{(\xdfCw[],\Lambda)}:=\min\Nset[\Di\in\Nz]{\daRa{\Di}{(\xdfCw[],\Lambda)}}$
    with $\daRa{\aDi{\ssY}(\xdfCw[])}{(\xdfCw[],\Lambda)}=\naRa{(\xdfCw[],\Lambda)}$.
\assEnd
\end{de}
For any $\pdDi,\mdDi\in\nset{1,\ssY}$ let us define 
\begin{multline*}
\mDi:=\min\set{\Di\in\nset{1,\mdDi}: \Vnormlp{\xdf_{\underline{0}}}^2\bias[\Di]^2(\xdf)\leq
  [\xdfCr^2+4\cpen]\dRa{\mdDi}{\xdfCw[]}}\quad\text{and}\\\pDi:=\max\set{\Di\in\nset{\pdDi,\ssY}:
   \penSv \leq 2[3\xdfCr^2+ 2\cpen] \dRa{\pdDi}{\xdfCw[]}}
\end{multline*}
where  the defining sets obviously contains $\mdDi$ and $\pdDi$, respectively, and hence, they are
not empty.
Hence we have, for any $\theta^{\circ}$ in $\rwCxdf$
 \begin{multline*}
    \E\Vnormlp{\txdfAg-\xdf}^2 \leq 3c([\xdfCr^2+\cpen]\daRa{\pDi}{(\xdfCw[])}) +2r^{2} [r^{2}+28 (\tfrac{3c}{2} \vee 1)][\xdfCr^2+\cpen]\daRa{\mdDi}{(\xdfCw[])} \\
    + \tfrac{2}{7}\sum\nolimits_{\Di\in\nsetlo{\pDi,\ssY}}\tfrac{21c}{2}[\xdfCr^2+\cpen]\daRa{\Di}{(\xdfCw[])}\exp[\eta n (\tfrac{21c}{2}[\xdfCr^2+\cpen]\daRa{\Di}{(\xdfCw[])}/2 - 14 (\tfrac{3c}{2} \vee 1) [\xdfCr^2+\cpen]\daRa{\pdDi}{(\xdfCw[])})]\\\hfill
    +2r^{2} (\sum\nolimits_{m = 1}^{m_{-}} \exp[-14 \eta n (\tfrac{3c}{2} \vee 1) [\xdfCr^2+\cpen]\daRa{\mdDi}{(\xdfCw[])}] + \exp[- \tfrac{c (c \wedge 1)}{4} m_{-}^{\dagger} \cmiSv[\mdDi]])\\
+2\sum_{\Di\in\nset{\pDi,\ssY}}6 n^{-1}\Lambda_{+}(m) \exp[- \tfrac{c}{4} m \cmiSv]  
+\tfrac{2}{7}\sum_{\Di\in\nsetlo{\pDi,\ssY}}\tfrac{21c}{2}[\xdfCr^2+\cpen]\daRa{\Di}{(\xdfCw[])}\exp[- \tfrac{c (c \wedge 1)}{4} m \cmiSv].
\end{multline*}

\begin{thm}\label{THM_FREQ_IGSSM_KNOWN_IID_MINIMAX_NP}
\reEnd
\end{thm}



\section{Inverse Gaussian sequence space model with partially known operator}\label{FREQ_IGSSM_UNKNOWN}
%%
\section{Circular deconvolution with independent data and known noise density}\label{FREQ_CIRCDECONV_KNOWN_IID}

As stated in \nref{BAYES}, the non-conjugated nature of the hierarchical Gaussian sieve in the context of circular deconvolution does not allow to compute the posterior mean analytically.
However, in this part we mimic the form of this posterior mean and construct an estimator from this idea.

We start by reminding the definition of the projection estimators, which we will use to surrogate the posterior mean of sieve priors, which appear in the structure of the posterior mean of hierarchical sieves.

\begin{rem}{\textsc{Projection estimators} \\}\label{REM_FREQ_CIRCDECONV_KNOWN_IID_PROJEST}
We recall the notation for the projection estimators, for any $m$ in $\mathds{Z}$, we have
\begin{alignat*}{3}
& \overline{\theta}_{m} && := && \mathds{1}_{m = 0} + \mathds{1}_{m \neq 0} \frac{1}{n}\sum\limits_{p = 1}^{n} \frac{e_{m}(Y_{p})}{\lambda_{m}};\\
& \left( \overline{\theta}^{m}_{j} \right)_{j \in \mathds{Z}} && := &&\left(\mathds{1}_{\vert j \vert \leq m} \overline{\theta}_{j}\right)_{j \in \mathds{Z}}.
\end{alignat*}
\end{rem}

As in the posterior mean of hierarchical sieves, we define a weight sequence, corresponding to the posterior distribution of the threshold parameter.

\begin{de}{\textsc{Weight sequence} \\}\label{DE_FREQ_CIRCDECONV_KNOWN_IID_WEIGHT}
Let be the following quantities:
\begin{alignat*}{3}
& \kappa && := && \frac{23}{2};\\
& \cmiSv && := && \frac{\log\left(\Di \iSv[\Di] \vee (\Di + 2)\right)^{2}}{\log\left(\Di + 2\right)^{2}};\\
& \DipenSv && := && \Di \iSv[\Di] \cmiSv; \\
& \pen(\Di) && := && \frac{9}{2} \cdot 12 \cdot \cpen \cdot \DipenSv;\\
& \Upsilon(Y, \Di) && := && n \left\Vert \overline{\theta}^{m} \right\Vert_{l^{2}}^{2}.
\end{alignat*}
Then, for any couple of natural integers $n$ and $\eta$, we define the distribution $\P_{M \vert Y^{n}}^{n, (\eta)}$, dominated by the counting measure on $\N^{\star}$ such that, for any $m$ in $\llbracket 1, n \rrbracket$
\[\P_{M \vert Y^{n}}^{n, (\eta)}(m) := \frac{\exp\left[\eta\left(- \pen(m) + \Upsilon(m, Y^{n})\right)\right]}{\sum\limits_{k = 1}^{n} \exp\left[\eta\left(- \pen(k) + \Upsilon(k, Y^{n})\right)\right]}.\]
\end{de}

With those definitions at hand, we are able to define an estimator that reproduces the structure of the posterior mean of iterated hierarchical sieves.

\begin{de}{\textsc{Aggregation/shrinkage estimator} \\}\label{DEFREQ_CIRCDECONV_KNOWN_IID_AGGREGEST}
Using the notations we just introduced, we define, for any strictly positive integer $\eta$ the shrinkage/aggregation estimator $\widehat{\theta}^{(\eta)}$ such that, for any $j$ in $\mathds{Z}$
\begin{alignat*}{3}
& \widehat{\theta}^{(\eta)}_{j} && := && \P_{M \vert Y^{n}}^{n, (\eta)}(\llbracket \vert j \vert, n \rrbracket) \overline{\theta}_{j};\\
& \widehat{\theta}^{\eta} && := && \sum\limits_{j = 1}^{n} \P_{M \vert Y^{n}}^{n, (\eta)}(j) \overline{\theta}^{j}.
\end{alignat*}
\end{de}

As previously, one can notice that, as $\eta$ tends to infinity, this estimator converges to the penalised contrast maximiser projection estimator with penalty function $\pen$ and contrast $\Upsilon$.

Using the method described in \nref{FREQ_STRATEGY}, we are able to show that, for any $\theta^{\circ}$, the sequence defined hereafter is a convergence rate.

\begin{de}{\textsc{Convergence rate} \\}\label{DE_FREQ_CIRCDECONV_KNOWN_IID_CONVRATE}
Let be the sequences:
\[m^{\dagger}_{n} := \argmin_{m \in \N}\left\{\left[\mathfrak{b}_{m}^{2}(\theta^{\circ})\mathfrak{b}_{0}^{-2}(\theta^{\circ}) \vee 2 \frac{m \Lambda_{(m)}}{n} \psi_{n}\right]\right\};\]
and
\[\Phi^{\dagger}_{n} := \left[\mathfrak{b}_{m^{\dagger}_{n}}^{2}(\theta^{\circ})\mathfrak{b}_{0}^{-2}(\theta^{\circ}) \vee 2 \frac{m^{\dagger}_{n} \Lambda_{(m^{\dagger}_{n})}}{n} \psi_{n}\right].\]
\end{de}

\begin{de}{\textsc{Accelerated convergence rate} \\}\label{DE_FREQ_CIRCDECONV_KNOWN_IID_CONVRATEFAST}
Let be the sequences:
\[m^{\circ}_{n} := \argmin_{m \in \N}\left\{\left[\mathfrak{b}_{m}^{2}(\theta^{\circ})\mathfrak{b}_{0}^{-2}(\theta^{\circ}) \vee 2 \frac{m \overline{\Lambda}_{m}}{n} \psi_{n}\right]\right\};\]
and
\[\Phi^{\circ}_{n} := \left[\mathfrak{b}_{m^{\circ}_{n}}^{2}(\theta^{\circ})\mathfrak{b}_{0}^{-2}(\theta^{\circ}) \vee 2 \frac{m^{\circ}_{n} \overline{\Lambda}_{m^{\circ}_{n}}}{n} \psi_{n}\right].\]
\end{de}

\bigskip

% ....................................................................
% <<Ass upper bound p>>
% ....................................................................
\begin{as}\label{ass:ub:p}
Let $\fxdf$ have a finite series expansion as definied in \ref{oo:xdf:p}, that is, either
\begin{inparaenum}[i]\renewcommand{\theenumi}{\dgrau\rm(\alph{enumi})}
\item\label{ass:ub:p:c1}
	$\fxdf=\left(\mathds{1}_{j = 0}\right)_{j \in \mathds{Z}}$, i.e., $\bias[0](\fxdf)=\Vnormlp{\Proj[{\mHiH[0]}]^\perp\fxdf}^2=0$ or
\item\label{ass:ub:p:c2}
	there is $K\in\N$ with $1\geq \bias[{K-1}](\fxdf)>0$ and $\bias[K](\fxdf)=0$.
\end{inparaenum}
In case  \ref{ass:ub:p:c1} set $\dr\ssY_{\fxdf,\iSv}:=\ceil{15(\tfrac{300}{\sqrt{\cpen}})^4}$ while in case \ref{ass:ub:p:c2} given $K_{\fydf}:=K\dr\vee 3(\tfrac{800\Vnormlp[1]{\fydf}}{\cpen})^2$ and $c_{\fxdf}:=\tfrac{2\Vnormlp{\Proj[{\mHiH[0]}]^\perp\fxdf}^2+484\cpen}{\Vnormlp{\Proj[{\mHiH[0]}]^\perp\fxdf}^2\bias[{K-1}]^2(\fxdf)}$ let there $\ssY_{\fxdf,\iSv}\in\N$ be with $\ssY_{\fxdf,\iSv}>\ceil{c_{\fxdf}\DipenSv[K_{\fydf}]\dr\vee15(\tfrac{300}{\sqrt{\cpen}})^4}$ such that $\sDi{\ssY}:=\max\{\Di\in\nset{K,\ssY}:c_{\fxdf}\,\DipenSv<\ssY\}$ where the defining set contains $K_{\fydf}$ and thus it is not empty, satisfies $\cmiSv[\sDi{\ssY}]\sDi{\ssY}\geq K_{\fydf}(\log\ssY)$ for all $\ssY\geq \ssY_{\fxdf,\iSv}$.
\end{as}
% ....................................................................
% <<Rem Ass upper bound p>>
% ....................................................................
\begin{rmk}\label{rem:ass:ub:p}
 Keep in mind that $\Nsuite[\Di]{\bias(\fxdf)}\subset[0,1]$ is monotonically non increasing with $\bias[1](\fxdf)\leq1$ and $\lim_{\Di\to\infty}\bias(\fxdf)=0$.
Thereby, in case \ref{ass:ub:p:c2} of \nref{ass:ub:p} holds $\Vnormlp{\Proj[{\mHiH[0]}]^\perp\fxdf}^2>0$ and $\tfrac{2\Vnormlp{\Proj[{\mHiH[0]}]^\perp\fxdf}^2+484\cpen}{\Vnormlp{\Proj[{\mHiH[0]}]^\perp\fxdf}^2\bias[{K-1}]^2(\fxdf)}\geq1$.
We shall stress that in \nref{re:ub:co1} and \nref{re:ub:co2} in the \nref{PRO_FREQ_CIRCDECONV_KNOWN_IID_ORACLE_P} we derive upper bounds for the partially data-driven aggregated OSE featuring a deterioration of the upper bound which, due to \nref{ass:ub:p} is avoided in the next assertion.
\end{rmk}
% ....................................................................
% <<Il Ass upper bound p>> \ref{il:ass:ub:p}
% ....................................................................
\begin{il}\label{il:ass:ub:p}
Let us illustrate \nref{ass:ub:p} considering as in \nref{il:oo} the commonly studied behaviours \ref{il:edf:o} and \ref{il:edf:s} for the sequence  $\Nsuite[j]{\iSv[j]}$.
\begin{Liste}[]
\item[\mylabel{il:ass:ub:p:o}{\dg\bfseries{(o)}}]
Let $\iSv[\Di]\sim \Di^{2a}$, $a>0$, then  we have $\cmSv\sim1$, $\miSv\sim \oiSv\sim\Di^{2a}$, $\DipenSv=\cmSv \Di \miSv\sim \Di^{2a+1}$ and hence $1\sim\DipenSv[\sDi{\ssY}]\ssY^{-1}\sim(\sDi{\ssY})^{2a+1}\ssY^{-1}$ implies $\sDi{\ssY}\sim\ssY^{1/(2a+1)}$ and $\sDi{\ssY}\cmSv[\sDi{\ssY}]\sim\ssY^{1/(2a+1)}$.
\item[\mylabel{il:ass:ub:p:s}{\dg\bfseries{(s)}}]
Let $\iSv[\Di]\sim \exp(\Di^{2a})$, $a>0$, then  we have $\cmSv\sim(\Di^{2a})^2$, $\DipenSv=\Di \cmSv \miSv\sim \Di^{1+4a}\exp(\Di^{2a})$ and hence $\ssY\sim\DipenSv[\sDi{\ssY}]\sim (\sDi{\ssY})^{1+4a}\exp((\sDi{\ssY})^{2a})$ implies $\sDi{\ssY}\sim(\log\ssY-\tfrac{1+4a}{2a}\log\log\ssY)^{1/(2a)}$ and $\sDi{\ssY}\cmSv[\sDi{\ssY}]\sim (\log \ssY)^{2+1/(2a)}$.
\end{Liste}
Clearly, in both cases \ref{il:ass:ub:p:o} and \ref{il:ass:ub:p:s}, there is ${\ssY}_{\fxdf,\iSv}\in\Nz$ such that $\cmiSv[\sDi{\ssY}]\sDi{\ssY}\geq K_{\fydf}(\log\ssY)$  for all $\ssY\geq{\ssY}_{\fxdf,\iSv}$ holds true.
\end{il}

More precisely, we obtain the following theorem, for which the proof is given in \nref{PRO_FREQ_CIRCDECONV_KNOWN_IID_ORACLE_P}.
% ....................................................................
% <<Re upper bound p>>
% ....................................................................
\begin{thm}\label{THM_FREQ_CIRCDECONV_KNOWN_IID_ORACLE_P}
Let $\fxdf$ have a finite series expansion as defined in \ref{oo:xdf:p}.
Under \nref{ass:ub:p} there is a finite numerical constant $\cst{}$ such that for all $\dr\ssY\in\N$,
\begin{equation}\label{re:ub:p:e1}
\FuEx[\ssY]{\fxdf}\left[\Vnormlp{\txdf-\xdf}^2\right]
\leq\cst{}\{\DipenSv[{\ssY_{\fxdf,\iSv}}]+\VnormLp{\Proj[{\mHiH[0]^\perp}]\xdf}^2\ssY_{\xdf,\iSv}+ \Vnormlp[1]{\fydf}^2\}\ssY^{-1}.
\end{equation}
\end{thm}
% ....................................................................
% <<Il upper bound co2>>
% ....................................................................
\begin{il}\label{il:ub:p}
Let us illustrate \nref{THM_FREQ_CIRCDECONV_KNOWN_IID_ORACLE_P}
  considering as in \nref{il:ass:ub:p} the behaviours
  \ref{il:edf:o} and \ref{il:edf:s} for the sequence
  $\Nsuite[j]{\iSv[j]}$.  Keeping in mind that as shown in \nref{il:ass:ub:p} there is ${\ssY}_{\xdf,\iSv}\in\Nz$ such that
    $\cmiSv[\sDi{\ssY}]\sDi{\ssY}\geq K_{\ydf}(\log\ssY)$  for all $\ssY\geq{\ssY}_{\xdf,\iSv}$ 
 holds true, due to \nref{re:ub:co2} there is a constant $\cst{\xdf,\edf}$
 depending only on the densities $\xdf$ and $\edf$ such that 
 $\FuEx[\ssY]{\rY}\VnormLp{\txdf-\xdf}^2\leq
 \cst{\xdf,\edf}\ssY^{-1}$ for all $\ssY\in\Nz$. Comparing the last result
 with the oracle rate derived in \ref{il:oo:po} and  \ref{il:oo:so}
 in \nref{il:oo} we conclude, that $\txdf$ is optimal  in an oracle sense in both cases \ref{il:oo:po} and  \ref{il:oo:so}.
\end{il}

\medskip

% ....................................................................
% <<Ass upper bound np>> \ref{ass:ub:np}
% ....................................................................
\begin{as}\label{ass:ub:np} Let $\xdf$  have an infinite series expansion
  as definied in \ref{oo:xdf:np}, that is, $1\geq \bias(\xdf)>0$ for all $\Di\in\Nz$.
Given   $\Di_{\ydf}:=\dr3(\tfrac{800\Vnormlp[1]{\fydf}}{\cpen})^2$ and
$\tDi_{\ydf}=\min\{\Di\in\Nz:\bias[\Di_{\ydf}](\xdf)>\bias[\Di](\xdf)\}$
 there is $\ssY_{\xdf,\iSv}\in\Nz$ with
$\ssY_{\xdf,\iSv}\geq\ceil{\tfrac{\DipenSv[\tDi_{\ydf}]}{\bias[\tDi_{\ydf}]^2(\xdf)}\vee\dr15(\tfrac{300}{\sqrt{\cpen}})^4}$
  such that either \begin{inparaenum}[i]\renewcommand{\theenumi}{\dgrau\rm(\alph{enumi})}\item\label{ass:ub:np:c1}
$\cmiSv[\aDi{\ssY}]\aDi{\ssY}\geq \Di_{\ydf}|\log\hRa{\xdf,\iSv}|$ 
for all
$\ssY\geq{\ssY}_{\xdf,\iSv}$ or \item\label{ass:ub:np:c2}  
$\aDi{\ssY}\leq  \Di_{\ydf}|\log\hRa{\xdf,\iSv}|$ for all
$\ssY\geq{\ssY}_{\xdf,\iSv}$.
\end{inparaenum}
We set in case \ref{ass:ub:np:c1}  $\sDi{\ssY}:=\aDi{\ssY}$ and  in case \ref{ass:ub:np:c2}  $\sDi{\ssY}:= \Di_{\ydf}|\log\hRa{\xdf,\iSv}|$.
\end{as}
% ....................................................................
% <<Rem Ass upper bound np>>
% ....................................................................
\begin{rmk}\label{rem:ass:ub:np}
Considering $\Di_{\ydf}:=\dr3(\tfrac{800\Vnormlp[1]{\fydf}}{\cpen})^2$ and
$\tDi_{\ydf}=\min\{\Di\in\Nz:\bias[\Di_{\ydf}](\xdf)>\bias[\Di](\xdf)\}$
as defined in \nref{ass:ub:np} the defining set is not empty since $\bias[\Di](\xdf)>0$ for all
$\Di\in\Nz$ and $\lim_{\Di\to\infty}\bias[\Di](\xdf)=0$. Moreover, it
holds $\tDi_{\ydf}>\Di_{\ydf}$ due to the the monotonicity of
$\bias(\xdf)$. Noting that 
$\ceil{\tfrac{\DipenSv[\tDi_{\ydf}]}{\bias[\tDi_{\ydf}]^2(\xdf)}\vee\dr15(\tfrac{300}{\sqrt{\cpen}})^4}\geq \DipenSv[\tDi_{\ydf}]\geq\tDi_{\ydf}$ by construction,
for all $\ssY\geq\ssY_{\xdf,\iSv}$ as in \nref{ass:ub:np} holds 
$\oRaDi{\Di_{\ydf},\xdf,\iSv}\geq\bias[\Di_{\ydf}]^2(\xdf)>\bias[\tDi_{\ydf}]^2(\xdf)% =\bias[\tDi_{\Sv}]^2(\So)[1\vee
% \tfrac{\tDi_{\Sv}\cmiSv[\tDi_{\Sv}]\miSv[\tDi_{\Sv}]/\nlIm}{\bias[\tDi_{\Sv}]^2(\So)}]
=\oRaDi{\tDi_{\ydf},\xdf,\iSv}$
and hence, for all
$\ssY\geq\ssY_{\xdf,\iSv}$ we have $\aDi{\ssY}>
\Di_{\ydf}$.
We use these preliminary findings in the proof of \nref{THM_FREQ_CIRCDECONV_KNOWN_IID_ORACLE_NP}.  
\end{rmk}
% ....................................................................
% <<Il Ass upper bound p>> \ref{il:ass:ub:np}
% ....................................................................
\begin{il}\label{il:ass:ub:np}
Let us illustrate \nref{ass:ub:np}
  considering as in \nref{il:oo} usual
  behaviour \ref{il:oo:oo}, \ref{il:oo:so} and \ref{il:oo:os}
 for the sequences $\Nsuite[\Di]{\bias[\Di](\xdf)}$ and
  $\Nsuite[\Di]{\iSv[\Di]}$:
 \begin{Liste}[]
\item[\mylabel{il:ass:ub:np:oo}{\dg\bfseries{[o-o]}}] Since
  $\bias^2(\xdf)\sim\Di^{-2p}$ and  $\DipenSv\sim\Di^{2a+1}$
    (cf.  \nref{il:ass:ub:p} \ref{il:ass:ub:p:o}) follows
    $\hRa{\xdf,\iSv}\sim(\aDi{\ssY})^{-2p}\sim\DipenSv[\aDi{\ssY}]\ssY^{-1}\sim(\aDi{\ssY})^{2a+1}\ssY^{-1}$ which
    implies $\aDi{\ssY}\sim\ssY^{1/(2p+2a+1)}$,
    $\cmiSv[\aDi{\ssY}]\aDi{\ssY}\sim\ssY^{1/(2p+2a+1)}$,
    $\hRa{\xdf,\iSv}\sim\ssY^{-2p/(2p+2a+1)}$ and $|\log\hRa{\xdf,\iSv}|\sim(\log\ssY)$.
 \item[\mylabel{il:ass:ub:np:os}{\dg\bfseries{[o-s]}}]
    Since
  $\bias^2(\xdf)\sim\Di^{-2p}$ and $\DipenSv\sim\Di^{1+4a}\exp(\Di^{2a})$ (cf. \nref{il:ass:ub:p} \ref{il:ass:ub:p:s}) follows
    $\hRa{\xdf,\iSv}\sim(\aDi{\ssY})^{-2p}\sim\DipenSv[\aDi{\ssY}]\ssY^{-1}\sim(\aDi{\ssY})^{1+4a}\exp((\aDi{\ssY})^{2a})$
    which implies  $\aDi{\ssY}\sim(\log\ssY)^{1/(2a)}$, $\cmiSv[\aDi{\ssY}]\aDi{\ssY}\sim(\log\ssY)^{2+1/(2a)}$, 
    $\hRa{\xdf,\iSv}\sim(\log\ssY)^{-p/a}$ and $|\log\hRa{\xdf,\iSv}|\sim(\log\log\ssY)$.
 \item[\mylabel{il:ass:ub:np:so}{\dg\bfseries{[s-o]}}]  Since
  $\bias^2(\xdf)\sim\exp(-\Di^{2p})$ and    $\DipenSv\sim\Di^{2a+1}$
    (cf.  \nref{il:ass:ub:p} \ref{il:ass:ub:p:o}) follows
    $\hRa{\xdf,\iSv}\sim\exp(-(\aDi{\ssY})^{2p})\sim\DipenSv[\aDi{\ssY}]\ssY^{-1}\sim
(\aDi{\ssY})^{2a+1}\ssY^{-1}$
    which implies  $\aDi{\ssY}\sim(\log\ssY)^{1/(2p)}$, $\cmiSv[\aDi{\ssY}]\aDi{\ssY}\sim(\log\ssY)^{1/(2p)}$,
    $\hRa{\xdf,\iSv}\sim(\log\ssY)^{(2a+1)/(2p)}\ssY^{-1}$
    and     $|\log\hRa{\xdf,\iSv}|\sim(\log\ssY)$.
  \end{Liste}
Clearly,  there is ${\ssY}_{\xdf,\iSv}\in\Nz$ such that for all
$\ssY\geq{\ssY}_{\xdf,\iSv}$ in the cases \ref{il:ass:ub:np:oo} and
\ref{il:ass:ub:np:os}   $\cmiSv[\aDi{\ssY}]\aDi{\ssY}\geq
\Di_{\ydf}|\log\hRa{\xdf,\iSv}|$, i.e., \nref{ass:ub:np}
\ref{ass:ub:np:c1} holds, while in case \ref{il:ass:ub:np:so}
$\aDi{\ssY}\leq \Di_{\ydf}|\log\hRa{\xdf,\iSv}|$ for $p\geq1/2$, i.e., \nref{ass:ub:np}
\ref{ass:ub:np:c2} holds, and $\cmiSv[\aDi{\ssY}]\aDi{\ssY}\geq
\Di_{\ydf}|\log\hRa{\xdf,\iSv}|$ for $p<1/2$, i.e., \nref{ass:ub:np}
\ref{ass:ub:np:c1} holds.
\end{il}

\begin{thm}\label{THM_FREQ_CIRCDECONV_KNOWN_IID_ORACLE_NP}
Let be the constants $K := \frac{\sqrt{2} - 1}{21 \sqrt{2}}$, and $C_{\lambda, \theta^{\circ}} \geq \sum\limits_{j = 1}^{\infty} \exp\left[- \eta \frac{\psi_{n} m \overline{\Lambda}_{m}}{2}\right]$ then, for any $n$ and $\eta$ integers greater than $1$, we have
\begin{alignat*}{3}
& \E_{\theta^{\circ}}^{n}\left[\Vert \widehat{\theta}^{(\eta)} - \theta^{\circ} \Vert_{l^{2}}^{2}\right] && \leq && 174 \mathfrak{b}_{0}^{2} \Phi^{\dagger}_{n} + \\
& && && \frac{1}{n} 32 C_{\lambda, \theta} \exp\left[- K\left(\frac{\psi_{n} 2 m^{\circ}_{n}}{\left\Vert \theta^{\circ} \right\Vert_{l^{2}}^{2} \left\Vert \lambda \right\Vert_{l^{2}}^{2}}\wedge \sqrt{n \psi_{n}}\right) + \log(\psi_{n}) + 2 \log\left(\Lambda_{n} \vee n\right)\right];
\end{alignat*}
assuming $m^{\dagger}_{n}$ tends to $\infty$, this gives
\[\E_{\theta^{\circ}}^{n}\left[\Vert \widehat{\theta}^{(\eta)} - \theta^{\circ} \Vert_{l^{2}}^{2}\right] \in \mathcal{O}_{n}\left(\Phi^{\dagger}_{n}\right).\]
\end{thm}

Comparison with the oracle rate of projection estimators reveals that in many cases, we obtain an oracle optimal estimator.

\begin{il}\label{IL_FREQ_CIRCDECONV_KNOWN_IID_ORACLE_NP}
Assume $m^{\dagger}_{n}$ tends to infinity and let be two positive real numbers $p$ and $a$.

If $\mathfrak{b}_{m}^{2} \asymp_{m \rightarrow \infty} m^{-2p}$ and $\Lambda_{m} \asymp_{m \rightarrow \infty} m^{2a}$, then we have $\psi_{n} \asymp_{n \rightarrow \infty} 4a$; $m^{\dagger}_{n} \asymp_{n \rightarrow \infty} n^{\frac{1}{2a + 2p + 1}}$; and $\Phi^{\dagger}_{n} \asymp_{n \rightarrow \infty} n^{-\frac{2p}{2a + 2p + 1}}$ and we have
\begin{alignat*}{3}
& \E_{\theta^{\circ}}^{n}\left[\left\Vert \widehat{\theta}^{(\eta)} - \theta^{\circ} \right\Vert_{l^{2}}^{2}\right] && \in && \mathcal{O}_{n}\left(n^{-\frac{2p}{2a + 2p + 1}} +\right.\\
& && && \left. \frac{1}{n} C_{\lambda, \theta^{\circ}} \exp\left[- K \left(\frac{8 a n^{\frac{1}{2a + 2p + 1}}}{\Vert \lambda \Vert_{l^{2}} \Vert \theta^{\circ} \Vert_{l_{2}}} \wedge \sqrt{4 a n}\right) + 4 a \log(n) + 2 n^{2a} \right]\right)\\
& && \in && \mathcal{O}_{n}(n^{-\frac{2p}{2a + 2p + 1}})
\end{alignat*}

\medskip

On the other hand, if $\Lambda_{m} \asymp_{m \rightarrow \infty} \exp\left[m^{2a}\right]$, then we have $\psi_{n} \asymp_{n \rightarrow \infty} \frac{n^{4a}}{\log(n)^{2}}$; $m^{\dagger}_{n} \asymp_{n \rightarrow \infty} \log(n)^{\frac{1}{2a}}$; and $\Phi^{\dagger}_{n} \asymp_{n \rightarrow \infty} \log(n)^{-\frac{p}{a}}$.
Which leads to
\begin{alignat*}{3}
& \E_{\theta^{\circ}}^{n}\left[\left\Vert \widehat{\theta}^{(\eta)} - \theta^{\circ} \right\Vert_{l^{2}}^{2}\right] && \in && \mathcal{O}_{n}\left(\log(n)^{-\frac{p}{a}} + \right. \\
& && && \left. \frac{1}{n} C_{\lambda, \theta^{\circ}} \exp\left[- K \left(\frac{n^{4a} 2 \log(n)^{\frac{1 - 4a}{2a}}}{\Vert \lambda \Vert_{l^{2}} \Vert \theta^{\circ} \Vert_{l_{2}}} \wedge \frac{n^{2a + \frac{1}{2}}}{\log(n)}\right) + 4 a \log(n) + 2 n^{2a} \right]\right)\\
& && \in && \mathcal{O}_{n}(\log(n)^{-\frac{p}{a}}).
\end{alignat*}
\end{il}

Alternatively, under stronger assumptions, we obtain the following theorem, for which the proof is given in \nref{PRO_FREQ_CIRCDECONV_KNOWN_IID_ORACLE_NP_FAST}.

\begin{thm}\label{THM_FREQ_CIRCDECONV_KNOWN_IID_ORACLE_NP_FAST}
Let be the constants $K := \frac{\sqrt{2} - 1}{21 \sqrt{2}}$, and $C_{\lambda, \theta^{\circ}} \geq \sum\limits_{j = 1}^{\infty} \exp\left[- \eta \frac{\psi_{n} m \overline{\Lambda}_{m}}{2}\right]$ then, for any $n$ greater than $1$ and $\eta$ greater than $118 K$, we have
\begin{alignat*}{3}
& \E_{\theta^{\circ}}^{n}\left[\Vert \widehat{\theta}^{(\eta)} - \theta^{\circ} \Vert_{l^{2}}^{2}\right] && \leq && 174 \mathfrak{b}_{0}^{2} \Phi^{\circ}_{n} + \\
& && && \frac{1}{n} 32 C_{\lambda, \theta} \exp\left[- K\left(\frac{\psi_{n} 2 m^{\circ}_{n} \overline{\Lambda}_{m^{\circ}_{n}}}{\Lambda_{(m^{\circ}_{n})} \left\Vert \theta^{\circ} \right\Vert_{l^{2}}^{2} \left\Vert \lambda \right\Vert_{l^{2}}^{2}}\wedge \sqrt{n \psi_{n}}\right) + \log(\psi_{n}) + 2 \log\left(\Lambda_{(n)} \vee n\right)\right];
\end{alignat*}
assuming $m^{\circ}_{n}$ tends to $\infty$, this gives
\[\E_{\theta^{\circ}}^{n}\left[\Vert \widehat{\theta}^{(\eta)} - \theta^{\circ} \Vert_{l^{2}}^{2}\right] \in \mathcal{O}_{n}\left(\Phi^{\dagger}_{n}\right).\]
\end{thm}

\section{Absolutely regular observation process with known noise density}
\subsection{Circular deconvolution with independent data and partially known noise density}\label{FREQ_CIRCDECONV_UNKNOWN_IID}

Finally, in this section, we consider the case of a partially known operator as described in \nref{INTRO_INVERSE_UNKNOWN}.

As a consequence, we need here to simultaneously estimate the distribution $\P_{\epsilon}$ of the noise random variable $\epsilon$ and the density of interest $f^{X}$.
We now assume that we have at hand two independent samples.
The first is an i.i.d. sample from $\P^{\epsilon}$, denoted $\epsilon^{q} = \left(\epsilon_{r}\right)_{r \in \llbracket 1, q \rrbracket}$; the second is, as in the known operator case, a sample from the convolved distribution, assumed here to be i.i.d., denoted $Y^{n} = \left(Y_{p}\right)_{p \in \llbracket 1, n \rrbracket}$.

\medskip

We want to adapt our aggregation estimator shape to this case. In this perspective let us define an estimators for $\lambda^{-1}$.

\begin{de}{\textsc{Thresholded estimator} \\}\label{DE_FREQ_CIRCDECONV_UNKNOWN_IID_THRESHOLDEDEST}
For any $m$ in $\mathds{Z}$ define, with the convention "$0/0 = 0$"
\begin{alignat*}{3}
& \widehat{\lambda}_{m} && := && \frac{1}{q} \sum\limits_{r = 1}^{q} e_{m}(\epsilon_{r});\\
& \widehat{\lambda}^{+}_{m} && := && \frac{1}{\widehat{\lambda}_{m}} \mathds{1}_{\vert \widehat{\lambda}_{m}\vert^{2} > \frac{1}{q}}.
\end{alignat*}
Mimicking notations we used until here we also note, for any $m$ in $\N$
\begin{alignat*}{3}
& \widehat{\Lambda}_{m} && := && \vert \widehat{\lambda}^{+}_{m} \vert^{2}\\
& \widehat{\Lambda}_{(m)} && := && \max\left\{\widehat{\Lambda}_{k}, k \in \llbracket 1, m \rrbracket \right\};
\end{alignat*}
\end{de}

With this definition, we define an alternative form for projection estimators which we aggregated in the two previous sections.

\begin{de}{\textsc{Thresholded projection estimators} \\}\label{DE_FREQ_CIRCDECONV_UNKNOWN_IID_THRESHOLDEDPROJEST}
For any $m$ in $\mathds{Z}$, let be
\begin{alignat*}{3}
& \overline{\theta}_{m}^{+} && := && \mathds{1}_{m = 0} + \mathds{1}_{m \neq 0} \frac{1}{n}\sum\limits_{p = 1}^{n} e_{m}(Y_{p}) \widehat{\lambda}_{m}^{+};\\
& \left( \overline{\theta}^{m, +}_{j} \right)_{j \in \mathds{Z}} && := &&\left(\mathds{1}_{\vert j \vert \leq m} \overline{\theta}^{+}_{j}\right)_{j \in \mathds{Z}}.
\end{alignat*}
\end{de}

We give the following shape to the weight sequence.

\begin{de}{\textsc{Weight sequence} \\}\label{DE_FREQ_CIRCDECONV_UNKNOWN_IID_WEIGHT}
Let be the following quantities:
\begin{alignat*}{3}
& \kappa && \geq && 1;\\
& \sqrt{\delta^{\widehat{\Lambda}}_{m}} && := && \frac{\log\left(k \widehat{\Lambda}_{(m)} \vee \left(m + 2\right)\right)}{\log\left(m + 2\right)}\\
& \Delta^{\widehat{\Lambda}}_{m} && := && \delta^{\widehat{\Lambda}}_{m} m \widehat{\Lambda}_{(m)}\\
& \pen(m) && := && \frac{9}{2} 12 \kappa \Delta^{\widehat{\Lambda}}_{m};\\
& \Upsilon(Y, \epsilon, m) && := && n \left\Vert \overline{\theta}^{m, +} \right\Vert_{l^{2}}^{2}.
\end{alignat*}
Then, for any couple of natural integers $n$ and $\eta$, we define the distribution $\P_{M \vert Y^{n}, \epsilon^{q}}^{n, (\eta)}$, dominated by the counting measure on $\N^{\star}$ such that, for any $m$ in $\llbracket 1, n \rrbracket$
\[\P_{M \vert Y^{n}, \epsilon^{q}}^{n, (\eta)}(m) := \frac{\exp\left[\eta\left(- \pen(m) + \Upsilon(Y^{n}, \epsilon^{q}, m)\right)\right]}{\sum\limits_{k = 1}^{n} \exp\left[\eta\left(- \pen(k) + \Upsilon(Y^{n}, \epsilon^{q}, k)\right)\right]}.\]
\end{de}

With those definitions at hand, we are able to define an estimator that reproduces the structure of the posterior mean of iterated hierarchical sieves.

\begin{de}{\textsc{Aggregation/shrinkage estimator} \\}\label{DE_FREQ_CIRCDECONV_UNKNOWN_IID_AGGREGEST}
Using the notations we just introduced, we define, for any strictly positive integer $\eta$ the shrinkage/aggregation estimator $\widehat{\theta}^{(\eta)}$ such that, for any $j$ in $\mathds{Z}$
\begin{alignat*}{3}
& \widehat{\theta}^{(\eta)}_{j} && := && \P_{M \vert Y^{n}, \epsilon^{q}}^{n, (\eta)}(\llbracket \vert j \vert, n \rrbracket) \overline{\theta}^{+}_{j};\\
& \widehat{\theta}^{\eta} && := && \sum\limits_{j = 1}^{n} \P_{M \vert Y^{n}, \epsilon^{q}}^{n, (\eta)}(j) \overline{\theta}^{j, +}.
\end{alignat*}
\end{de}

As previously, one can notice that, as $\eta$ tends to infinity, this estimator converges to the penalised contrast maximiser projection estimator with penalty function $\pen$ and contrast $\Upsilon$.

In addition, this time, it is important to note that this estimator does not depend on characteristics of $\lambda$ nor $\theta^{\circ}$ and is hence fully data driven and well designed for the context of a partially unknown operator problem.

\medskip

Using the method described in \nref{FREQ_STRATEGY}, we are able to show that, for any $\theta^{\circ}$, the sequence defined hereafter is a convergence rate.

\begin{de}{\textsc{Convergence rate} \\}\label{DE_FREQ_CIRCDECONV_UNKNOWN_IID_CONVRATE}
Let be the sequences:
\[m^{\dagger}_{n} := \argmin_{m \in \N}\left\{\left[\mathfrak{b}_{m}^{2}(\theta^{\circ})\mathfrak{b}_{0}^{-2}(\theta^{\circ}) \vee 2 \frac{m \Lambda_{(m)}}{n} \psi_{n}\right]\right\};\]
and
\[\Phi^{\dagger}_{n} := \left[\mathfrak{b}_{m^{\dagger}_{n}}^{2}(\theta^{\circ})\mathfrak{b}_{0}^{-2}(\theta^{\circ}) \vee 2 \frac{m^{\dagger}_{n} \Lambda_{(m^{\dagger}_{n})}}{n} \psi_{n}\right].\]
\end{de}

\begin{te}
For each $\Di\in\Nz$ keep in mind that
$\VnormLp{\ProjC[{\mHiH}]\xdf}^2=\VnormLp{\ProjC[{\mHiH[0]}]\xdf}^2\bias[k]^2(\xdf)$,  
$\hRaDi{\Di,\xdf,\iSv}:=[\bias^2(\xdf)\vee \DipenSv\,\ssY^{-1}]$
and
introduce in addition
$\dxdfPr=\sum_{j\in\nset{-\Di,\Di}}\hfedfmpI[j]\fydf[j]\bas_j$. Note
that  $\dxdfPr=\Proj[{\mHiH[\Di]}]\dxdfPr[\ssY]$
and $\VnormLp{\ProjC[{\mHiH[\Di]}]\dxdfPr[\ssY]}^2=2\sum_{j\in\nsetlo{\Di,\ssY}}\eiSv[j]|\fydf[j]|^2$. For any $\pdDi,\mdDi\in\nset{1,\ssY}$ let us define 
\begin{multline}\label{de:au:*Di}
\mDi:=\min\set{k\in\nset{1,\mdDi}: \VnormLp{\ProjC[{\mHiH[0]}]\xdf}^2\bias[k]^2(\xdf)\leq
  [2\VnormLp{\ProjC[{\mHiH[0]}]\xdf}^2+7576\cpen]\hRaDi{\mdDi,\xdf,\iSv}}\quad\text{and}\\\pDi:=\max\set{k\in\nset{\pdDi,\ssY}:
   \epenSv[k] \leq [3\VnormLp{\ProjC[{\mHiH[\pdDi]}]\dxdfPr[\ssY]}^2+9*\epenSv[\pdDi]]}
\end{multline}
where  the defining set obviously contains $\mdDi$ and $\pdDi$, respectively, 
and hence, they are
not empty. Keep in mind that $\pDi:=\pDi(\rE_1,\dotsc,\rE_{\ssE})$ is
random but does not depend on the sample $\rY_1,\dotsc,\rY_{\ssY}$.
\end{te}

\begin{as}\label{ass:au:ub:p}
Let $\xdf$ have a finite series expansion as defined in \ref{oo:xdf:p}, that is, either
\begin{inparaenum}[i]
\renewcommand{\theenumi}{\dgrau\rm(\alph{enumi})}
\item\label{ass:au:ub:p:c1} $\xdf=\bas_0$, i.e., $\bias[0](\xdf)=\VnormLp{\Proj[{\mHiH[0]}]^\perp\xdf}^2=0$ or
\item\label{ass:au:ub:p:c2} there is $K\in\Nz$ with $1\geq \bias[{[K-1] }](\xdf)>0$ and $\bias[K](\xdf)=0$.
\end{inparaenum}
In case \ref{ass:au:ub:p:c1} set $\dr K_{\ydf}:=\ceil{15\tfrac{300^4}{\cpen^2}\vee3\tfrac{800^2}{\cpen^2}}$ while in case \ref{ass:au:ub:p:c2} given $K_{\ydf}:=K\dr\vee
3\tfrac{800^2\Vnormlp[1]{\fydf}^2}{\cpen^2}$ and $c_{\xdf}:=\tfrac{2\VnormLp{\Proj[{\mHiH[0]}]^\perp\xdf}^2+7576\cpen}{\VnormLp{\Proj[{\mHiH[0]}]^\perp\xdf}^2\bias[{[K-1]}]^2(\xdf)}$ let there $\ssY_{\xdf,\iSv},\ssE_{\xdf,\iSv}\in\Nz$ be with $\ssY_{\xdf,\iSv}>\ceil{c_{\xdf}\DipenSv[K_{\ydf}]\dr\vee15\tfrac{300^4}{\cpen^2}}$ and $\ssE_{\xdf,\iSv}>\ceil{289\log(K_{\ydf}+2)\cmiSv[K_{\ydf}]\miSv[K_{\ydf}]}$ such that $\sDi{\ssY}:=\max\{\Di\in\nset{K,\ssY}:c_{\xdf}\,\DipenSv<\ssY\}$ and $\sDi{\ssE}:=\max\{\Di\in\nset{K_{\ydf},\ssE}:289\log(\Di+2)\cmiSv\miSv\leq\ssE\}$ where the defining sets contain $K_{\ydf}$ and thus they are not empty, satisfies $\cmiSv[\sDi{\ssY}]\sDi{\ssY}\geq K_{\ydf}(\log\ssY)$ for all $\ssY\geq \ssY_{\xdf,\iSv}$ and $\cmiSv[\sDi{\ssE}]\sDi{\ssE}\geq K_{\ydf}(\log\ssE)$ for all $\ssE\geq \ssE_{\xdf,\iSv}$, respectively.
\end{as}

\begin{il}\label{il:ass:au:ub:p}
Let us illustrate \nref{ass:au:ub:p} considering as in \nref{il:oo} the commonly studied behaviours \ref{il:edf:o} and \ref{il:edf:s} for the sequence  $\Nsuite[j]{\iSv[j]}$.
\begin{Liste}[]
\item[\mylabel{il:ass:au:ub:p:o}{\dg\bfseries{(o)}}]
	Let $\iSv[\Di]\sim \Di^{2a}$, $a>0$, then $\sDi{\ssY}\cmSv[\sDi{\ssY}]\sim\ssY^{1/(2a+1)}$ (cf. \nref{il:ass:ub:p} \ref{il:ass:ub:p:o}), while $\ssE\sim(\log\sDi{\ssE})\cmSv[\sDi{\ssY}]\miSv[\sDi{\ssE}]\sim(\log\sDi{\ssE})(\sDi{\ssE})^{2a}$ implies $\sDi{\ssE}\sim(\ssE/\log\ssE)^{1/(2a)}$ and $\sDi{\ssE}\cmSv[\sDi{\ssE}]\sim (\ssE/\log\ssE)^{1/(2a)}$.
\item[\mylabel{il:ass:au:ub:p:s}{\dg\bfseries{(s)}}]
	Let $\iSv[\Di]\sim \exp(\Di^{2a})$, $a>0$, then $\sDi{\ssY}\cmSv[\sDi{\ssY}]\sim (\log \ssY)^{2+1/(2a)}$ (cf. \nref{il:ass:ub:p} \ref{il:ass:ub:p:s}), while $\ssE\sim(\log\sDi{\ssE})\cmSv[\sDi{\ssE}]\miSv[\sDi{\ssE}]\sim (\log\sDi{\ssE})(\sDi{\ssE})^{4a}\exp((\sDi{\ssE})^{2a})$ implies $\sDi{\ssE}\sim(\log \ssE-\tfrac{1+4a}{2a}\log\log\ssE-\tfrac{1}{2a}\log\log\log\ssE)^{1/(2a)}$ and $\sDi{\ssE}\cmSv[\sDi{\ssE}]\sim (\log \ssE)^{2+1/(2a)}$.
\end{Liste}
Clearly, in both cases \ref{il:ass:au:ub:p:o} and \ref{il:ass:au:ub:p:s}, there are ${\ssY}_{\xdf,\iSv},{\ssE}_{\xdf,\iSv}\in\Nz$ such that $\cmiSv[\sDi{\ssY}]\sDi{\ssY}\geq K_{\ydf}(\log\ssY)$ for all $\ssY\geq{\ssY}_{\xdf,\iSv}$ and  $\cmiSv[\sDi{\ssE}]\sDi{\ssE}\geq K_{\ydf}(\log\ssE)$ for all $\ssE\geq{\ssE}_{\xdf,\iSv}$  hold true.
\end{il}


\begin{thm}\label{THM_FREQ_CIRCDECONV_UNKNOWN_IID_ORACLE_P}
Let $\xdf$ have a finite series expansion as defined in \ref{oo:xdf:p}.
Under \nref{ass:au:ub:p} there is a finite numerical constant $\cst{}$ such that for all $\dr\ssY,\ssE\in\Nz$
\begin{equation}\label{re:au:ub:p:e1}
\FuEx[\ssY,\ssE]{\rY,\rE}\VnormLp{\hxdf-\xdf}^2
\leq\cst{}(1\vee\VnormLp{\Proj[{\mHiH[0]^\perp}]\xdf}^2)(\DipenSv[\ssY_{\xdf,\iSv}]\ssY^{-1}+K_{\ydf}\miSv[K_{\ydf}]^2\ssE^{-1}+\ssE_{\xdf,\iSv}\ssE^{-1})
\end{equation}
\end{thm}

Comparison with the oracle rate of projection estimators reveals that in many cases, we obtain an oracle optimal estimator.

\begin{il}\label{IL_FREQ_CIRCDECONV_UNKNOWN_IID_ORACLE_P}
Let us illustrate \nref{THM_FREQ_CIRCDECONV_UNKNOWN_IID_ORACLE_P} considering as in \nref{il:ass:au:ub:p} the behaviours \ref{il:edf:o} and \ref{il:edf:s} for the sequence $\Nsuite[j]{\iSv[j]}$.  
Keeping in mind that as shown in \nref{il:ass:au:ub:p} there are ${\ssY}_{\xdf,\iSv},{\ssE}_{\xdf,\iSv}\in\Nz$ such that $\cmiSv[\sDi{\ssY}]\sDi{\ssY}\geq K_{\ydf}(\log\ssY)$  for all $\ssY\geq{\ssY}_{\xdf,\iSv}$ and $\cmiSv[\sDi{\ssE}]\sDi{\ssE}\geq K_{\ydf}(\log\ssE)$  for all $\ssE\geq{\ssE}_{\xdf,\iSv}$ hold true, due to \nref{THM_FREQ_CIRCDECONV_UNKNOWN_IID_ORACLE_P} there is a constant $\cst{\xdf,\edf}$ depending only on the densities $\xdf$ and $\edf$ such that $\FuEx[\ssY,\ssE]{\rY,\rE}\VnormLp{\hxdf-\xdf}^2\leq \cst{\xdf,\edf}(\ssY^{-1}+\ssE^{-1})$ for all $\ssY,\ssE\in\Nz$.
Comparing the last result with the oracle rates derived in \nref{il:ee} we conclude, that $\hxdf$ is optimal in an oracle sense in both, the case \ref{il:oo:po} and  \ref{il:oo:so}.
\end{il}

\begin{as}\label{ass:au:ub:np}
Let $\xdf$  have an infinite series expansion as definied in \ref{oo:xdf:np}, that is, $1\geq \bias(\xdf)>0$ for all $\Di\in\Nz$.
Given   $\Di_{\ydf}:=\dr3*800^2\Vnormlp[1]{\fydf}^2\cpen^{-2}$ and $\tDi_{\ydf}=\min\{\Di\in\Nz:\bias[\Di_{\ydf}](\xdf)>\bias[\Di](\xdf)\}$ there are $\ssY_{\xdf,\iSv},\ssE_{\xdf,\iSv}\in\Nz$ with $\ssY_{\xdf,\iSv}\geq\DipenSv[\tDi_{\ydf}]\bias[\tDi_{\ydf}]^{-2}(\xdf)\vee\dr15*300^4\cpen^{-2}$ and $\ssE_{\xdf,\iSv}\geq289\log(\Di_{\ydf}+2)\cmiSv[\Di_{\ydf}]\miSv[\Di_{\ydf}]$ such that \begin{inparaenum}[i]\renewcommand{\theenumi}{\dgrau\rm(\alph{enumi})} \item\label{ass:au:ub:np:c0} for all $\ssE\geq\ssE_{\xdf,\iSv}$, $\sDi{\ssE}:=\max\{\Di\in\nset{\Di_{\ydf},\ssE}:289\log(\Di+2)\cmiSv[\Di]\miSv[\Di]\leq\ssE\}$, where the defining set containing $\Di_{\ydf}$ is not empty, satisfies $\cmiSv[\sDi{\ssE}]\sDi{\ssE}\geq \Di_{\ydf}|\log\mRa{\xdf,\iSv}|$ and  either 
\item\label{ass:au:ub:np:c1}
$\cmiSv[\aDi{\ssY}]\aDi{\ssY}\geq \Di_{\ydf}|\log\hRa{\xdf,\iSv}|$ 
for all
$\ssY\geq{\ssY}_{\xdf,\iSv}$ or \item\label{ass:au:ub:np:c2}  
$\aDi{\ssY}\leq  \Di_{\ydf}|\log\hRa{\xdf,\iSv}|$ for all
$\ssY\geq{\ssY}_{\xdf,\iSv}$. \end{inparaenum} We set 
$\sDi{\ssY}:= \ceil{\Di_{\ydf}|\log\hRa{\xdf,\iSv}|}\wedge\ssY$ for
$\ssY<\ssY_{\xdf,\iSv}$ and $\sDi{\ssY}:= \ceil{\Di_{\ydf}|\log\hRa{\xdf,\iSv}|}\vee\aDi{\ssY}$ for
$\ssY\geq\ssY_{\xdf,\iSv}$, and in addition $\sDi{\ssE}:=\sDi{\ssY}$
for $\ssE<\ssE_{\xdf,\iSv}$, where consequently in case
\ref{ass:au:ub:np:c1}  $\sDi{\ssY}= \Di_{\ydf}|\log\hRa{\xdf,\iSv}|$ for
$\ssY<\ssY_{\xdf,\iSv}$, $\sDi{\ssY}=\aDi{\ssY}$ for
$\ssY\geq\ssY_{\xdf,\iSv}$ and in case \ref{ass:au:ub:np:c2}
$\sDi{\ssY}= \Di_{\ydf}|\log\hRa{\xdf,\iSv}|$ for all $\ssY\in\Nz$.
\end{as}

\begin{il}\label{il:ass:au:ub:np}
Let us illustrate  \nref{ass:au:ub:np}
  considering as in \nref{il:oo} usual
  behaviour \ref{il:oo:oo}, \ref{il:oo:so} and \ref{il:oo:os}
 for the sequences $\Nsuite[\Di]{\bias[\Di](\xdf)}$ and
  $\Nsuite[\Di]{\iSv[\Di]}$:
 \begin{Liste}[]
\item[\mylabel{il:ass:au:ub:np:oo}{\dg\bfseries{[o-o]}}]
$\cmiSv[\aDi{\ssY}]\aDi{\ssY}\sim\ssY^{1/(2p+2a+1)}$ and
$|\log\hRa{\xdf,\iSv}|\sim(\log\ssY)$ (cf.  \nref{il:ass:ub:np}
\ref{il:ass:ub:np:oo}) while
$\sDi{\ssE}\cmSv[\sDi{\ssE}]\sim (\ssE/\log\ssE)^{1/(2a)}$ (cf.  \nref{il:ass:au:ub:p}
\ref{il:ass:au:ub:p:o})  and $|\log\mRa{\xdf,\iSv}|\sim(\log\ssE)$ (cf.  \nref{il:ee}
\ref{il:ee:oo})
 \item[\mylabel{il:ass:au:ub:np:os}{\dg\bfseries{[o-s]}}]
$\cmiSv[\aDi{\ssY}]\aDi{\ssY}\sim(\log \ssY)^{2+1/(2a)}$ and
$|\log\hRa{\xdf,\iSv}|\sim(\log\log\ssY)$ (cf.  \nref{il:ass:ub:np}
\ref{il:ass:ub:np:os}) while
$\sDi{\ssE}\cmSv[\sDi{\ssE}]\sim (\log\ssE)^{2+1/(2a)}$ (cf.  \nref{il:ass:au:ub:p}
\ref{il:ass:ub:p:s})  and $|\log\mRa{\xdf,\iSv}|\sim(\log\log\ssE)$ (cf.  \nref{il:ee}
\ref{il:ee:os})
 \item[\mylabel{il:ass:au:ub:np:so}{\dg\bfseries{[s-o]}}]
$\cmiSv[\aDi{\ssY}]\aDi{\ssY}\sim(\log \ssY)^{1/(2p)}$ and
$|\log\hRa{\xdf,\iSv}|\sim(\log\ssY)$ (cf.  \nref{il:ass:ub:np}
\ref{il:ass:ub:np:so}) while
$\sDi{\ssE}\cmSv[\sDi{\ssE}]\sim (\ssE/\log\ssE)^{1/(2a)}$ (cf.  \nref{il:ass:au:ub:p}
\ref{il:ass:au:ub:p:o}) and $|\log\mRa{\xdf,\iSv}|\sim(\log\ssE)$ (cf.  \nref{il:ee}
\ref{il:ee:so})
\end{Liste}
Clearly, there is ${\ssE}_{\xdf,\iSv}\in\Nz$ such that for all
$\ssE\geq{\ssE}_{\xdf,\iSv}$ in the three cases
\ref{il:ass:au:ub:np:oo}, \ref{il:ass:ub:np:os} and
\ref{il:ass:ub:np:so}   $\cmiSv[\sDi{\ssE}]\sDi{\ssE}\geq
\Di_{\ydf}|\log\mRa{\xdf,\iSv}|$, i.e., \nref{ass:au:ub:np}
\ref{ass:au:ub:np:c0} holds.
On the other hand side,  there is ${\ssY}_{\xdf,\iSv}\in\Nz$ such that for all
$\ssY\geq{\ssY}_{\xdf,\iSv}$ in the cases \ref{il:ass:ub:np:oo} and
\ref{il:ass:ub:np:os}   $\cmiSv[\aDi{\ssY}]\aDi{\ssY}\geq
\Di_{\ydf}|\log\hRa{\xdf,\iSv}|$, i.e., \nref{ass:au:ub:np}
\ref{ass:au:ub:np:c1} holds,   while in case \ref{il:ass:au:ub:np:so}
$\aDi{\ssY}\leq \Di_{\ydf}|\log\hRa{\xdf,\iSv}|$ for $p\geq1/2$, i.e., \nref{ass:au:ub:np}
\ref{ass:au:ub:np:c2} holds, and $\cmiSv[\aDi{\ssY}]\aDi{\ssY}\geq
\Di_{\ydf}|\log\hRa{\xdf,\iSv}|$ for $p<1/2$, i.e., \nref{ass:au:ub:np}
\ref{ass:au:ub:np:c1} holds.
\end{il}

\begin{thm}\label{THM_FREQ_CIRCDECONV_UNKNOWN_IID_ORACLE_NP}
\label{re:au:ub:np} Let $\xdf$ have an infinite series expansion
  as definied in \ref{oo:xdf:np}. Under \nref{ass:au:ub:np} there is a finite numerical
  constant $\cst{}$ such that for all $\dr\ssY,\ssE\in\Nz$
\begin{multline}\label{re:au:ub:np:e1}
\FuEx[\ssY,\ssE]{\rY,\rE}\VnormLp{\hxdfPr[]-\xdf}^2\leq
\cst{}\big\{[1\vee\VnormLp{\Proj[{\mHiH[0]^\perp}]\xdf}^2]\,\hRaDi{\sDi{\ssY}\wedge\sDi{\ssE},\xdf,\iSv}+\mRa{\xdf,\iSv}\\
\hfill+[1\vee\VnormLp{\Proj[{\mHiH[0]^\perp}]\xdf}^2](\DipenSv[\ssY_{\xdf,\iSv}]\ssY^{-1}+\ssE_{\xdf,\iSv}\ssE^{-1})+\Vnormlp[1]{\fydf}^2\ssY^{-1}+\ssE_{\xdf,\iSv}^2\ssE^{-1}\big\}.
\end{multline}
\end{thm}

Comparison with the oracle rate of projection estimators reveals that in many cases, we obtain an oracle optimal estimator.

\begin{cor}\label{COR_FREQ_CIRCDECONV_UNKNOWN_IID_ORACLE_NP}
Under the assumptions of
  \nref{re:au:ub:np}  for  $\ssY\in\Nz$ let
  $\ssE_{\ssY}:=\ssE(\ssY)\in\Nz$ such that
  $\aDi{\ssY}\leq\sDi{\ssE_{\ssY}}$. If in addition
  $\lim_{\ssY\to\infty}\cmiSv[\aDi{\ssY}]\aDi{\ssY}|\log\hRa{\xdf,\iSv}|^{-1}=\infty$, then there is a finite constant $\cst{\xdf,\edf}$ depending on the densities 
$\xdf$ and $\edf$ such that
\begin{equation*}
\FuEx[\ssY,\ssE]{\rY,\rE}\VnormLp{\hxdf-\xdf}^2\leq\cst{\xdf,\edf}(\hRa{\xdf,\iSv}+\mRa[\ssE_{\ssY}]{\xdf,\iSv})\text{
  for all } \ssY\in\Nz .
\end{equation*}
\end{cor}

\begin{il}\label{IL_FREQ_CIRCDECONV_UNKNOWN_IID_ORACLE_NP}
Let us illustrate  \nref{re:au:ub:np} considering as in \nref{il:ass:au:ub:np} usual
  behaviour  \ref{il:ass:au:ub:np:oo}, \ref{il:ass:au:ub:np:os} and \ref{il:ass:au:ub:np:so}
 for the sequences $\Nsuite[\Di]{\bias[\Di](\xdf)}$ and
  $\Nsuite[\Di]{\iSv[\Di]}$. In light of \nref{il:ass:au:ub:np}, we
  apply \nref{re:au:ub:np}, where  we need only check \nref{ass:au:ub:np}. The rates then follow by an evaluation of the upper bound. Let
  $\Nsuite[\ssY]{\ssE_{\ssY}}$ be a sequence of positive integers and
  suppose that the limits  $q_{\text{o-o}}$, $q_{\text{o-s}}$, and
$q_{\text{s-o}}$  defined in \nref{il:ee} exists in the respective cases.
\begin{Liste}[]
\item[\mylabel{il:au:ub:np:oo}{\dg\bfseries{[o-o]}}] 
Since \nref{ass:au:ub:np} \ref{ass:au:ub:np:c0} with $\sDi{\ssE}\sim
(\ssE/\log\ssE)^{1/(2a)}$ and \ref{ass:au:ub:np:c1} with
$\oDi{\ssY}\sim \ssY^{1/(2p+2a+1)}$ hold true
(cf., respectively, \nref{il:ass:au:ub:p} \ref{il:ass:au:ub:p:o} and
\nref{il:ass:ub:np} \ref{il:ass:ub:np:oo}), due to
\nref{re:au:ub:np} and \nref{il:ee} \ref{il:ee:oo}
there is a constant $\cst{\xdf,\edf}$ depending on $\xdf$ and $\edf$
such that
\begin{equation}\label{il:au:ub:np:oo:e1}\FuEx[\ssY,\ssE]{\rY,\rE}\VnormLp{\hxdf-\xdf}^2\leq\cst{\xdf,\edf}\{(\aDi{\ssY}\wedge\sDi{\ssE})^{-2p}+\ssE^{-(p\wedge
    a)/a}\},\quad\forall\;\ssY,\ssE\in\Nz.\end{equation} 
We consider two cases. Firstly, let $p> a$. If
$q_{\text{o-o}}=\lim_{\ssY\to\infty}\ssY^{2p/(2p+2a+1)}\ssE_{\ssY}^{-1}<\infty$,
then
\begin{equation*}
\frac{\aDi{\ssY}}{\sDi{\ssE}}\sim\frac{\ssY^{1/(2p+2a+1)}}{(\ssE_{\ssY}/\log\ssE_{\ssY})^{1/(2a)}}=\frac{\ssY^{1/(2p+2a+1)}}{(\ssE_{\ssY})^{1/(2p)}}\frac{(\log\ssE_{\ssY})^{1/(2a)}}{(\ssE_{\ssY})^{1/(2a)-1/(2p)}}=o(1).
\end{equation*}
This means $\aDi{\ssY}\lesssim\sDi{\ssE}$ so the resulting upper bound
is of order
$(\aDi{\ssY})^{-2p}+\ssE_{\ssY}^{-1}\lesssim(\aDi{\ssY})^{-2p}$. Suppose
now that $q_{\text{o-o}}=\infty$. If in addition
$q^b_{\text{o-o}}=\lim_{\ssY\to\infty}\aDi{\ssY}(\sDi{\ssE})^{-1}<\infty$
then the first summand in the upper bound in \eqref{il:au:ub:np:oo:e1}
reduces to $(\aDi{\ssY})^{-2p}$ and thus (keep $q_{\text{o-o}}=\infty$
in mind) the resulting upper bound
is of order $\ssE_{\ssY}^{-1}$. Now consider
$q^b_{\text{o-o}}=\infty$, then   the upper bound in
\eqref{il:au:ub:np:oo:e1} is of order $(\sDi{\ssE})^{-2p}+\ssE_{\ssY}^{-1}\lesssim\ssE_{\ssY}^{-1}$ because $p>a$. Combining both
  cases, we obtain in case $p>a$ that as
 $\ssY\to\infty$
\begin{equation*}
\FuEx[\ssY,\ssE]{\rY,\rE}\VnormLp{\hxdf-\xdf}^2=\left\{\begin{array}{ll}
O(\ssY^{-2p/(2p+2a+1)}),& \text{if }q_{\text{o-o}}<\infty,\\
O(\ssE_{\ssY}^{-1}),& \text{otherwise }.
\end{array}\right.
\end{equation*}
Now assume $p\leq a$. First, suppose 
$q^b_{\text{o-o}}=\lim_{\ssY\to\infty}\aDi{\ssY}(\sDi{\ssE})^{-1}<\infty$
then the first summand in the upper bound in \eqref{il:au:ub:np:oo:e1}
reduces to $(\aDi{\ssY})^{-2p}$ and moreover, it follows that
$q_{\text{o-o}}<\infty$. Therefore, the resulting upper bound
is of order $(\aDi{\ssY})^{-2p}$. Now consider
$q^b_{\text{o-o}}=\infty$, then   the upper bound in
\eqref{il:au:ub:np:oo:e1} is of order $(\ssE_{\ssY}/\log\ssE_{\ssY})^{-p/a}+\ssE_{\ssY}^{-p/a}\lesssim(\ssE_{\ssY}/\log\ssE_{\ssY})^{-p/a}$. Combining both
  cases, we obtain in case $p\leq a$ that as
 $\ssY\to\infty$
\begin{equation*}
\FuEx[\ssY,\ssE]{\rY,\rE}\VnormLp{\hxdf-\xdf}^2=\left\{\begin{array}{ll}
O(\ssY^{-2p/(2p+2a+1)}),& \text{if }q^b_{\text{o-o}}<\infty,\\
O((\ssE_{\ssY}/\log\ssE_{\ssY})^{-p/a}),& \text{otherwise }.
\end{array}\right.
\end{equation*}
 \item[\mylabel{il:au:ub:np:os}{\dg\bfseries{[o-s]}}]
Since \nref{ass:au:ub:np} \ref{ass:au:ub:np:c0} with $\sDi{\ssE}\sim
(\log\ssE)^{1/(2a)}$ and \ref{ass:au:ub:np:c1} with
$\oDi{\ssY}\sim (\log\ssY)^{1/(2a)}$ hold true
(cf., respectively, \nref{il:ass:au:ub:p} \ref{il:ass:au:ub:p:s} and
\nref{il:ass:ub:np} \ref{il:ass:ub:np:os}), due to
\nref{re:au:ub:np} and \nref{il:ee} \ref{il:ee:os}
there is a constant $\cst{\xdf,\edf}$ depending on $\xdf$ and $\edf$
such that
\begin{equation}\label{il:au:ub:np:os:e2}\FuEx[\ssY,\ssE]{\rY,\rE}\VnormLp{\hxdf-\xdf}^2\leq\cst{\xdf,\edf}\{(\log\ssY)^{-p/a}+(\log\ssE)^{-p/a}\},\quad\forall\;\ssY,\ssE\in\Nz.\end{equation} 
Considering $q_{\text{o-s}}=\lim_{\ssY\to\infty}(\log \ssY)(\log \ssE_{\ssY})^{-1}$  it follows that as
 $\ssY\to\infty$
\begin{equation*}
\FuEx[\ssY,\ssE]{\rY,\rE}\VnormLp{\hxdf-\xdf}^2=\left\{\begin{array}{ll}
O\big((\log\ssY)^{-p/a}\big),& \text{if }q_{\text{o-s}}<\infty,\\
O\big((\log \ssE_{\ssY})^{-p/a}\big),& \text{otherwise }.
\end{array}\right.
\end{equation*}
\item[\mylabel{il:au:ub:np:so}{\dg\bfseries{[s-o]}}] 
Since \nref{ass:au:ub:np} \ref{ass:au:ub:np:c0} with $\sDi{\ssE}\sim
(\ssE/\log\ssE)^{1/(2a)}$, \ref{ass:au:ub:np:c2} with
$\sDi{\ssY}\sim (\log\ssY)$ for $p\geq1/2$ and \ref{ass:au:ub:np:c1} with
$\sDi{\ssY}\sim (\log\ssY)^{1/(2p)}$ for $p<1/2$  hold true
(cf., respectively, \nref{il:ass:au:ub:p} \ref{il:ass:au:ub:p:o} and
\nref{il:ass:ub:np} \ref{il:ass:ub:np:so}), 
 due to
\nref{re:au:ub:np} and \nref{il:ee} \ref{il:ee:os}
there is a constant $\cst{\xdf,\edf}$ depending on $\xdf$ and $\edf$
such that
\begin{equation}\label{il:au:ub:np:os:e3}\FuEx[\ssY,\ssE]{\rY,\rE}\VnormLp{\hxdf-\xdf}^2\leq\cst{\xdf,\edf}\{\hRaDi{\sDi{\ssY}\wedge(\ssE/\log\ssE)^{1/(2a)},\xdf,\iSv}+\ssE^{-1}\},\quad\forall\;\ssY,\ssE\in\Nz.\end{equation} 
Clearly, if
$q^b_{\text{s-o}}=\lim_{\ssY\to\infty}\ssY(\sDi{\ssY})^{-(2a+1)}\ssE_{\ssY}^{-1}<\infty$
then holds $\sDi{\ssY}=(\log\ssY)^{1\vee1/(2p)}\lesssim(\ssE_{\ssY}/\log\ssE_{\ssY})^{1/(2a)}$ and hence 
$\hRaDi{\sDi{\ssY}\wedge(\ssE_{\ssY}/\log\ssE_{\ssY})^{1/(2a)},\xdf,\iSv}+\ssE_{\ssY}^{-1}\lesssim(\sDi{\ssY})^{2a+1}\ssY^{-1}$. Suppose
now that $q^b_{\text{s-o}}=\infty$, then 
\begin{multline*}
\hRaDi{\sDi{\ssY}\wedge(\ssE_{\ssY}/\log\ssE_{\ssY})^{1/(2a)},\xdf,\iSv}+\ssE_{\ssY}^{-1}\\
\hfill\lesssim(\sDi{\ssY})^{2a+1}\ssY^{-1}\vee\hRaDi{(\ssE_{\ssY}/\log\ssE_{\ssY})^{1/(2a)},\xdf,\iSv}+\ssE_{\ssY}^{-1}\\
\lesssim\exp(-(\ssE_{\ssY}/\log\ssE_{\ssY})^{p/a})\vee\ssY^{-1}(\ssE_{\ssY}/\log\ssE_{\ssY})^{-(2a+1)/(2a)}+\ssE_{\ssY}^{-1}\lesssim \ssE_{\ssY}^{-1}.
\end{multline*}
Consequently, 
it follows that as
 $\ssY\to\infty$
\begin{equation*}
\FuEx[\ssY,\ssE]{\rY,\rE}\VnormLp{\hxdf-\xdf}^2=\left\{\begin{array}{ll}
O\big(\ssY^{-1}(\log\ssY)^{(2a+1)[1\vee1/(2p)]}\big),& \text{if }q^b_{\text{o-s}}<\infty,\\
O\big(\ssE_{\ssY}^{-1}),& \text{otherwise }.
\end{array}\right.
\end{equation*}
  \end{Liste}
Comparing the last rates with the oracle rates derived in
 \nref{il:oo} \ref{il:oo:oo}, \ref{il:oo:os} and \ref{il:oo:so} we see
 that in case \ref{il:au:ub:np:oo} with $p>a$, \ref{il:au:ub:np:os} and
 \ref{il:au:ub:np:so} with $p<1/2$ $\hxdf$ attains
 the oracle rate, while in case \ref{il:au:ub:np:oo} with $p\leq a$
 and  \ref{il:au:ub:np:so} with $p\geq1/2$ the rate of the fully data-driven estimator $\hxdf$ features a detoriation  by a logarithmic factor compared to the
 oracle rate.
\end{il}
%\section{Circular deconvolution with beta mixing data and partially known noise density}
%%
\appendix
%%%
\chapter{Useful results}\label{USEFULRESULTS}
%\section{Lemmata for circular deconvolution}
%
%The following lemma, Talagrand's inequality, allows to derive upper bounds for empirical processes  both in expectation and probability and will be used throughout the proofs concerning circular density deconvolution.
%This version of the inequality can be found in \textcolor{red}{Johannes et al.}
%
%% --------------------------------------------------------------------
%% <<Lemma \ref{re:tal}>>
%% --------------------------------------------------------------------
%\begin{lm}\label{re:tal}\label{LM_TALAGRAND}
%Let $\left(Y_{p}\right)_{p \in \llbracket 1, n \llbracket}$ be independent $\mathds{Y}$-valued random variables, $\mathds{B}$ be a countable space and $S$ be a space of measurable functions from $\left(\mathds{Y}, \mathcal{Y}\right)$ to $\left(\R, \mathcal{B}(\R)\right)$ indexed by $\mathds{B}$.
%For any $t$ in $\mathds{B}$, consider $\nu_{t}$, the function belonging to $S$ indexed by $t$ and define $\overline{\nu_{t}} := \frac{1}{n} \sum_{p = 1}^{n} \nu_{t}(Y_{p}) - \mathds{E}\left[\nu_{t}(Y_{p})\right]$.
%Then, with $K := ({\sqrt{2}-1})/({21\sqrt{2}})$ and any numerical constants verifying $\lambda > 0$, and $C>0$ and $h$, $H$ and $v$ verifying
%\begin{equation*}
%	\sup_{t \in \mathds{B}}\sup_{y \in \mathds{Y}}\left\vert \nu_{t}(y)\right\vert \leq h,\qquad \E\left[\sup_{t \in \mathds{B}}\left\vert \overline{\nu_{t}}\right\vert \right]\leq H,\qquad \sup_{t \in \mathds{B}} \frac{1}{n} \sum_{p = 1}^{n} \V\left[\nu_{t}(Y_{p})\right]\leq v.
%\end{equation*}
%we have
%\begin{align}
%	 &\E\left[\left(\sup_{t \in \mathds{B}} \left\vert \overline{\nu_{t}} \right\vert^{2}-6 H^2\right)_{+}\right]\leq C \left[\frac{v}{n}\exp\left(\frac{-n H^2}{6 v}\right)+\frac{h^{2}}{n^{2}}\exp\left(\frac{-K n H}{h}\right) \right]\label{re:tal:e1} \\
%	&\P\left(\sup_{t \in \mathds{B}}\left\vert \overline{\nu_{t}} \right\vert \geq 2 H + \lambda\right) \leq 3 \exp\left[-K n \left( \frac{\lambda^{2}}{v} \wedge \frac{\lambda}{h} \right)\right]\leq 3 \left(\exp\left[\frac{- K n \lambda^{2}}{v}\right] + \exp\left[\frac{-Kn\lambda}{h}\right])\right)\label{re:tal:e2}
%\end{align}
%\end{lm}
%% --------------------------------------------------------------------
%% <<Remark Talagrand>>
%% --------------------------------------------------------------------
%In particular, we will use the specific form this lemma takes with a specific choice for $\lambda$ and with bounds for $K$ which will not influence the convergence rate, as specified in the next remark.
%\begin{rmk}
%Setting $\lambda=\sqrt{2}(\sqrt{3}-\sqrt{2}) H =\frac{(\sqrt{6}-\sqrt{4})(\sqrt{6}+\sqrt{4})}{(\sqrt{6}+\sqrt{4})}H = \frac{\sqrt{2}}{(\sqrt{3}+\sqrt{2})}H$, and hence $\sqrt{2}\sqrt{3} H = \sqrt{2}\sqrt{2}H + \sqrt{2}(\sqrt{3}-\sqrt{2}) H$ together with $K\frac{2}{(\sqrt{3}+\sqrt{2})^2}=\frac{(\sqrt{2}-1)}{(21\sqrt{2})}\frac{2}{(\sqrt{3}+\sqrt{2})^2}=\frac{(2-\sqrt{2})}{21(\sqrt{3}+\sqrt{2})^2}\geq\frac{1}{400}$
%and
%$K\frac{\sqrt{2}}{(\sqrt{3}+\sqrt{2})}=\frac{\sqrt{2}-1}{21(\sqrt{3}+\sqrt{2})}\geq\frac{1}{200}$
%and $K\geq \frac{1}{100}$ implies
%\begin{align}
%	 &\E\left[\left(\sup_{t \in \mathds{B}}\left\vert \overline{\nu_{t}}\right\vert^{2}- 6 H^{2}\right)_{+}\right] \leq C \left[\frac{v}{n}\exp\left(\frac{-n H^{2}}{6 v}\right)+\frac{h^{2}}{n^{2}}\exp\left(\frac{-n H}{100 h}\right) \right]\label{re:tal:e3} \\
%	&\P\left(\sup_{t \in \mathds{B}} \left\vert \overline{\nu_{t}} \right\vert^{2} \geq 6 H^{2}\right) \leq 3 \left(\exp\left[\frac{-n H^{2}}{400 v}\right]+\exp\left[\frac{-n H}{200 h}\right]\right).\label{re:tal:e4}
%\end{align}
%\remEnd
%\end{rmk}
%
%We show here how Talagrand's inequality may be linked to our model.
%% --------------------------------------------------------------------
%% <<Remark Talagrand>>
%% --------------------------------------------------------------------
%\begin{rmk}\label{rem:re:tal}
%Remind the notation of the unit ball for any $m$ in $\N$,
%
%$\mathds{B}_{\overline{m}}:=\left\{h \in \mathds{L}^{2} : \forall x \in \mathds{T}, \left\vert x \right\vert > m, h(x) = 0 \vee \left\Vert h \right\Vert_{L^{2}} \leq 1\right\}$.
%Note that $\mathds{B}_{\overline{m}}$ is contained in the linear subspace $\mathds{U}_{\overline{m}} = \Span\left\{ e_{j}, \left\vert j \right\vert \in \llbracket 1, m \rrbracket \right\}$.
%Defining for any $t$ in $\mathds{B}_{\overline{m}}$ the function $\nu_{t}(Y)=\sum_{|j|\in\nset{1, m}}\frac{\overline{\left[t\right]}_{j}}{\lambda_{j}} e_{j}(-Y)$ we have $\E_{\theta^{\circ}}^{n}\left[\nu_{t}(Y)\right] = \sum_{\left\vert j \right\vert \in \llbracket 1, m \rrbracket} \frac{\overline{\left[t\right]}_{j}}{\lambda_{j}} \phi_{j}$, hence, keeping in mind the notations from \nref{LM_TALAGRAND} we have $\overline{\nu_{t}}=\frac{1}{n}\sum_{p = 1}^{n}\sum_{\left\vert j \right\vert \in \llbracket 1,m \rrbracket} \frac{\overline{\left[t\right]}_{j}}{\lambda_{j}} (e_{j}(-Y_{p})-\phi_{j})$ and keeping in mind that $\overline{\left[t\right]}_{j}=\left\langle e_{j} \vert t \right\rangle_{L^{2}}$ we have
%\begin{multline*}
%\left\Vert f_{n, \overline{m}} - f_{\overline{m}} \right\Vert=\sup_{t \in \mathds{B}_{\overline{m}}}\left\vert \left\langle t \vert f_{n, \overline{m}} - f_{\overline{m}} \right\rangle_{L^{2}} \right\vert^{2}=\sup_{t \in \mathds{B}_{\overline{m}}} \left\vert \sum_{\vert j \vert \in \llbracket 1, m \rrbracket}\frac{\overline{\left[t\right]}_{j}}{\lambda_{j}} \left(\phi_{n, j} - \phi_{j}\right) \right\vert^{2}\\
%=\sup_{t \in \mathds{B}_{\overline{m}}} \left\vert \sum_{\vert j \vert \in \llbracket 1, m \rrbracket} \frac{1}{\lambda_{j}} \left\{\frac{1}{n}\sum_{p = 1}^{n}(e_{j}(-Y_{p}) - \phi_{j})\right\}\overline{\left[t\right]}_{j}\right\vert^{2} = \sup_{t \in \mathds{B}_{\overline{m}}}\left\vert\overline{\nu_{t}}\right\vert^{2}.
%\end{multline*}
%
%Note that, the unit ball $\mathds{B}_{\overline{m}}$ is not a countable set of functions, however, it contains a countable dense subset, say $\mathds{B}$, since $\mathds{L}^2$ is separable, and it is straightforward to see that $\sup_{t \in \mathds{B}_{\overline{m}}} \left\vert \overline{\nu_{t}}\right\vert^{2}=\sup_{t \in \mathds{B}} \left\vert \overline{\nu_{t}}\right\vert^2$.
%
%The last identity will be used to link the contraction of the hyper-parameter and convergence of the projection estimators to Talagrand's inequality.
%
%\remEnd
%\end{rmk}

\section{Commonly used inequalities}

\begin{lm}{\textsc{Young's inequality} \\}\label{A.3.1}
Consider $p$, $q$, and $r$, three real numbers greater than 1 such that $\frac{1}{p} + \frac{1}{q} = 1 + \frac{1}{r}$; as well as $x$ and $y$, respectively in $\mathcal{L}^{p}$ and $\mathcal{L}^{q}$.
Then,
\[\left\Vert x \star y\right\Vert_{r} \leq \Vert f \Vert_{p} \cdot \Vert g \Vert_{q}.\]
\end{lm}

\begin{lm}{\textsc{Cauchy Schwarz inequality} \\}\label{A.3.2}
Let be $x$ and $y$ in an inner product space (e.g. $\mathcal{L}^{2}$), then we have
\[\vert \langle x \vert y \rangle \vert^{2} \leq \Vert x \Vert^{2} \cdot \Vert y \Vert^{2}.\]
\end{lm}

\begin{lm}{\textsc{Hölder's inequality} \\}\label{A.3.3}
Let $p$ and $q$ be elements of $[1, \infty]$ such that $\frac{1}{p} + \frac{1}{q} = 1$. Then, for any measurable complex-valued functions $f$ and $g$,
\[\Vert f \cdot g \Vert_{L^{1}} = \Vert f \Vert_{L^{p}} \cdot \Vert g \Vert_{L^{q}}\]
\end{lm}

\begin{lm}{\textsc{Chebyshev's inequality} \\}\label{A.3.4}
Let $X$ be a random variable with finite expectation $\E[ X ]$ and finite, strictly positive variance $\V[X]$. Then for any real number $\alpha$ we have $\P(\vert X - \E[X] \vert \geq \alpha \sqrt{\V[X]}) \leq 1/ \alpha^{2}$
\end{lm}
%
\chapter{Dependent data}\label{DEPENDENTDATA}
We present in this annex definitions, and results which are used in \nref{FREQ_CIRCDECONV_KNOWN_BETA} in order to compute the convergence rate of the adaptive aggregation estimator for strictly stationary, absolutely regular process.

\begin{de}{\textsc{Strict stationarity} \\}\label{DE_DEPENDENTDATA_STRICTSTATIONARITY}
Consider a sequence of random variables $\left(Y_{p}\right)_{p \in \N}$.
We say that $\left(Y_{p}\right)_{p \in \N}$ is strictly stationary if, for any integer $q$, any finite family of integers $\left(p_{r}\right)_{r \in \llbracket 1, q\rrbracket}$, and integer $h$, $\left(Y_{p_{r}}\right)_{r \in \llbracket 1, q \rrbracket}$ is identically distributed to $\left(Y_{p_{r} + h}\right)_{r \in \llbracket 1, q \rrbracket}$. In particular, for any integers $p$ and $q$, $Y_{p}$ and $Y_{q}$ have same distribution.
\end{de}

\begin{de}{\textsc{$\beta$-mixing coefficients} \\}\label{DE_DEPENDENTDATA_BETAMIXING}
Let $\left(\Omega, \mathcal{A}, \P\right)$ be a probability space and $\mathcal{U}$ and $\mathcal{V}$ be two sub $\sigma$-algebras of $\mathcal{A}$.
Then, we define the $\beta$-mixing coefficient of $\mathcal{U}$ and $\mathcal{V}$:
\[\beta(\mathcal{U}, \mathcal{V}) := \frac{1}{2} \sup\limits_{(U_{j})_{j \in I}(V_{j})_{j \in J}}\left\{\sum\sum \vert \P(U_{j}\P(V_{k}) - \P(U_{j} \cap V_{k}) \vert \right\}\]
where the $\sup$ is taken over all possible finite partition of $\Omega$ which are respectively $\mathcal{U}$ and $\mathcal{V}$ measurable.

In addition for two random variables $Y_{1}$ and $Y_{2}$ we note $\sigma(Y_{1})$ and $\sigma(Y_{2})$ the $\sigma$-algebra they generate and $\beta\left(Y_{1}, Y_{2}\right) = \beta\left(\sigma(Y_{1}), \sigma(Y_{2})\right)$.
\end{de}

\begin{de}{\textsc{Absolutely regular process} \\}\label{DE_DEPENDENTDATA_ABSOLUTELYREGULAR}
Consider a stochastic process $(Y_{p})_{p \in \mathds{Z}}$.
Denote, for any $p$ in $\N$, by $\mathcal{F}^{-}_{p} := \sigma\left((Y_{q})_{q \leq p}\right)$ and $\mathcal{F}^{+}_{p} := \sigma\left((Y_{q})_{q \geq p}\right)$.
The stochastic process $(Y_{p})_{p \in \mathds{Z}}$ is said to be absolutely regular if
\[\lim\limits_{p \rightarrow \infty} \beta(\mathcal{F}_{0}^{-}, \mathcal{F}_{p}^{+}) = 0.\]
\end{de}

%\begin{de}{\textsc{Space $\mathcal{L}(q, \omega, \P)$} \\}\label{DE2.5.3}
%Given $q \geq 2$, a non negative sequence $\omega = \left(\omega_{p}\right)_{p \in \N}$ and a probability measure $\P$, let $\mathcal{L}(q, \omega, \P)$ be the set of functions $b : \R \rightarrow \overline{\R_{+}}$ such that there exists a sequence $(b_{p})_{p \in \N}$ of measurable functions $b_{p}: \R \rightarrow [0, 1]$ with $b_{0} = \mathds{1}$ and $\E_{\P} b_{p} \leq \omega_{p}$ satisfying $b = \sum\limits_{p = 0}^{\infty} (p + 1)^{q - 2} b_{p}$.
%
%A sufficient condition for elements of $\mathcal{L}(q, \omega, \P)$ to be non-negative $\P$-integrable functions is $\sum\limits_{j = 0}^{\infty} (j - 1)^{q-2} \omega_{j} y \infty$.
%\end{de}
%
%
%\begin{lm}\label{LMB.0.1}
%Consider a strictly stationary process $(Y_{p})_{p \in \mathds{Z}}$.
%Denote $\P_{Y}$ the common marginal distribution and $\left(\beta_{p}\right)_{p \in \N} = \left(\beta(\mathcal{F}_{0}^{-}, \mathcal{F}_{p}^{+})\right)_{p \in \N}$ the sequence of mixing coefficients with the convention $\beta_{0} = 1$.
%
%Then, there exist a function $b$ belonging to $\mathcal{L}(2, \beta, \P)$ such that for any measurable function $h$ with $\E\left[\left\vert h(\epsilon_{0})\right\vert^{2}\right] < \infty$ and any integer $n$, we have
%\[\V\left[\sum\limits_{p = 0}^{n} h(\epsilon_{p})\right] \leq 4 n \E\left[\left\vert h(\epsilon_{0})\right\vert^{2} b(\epsilon_{0})\right].\]
%
%Assuming in addition $\sum\limits_{p \in \N} \beta(\epsilon_{0}, \epsilon_{p}) < \infty$ and $\Vert h \Vert_{\infty} < \infty$ gives
%\[\E\left[\left\vert h(\epsilon_{0})\right\vert^{2} b(\epsilon_{0})\right] \leq \Vert h \Vert_{\infty} \sum\limits_{p\in \N} \beta(\epsilon_{0}, \epsilon_{p}) \leq \infty.\]
%
%Alternatively, assuming, for $r$ and $q$ exponents as in Hölder's inequality, that $\E\left[\vert b(\epsilon_{0}) \vert^{r}\right] < \infty$ and $\E\left[\vert h(\epsilon_{0})\vert^{2q}\right] < \infty$
%\[\E\left[\left\vert h(\epsilon_{0})\right\vert^{2} b(\epsilon_{0})\right] \leq \E\left[\left\vert h(\epsilon_{0})\right\vert^{2q}\right]^{1/q} \E\left[ b(\epsilon_{0})^{r}\right]^{1/r}.\]
%
%If one assumes that, for some $r$, $\omega_{p}$ tends to $0$ as $p \rightarrow \infty$ with $\omega_{0} = 1$ and such that $\sum\limits_{p \in \N} (p+1)^{r-1} \omega_{p} < \infty$ then, for each function $b$ in $\mathcal{L}(2, \omega, \P)$, the function $b^{r}$ is $\P$-integrable and $\E_{\P}\left[\vert b \vert^{r}\right] \leq p \sum\limits_{p \in \N} (p+1)^{r-1} \omega_{p}$.
%\end{lm}
%
%\begin{as}\label{ASB.0.1}
%For any integer $p$, the joint distribution $\P_{Y_{0}^{n}, Y_{p}^{n}}$ of $Y_{0}^{n}$ and $Y_{p}^{n}$ admits a density $f_{Y_{0}^{n}, Y_{p}^{n}}$ which is square integrable.
%Let $\Vert f_{Y_{0}^{n}, Y_{p}^{n}} \Vert_{L^{2}}^{2} := \int_{0}^{1} \int_{0}^{1} \vert f_{Y_{0}^{n}, Y_{p}^{n}}(x, y)\vert^{2}dx \, dy < \infty$ with a slight abuse of notations.
%If we denote further by $h \otimes g : [0, 1]^{2} \rightarrow \R$ the bivariate function $[h \otimes g](x, y) := h(x) g(y)$ then let assume $\gamma_{f} := \sup\limits_{p \geq 1} \Vert f_{Y_{0}^{n}, Y_{p}^{n}} - f \otimes f \Vert_{L^{2}} < \infty$.
%\end{as}
%
%\begin{lm}\label{LMB.0.2}
%Let $(Y_{p})_{p \in \N}$ be a strictly stationary process with associated sequence of mixing coefficients $\left(\beta(Y_{0}, Y_{p})\right)_{p \in \N}$.
%Under \nref{ASB.0.1}, for any $n \geq 1$ and $K \in \llbracket 0, n-1\rrbracket$, it holds
%\[\sum\limits_{m \leq \vert j \vert \leq l} \V\left[\sum\limits_{p = 1}^{n} e_{j}(Y_{p})\right] \leq n 2 (l-m+1) \left\{1 + 2\left[\gamma_{f} \frac{K}{\sqrt{(l-m+1)}} + 2 \sum\limits_{p = K + 1}^{n - 1} \beta(Y_{0}, Y_{p})\right]\right\}\]
%\end{lm}
%
%If we assume in addition that $\sum\limits_{p = 1}^{\infty} \beta(Y_{0}, Y_{p}) < \infty$ and $\gamma := \sup\limits_{\theta \in \Theta(\mathfrak{a}, r)} \gamma_{\theta} < \infty$ then, for any constant $C$ and any unbounded, increasing sequence $P_{n}$, there exist two integers $P^{\circ}$ and $n^{\circ}$ such that $\sum\limits_{p = P_{\circ} + 1}^{\infty} \beta(Y_{0}, Y_{p}) < C$ and $P_{n} > P^{\circ}$ for all $n$ greater than $n^{\circ}$.
%Hence, for any $n \geq n^{\circ}$
%\[\sum\limits_{m \leq \vert j \vert \leq l} \V\left[\sum\limits_{p=1}^{n} e_{j}(Y_{p})\right]\]
%

\begin{as}{\textsc{Rich space \\}}\label{AS_DEPENDENTDATA_RICHSPACE}
Assume that the universe is rich enough in the sense that there exist a sequence of random variables with uniform distribution on $[0,1]$ which is independent of $(Y_{p})_{p \in \mathds{N}}$.

As a consequence, there exist a sequence $(Y_{p}^{\perp})_{p \in \mathds{N}}$ satisfying the following properties.
For any positive integer $s$ and for any strictly positive integer $q$, define the sets $(I_{q, p}^{e})_{p \in \llbracket 1, s\rrbracket} := \llbracket 2(q-1) s + 1, (2q - 1) s\rrbracket$ and $\left(I_{q, p}^{o}\right)_{p \in \llbracket 1, s \rrbracket} := \llbracket (2q-1) s + 1, 2q s\rrbracket$.

Define for any $q$ in $\N$ the vectors of random variables $E_{q} := (Y_{I_{q, p}^{e}}^{n})_{p \in \llbracket 1, s \rrbracket}$; $O_{q} := (Y_{I_{q, p}^{o}}^{n})_{p \in \llbracket 1, s \rrbracket}$; and their counterparts $E_{q}^{\perp} := (Y_{I_{q, p}^{e}}^{n, \perp})_{p \in \llbracket 1, s \rrbracket}$ and $O_{q}^{\perp} := (Y_{I_{q, p}^{o}}^{n, \perp})_{p \in \llbracket 1, s \rrbracket}$.

Then, $\left(Y^{\perp}_{p}\right)_{p \in \N}$ satisfies:
\begin{itemize}
\item for any integer $q$, $E^{\perp}_{q}$, $E_{q}$, $O^{\perp}_{q}$, and $O_{q}$ are identically distributed;
\item for any integer $q$, $\P_{\theta^{\circ}}^{n}\left(E_{q} \neq E^{\perp}_{q}\right) \leq \beta_{s}$ and $\P_{\theta^{\circ}}^{n}\left(O_{q} \neq O^{\perp}_{q}\right) \leq \beta_{s}$;
\item $\left(E^{\perp}_{q}\right)_{q \in \N}$ are independent and identically distributed and $\left(O^{\perp}_{q}\right)_{q \in \N}$ as well.
\end{itemize}
\end{as}



%
\chapter{Proof of \textsc{\nref{THM_BAYES_IGSSM_KNOWN_IID_ORACLE_NP}}}\label{PRO_BAYES_IGSSM_KNOWN_IID_ORACLE_NP}
\subsection{Intermediate results}
To prove this result, we will apply \nref{THM_BAYES_STRATEGIES_EXPOLIM}.
Hence, we will verify \nref{AS_BAYES_STRATEGIES_EXPOLIM}.
We will take the following expressions for the sequences $G^{+}_{n}$, and $G^{-}_{n}$.
\begin{de}\label{deB.1.1}
Define the following quantities :
\begin{alignat*}{3}
& G_{n}^{-} &&:=&& \min\{m \in \llbracket 1, m_{n}^{\circ} \rrbracket : \quad \mathfrak{b}_{m}^{2}(\theta^{\circ}) \leq 9 L \Phi_{n}^{\circ}(\theta^{\circ}, \lambda)\},\\
& G_{n}^{+} &&:=&& \max \{m \in \llbracket m_{n}^{\circ}, G_{n} \rrbracket : n^{-1}( m - m_{n}^{\circ} ) \leq 3 \Lambda(m_{n}^{\circ})^{-1} \Phi_{n}^{\circ}(\theta^{\circ}, \lambda)\}.
\end{alignat*}
\assEnd
\end{de}

With this choice, we have the following results, for which the proofs are given underneath.

\begin{pr}\label{prB.1.1}
Under \textsc{\cref{AS_BAYES_GAUSS_CONTRACT_HIERARCHICAL_LAMBDA}}, we have, for all $m$ in $\llbracket 1, G_{n} \rrbracket$
\begin{alignat*}{3}
&\P[\Vert \theta_{n, \overline{m}} - \theta^{\circ} \Vert_{l^{2}}^{2} < [\mathfrak{b}_{m}^{2}(\theta^{\circ}) \vee n^{-1}m \Lambda_{\circ}(m)]/2] &&\leq&& \exp[-m/(16 L)],\\
&\P[\Vert \theta_{n, \overline{m}} - \theta^{\circ} \Vert_{l^{2}}^{2} > 4 [\mathfrak{b}_{m}^{2}(\theta^{\circ}) \vee n^{-1}m \Lambda_{\circ}(m)]] &&\leq&& \exp[-m/(9 L)].
\end{alignat*}
\reEnd
\end{pr}
This first result implies the third condition of \nref{AS_BAYES_STRATEGIES_EXPOLIM}.

\begin{pr}\label{prB.1.2}
Under \textsc{\cref{AS_BAYES_GAUSS_CONTRACT_HIERARCHICAL_LAMBDA}}, we have the following concentration inequalities for the threshold hyper parameter :
\begin{alignat*}{3}
& \sum\nolimits_{m > G^{+}_{n}} \P(\Upsilon(m, \phi_{n}) - \Upsilon(m^{\circ}_{n}, \phi_{n}) < \pen(m^{\circ}_{n}) - \pen(m)) && \leq && \exp[- 5 m_{n}^{\circ}/(9 L) + \log (G_{n})];\\
& \sum\nolimits_{m < G^{-}_{n}} \P(\Upsilon(m, \phi_{n}) - \Upsilon(m^{\circ}_{n}, \phi_{n}) < \pen(m^{\circ}_{n}) - \pen(m)) && \leq && \exp[- 4 m_{n}^{\circ}/9 + \log (G_{n})].
\end{alignat*}
\reEnd
\end{pr}
This second result implies the two remaining conditions of \nref{AS_BAYES_STRATEGIES_EXPOLIM}.

And we can hence directly apply \nref{THM_BAYES_STRATEGIES_EXPOLIM} to obtain the considered theorem.

\subsection{Detailed proofs}
\begin{pro}{\textsc{Proof of \textsc{\cref{prB.1.1}}} \\}\label{proB.1.1}
Let be $m$ in $\llbracket 1, G_{n} \rrbracket$ and note that $\Vert \theta_{n, \overline{m}} - \theta^{\circ} \Vert^{2} = \sum\nolimits_{s = 1}^{m} (\phi_{n}(s)\lambda(s)^{-1} - \theta^{\circ}(s))^{2} + \mathfrak{b}_{m}^{2}(\theta^{\circ})$.

We will use \textsc{\cref{lmA.1.1}}; therefor, define, for any $s$ in $\llbracket 1, m \rrbracket$
\begin{alignat*}{7}
& S_{m} &&:=&& \sum\nolimits_{j = 1}^{m} (\phi_{n}(s) \lambda(s)^{-1} - \theta^{\circ}(s))^{2}; \quad && && \mu_{m}&&:=&&\E[S_{m}] = n^{-1}m \Lambda_{\circ}(m); \\
& \beta(s)^{2} &&:=&& \V[\phi_{n}(s) \lambda(s)^{-1} - \theta^{\circ}(s)] = n^{-1} \Lambda(s); \quad && && v_{m} &&:=&& \sum\nolimits_{j = 1}^{m}\beta(s)^{2} = n^{-1} m \Lambda_{\circ}(m);\\
& \alpha(s)^{2} &&:=&& \E[\phi_{n}(s) \lambda(s)^{-1} - \theta^{\circ}(s)] = 0; && && t_{m} &&:=&& \max\nolimits_{j \in \llbracket 1, m \rrbracket}\beta(s)^{2} = n^{-1} \Lambda(m).
\end{alignat*}

We then control the concentration of $S_{m}$, first from above, using that, for any $a$ and $b$ in $\R_{+}$, we have $a \vee b \leq a+b$; \nref{lmA.1.1}; and \nref{AS_BAYES_GAUSS_CONTRACT_HIERARCHICAL_LAMBDA}.
We obtain
\begin{alignat*}{2}
& \P&&[S_{m} + \mathfrak{b}_{m}^{2}(\theta^{\circ}) \leq [n^{-1} m \Lambda_{\circ}(m) \vee \mathfrak{b}_{m}^{2}(\theta^{\circ})]/2 ]\\
& && \leq \P[S_{m} + \mathfrak{b}_{m}^{2}(\theta^{\circ}) \leq (n^{-1}m \Lambda_{\circ}(m) + \mathfrak{b}_{m}^{2}(\theta^{\circ}))/2 ]\\
& && \leq \P[S_{m} + \mathfrak{b}_{m}^{2}(\theta^{\circ}) \leq (2n)^{-1} m \Lambda_{\circ}(m) + \mathfrak{b}_{m}^{2}(\theta^{\circ}) ] \leq \P[S_{m} \leq (2n)^{-1} m \Lambda_{\circ}(m) ] \\
& &&\leq \P[S_{m} - \mu_{m} \leq - (2n)^{-1} m \Lambda_{\circ}(m)] \leq \exp[- (n m \Lambda_{\circ}(m))(16 n \Lambda(m))^{-1}]\\
\leq \exp[- m(16 L)^{-1}].
\end{alignat*}
Finally, we control the concentration of $S_{m}$ from bellow using that, for any $a$ and $b$ in $\R_{+}$, we have $a \vee b \geq (a+b)/2$; \nref{lmA.1.1}; and \nref{AS_BAYES_GAUSS_CONTRACT_HIERARCHICAL_LAMBDA};
We obtain
\begin{alignat*}{2}
& \P &&[S_{m} + \mathfrak{b}_{m}^{2}(\theta^{\circ}) \geq 4 [n^{-1} m \Lambda_{\circ}(m) \vee \mathfrak{b}_{m}^{2}(\theta^{\circ})] ]\\
& && \leq \P[S_{m} + \mathfrak{b}_{m}^{2}(\theta^{\circ}) \geq 2 (n^{-1} m \Lambda_{\circ}(m) + \mathfrak{b}_{m}^{2}(\theta^{\circ})) ]\\
& && \leq \P[S_{m} + \mathfrak{b}_{m}^{2}(\theta^{\circ}) \geq 2 n^{-1} m \Lambda_{\circ}(m) + \mathfrak{b}_{m}^{2}(\theta^{\circ}) ] \leq \P[S_{m} \geq 2 n^{-1} m \Lambda_{\circ}(m)]\\
& && \leq \P[S_{m} - \mu_{m} \geq n^{-1} m \Lambda_{\circ}(m)] \leq \exp[- (n m \Lambda_{\circ}(m))(9 n \Lambda(m))^{-1}] \leq \exp[- m(9 L)^{-1}].
\end{alignat*}
\proEnd
\end{pro}

\begin{pro}{\textsc{Proof for \nref{prB.1.2}} \\}\label{proB.1.2}
First, let's proof the first inequality.
Use the fact that : 
\begin{alignat*}{2}
& \sum\nolimits && _{m > G^{+}_{n}} \P(\Upsilon(m, \phi_{n}) - \Upsilon(m^{\circ}_{n}, \phi_{n}) < \pen(m^{\circ}_{n}) - \pen(m))\\
& && = \sum\nolimits_{m = G_{n}^{+} + 1}^{G_{n}} \P[0 < 3 n^{-1} (m_{n}^{\circ} - m ) + \sum\nolimits_{s = m_{n}^{\circ} + 1}^{m} \phi_{n}(s)^{2}]
\end{alignat*}
We will now use \textsc{\cref{lmA.1.1}}. For this purpose, define then for all $m$ in $\llbracket G_{n}^{+} + 1, G_{n} \rrbracket$ : $S_{m} := \sum\nolimits_{j = m_{n}^{\circ} + 1}^{m} \phi_{n}(s)^{2}$, we then have $\mu_{m} := \E[S_{m}] = n^{-1} (m- m_{n}^{\circ}) + \sum\nolimits_{s = m_{n}^{\circ} + 1}^{m} (\theta^{\circ}(s)\lambda(s))^{2}$, $\alpha(s)^{2} := \E[\phi_{n}(s)]^{2} = (\theta^{\circ}(s)\lambda(s))^{2}$ and $\beta(s)^{2} := \V[\phi_{n}(s)] = n^{-1}$.
Now, using that $\lambda$ is monotonically decreasing and $\b_{m_{n}^{\circ}}^{2}(\theta^{\circ}) \leq \Phi_{n}^{\circ}(\theta^{\circ}, \lambda)$, we note
\begin{alignat*}{3}
&\sum\nolimits_{s = m_{n}^{\circ} + 1}^{m} \alpha(s)^{2} &&=&& \sum\nolimits_{s = m_{n}^{\circ} + 1}^{m}(\theta^{\circ}(s)\lambda(s))^{2} \leq \Lambda(m_{n}^{\circ})^{-1} \sum\nolimits_{j = m_{n}^{\circ} + 1}^{m}(\theta^{\circ}(s))^{2} \\
& &&\leq&& \Lambda(m_{n}^{\circ})^{-1} \b_{m_{n}^{\circ}}^{2}(\theta^{\circ}) \leq \Lambda(m_{n}^{\circ})^{-1} \Phi_{n}^{\circ}(\theta^{\circ}, \lambda) =: r_{m};\\
&\sum\nolimits_{s = m_{n}^{\circ} + 1}^{m} \beta(s)^{2} &&=&& n^{-1}(m - m_{n}^{\circ}) =: v_{m}; \quad \max\nolimits_{j \in \llbracket m_{n}^{\circ} + 1, m \rrbracket} \beta_{j} = n^{-1} =: t_{m}
\end{alignat*}
Hence, we have, for all $m$ in $\llbracket G_{n}^{+}, G_{n}\rrbracket$
\begin{alignat*}{2}
&\P && [\sum\nolimits_{s = m_{n}^{\circ} + 1}^{m} \phi_{n}(s)^{2} - 3 n^{-1}(m - m_{n}^{\circ}) > 0] \\
& && =\P[S_{m} - n^{-1} (m - m_{n}^{\circ}) > 2 n^{-1} (m - m_{n}^{\circ})]\\
%& && = \P[S_{m} - n^{-1} (m - m_{n}^{\circ}) - \sum\nolimits_{s = m_{n}^{\circ} + 1}^{m} (\theta^{\circ}(s)\lambda(s))^{2} > 2 n^{-1} (m - m_{n}^{\circ}) - \sum\nolimits_{s = m_{n}^{\circ} + 1}^{m} (\theta^{\circ}(s)\lambda(s))^{2}]\\
& && \leq \P[S_{m} - \mu_{m} > 2 n^{-1} (m - m_{n}^{\circ}) - \Lambda(m_{n}^{\circ})^{-1} \Phi_{n}^{\circ}(\theta^{\circ}, \lambda)].
\end{alignat*}
Using the definition of $G_{n}^{+},$ we have $n^{-1} (m - m_{n}^{\circ}) > 3 \Lambda(m_{n}^{\circ})^{-1}\Phi_{n}^{\circ}(\theta^{\circ}, \lambda).$
Hence, we can write, using \textsc{\cref{AS_BAYES_GAUSS_CONTRACT_HIERARCHICAL_LAMBDA}} and \textsc{\cref{lmA.1.1}} with $c = 2/3$ :
\begin{alignat*}{2}
& \P&&[\sum\nolimits_{s = m_{n}^{\circ}}^{m} \phi_{n}(s)^{2} -  3 n^{-1} (m - m_{n}^{\circ}) > 0 ]\\
& && \leq \P[S_{m} - \mu_{m} > n^{-1} (m - m_{n}^{\circ}) + 2 \Lambda(m_{n}^{\circ})^{-1} \Phi_{n}^{\circ}(\theta^{\circ}, \lambda)]\\
& && \leq \P[S_{m} - \mu_{m} > v_{m} + 2 r_{m}] \leq \exp[- n (n^{-1} (m - m_{n}^{\circ}) + 2 \Lambda(m_{n}^{\circ})^{-1} \Phi_{n}^{\circ}(\theta^{\circ}, \lambda))/9]\\
& && \leq \exp[- n (5 \Lambda(m_{n}^{\circ})^{-1} \Phi_{n}^{\circ}(\theta^{\circ}, \lambda))/9 ] \leq \exp[- 5 m_{n}^{\circ}/(9 L)].
\end{alignat*}
Which gives
\[ \sum\nolimits_{m > G^{+}_{n}} \P(\Upsilon(m, \phi_{n}) - \Upsilon(m^{\circ}_{n}, \phi_{n}) < \pen(m^{\circ}_{n}) - \pen(m)) \leq \exp[- 5 m_{n}^{\circ}/(9 L) + \log (G_{n})]\]
Hence, the hypothesis is verified.

We now prove the second inequality.
We begin by noting that:
\begin{alignat*}{2}
& \sum\nolimits&&_{m < G^{-}_{n}} \P(\Upsilon(m, \phi_{n}) - \Upsilon(m^{\circ}_{n}, \phi_{n}) < \pen(m^{\circ}_{n}) - \pen(m))\\
& && = \sum\nolimits_{m = 1}^{G_{n}^{-}} \P [\sum\nolimits_{s = m + 1}^{m_{n}^{\circ}} \phi_{n}(s)^{2} < 3 n^{-1} (m_{n}^{\circ} - m) ].
\end{alignat*}
The \textsc{\cref{lmA.1.1}} steps in again.
Define $S_{m} := \sum\nolimits_{s = m + 1}^{m_{n}^{\circ}} \phi_{n}(s)^{2}$ and we want to control the concentration of this sum, hence we take the following notations :
\begin{alignat*}{3}
& \mu_{m} &&:=&& \E[S_{m}] = n^{-1} (m_{n}^{\circ} - m) + \sum\nolimits_{s = m + 1}^{m_{n}^{\circ}}(\theta^{\circ}(s)\lambda(s))^{2}\\
&r_{m} &&:=&& \sum\nolimits_{j = m+1}^{m_{n}^{\circ}}(\theta^{\circ}(s)\lambda(s))^{2}; \quad v_{m} := n^{-1} (m_{n}^{\circ} - m); \quad t_{m} := n^{-1}.
\end{alignat*}
Hence, we have, using \textsc{\cref{AS_BAYES_GAUSS_CONTRACT_HIERARCHICAL_LAMBDA}} and the definition of $G_{n}^{-}$
\begin{alignat*}{2}
&\P&&[S_{m} < 3 n^{-1} (m_{n}^{\circ} - m)]\\
& && = \P[S_{m} - \mu_{m} < 3 n^{-1} (m_{n}^{\circ} - m) - n^{-1} (m_{n}^{\circ} - m) - \sum\nolimits_{s = m + 1}^{m_{n}^{\circ}}(\theta^{\circ}(s)\lambda(s))^{2}]\\
% & && = \P[S_{m} - \mu_{m} < 3 n^{-1} (m_{n}^{\circ} - m) - 2 n^{-1} (m_{n}^{\circ} - m) / 3 - \sum\nolimits_{s = m + 1}^{m_{n}^{\circ}}(\theta^{\circ}(s)\lambda(s))^{2}/3 - [v_{m} + 2 r_{m}]/3]\\
%& && \leq \P[S_{m} - \mu_{m} <  - [v_{m} + 2 r_{m}]/3 + 7 n^{-1} m_{n}^{\circ}/3 - 7 n^{-1} m / 3 - \Lambda(m_{n}^{\circ})^{-1}\sum\nolimits_{s = m + 1}^{m_{n}^{\circ}}(\theta^{\circ}(s))^{2} / 3]\\
& && \leq \P[S_{m} - \mu_{m} <  - [v_{m} + 2 r_{m}]/3 + 7 n^{-1} m_{n}^{\circ}/3 + \Lambda(m_{n}^{\circ})^{-1} (\b_{m_{n}^{\circ}}^{2}(\theta^{\circ}) - \mathfrak{b}_{m}^{2}(\theta^{\circ})) / 3]\\
% & && \leq \P[S_{m} - \mu_{m} < -  [v_{m} + 2 r_{m}]/3 + 7 L \Phi_{n}^{\circ}(\theta^{\circ}, \lambda) \Lambda(m_{n}^{\circ})^{-1}/3 + \Lambda(m_{n}^{\circ})^{-1} (\Phi_{n}^{\circ}(\theta^{\circ}, \lambda) - 9 L \Phi_{n}^{\circ}(\theta^{\circ}, \lambda))/3]\\
& && \leq \P[S_{m} - \mu_{m} < - [v_{m} + 2 r_{m}]/3 + (1 - 2L)(\Phi_{n}^{\circ}(\theta^{\circ}, \lambda) \Lambda(m_{n}^{\circ})^{-1})/3 ]\\
\end{alignat*}
we now use \textsc{\cref{lmA.1.1}}
\begin{alignat*}{3}
&\P[S_{m} < 3 n^{-1} (m_{n}^{\circ} - m)] && \leq && \P[S_{m} - \mu_{m} < - [v_{m} + 2 r_{m}]/3]\\
& &&\leq&& \exp[- n(n^{-1}(m_{n}^{\circ} - m) + 2 \sum\nolimits_{s = m + 1}^{m_{n}^{\circ}}(\theta^{\circ}(s)\lambda(s))^{2})/36]\\
%& &&\leq&& \exp[- n(n^{-1} (m_{n}^{\circ} - m) + 2 \Lambda(m_{n}^{\circ})^{-1} \mathfrak{b}_{m}^{2}(\theta^{\circ}) - 2 \Lambda(m_{n}^{\circ})^{-1} \b_{m_{n}^{\circ}}^{2}(\theta^{\circ}))/36]\\
& &&\leq&& \exp[- n (16 L \Phi_{n}^{\circ}(\theta^{\circ}, \lambda) \Lambda(m_{n}^{\circ})^{-1})/36] \leq \exp [ - 4 m_{n}^{\circ}/9 ].
\end{alignat*}
And hence
\[\sum\nolimits_{m < G^{-}_{n}} \P(\Upsilon(m, \phi_{n}) - \Upsilon(m^{\circ}_{n}, \phi_{n}) < \pen(m^{\circ}_{n}) - \pen(m)) \leq\exp[- 4 m_{n}^{\circ}/9 + \log (G_{n})].\]
\proEnd
\end{pro}
\chapter{Proof of \textsc{\nref{THM_BAYES_IGSSM_KNOWN_IID_MINIMAX_NP}}}\label{PRO_BAYES_IGSSM_KNOWN_IID_MINIMAX_NP}
\subsection{Intermediate results}
Let us first consider the following intermediate results, for which the proofs are given later.

\begin{pr}\label{prB.2.1}
Under \textsc{\cref{AS_BAYES_GAUSS_CONTRACT_HIERARCHICAL_LAMBDA}}, we have, for all $m$ in $\llbracket 1, G_{n} \rrbracket$ and $c$ greater than $3/2$,
\[\P[\Vert \theta_{n, \overline{m}} - \theta^{\circ} \Vert_{l^{2}}^{2} > 4 c [\b_{m}^{2}(\theta^{\circ}) \vee n^{-1}(m \Lambda_{\circ}(m))]] \leq \exp[-c m(6 L)^{-1}].\]
\reEnd
\end{pr}

\begin{de}\label{deB.2.1}
Define the following quantities :
\begin{alignat*}{3}
&G_{n}^{\star-} && := && \min\{m \in \llbracket 1, m_{n}^{\star} \rrbracket : \quad \b_{m}^{2}(\theta^{\circ}) \leq 9 (1 \vee r) L \Phi_{n}^{\star}(\mathfrak{a}, \lambda)\},\\
& G_{n}^{\star+} && := && \max \{m \in \llbracket m_{n}^{\star}, G_{n} \rrbracket : n^{-1}(m - m_{n}^{\star}) \leq 3 \Lambda(m_{n}^{\star})^{-1} (1 \vee r ) \Phi_{n}^{\star}(\mathfrak{a}, \lambda)\}.
\end{alignat*}
\assEnd
\end{de}

\begin{pr}\label{prB.2.2}
Under \textsc{\cref{AS_BAYES_GAUSS_CONTRACT_HIERARCHICAL_LAMBDA}}, we have the following concentration inequalities for the threshold hyper parameter :
\begin{alignat*}{3}
& \P[\widehat{m} > G_{n}^{\star+}] && \leq && \exp[- (9L)^{-1}5 (1 \vee r) m_{n}^{\star} + \log (G_{n})],\\
& \P[\widehat{m} < G_{n}^{\star-}] && \leq && \exp[- 7 (1 \vee r) m_{n}^{\star} / 9 + \log (G_{n})].
\end{alignat*}
\reEnd
\end{pr}

Then the proof of the theorem goes as follows.
\begin{pro}{\textsc{Proof of \nref{THM_BAYES_IGSSM_KNOWN_IID_MINIMAX_NP}} \\}\label{proB.2.3}
By the the total probability formula, we have :
\[ \E[\P_{\boldsymbol{\theta}_{\overline{M}}\vert \phi_{n}}^{(\infty)}(\Vert \boldsymbol{\theta}_{\overline{M}} - \theta^{\circ} \Vert_{l^{2}}^{2} \leq K^{\star} \Phi_{n}^{\star}(\mathfrak{a}, \lambda))] = 1 - \E[\P_{\boldsymbol{\theta}_{\overline{M}}\vert \phi_{n}}^{(\infty)}(K^{\star} \Phi_{n}^{\star}(\mathfrak{a}, \lambda) < \Vert \boldsymbol{\theta}_{\overline{M}} - \theta^{\circ} \Vert_{l^{2}}^{2})].\]
Hence, we will control $\E[\P_{\boldsymbol{\theta}_{\overline{M}}\vert \phi_{n}}^{(\infty)}(K^{\star} \Phi_{n}^{\star}(\mathfrak{a}, \lambda) < \Vert \boldsymbol{\theta}_{\overline{M}} - \theta^{\circ} \Vert_{l^{2}}^{2})]$.

We can write :
\begin{alignat*}{2}
& \E && [\P_{\boldsymbol{\theta}_{\overline{M}}\vert \phi_{n}}^{(\infty)}(K^{\star} \Phi_{n}^{\star}(\mathfrak{a}, \lambda) < \Vert \boldsymbol{\theta}_{\overline{M}} - \theta^{\circ} \Vert_{l^{2}}^{2})]\\
& && = \sum\nolimits_{m = 1}^{G_{n}} \E[\P_{\boldsymbol{\theta}_{\overline{M}} \vert \phi_{n}}^{(\infty)}(\{K^{\star} \Phi_{n}^{\star}(\mathfrak{a}, \lambda) < \Vert \boldsymbol{\theta}_{\overline{M}} - \theta^{\circ} \Vert_{l^{2}}^{2}\} \cap \{M = m\})]\\
& && \leq \sum\nolimits_{m = 1}^{G_{n}^{\star-} - 1} \E[\P_{M \vert \phi_{n}}^{(\infty)}(\{M = m\})] + \sum\nolimits_{m = G_{n}^{\star+} + 1}^{G_{n}} \E[\P_{\boldsymbol{\theta}_{\overline{M}} \vert \phi_{n}{n}}^{(\infty)}(\{M = m\})]\\
& && + \sum\nolimits_{m = G_{n}^{\star-}}^{G_{n}^{\star+}} \E[\P_{\boldsymbol{\theta}_{\overline{M}}\vert \phi_{n}, M = m}^{(\infty)}(\{K^{\star} \Phi_{n}^{\star}(\mathfrak{a}, \lambda) < \Vert \boldsymbol{\theta}_{\overline{M}} - \theta^{\circ} \Vert_{l^{2}}^{2}\})]\\
& && \leq \underbrace{\sum\nolimits_{m = 1}^{G_{n}^{\star-} - 1} \P[\{\widehat{m} = m\}]}_{=: A} + \underbrace{\sum\nolimits_{m = G_{n}^{\star+} + 1}^{G_{n}} \P[\{\widehat{m} = m\}]}_{=: B}\\
& && + \sum\nolimits_{m = G_{n}^{\star-}}^{G_{n}^{\star+}} \underbrace{\P[\{K^{\star} \Phi_{n}^{\star}(\mathfrak{a}, \lambda) < \Vert \theta_{n, \overline{m}} - \theta^{\circ} \Vert_{l^{2}}^{2}\}]}_{=: C_{m}}.
\end{alignat*}

\medskip

We control $A$ and $B$ using \textsc{\cref{prB.2.2}}.
Hence, we now control $\sum\nolimits_{m = G_{n}^{\star-}}^{G_{n}^{\star+}} C_{m}$.
Using \nref{AS_BAYES_GAUSS_CONTRACT_HIERARCHICAL_MINIMAX} we have
\[[a_{m_{n}^{\star}} \wedge n^{-1} m_{n}^{\star} \Lambda(m_{n}^{\star}) ] \leq \Phi_{n}^{\star}(\mathfrak{a}, \lambda) \leq  (\kappa^{\star})^{-1}[a_{m_{n}^{\star}} \wedge n^{-1} m_{n}^{\star} \Lambda(m_{n}^{\star})].\]
Hence, for any $m$ in $\llbracket m_{n}^{\star}, G_{n}^{\star+} \rrbracket$ we have, using the definition of $G_{n}^{\star +}$
\begin{alignat*}{3}
& m && \leq && 3 \Lambda(m_{n}^{\star})^{-1} (1 \vee r) \Phi_{n}^{\star}(\mathfrak{a}, \lambda) n + m_{n}^{\star} \leq 3 (1 \vee r) n (\Lambda(m_{n}^{\star}) \kappa^{\star})^{-1} [\mathfrak{a}_{m_{n}^{\star}} \wedge n^{-1} m_{n}^{\star} \Lambda_{\circ}(m_{n}^{\star})] + m_{n}^{\star}\\
& && \leq && 3 (1 \vee r)(\kappa^{\star})^{-1} m_{n}^{\star} \Lambda_{\circ}(m_{n}^{\star}) \Lambda(m_{n}^{\star})^{-1} + m_{n}^{\star} \leq ( 3 (1 \vee r)(\kappa^{\star} L)^{-1} + 1) m_{n}^{\star} \leq D^{\star} m_{n}^{\star};
\end{alignat*}
and $\Lambda_{\circ}(m) \leq \Lambda(m) \leq \Lambda(D^{\star} m_{n}^{\star}) \leq \Lambda(D^{\star}) \Lambda(m_{n}^{\star}) \leq \Lambda(D^{\star}) L \Lambda_{\circ}(m_{n}^{\star})$;
which give together
\[n^{-1} m \Lambda_{\circ}(m) \leq D^{\star} \Lambda(D^{\star}) L n^{-1} m_{n}^{\star} \Lambda(m_{n}^{\star}) \leq D^{\star} \Lambda(D^{\star}) L \Phi_{n}^{\star}(\mathfrak{a}, \lambda);\]
moreover, we have $\b_{m}^{2}(\theta^{\circ}) \leq \b_{m_{n}^{\star}}^{2}(\theta^{\circ})^{2}(\theta^{\circ}) \leq \Phi_{n}^{\star}(\mathfrak{a}, \lambda)$, which leads to the conclusion
\[[\b_{m}^{2}(\theta^{\circ}) \vee n^{-1} m \Lambda_{\circ}(m) ] \leq D^{\star} \Lambda(D^{\star}) ( 1 \vee r) \Phi_{n}^{\star}(\mathfrak{a}, \lambda).\]
On the other hand, for and $m$ in $\llbracket G_{n}^{\star -}, m_{n}^{\star} \rrbracket$, the definition of $G_{n}^{\star -}$ directly gives us
\[[\b_{m}^{2}(\theta^{\circ}) \vee n^{-1} m \Lambda_{\circ}(m) ] \leq 9 L (1 \vee r) \Phi_{n}^{\star}(\mathfrak{a}, \lambda).\]
Using \textsc{\cref{prB.2.1}}, we have that, for all $m$ in $\llbracket G_{n}^{\star-}, G_{n}^{\star+} \rrbracket$ and $c$ greater than $3/2$ :
\[\P[\{\Vert \theta_{n, \overline{m}} - \theta^{\circ}\Vert_{l^{2}}^{2} > 4c (9 L \vee D^{\star} \Lambda(D^{\star})) (1 \vee r) \Phi_{n}^{\star}(\mathfrak{a}, \lambda)\}] \leq \exp[-c m/(6 L)].\]
Hence, we set $K^{\star} := 6 (9L \vee D^{\star} \Lambda(D^{\star})) (1 \vee r),$ which leads us to the upper bound:
\[\sum\nolimits_{m = G_{n}^{\star-}}^{G_{n}^{\star+}}C_{m} \leq \sum\nolimits_{m = G_{n}^{\star-}}^{G_{n}^{\star+}} \exp[- 3 m/(8 L)] \leq 4 L \exp[-G_{n}^{\star-}(4 L)].\]
Finally, we can conclude :
\begin{alignat*}{2}
& \E && [\P_{\boldsymbol{\theta}_{\overline{M}}\vert \phi_{n}}^{(\infty)}(\Vert \boldsymbol{\theta}_{\overline{M}} - \theta^{\circ} \Vert_{l^{2}}^{2} \leq K^{\star} \Phi_{n}^{\star}(\mathfrak{a}, \lambda))]\\
& && \geq 1 - \exp[- (5 / (9 L)) (1 \vee r) m_{n}^{\star} + \log(G_{n})] - \exp[- (7/9) (1 \vee r) m_{n}^{\star} + \log(G_{n})]\\
& && - 4 L \exp[- G_{n}^{\star-}/(4 L)].
\end{alignat*}
This proves the first part of the theorem for any $\theta^{\circ}$ such that $G_{n}^{\star-}$ tends to infinity when $n$ tends to $\infty$.
In the opposite case, it means that there exist $n^{\circ}$ such that for all $n$ larger than $n^{\circ}$, $G_{n}^{\star-} = G_{n^{\circ}}^{\star-}.$
This means that $n \mapsto \b_{G_{n}^{\star-}}$ is constant function for $n$ larger that $n^{\circ}$ but, by definition of $G_{n}^{\star-}$, we also have $\b_{G_{n}^{\star-}} \leq 9 (1 \vee r) L \Phi_{n}^{\star}(\mathfrak{a}, \lambda) \rightarrow 0$ which leads to the conclusion that for all $m$ greater than $G_{n^{\circ}}^{\star-}$, $\b_{m}^{2}(\theta^{\circ}) = 0.$
Hence, for all $m$ greater than $G_{n^{\circ}}^{\star-}$, we can write
\begin{alignat*}{2}
& K^{\star} && \cdot \Phi_{n}^{\star}(\mathfrak{a}, \lambda) [\b_{m}^{2}(\theta^{\circ}) \vee n^{-1} m \Lambda_{\circ}(m) ]^{-1} = K^{\star} \Phi_{n}^{\star}(\mathfrak{a}, \lambda) [n^{-1} m \Lambda_{\circ}(m) ]^{-1}\\
& && \geq K^{\star} n^{-1}m_{n}^{\star} \Lambda_{\circ}(m_{n}^{\star}) [n^{-1} m \Lambda_{\circ}(m) ]^{-1} \geq K^{\star} m_{n}^{\star} (L m)^{-1} \geq 9 D^{\star} \Lambda(D^{\star}) (1 \vee r) m^{-1} m_{n}^{\star} \geq 1.
\end{alignat*}
Hence, we set $c := \frac{9}{4} D^{\star} \Lambda(D^{\star}) (1 \vee r) \frac{m_{n}^{\star}}{m}$ and can write in this case for all $n$ larger than $n^{\circ}$:
\begin{alignat*}{2}
& \sum\nolimits&&_{m = G_{n}^{\star-}}^{G_{n}^{\star+}} \P[K^{\star} \Phi_{n}^{\star}(\mathfrak{a}, \lambda) < \Vert \theta_{n, \overline{m}} - \theta^{\circ} \Vert_{l^{2}}^{2}]\\
& && \leq \sum\nolimits_{m = G_{n}^{\star-}}^{G_{n}^{\star+}} \P[ 4c [\b_{m}^{2}(\theta^{\circ}) \vee n^{-1} m \Lambda_{\circ}(m)] < \Vert \theta_{n, \overline{m}} - \theta^{\circ} \Vert_{l^{2}}^{2}]\\
& && \leq \sum\nolimits_{m = G_{n}^{\star-}}^{G_{n}^{\star+}} \exp[- c m / (6 L) ] \leq \sum\nolimits_{m = G_{n}^{\star-}}^{G_{n}^{\star+}} \exp[- 3/(8 L) D^{\star} \Lambda(D^{\star}) (1 \vee r) m_{n}^{\star}] \\
& && \leq \exp[-3/(8 L) D^{\star} \Lambda(D^{\star}) (1 \vee r ) m_{n}^{\star} + \log(G_{n})].
\end{alignat*}
We can hence conclude that
\begin{alignat*}{2}
& \E&&[\P_{\boldsymbol{\theta}_{\overline{M}}\vert \phi_{n}}^{(\infty)}(\Vert \boldsymbol{\theta}_{\overline{M}} - \theta^{\circ} \Vert_{l^{2}}^{2} \leq K^{\star} \Phi_{n}^{\star}(\mathfrak{a}, \lambda))]\\
& && > 1 - \exp[-3/(8 L) D^{\star} \Lambda(D^{\star}) (1 \vee r) m_{n}^{\star} + \log(G_{n})]\\
& &&- \exp[- 5 m_{n}^{\star} / (9 L) + \log(G_{n})] - \exp[- (7/9) m_{n}^{\star} + \log(G_{n})].
\end{alignat*}
Hence, we have shown here that $\Phi_{n}^{\star}(\mathfrak{a}, \lambda)$ is an upper bound for the contraction rate under the quadratic risk.
We will now use this to prove that it is also for the maximal risk.
Note that $K^{\star} \, \Phi_{n}^{\star}(\mathfrak{a}, \lambda) \geq 4 [\b_{m}^{2}(\theta^{\circ}) \vee n^{-1} m \Lambda_{\circ}(m)]$ for all $m$ in $\llbracket G_{n}^{\star-}, G_{n}^{\star+} \rrbracket$.
Hence, for any increasing function $K_{n}$ such that $\lim\nolimits_{n \rightarrow \infty} K_{n} = \infty,$ we have
\[K_{n} \Phi_{n}^{\star}(\mathfrak{a}, \lambda) \geq 4 K_{n} (K^{\star})^{-1} [\b_{m}^{2}(\theta^{\circ}) \vee n^{-1} m \Lambda_{\circ}(m) ].\]
So, if we define $\tilde{n}^{\circ},$ the smallest integer such that $K_{n}(K^{\star})^{-1} \geq 1$, we can apply \textsc{\cref{prB.2.1}} and we have:

\begin{alignat*}{2}
& \sum\nolimits&&_{m = G_{n}^{\star -}}^{G_{n}^{\star +}} \P[K_{n} \Phi_{n}^{\star}(\mathfrak{a}, \lambda) < \Vert \theta_{n, \overline{m}} - \theta^{\circ} \Vert_{l^{2}}^{2}]\\
& && \leq \sum\nolimits_{m = G_{n}^{\star-}}^{G_{n}^{\star+}} \P[4 K_{n}(K^{\star})^{-1} [\b_{m}^{2}(\theta^{\circ}) \vee n^{-1}m \Lambda_{\circ}(m) ] < \Vert \theta_{n, \overline{m}} - \theta^{\circ} \Vert_{l^{2}}^{2}] \\
& && \leq \sum\nolimits_{m = G_{n}^{\star-}}^{G_{n}^{\star+}} \exp[-4 K_{n} m(9 K^{\star} L)^{-1}] \leq \exp[-4 K_{n}(9 K^{\star} L)^{-1}].
\end{alignat*}
We hence here have a uniform upper bound for the maximal risk which concludes the proof.
\proEnd
\end{pro}

\subsection{Detailed proofs}
\begin{pro}{\textsc{Proof of \nref{prB.2.1}} \\}\label{proB.2.1}
Let be $m$ in $\llbracket 1, G_{n} \rrbracket$ and note that
$\Vert \theta_{n, \overline{m}} - \theta^{\circ} \Vert_{l^{2}}^{2} = \sum\nolimits_{s = 1}^{m} (\phi_{n}(s)\lambda(s)^{-1} - \theta^{\circ}(s))^{2} + \b_{m}^{2}(\theta^{\circ})$, 
hence, we will use \textsc{\cref{lmA.1.1}}.
We then define for any $s$ in $\llbracket 1, m \rrbracket$
\begin{alignat*}{4}
& S_{m} && := && \sum\nolimits_{s = 1}^{m} (\phi_{n}(s)\lambda(s)^{-1} - \theta^{\circ}(s))^{2}; \quad && \mu_{m} := \E[S_{m}] = n^{-1} m \Lambda_{\circ}(m),\\
& \beta(s)^{2} && := && \V[\phi_{n}(s)\lambda(s)^{-1} - \theta^{\circ}(s)] = n^{-1}\Lambda(s); \quad && v_{m} := \sum\nolimits_{s = 1}^{m}\beta(s)^{2} = n^{-1} m \Lambda_{\circ}(m);\\
& \alpha(s)^{2} && := && \E[\phi_{n}(s)\lambda(s)^{-1} - \theta^{\circ}(s)] = 0; \quad && t_{m} := \max\nolimits_{s \in \llbracket 1, m \rrbracket}\beta(s)^{2} = n^{-1}\Lambda(m).
\end{alignat*}

We then control the concentration of $S_{m}$, define $c$ a constant greater than $3/2$.
Using that, for any $a$ and $b$ in $\R_{+}$ we have $a \vee b \geq (a + b)/2$; $c > 1$; and \nref{lmA.1.1}, we obtain
\begin{alignat*}{2}
& \P&&[S_{m} + \b_{m}^{2}(\theta^{\circ}) \geq 4 c (n^{-1} m \Lambda_{\circ}(m) \vee \b_{m}^{2}(\theta^{\circ})) ]\\
& && \leq \P[S_{m} + \b_{m}^{2}(\theta^{\circ}) \geq 2 c n^{-1} m \Lambda_{\circ}(m) + \b_{m}^{2}(\theta^{\circ}) ] \leq \P[S_{m} \geq 2 c n^{-1} m \Lambda_{\circ}(m) ]\\
& && \leq \P[S_{m} - \mu_{m} \geq c n^{-1} m \Lambda_{\circ}(m) ] \leq \exp[- c n m \Lambda_{\circ}(m) (6 n \Lambda(m))^{-1}] \leq \exp[- c m / (6 L)];
\end{alignat*}
which proves the claim
\proEnd
\end{pro}

\begin{pro}{\textsc{Proof of \textsc{\cref{prB.2.2}}} \\}
First, let's proof the first inequality.
Use the fact that : 
\begin{alignat*}{2}
& \P&&[G_{n}^{\star +} < \widehat{m} \leq G_{n}] \\
& &&= \P [\forall l \in \llbracket 1, G_{n}^{\star+}\rrbracket, \quad n^{-1} 3 \widehat{m} - \sum\nolimits_{s=1}^{\widehat{m}} \phi_{n}(s)^{2} < n^{-1} 3 l - \sum\nolimits_{s=1}^{l} \phi_{n}(s)^{2} ]\\
& && \leq \P[\exists m \in \llbracket G_{n}^{\star+} + 1, G_{n}\rrbracket : \quad n^{-1} 3 m - \sum\nolimits_{s=1}^{m} \phi_{n}(s)^{2} < n^{-1} 3 m_{n}^{\star} - \sum\nolimits_{s=1}^{m_{n}^{\star}} \phi_{n}(s)^{2} ]\\
& && \leq \sum\nolimits_{m = G_{n}^{\star+} + 1}^{G_{n}} \P[n^{-1} 3 m - \sum\nolimits_{s=1}^{m} \phi_{n}(s)^{2} < n^{-1} 3 m_{n}^{\star} - \sum\nolimits_{s=1}^{m_{n}^{\star}} \phi_{n}(s)^{2}]\\
& && \leq \sum\nolimits_{m = G_{n}^{\star+} + 1}^{G_{n}} \P[0 <  3 n^{-1} m_{n}^{\star} - m + \sum\nolimits_{s= m_{n}^{\star} + 1}^{m} \phi_{n}(s)^{2}]
\end{alignat*}
We will now use \textsc{\cref{lmA.1.1}}. For this purpose, define then for all $m$ in $\llbracket G_{n}^{\star+} + 1, G_{n} \rrbracket$ : $S_{m} := \sum\nolimits_{s= m_{n}^{\star} + 1}^{m} \phi_{n}(s)^{2}$, we then have $\mu_{m} := \E[S_{m}] = n^{-1} m- m_{n}^{\star} + \sum\nolimits_{s= m_{n}^{\star} + 1}^{m} \phi(s)^{2}$, $\alpha(s)^{2} := \E[\phi_{n}(s)]^{2} =  \phi(s)^{2}$ and $\beta(s)^{2} := \V[\phi_{n}(s)] = n^{-1}$.
Now we note, using the definition of $\Theta(\mathfrak{a}, r)$
\begin{alignat*}{5}
& \sum\nolimits_{s= m_{n}^{\star} + 1}^{m} \alpha(s)^{2} && = && \sum\nolimits_{s= m_{n}^{\star} + 1}^{m}\phi(s)^{2} \leq \Lambda(m_{n}^{\star})^{-1} \sum\nolimits_{s= m_{n}^{\star} + 1}^{m}(\theta^{\circ}(s))^{2}\\
& && \leq && \Lambda(m_{n}^{\star})^{-1} \b_{m_{n}^{\star}}^{2}(\theta^{\circ}) \leq \Lambda(m_{n}^{\star})^{-1} ( 1 \vee r) \Phi_{n}^{\star}(\mathfrak{a}, \lambda) && =: && r_{m};\\
& \sum\nolimits_{s= m_{n}^{\star} + 1}^{m} \beta(s)^{2} && = && n^{-1} (m - m_{n}^{\star}) =: v_{m}; \quad \max\nolimits_{j \in \llbracket m_{n}^{\star}, m \rrbracket} \beta(s) = n^{-1} && =: && t_{m}.
\end{alignat*}

Hence, we have, for all $m$ in $\llbracket G_{n}^{\star+} + 1, G_{n}\rrbracket$
\begin{alignat*}{2}
& \P&&[\sum\nolimits_{s= m_{n}^{\star} + 1}^{m} \phi_{n}(s)^{2} - 3 n^{-1} (m - m_{n}^{\star}) > 0] = \P[S_{m} - n^{-1}(m - m_{n}^{\star}) > 2 n^{-1}(m - m_{n}^{\star}) ]\\
& && = \P[S_{m} - n^{-1}(m - m_{n}^{\star}) - \sum\nolimits_{s= m_{n}^{\star} + 1}^{m} \phi(s)^{2} > 2 n^{-1}(m - m_{n}^{\star}) - \sum\nolimits_{s= m_{n}^{\star} + 1}^{m} \phi(s)^{2}]\\
& && \leq \P[S_{m} - \mu_{m} > 2 n^{-1}(m - m_{n}^{\star}) - \Lambda(m_{n}^{\star})^{-1} \b_{m_{n}^{\star}}^{2}(\theta^{\circ})]\\
& && \leq \P[S_{m} - \mu_{m} > 2 n^{-1}(m - m_{n}^{\star}) - \Lambda(m_{n}^{\star})^{-1} (1 \vee r) \Phi_{n}^{\star}(\mathfrak{a}, \lambda) ].
\end{alignat*}
Using the definition of $G_{n}^{\star+},$ we have $n^{-1}(m - m_{n}^{\star}) > 3 \Lambda(m_{n}^{\star})^{-1}(1 \vee r) \Phi_{n}^{\star}(\mathfrak{a}, \lambda)$.
Hence, we can write :
\begin{alignat*}{2}
& \P&&[\sum\nolimits_{s= m_{n}^{\star}}^{m} \phi_{n}(s)^{2} - 3 n^{-1}(m - m_{n}^{\star}) > 0]\\
& &&\leq \P[S_{m} - \mu_{m} > n^{-1}(m - m_{n}^{\star}) + 2 \Lambda(m_{n}^{\star})^{-1} (1 \vee r ) \Phi_{n}^{\star}(\mathfrak{a}, \lambda)] \leq \P[S_{m} - \mu_{m} > v_{m} + 2 r_{m}]\\
& &&\leq \exp[- n (n^{-1} (m - m_{n}^{\star}) + 2 \Lambda(m_{n}^{\star})^{-1} (1 \vee r) \Phi_{n}^{\star}(\mathfrak{a}, \lambda))/9 ]\\
& &&\leq \exp[- n (5 \Lambda(m_{n}^{\star})^{-1} ( 1 \vee r) \Phi_{n}^{\star}(\mathfrak{a}, \lambda))/9] \leq \exp[- 5 ( 1 \vee r) m_{n}^{\star}/(9 L)].
\end{alignat*}
Finally we can conclude that
\[\P[G_{n}^{\star+} < \widehat{m} \leq G_{n}] \leq \exp[- 5 ( 1 \vee r) m_{n}^{\star}/(9 L) + \log(G_{n})].\]
We now prove the second inequality.
We begin by writing the same kind of inclusion of events as for the first inequality :
\begin{alignat*}{2}
& \P &&[1 \leq \widehat{m} < G_{n}^{\star-}]\\
& &&= \P [ \forall m \in \llbracket G_{n}^{-}, G_{n} \rrbracket, \quad 3 n^{-1} \widehat{m} - \sum\nolimits_{s= 1}^{\widehat{m}} \phi_{n}(s)^{2} < 3 n^{-1}m - \sum\nolimits_{s= 1}^{m} \phi_{n}(s)^{2}]\\
& && \leq \P [ \exists m \in \llbracket 1, G_{n}^{\star-} -1 \rrbracket, \quad 3 n^{-1}m - \sum\nolimits_{s= 1}^{m} \phi_{n}(s)^{2} < 3 n^{-1} m_{n}^{\star} - \sum\nolimits_{s= 1}^{m_{n}^{\star}} \phi_{n}(s)^{2}]\\
& && \leq \sum\nolimits_{m = 1}^{G_{n}^{\star-}} \P [3 n^{-1}m - \sum\nolimits_{s= 1}^{m} \phi_{n}(s)^{2} < 3 n^{-1}m_{n}^{\star} - \sum\nolimits_{s= 1}^{m_{n}^{\star}} \phi_{n}(s)^{2}]\\
& && \leq \sum\nolimits_{m = 1}^{G_{n}^{\star-}} \P[\sum\nolimits_{s= m + 1}^{m_{n}^{\star}} \phi_{n}(s)^{2} < 3 n^{-1}(m_{n}^{\star} - m) ].
\end{alignat*}
The \textsc{\cref{lmA.1.1}} steps in again, define $S_{m} := \sum\nolimits_{s= m + 1}^{m_{n}^{\star}} \phi_{n}(s)^{2}$ and we want to control the concentration of this sum, hence we take the following notations :
\begin{alignat*}{3}
& \mu_{m} && := && \E[S_{m}] = n^{-1} (m_{n}^{\star} - m) + \sum\nolimits_{s= m+1}^{m_{n}^{\star}}\phi(s)^{2};\\
& r_{m} && := && \sum\nolimits_{s= m+1}^{m_{n}^{\star}}\phi(s)^{2}; \quad v_{m} := n^{-1} (m_{n}^{\star} - m); \quad t_{m} := n^{-1}.
\end{alignat*}
Hence, we have
\begin{alignat*}{2}
& \P && [S_{m} < 3 n^{-1} (m_{n}^{\star} - m) ] \\
& && = \P[S_{m} - \mu_{m} < 3 n^{-1} (m_{n}^{\star} - m) - n^{-1} (m_{n}^{\star} - m) - \sum\nolimits_{s= m+1}^{m_{n}^{\star}}\phi(s)^{2}]\\
% & && = \P[S_{m} - \mu_{m} < 3 n^{-1} (m_{n}^{\star} - m) - (2/3) n^{-1} (m_{n}^{\star} - m) - \sum\nolimits_{s= m+1}^{m_{n}^{\star}}\phi(s)^{2}/3 - [v_{m} + 2 r_{m}]/3]\\
& && \leq \P[S_{m} - \mu_{m} < (7/3) n^{-1}(m_{n}^{\star} - m) - (1/3) \Lambda(m_{n}^{\star})^{-1}\sum\nolimits_{s= m+1}^{m_{n}^{\star}}(\theta^{\circ}(s))^{2} - [v_{m} + 2 r_{m}]/3]\\
% & && \leq \P[S_{m} - \mu_{m} <  -  [v_{m} + 2 r_{m}]/3 + (7/3) n^{-1} m_{n}^{\star} + (1/3) \Lambda(m_{n}^{\star})^{-1}\b_{m_{n}^{\star}}^{2}(\theta^{\circ}) - (1/3) \Lambda(m_{n}^{\star})^{-1}\b_{m}^{2}(\theta^{\circ})]\\
& && \leq \P[S_{m} - \mu_{m} < - [v_{m} + 2 r_{m}]/3 + 3 (1 \vee r) \Phi_{n}^{\star}(\mathfrak{a}, \lambda) \Lambda_{\circ}(m_{n}^{\star})^{-1} - (1/3) \Lambda(m_{n}^{\star})^{-1}\b_{m}^{2}(\theta^{\circ})]
\end{alignat*}
now, using the definition of $G_{n}^{-}$, we have $\b_{m}^{2}(\theta^{\circ}) > 9 L (1 \vee r)\Phi_{n}^{\star}(\mathfrak{a}, \lambda)$ so
\begin{alignat*}{2}
& \P&&[S_{m} < 3 n^{-1}(m_{n}^{\star} - m) ]\\
& && \leq \P[S_{m} - \mu_{m} < - (1/3) [v_{m} + 2 r_{m}]]\\
& && \leq \exp[- (n/36)(n^{-1} (m_{n}^{\star} - m) + 2 \sum\nolimits_{s= m + 1}^{m_{n}^{\star}}\phi(s)^{2})]\\
& && \leq \exp[- (n/36) (n^{-1}(m_{n}^{\star} - m) + 2 \Lambda(m_{n}^{\star})^{-1}\b_{m}^{2}(\theta^{\circ}) - 2 \Lambda(m_{n}^{\star})^{-1}\b_{m_{n}^{\star}}^{2}(\theta^{\circ})) ]\\
& && \leq \exp[- (n/36) (16 L (1 \vee r) \Phi_{n}^{\star}(\mathfrak{a}, \lambda) \Lambda(m_{n}^{\star})^{-1})]\\
& && \leq \exp [ - (4/9) (1 \vee r)m_{n}^{\star} ]
\end{alignat*}
Which in turn implies
\[\P[1 \leq \widehat{m} < G_{n}^{\star-}] \leq \exp[ - (4/9) (1 \vee r) m_{n}^{\star} + \log(G_{n})].\]
\proEnd
\end{pro}
%
\chapter{Proof for \nref{THM_FREQ_IGSSM_KNOWN_IID_ORACLE_NP}}\label{PRO_FREQ_IGSSM_KNOWN_IID_ORACLE_NP}
\section{Proof of \nref{THM_FREQ_IGSSM_KNOWN_IID_ORACLE_NP}}
The $l^{2}$ risk can be written :
\[\E[\Vert \theta_{n, \overline{\widehat{m}}} - \theta^{\circ} \Vert^{2}] = \E[\sum\nolimits_{s = 1}^{G_{n}} ( \theta_{n, \overline{\widehat{m}}}(s) - \theta^{\circ}(s) )^{2}] + \E[\sum\nolimits_{s = G_{n} + 1}^{\infty} \theta^{\circ}(s)^{2}].\]
Together with the fact that, for all $s$ in $\llbracket 1, G_{n} \rrbracket$
\[\theta_{n, \overline{\widehat{m}}}(s) - \theta^{\circ}(s) = (\phi_{n}(s)\lambda(s)^{-1} - \theta^{\circ}(s)) \mathds{1}_{\{\widehat{m} \in \llbracket s, G_{n} \rrbracket\}} + \theta^{\circ}(s) \mathds{1}_{\{\widehat{m} \in \llbracket 1, s-1 \rrbracket\}},\]
implies that
\begin{alignat*}{3}
& \E[\Vert \theta_{n, \overline{\widehat{m}}} - \theta^{\circ} \Vert^{2}] &&\leq&& \underbrace{\sum\nolimits_{s = 1}^{G_{n}}\E[(\phi_{n}(s)\lambda(s)^{-1} - \theta^{\circ}(s))^{2} \mathds{1}_{\{\widehat{m} \in \llbracket s, G_{n} \rrbracket\}}]}_{=: A}\\
& && && + \underbrace{\sum\nolimits_{s = 1}^{G_{n}}\E[\theta^{\circ}(s)^{2} \mathds{1}_{\{\widehat{m} \in \llbracket 1, s-1 \rrbracket\}}]}_{=: B}+ \underbrace{\sum\nolimits_{s = G_{n} + 1}^{\infty}\E[( \theta^{\circ}(s))^{2}]}_{=: C}.
\end{alignat*}
We will now control each of the three parts of the sum using \textsc{\cref{lmA.1.2}} and \textsc{\cref{prB.1.1}}.
Remind that we have $\theta_{n}(s) = \phi_{n}(s)\lambda(s)^{-1}$ for any $s$ in $\N$.
First, consider $A$ and let be some positive real number $p$ to be specified later.
Then we can write
\begin{alignat*}{3}
& A && \leq && \sum\nolimits_{s = 1}^{G_{n}^{+}} \E[(\theta_{n}(s) - \theta^{\circ}(s))^{2}] + \sum\nolimits_{s = G_{n}^{+} + 1}^{G_{n}} \E [ (\theta_{n}(s) - \theta^{\circ}(s) )^{2} \mathds{1}_{\{\widehat{m} \in \llbracket s, G_{n} \rrbracket\}}]\\
%& &&\leq&& \sum\nolimits_{s = 1}^{G_{n}^{+}} \E[(\theta_{n}(s) - \theta^{\circ}(s))^{2}] + \sum\nolimits_{s = G_{n}^{+} + 1}^{G_{n}} \E [ (\theta_{n}(s) - \theta^{\circ}(s) )^{2} \mathds{1}_{\{\widehat{m} \in \llbracket G_{n}^{+}+1, G_{n} \rrbracket\}}]\\
%& &&\leq&& \sum\nolimits_{s = 1}^{G_{n}^{+}} \E[(\theta_{n}(s) - \theta^{\circ}(s))^{2}] + \sum\nolimits_{s = G_{n}^{+} + 1}^{G_{n}} \E [ (\theta_{n}(s) - \theta^{\circ}(s) )^{2} \mathds{1}_{\{\widehat{m} \in \llbracket G_{n}^{+}+1, G_{n} \rrbracket\}}] \\
%& && && - p\E[\mathds{1}_{\{\widehat{m} \in \llbracket G_{n}^{+}+1, G_{n} \rrbracket}] + p \E[\mathds{1}_{\{\widehat{m} \in \llbracket G_{n}^{+}+1, G_{n} \rrbracket}]\\
%& &&\leq&& \sum\nolimits_{s = 1}^{G_{n}^{+}} \E[(\theta_{n}(s) - \theta^{\circ}(s))^{2}] + \E [ ( \sum\nolimits_{s = G_{n}^{+} + 1}^{G_{n}}(\theta_{n}(s) - \theta^{\circ}(s) )^{2} - p) \mathds{1}_{\{\widehat{m} \in \llbracket G_{n}^{+}+1, G_{n} \rrbracket\}}]\\
%& && &&+ p \E[\mathds{1}_{\{\widehat{m} \in \llbracket G_{n}^{+}+1, G_{n} \rrbracket}]\\
& &&\leq&& \underbrace{\sum\nolimits_{s = 1}^{G_{n}^{+}} \E[(\theta_{n}(s) - \theta^{\circ}(s))^{2}]}_{=: A_{3}} + \underbrace{\E[ (\sum\nolimits_{s = G_{n}^{+} + 1}^{G_{n}}(\theta_{n}(s) - \theta^{\circ}(s) )^{2} - p)_{+}]}_{=: A_{1}} + \underbrace{p \E[\mathds{1}_{\{\widehat{m} \in \llbracket G_{n}^{+}+1, G_{n} \rrbracket}]}_{=: A_{2}} .
\end{alignat*}
First, we will control $A_{1}$ using \textsc{\cref{lmA.1.2}}.
The goal is to give $p$ a value that is large enough to control this object but small enough so $p \cdot \P [G_{n}^{+} < \widehat{m} \leq G_{n}]$ is still for the most $\Phi_{n}^{\circ}$.
Define $S_{n} := \sum\nolimits_{s = G_{n}^{+} + 1}^{G_{n}}(\theta_{n}(s) - \theta^{\circ}(s))^{2}.$
We have, for all $s$ in $\llbracket G_{n}^{+} + 1, G_{n} \rrbracket$, $\theta_{n}(s) - \theta^{\circ}(s) \sim \mathcal{N}(0, n^{-1}\Lambda(s) ),$
so $\E[S_{n}] = \frac{1}{n} \sum\nolimits_{s = G_{n}^{+} + 1}^{G_{n}} \Lambda(s)$.
And define, using the definition of $G_{n}$ and \nref{AS_BAYES_GAUSS_CONTRACT_HIERARCHICAL_LAMBDA}
\begin{alignat*}{3}
& t_{m} &&:=&& \Lambda(1) \geq \frac{\Lambda_{G_{n}}}{n} \geq \max\nolimits_{s \in \llbracket G_{n}^{+} + 1, G_{n} \rrbracket} n^{-1}\Lambda(s); \quad v_{m} := G_{n} \Lambda(1) \geq \frac{G_{n} \Lambda_{G_{n}}}{n} \geq  \frac{1}{n} \sum\nolimits_{s = G_{n}^{+} + 1}^{G_{n}} \Lambda(s).
\end{alignat*}
We can take $p = \E[S_{n}] + 3 v_{m}$ which, using the definition of $G_{n}^{+}$ and $G_{n} > G_{n}^{+}$, gives, $A_{1} \leq 6 \Lambda(1) \exp[- 2 m_{n}^{\circ}/L]$.
Thanks to \textsc{\cref{prB.1.1}} it is easily shown that $A_{2} < 4 \Lambda(1) \exp[-5 m_{n}^{\circ}/(9 L) + 2 \log (G_{n})]$.
Finally, we control $A_{3}$ by using the definition of $G_{n}^{+}$, which gives $A_{3} \leq \Lambda_{\circ}(G_{n}^{+})\Lambda(m_{n}^{\circ})^{-1} 3 \Phi_{n}^{\circ}$.
Note that, using \nref{AS_BAYES_GAUSS_CONTRACT_HIERARCHICAL_ORACLE} and the definition of $G_{n}^{+}$, we have that $\Lambda_{\circ}(G_{n}^{+})\Lambda(m_{n}^{\circ})^{-1}$ is bounded for $n$ large enough; indeed with $D^{\circ} := \lceil 3/\kappa^{\circ} + 1\rceil$,
\[G_{n}^{+} \leq 3 n \Phi_{n}^{\circ}\Lambda(m_{n}^{\circ})^{-1} + m_{n}^{\circ} \leq 3 n m_{n}^{\circ} \Lambda_{\circ}(m_{n}^{\circ})(\Lambda(m_{n}^{\circ}) n \kappa^{\circ}){-1} + m_{n}^{\circ} \leq (\frac{3}{\kappa^{\circ}} + 1) m_{n}^{\circ} \leq D^{\circ} m_{n}^{\circ};\]
which implies $\Lambda_{\circ}(G_{n}^{+})\Lambda(m_{n}^{\circ})^{-1} \leq \Lambda_{\circ}(D^{\circ} \cdot m_{n}^{\circ})\Lambda(D^{\circ} \cdot m_{n}^{\circ})^{-1} \cdot \Lambda(D^{\circ} \cdot m_{n}^{\circ})\Lambda(m_{n}^{\circ})^{-1} \leq \Lambda(D^{\circ})$.
Hence, we have
\[A \leq 6 \Lambda(1) \exp[- 2 m_{n}^{\circ}/L] + 4 \Lambda(1) \exp[-5 m_{n}^{\circ}/(9 L) + 2 \log (G_{n})] +  \Lambda(D^{\circ}) 3 \Phi_{n}^{\circ}.\]

Now we control $B$ We use a decomposition similar to the one used for $A$ :
\[ B \leq \underbrace{\sum\nolimits_{s = G_{n}^{-} + 1}^{G_{n}} \theta^{\circ}(s)^{2}}_{=: B_{1}} + \underbrace{\sum\nolimits_{s = 1}^{G_{n}^{-}} \theta^{\circ}(s)^{2} \P[1 \leq \widehat{m} < G_{n}^{-}]}_{=: B_{2}}.\]
First, notice that $B_{1} + C = \mathfrak{b}_{G_{n}^{-}} \leq 9 L \Phi_{n}^{\circ}$ by the definition of $G_{n}^{-}.$
To control $B_{2},$ we use the fact that $\theta^{\circ}$ is square summable and the \textsc{\cref{prB.1.1}}: $B_{2} \leq \exp[-7 m_{n}^{\circ}/9 + \log(G_{n})] \cdot \Vert \theta^{\circ} \Vert^{2}$.
So we have $B + C \leq 9 L \Phi_{n}^{\circ} + \Vert \theta^{\circ} \Vert^{2} \cdot \exp[-\frac{7 m_{n}^{\circ}}{9} + \log(G_{n})]$.
Which leads to :
\begin{multline*}
\E[\Vert \theta_{n, \overline{\widehat{m}}} - \theta^{\circ} \Vert^{2}] \leq 3 ( \Lambda(D^{\circ}) + 3 L ) \Phi_{n}^{\circ}\\
  + (6 \Lambda(1) \exp[- (2 m_{n}^{\circ}/L + \log(\Phi_{n}^{\circ}))] + 4 \Lambda(1) \exp[-5 m_{n}^{\circ}/(9 L) + \log (G_{n}^{2}/\Phi_{n}^{\circ})]\\
  + \Vert \theta^{\circ} \Vert^{2} \cdot \exp[-7 m_{n}^{\circ}/9 + \log(G_{n}/\Phi_{n}^{\circ})]) \Phi_{n}^{\circ},
\end{multline*}
which proves that there exist $C$ such that, for $n$ large enough,
\[\E[\Vert \theta_{n, \overline{\widehat{m}}} - \theta^{\circ} \Vert^{2}] \leq C \Phi_{n}^{\circ}.\]
\proEnd
\chapter{Proof for \nref{THM_FREQ_IGSSM_KNOWN_IID_MINIMAX_NP}}\label{PRO_FREQ_IGSSM_KNOWN_IID_MINIMAX_NP}
\section{Intermediate results}

\section{Detailed proofs}

The $L^{2}$ risk can be written :

\[\mathds{E}_{\theta^{\circ}}^{n}\left[\left\Vert \overline{\theta}^{\widehat{m}} - \theta^{\circ} \right\Vert^{2}\right] = \mathds{E}_{\theta^{\circ}}^{n}\left[\sum\limits_{j = 1}^{G_{n}} \left( \overline{\theta}_{j} - \theta^{\circ}_{j} \right)^{2}\right] + \mathds{E}_{\theta^{\circ}}^{n}\left[\sum\limits_{j = G_{n} + 1}^{\infty} \left(\theta^{\circ}_{j}\right)^{2}\right].\]

Together with
\[\forall j \in \llbracket 1, G_{n} \rrbracket, \quad \overline{\theta}^{\widehat{m}}_{j} - \theta^{\circ}_{j} = \left(\frac{Y_{j}}{\lambda_{j}} - \theta^{\circ}_{j}\right) \mathds{1}_{\{\widehat{m} \in \llbracket j, G_{n} \rrbracket\}} + \theta^{\circ}_{j} \mathds{1}_{\{\widehat{m} \in \llbracket 1, j-1 \rrbracket\}},\]
implies that

\begin{alignat*}{3}
& \mathds{E}_{\theta^{\circ}}^{n}\left[\left\Vert \overline{\theta}^{\widehat{m}} - \theta^{\circ} \right\Vert^{2}\right] && \leq && \underbrace{\sum\limits_{j = 1}^{G_{n}}\mathds{E}_{\theta^{\circ}}^{n}\left[\left(\frac{Y_{j}}{\lambda_{j}} - \theta^{\circ}_{j}\right)^{2} \mathds{1}_{\{\widehat{m} \in \llbracket j, G_{n} \rrbracket\}}\right]}_{=: A}\\
& && && + \underbrace{\sum\limits_{j = 1}^{G_{n}}\mathds{E}_{\theta^{\circ}}^{n}\left[\left(\theta^{\circ}_{j}\right)^{2} \mathds{1}_{\{\widehat{m} \in \llbracket 1, j-1 \rrbracket\}}\right]}_{=: B}+ \underbrace{\sum\limits_{j = G_{n} + 1}^{\infty}\left( \theta^{\circ}_{j}\right)^{2}}_{=: C}.
\end{alignat*}

\bigskip

We will now control each of the three parts of the sum.

\bigskip

First, consider $A$ and let be some positive real number $p$ to be specified later.

Then we can write
\begin{alignat*}{3}
& A && \leq && \sum\limits_{j = 1}^{G_{n}^{\star+}} \mathds{E}_{\theta^{\circ}}^{n}\left[\left(\frac{Y_{j}}{\lambda_{j}} - \theta^{\circ}_{j}\right)^{2}\right] + \sum\limits_{j = G_{n}^{\star+} + 1}^{G_{n}} \mathds{E}_{\theta^{\circ}}^{n} \left[ \left(\frac{Y_{j}}{\lambda_{j}} - \theta^{\circ}_{j} \right)^{2} \mathds{1}_{\{\widehat{m} \in \llbracket j, G_{n} \rrbracket\}}\right]\\
& && \leq && \sum\limits_{j = 1}^{G_{n}^{\star+}} \mathds{E}_{\theta^{\circ}}^{n}\left[\left(\frac{Y_{j}}{\lambda_{j}} - \theta^{\circ}_{j}\right)^{2}\right] + \sum\limits_{j = G_{n}^{\star+} + 1}^{G_{n}} \mathds{E}_{\theta^{\circ}}^{n} \left[ \left(\frac{Y_{j}}{\lambda_{j}} - \theta^{\circ}_{j} \right)^{2} \mathds{1}_{\{\widehat{m} \in \llbracket G_{n}^{\star+}, G_{n} \rrbracket\}}\right]\\
& && \leq && \sum\limits_{j = 1}^{G_{n}^{\star+}} \mathds{E}_{\theta^{\circ}}^{n}\left[\left(\frac{Y_{j}}{\lambda_{j}} - \theta^{\circ}_{j}\right)^{2}\right] + \sum\limits_{j = G_{n}^{\star+} + 1}^{G_{n}} \mathds{E}_{\theta^{\circ}}^{n} \left[ \left(\frac{Y_{j}}{\lambda_{j}} - \theta^{\circ}_{j} \right)^{2} \mathds{1}_{\{\widehat{m} \in \llbracket G_{n}^{\star+}, G_{n} \rrbracket\}}\right] \\
& && && - p\mathds{E}_{\theta^{\circ}}^{n}\left[\mathds{1}_{\{\widehat{m} \in \llbracket G_{n}^{\star+}, G_{n} \rrbracket}\right] + p \mathds{E}_{\theta^{\circ}}^{n}\left[\mathds{1}_{\{\widehat{m} \in \llbracket G_{n}^{\star+}, G_{n} \rrbracket}\right]\\
& && \leq && \sum\limits_{j = 1}^{G_{n}^{\star+}} \mathds{E}_{\theta^{\circ}}^{n}\left[\left(\frac{Y_{j}}{\lambda_{j}} - \theta^{\circ}_{j}\right)^{2}\right] + \mathds{E}_{\theta^{\circ}} \left[ \left(\sum\limits_{j = G_{n}^{\star+} + 1}^{G_{n}}\left(\frac{Y_{j}}{\lambda_{j}} - \theta^{\circ}_{j} \right)^{2} - p\right) \mathds{1}_{\{\widehat{m} \in \llbracket G_{n}^{\star+}, G_{n} \rrbracket\}}\right]\\
& && && + p \mathds{E}_{\theta^{\circ}}^{n}\left[\mathds{1}_{\{\widehat{m} \in \llbracket G_{n}^{\star+}, G_{n} \rrbracket}\right]\\
& && \leq && \underbrace{\sum\limits_{j = 1}^{G_{n}^{\star+}} \mathds{E}_{\theta^{\circ}}^{n}\left[\left(\frac{Y_{j}}{\lambda_{j}} - \theta^{\circ}_{j}\right)^{2}\right]}_{=: A_{3}} + \underbrace{\mathds{E}_{\theta^{\circ}}^{n} \left[ \left(\sum\limits_{j = G_{n}^{\star+} + 1}^{G_{n}}\left(\frac{Y_{j}}{\lambda_{j}} - \theta^{\circ}_{j} \right)^{2} - p\right)_{+}\right]}_{=: A_{1}} + \underbrace{p \mathds{E}_{\theta^{\circ}}^{n}\left[\mathds{1}_{\{\widehat{m} \in \llbracket G_{n}^{\star+}, G_{n} \rrbracket}\right]}_{=: A_{2}} .
\end{alignat*}

\medskip

First, we will control $A_{1}$ using \textsc{\cref{lmA.1.2}}.

The goal is to give $p$ a value that is large enough to control this object but small enough so $p \cdot \mathds{P}_{\theta^{\circ}}^{n} \left[G_{n}^{\star+} < \widehat{m} \leq G_{n}\right]$ is still for the most $\Psi_{n}^{\star}$.

Define $S_{n} := \sum\limits_{j = G_{n}^{\star+} + 1}^{G_{n}}\left(\frac{Y_{j}}{\lambda_{j}} - \theta^{\circ}_{j}\right)^{2}.$

We have, for all $j$ in $\llbracket G_{n}^{\star+} + 1, G_{n} \rrbracket$,
\[ \frac{Y_{j}}{\lambda_{j}} - \theta^{\circ}_{j} \sim \mathcal{N}\left(0, \frac{\Lambda_{j}}{n} \right), \]
so $\mathds{E}_{\theta^{\circ}}^{n}\left[S_{n}\right] = \frac{1}{n} \sum\limits_{j = G_{n}^{\star+} + 1}^{G_{n}} \Lambda_{j}$.

And define
\begin{alignat*}{7}
& t_{m} && := && \Lambda_{1} && \geq && \frac{\Lambda_{G_{n}}}{n} && \geq && \max\limits_{j \in \llbracket G_{n}^{\star+} + 1, G_{n} \rrbracket} \frac{\Lambda_{j}}{n},\\
& v_{m} && := && G_{n} \Lambda_{1} && \geq && \frac{G_{n} \Lambda_{G_{n}}}{n} && \geq &&  \frac{1}{n} \sum\limits_{j = G_{n}^{\star+} + 1}^{G_{n}} \Lambda_{j}.
\end{alignat*}

We can take $p = \mathds{E}_{\theta^{\circ}}^{n}\left[S_{n}\right] + 3 v_{m}$ which gives
\begin{alignat*}{3}
& A_{1} && = && \mathds{E}_{\theta^{\circ}}^{n}\left[\left(S_{n} - \mathds{E}_{\theta^{\circ}}^{n}\left[S_{n}\right] - 3 v_{m}\right)_{+}\right]\\
& && \leq && 6 \Lambda_{1} \exp\left[- \frac{G_{n}}{2}\right]\\
& && \leq && 6 \Lambda_{1} \exp\left[- \frac{n G_{n}}{2 n}\right]\\
& && \leq && 6 \Lambda_{1} \exp\left[- n \frac{3 \Lambda_{m_{n}^{\star}}^{-1} \left(1 \vee r \right) \Psi_{n}^{\star}}{2} - \frac{m_{n}^{\star}}{2}\right]\\
& A_{1} && \leq && 6 \Lambda_{1} \exp\left[- \frac{2 m_{n}^{\star}}{L}\right].
\end{alignat*}

\medskip

Thanks to \textsc{\cref{prB.1.2}} it is easily shown that
\[A_{2} \leq 4 \Lambda_{1} \exp\left[-\frac{5 m_{n}^{\star}}{9 L} + 2 \log \left(G_{n}\right)\right].\]

\medskip

Finally, we control $A_{3}.$
Using the definition of $G_{n}^{\star+}$ we have
\begin{alignat*}{3}
& A_{3} && = && \sum\limits_{j = 1}^{G_{n}^{\star+}} \mathds{E}_{\theta^{\circ}}^{n}\left[\left(\overline{\theta}_{j} - \theta^{\circ}_{j}\right)^{2}\right]\\
& && = && \sum\limits_{j = 1}^{G_{n}^{\star+}} \frac{\Lambda_{j}}{n}\\
& && = && \frac{1}{n} G_{n}^{\star+} \overline{\Lambda}_{G_{n}^{\star+}}\\
& A_{3} && \leq && \frac{\overline{\Lambda}_{G_{n}^{\star+}}}{\Lambda_{m_{n}^{\star}}} \Psi_{n}^{\star} \left(3 \left(1 \vee r\right) + \frac{\Lambda_{m_{n}^{\star}}}{\overline{\Lambda}_{m_{n}^{\star}}}\right).
\end{alignat*}

Note that, using \textsc{\cref{as2.4.2}} and the definition of $G_{n}^{\star+}$, we have that $\frac{\overline{\Lambda}_{G_{n}^{\star+}}}{\Lambda_{m_{n}^{\star}}}$ is bounded for $n$ large enough; indeed with $D^{\star} := \left\lceil \frac{3 \left( 1 \vee r \right) }{\kappa^{\star}} + 1\right\rceil$,

\begin{alignat*}{3}
& G_{n}^{\star+} && \leq && \frac{3 \left(1 \vee r \right) \Psi_{n}^{\star} n}{\Lambda_{m_{n}^{\star}}} + m_{n}^{\star} \leq \frac{3 \left(1 \vee r \right) m_{n}^{\star} \overline{\Lambda}_{m_{n}^{\star}} n}{ n\Lambda_{m_{n}^{\star}} \kappa^{\star}} + m_{n}^{\star} \leq \left(\frac{3\left(1 \vee r \right)}{\kappa^{\star}} + 1\right) m_{n}^{\star} \leq D^{\star} m_{n}^{\star}\\ 
& && \Rightarrow && \frac{\overline{\Lambda}_{G_{n}^{\star+}}}{\Lambda_{m_{n}^{\star}}}\leq \Lambda_{D^{\star}}.
\end{alignat*}

\medskip

Hence, we have
\[A \leq 6 \Lambda_{1} \exp\left[- \frac{2 m_{n}^{\star}}{L}\right] + 4 \Lambda_{1} \exp\left[-\frac{5 m_{n}^{\star}}{9 L} + 2 \log \left(G_{n}\right)\right] +  4 \Lambda_{D^{\star}} L \left(1 \vee r\right) \Psi_{n}^{\star}.\]

\bigskip

Now we control $B$.
We use a similar decomposition to the one used for $A$ :

\begin{alignat*}{3}
& B && \leq && \sum\limits_{j = G_{n}^{\star-} + 1}^{G_{n}} \left(\theta^{\circ}_{j}\right)^{2} + \sum\limits_{j = 1}^{G_{n}^{\star-}} \mathds{E}_{\theta^{\circ}}^{n}\left[\left(\theta^{\circ}_{j}\right)^{2} \mathds{1}_{\left\{\widehat{m} \in \llbracket 1, j - 1 \rrbracket\right\}}\right]\\
& && \leq && \sum\limits_{j = G_{n}^{\star-} + 1}^{G_{n}} \left(\theta^{\circ}_{j}\right)^{2} + \sum\limits_{j = 1}^{G_{n}^{\star-}} \mathds{E}_{\theta^{\circ}}^{n}\left[\left(\theta^{\circ}_{j}\right)^{2} \mathds{1}_{\left\{\widehat{m} \in \llbracket 1, G_{n}^{\star-} - 1 \rrbracket\right\}}\right]\\
& B && \leq && \underbrace{\sum\limits_{j = G_{n}^{\star-} + 1}^{G_{n}} \left(\theta^{\circ}_{j}\right)^{2}}_{=: B_{1}} + \underbrace{\sum\limits_{j = 1}^{G_{n}^{\star-}} \left(\theta^{\circ}_{j}\right)^{2} \mathds{P}_{\theta^{\circ}}^{n}\left[1 \leq \widehat{m} < G_{n}^{\star-}\right]}_{=: B_{2}}.
\end{alignat*}

\medskip

First, notice that $B_{1} + C = \mathfrak{b}_{G_{n}^{\star-}} \leq 9 L \left(1 \vee r\right) \Psi_{n}^{\star}$ by the definition of $G_{n}^{\star-}.$

\medskip

To control $B_{2},$ we use the definition of the Sobolev ellipsoid and the \textsc{\cref{prB.1.2}}:
 
\begin{alignat*}{3}
& B_{2} && = && \mathds{P}_{\theta^{\circ}}^{n}\left[1 \leq \widehat{m} < G_{n}^{\star-}\right] \sum\limits_{j = 1}^{G_{n}^{\star-}} \left(\theta^{\circ}_{j}\right)^{2} \\
& && \leq && \exp\left[-\frac{7 \left(1 \vee r\right) m_{n}^{\star}}{9} + \log\left(G_{n}\right)\right] \cdot \sum\limits_{j = 1}^{G_{n}^{\star-}} \frac{\mathfrak{a}_{j}}{\mathfrak{a}_{j}}\left(\theta^{\circ}_{j}\right)^{2}\\
& && \leq && \exp\left[-\frac{7 \left(1 \vee L^{\circ} \right) m_{n}^{\star}}{9} + \log\left(G_{n}\right)\right] \cdot \mathfrak{a}_{1}\sum\limits_{j = 1}^{G_{n}^{\star-}} \frac{1}{\mathfrak{a}_{j}}\left(\theta^{\circ}_{j}\right)^{2}\\
& && \leq && \exp\left[-\frac{7 m_{n}^{\star}}{9} + \log\left(G_{n}\right)\right] \cdot \mathfrak{a}_{1}r\\
 \end{alignat*}

\medskip

So we have
\[B + C \leq 9 L \left(1 \vee r \right) \Psi_{n}^{\star} + \mathfrak{a}_{1} L^{\circ} \cdot \exp\left[-\frac{7 m_{n}^{\star}}{9} + \log\left(G_{n}\right)\right].\]

\bigskip

Which leads to :
\begin{alignat*}{3}
& \mathds{E}_{\theta^{\circ}}^{n}\left[\left\Vert \overline{\theta}^{\widehat{m}} - \theta^{\circ} \right\Vert^{2}\right] && \leq && L \left( 1 \vee L^{\circ} \right) \left( 4 \Lambda_{D^{\star}} + 9\right) \Psi_{n}^{\star} +\\
& && && \left(6 \Lambda_{1} \exp\left[- \frac{2 m_{n}^{\star}}{L} - \log\left(\Psi_{n}^{\star}\right)\right] + 4 \Lambda_{1} \exp\left[-\frac{5 m_{n}^{\star}}{9 L} + \log \left(\frac{G_{n}^{2}}{\Psi_{n}^{\star}}\right)\right] + \right.\\
& && && \left.\mathfrak{a}_{1} r \cdot \exp\left[-\frac{7 m_{n}^{\star}}{9} + \log\left(\frac{G_{n}}{\Psi_{n}^{\star}}\right)\right]\right) \Psi_{n}^{\star},\\
\end{alignat*}

which proves that there exist $K$ such that, for $n$ large enough,
\[\mathds{E}_{\theta^{\circ}}^{n}\left[\left\Vert \overline{\theta}^{\widehat{m}} - \theta^{\circ} \right\Vert^{2}\right] \leq C \Psi_{n}^{\star}.\]

\textcolor{red}{Add remark stating that the bound is uniform over the ellipsoid hence rate for the maximal risk?}
\chapter{Proof for \nref{THM_FREQ_CIRCDECONV_KNOWN_IID_ORACLE_P}}\label{PRO_FREQ_CIRCDECONV_KNOWN_IID_ORACLE_P}


%======================================================================================================================
%                                                                 
% Title:  Appendix:  known error density
% Author: Jan JOHANNES, Institut für Angewandte Mathematik, Ruprecht-Karls Universität Heidelberg, Deutschland  
% 
% Email: johannes@math.uni-heidelberg.de
% Date: %%ts latex start%%[2018-03-29 Thu 13:22]%%ts latex end%%
%
% ======================================================================================================================
% --------------------------------------------------------------------
% section <<Appendix: Proofs of \nref{ak}>>\ref{a:ak}
% --------------------------------------------------------------------
% --------------------------------------------------------------------
% <<Proof of Re key argument>>
% --------------------------------------------------------------------
%\begin{pro}[Proof of \nref{co:agg}.]
%We start the proof with the observation that
%$\overline{\theta_{n}}(s)-\ofxdf[(s)]=\theta_{n}(-s)-\fxdf[(-s)]$ for all $s\in\Zz$ and 
%\begin{multline*}
%  \theta_{n}(s)-\fxdf[(s)]=\fedfI[(s)](\hfydf[(s)]-\fydf[(s)])\We[](\nset{s,\ssY})-\fxdf[(s)]\We[](\nsetro{1,s})\text{ for all }s\in\nset{1,\ssY},\\\theta_{n}(0)-\fxdf[(0)]=0\text{ and }\theta_{n}(s)-\fxdf[(s)]=-\fxdf[(s)]\text{ for all }s>\ssY.
%\end{multline*}
%Consequently, (keep in mind that $\vert\fedfI[(s)]\vert^2=\iSv[s]$)  we  have
%  \begin{multline}\label{co:agg:pro1}
%    \Vnormlp{\theta_{n}-\xdf}^2=
%   2\sum_{s\in\nset{1,\ssY}}\vert\fedfI[(s)](\hfydf[(s)]-\fydf[(s)])\We[](\nset{s,\ssY})-\fxdf[(s)]\We[](\nsetro{1,s})\vert^2+2\sum_{s>\ssY}\vert\fxdf[(s)]\vert^2\\
%\leq
%   \sum_{s\in\nset{1,\ssY}}4\{\iSv[s]\vert\hfydf[(s)]-\fydf[(s)]\vert^2 \We[](\nset{s,\ssY})\} + \sum_{s\in\nset{1,\ssY}}4\vert\fxdf[(s)]\vert^2\We[](\nsetro{1,s})+2\sum_{s>n}\vert\fxdf[(s)]\vert^2,%\\
%% \leq 2\{n^{-1} \peDi \oEvs[\peDi]+6\Evs_1\exp(-\DiMa/3)+6\Evs_1\exp
%%   \big(-\frac{n\dnRa}{2}+ 2\log \DiMa \big)\}\\
%% + 2\{\gb_{\meDi}^2 +2\Vnormlp{\xdf}^2\exp\big(-\frac{n\dnRa}{2}+ 2\log \DiMa \big)\}.
% \end{multline}
%where we consider the first r.h.s and the two other r.h.s. terms
%separately. Consider the first r.h.s. term in \eqref{co:agg:pro1}. We split the sum into two parts which we bound separately.  Precisely,
%\begin{multline}\label{co:agg:pro2}
%2\sum_{s\in\nset{1,\ssY}}\iSv[s]\vert\hfydf[(s)]-\fydf[(s)]\vert^2
%\We[](\nset{s,\ssY})\\
%% \leq 2\sum_{s\in\nset{1,\pDi}}\iSv[s]\vert\hfydf[(s)]-\fydf[(s)]\vert^2 +
%% 2\sum_{s\in\nsetlo{\pDi,\ssY}}\iSv[s]\vert\hfydf[(s)]-\fydf[(s)]\vert^2\sum_{l\in\nset{s,\ssY}}\We[(l)]\\
%% = 2\sum_{s\in\nset{1,\pDi}}\iSv[s]\vert\hfydf[(s)]-\fydf[(s)]\vert^2 +
%% \sum_{l\in\nsetlo{\pDi,\ssY}}\We[(l)]\;2\sum_{s\in\nsetlo{\pDi,l}}\iSv[s]\vert\hfydf[(s)]-\fydf[(s)]\vert^2\\
%\leq \Vnormlp{\txdfPr[\pDi]-\xdfPr[\pDi]}^2
%+\sum_{l\in\nsetlo{\pDi,\ssY}}\We[(l)]\Vnormlp{\txdfPr[l]-\xdfPr[l]}^2\\
%% \leq\Vnormlp{\txdfPr[\pDi]-\xdfPr[\pDi]}^2
%% +\sum_{l\in\nsetlo{\pDi,\ssY}}\We[(l)]\Vnormlp{\txdfPr[l]-\xdfPr[l]}^2\Ind{\{\Vnormlp{\txdfPr[l]-\xdfPr[l]}^2\geq\penSv[l]\}}
%% \\
%% \hfill+(12\cpen/\ssY)\sum_{l\in\nsetlo{\pDi,\ssY}}\DipenSv[l]\We[(l)]\Ind{\{\Vnormlp{\txdfPr[l]-\xdfPr[l]}^2<\penSv[l]\}}\\
%% =\Vnormlp{\txdfPr[\pDi]-\xdfPr[\pDi]}^2
%% +\sum_{l\in\nsetlo{\pDi,\ssY}}\We[(l)]\vect{\Vnormlp{\txdfPr[l]-\xdfPr[l]}^2-\cst{1}\penSv[l]}\Ind{\{\Vnormlp{\txdfPr[l]-\xdfPr[l]}^2\geq\penSv[l]\}}\\
%% \hfill+(12\cst{1}\cpen/\ssY)\sum_{l\in\nsetlo{\pDi,\ssY}}\We[(l)]\DipenSv[l]\Ind{\{\Vnormlp{\txdfPr[l]-\xdfPr[l]}^2\geq\penSv[l]\}}
%%  +(12\cpen/\ssY)\sum_{l\in\nsetlo{\pDi,\ssY}}\DipenSv[l]\We[(l)]\Ind{\{\Vnormlp{\txdfPr[l]-\xdfPr[l]}^2<\penSv[l]\}}\\
%\leq\Vnormlp{\txdfPr[\pDi]-\xdfPr[\pDi]}^2
%+\sum_{l\in\nsetlo{\pDi,\ssY}}\We[(l)]\vectp{\Vnormlp{\txdfPr[l]-\xdfPr[l]}^2-\pen(l)/7}\\
%+\tfrac{1}{7}\sum_{l\in\nsetlo{\pDi,\ssY}}\We[(l)]\pen(l)\Ind{\{\Vnormlp{\txdfPr[l]-\xdfPr[l]}^2\geq\pen(l)/7\}}
%+\tfrac{1}{7}\sum_{l\in\nsetlo{\pDi,\ssY}}\pen(l)\We[(l)]\Ind{\{\Vnormlp{\txdfPr[l]-\xdfPr[l]}^2<\pen(l)/7\}}\\
%\leq\tfrac{1}{7}\pen(\pDi)
%+\sum_{l\in\nset{\pDi,\ssY}}\vectp{\Vnormlp{\txdfPr[l]-\xdfPr[l]}^2-\pen(l)/7}\\
%+\tfrac{1}{7}\sum_{l\in\nsetlo{\pDi,\ssY}}\We[(l)]\pen(l)\Ind{\{\Vnormlp{\txdfPr[l]-\xdfPr[l]}^2\geq\pen(l)/7\}}
%+\tfrac{1}{7}\sum_{l\in\nsetlo{\pDi,\ssY}}\pen(l)\We[(l)]\Ind{\{\Vnormlp{\txdfPr[l]-\xdfPr[l]}^2<\pen(l)/7\}}%\\
%\end{multline}
%Consider the second and third r.h.s. term in \eqref{co:agg:pro1}.  Splitting the first sum into two parts we obtain
%\begin{multline}\label{co:agg:pro3}
%2\sum_{s\in\nset{1,\ssY}}\vert\fxdf[(s)]\vert^2\We[](\nsetro{1,s})+2\sum_{s>\ssY}\vert\fxdf[(s)]\vert^2\\
%\hspace*{5ex}\leq  2\sum_{s\in\nset{1,\mDi}}\vert\fxdf[(s)]\vert^2\We[](\nsetro{1,s})+ 2\sum_{s\in\nsetlo{\mDi,n}}\vert\fxdf[(s)]\vert^2+
%  2\sum_{s>n}\vert\fxdf[(s)]\vert^2\\\hfill
%\leq \Vnormlp{\theta^{\circ}_{\underline{0}}}^2\{\We[](\nsetro{1,\mDi})+\bias[\mDi]^2(\xdf)\}
%\end{multline}
%Combining  \eqref{co:agg:pro1} and the upper bounds \eqref{co:agg:pro2}
%and \eqref{co:agg:pro3} we obtain   the assertion, which completes the proof.\proEnd
%\end{pro}

\section{Proofs of \nref{AK:RB:OR}}\label{a:ak:rb}
% ....................................................................
% Te <<Upper bound random weights>>
% ....................................................................
\begin{te}
 Below  we state the proofs of  \nref{ak:re:SrWe:ag} and \nref{ak:re:SrWe:ms}. The
  proof of \nref{ak:re:SrWe:ag} is based on \nref{re:rWe} given first.
\end{te}
% ....................................................................
% <<Re Random weights>>
% ....................................................................
\begin{lm}\label{re:rWe} Consider the data-driven aggregation weights
  $\rWe[]$ as in \eqref{ak:de:rWe}. Under condition
  \nref{ak:ass:pen:oo} for any $l\in\nset{1,\ssY}$ with
  $\daRaS{l}{\xdf,\Lambda}:=\daRa{l}{(\xdf,\Lambda)}$ holds
  \begin{resListeN}[]
  \item\label{re:rWe:i} for all $k\in\nsetro{1,l}$ we have\\
    $\rWe\Ind{\setB{\Vnormlp{\txdfPr[l]-\xdfPr[l]}^2<\cpen\daRaS{l}{\xdf,\Lambda}/7}} 
    \leq\exp\big(\rWn\big\{-\tfrac{\Vnormlp{\xdf_{\underline{0}}}^2}{2}\bias^2(\xdf)
    +[\tfrac{25\cpen}{14}+\tfrac{\Vnormlp{\xdf_{\underline{0}}}^2}{2}]\daRaS{l}{\xdf,\Lambda}-\penSv\big\}\big)$%\\
    % $\rWe\Ind{\setB{\Vnormlp{\txdfPr[l]-\xdfPr[l]}^2<\penSv[l]}} 
    % \leq\exp\big(\rWn\big\{-\tfrac{\Vnormlp{\Proj[{\mHiH[0]}]^\perp\xdf}^2}{2}\bias^2(\xdf)
    % +[120\cpen+\tfrac{\Vnormlp{\Proj[{\mHiH[0]}]^\perp\xdf}^2}{2}]\hRaDi{l,\xdf,\Lambda}\big\}\big)$
  \item\label{re:rWe:ii} for all $\Di\in\nsetlo{l,\ssY}$ we have\\
    $\rWe\Ind{\setB{\Vnormlp{\txdfPr-\xdfPr}^2<\penSv/7}}\leq\exp\big(\rWn\big\{-\tfrac{1}{2}\penSv
    +[\tfrac{3}{2}\Vnormlp{\xdf_{\underline{0}}}^2+\cpen]\daRaS{l}{\xdf,\Lambda}\big\}\big)$.
    % $\rWe\Ind{\setB{\Vnormlp{\txdfPr-\xdfPr}^2<\penSv}}\leq
    % \exp\big(\rWn\big\{-\penSv
    % +[\tfrac{3}{2}\Vnormlp{\Proj[{\mHiH[0]}]^\perp\xdf}^2+54\cpen]\hRaDi{l,\xdf,\Lambda}\big\}\big)$.
  \end{resListeN}
\end{lm}
% --------------------------------------------------------------------
% <<Proof Re Random weights>>
% --------------------------------------------------------------------
\begin{pro}[Proof of \nref{re:rWe}.]
  Given $\Di,l\in\nset{1,\ssY}$ and an event $\dmEv{\Di}{l}$ (to be
  specified below) it clearly follows
  \begin{multline}\label{re:rWe:pro1}
    \rWe\Ind{\dmEv{\Di}{l}}
    =\frac{\exp(-\rWn\{-\Vnormlp{\txdfPr}^2+\penSv\})}
    {\sum_{l\in\nset{1,\ssY}}\exp(-\rWn\{-\Vnormlp{\txdfPr[l]}^2+\penSv[l]\})}\Ind{\dmEv{\Di}{l}}\\
    \leq
    \exp\big(\rWn\big\{\Vnormlp{\txdfPr}^2-\Vnormlp{\txdfPr[l]}^2+(\penSv[l]-\penSv)\big\}\big)\Ind{\dmEv{\Di}{l}}
  \end{multline}
  We distinguish the two cases \ref{re:rWe:i} $\Di\in\nsetro{1,l}$ and \ref{re:rWe:ii}
  $\Di\in\nsetlo{l,n}$.  Consider first \ref{re:rWe:i} $\Di\in\nsetro{1,l}$. From
  \ref{re:contr:e1} in \nref{re:contr} (with
  $\dxdfPr[]=\txdfPr[\ssY]$) follows that
  \begin{multline*}%\label{re:rWe:pro2}
    \rWe\Ind{\dmEv{\Di}{l}}
    \leq
    \exp\big(\rWn\big\{\Vnormlp{\txdfPr}^2-\Vnormlp{\txdfPr[l]}^2+(\penSv[l]-\penSv)\big\}\big)\Ind{\dmEv{\Di}{l}}\\
    % \exp\big(\rWn\big\{\contr[](\txdfPr[l])-\contr[](\txdfPr)+\tfrac{9}{2}(\penSv[l]-\penSv)\big\}\big)\Ind{\dmEv{\Di}{l}}\\
    \leq \exp\big(\rWn\big\{\tfrac{11}{2}\Vnormlp{\txdfPr[l]-\xdfPr[l]}^2-\tfrac{1}{2}\Vnormlp{\xdf_{\underline{0}}}^2(\bias[k]^2(\xdf)-\bias[l]^2(\xdf))+(\penSv[l]-\penSv[k])\big\}\big)\Ind{\dmEv{k}{l}}
  \end{multline*}
  If we define
  $\dmEv{\Di}{l}:=\setB{\Vnormlp{\txdfPr[l]-\xdfPr[l]}^2<\cpen\daRa{l}{(\xdf,\Lambda)}/7}$
  then the last bound togehter with \ref{ak:ass:pen:oo}, i.e.,
  $[\Vnormlp{\xdf_{\underline{0}}}^2+\cpen]\daRa{\Di}{(\xdf,\Lambda)}\geq
  \Vnormlp{\xdf_{\underline{0}}}^2\bias^2(\xdf)\vee\penSv$, implies the
  assertion \ref{re:rWe:i}, that is
  \begin{multline*}
    \rWe\Ind{\setB{\Vnormlp{\txdfPr[l]-\xdfPr[l]}^2<\cpen\daRa{l}{(\xdf,\Lambda)}/7}}
    \\\leq\exp\big(\rWn\big\{\tfrac{11}{14}\cpen\daRa{l}{(\xdf,\Lambda)}
    +\tfrac{1}{2}\Vnormlp{\xdf_{\underline{0}}}^2\bias[l]^2(\xdf) +\penSv[l]
    -\tfrac{1}{2}\Vnormlp{\xdf_{\underline{0}}}^2\bias^2(\xdf)-\penSv\big\}\big)\\
    \leq\exp\big(\rWn\big\{[\tfrac{25}{14}\cpen+\tfrac{1}{2}\Vnormlp{\xdf_{\underline{0}}}^2]\daRa{l}{(\xdf,\Lambda)}
    -\tfrac{1}{2}\Vnormlp{\xdf_{\underline{0}}}^2\bias^2(\xdf)-\penSv\big\}\big).
    % =\exp\big(\rWn\big\{10*\penSv[l]-\tfrac{1}{2}\Vnormlp{\Proj[{\mHiH[0]}]^\perp\xdf}^2(\bias[k]^2(\xdf)-\bias[l]^2(\xdf))-\tfrac{9}{2}\penSv[k]\big\}\big)
  \end{multline*}
  % and hence, by exploiting that $\penSv[k]\geq0$ and
  % $\hRa{l,\xdf,\Lambda}=[\bias[l]^2(\xdf)\vee \DipenSv[l] \ssY^{-1}]$
  % follows the assertion \ref{re:rWe:i}, that is
  % \begin{multline*}
  %   \rWe[(k)]\Ind{\setB{\Vnormlp{\txdfPr[l]-\xdfPr[l]}^2<\penSv[l]}}\leq
  %   \exp\big(\rWn\big\{-\tfrac{\Vnormlp{\Proj[{\mHiH[0]}]^\perp\xdf}^2}{2}\bias[k]^2(\xdf)+[10*12\cpen+\tfrac{\Vnormlp{\Proj[{\mHiH[0]}]^\perp\xdf}^2}{2}]\hRa{l,\xdf,\Lambda})\big\}\big).
  % \end{multline*}
  Consider secondly \ref{re:rWe:ii} $\Di\in\nsetlo{l,n}$. From \ref{re:contr:e2}
  in \nref{re:contr} (with $\dxdfPr[]=\txdfPr[\ssY]$) and
  \eqref{re:rWe:pro1} follows
  \begin{multline*}
    \rWe[(k)]\Ind{\dmEv{l}{k}}
    \leq\exp\big(\rWn\big\{\Vnormlp{\txdfPr}^2-\Vnormlp{\txdfPr[l]}^2
    +(\penSv[l]-\penSv)\big\}\big)\Ind{\dmEv{\Di}{l}}\\
    \leq
    \exp\big(\rWn\big\{\tfrac{7}{2}\Vnormlp{\txdfPr[k]-\xdfPr[k]}^2
    +\tfrac{3}{2}\Vnormlp{\xdf_{\underline{0}}}^2(\bias[l]^2(\xdf)-\bias^2(\xdf))
    +(\penSv[l]-\penSv)\big\}\big)\Ind{\dmEv{l}{k}}
  \end{multline*}
  If we set $\dmEv{l}{\Di}:=\{\Vnormlp{\txdfPr-\xdfPr}^2<\penSv/7\}$
  then we clearly have
  \begin{multline*}
    \rWe\Ind{\setB{\Vnormlp{\txdfPr-\xdfPr}^2<\penSv/7}}\\
    \leq \exp\big(\rWn\big\{-\tfrac{1}{2}\penSv+\penSv[l]+
    \tfrac{3}{2}\Vnormlp{\xdf_{\underline{0}}}^2\bias[l]^2(\xdf)
    -\tfrac{3}{2}\Vnormlp{\xdf_{\underline{0}}}^2\bias^2(\xdf)\big\}\big)
  \end{multline*}
  and hence, by exploiting $\bias^2(\xdf)\geq0$ and
  \nref{ak:ass:pen:oo} follows the assertion \ref{re:rWe:ii}, that is
  \begin{equation*}
    \rWe[(k)]\Ind{\setB{\Vnormlp{\txdfPr[k]-\xdfPr[k]}^2<\penSv}}
    % \leq \exp\big(\rWn\big\{-\tfrac{1}{2}\penSv
    % +[\tfrac{3}{2}\Vnormlp{\Proj[{\mHiH[0]}]^\perp\xdf}^2+\tfrac{9}{2}*12\cpen]\hRa{l,\xdf,\Lambda}\big\}\big).
    \leq \exp\big(\rWn\big\{-\tfrac{1}{2}\penSv+[\tfrac{3}{2}\Vnormlp{\xdf_{\underline{0}}}^2+\cpen]\daRa{l}{(\xdf,\Lambda)}\big\}\big),
 \end{equation*}
which completes the proof.\proEnd
\end{pro}
% ....................................................................
% <<Proof Re Sum Random weights>>
% ....................................................................
\begin{pro}[Proof of \nref{ak:re:SrWe:ag}.]
  Consider \ref{ak:re:SrWe:ag:i}. For the non trivial case $\mDi>1$
  from \nref{re:rWe} \ref{re:rWe:i} with $l=\mdDi$ follows for all
  $\Di<\mDi\leq \mdDi$, and hence due to the definition
  \eqref{ak:de:*Di:ag}
  $\Vnormlp{\xdf_{\underline{0}}}^2\bias^2\geq
  \Vnormlp{\xdf_{\underline{0}}}^2\bias[\mDi-1]^2>2[\Vnormlp{\xdf_{\underline{0}}}^2+2\cpen]\daRaS{\mdDi}{\xdf,\Lambda}$.
  Exploiting the last bound we obtain for each $\Di\in\nsetro{1,\mDi}$
  \begin{multline*}
    \rWe\Ind{\setB{\Vnormlp{\txdfPr[\mdDi]-\xdfPr[\mdDi]}^2<\cpen\daRaS{\mdDi}{\xdf,\Lambda}/7}}
    \leq
    \exp\big(\rWn\big\{-\tfrac{\Vnormlp{\xdf_{\underline{0}}}^2}{2}\bias^2(\xdf)
    +[\tfrac{25\cpen}{14}+\tfrac{\Vnormlp{\xdf_{\underline{0}}}^2}{2}]\daRaS{\mdDi}{\xdf,\Lambda}-\penSv\big\}\big)\\
    % \hfill=\exp\big(\rWn\big\{\underbrace{-\tfrac{1}{2}\Vnormlp{\So}^2\bias^2(\So)
    % +[\tfrac{28}{14}\cpen+\tfrac{1}{2}\Vnormlp{\So}^2]\dRa{\mdDi}(\So)}_{\leq0}\}\big)\hfill\\
    % \hfill\times\exp\big(-\tfrac{3}{14}\rWc\cpen
    % n\dRa{\mdDi}(\So)\big)\\
    \hfill
    \leq\exp\big(-\tfrac{3}{14}\rWc\cpen \ssY\daRaS{\mdDi}{\xdf,\Lambda}-\rWn\penSv\big)
  \end{multline*}
  which in turn with
  $\penSv=\cpen \Di\cmiSv\miSv \ssY^{-1}\geq \cpen\Di\ssY^{-1}$ and
  $\sum_{\Di\in\Nz}\exp(-\mu\Di)\leq \mu^{-1}$ for any $\mu>0$
  implies \ref{ak:re:SrWe:ag:i}, that is,
  \begin{multline*}
    \rWe[](\nsetro{1,\mDi})\leq
    \rWe[](\nsetro{1,\mDi})\Ind{\setB{\Vnormlp{\txdfPr[\mdDi]-\xdfPr[\mdDi]}^2<\cpen\daRaS{\mdDi}{\xdf,\Lambda}/7}}
    +\Ind{\setB{\Vnormlp{\txdfPr[\mdDi]-\xdfPr[\mdDi]}^2\geq\cpen\daRaS{\mdDi}{\xdf,\Lambda}/7}}\\
    \hfill\leq\exp\big(-\tfrac{3\rWc\cpen}{14}\ssY\daRaS{\mdDi}{\xdf,\Lambda}\big)\sum_{k=1}^{\mDi-1}\exp(-\rWc\cpen\Di)
    +\Ind{\setB{\Vnormlp{\txdfPr[\mdDi]-\xdfPr[\mdDi]}^2\geq\cpen\daRaS{\mdDi}{\xdf,\Lambda}/7}}\\
    \leq \tfrac{1}{\rWc\cpen}\exp\big(-\tfrac{3\rWc\cpen}{14}\ssY\daRaS{\mdDi}{\xdf,\Lambda}\big)
    +\Ind{\setB{\Vnormlp{\txdfPr[\mdDi]-\xdfPr[\mdDi]}^2\geq\cpen\daRaS{\mdDi}{\xdf,\Lambda}/7}}.
  \end{multline*} 
  Consider \ref{ak:re:SrWe:ag:ii}. From \nref{re:rWe} \ref{re:rWe:ii}
  with $l=\pdDi$ follows for all $\Di>\pDi\geq \pdDi$, and hence due
  to the definition \eqref{ak:de:*Di:ag}
  $\penSv > 2[3\Vnormlp{\xdf_{\underline{0}}}^2+ 2\cpen]\daRaS{\pdDi}{\xdf,\Lambda}$. Thereby, we
  obtain for $\Di\in\nsetlo{\mDi,n}$
  \begin{multline*}
    \rWe\Ind{\setB{\Vnormlp{\txdfPr-\xdfPr}^2<\penSv/7}}
    \leq % \exp\big(\rWn\big\{-\tfrac{1}{2}\pen +[\tfrac{3}{2}\Vnormlp{\So}^2+\cpen]\dRa{\pdDi}(\So)\big\}\big)
    % \\
    % =
    \exp\big(\rWn\big\{-\tfrac{1}{4} \penSv
    -\tfrac{1}{4}\penSv 
    +[\tfrac{3}{2}\Vnormlp{\xdf_{\underline{0}}}^2+\cpen]\daRaS{\pdDi}{\xdf,\Lambda}\big\}\big)\\
    \leq \exp\big(\rWn\big\{-\tfrac{1}{4} \penSv\big\}\big).
  \end{multline*}
   which in turn with $\penSv=\cpen \Di\cmiSv\miSv \ssY^{-1}$  implies
  \begin{multline}\label{ak:re:SrWe:ag:pe1}
    \sum_{\Di\in\nsetlo{\pDi,n}}\penSv\rWe\Ind{\{\Vnormlp{\txdfPr-\xdfPr}^2\leq\pen/7\}}\\
    \leq \cpen\ssY^{-1}\sum_{\Di\in\nsetlo{\pDi,n}} \Di\cmiSv\miSv\exp\big(-\tfrac{\rWc\cpen}{4}\Di\cmiSv\miSv\big)
    % \tfrac{4}{\rWc}n{^{-1}}\sum_{\Di\in\nsetlo{\pDi,n}}\tfrac{\rWc\cpen}{4}\Di
    %\exp\big(-\tfrac{\rWc\cpen}{4}\Di\big)\leq\tfrac{16}{\rWc^2\cpen}n^{-1},\hfill
  \end{multline}
  Exploiting that
  $\sqrt{\cmiSv}=\tfrac{\log (\Di\miSv \vee
    (\Di+2))}{\log(\Di+2)}\geq1$, $\cpen/4\geq2\log(3e)$ and
  $\rWc\geq1$, then for all $k\in\Nz$ we have
  $\tfrac{\rWc\cpen}{4} k-\log(k+2)\geq1$, and hence by
  $a\exp(-ab)\leq \exp(-b)$ for $a,b\geq1$, it follows
  \begin{multline*}
    \cmiSv\Di \miSv\exp\big(-\tfrac{\rWc\cpen}{4}\cmiSv\Di\miSv\big)
    \leq\cmiSv\exp\big(-\tfrac{\rWc\cpen}{4}\cmiSv\Di\miSv + \sqrt{\cmiSv}\log(\Di+2)\big)
    \\\hfill\leq
    \cmiSv\exp\big(-\cmiSv(\tfrac{\rWc\cpen}{4}\Di-\log(\Di+2))\big)
    \leq\exp\big(-(\tfrac{\rWc\cpen}{4}\Di-\log(\Di+2))\big)\\
    =(\Di+2)\exp\big(-\tfrac{\rWc\cpen}{4}\Di\big).
  \end{multline*}
  Exploiting $\sum_{\Di\in\Nz}\mu\Di\exp(-\mu\Di)\leq \mu^{-1}$ and
  $\sum_{\Di\in\Nz}\mu\exp(-\mu\Di)\leq 1$ for any $\mu$; we obtain
  \begin{displaymath}
    \sum_{k=\pDi+1}^{\ssY}\cmiSv\Di \miSv\exp\big(-\tfrac{\rWc\cpen}{4}\cmiSv\Di\miSv\big)
    \leq \sum_{k=\pDi+1}^\infty(\Di+2)\exp\big(-\tfrac{\rWc\cpen}{4}\Di\big)
    \leq \tfrac{16}{\cpen^2\rWc^{2}}+ \tfrac{8}{\cpen\rWc}.
  \end{displaymath}
  Combining the last bound and \eqref{ak:re:SrWe:ag:pe1} we obtain the
  assertion \ref{ak:re:SrWe:ag:ii}, that is
  \begin{displaymath}
    \sum_{\Di\in\nsetlo{\pDi,n}}\penSv\rWe\Ind{\{\Vnormlp{\txdfPr-\xdfPr}^2\leq\pen/7\}}
    \leq \ssY^{-1}\{\tfrac{16}{\cpen\rWc^{2}}+ \tfrac{8}{\rWc}\}
  \end{displaymath}
  which completes the proof.\proEnd
\end{pro}
% --------------------------------------------------------------------
% <<Proof Re Sum MS Random weights>>
% --------------------------------------------------------------------
\begin{pro}[Proof of \nref{ak:re:SrWe:ms}.]
  By definition of $\hDi$ it holds
  $-\Vnormlp{\txdfPr[\hDi]}^2+\penSv[\hDi]\leq
  -\Vnormlp{\txdfPr}^2+\penSv$ for all $\Di\in\nset{1,\ssY}$, and
  hence
  \begin{equation}\label{ak:re:SrWe:ms:pr:e1}
    \Vnormlp{\txdfPr[\hDi]}^2-\Vnormlp{\txdfPr}^2\geq
    \penSv[\hDi]-\penSv\text{ for all }\Di\in\nset{1,\ssY}.
  \end{equation}
  Consider \ref{ak:re:SrWe:ms:i}. It is sufficient to show, that
  $\{\hDi\in\nsetro{1,\mDi}\}\subseteq
  \{\Vnormlp{\txdfPr-\xdfPr}^2\geq\cpen\daRaS{\mdDi}{\xdf,\Lambda}/7\}$
  for $\mDi>1$ holds.  On the event $\{\hDi\in\nsetro{1,\mDi}\}$ holds
  $1\leq\hDi<\mDi\leq\mdDi$ and thus by definition
  \eqref{ak:de:*Di:ag}
  \begin{equation}\label{ak:re:SrWe:ms:pr:e2}
    \Vnormlp{\xdf_{\underline{0}}}^2\bias[\hDi]^2(\xdf)>
    [\Vnormlp{\xdf_{\underline{0}}}^2+4\cpen]\daRaS{\mdDi}{\xdf,\Lambda}
  \end{equation}
  and due to \nref{re:contr} \ref{re:contr:e1} also
  \begin{equation}\label{ak:re:SrWe:ms:pr:e3}
    \Vnormlp{\txdfPr[\hDi]}^2-\Vnormlp{\txdfPr[\mdDi]}^2\leq
    \tfrac{11}{2}\Vnormlp{\txdfPr[\mdDi]-\xdfPr[\mdDi]}^2
    -\tfrac{1}{2}\Vnormlp{\xdf_{\underline{0}}}^2\{\bias[\hDi]^2(\xdf)-\bias[\mdDi]^2(\xdf)\}.
  \end{equation}
  Combining, \eqref{ak:re:SrWe:ms:pr:e1} and
  \eqref{ak:re:SrWe:ms:pr:e3} it follows that
  \begin{multline*}
    \tfrac{11}{2}\Vnormlp{\txdfPr[\mdDi]-\xdfPr[\mdDi]}^2\geq
    \penSv[\hDi]-\penSv[\mdDi]
    +\tfrac{1}{2}\Vnormlp{\xdf_{\underline{0}}}^2\{\bias[\hDi]^2(\xdf)-\bias[\mdDi]^2(\xdf)\}\hfill
  \end{multline*}
  and hence together with $\penSv[\hDi]\geq0$, \eqref{ak:re:SrWe:ms:pr:e2}
  and \ref{ak:ass:pen:oo} we obtain the claim, that is
  \begin{multline*}
    \tfrac{11}{2}\Vnormlp{\txdfPr[\mdDi]-\xdfPr[\mdDi]}^2\geq
    \tfrac{1}{2}\Vnormlp{\xdf_{\underline{0}}}^2\bias[\hDi]^2(\xdf)-
    \tfrac{1}{2}\Vnormlp{\xdf_{\underline{0}}}^2\bias[\mdDi]^2(\xdf)
    -\penSv[\mdDi]\\
    >[\tfrac{1}{2}\Vnormlp{\xdf_{\underline{0}}}^2+2\cpen]\daRaS{\mdDi}{\xdf,\Lambda}
    -\tfrac{1}{2}\Vnormlp{\xdf_{\underline{0}}}^2\bias[\mdDi]^2(\xdf)-\penSv[\mdDi]
    \geq\tfrac{11}{14}\cpen\daRaS{\mdDi}{\xdf,\Lambda},
  \end{multline*}
  and shows \ref{ak:re:SrWe:ms:i}.  Consider \ref{ak:re:SrWe:ms:ii}. It is sufficient to show that,
  $\{\hDi\in\nsetlo{\pDi,\ssY}\}\subseteq
  \{\Vnormlp{\txdfPr[\hDi]-\xdfPr[\hDi]}^2\geq\penSv[\hDi]/7\}$.  On the
  event $\{\hDi\in\nsetlo{\pDi,\ssY}\}$ holds $\hDi>\pDi\geq\pdDi$ and
  thus by definition \eqref{ak:de:*Di:ag}
  \begin{equation}\label{ak:re:SrWe:ms:pr:e4}
    \penSv[\hDi] > [6\Vnormlp{\So}^2+ 4\cpen] \daRaS{\pdDi}{\xdf,\Lambda}
  \end{equation}
  and due to \nref{re:contr} \ref{re:contr:e2} also
  \begin{equation}\label{ak:re:SrWe:ms:pr:e5}
    \Vnormlp{\txdfPr[\hDi]}^2-\Vnormlp{\txdfPr[\pdDi]}^2\leq
    \tfrac{7}{2}\Vnormlp{\txdfPr[\hDi]-\xdfPr[\hDi]}^2+\tfrac{3}{2}\Vnormlp{\xdf_{\underline{0}}}^2
    \{\bias[\pdDi]^2(\xdf)-\bias[\hDi]^2(\xdf)\}.
  \end{equation}
  Combining, \eqref{ak:re:SrWe:ms:pr:e1} and \eqref{ak:re:SrWe:ms:pr:e5} it
  follows that
  \begin{multline*}
    \tfrac{7}{2}\Vnormlp{\txdfPr[\hDi]-\xdfPr[\hDi]}^2\geq
    \penSv[\hDi]-\penSv[\pdDi]  -\tfrac{3}{2}\Vnormlp{\xdf_{\underline{0}}}^2
    \{\bias[\pdDi]^2(\xdf)-\bias[\hDi]^2(\xdf)\}\hfill
  \end{multline*}
  and hence together with $\bias[\hDi]^2(\xdf)\geq0$,
  \eqref{ak:re:SrWe:ms:pr:e4} and \nref{ak:ass:pen:oo} we obtain the claim,
  that is
  \begin{multline*}
    \tfrac{7}{2}\Vnormlp{\txdfPr[\hDi]-\xdfPr[\hDi]}^2\geq
    (\tfrac{1}{2}+\tfrac{1}{2})\penSv[\hDi]-\penSv[\pdDi]  -\tfrac{3}{2}\Vnormlp{\xdf_{\underline{0}}}^2
    \bias[\pdDi]^2(\xdf)\\
    >\tfrac{1}{2}\penSv[\hDi]+\tfrac{1}{2}[6\Vnormlp{\xdf_{\underline{0}}}^2+ 4\cpen]
    \daRaS{\pdDi}{\xdf,\Lambda}-\penSv[\pdDi]-\tfrac{3}{2}\Vnormlp{\xdf_{\underline{0}}}^2
    \bias[\pdDi]^2(\xdf)
    \geq\tfrac{1}{2}\penSv[\hDi],
  \end{multline*}
  which shows \ref{ak:re:SrWe:ms:ii} and completes the proof.\proEnd
\end{pro}



% ....................................................................
% <<Re rest>>
% ....................................................................
%\begin{lm}\label{ak:re:rest}Let $\DipenSv=\cmSv \Di \miSv$
%  with
%  $\sqrt{\cmiSv}=\tfrac{\log (\Di\miSv \vee
%    (\Di+2))}{\log(\Di+2)}\geq1$, then there is a numerical constant
%  $\cst{}$ such that for all $\ssY\in\Nz$ and
%  $\Di\in\nset{1,\ssY}$ hold
%  \begin{resListeN}[]
%  \item\label{ak:re:rest:i} let $\Di_{\ydf}:=\floor{  3(6\Vnormlp[1]{\fydf})^2}$ and $\ssY_{o}:={15(200)^4}$ then\\ 
%    $\sum_{\Di=1}^{\ssY}\E
%    \vectp{\Vnormlp{\txdfPr-\xdfPr}^2-12\DipenSv/\ssY}
%    \leq \cst{}\ssY^{-1}\big[\miSv[\Di_{\ydf}]\Di_{\ydf}+ \miSv[\ssY_{o}]\big]$
%  \item\label{ak:re:rest:ii} let
%    $\Di_{\ydf}:=\floor{3(800\Vnormlp[1]{\fydf})^2}$ and
%    $\ssY_{o}:=15({300})^4$ then\\
%    $\sum_{\Di=1}^{\ssY}\DipenSv\P\big(\Vnormlp{\txdfPr-\xdfPr}^2
%    \geq12\DipenSv/\ssY\big)\leq\cst{}\big[\miSv[\Di_{\ydf}]^2\Di_{\ydf}^2+\miSv[\ssY_{o}]^2\big]$
%  \item\label{ak:re:rest:iii} 
%  $\P\big(\Vnormlp{\txdfPr-\xdfPr}^2 \geq 12\daRaS{\Di}{\xdf,\Lambda}\big)\leq 
%    \cst{} \big[\exp\big(\tfrac{-\ssY\daRaS{\Di}{\xdf,\Lambda}}{200\Vnormlp[1]{\fydf}\miSv}\big)+\ssY^{-1}\big]$
%  \end{resListeN}
%\end{lm}
% ....................................................................
% <<Proof Re rest>>
% ....................................................................
%\begin{pro}[Proof of \nref{ak:re:rest}.]Consider \ref{ak:re:rest:i}.
%  Since $\cmiSv\geq1$ for
%  $\Di\geq3({6\Vnormlp[1]{\fydf}})^2$ holds
%  $\tfrac{\sqrt{\cmSv}\Di}{6\Vnormlp[1]{\fydf}}-\log(\Di+2)\geq0$
%  and%
%  \begin{multline*}
%    \miSv\exp\big(\tfrac{-\cmSv\Di}{3\Vnormlp[1]{\fydf}}\big)\leq
%    \exp\big(\tfrac{-\cmSv\Di}{6\Vnormlp[1]{\fydf}}\big)
%    \exp\big(-\sqrt{\cmSv}[\tfrac{\sqrt{\cmSv}\Di}{6\Vnormlp[1]{\fydf}}-\log(\Di+2)]\big)\\
%    \leq\exp\big(\tfrac{-\cmSv\Di}{6\Vnormlp[1]{\fydf}}\big)
%    \leq\exp\big(-\tfrac{1}{6\Vnormlp[1]{\fydf}}\Di\big)
%  \end{multline*}
%  consequently, for
%  $\Di_{\ydf}:=\floor{3({6\Vnormlp[1]{\fydf}})^2}$ then exploiting
%  $\sum_{\Di\in\Nz}\exp(-\mu\Di)\leq \mu^{-1}$ follows
%  \begin{displaymath}
%    \sum_{\Di=1+\Di_{\ydf}}^{\ssY}\miSv\exp\big(\tfrac{-\cmSv\Di}{3\Vnormlp[1]{\fydf}}\big)\leq
%    \sum_{\Di=1+\Di_{\ydf}}^{\ssY}\exp\big(-\tfrac{1}{6\Vnormlp[1]{\fydf}}\Di\big)
%    \leq {6\Vnormlp[1]{\fydf}}
%  \end{displaymath}
%  while
%  \begin{displaymath}
%   \sum_{\Di=1}^{\Di_{\ydf}}\miSv\exp\big(\tfrac{-\cmSv\Di}{3\Vnormlp[1]{\fydf}}\big)\leq
%   \miSv[\Di_{\ydf}]\sum_{\Di=1}^{\Di_{\ydf}}\exp\big(\tfrac{-\Di}{3\Vnormlp[1]{\fydf}}\big)
%   \leq \miSv[\Di_{\ydf}]{3\Vnormlp[1]{\fydf}}
% \end{displaymath}
% hence
% \begin{displaymath}
%   \sum_{\Di=1}^{\ssY}\miSv\exp\big(\tfrac{-\cmSv\Di}{3\Vnormlp[1]{\fydf}}\big)\leq
%   {6\Vnormlp[1]{\fydf}}+3\miSv[\Di_{\ydf}]\Vnormlp[1]{\fydf}\leq 9\miSv[\Di_{\ydf}]{\Vnormlp[1]{\fydf}}
% \end{displaymath}
% Using for all $ \ssY>\ssY_{o}:=15({200})^4$ holds 
% $\sqrt{n}\geq{200}\log(n+2)$ it follows for all $\Di\in\nset{1,n}$
% \begin{displaymath}
%   \tfrac{\Di\miSv}{\ssY}\exp\big(\tfrac{-\sqrt{n\cmSv}}{200}\big)
%   \leq
%   \tfrac{1}{\ssY}\exp\big(-\sqrt{\cmSv}[\tfrac{\sqrt{\ssY}}{200}-\log(\Di+2)]\big)\leq \tfrac{1}{\ssY}
% \end{displaymath}
% consequently, 
% \begin{equation*}
%   \sum_{\Di=1}^{\ssY}\tfrac{\Di\miSv}{\ssY}\exp\big(\tfrac{-\sqrt{n\cmSv}}{200}\big)
%   \leq\sum_{\Di=1}^{\ssY}\tfrac{1}{\ssY}\leq1
% \end{equation*}
% while for $\ssY\leq \ssY_{o}$ with
% $\miSv[\ssY]\leq\miSv[\ssY_{o}]$ follows
%\begin{equation*}
%   \sum_{\Di=1}^{\ssY}\tfrac{\Di\miSv}{\ssY}\exp\big(\tfrac{-\sqrt{\ssY\cmSv}}{200}\big)\leq 
%    \miSv[\ssY]\ssY\exp\big(\tfrac{-\sqrt{\ssY}}{200}\big)\leq\ssY_{o}\miSv[\ssY_{o}]
%  \end{equation*}
% consequently, for all $\ssY\in\Nz$ holds
% \begin{displaymath}
% \sum_{\Di=1}^{\ssY}\tfrac{\Di\miSv}{\ssY}\exp\big(\tfrac{-\sqrt{n\cmSv}}{200}\big)\leq \miSv[\ssY_{o}]\ssY_{o}
%\end{displaymath}
%Combining the last two bounds and \nref{re:conc} \ref{re:conc:i}  we obtain \ref{ak:re:rest:i}, that is 
%\begin{multline*}
%\sum_{\Di=1}^{\ssY}\E\vectp{\Vnormlp{\txdfPr[\Di]-\xdfPr[\Di]}^2-12\DipenSv/\ssY}\\\hfill\leq \cst{}\bigg[\tfrac{\Vnormlp[1]{\fydf}}{\ssY}\sum_{\Di=1}^{\ssY}
%\miSv\exp\big(\tfrac{-\cmSv\Di}{3\Vnormlp[1]{\fydf}}\big)+\tfrac{4}{n}\sum_{\Di=1}^{\ssY}\tfrac{\Di\miSv}{n}\exp\big(\tfrac{-\sqrt{n\cmSv}}{200}\big)
%\bigg]\\\leq \cst{}\ssY^{-1}\big[9\miSv[\Di_{\ydf}]{\Vnormlp[1]{\fydf}^2}+4 \miSv[\ssY_{o}]\ssY_{o}\big]
%\end{multline*}
%Consider  \ref{ak:re:rest:ii}. If $\Di\geq 3({400\Vnormlp[1]{\fydf}})^2$ then 
%$\Di\geq({400\Vnormlp[1]{\fydf}})\log(\Di+2)$ and
%hence
%$\Di-{200\Vnormlp[1]{\fydf}}\log(\Di+2)\geq{200\Vnormlp[1]{\fydf}}\log(\Di+2)$
%or equivalently,
%$\tfrac{\Di}{200\Vnormlp[1]{\fydf}}-\log(\Di+2)\geq\log(\Di+2)\geq1$
%and thus
%\begin{multline*}
%\Di\cmSv\miSv\exp\big(\tfrac{-\cmSv\Di}{200\Vnormlp[1]{\fydf}}\big)\leq
%\cmSv\exp\big(-\cmSv\,[\tfrac{\Di}{200\Vnormlp[1]{\fydf}}-\log(\Di+2)]\big)\\\leq
%(\Di+2)\exp\big(-\tfrac{\Di}{200\Vnormlp[1]{\fydf}}\big)% =
%% \\\cmSv\exp\big(-\tfrac{\cpen}{800\Vnormlp[1]{\fydf}}\cmSv\Di\big)\exp\big(-\sqrt{\cmSv}[\tfrac{\cpen\sqrt{\cmSv}}{800\Vnormlp[1]{\fydf}}\Di-\log(\Di+2)]\big)\leq\\
%\end{multline*}
%consequently, if $\Di>\Di_{\ydf}:=\floor{3({400\Vnormlp[1]{\fydf}})^2}$ exploiting $\sum_{\Di\in\Nz}(\Di+2)\exp(-\mu\Di)\leq \mu^{-2}+ 2\mu^{-1}$
%follows
%\begin{multline*}
%\sum_{\Di=1+\Di_{\ydf}}^{\ssY}\Di\cmSv\miSv\exp\big(\tfrac{-\cmSv\Di}{200\Vnormlp[1]{\fydf}}\big)\leq
%\sum_{\Di=1+\Di_{\ydf}}^{\ssY}(k+2)\exp\big(-\tfrac{\Di}{200\Vnormlp[1]{\fydf}}\big)
%\\\leq
%({200\Vnormlp[1]{\fydf}})^2+{400\Vnormlp[1]{\fydf}}\leq \Di_{\ydf}^2
%\end{multline*}
%while $\log(\Di\miSv)\leq \tfrac{1}{e}\Di\miSv$ implies
%$\cmSv\leq\Di\miSv$ it follows
%\begin{multline*}
%  \sum_{\Di=1}^{\Di_{\ydf}}\Di\cmSv\miSv\exp\big(\tfrac{-\cmSv\Di}{200\Vnormlp[1]{\fydf}}\big)\leq
%  \cmSv[\Di_{\ydf}]\miSv[\Di_{\ydf}]\sum_{\Di=1}^{\Di_{\ydf}}\Di\exp\big(\tfrac{-\Di}{200\Vnormlp[1]{\fydf}}\big)\\\leq
%  \cmSv[\Di_{\ydf}]\miSv[\Di_{\ydf}]({200\Vnormlp[1]{\fydf}})^2\leq\miSv[\Di_{\ydf}]^2\Di_{\ydf}^2
%\end{multline*}
%consequently for all $\ssY\in\Nz$ we have
%\begin{displaymath}
%  \sum_{\Di=1}^{\ssY}\Di\cmSv\miSv\exp\big(\tfrac{-\cmSv\Di}{200\Vnormlp[1]{\fydf}}\big)\leq(1+\miSv[\Di_{\ydf}]^2)\Di_{\ydf}^2\leq 2\miSv[\Di_{\ydf}]^2\Di_{\ydf}^2
%\end{displaymath}
%Since  $\cmSv\leq\Di\miSv$,
%and  for all $ \ssY>\ssY_{o}:=\floor{15({600})^4}$ holds $\sqrt{\ssY}\geq{600}\log(\ssY+2)$
%\begin{multline*}
%\Di\cmSv\miSv\exp\big(\tfrac{-\sqrt{\ssY\cmSv}}{200}\big)\leq
%\Di^2\miSv^2\exp\big(\tfrac{-\sqrt{\ssY\cmSv}}{200}\big)\\\leq
%\tfrac{1}{\ssY}\exp\big(-\sqrt{\cmSv}[\tfrac{\sqrt{\ssY}}{200}-2\log(\Di+2)]+\log(\ssY+2)\big)
%\leq\tfrac{1}{\ssY}\exp\big(-3\sqrt{\cmSv}[\tfrac{\sqrt{\ssY}}{600}-\log(\ssY+2)]\big)
%\\
%\leq \tfrac{1}{\ssY}
%  \end{multline*}
%consequently, 
%\begin{equation*}
%\sum_{\Di=1}^{\ssY}\Di\cmSv\miSv\exp\big(\tfrac{-\sqrt{\ssY\cmSv}}{200}\big)\leq\sum_{\Di=1}^{\ssY}\tfrac{1}{\ssY}\leq1
%\end{equation*}
%On the other hand side for $\ssY\leq\ssY_{o}$ with  $\ssY^b\exp(-a\ssY^{1/c})\leq (\tfrac{cb}{ea})^{cb}$ for all $c>0$ and $a,b\geq0$  follows
%\begin{multline*}
%\sum_{\Di=1}^{\ssY}\Di\cmSv\miSv\exp\big(\tfrac{-\sqrt{\ssY\cmSv}}{200}\big)\leq\ssY^2\cmSv[\ssY]\miSv[\ssY]\exp\big(\tfrac{-\sqrt{\ssY}}{200}\big)\leq
%\miSv[\ssY]^2\ssY^3\exp\big(\tfrac{-\sqrt{\ssY}}{200}\big)\\\leq \miSv[\ssY_{o}]^2\big({600}\big)^6\leq\miSv[\ssY_{o}]^2\ssY_{o}^2
%\end{multline*}
% consequently, for all $\ssY\in\Nz$ holds
% \begin{displaymath}
% \sum_{\Di=1}^{\ssY}\Di\cmSv\miSv\exp\big(\tfrac{-\sqrt{\ssY\cmSv}}{200}\big)\leq \miSv[\ssY_{o}]^2\ssY_{o}^2
%\end{displaymath}
%Combining the last two bounds and \nref{re:conc} \ref{re:conc:ii} we obtain \ref{ak:re:rest:ii}, that is 
%\begin{multline*}
%\sum_{\Di=1}^{\ssY}\cmiSv \Di \miSv\P\big(\Vnormlp{\txdfPr[\Di]-\xdfPr[\Di]}^2\geq12\DipenSv/\ssY\big)\\\hfill\leq 3\sum_{\Di=1}^{\ssY}\cmiSv \Di \miSv\bigg[\exp\big(\tfrac{-\cmSv\Di}{200\Vnormlp[1]{\fydf}}\big)+\exp\big(\tfrac{-\sqrt{\ssY\cmSv}}{200}\big)
%\bigg]
%\leq3\bigg[2\miSv[\Di_{\ydf}]^2\Di_{\ydf}^2+\miSv[\ssY_{o}]^2\ssY_{o}^2\bigg]
%\end{multline*}
%Consider \ref{ak:re:rest:iii}. Since
%$\tfrac{\ssY\sqrt{\daRa{\Di}{(\xdf,\Lambda)}}}{200\sqrt{\Di\miSv}}\geq\tfrac{\sqrt{\ssY\cmiSv}}{200}\geq\tfrac{\sqrt{\ssY}}{200}$
%and $\ssY\exp(-\tfrac{\sqrt{\ssY}}{200})\leq(200)^2$ 
%from \nref{re:conc} \ref{re:conc:iii} follows \ref{ak:re:rest:iii}, that is 
%\begin{multline*}
% \P\big(\Vnormlp{\txdfPr-\xdfPr}^2 \geq 12\daRa{\Di}{(\xdf,\Lambda)}\big)\leq 
%    3 \big[\exp\big(\tfrac{-\ssY\daRa{\Di}{(\xdf,\Lambda)}}{200\Vnormlp[1]{\fydf}\miSv}\big)
%    +\exp\big(\tfrac{-\ssY\sqrt{\daRa{\Di}{(\xdf,\Lambda)}}}{200\sqrt{\Di\miSv}}\big)\big]\\\leq 3 \big[\exp\big(\tfrac{-\ssY\daRa{\Di}{(\xdf,\Lambda)}}{200\Vnormlp[1]{\fydf}\miSv}\big)
%    +(200)^2\ssY^{-1}\big] 
%\end{multline*}
%which  completes the proof.\proEnd\end{pro}
% --------------------------------------------------------------------
% <<Proof Re ND rest>>
% --------------------------------------------------------------------
%\begin{pro}[Proof of \nref{ak:re:nd:rest}.]
%  Since $\cpen/7\geq 12$ and $\penSv/7\geq12\DipenSv\ssY^{-1}$,
%  $\Di\in\nset{1,n}$, by exploiting \nref{ak:re:rest}
%  \ref{ak:re:rest:i}, \ref{ak:re:rest:ii} and \ref{ak:re:rest:iii} we
%  obtain immediately the claim \ref{ak:re:nd:rest1},
%  \ref{ak:re:nd:rest2} and \ref{ak:re:nd:rest3}, respectively, which  completes the proof.
%\proEnd\end{pro}

%\begin{pro}[Proof of \nref{freq:ge:strat:kn:qu:ub}.]
%  Consider firstly the aggregation using the aggregation weights
%  $\rWe[]$ as in \eqref{freq:ge:shape:kn:we}.  Combining
%  \nref{freq:ge:strat:uk:qu:as} and the upper bound given in \eqref{freq:ge:strat:kn:qu:e1}
%  we obtain
%  \begin{multline}\label{ak:ag:ub:p1}
%    \E\Vnormlp{\txdfAg-\xdf}^2\leq \tfrac{2}{7}\penSv[\pDi]
%    +2\Vnormlp{\xdf_{\underline{0}}}^2\bias[\mDi]^2(\xdf)
%    \\\hfill
% + \cst{}\Vnormlp{\xdf_{\underline{0}}}^2\Ind{\{\mDi>1\}}\big[ \tfrac{1}{\rWc}
%     \exp\big(-18\rWc\ssY\daRaS{\mdDi}{\xdf,\Lambda}\big)+
% \exp\big(\tfrac{-1}{200\Vnormlp[1]{\fydf}}\ssY\daRaS{\mdDi}{\xdf,\Lambda}\miSv[\mdDi]^{-1}\big)\big]
% \\
% +\cst{}\big[\tfrac{1}{\rWc}+
% \Vnormlp{\xdf_{\underline{0}}}^2\Ind{\{\mDi>1\}}
%+\miSv[\Di_{\ydf}]^2\Di_{\ydf}^2+\miSv[\ssY_{o}]^2 \big]\ssY^{-1}
%  \end{multline}
% Moreover, since $1\geq\miSv[\mdDi]^{-1}$  it holds
%$\ssY\daRaS{\mdDi}{\xdf,\Lambda}\geq\ssY\daRaS{\mdDi}{\xdf,\Lambda}\miSv[\mdDi]^{-1}$. From
%\eqref{ak:ag:ub:p1} with $18\rWc>\tfrac{1}{200\Vnormlp[1]{\fydf}}$
%(since $\rWc\geq1$ and $\Vnormlp[1]{\fydf}\geq\vert\fydf[(0)]\vert=1$) follows
%  \begin{multline}\label{ak:ag:ub:p2}
%    \E\Vnormlp{\txdfAg-\xdf}^2\leq \tfrac{2}{7}\penSv[\pDi]
%    +2\Vnormlp{\xdf_{\underline{0}}}^2\bias[\mDi]^2(\xdf)
%    \\\hfill
% + \cst{}\Vnormlp{\xdf_{\underline{0}}}^2\Ind{\{\mDi>1\}}
% \exp\big(\tfrac{-1}{200\Vnormlp[1]{\fydf}}\ssY\daRaS{\mdDi}{\xdf,\Lambda}\miSv[\mdDi]^{-1}\big)
% \\
% +\cst{}\big[
% \Vnormlp{\xdf_{\underline{0}}}^2\Ind{\{\mDi>1\}}
%+\miSv[\Di_{\ydf}]^2\Di_{\ydf}^2+\miSv[\ssY_{o}]^2 \big]\ssY^{-1}.
%  \end{multline}
%  Consider secondly the aggregation using the model selection weights $\msWe[]$
%  as in \eqref{freq:ge:shape:kn:de:msWe}. Combining
%  \nref{freq:ge:strat:uk:qu:as} and the upper bound given in \label{freq:ge:strat:kn:qu:e2}
%  we obtain
%  \begin{multline}\label{ak:ag:ub:p3}
%    \E\Vnormlp{\txdfAg[{\msWe[]}]-\xdf}^2\leq \tfrac{2}{7}\penSv[\pDi]
%    +2\Vnormlp{\xdf_{\underline{0}}}^2\bias[\mDi]^2(\xdf)
%    \\\hfill
% + \cst{}\Vnormlp{\xdf_{\underline{0}}}^2\Ind{\{\mDi>1\}}
% \exp\big(\tfrac{-1}{200\Vnormlp[1]{\fydf}}\ssY\daRaS{\mdDi}{\xdf,\Lambda}\miSv[\mdDi]^{-1}\big)
% \\
% +\cst{}\big[
% \Vnormlp{\xdf_{\underline{0}}}^2\Ind{\{\mDi>1\}}
%+\miSv[\Di_{\ydf}]^2\Di_{\ydf}^2+\miSv[\ssY_{o}]^2 \big]\ssY^{-1}.
%  \end{multline}  
%From \eqref{ak:ag:ub:p2} and \eqref{ak:ag:ub:p3} together with
%$\ssY\daRaS{\mdDi}{\xdf,\Lambda}\miSv[\mdDi]^{-1}\geq\cmiSv[\mdDi]\mdDi$
%follows the claim, which  completes the proof.
%\proEnd\end{pro}


% --------------------------------------------------------------------
% <<Proof Re upper bound ag>>
% --------------------------------------------------------------------
\begin{pro}[Proof of \nref{ak:ag:ub}.]
  Consider firstly the aggregation using the aggregation weights
  $\rWe[]$ as in \eqref{ak:de:rWe}.  Combining
  \nref{ak:re:nd:rest} and the upper bound given in \eqref{co:agg:ag}
  we obtain
  \begin{multline}\label{ak:ag:ub:p1}
    \E\Vnormlp{\txdfAg-\xdf}^2\leq \tfrac{2}{7}\penSv[\pDi]
    +2\Vnormlp{\xdf_{\underline{0}}}^2\bias[\mDi]^2(\xdf)
    \\\hfill
 + \cst{}\Vnormlp{\xdf_{\underline{0}}}^2\Ind{\{\mDi>1\}}\big[ \tfrac{1}{\rWc}
     \exp\big(-18\rWc\ssY\daRaS{\mdDi}{\xdf,\Lambda}\big)+
 \exp\big(\tfrac{-1}{200\Vnormlp[1]{\fydf}}\ssY\daRaS{\mdDi}{\xdf,\Lambda}\miSv[\mdDi]^{-1}\big)\big]
 \\
 +\cst{}\big[\tfrac{1}{\rWc}+
 \Vnormlp{\xdf_{\underline{0}}}^2\Ind{\{\mDi>1\}}
+\miSv[\Di_{\ydf}]^2\Di_{\ydf}^2+\miSv[\ssY_{o}]^2 \big]\ssY^{-1}
  \end{multline}
 Moreover, since $1\geq\miSv[\mdDi]^{-1}$  it holds
$\ssY\daRaS{\mdDi}{\xdf,\Lambda}\geq\ssY\daRaS{\mdDi}{\xdf,\Lambda}\miSv[\mdDi]^{-1}$. From
\eqref{ak:ag:ub:p1} with $18\rWc>\tfrac{1}{200\Vnormlp[1]{\fydf}}$
(since $\rWc\geq1$ and $\Vnormlp[1]{\fydf}\geq\vert\fydf[(0)]\vert=1$) follows
  \begin{multline}\label{ak:ag:ub:p2}
    \E\Vnormlp{\txdfAg-\xdf}^2\leq \tfrac{2}{7}\penSv[\pDi]
    +2\Vnormlp{\xdf_{\underline{0}}}^2\bias[\mDi]^2(\xdf)
    \\\hfill
 + \cst{}\Vnormlp{\xdf_{\underline{0}}}^2\Ind{\{\mDi>1\}}
 \exp\big(\tfrac{-1}{200\Vnormlp[1]{\fydf}}\ssY\daRaS{\mdDi}{\xdf,\Lambda}\miSv[\mdDi]^{-1}\big)
 \\
 +\cst{}\big[
 \Vnormlp{\xdf_{\underline{0}}}^2\Ind{\{\mDi>1\}}
+\miSv[\Di_{\ydf}]^2\Di_{\ydf}^2+\miSv[\ssY_{o}]^2 \big]\ssY^{-1}.
  \end{multline}
  Consider secondly the aggregation using the model selection weights $\msWe[]$
  as in \eqref{ak:de:msWe}. Combining
  \nref{ak:re:nd:rest} and the upper bound given in \eqref{co:agg:ms}
  we obtain
  \begin{multline}\label{ak:ag:ub:p3}
    \E\Vnormlp{\txdfAg[{\msWe[]}]-\xdf}^2\leq \tfrac{2}{7}\penSv[\pDi]
    +2\Vnormlp{\xdf_{\underline{0}}}^2\bias[\mDi]^2(\xdf)
    \\\hfill
 + \cst{}\Vnormlp{\xdf_{\underline{0}}}^2\Ind{\{\mDi>1\}}
 \exp\big(\tfrac{-1}{200\Vnormlp[1]{\fydf}}\ssY\daRaS{\mdDi}{\xdf,\Lambda}\miSv[\mdDi]^{-1}\big)
 \\
 +\cst{}\big[
 \Vnormlp{\xdf_{\underline{0}}}^2\Ind{\{\mDi>1\}}
+\miSv[\Di_{\ydf}]^2\Di_{\ydf}^2+\miSv[\ssY_{o}]^2 \big]\ssY^{-1}.
  \end{multline}  
From \eqref{ak:ag:ub:p2} and \eqref{ak:ag:ub:p3} together with
$\ssY\daRaS{\mdDi}{\xdf,\Lambda}\miSv[\mdDi]^{-1}\geq\cmiSv[\mdDi]\mdDi$
follows the claim \eqref{ak:ag:ub:e1}, which  completes the proof.
\proEnd\end{pro}
% ....................................................................
% <<Pro upper bound ag p>>
% ....................................................................
\begin{pro}[Proof of \nref{re:cconc}.]
From
\eqref{ak:ag:ub:e1} follows for any $\mdDi,\pdDi\in\nset{1,n}$ and associated
$\mDi,\pDi\in\nset{1,n}$ as defined in  \eqref{ak:de:*Di:ag}%
 \begin{multline}\label{ak:ag:ub:pnp:p1}
     \E\Vnormlp{\txdfAg[{\erWe[]}]-\xdf}^2\leq \tfrac{2}{7}\penSv[\pDi]
    +2\Vnormlp{\xdf_{\underline{0}}}^2\bias[\mDi]^2(\xdf)
 + \cst{}\Vnormlp{\xdf_{\underline{0}}}^2\Ind{\{\mDi>1\}}\exp\big(\tfrac{-\cmiSv[\mdDi]\mdDi}{200\Vnormlp[1]{\fydf}}\big)\\
    +\cst{}\big[\Vnormlp{\xdf_{\underline{0}}}^2\Ind{\{\mDi>1\}}
+\miSv[\Di_{\ydf}]^2\Di_{\ydf}^2+\miSv[\ssY_{o}]^2 \big]\ssY^{-1}
\end{multline}
We destinguish the two cases \ref{ak:ag:ub:pnp:p} and
\ref{ak:ag:ub:pnp:np}. Consider first \ref{ak:ag:ub:pnp:p}, and hence there is $K\in\Nz_0$   with   $1\geq \bias[{[K-1] }](\xdf)>0$ and
$\bias(\xdf)=0$ for all $\Di\geq K$. Consider first $K=0$, then $\bias[0](\xdf)=0$
and hence $\Vnormlp{\xdf_{\underline{0}}}^2=0$. From \eqref{ak:ag:ub:pnp:p1}
follows 
 \begin{equation}\label{ak:ag:ub:pnp:p2}
     \E\Vnormlp{\txdfAg[{\erWe[]}]-\xdf}^2\leq \tfrac{2}{7}\penSv[\pDi]
    +\cst{}\big[\miSv[\Di_{\ydf}]^2\Di_{\ydf}^2+\miSv[\ssY_{o}]^2 \big]\ssY^{-1}
\end{equation}
Setting  $\pdDi:=1$ it follows from the definition
\eqref{ak:de:*Di:ag} of  $\pDi$ that
$\penSv[\pDi]\leq4\cpen\daRaS{1}{\xdf,\Lambda}$, where
$\daRaS{1}{\xdf,\Lambda}=\DipenSv[1]\ssY^{-1}$ and
$\DipenSv[1]=\cmSv[1] \miSv[1]\leq\miSv[1]^2$. Thereby with numerical
constant $\cpen\geq84$, \eqref{ak:ag:ub:pnp:p2} implies
 \begin{equation}\label{ak:ag:ub:pnp:p3}
     \E\Vnormlp{\txdfAg[{\erWe[]}]-\xdf}^2\leq\cst{}\big[\miSv[1]^2+\miSv[\Di_{\ydf}]^2\Di_{\ydf}^2+\miSv[\ssY_{o}]^2 \big]\ssY^{-1}
\end{equation}
Consider now $K\in\Nz$, and hence $\Vnormlp{\xdf_{\underline{0}}}^2>0$. Let 
$c_{\xdf}:=\tfrac{\Vnormlp{\xdf_{\underline{0}}}^2+4\cpen}{\Vnormlp{\xdf_{\underline{0}}}^2\bias[{[K-1]}]^2(\xdf)}>1$
and $\ssY_{\xdf}:=\ceil{c_{\xdf}\DipenSv[K])}\in\Nz$. We distinguish for $n\in\Nz$ the following two
 cases, \begin{inparaenum}[i]\renewcommand{\theenumi}{\dgrau\rm(\alph{enumi})}\item\label{ak:ag:ub:pnp:p:c1}
$\ssY\in\nset{1,\ssY_{\xdf}}$ and \item\label{ak:ag:ub:pnp:p:c2}
$\ssY> \ssY_{\xdf}$. \end{inparaenum} Firstly, consider
\ref{ak:ag:ub:pnp:p:c1} with $\ssY\in\nset{1,\ssY_{\xdf}}$, then setting $\mdDi:=1$, $\pdDi:=1$ we have
$\mDi=1$, $1\geq\bias[\mDi]$ and from the definition
\eqref{ak:de:*Di:ag} of  $\pDi$ also
$\penSv[\pDi]\leq2[3\Vnormlp{\xdf_{\underline{0}}}^2+2\cpen]\daRaS{1}{\xdf,\Lambda}\leq10\cpen\,[\Vnormlp{\xdf_{\underline{0}}}^2\vee1]\miSv[1]^2$
exploiting $\bias[1]\leq1\leq\DipenSv[1]=\cmSv[1]
\miSv[1]\leq\miSv[1]^2$. Thereby,  from \nref{ak:ag:ub:pnp:p1} 
follows
 \begin{multline*}
     \E\Vnormlp{\txdfAg[{\erWe[]}]-\xdf}^2\leq \tfrac{20}{7}\cpen(\Vnormlp{\xdf_{\underline{0}}}^2\vee1)\miSv[1]^2   +2\Vnormlp{\xdf_{\underline{0}}}^2
    +\cst{}\big[\miSv[\Di_{\ydf}]^2\Di_{\ydf}^2+\miSv[\ssY_{o}]^2
    \big]\ssY^{-1}\\
    \leq \cst{}\big[(\Vnormlp{\xdf_{\underline{0}}}^2\vee1)\miSv[1]^2\ssY+\miSv[\Di_{\ydf}]^2\Di_{\ydf}^2+\miSv[\ssY_{o}]^2\big]\ssY^{-1}
\end{multline*}
Moreover, for all $\ssY\in\nset{1,\ssY_{\xdf}}$ with
$\ssY_{\xdf}=\ceil{c_{\xdf}\DipenSv[K]}$ and
$\DipenSv[K]=K\cmSv[K] \miSv[K]\leq K^2\miSv[K]^2$ holds
$\ssY\leq\cst{}\tfrac{(\Vnormlp{\xdf_{\underline{0}}}^2\vee1)}{\Vnormlp{\xdf_{\underline{0}}}^2\bias[{[K-1]}]^2(\xdf)}
K^2\miSv[K]^2$and thereby, 
\begin{equation}\label{ak:ag:ub:pnp:p4}
  \E\Vnormlp{\txdfAg[{\erWe[]}]-\xdf}^2\leq
  \cst{}\big[(\Vnormlp{\xdf_{\underline{0}}}^2\vee1)\miSv[1]^2\tfrac{K^2\miSv[K]^2}{\Vnormlp{\xdf_{\underline{0}}}^2\bias[{[K-1]}]^2(\xdf)}+\miSv[\Di_{\ydf}]^2\Di_{\ydf}^2+\miSv[\ssY_{o}]^2\big]\ssY^{-1}.
\end{equation}
Secondly, consider \ref{ak:ag:ub:pnp:p:c2}, i.e., $\ssY>
\ssY_{\xdf}$. Setting
$\pdDi:=K< \ceil{c_{\xdf}\DipenSv[K]}=\ssY_{\xdf}$, i.e.,
$\pdDi\in\nset{1,\ssY}$, it follows $\bias[\pdDi](\xdf)=0$ and hence
$\daRaS{\pdDi}{\xdf,\Lambda}= \DipenSv[K] \ssY^{-1}$.  Therefore, the
definition \eqref{ak:de:*Di:ag} of $\pDi$ implies
$\pen(\pDi)\leq[6\Vnormlp{\xdf_{\underline{0}}}^2+ 4\cpen]\DipenSv[K]
\ssY^{-1}\leq
\cst{}(\Vnormlp{\xdf_{\underline{0}}}^2\vee1)K^2\miSv[K]^2\ssY^{-1}$. From
\eqref{ak:ag:ub:pnp:p1} follows for all $\ssY> \ssY_{\xdf}$ thus
\begin{multline}\label{ak:ag:ub:pnp:p5}
  \E\Vnormlp{\txdfAg[{\erWe[]}]-\xdf}^2\leq 2\Vnormlp{\xdf_{\underline{0}}}^2\bias[\mDi]^2(\xdf)
    + \cst{}\Vnormlp{\xdf_{\underline{0}}}^2\Ind{\{\mDi>1\}}
    \exp\big(\tfrac{-\cmiSv[\mdDi]\mdDi}{200\Vnormlp[1]{\fydf}}\big)\\
    +\cst{}\big[(\Vnormlp{\xdf_{\underline{0}}}^2\vee1)K^2\miSv[K]^2
    +\miSv[\Di_{\ydf}]^2\Di_{\ydf}^2+\miSv[\ssY_{o}]^2 \big]\ssY^{-1}.
\end{multline}
Since
$\ssY> \ssY_{\xdf}:=\ceil{c_{\xdf}\DipenSv[K]}$ with
$c_{\xdf}=\tfrac{\Vnormlp{\xdf_{\underline{0}}}^2+4\cpen}{\Vnormlp{\xdf_{\underline{0}}}^2\bias[{[K-1]}]^2(\xdf)}>1$
the defining set of
$\sDi{\ssY}:=\max\{\Di\in\nset{K,\ssY}:\ssY>c_{\xdf}\DipenSv\}$
evenutally containing $K$ is not empty. Consequently,  $\sDi{\ssY}\geq
K$ and, hence 
$\bias[\sDi{\ssY}](\xdf)=0$, and
$\daRaS{\sDi{\ssY}}{\xdf,\Lambda}=\DipenSv[\sDi{\ssY}]\ssY^{-1}<c_{\xdf}^{-1}=\tfrac{\Vnormlp{\xdf_{\underline{0}}}^2\bias[{[K-1]}]^2(\xdf)}{\Vnormlp{\xdf_{\underline{0}}}^2+4\cpen}$,
it follows
$\Vnormlp{\xdf_{\underline{0}}}^2\bias[{[K-1]}]^2(\xdf)>[\Vnormlp{\xdf_{\underline{0}}}^2+4\cpen]\daRaS{\sDi{\ssY}}{\xdf,\Lambda}$
and trivially
$\Vnormlp{\xdf_{\underline{0}}}^2\bias[{K}]^2(\xdf)=0<[\Vnormlp{\xdf_{\underline{0}}}^2+4\cpen]\daRaS{\sDi{\ssY}}{\xdf,\Lambda}$. Therefore,
setting $\mdDi:=\sDi{\ssY}$ the definition \eqref{ak:de:*Di:ag}
implies $\mDi=K$ and hence
$\bias[\mDi]^2(\xdf)=\bias[K]^2(\xdf)=0$. From \eqref{ak:ag:ub:pnp:p5}  follows
now for all $\ssY> \ssY_{\xdf}$ thus
\begin{multline}\label{ak:ag:ub:pnp:p6}
  \E\Vnormlp{\txdfAg[{\erWe[]}]-\xdf}^2\leq  \cst{}\Vnormlp{\xdf_{\underline{0}}}^2\exp\big(\tfrac{-1}{200\Vnormlp[1]{\fydf}}\cmiSv[\sDi{\ssY}]\sDi{\ssY}\big)\\
  +\cst{}\big[(\Vnormlp{\xdf_{\underline{0}}}^2\vee1)K^2\miSv[K]^2
  +\miSv[\Di_{\ydf}]^2\Di_{\ydf}^2+\miSv[\ssY_{o}]^2 \big]\ssY^{-1}.
\end{multline}
Combining \eqref{ak:ag:ub:pnp:p4} and
    \eqref{ak:ag:ub:pnp:p6}  for $K\geq1$ with \ref{ak:ag:ub:pnp:p:c1}
$\ssY\in\nsetro{1,\ssY_{\xdf}}$ and \ref{ak:ag:ub:pnp:p:c2}
$\ssY\geq \ssY_{\xdf}$, respectively, and \eqref{ak:ag:ub:pnp:p3}  for
$K=0$ implies for all $K\in\Nz_0$ and for all $\ssY\in\Nz$ the claim
\eqref{ak:ag:ub:pnp:e1} in case \ref{ak:ag:ub:pnp:p}, that is
\begin{multline}\label{ak:ag:ub:pnp:p7}
  \E\Vnormlp{\txdfAg[{\erWe[]}]-\xdf}^2\leq
  \cst{}\Vnormlp{\xdf_{\underline{0}}}^2\big[  \ssY^{-1}\vee\exp\big(\tfrac{-\cmiSv[\sDi{\ssY}]\sDi{\ssY}}{200\Vnormlp[1]{\fydf}}\big)\big]\\
  +\cst{}\big[\miSv[1]^2\{\tfrac{(\Vnormlp{\xdf_{\underline{0}}}^2\vee1)K^2\miSv[K]^2}{\Vnormlp{\xdf_{\underline{0}}}^2\bias[{[K-1]}]^2(\xdf)}\Ind{K\geq1}+\Ind{K=0}\}
  +\miSv[\Di_{\ydf}]^2\Di_{\ydf}^2+\miSv[\ssY_{o}]^2 \big]\ssY^{-1}.
\end{multline}
Consider the case \ref{ak:ag:ub:pnp:np}. For $\aDi{\ssY}(\xdf)\in\nset{1,n}$
as in \ref{ak:ass:pen:oo}  set $\pdDi:=\aDi{\ssY}(\xdf)$ and $\mdDi:=\sDi{\ssY}\in\nset{\aDi{\ssY}(\xdf),\ssY}$ by exploiting the definition
\eqref{ak:de:*Di:ag} of $\pDi$ and $\mDi$ it follows
$\penSv[\pDi] \leq 2[3\Vnormlp{\xdf_{\underline{0}}}^2+ 2\cpen]
\daRa{\pdDi}{(\xdf)}$ and 
$\Vnormlp{\xdf_{\underline{0}}}^2\bias[\mDi]^2(\xdf)\leq
      [\Vnormlp{\xdf_{\underline{0}}}^2+4\cpen]\daRa{\sDi{\ssY}}{(\xdf)}$
which together with
$\daRa{\sDi{\ssY}}{(\xdf)}\geq\naRa{(\xdf)}=\daRa{\pdDi}{(\xdf)}\geq\ssY^{-1}$
and exploiting 
\eqref{ak:ag:ub:pnp:p1} implies%
 \begin{multline}\label{ak:ag:ub:pnp:p8}
   \E\Vnormlp{\txdfAg[{\erWe[]}]-\xdf}^2% \leq
 %   \tfrac{2}{7}\penSv[\pDi]
 %    +2\Vnormlp{\xdf_{\underline{0}}}^2\bias[\mDi]^2(\xdf)
 % + \cst{}\Vnormlp{\xdf_{\underline{0}}}^2\Ind{\{\mDi>1\}}\exp\big(\tfrac{-\cmiSv[\mdDi]\mdDi}{200\Vnormlp[1]{\fydf}}\big)\\
 %    +\cst{}\big[\Vnormlp{\xdf_{\underline{0}}}^2\Ind{\{\mDi>1\}}
 %    +\miSv[\Di_{\ydf}]^2\Di_{\ydf}^2+\miSv[\ssY_{o}]^2 \big]\ssY^{-1}\\
    \leq 
   \cst{}(\Vnormlp{\xdf_{\underline{0}}}^2\vee1)\big[\daRa{\sDi{\ssY}}{(\xdf,\Lambda)}\vee\exp\big(\tfrac{-\cmiSv[\sDi{\ssY}]\sDi{\ssY}}{200\Vnormlp[1]{\fydf}}\big)\big]\\
   +\cst{}\big[\miSv[\Di_{\ydf}]^2\Di_{\ydf}^2+\miSv[\ssY_{o}]^2 \big]\ssY^{-1}
\end{multline}

which shows the assertion \eqref{ak:ag:ub:pnp:e2} and  completes the
proof of \nref{ak:ag:ub:pnp}.\proEnd\end{pro}
% ....................................................................
% <<Pro upper bound ag p>>
% ....................................................................
\begin{pro}[Proof of \nref{ak:ag:ub2:pnp}.]
  Consider the case \ref{ak:ag:ub2:pnp:p}.  If the additional
  assumption \ref{ak:ag:ub2:pnp:pc} is satisfied, then we have
  trivially
  $\exp\big(\tfrac{-\cmiSv[\sDi{\ssY}]\sDi{\ssY}}{\Di_{\ydf}}\big)\leq\ssY^{-1}$
  $\ssY> \ssY_{\xdf,\Lambda}$ while for
  $\ssY\in\nset{1,\ssY_{\xdf,\Lambda}}$ we have
  $\exp\big(\tfrac{-\cmiSv[\sDi{\ssY}]\sDi{\ssY}}{\Di_{\ydf}}\big)\leq1\leq
  \ssY_{\xdf,\Lambda} \ssY^{-1}$. Thereby, from \eqref{ak:ag:ub:pnp:e1}
  follows immediately the assertion
  $\nRi{\txdfAg[{\erWe[]}]}{\xdf,\Lambda} \leq \cst{\xdf,\Lambda}\ssY^{-1}$
  for all $\ssY\in\Nz$. On the other hand side, in case
  \ref{ak:ag:ub2:pnp:np} if the additional assumption
  \ref{ak:ag:ub2:pnp:npc} is satisfied, then we have trivially
  $\exp\big(\tfrac{-\cmiSv[\aDi{\ssY}(\xdf)]\aDi{\ssY}(\xdf)}{\Di_{\ydf}}\big)\leq
  \naRa{(\xdf,\Lambda)}$ while for $\ssY< \ssY_{\xdf,\Lambda}$ we have
  $\exp\big(\tfrac{-\cmiSv[\aDi{\ssY}(\xdf)]\aDi{\ssY}(\xdf)}{\Di_{\ydf}}\big)\leq1\leq
  \ssY\naRa{(\xdf,\Lambda)}\leq \ssY_{\xdf,\Lambda}
  \naRaS{(\xdf,\Lambda)}$. Thereby, from \eqref{ak:ag:ub:pnp:e2} with
  $\min_{\Di\in\nset{1,\ssY}}\dRa{\Di}{\xdf,\Lambda}=\naRa{(\xdf,\Lambda)}$
  follows immediately
  $ \nRi{\txdfAg[{\erWe[]}]}{\xdf,\Lambda} \leq \cst{\xdf,\Lambda}
  \naRa{(\xdf,\Lambda)}$ for all $\ssY\in\Nz$, which completes the proof of
  \nref{ak:ag:ub2:pnp}.\proEnd
  \end{pro}
\section{Proofs of \nref{ak:mrb}}\label{a:ak:mrb}
% ....................................................................
% Te <<Upper bound random weights>>
% ....................................................................
\begin{te}
 Below  we state the proofs of  \nref{ak:re:SrWe:ag:mm} and \nref{ak:re:SrWe:ms:mm}. The
  proof of \nref{ak:re:SrWe:ag} is based on \nref{re:rWe:mm} given first.
\end{te}
% ....................................................................
% <<Upper bound random weights>>
% ....................................................................
\begin{lm}\label{re:rWe:mm} Consider the data-driven aggregation weights
  $\rWe[]$ as in \eqref{ak:de:rWe}. Under definition
  \ref{ak:ass:pen:oo} for any $l\in\nset{1,\ssY}$ with
  $\daRaS{l}{\xdfCw[],\Lambda}:=\daRa{l}{(\xdfCw[])}$ holds
  \begin{resListeN}[]
  \item\label{re:rWe:mm:i} for all $k\in\nsetro{1,l}$ we have\\
    $\rWe\Ind{\setB{\Vnormlp{\txdfPr[l]-\xdfPr[l]}^2<\cpen\daRa{l}{(\xdfCw[])}/7}} 
    \leq\exp\big(\rWn\big\{-\tfrac{\Vnormlp{\xdf_{\underline{0}}}^2}{2}\bias^2(\xdf)
    +[\tfrac{25\cpen}{14}+\tfrac{\Vnormlp{\xdf_{\underline{0}}}^2}{2}]\daRaS{l}{\xdfCw[],\Lambda}-\penSv\big\}\big)$
  \item\label{re:rWe:mm:ii} for all $\Di\in\nsetlo{l,\ssY}$ we have\\
    $\rWe\Ind{\setB{\Vnormlp{\txdfPr-\xdfPr}^2<\penSv/7}}\leq\exp\big(\rWn\big\{-\tfrac{1}{2}\penSv
    +[\tfrac{3}{2}\Vnormlp{\xdf_{\underline{0}}}^2+\cpen]\daRaS{l}{\xdfCw[],\Lambda}\big\}\big)$.
  \end{resListeN}
\end{lm}
% --------------------------------------------------------------------
% <<Proof Re Random weights>> angepasst
% --------------------------------------------------------------------
\begin{pro}[Proof of \nref{re:rWe:mm}.]
The proof follows line by line the proof of \nref{re:rWe} using
\eqref{ak:ass:pen:mm:c} rather than \eqref{freq:ge:strat:kn:qu:de:rate:e1}, and we omit the details.\proEnd
\end{pro}
% ....................................................................
% <<Proof Re Sum Random weights>>
% ....................................................................
\begin{pro}[Proof of \nref{ak:re:SrWe:ag:mm}.]
The proof follows line by line the proof of \nref{ak:re:SrWe:ag} using
\nref{re:rWe:mm} rather than \nref{re:rWe}, and we omit the details.\proEnd  
\end{pro}
% --------------------------------------------------------------------
% Proof <<Upper bound random weights>> ms mm
% --------------------------------------------------------------------
\begin{pro}[Proof of \nref{ak:re:SrWe:ms:mm}.]
The proof follows line by line the proof of \nref{ak:re:SrWe:ms} using
\eqref{ak:ass:pen:mm:c} rather than \eqref{freq:ge:strat:kn:qu:de:rate:e1}, and we omit the details.\proEnd
\end{pro}
% ....................................................................
% <<Pro upper bound ag p>>
% ....................................................................
\begin{pro}[Proof of \nref{ak:ag:ub:pnp:mm}.]Keep in mind that
  $\Vnormlp{\xdf_{\underline{0}}}^2\leq\xdfCr^2$ for all $\xdf\in\rwCxdf$.
From
\eqref{ak:ag:ub:mm:e1} follows for any $\xdf\in\rwCxdf$, $\mdDi,\pdDi\in\nset{1,n}$ and associated
$\mDi,\pDi\in\nset{1,n}$ as defined in  \eqref{freq:ge:strat:kn:ma:de:mDipDi}%
 \begin{multline}\label{ak:ag:ub:pnp:mm:p1}
    \E\Vnormlp{\txdfAg[{\erWe[]}]-\xdf}^2\leq \tfrac{2}{7}\penSv[\pDi]
    +2\Vnormlp{\xdf_{\underline{0}}}^2\bias[\mDi]^2(\xdf)
    %\\\hfill
 + \cst{}\xdfCr^2
 \exp\big(\tfrac{-\cmiSv[\mdDi]\mdDi}{200\Vnorm[{\xdfCw[]}]{\edf}\xdfCr}\big)%\big]
\\ +\cst{}\big[%\tfrac{1}{\rWc}+
 \xdfCr^2+\miSv[\Di_{\edf,\xdfCr}]^2\Di_{\edf,\xdfCr}^2+\miSv[\ssY_{o}]^2 \big]\ssY^{-1}% \\
\end{multline}
 For $\aDi{\ssY}(\xdfCw[])\in\nset{1,n}$
as in \ref{ak:ass:pen:mm}  set $\pdDi:=\aDi{\ssY}(\xdfCw[])$ and $\mdDi:=\sDi{\ssY}\in\nset{\aDi{\ssY}(\xdfCw[]),\ssY}$ by exploiting the definition
\eqref{freq:ge:strat:kn:ma:de:mDipDi} of $\pDi$ and $\mDi$ it follows
$\penSv[\pDi] \leq 2[3\xdfCr^2+ 2\cpen]
\daRa{\pdDi}{(\xdfCw[])}$ and 
$\Vnormlp{\xdf_{\underline{0}}}^2\bias[\mDi]^2(\xdf)\leq
      [\xdfCr^2+4\cpen]\daRa{\sDi{\ssY}}{(\xdfCw[])}$
which together with
$\daRa{\sDi{\ssY}}{(\xdfCw[])}\geq\naRa{(\xdfCw[])}=\daRa{\pdDi}{(\xdfCw[])}\geq\ssY^{-1}$
and exploiting 
\eqref{ak:ag:ub:pnp:mm:p1} implies%
 \begin{multline}\label{ak:ag:ub:pnp:mm:p2}
   \sup_{\xdf\in\rwCxdf}\E\Vnormlp{\txdfAg[{\erWe[]}]-\xdf}^2% \leq
 %   \tfrac{2}{7}\penSv[\pDi]
 %    +2\Vnormlp{\xdf_{\underline{0}}}^2\bias[\mDi]^2(\xdf)
 % + \cst{}\Vnormlp{\xdf_{\underline{0}}}^2\Ind{\{\mDi>1\}}\exp\big(\tfrac{-\cmiSv[\mdDi]\mdDi}{200\Vnormlp[1]{\fydf}}\big)\\
 %    +\cst{}\big[\Vnormlp{\xdf_{\underline{0}}}^2\Ind{\{\mDi>1\}}
 %    +\miSv[\Di_{\ydf}]^2\Di_{\ydf}^2+\miSv[\ssY_{o}]^2 \big]\ssY^{-1}\\
    \leq 
   \cst{}(\xdfCr^2\vee1)\min_{\Di\in\nset{1,\ssY}}\big[\daRa{\Di}{(\xdfCw[])}\vee\exp\big(\tfrac{-\cmiSv\Di}{200\Vnorm[{\xdfCw[]}]{\edf}\xdfCr}\big)\big]\\
   +\cst{}\big[\miSv[\Di_{\edf,\xdfCr}]^2\Di_{\edf,\xdfCr}^2+\miSv[\ssY_{o}]^2 \big]\ssY^{-1}
\end{multline}
which shows the assertion \eqref{ak:ag:ub:pnp:mm:e1} and  completes the
proof of \nref{ak:ag:ub:pnp:mm}.\proEnd\end{pro}
% ....................................................................
% <<Pro upper bound ag p>>
% ....................................................................
\begin{pro}[Proof of \nref{ak:ag:ub2:pnp:mm}.] Under
  \ref{ak:ag:ub2:pnp:mm:c} holds  
  $\exp\big(\tfrac{-\cmiSv[\aDi{\ssY}({\xdfCw[]})]\aDi{\ssY}({\xdfCw[]})}{200\Vnorm[{\xdfCw[]}]{\edf}\xdfCr}\big)\leq
  \naRa{(\xdfCw[])}$  for $\ssY> \ssY_{\xdfCw[],\xdfCr,\Lambda}$, while 
  $\exp\big(\tfrac{-\cmiSv[\aDi{\ssY}({\xdfCw[]})]\aDi{\ssY}({\xdfCw[]})}{200\Vnorm[{\xdfCw[]}]{\edf}\xdfCr}\big)\leq1\leq
  \ssY\naRa{(\xdfCw[])}\leq \ssY_{\xdfCw[],\xdfCr,\Lambda}
  \naRa{{(\xdfCw[])}}$ for $\ssY\in\nset{1,\ssY_{\xdfCw[],\xdfCr,\Lambda}}$. Thereby, from
  \eqref{ak:ag:ub:pnp:mm:e1} with $\sDi{\ssY}:=\aDi{\ssY}({\xdfCw[]})$
  follows immediately the assertion
    $\nRi{\txdfAg[{\erWe[]}]}{\rwCxdf,\Lambda}
    \leq \cst{\xdfCw[],\xdfCr,\Lambda} \naRa{\xdfCw[],\Lambda}$ for all
    $\ssY\in\Nz$, which  completes the
proof of \nref{ak:ag:ub2:pnp:mm}.\proEnd\end{pro}
\chapter{Proof for \nref{THM_FREQ_CIRCDECONV_KNOWN_IID_ORACLE_NP}}\label{PRO_FREQ_CIRCDECONV_KNOWN_IID_ORACLE_NP}
%======================================================================================================================
%                                                                 
% Title:  Appendix:  known error density
% Author: Jan JOHANNES, Institut für Angewandte Mathematik, Ruprecht-Karls Universität Heidelberg, Deutschland  
% 
% Email: johannes@math.uni-heidelberg.de
% Date: %%ts latex start%%[2018-03-29 Thu 13:22]%%ts latex end%%
%
% ======================================================================================================================
% --------------------------------------------------------------------
% section <<Appendix: Proofs of \cref{ak}>>\ref{a:ak}
% --------------------------------------------------------------------
\subsection{Proofs of \cref{ak}}\label{a:ak}
\begin{te}
For each
  $\Di\in\Nz$ the projection $\xdfPr=\sum_{j=-\Di}^{\Di}\fxdf[j]\bas_j$ and
  the  orthogonal series estimator
  $\txdfPr=\sum_{j=1}^{\Di}\fedfI[j]\hfydf[j]\bas_j$  is constructed, respectively, using
the sequences  $\fxdf=\Nsuite{\fxdf[j]}$, $\fedfI=\Nsuite{\fedfI[j]}$ and  $\hfydf=\Nsuite{\hfydf[j]}$. %    In this section considering the data-driven aggregation weights and
  % the model selection weights.
\end{te}
% --------------------------------------------------------------------
% <<Proof of Re key argument>>
% --------------------------------------------------------------------
\begin{pro}[Proof of \cref{co:agg}.]
We start the proof with the observation that
$\oftxdf{j}-\ofxdf[j]=\ftxdf{-j}-\fxdf[-j]$ for all $j\in\Zz$ and 
\begin{multline*}
  \ftxdf{j}-\fxdf[j]=\fedfI[j](\hfydf[j]-\fydf[j])\FuVg{\We[]}(\nset{j,\ssY})-\fxdf[j]\FuVg{\We[]}(\nsetro{1,j})\text{ for all }j\in\nset{1,\ssY},\\\ftxdf{0}-\fxdf[0]=0\text{ and }\ftxdf{j}-\fxdf[j]=-\fxdf[j]\text{ for all }j>\ssY.
\end{multline*}
Consequently, (keep in mind that $|\fedfI[j]|^2=\iSv[j]$)  we  have
  \begin{multline}\label{co:agg:pro1}
    \VnormLp{\txdf-\xdf}^2=
   2\sum_{j\in\nset{1,\ssY}}|\fedfI[j](\hfydf[j]-\fydf[j])\FuVg{\We[]}(\nset{j,\ssY})-\fxdf[j]\FuVg{\We[]}(\nsetro{1,j})|^2+2\sum_{j>\ssY}|\fxdf[j]|^2\\
\leq
   \sum_{j\in\nset{1,\ssY}}4\{\iSv[j]|\hfydf[j]-\fydf[j]|^2 \FuVg{\We[]}(\nset{j,\ssY})\} + \sum_{j\in\nset{1,\ssY}}4|\fxdf[j]|^2\FuVg{\We[]}(\nsetro{1,j})+2\sum_{j>n}|\fxdf[j]|^2,%\\
% \leq 2\{n^{-1} \peDi \oEvs[\peDi]+6\Evs_1\exp(-\DiMa/3)+6\Evs_1\exp
%   \big(-\frac{n\dnRa}{2}+ 2\log \DiMa \big)\}\\
% + 2\{\gb_{\meDi}^2 +2\VnormLp{\xdf}^2\exp\big(-\frac{n\dnRa}{2}+ 2\log \DiMa \big)\}.
 \end{multline}
where we consider the first r.h.s and the two other r.h.s. terms
separatly. Consider the first r.hs. term in \eqref{co:agg:pro1}. We split the sum into two parts which we bound separately.  Precisely,
\begin{multline}\label{co:agg:pro2}
2\sum_{j\in\nset{1,\ssY}}\iSv[j]|\hfydf[j]-\fydf[j]|^2
\FuVg{\We[]}(\nset{j,\ssY})\\
% \leq 2\sum_{j\in\nset{1,\pDi}}\iSv[j]|\hfydf[j]-\fydf[j]|^2 +
% 2\sum_{j\in\nsetlo{\pDi,\ssY}}\iSv[j]|\hfydf[j]-\fydf[j]|^2\sum_{l\in\nset{j,\ssY}}\We[l]\\
% = 2\sum_{j\in\nset{1,\pDi}}\iSv[j]|\hfydf[j]-\fydf[j]|^2 +
% \sum_{l\in\nsetlo{\pDi,\ssY}}\We[l]\;2\sum_{j\in\nsetlo{\pDi,l}}\iSv[j]|\hfydf[j]-\fydf[j]|^2\\
\leq \VnormLp{\txdfPr[\pDi]-\xdfPr[\pDi]}^2
+\sum_{l\in\nsetlo{\pDi,\ssY}}\We[l]\VnormLp{\txdfPr[l]-\xdfPr[l]}^2\\
% \leq\VnormLp{\txdfPr[\pDi]-\xdfPr[\pDi]}^2
% +\sum_{l\in\nsetlo{\pDi,\ssY}}\We[l]\VnormLp{\txdfPr[l]-\xdfPr[l]}^2\Ind{\{\VnormLp{\txdfPr[l]-\xdfPr[l]}^2\geq\penSv[l]\}}
% \\
% \hfill+(12\cpen/\ssY)\sum_{l\in\nsetlo{\pDi,\ssY}}\DipenSv[l]\We[l]\Ind{\{\VnormLp{\txdfPr[l]-\xdfPr[l]}^2<\penSv[l]\}}\\
% =\VnormLp{\txdfPr[\pDi]-\xdfPr[\pDi]}^2
% +\sum_{l\in\nsetlo{\pDi,\ssY}}\We[l]\vect{\VnormLp{\txdfPr[l]-\xdfPr[l]}^2-\cst{1}\penSv[l]}\Ind{\{\VnormLp{\txdfPr[l]-\xdfPr[l]}^2\geq\penSv[l]\}}\\
% \hfill+(12\cst{1}\cpen/\ssY)\sum_{l\in\nsetlo{\pDi,\ssY}}\We[l]\DipenSv[l]\Ind{\{\VnormLp{\txdfPr[l]-\xdfPr[l]}^2\geq\penSv[l]\}}
%  +(12\cpen/\ssY)\sum_{l\in\nsetlo{\pDi,\ssY}}\DipenSv[l]\We[l]\Ind{\{\VnormLp{\txdfPr[l]-\xdfPr[l]}^2<\penSv[l]\}}\\
\leq\VnormLp{\txdfPr[\pDi]-\xdfPr[\pDi]}^2
+\sum_{l\in\nsetlo{\pDi,\ssY}}\We[l]\vectp{\VnormLp{\txdfPr[l]-\xdfPr[l]}^2-\pen[l]/7}\\
+\tfrac{1}{7}\sum_{l\in\nsetlo{\pDi,\ssY}}\We[l]\pen[l]\Ind{\{\VnormLp{\txdfPr[l]-\xdfPr[l]}^2\geq\pen[l]/7\}}
+\tfrac{1}{7}\sum_{l\in\nsetlo{\pDi,\ssY}}\pen[l]\We[l]\Ind{\{\VnormLp{\txdfPr[l]-\xdfPr[l]}^2<\pen[l]/7\}}\\
\leq\tfrac{1}{7}\pen[\pDi]
+\sum_{l\in\nset{\pDi,\ssY}}\vectp{\VnormLp{\txdfPr[l]-\xdfPr[l]}^2-\pen[l]/7}\\
+\tfrac{1}{7}\sum_{l\in\nsetlo{\pDi,\ssY}}\We[l]\pen[l]\Ind{\{\VnormLp{\txdfPr[l]-\xdfPr[l]}^2\geq\pen[l]/7\}}
+\tfrac{1}{7}\sum_{l\in\nsetlo{\pDi,\ssY}}\pen[l]\We[l]\Ind{\{\VnormLp{\txdfPr[l]-\xdfPr[l]}^2<\pen[l]/7\}}%\\
\end{multline}
Consider the second and third r.hs. term in \eqref{co:agg:pro1}.  Splitting the first sum into two parts we obtain
\begin{multline}\label{co:agg:pro3}
2\sum_{j\in\nset{1,\ssY}}|\fxdf[j]|^2\FuVg{\We[]}(\nsetro{1,j})+2\sum_{j>\ssY}|\fxdf[j]|^2\\
\hspace*{5ex}\leq  2\sum_{j\in\nset{1,\mDi}}|\fxdf[j]|^2\FuVg{\We[]}(\nsetro{1,j})+ 2\sum_{j\in\nsetlo{\mDi,n}}|\fxdf[j]^2+
  2\sum_{j>n}|\fxdf[j]|^2\\\hfill
\leq \VnormLp{\Proj[{\mHiH[0]^\perp}]\xdf}^2\{\FuVg{\We[]}(\nsetro{1,\mDi})+\bias[\mDi]^2(\xdf)\}
\end{multline}
Combining  \eqref{co:agg:pro1} and the upper bounds \eqref{co:agg:pro2}
and \eqref{co:agg:pro3} we obtain   the assertion, which completes the proof.\proEnd
\end{pro}
\subsubsection{Proofs of \cref{ak:rb}}\label{a:ak:rb}
% ....................................................................
% Te <<Upper bound random weights>>
% ....................................................................
\begin{te}
 Below  we state the proofs of  \cref{ak:re:SrWe:ag} and \cref{ak:re:SrWe:ms}. The
  proof of \cref{ak:re:SrWe:ag} is based on \cref{re:rWe} given first.
\end{te}
% ....................................................................
% <<Re Random weights>>
% ....................................................................
\begin{lm}\label{re:rWe} Consider the data-driven aggreagtion weights
  $\rWe[]$ as in \eqref{ak:de:rWe}. Under condition
  \ref{ak:ass:pen:oo} for any $l\in\nset{1,\ssY}$ with
  $\daRaS{l}{\xdf,\iSv}:=\daRa{l}{\xdf,\iSv}$ holds
  \begin{resListeN}[]
  \item\label{re:rWe:i} for all $k\in\nsetro{1,l}$ we have\\
    $\rWe\Ind{\setB{\VnormLp{\txdfPr[l]-\xdfPr[l]}^2<\cpen\daRaS{l}{\xdf,\iSv}/7}} 
    \leq\exp\big(\rWn\big\{-\tfrac{\VnormLp{\ProjC[0]\xdf}^2}{2}\bias^2(\xdf)
    +[\tfrac{25\cpen}{14}+\tfrac{\VnormLp{\ProjC[0]\xdf}^2}{2}]\daRaS{l}{\xdf,\iSv}-\penSv\big\}\big)$%\\
    % $\rWe\Ind{\setB{\VnormLp{\txdfPr[l]-\xdfPr[l]}^2<\penSv[l]}} 
    % \leq\exp\big(\rWn\big\{-\tfrac{\VnormLp{\Proj[{\mHiH[0]}]^\perp\xdf}^2}{2}\bias^2(\xdf)
    % +[120\cpen+\tfrac{\VnormLp{\Proj[{\mHiH[0]}]^\perp\xdf}^2}{2}]\hRaDi{l,\xdf,\iSv}\big\}\big)$
  \item\label{re:rWe:ii} for all $\Di\in\nsetlo{l,\ssY}$ we have\\
    $\rWe\Ind{\setB{\VnormLp{\txdfPr-\xdfPr}^2<\penSv/7}}\leq\exp\big(\rWn\big\{-\tfrac{1}{2}\penSv
    +[\tfrac{3}{2}\VnormLp{\ProjC[0]\xdf}^2+\cpen]\daRaS{l}{\xdf,\iSv}\big\}\big)$.
    % $\rWe\Ind{\setB{\VnormH{\txdfPr-\xdfPr}^2<\penSv}}\leq
    % \exp\big(\rWn\big\{-\penSv
    % +[\tfrac{3}{2}\VnormLp{\Proj[{\mHiH[0]}]^\perp\xdf}^2+54\cpen]\hRaDi{l,\xdf,\iSv}\big\}\big)$.
  \end{resListeN}
\end{lm}
% --------------------------------------------------------------------
% <<Proof Re Random weights>>
% --------------------------------------------------------------------
\begin{pro}[Proof of \cref{re:rWe}.]
  Given $\Di,l\in\nset{1,\ssY}$ and an event $\dmEv{\Di}{l}$ (to be
  specified below) it clearly follows
  \begin{multline}\label{re:rWe:pro1}
    \rWe\Ind{\dmEv{\Di}{l}}
    =\frac{\exp(-\rWn\{-\VnormLp{\txdfPr}^2+\penSv\})}
    {\sum_{l\in\nset{1,\ssY}}\exp(-\rWn\{-\VnormLp{\txdfPr[l]}^2+\penSv[l]\})}\Ind{\dmEv{\Di}{l}}\\
    \leq
    \exp\big(\rWn\big\{\VnormLp{\txdfPr}^2-\VnormLp{\txdfPr[l]}^2+(\penSv[l]-\penSv)\big\}\big)\Ind{\dmEv{\Di}{l}}
  \end{multline}
  We distinguish the two cases \ref{re:rWe:i} $\Di\in\nsetro{1,l}$ and \ref{re:rWe:ii}
  $\Di\in\nsetlo{l,n}$.  Consider first \ref{re:rWe:i} $\Di\in\nsetro{1,l}$. From
  \ref{re:contr:e1} in \cref{re:contr} (with
  $\dxdfPr[]=\txdfPr[\ssY]$) follows that
  \begin{multline*}%\label{re:rWe:pro2}
    \rWe\Ind{\dmEv{\Di}{l}}
    \leq
    \exp\big(\rWn\big\{\VnormLp{\txdfPr}^2-\VnormLp{\txdfPr[l]}^2+(\penSv[l]-\penSv)\big\}\big)\Ind{\dmEv{\Di}{l}}\\
    % \exp\big(\rWn\big\{\contr[](\txdfPr[l])-\contr[](\txdfPr)+\tfrac{9}{2}(\penSv[l]-\penSv)\big\}\big)\Ind{\dmEv{\Di}{l}}\\
    \leq \exp\big(\rWn\big\{\tfrac{11}{2}\VnormLp{\txdfPr[l]-\xdfPr[l]}^2-\tfrac{1}{2}\VnormLp{\ProjC[0]\xdf}^2(\bias[k]^2(\xdf)-\bias[l]^2(\xdf))+(\penSv[l]-\penSv[k])\big\}\big)\Ind{\dmEv{k}{l}}
  \end{multline*}
  If we define
  $\dmEv{\Di}{l}:=\setB{\VnormLp{\txdfPr[l]-\xdfPr[l]}^2<\cpen\daRa{l}{\xdf,\iSv}/7}$
  then the last bound togehter with \ref{ak:ass:pen:oo}, i.e.,
  $\db[\VnormLp{\ProjC[0]\xdf}^2+\cpen]\daRa{\Di}{\xdf,\iSv}\geq
  \VnormLp{\ProjC[0]\xdf}^2\bias^2(\xdf)\vee\penSv$, implies the
  assertion \ref{re:rWe:i}, that is
  \begin{multline*}
    \rWe\Ind{\setB{\VnormLp{\txdfPr[l]-\xdfPr[l]}^2<\cpen\daRa{l}{\xdf,\iSv}/7}}
    \\\leq\exp\big(\rWn\big\{\tfrac{11}{14}\cpen\daRa{l}{\xdf,\iSv}
    +\tfrac{1}{2}\VnormLp{\ProjC[0]\xdf}^2\bias[l]^2(\xdf) +\penSv[l]
    -\tfrac{1}{2}\VnormLp{\ProjC[0]\xdf}^2\bias^2(\xdf)-\penSv\big\}\big)\\
    \leq\exp\big(\rWn\big\{[\tfrac{25}{14}\cpen+\tfrac{1}{2}\VnormLp{\ProjC[0]\xdf}^2]\daRa{l}{\xdf,\iSv}
    -\tfrac{1}{2}\VnormLp{\ProjC[0]\xdf}^2\bias^2(\xdf)-\penSv\big\}\big).
    % =\exp\big(\rWn\big\{10*\penSv[l]-\tfrac{1}{2}\VnormLp{\Proj[{\mHiH[0]}]^\perp\xdf}^2(\bias[k]^2(\xdf)-\bias[l]^2(\xdf))-\tfrac{9}{2}\penSv[k]\big\}\big)
  \end{multline*}
  % and hence, by exploiting that $\penSv[k]\geq0$ and
  % $\hRa{l,\xdf,\iSv}=[\bias[l]^2(\xdf)\vee \DipenSv[l] \ssY^{-1}]$
  % follows the assertion \ref{re:rWe:i}, that is
  % \begin{multline*}
  %   \rWe[k]\Ind{\setB{\VnormLp{\txdfPr[l]-\xdfPr[l]}^2<\penSv[l]}}\leq
  %   \exp\big(\rWn\big\{-\tfrac{\VnormLp{\Proj[{\mHiH[0]}]^\perp\xdf}^2}{2}\bias[k]^2(\xdf)+[10*12\cpen+\tfrac{\VnormLp{\Proj[{\mHiH[0]}]^\perp\xdf}^2}{2}]\hRa{l,\xdf,\iSv})\big\}\big).
  % \end{multline*}
  Consider secondly \ref{re:rWe:ii} $\Di\in\nsetlo{l,n}$. From \ref{re:contr:e2}
  in \cref{re:contr} (with $\dxdfPr[]=\txdfPr[\ssY]$) and
  \eqref{re:rWe:pro1} follows
  \begin{multline*}
    \rWe[k]\Ind{\dmEv{l}{k}}
    \leq\exp\big(\rWn\big\{\VnormLp{\txdfPr}^2-\VnormLp{\txdfPr[l]}^2
    +(\penSv[l]-\penSv)\big\}\big)\Ind{\dmEv{\Di}{l}}\\
    \leq
    \exp\big(\rWn\big\{\tfrac{7}{2}\VnormLp{\txdfPr[k]-\xdfPr[k]}^2
    +\tfrac{3}{2}\VnormLp{\ProjC[0]\xdf}^2(\bias[l]^2(\xdf)-\bias^2(\xdf))
    +(\penSv[l]-\penSv)\big\}\big)\Ind{\dmEv{l}{k}}
  \end{multline*}
  If we set $\dmEv{l}{\Di}:=\{\VnormLp{\txdfPr-\xdfPr}^2<\penSv/7\}$
  then we clearly have
  \begin{multline*}
    \rWe\Ind{\setB{\VnormLp{\txdfPr-\xdfPr}^2<\penSv/7}}\\
    \leq \exp\big(\rWn\big\{-\tfrac{1}{2}\penSv+\penSv[l]+
    \tfrac{3}{2}\VnormLp{\ProjC[0]\xdf}^2\bias[l]^2(\xdf)
    -\tfrac{3}{2}\VnormLp{\ProjC[0]\xdf}^2\bias^2(\xdf)\big\}\big)
  \end{multline*}
  and hence, by exploiting $\bias^2(\xdf)\geq0$ and
  \ref{ak:ass:pen:oo} follows the assertion \ref{re:rWe:ii}, that is
  \begin{equation*}
    \rWe[k]\Ind{\setB{\VnormLp{\txdfPr[k]-\xdfPr[k]}^2<\penSv}}
    % \leq \exp\big(\rWn\big\{-\tfrac{1}{2}\penSv
    % +[\tfrac{3}{2}\VnormLp{\Proj[{\mHiH[0]}]^\perp\xdf}^2+\tfrac{9}{2}*12\cpen]\hRa{l,\xdf,\iSv}\big\}\big).
    \leq \exp\big(\rWn\big\{-\tfrac{1}{2}\penSv+[\tfrac{3}{2}\VnormLp{\ProjC[0]\xdf}^2+\cpen]\daRa{l}{\xdf,\iSv}\big\}\big),
 \end{equation*}
which completes the proof.\proEnd
\end{pro}
% ....................................................................
% <<Proof Re Sum Random weights>>
% ....................................................................
\begin{pro}[Proof of \cref{ak:re:SrWe:ag}.]
  Consider \ref{ak:re:SrWe:ag:i}. For the non trivial case $\mDi>1$
  from \cref{re:rWe} \ref{re:rWe:i} with $l=\mdDi$ follows for all
  $\Di<\mDi\leq \mdDi$, and hence due to the definition
  \eqref{ak:de:*Di:ag}
  $\VnormLp{\ProjC[0]\xdf}^2\bias^2\geq
  \VnormLp{\ProjC[0]\xdf}^2\bias[\mDi-1]^2>2[\VnormLp{\ProjC[0]\xdf}^2+2\cpen]\daRaS{\mdDi}{\xdf,\iSv}$.
  Exploiting the last bound we obtain for each $\Di\in\nsetro{1,\mDi}$
  \begin{multline*}
    \rWe\Ind{\setB{\VnormLp{\txdfPr[\mdDi]-\xdfPr[\mdDi]}^2<\cpen\daRaS{\mdDi}{\xdf,\iSv}/7}}
    \leq
    \exp\big(\rWn\big\{-\tfrac{\VnormLp{\ProjC[0]\xdf}^2}{2}\bias^2(\xdf)
    +[\tfrac{25\cpen}{14}+\tfrac{\VnormLp{\ProjC[0]\xdf}^2}{2}]\daRaS{\mdDi}{\xdf,\iSv}-\penSv\big\}\big)\\
    % \hfill=\exp\big(\rWn\big\{\underbrace{-\tfrac{1}{2}\VnormLp{\So}^2\bias^2(\So)
    % +[\tfrac{28}{14}\cpen+\tfrac{1}{2}\VnormLp{\So}^2]\dRa{\mdDi}(\So)}_{\leq0}\}\big)\hfill\\
    % \hfill\times\exp\big(-\tfrac{3}{14}\rWc\cpen
    % n\dRa{\mdDi}(\So)\big)\\
    \hfill
    \leq\exp\big(-\tfrac{3}{14}\rWc\cpen \ssY\daRaS{\mdDi}{\xdf,\iSv}-\rWn\penSv\big)
  \end{multline*}
  which in turn with
  $\dr\penSv=\cpen \Di\cmiSv\miSv \ssY^{-1}\geq \cpen\Di\ssY^{-1}$ and
  $\dr\sum_{\Di\in\Nz}\exp(-\mu\Di)\leq \mu^{-1}$ for any $\mu>0$
  implies \ref{ak:re:SrWe:ag:i}, that is,
  \begin{multline*}
    \FuVg{\rWe[]}(\nsetro{1,\mDi})\leq
    \FuVg{\rWe[]}(\nsetro{1,\mDi})\Ind{\setB{\VnormLp{\txdfPr[\mdDi]-\xdfPr[\mdDi]}^2<\cpen\daRaS{\mdDi}{\xdf,\iSv}/7}}
    +\Ind{\setB{\VnormLp{\txdfPr[\mdDi]-\xdfPr[\mdDi]}^2\geq\cpen\daRaS{\mdDi}{\xdf,\iSv}/7}}\\
    \hfill\leq\exp\big(-\tfrac{3\rWc\cpen}{14}\ssY\daRaS{\mdDi}{\xdf,\iSv}\big)\sum_{k=1}^{\mDi-1}\exp(-\rWc\cpen\Di)
    +\Ind{\setB{\VnormLp{\txdfPr[\mdDi]-\xdfPr[\mdDi]}^2\geq\cpen\daRaS{\mdDi}{\xdf,\iSv}/7}}\\
    \leq \tfrac{1}{\rWc\cpen}\exp\big(-\tfrac{3\rWc\cpen}{14}\ssY\daRaS{\mdDi}{\xdf,\iSv}\big)
    +\Ind{\setB{\VnormLp{\txdfPr[\mdDi]-\xdfPr[\mdDi]}^2\geq\cpen\daRaS{\mdDi}{\xdf,\iSv}/7}}.
  \end{multline*} 
  Consider \ref{ak:re:SrWe:ag:ii}. From \cref{re:rWe} \ref{re:rWe:ii}
  with $l=\pdDi$ follows for all $\Di>\pDi\geq \pdDi$, and hence due
  to the definition \eqref{ak:de:*Di:ag}
  $\penSv > 2[3\VnormLp{\ProjC[0]\xdf}^2+ 2\cpen]\daRaS{\pdDi}{\xdf,\iSv}$. Thereby, we
  obtain for $\Di\in\nsetlo{\mDi,n}$
  \begin{multline*}
    \rWe\Ind{\setB{\VnormLp{\txdfPr-\xdfPr}^2<\penSv/7}}
    \leq % \exp\big(\rWn\big\{-\tfrac{1}{2}\pen +[\tfrac{3}{2}\VnormLp{\So}^2+\cpen]\dRa{\pdDi}(\So)\big\}\big)
    % \\
    % =
    \exp\big(\rWn\big\{-\tfrac{1}{4} \penSv
    -\tfrac{1}{4}\penSv 
    +[\tfrac{3}{2}\VnormLp{\ProjC[0]\xdf}^2+\cpen]\daRaS{\pdDi}{\xdf,\iSv}\big\}\big)\\
    \leq \exp\big(\rWn\big\{-\tfrac{1}{4} \penSv\big\}\big).
  \end{multline*}
   which in turn with $\dr\penSv=\cpen \Di\cmiSv\miSv \ssY^{-1}$  implies
  \begin{equation}\label{ak:re:SrWe:ag:pe1}
    \sum_{\Di\in\nsetlo{\pDi,n}}\penSv\rWe\Ind{\{\VnormLp{\hSoPr-\SoPr}^2\leq\pen/7\}}
    \leq \cpen\ssY^{-1}\sum_{\Di\in\nsetlo{\pDi,n}} \Di\cmiSv\miSv\exp\big(-\tfrac{\rWc\cpen}{4}\Di\cmiSv\miSv\big)
    % \tfrac{4}{\rWc}n{^{-1}}\sum_{\Di\in\nsetlo{\pDi,n}}\tfrac{\rWc\cpen}{4}\Di
    %\exp\big(-\tfrac{\rWc\cpen}{4}\Di\big)\leq\tfrac{16}{\rWc^2\cpen}n^{-1},\hfill
  \end{equation}
  Exploiting that
  $\dr\sqrt{\cmiSv}=\tfrac{\log (\Di\miSv \vee
    (\Di+2))}{\log(\Di+2)}\geq1$, $\dr\cpen/4\geq2\log(3e)$ and
  $\dr\rWc\geq1$, then for all $k\in\Nz$ we have
  $\tfrac{\rWc\cpen}{4} k-\log(k+2)\geq1$, and hence by
  $a\exp(-ab)\leq \exp(-b)$ for $a,b\geq1$, it follows
  \begin{multline*}
    \cmiSv\Di \miSv\exp\big(-\tfrac{\rWc\cpen}{4}\cmiSv\Di\miSv\big)
    \leq\cmiSv\exp\big(-\tfrac{\rWc\cpen}{4}\cmiSv\Di\miSv + \sqrt{\cmiSv}\log(\Di+2)\big)
    \\\hfill\leq
    \cmiSv\exp\big(-\cmiSv(\tfrac{\rWc\cpen}{4}\Di-\log(\Di+2))\big)
    \leq\exp\big(-(\tfrac{\rWc\cpen}{4}\Di-\log(\Di+2))\big)\\
    =(\Di+2)\exp\big(-\tfrac{\rWc\cpen}{4}\Di\big).
  \end{multline*}
  Exploiting $\sum_{\Di\in\Nz}\mu\Di\exp(-\mu\Di)\leq \mu^{-1}$ und
  $\sum_{\Di\in\Nz}\mu\exp(-\mu\Di)\leq 1$ we obtain
  \begin{displaymath}
    \sum_{k=\pDi+1}^{\ssY}\cmiSv\Di \miSv\exp\big(-\tfrac{\rWc\cpen}{4}\cmiSv\Di\miSv\big)
    \leq \sum_{k=\pDi+1}^\infty(\Di+2)\exp\big(-\tfrac{\rWc\cpen}{4}\Di\big)
    \leq \tfrac{16}{\cpen^2\rWc^{2}}+ \tfrac{8}{\cpen\rWc}.
  \end{displaymath}
  Combining the last bound and \eqref{ak:re:SrWe:ag:pe1} we obtain the
  assertion \ref{ak:re:SrWe:ag:ii}, that is
  \begin{displaymath}
    \sum_{\Di\in\nsetlo{\pDi,n}}\penSv\rWe\Ind{\{\VnormLp{\hSoPr-\SoPr}^2\leq\pen/7\}}
    \leq \ssY^{-1}\{\tfrac{16}{\cpen\rWc^{2}}+ \tfrac{8}{\rWc}\}
  \end{displaymath}
  which completes the proof.\proEnd
\end{pro}
% --------------------------------------------------------------------
% <<Proof Re Sum MS Random weights>>
% --------------------------------------------------------------------
\begin{pro}[Proof of \cref{ak:re:SrWe:ms}.]
  By definition of $\hDi$ it holds
  $-\VnormLp{\txdfPr[\hDi]}^2+\penSv[\hDi]\leq
  -\VnormLp{\txdfPr}^2+\penSv$ for all $\Di\in\nset{1,\ssY}$, and
  hence
  \begin{equation}\label{ak:re:SrWe:ms:pr:e1}
    \VnormLp{\txdfPr[\hDi]}^2-\VnormLp{\txdfPr}^2\geq
    \penSv[\hDi]-\penSv\text{ for all }\Di\in\nset{1,\ssY}.
  \end{equation}
  Consider \ref{ak:re:SrWe:ms:i}. It is sufficient to show, that
  $\{\hDi\in\nsetro{1,\mDi}\}\subseteq
  \{\VnormLp{\txdfPr-\xdfPr}^2\geq\cpen\daRaS{\mdDi}{\xdf,\iSv}/7\}$
  for $\mDi>1$ holds.  On the event $\{\hDi\in\nsetro{1,\mDi}\}$ holds
  $1\leq\hDi<\mDi\leq\mdDi$ and thus by definition
  \eqref{ak:de:*Di:ag}
  \begin{equation}\label{ak:re:SrWe:ms:pr:e2}
    \VnormLp{\ProjC[0]\xdf}^2\bias[\hDi]^2(\xdf)>
    [\VnormLp{\ProjC[0]\xdf}^2+4\cpen]\daRaS{\mdDi}{\xdf,\iSv}
  \end{equation}
  and due to \cref{re:contr} \ref{re:contr:e1} also
  \begin{equation}\label{ak:re:SrWe:ms:pr:e3}
    \VnormLp{\txdfPr[\hDi]}^2-\VnormLp{\txdfPr[\mdDi]}^2\leq
    \tfrac{11}{2}\VnormLp{\txdfPr[\mdDi]-\xdfPr[\mdDi]}^2
    -\tfrac{1}{2}\VnormLp{\ProjC[0]\xdf}^2\{\bias[\hDi]^2(\xdf)-\bias[\mdDi]^2(\xdf)\}.
  \end{equation}
  Combining, \eqref{ak:re:SrWe:ms:pr:e1} and
  \eqref{ak:re:SrWe:ms:pr:e3} it follows that
  \begin{multline*}
    \tfrac{11}{2}\VnormLp{\txdfPr[\mdDi]-\xdfPr[\mdDi]}^2\geq
    \penSv[\hDi]-\penSv[\mdDi]
    +\tfrac{1}{2}\VnormLp{\ProjC[0]\xdf}^2\{\bias[\hDi]^2(\xdf)-\bias[\mdDi]^2(\xdf)\}\hfill
  \end{multline*}
  and hence together with $\penSv[\hDi]\geq0$, \eqref{ak:re:SrWe:ms:pr:e2}
  and \ref{ak:ass:pen:oo} we obtain the claim, that is
  \begin{multline*}
    \tfrac{11}{2}\VnormLp{\txdfPr[\mdDi]-\xdfPr[\mdDi]}^2\geq
    \tfrac{1}{2}\VnormLp{\ProjC[0]\xdf}^2\bias[\hDi]^2(\xdf)-
    \tfrac{1}{2}\VnormLp{\ProjC[0]\xdf}^2\bias[\mdDi]^2(\xdf)
    -\penSv[\mdDi]\\
    >[\tfrac{1}{2}\VnormLp{\ProjC[0]\xdf}^2+2\cpen]\daRaS{\mdDi}{\xdf,\iSv}
    -\tfrac{1}{2}\VnormLp{\ProjC[0]\xdf}^2\bias[\mdDi]^2(\xdf)-\penSv[\mdDi]
    \geq\tfrac{11}{14}\cpen\daRaS{\mdDi}{\xdf,\iSv},
  \end{multline*}
  and shows \ref{ak:re:SrWe:ms:i}.  Consider \ref{ak:re:SrWe:ms:ii}. It is sufficient to show that,
  $\{\hDi\in\nsetlo{\pDi,\ssY}\}\subseteq
  \{\VnormLp{\txdfPr[\hDi]-\xdfPr[\hDi]}^2\geq\penSv[\hDi]/7\}$.  On the
  event $\{\hDi\in\nsetlo{\pDi,\ssY}\}$ holds $\hDi>\pDi\geq\pdDi$ and
  thus by definition \eqref{ak:de:*Di:ag}
  \begin{equation}\label{ak:re:SrWe:ms:pr:e4}
    \penSv[\hDi] > [6\VnormH{\So}^2+ 4\cpen] \daRaS{\pdDi}{\xdf,\iSv}
  \end{equation}
  and due to \cref{re:contr} \ref{re:contr:e2} also
  \begin{equation}\label{ak:re:SrWe:ms:pr:e5}
    \VnormLp{\txdfPr[\hDi]}^2-\VnormLp{\txdfPr[\pdDi]}^2\leq
    \tfrac{7}{2}\VnormLp{\txdfPr[\hDi]-\xdfPr[\hDi]}^2+\tfrac{3}{2}\VnormLp{\ProjC[0]\xdf}^2
    \{\bias[\pdDi]^2(\xdf)-\bias[\hDi]^2(\xdf)\}.
  \end{equation}
  Combining, \eqref{ak:re:SrWe:ms:pr:e1} and \eqref{ak:re:SrWe:ms:pr:e5} it
  follows that
  \begin{multline*}
    \tfrac{7}{2}\VnormLp{\txdfPr[\hDi]-\xdfPr[\hDi]}^2\geq
    \penSv[\hDi]-\penSv[\pdDi]  -\tfrac{3}{2}\VnormLp{\ProjC[0]\xdf}^2
    \{\bias[\pdDi]^2(\xdf)-\bias[\hDi]^2(\xdf)\}\hfill
  \end{multline*}
  and hence together with $\bias[\hDi]^2(\xdf)\geq0$,
  \eqref{ak:re:SrWe:ms:pr:e4} and \ref{ak:ass:pen:oo} we obtain the claim,
  that is
  \begin{multline*}
    \tfrac{7}{2}\VnormLp{\txdfPr[\hDi]-\xdfPr[\hDi]}^2\geq
    (\tfrac{1}{2}+\tfrac{1}{2})\penSv[\hDi]-\penSv[\pdDi]  -\tfrac{3}{2}\VnormLp{\ProjC[0]\xdf}^2
    \bias[\pdDi]^2(\xdf)\\
    >\tfrac{1}{2}\penSv[\hDi]+\tfrac{1}{2}[6\VnormLp{\ProjC[0]\So}^2+ 4\cpen]
    \daRaS{\pdDi}{\xdf,\iSv}-\penSv[\pdDi]-\tfrac{3}{2}\VnormLp{\ProjC[0]\xdf}^2
    \bias[\pdDi]^2(\xdf)
    \geq\tfrac{1}{2}\penSv[\hDi],
  \end{multline*}
  which shows \ref{ak:re:SrWe:ms:ii} and completes the proof.\proEnd
\end{pro}



% ....................................................................
% <<Re rest>>
% ....................................................................
\begin{lm}\label{ak:re:rest}Let $\DipenSv=\cmSv \Di \miSv$
  with
  $\sqrt{\cmiSv}=\tfrac{\log (\Di\miSv \vee
    (\Di+2))}{\log(\Di+2)}\geq1$, then there is a numerical constant
  $\cst{}$ such that for all $\ssY\in\Nz$ and
  $\Di\in\nset{1,\ssY}$ hold
  \begin{resListeN}[]
  \item\label{ak:re:rest:i} let $\dr\Di_{\ydf}:=\floor{  3(6\Vnormlp[1]{\fydf})^2}$ and $\dr \ssY_{o}:={15(200)^4}$ then\\ 
    $\sum_{\Di=1}^{\ssY}\FuEx[\ssY]{\rY}
    \vectp{\VnormLp{\txdfPr-\xdfPr}^2-12\DipenSv/\ssY}
    \leq \cst{}\ssY^{-1}\big[\miSv[\Di_{\ydf}]\Di_{\ydf}+ \miSv[\ssY_{o}]\big]$
  \item\label{ak:re:rest:ii} let
    $\dr\Di_{\ydf}:=\floor{3(800\Vnormlp[1]{\fydf})^2}$ and
    $\dr \ssY_{o}:=15({300})^4$ then\\
    $\sum_{\Di=1}^{\ssY}\DipenSv\FuVg[\ssY]{\rY}\big(\VnormLp{\txdfPr-\xdfPr}^2
    \geq12\DipenSv/\ssY\big)\leq\cst{}\big[\miSv[\Di_{\ydf}]^2\Di_{\ydf}^2+\miSv[\ssY_{o}]^2\big]$
  \item\label{ak:re:rest:iii} 
  $\FuVg[\ssY]{\rY}\big(\VnormLp{\txdfPr-\xdfPr}^2 \geq 12\daRaS{\Di}{\xdf,\iSv}\big)\leq 
    \cst{} \big[\exp\big(\tfrac{-\ssY\daRaS{\Di}{\xdf,\iSv}}{200\Vnormlp[1]{\fydf}\miSv}\big)+\ssY^{-1}\big]$
  \end{resListeN}
\end{lm}
% ....................................................................
% <<Proof Re rest>>
% ....................................................................
\begin{pro}[Proof of \cref{ak:re:rest}.]Consider \ref{ak:re:rest:i}.
  Since $\cmiSv\geq1$ for
  $\dr\Di\geq3({6\Vnormlp[1]{\fydf}})^2$ holds
  $\tfrac{\sqrt{\cmSv}\Di}{6\Vnormlp[1]{\fydf}}-\log(\Di+2)\geq0$
  and%
  \begin{multline*}
    \miSv\exp\big(\tfrac{-\cmSv\Di}{3\Vnormlp[1]{\fydf}}\big)\leq
    \exp\big(\tfrac{-\cmSv\Di}{6\Vnormlp[1]{\fydf}}\big)
    \exp\big(-\sqrt{\cmSv}[\tfrac{\sqrt{\cmSv}\Di}{6\Vnormlp[1]{\fydf}}-\log(\Di+2)]\big)\\
    \leq\exp\big(\tfrac{-\cmSv\Di}{6\Vnormlp[1]{\fydf}}\big)
    \leq\exp\big(-\tfrac{1}{6\Vnormlp[1]{\fydf}}\Di\big)
  \end{multline*}
  consequently, for
  $\dr\Di_{\ydf}:=\floor{3({6\Vnormlp[1]{\fydf}})^2}$ then exploiting
  $\sum_{\Di\in\Nz}\exp(-\mu\Di)\leq \mu^{-1}$ follows
  \begin{displaymath}
    \sum_{\Di=1+\Di_{\ydf}}^{\ssY}\miSv\exp\big(\tfrac{-\cmSv\Di}{3\Vnormlp[1]{\fydf}}\big)\leq
    \sum_{\Di=1+\Di_{\ydf}}^{\ssY}\exp\big(-\tfrac{1}{6\Vnormlp[1]{\fydf}}\Di\big)
    \leq {6\Vnormlp[1]{\fydf}}
  \end{displaymath}
  while
  \begin{displaymath}
   \sum_{\Di=1}^{\Di_{\ydf}}\miSv\exp\big(\tfrac{-\cmSv\Di}{3\Vnormlp[1]{\fydf}}\big)\leq
   \miSv[\Di_{\ydf}]\sum_{\Di=1}^{\Di_{\ydf}}\exp\big(\tfrac{-\Di}{3\Vnormlp[1]{\fydf}}\big)
   \leq \miSv[\Di_{\ydf}]{3\Vnormlp[1]{\fydf}}
 \end{displaymath}
 hence
 \begin{displaymath}
   \sum_{\Di=1}^{\ssY}\miSv\exp\big(\tfrac{-\cmSv\Di}{3\Vnormlp[1]{\fydf}}\big)\leq
   {6\Vnormlp[1]{\fydf}}+3\miSv[\Di_{\ydf}]\Vnormlp[1]{\fydf}\leq 9\miSv[\Di_{\ydf}]{\Vnormlp[1]{\fydf}}
 \end{displaymath}
 Using for all $\dr \ssY>\ssY_{o}:=15({200})^4$ holds 
 $\sqrt{n}\geq{200}\log(n+2)$ it follows for all $\Di\in\nset{1,n}$
 \begin{displaymath}
   \tfrac{\Di\miSv}{\ssY}\exp\big(\tfrac{-\sqrt{n\cmSv}}{200}\big)
   \leq
   \tfrac{1}{\ssY}\exp\big(-\sqrt{\cmSv}[\tfrac{\sqrt{\ssY}}{200}-\log(\Di+2)]\big)\leq \tfrac{1}{\ssY}
 \end{displaymath}
 consequently, 
 \begin{equation*}
   \sum_{\Di=1}^{\ssY}\tfrac{\Di\miSv}{\ssY}\exp\big(\tfrac{-\sqrt{n\cmSv}}{200}\big)
   \leq\sum_{\Di=1}^{\ssY}\tfrac{1}{\ssY}\leq1
 \end{equation*}
 while for $\ssY\leq \ssY_{o}$ with
 $\miSv[\ssY]\leq\miSv[\ssY_{o}]$ follows
\begin{equation*}
   \sum_{\Di=1}^{\ssY}\tfrac{\Di\miSv}{\ssY}\exp\big(\tfrac{-\sqrt{\ssY\cmSv}}{200}\big)\leq 
    \miSv[\ssY]\ssY\exp\big(\tfrac{-\sqrt{\ssY}}{200}\big)\leq\ssY_{o}\miSv[\ssY_{o}]
  \end{equation*}
 consequently, for all $\ssY\in\Nz$ holds
 \begin{displaymath}
 \sum_{\Di=1}^{\ssY}\tfrac{\Di\miSv}{\ssY}\exp\big(\tfrac{-\sqrt{n\cmSv}}{200}\big)\leq \miSv[\ssY_{o}]\ssY_{o}
\end{displaymath}
Combining the last two bounds and \cref{re:conc} \ref{re:conc:i}  we obtain \ref{ak:re:rest:i}, that is 
\begin{multline*}
\sum_{\Di=1}^{\ssY}\FuEx[\ssY]{\rY}\vectp{\VnormLp{\txdfPr[\Di]-\xdfPr[\Di]}^2-12\DipenSv/\ssY}\\\hfill\leq \cst{}\bigg[\tfrac{\Vnormlp[1]{\fydf}}{\ssY}\sum_{\Di=1}^{\ssY}
\miSv\exp\big(\tfrac{-\cmSv\Di}{3\Vnormlp[1]{\fydf}}\big)+\tfrac{4}{n}\sum_{\Di=1}^{\ssY}\tfrac{\Di\miSv}{n}\exp\big(\tfrac{-\sqrt{n\cmSv}}{200}\big)
\bigg]\\\leq \cst{}\ssY^{-1}\big[9\miSv[\Di_{\ydf}]{\Vnormlp[1]{\fydf}^2}+4 \miSv[\ssY_{o}]\ssY_{o}\big]
\end{multline*}
Consider  \ref{ak:re:rest:ii}. If   $\dr\Di\geq 3({400\Vnormlp[1]{\fydf}})^2$ then 
$\Di\geq  ({400\Vnormlp[1]{\fydf}})\log(\Di+2)$ and
hence
$\Di-{200\Vnormlp[1]{\fydf}}\log(\Di+2)\geq{200\Vnormlp[1]{\fydf}}\log(\Di+2)$
or equivalently,
$\tfrac{\Di}{200\Vnormlp[1]{\fydf}}-\log(\Di+2)\geq\log(\Di+2)\geq1$
and thus
\begin{multline*}
\Di\cmSv\miSv\exp\big(\tfrac{-\cmSv\Di}{200\Vnormlp[1]{\fydf}}\big)\leq
\cmSv\exp\big(-\cmSv\,[\tfrac{\Di}{200\Vnormlp[1]{\fydf}}-\log(\Di+2)]\big)\\\leq
(\Di+2)\exp\big(-\tfrac{\Di}{200\Vnormlp[1]{\fydf}}\big)% =
% \\\cmSv\exp\big(-\tfrac{\cpen}{800\Vnormlp[1]{\fydf}}\cmSv\Di\big)\exp\big(-\sqrt{\cmSv}[\tfrac{\cpen\sqrt{\cmSv}}{800\Vnormlp[1]{\fydf}}\Di-\log(\Di+2)]\big)\leq\\
\end{multline*}
consequently, if $\dr\Di>\Di_{\ydf}:=\floor{3({400\Vnormlp[1]{\fydf}})^2}$ exploiting $\sum_{\Di\in\Nz}(\Di+2)\exp(-\mu\Di)\leq \mu^{-2}+ 2\mu^{-1}$
follows
\begin{multline*}
\sum_{\Di=1+\Di_{\ydf}}^{\ssY}\Di\cmSv\miSv\exp\big(\tfrac{-\cmSv\Di}{200\Vnormlp[1]{\fydf}}\big)\leq
\sum_{\Di=1+\Di_{\ydf}}^{\ssY}(k+2)\exp\big(-\tfrac{\Di}{200\Vnormlp[1]{\fydf}}\big)
\\\leq
({200\Vnormlp[1]{\fydf}})^2+{400\Vnormlp[1]{\fydf}}\leq \Di_{\ydf}^2
\end{multline*}
while $\log(\Di\miSv)\leq \tfrac{1}{e}\Di\miSv$ implies
$\cmSv\leq\Di\miSv$ it follows
\begin{multline*}
  \sum_{\Di=1}^{\Di_{\ydf}}\Di\cmSv\miSv\exp\big(\tfrac{-\cmSv\Di}{200\Vnormlp[1]{\fydf}}\big)\leq
  \cmSv[\Di_{\ydf}]\miSv[\Di_{\ydf}]\sum_{\Di=1}^{\Di_{\ydf}}\Di\exp\big(\tfrac{-\Di}{200\Vnormlp[1]{\fydf}}\big)\\\leq
  \cmSv[\Di_{\ydf}]\miSv[\Di_{\ydf}]({200\Vnormlp[1]{\fydf}})^2\leq\miSv[\Di_{\ydf}]^2\Di_{\ydf}^2
\end{multline*}
consequently for all $\ssY\in\Nz$ we have
\begin{displaymath}
  \sum_{\Di=1}^{\ssY}\Di\cmSv\miSv\exp\big(\tfrac{-\cmSv\Di}{200\Vnormlp[1]{\fydf}}\big)\leq(1+\miSv[\Di_{\ydf}]^2)\Di_{\ydf}^2\leq 2\miSv[\Di_{\ydf}]^2\Di_{\ydf}^2
\end{displaymath}
Since  $\cmSv\leq\Di\miSv$,
and  for all $\dr \ssY>\ssY_{o}:=\floor{15({600})^4}$ holds $\sqrt{\ssY}\geq{600}\log(\ssY+2)$
\begin{multline*}
\Di\cmSv\miSv\exp\big(\tfrac{-\sqrt{\ssY\cmSv}}{200}\big)\leq
\Di^2\miSv^2\exp\big(\tfrac{-\sqrt{\ssY\cmSv}}{200}\big)\\\leq
\tfrac{1}{\ssY}\exp\big(-\sqrt{\cmSv}[\tfrac{\sqrt{\ssY}}{200}-2\log(\Di+2)]+\log(\ssY+2)\big)
\leq\tfrac{1}{\ssY}\exp\big(-3\sqrt{\cmSv}[\tfrac{\sqrt{\ssY}}{600}-\log(\ssY+2)]\big)
\\
\leq \tfrac{1}{\ssY}
  \end{multline*}
consequently, 
\begin{equation*}
\sum_{\Di=1}^{\ssY}\Di\cmSv\miSv\exp\big(\tfrac{-\sqrt{\ssY\cmSv}}{200}\big)\leq\sum_{\Di=1}^{\ssY}\tfrac{1}{\ssY}\leq1
\end{equation*}
On the other hand side for $\ssY\leq\ssY_{o}$ with  $\ssY^b\exp(-a\ssY^{1/c})\leq (\tfrac{cb}{ea})^{cb}$ for all $c>0$ and $a,b\geq0$  follows
\begin{multline*}
\sum_{\Di=1}^{\ssY}\Di\cmSv\miSv\exp\big(\tfrac{-\sqrt{\ssY\cmSv}}{200}\big)\leq\ssY^2\cmSv[\ssY]\miSv[\ssY]\exp\big(\tfrac{-\sqrt{\ssY}}{200}\big)\leq
\miSv[\ssY]^2\ssY^3\exp\big(\tfrac{-\sqrt{\ssY}}{200}\big)\\\leq \miSv[\ssY_{o}]^2\big({600}\big)^6\leq\miSv[\ssY_{o}]^2\ssY_{o}^2
\end{multline*}
 consequently, for all $\ssY\in\Nz$ holds
 \begin{displaymath}
 \sum_{\Di=1}^{\ssY}\Di\cmSv\miSv\exp\big(\tfrac{-\sqrt{\ssY\cmSv}}{200}\big)\leq \miSv[\ssY_{o}]^2\ssY_{o}^2
\end{displaymath}
Combining the last two bounds and \cref{re:conc} \ref{re:conc:ii} we obtain \ref{ak:re:rest:ii}, that is 
\begin{multline*}
\sum_{\Di=1}^{\ssY}\cmiSv \Di \miSv\FuVg[\ssY]{\rY}\big(\VnormLp{\txdfPr[\Di]-\xdfPr[\Di]}^2\geq12\DipenSv/\ssY\big)\\\hfill\leq 3\sum_{\Di=1}^{\ssY}\cmiSv \Di \miSv\bigg[\exp\big(\tfrac{-\cmSv\Di}{200\Vnormlp[1]{\fydf}}\big)+\exp\big(\tfrac{-\sqrt{\ssY\cmSv}}{200}\big)
\bigg]
\leq3\bigg[2\miSv[\Di_{\ydf}]^2\Di_{\ydf}^2+\miSv[\ssY_{o}]^2\ssY_{o}^2\bigg]
\end{multline*}
Consider \ref{ak:re:rest:iii}. Since
$\tfrac{\ssY\sqrt{\daRa{\Di}{\xdf,\iSv}}}{200\sqrt{\Di\miSv}}\geq\tfrac{\sqrt{\ssY\cmiSv}}{200}\geq\tfrac{\sqrt{\ssY}}{200}$
and $\ssY\exp(-\tfrac{\sqrt{\ssY}}{200})\leq(200)^2$ 
from \cref{re:conc} \ref{re:conc:iii} follows \ref{ak:re:rest:iii}, that is 
\begin{multline*}
 \FuVg[\ssY]{\rY}\big(\VnormLp{\txdfPr-\xdfPr}^2 \geq 12\daRa{\Di}{\xdf,\iSv}\big)\leq 
    3 \big[\exp\big(\tfrac{-\ssY\daRa{\Di}{\xdf,\iSv}}{200\Vnormlp[1]{\fydf}\miSv}\big)
    +\exp\big(\tfrac{-\ssY\sqrt{\daRa{\Di}{\xdf,\iSv}}}{200\sqrt{\Di\miSv}}\big)\big]\\\leq 3 \big[\exp\big(\tfrac{-\ssY\daRa{\Di}{\xdf,\iSv}}{200\Vnormlp[1]{\fydf}\miSv}\big)
    +(200)^2\ssY^{-1}\big] 
\end{multline*}
which  completes the proof.\proEnd\end{pro}
% --------------------------------------------------------------------
% <<Proof Re ND rest>>
% --------------------------------------------------------------------
\begin{pro}[Proof of \cref{ak:re:nd:rest}.]
  Since $\dr\cpen/7\geq 12$ and $\dr\penSv/7\geq12\DipenSv\ssY^{-1}$,
  $\Di\in\nset{1,n}$, by exploiting \cref{ak:re:rest}
  \ref{ak:re:rest:i}, \ref{ak:re:rest:ii} and \ref{ak:re:rest:iii} we
  obtain immediately the claim \ref{ak:re:nd:rest1},
  \ref{ak:re:nd:rest2} and \ref{ak:re:nd:rest3}, respectively, which  completes the proof.
\proEnd\end{pro}
% --------------------------------------------------------------------
% <<Proof Re upper bound ag>>
% --------------------------------------------------------------------
\begin{pro}[Proof of \cref{ak:ag:ub}.]
  Consider firstly the aggregation using the aggregation weights
  $\erWe[]:=\rWe[]$ as in \eqref{ak:de:rWe}.  Combining
  \cref{ak:re:nd:rest} and the upper bound given in \eqref{co:agg:ag}
  we obtain
  \begin{multline}\label{ak:ag:ub:p1}
    \FuEx[\ssY]{\rY}\VnormLp{\txdfAg-\xdf}^2\leq \tfrac{2}{7}\penSv[\pDi]
    +2\VnormLp{\ProjC[0]\xdf}^2\bias[\mDi]^2(\xdf)
    \\\hfill
 + \cst{}\VnormLp{\ProjC[0]\xdf}^2\Ind{\{\mDi>1\}}\big[ \tfrac{1}{\rWc}
     \exp\big(-18\rWc\ssY\daRaS{\mdDi}{\xdf,\iSv}\big)+
 \exp\big(\tfrac{-1}{200\Vnormlp[1]{\fydf}}\ssY\daRaS{\mdDi}{\xdf,\iSv}\miSv[\mdDi]^{-1}\big)\big]
 \\
 +\cst{}\big[\tfrac{1}{\rWc}+
 \VnormLp{\ProjC[0]\xdf}^2\Ind{\{\mDi>1\}}
+\miSv[\Di_{\ydf}]^2\Di_{\ydf}^2+\miSv[\ssY_{o}]^2 \big]\ssY^{-1}
  \end{multline}
 Moreover, since $1\geq\miSv[\mdDi]^{-1}$  it holds
$\ssY\daRaS{\mdDi}{\xdf,\iSv}\geq\ssY\daRaS{\mdDi}{\xdf,\iSv}\miSv[\mdDi]^{-1}$. From
\eqref{ak:ag:ub:p1} with $18\rWc>\tfrac{1}{200\Vnormlp[1]{\fydf}}$
(since $\rWc\geq1$ and $\Vnormlp[1]{\fydf}\geq|\fydf[0]|=1$) follows
  \begin{multline}\label{ak:ag:ub:p2}
    \FuEx[\ssY]{\rY}\VnormLp{\txdfAg-\xdf}^2\leq \tfrac{2}{7}\penSv[\pDi]
    +2\VnormLp{\ProjC[0]\xdf}^2\bias[\mDi]^2(\xdf)
    \\\hfill
 + \cst{}\VnormLp{\ProjC[0]\xdf}^2\Ind{\{\mDi>1\}}
 \exp\big(\tfrac{-1}{200\Vnormlp[1]{\fydf}}\ssY\daRaS{\mdDi}{\xdf,\iSv}\miSv[\mdDi]^{-1}\big)
 \\
 +\cst{}\big[
 \VnormLp{\ProjC[0]\xdf}^2\Ind{\{\mDi>1\}}
+\miSv[\Di_{\ydf}]^2\Di_{\ydf}^2+\miSv[\ssY_{o}]^2 \big]\ssY^{-1}.
  \end{multline}
  Consider secondly the aggregation using the model selection weights $\erWe[]:=\msWe[]$
  as in \eqref{ak:de:msWe}. Combining
  \cref{ak:re:nd:rest} and the upper bound given in \eqref{co:agg:ms}
  we obtain
  \begin{multline}\label{ak:ag:ub:p3}
    \FuEx[\ssY]{\rY}\VnormLp{\txdfAg[{\msWe[]}]-\xdf}^2\leq \tfrac{2}{7}\penSv[\pDi]
    +2\VnormLp{\ProjC[0]\xdf}^2\bias[\mDi]^2(\xdf)
    \\\hfill
 + \cst{}\VnormLp{\ProjC[0]\xdf}^2\Ind{\{\mDi>1\}}
 \exp\big(\tfrac{-1}{200\Vnormlp[1]{\fydf}}\ssY\daRaS{\mdDi}{\xdf,\iSv}\miSv[\mdDi]^{-1}\big)
 \\
 +\cst{}\big[
 \VnormLp{\ProjC[0]\xdf}^2\Ind{\{\mDi>1\}}
+\miSv[\Di_{\ydf}]^2\Di_{\ydf}^2+\miSv[\ssY_{o}]^2 \big]\ssY^{-1}.
  \end{multline}  
From \eqref{ak:ag:ub:p2} and \eqref{ak:ag:ub:p3} together with
$\ssY\daRaS{\mdDi}{\xdf,\iSv}\miSv[\mdDi]^{-1}\geq\cmiSv[\mdDi]\mdDi$
follows the claim \eqref{ak:ag:ub:e1}, which  completes the proof.
\proEnd\end{pro}
% ....................................................................
% <<Pro upper bound ag p>>
% ....................................................................
\begin{pro}[Proof of \cref{ak:ag:ub:pnp}.]
From
\eqref{ak:ag:ub:e1} follows for any $\mdDi,\pdDi\in\nset{1,n}$ and associated
$\mDi,\pDi\in\nset{1,n}$ as defined in  \eqref{ak:de:*Di:ag}%
 \begin{multline}\label{ak:ag:ub:pnp:p1}
     \FuEx[\ssY]{\rY}\VnormLp{\txdfAg[{\erWe[]}]-\xdf}^2\leq \tfrac{2}{7}\penSv[\pDi]
    +2\VnormLp{\ProjC[0]\xdf}^2\bias[\mDi]^2(\xdf)
 + \cst{}\VnormLp{\ProjC[0]\xdf}^2\Ind{\{\mDi>1\}}\exp\big(\tfrac{-\cmiSv[\mdDi]\mdDi}{200\Vnormlp[1]{\fydf}}\big)\\
    +\cst{}\big[\VnormLp{\ProjC[0]\xdf}^2\Ind{\{\mDi>1\}}
+\miSv[\Di_{\ydf}]^2\Di_{\ydf}^2+\miSv[\ssY_{o}]^2 \big]\ssY^{-1}
\end{multline}
We destinguish the two cases \ref{ak:ag:ub:pnp:p} and
\ref{ak:ag:ub:pnp:np}. Consider first \ref{ak:ag:ub:pnp:p}, and hence there is $K\in\Nz_0$   with   $1\geq \bias[{[K-1] }](\xdf)>0$ and
$\bias(\xdf)=0$ for all $\Di\geq K$. Consider first $K=0$, then $\bias[0](\xdf)=0$
and hence $\VnormLp{\ProjC[0]\xdf}^2=0$. From \eqref{ak:ag:ub:pnp:p1}
follows 
 \begin{equation}\label{ak:ag:ub:pnp:p2}
     \FuEx[\ssY]{\rY}\VnormLp{\txdfAg[{\erWe[]}]-\xdf}^2\leq \tfrac{2}{7}\penSv[\pDi]
    +\cst{}\big[\miSv[\Di_{\ydf}]^2\Di_{\ydf}^2+\miSv[\ssY_{o}]^2 \big]\ssY^{-1}
\end{equation}
Setting  $\pdDi:=1$ it follows from the definition
\eqref{ak:de:*Di:ag} of  $\pDi$ that
$\penSv[\pDi]\leq4\cpen\daRaS{1}{\xdf,\iSv}$, where
$\daRaS{1}{\xdf,\iSv}=\DipenSv[1]\ssY^{-1}$ and
$\DipenSv[1]=\cmSv[1] \miSv[1]\leq\miSv[1]^2$. Thereby with numerical
constant $\cpen\geq84$, \eqref{ak:ag:ub:pnp:p2} implies
 \begin{equation}\label{ak:ag:ub:pnp:p3}
     \FuEx[\ssY]{\rY}\VnormLp{\txdfAg[{\erWe[]}]-\xdf}^2\leq\cst{}\big[\miSv[1]^2+\miSv[\Di_{\ydf}]^2\Di_{\ydf}^2+\miSv[\ssY_{o}]^2 \big]\ssY^{-1}
\end{equation}
Consider now $K\in\Nz$, and hence $\VnormLp{\ProjC[0]\xdf}^2>0$. Let 
$c_{\xdf}:=\tfrac{\VnormLp{\ProjC[0]\xdf}^2+4\cpen}{\VnormLp{\ProjC[0]\xdf}^2\bias[{[K-1]}]^2(\xdf)}>1$
and $\ssY_{\xdf}:=\ceil{c_{\xdf}\DipenSv[K])}\in\Nz$. We distinguish for $n\in\Nz$ the following two
 cases, \begin{inparaenum}[i]\renewcommand{\theenumi}{\dgrau\rm(\alph{enumi})}\item\label{ak:ag:ub:pnp:p:c1}
$\ssY\in\nset{1,\ssY_{\xdf}}$ and \item\label{ak:ag:ub:pnp:p:c2}
$\ssY> \ssY_{\xdf}$. \end{inparaenum} Firstly, consider
\ref{ak:ag:ub:pnp:p:c1} with $\ssY\in\nset{1,\ssY_{\xdf}}$, then setting $\mdDi:=1$, $\pdDi:=1$ we have
$\mDi=1$, $1\geq\bias[\mDi]$ and from the definition
\eqref{ak:de:*Di:ag} of  $\pDi$ also
$\penSv[\pDi]\leq2[3\VnormLp{\ProjC[0]\xdf}^2+2\cpen]\daRaS{1}{\xdf,\iSv}\leq10\cpen\,[\VnormLp{\ProjC[0]\xdf}^2\vee1]\miSv[1]^2$
exploiting $\bias[1]\leq1\leq\DipenSv[1]=\cmSv[1]
\miSv[1]\leq\miSv[1]^2$. Thereby,  from \cref{ak:ag:ub:pnp:p1} 
follows
 \begin{multline*}
     \FuEx[\ssY]{\rY}\VnormLp{\txdfAg[{\erWe[]}]-\xdf}^2\leq \tfrac{20}{7}\cpen(\VnormLp{\ProjC[0]\xdf}^2\vee1)\miSv[1]^2   +2\VnormLp{\ProjC[0]\xdf}^2
    +\cst{}\big[\miSv[\Di_{\ydf}]^2\Di_{\ydf}^2+\miSv[\ssY_{o}]^2
    \big]\ssY^{-1}\\
    \leq \cst{}\big[(\VnormLp{\ProjC[0]\xdf}^2\vee1)\miSv[1]^2\ssY+\miSv[\Di_{\ydf}]^2\Di_{\ydf}^2+\miSv[\ssY_{o}]^2\big]\ssY^{-1}
\end{multline*}
Moreover, for all $\ssY\in\nset{1,\ssY_{\xdf}}$ with
$\ssY_{\xdf}=\ceil{c_{\xdf}\DipenSv[K]}$ and
$\DipenSv[K]=K\cmSv[K] \miSv[K]\leq K^2\miSv[K]^2$ holds
$\ssY\leq\cst{}\tfrac{(\VnormLp{\ProjC[0]\xdf}^2\vee1)}{\VnormLp{\ProjC[0]\xdf}^2\bias[{[K-1]}]^2(\xdf)}
K^2\miSv[K]^2$and thereby, 
\begin{equation}\label{ak:ag:ub:pnp:p4}
  \FuEx[\ssY]{\rY}\VnormLp{\txdfAg[{\erWe[]}]-\xdf}^2\leq
  \cst{}\big[(\VnormLp{\ProjC[0]\xdf}^2\vee1)\miSv[1]^2\tfrac{K^2\miSv[K]^2}{\VnormLp{\ProjC[0]\xdf}^2\bias[{[K-1]}]^2(\xdf)}+\miSv[\Di_{\ydf}]^2\Di_{\ydf}^2+\miSv[\ssY_{o}]^2\big]\ssY^{-1}.
\end{equation}
Secondly, consider \ref{ak:ag:ub:pnp:p:c2}, i.e., $\ssY>
\ssY_{\xdf}$. Setting
$\pdDi:=K< \ceil{c_{\xdf}\DipenSv[K]}=\ssY_{\xdf}$, i.e.,
$\pdDi\in\nset{1,\ssY}$, it follows $\bias[\pdDi](\xdf)=0$ and hence
$\daRaS{\pdDi}{\xdf,\iSv}= \DipenSv[K] \ssY^{-1}$.  Therefore, the
definition \eqref{ak:de:*Di:ag} of $\pDi$ implies
$\pen[\pDi]\leq[6\VnormLp{\ProjC[0]\xdf}^2+ 4\cpen]\DipenSv[K]
\ssY^{-1}\leq
\cst{}(\VnormLp{\ProjC[0]\xdf}^2\vee1)K^2\miSv[K]^2\ssY^{-1}$. From
\eqref{ak:ag:ub:pnp:p1} follows for all $\ssY> \ssY_{\xdf}$ thus
\begin{multline}\label{ak:ag:ub:pnp:p5}
  \FuEx[\ssY]{\rY}\VnormLp{\txdfAg[{\erWe[]}]-\xdf}^2\leq 2\VnormLp{\ProjC[0]\xdf}^2\bias[\mDi]^2(\xdf)
    + \cst{}\VnormLp{\ProjC[0]\xdf}^2\Ind{\{\mDi>1\}}
    \exp\big(\tfrac{-\cmiSv[\mdDi]\mdDi}{200\Vnormlp[1]{\fydf}}\big)\\
    +\cst{}\big[(\VnormLp{\ProjC[0]\xdf}^2\vee1)K^2\miSv[K]^2
    +\miSv[\Di_{\ydf}]^2\Di_{\ydf}^2+\miSv[\ssY_{o}]^2 \big]\ssY^{-1}.
\end{multline}
Since
$\ssY> \ssY_{\xdf}:=\ceil{c_{\xdf}\DipenSv[K]}$ with
$c_{\xdf}=\tfrac{\VnormLp{\ProjC[0]\xdf}^2+4\cpen}{\VnormLp{\ProjC[0]\xdf}^2\bias[{[K-1]}]^2(\xdf)}>1$
the defining set of
$\sDi{\ssY}:=\max\{\Di\in\nset{K,\ssY}:\ssY>c_{\xdf}\DipenSv\}$
evenutally containing $K$ is not empty. Consequently,  $\sDi{\ssY}\geq
K$ and, hence 
$\bias[\sDi{\ssY}](\xdf)=0$, and
$\daRaS{\sDi{\ssY}}{\xdf,\iSv}=\DipenSv[\sDi{\ssY}]\ssY^{-1}<c_{\xdf}^{-1}=\tfrac{\VnormLp{\ProjC[0]\xdf}^2\bias[{[K-1]}]^2(\xdf)}{\VnormLp{\ProjC[0]\xdf}^2+4\cpen}$,
it follows
$\VnormLp{\ProjC[0]\xdf}^2\bias[{[K-1]}]^2(\xdf)>[\VnormLp{\ProjC[0]\xdf}^2+4\cpen]\daRaS{\sDi{\ssY}}{\xdf,\iSv}$
and trivially
$\VnormLp{\ProjC[0]\xdf}^2\bias[{K}]^2(\xdf)=0<[\VnormLp{\ProjC[0]\xdf}^2+4\cpen]\daRaS{\sDi{\ssY}}{\xdf,\iSv}$. Therefore,
setting $\mdDi:=\sDi{\ssY}$ the definition \eqref{ak:de:*Di:ag}
implies $\mDi=K$ and hence
$\bias[\mDi]^2(\xdf)=\bias[K]^2(\xdf)=0$. From \eqref{ak:ag:ub:pnp:p5}  follows
now for all $\ssY> \ssY_{\xdf}$ thus
\begin{multline}\label{ak:ag:ub:pnp:p6}
  \FuEx[\ssY]{\rY}\VnormLp{\txdfAg[{\erWe[]}]-\xdf}^2\leq  \cst{}\VnormLp{\ProjC[0]\xdf}^2\exp\big(\tfrac{-1}{200\Vnormlp[1]{\fydf}}\cmiSv[\sDi{\ssY}]\sDi{\ssY}\big)\\
  +\cst{}\big[(\VnormLp{\ProjC[0]\xdf}^2\vee1)K^2\miSv[K]^2
  +\miSv[\Di_{\ydf}]^2\Di_{\ydf}^2+\miSv[\ssY_{o}]^2 \big]\ssY^{-1}.
\end{multline}
Combining \eqref{ak:ag:ub:pnp:p4} and
    \eqref{ak:ag:ub:pnp:p6}  for $K\geq1$ with \ref{ak:ag:ub:pnp:p:c1}
$\ssY\in\nsetro{1,\ssY_{\xdf}}$ and \ref{ak:ag:ub:pnp:p:c2}
$\ssY\geq \ssY_{\xdf}$, respectively, and \eqref{ak:ag:ub:pnp:p3}  for
$K=0$ implies for all $K\in\Nz_0$ and for all $\ssY\in\Nz$ the claim
\eqref{ak:ag:ub:pnp:e1} in case \ref{ak:ag:ub:pnp:p}, that is
\begin{multline}\label{ak:ag:ub:pnp:p7}
  \FuEx[\ssY]{\rY}\VnormLp{\txdfAg[{\erWe[]}]-\xdf}^2\leq
  \cst{}\VnormLp{\ProjC[0]\xdf}^2\big[  \ssY^{-1}\vee\exp\big(\tfrac{-\cmiSv[\sDi{\ssY}]\sDi{\ssY}}{200\Vnormlp[1]{\fydf}}\big)\big]\\
  +\cst{}\big[\miSv[1]^2\{\tfrac{(\VnormLp{\ProjC[0]\xdf}^2\vee1)K^2\miSv[K]^2}{\VnormLp{\ProjC[0]\xdf}^2\bias[{[K-1]}]^2(\xdf)}\Ind{K\geq1}+\Ind{K=0}\}
  +\miSv[\Di_{\ydf}]^2\Di_{\ydf}^2+\miSv[\ssY_{o}]^2 \big]\ssY^{-1}.
\end{multline}
Consider the case \ref{ak:ag:ub:pnp:np}. For $\aDi{\ssY}(\xdf)\in\nset{1,n}$
as in \ref{ak:ass:pen:oo}  set $\pdDi:=\aDi{\ssY}(\xdf)$ and $\mdDi:=\sDi{\ssY}\in\nset{\aDi{\ssY}(\xdf),\ssY}$ by exploiting the definition
\eqref{ak:de:*Di:ag} of $\pDi$ and $\mDi$ it follows
$\penSv[\pDi] \leq 2[3\VnormLp{\ProjC[0]\xdf}^2+ 2\cpen]
\daRa{\pdDi}{\xdf}$ and 
$\VnormLp{\ProjC[0]\xdf}^2\bias[\mDi]^2(\xdf)\leq
      [\VnormLp{\ProjC[0]\xdf}^2+4\cpen]\daRa{\sDi{\ssY}}{\xdf}$
which together with
$\daRa{\sDi{\ssY}}{\xdf}\geq\naRa{\xdf}=\daRa{\pdDi}{\xdf}\geq\ssY^{-1}$
and exploiting 
\eqref{ak:ag:ub:pnp:p1} implies%
 \begin{multline}\label{ak:ag:ub:pnp:p8}
   \FuEx[\ssY]{\rY}\VnormLp{\txdfAg[{\erWe[]}]-\xdf}^2% \leq
 %   \tfrac{2}{7}\penSv[\pDi]
 %    +2\VnormLp{\ProjC[0]\xdf}^2\bias[\mDi]^2(\xdf)
 % + \cst{}\VnormLp{\ProjC[0]\xdf}^2\Ind{\{\mDi>1\}}\exp\big(\tfrac{-\cmiSv[\mdDi]\mdDi}{200\Vnormlp[1]{\fydf}}\big)\\
 %    +\cst{}\big[\VnormLp{\ProjC[0]\xdf}^2\Ind{\{\mDi>1\}}
 %    +\miSv[\Di_{\ydf}]^2\Di_{\ydf}^2+\miSv[\ssY_{o}]^2 \big]\ssY^{-1}\\
    \leq 
   \cst{}(\VnormLp{\ProjC[0]\xdf}^2\vee1)\big[\daRa{\sDi{\ssY}}{\xdf,\iSv}\vee\exp\big(\tfrac{-\cmiSv[\sDi{\ssY}]\sDi{\ssY}}{200\Vnormlp[1]{\fydf}}\big)\big]\\
   +\cst{}\big[\miSv[\Di_{\ydf}]^2\Di_{\ydf}^2+\miSv[\ssY_{o}]^2 \big]\ssY^{-1}
\end{multline}

which shows the assertion \eqref{ak:ag:ub:pnp:e2} and  completes the
proof of \cref{ak:ag:ub:pnp}.\proEnd\end{pro}
% ....................................................................
% <<Pro upper bound ag p>>
% ....................................................................
\begin{pro}[Proof of \cref{ak:ag:ub2:pnp}.]
  Consider the case \ref{ak:ag:ub2:pnp:p}.  If the additional
  assumption \ref{ak:ag:ub2:pnp:pc} is satisfied, then we have
  trivially
  $\exp\big(\tfrac{-\cmiSv[\sDi{\ssY}]\sDi{\ssY}}{\Di_{\ydf}}\big)\leq\ssY^{-1}$
  $\ssY> \ssY_{\xdf,\iSv}$ while for
  $\ssY\in\nset{1,\ssY_{\xdf,\iSv}}$ we have
  $\exp\big(\tfrac{-\cmiSv[\sDi{\ssY}]\sDi{\ssY}}{\Di_{\ydf}}\big)\leq1\leq
  \ssY_{\xdf,\iSv} \ssY^{-1}$. Thereby, from \eqref{ak:ag:ub:pnp:e1}
  follows immediately the assertion
  $\nRi{\txdfAg[{\erWe[]}]}{\xdf,\iSv} \leq \cst{\xdf,\iSv}\ssY^{-1}$
  for all $\ssY\in\Nz$. On the other hand side, in case
  \ref{ak:ag:ub2:pnp:np} if the additional assumption
  \ref{ak:ag:ub2:pnp:npc} is satisfied, then we have trivially
  $\exp\big(\tfrac{-\cmiSv[\aDi{\ssY}(\xdf)]\aDi{\ssY}(\xdf)}{\Di_{\ydf}}\big)\leq
  \naRa{\xdf,\iSv}$ while for $\ssY< \ssY_{\xdf,\iSv}$ we have
  $\exp\big(\tfrac{-\cmiSv[\aDi{\ssY}(\xdf)]\aDi{\ssY}(\xdf)}{\Di_{\ydf}}\big)\leq1\leq
  \ssY\naRa{\xdf,\iSv}\leq \ssY_{\xdf,\iSv}
  \naRaS{\xdf,\iSv}$. Thereby, from \eqref{ak:ag:ub:pnp:e2} with
  $\min_{\Di\in\nset{1,\ssY}}\dRa{\Di}{\xdf,\iSv}=\naRa{\xdf,\iSv}$
  follows immediately
  $ \nRi{\txdfAg[{\erWe[]}]}{\xdf,\iSv} \leq \cst{\xdf,\iSv}
  \naRa{\xdf,\iSv}$ for all $\ssY\in\Nz$, which completes the proof of
  \cref{ak:ag:ub2:pnp}.\proEnd
  \end{pro}
\subsubsection{Proofs of \cref{ak:mrb}}\label{a:ak:mrb}
% ....................................................................
% Te <<Upper bound random weights>>
% ....................................................................
\begin{te}
 Below  we state the proofs of  \cref{ak:re:SrWe:ag:mm} and \cref{ak:re:SrWe:ms:mm}. The
  proof of \cref{ak:re:SrWe:ag} is based on \cref{re:rWe:mm} given first.
\end{te}
% ....................................................................
% <<Upper bound random weights>>
% ....................................................................
\begin{lm}\label{re:rWe:mm} Consider the data-driven aggreagtion weights
  $\rWe[]$ as in \eqref{ak:de:rWe}. Under definition
  \ref{ak:ass:pen:oo} for any $l\in\nset{1,\ssY}$ with
  $\daRaS{l}{\xdfCw[],\iSv}:=\daRa{l}{\xdfCw[]}$ holds
  \begin{resListeN}[]
  \item\label{re:rWe:mm:i} for all $k\in\nsetro{1,l}$ we have\\
    $\rWe\Ind{\setB{\VnormLp{\txdfPr[l]-\xdfPr[l]}^2<\cpen\daRa{l}{\xdfCw[]}/7}} 
    \leq\exp\big(\rWn\big\{-\tfrac{\VnormLp{\ProjC[0]\xdf}^2}{2}\bias^2(\xdf)
    +[\tfrac{25\cpen}{14}+\tfrac{\VnormLp{\ProjC[0]\xdf}^2}{2}]\daRaS{l}{\xdfCw[],\iSv}-\penSv\big\}\big)$
  \item\label{re:rWe:mm:ii} for all $\Di\in\nsetlo{l,\ssY}$ we have\\
    $\rWe\Ind{\setB{\VnormLp{\txdfPr-\xdfPr}^2<\penSv/7}}\leq\exp\big(\rWn\big\{-\tfrac{1}{2}\penSv
    +[\tfrac{3}{2}\VnormLp{\ProjC[0]\xdf}^2+\cpen]\daRaS{l}{\xdfCw[],\iSv}\big\}\big)$.
  \end{resListeN}
\end{lm}
% --------------------------------------------------------------------
% <<Proof Re Random weights>> angepasst
% --------------------------------------------------------------------
\begin{pro}[Proof of \cref{re:rWe:mm}.]
The proof follows line by line the proof of \cref{re:rWe} using
\eqref{ak:ass:pen:mm:c} rather than \eqref{ak:ass:pen:oo:c}, and we omit the details.\proEnd
\end{pro}
% ....................................................................
% <<Proof Re Sum Random weights>>
% ....................................................................
\begin{pro}[Proof of \cref{ak:re:SrWe:ag:mm}.]
The proof follows line by line the proof of \cref{ak:re:SrWe:ag} using
\cref{re:rWe:mm} rather than \cref{re:rWe}, and we omit the details.\proEnd  
\end{pro}
% --------------------------------------------------------------------
% Proof <<Upper bound random weights>> ms mm
% --------------------------------------------------------------------
\begin{pro}[Proof of \cref{ak:re:SrWe:ms:mm}.]
The proof follows line by line the proof of \cref{ak:re:SrWe:ms} using
\eqref{ak:ass:pen:mm:c} rather than \eqref{ak:ass:pen:oo:c}, and we omit the details.\proEnd
\end{pro}
% ....................................................................
% <<Pro upper bound ag p>>
% ....................................................................
\begin{pro}[Proof of \cref{ak:ag:ub:pnp:mm}.]Keep in mind that
  $\VnormLp{\ProjC[0]\xdf}^2\leq\xdfCr^2$ for all $\xdf\in\rwCxdf$.
From
\eqref{ak:ag:ub:mm:e1} follows for any $\xdf\in\rwCxdf$, $\mdDi,\pdDi\in\nset{1,n}$ and associated
$\mDi,\pDi\in\nset{1,n}$ as defined in  \eqref{ak:de:*Di:ag:mm}%
 \begin{multline}\label{ak:ag:ub:pnp:mm:p1}
    \FuEx[\ssY]{\rY}\VnormLp{\txdfAg[{\erWe[]}]-\xdf}^2\leq \tfrac{2}{7}\penSv[\pDi]
    +2\VnormLp{\ProjC[0]\xdf}^2\bias[\mDi]^2(\xdf)
    %\\\hfill
 + \cst{}\xdfCr^2
 \exp\big(\tfrac{-\cmiSv[\mdDi]\mdDi}{200\Vnorm[{\xdfCw[]}]{\edf}\xdfCr}\big)%\big]
\\ +\cst{}\big[%\tfrac{1}{\rWc}+
 \xdfCr^2+\miSv[\Di_{\edf,\xdfCr}]^2\Di_{\edf,\xdfCr}^2+\miSv[\ssY_{o}]^2 \big]\ssY^{-1}% \\
\end{multline}
 For $\aDi{\ssY}(\xdfCw[])\in\nset{1,n}$
as in \ref{ak:ass:pen:mm}  set $\pdDi:=\aDi{\ssY}(\xdfCw[])$ and $\mdDi:=\sDi{\ssY}\in\nset{\aDi{\ssY}(\xdfCw[]),\ssY}$ by exploiting the definition
\eqref{ak:de:*Di:ag:mm} of $\pDi$ and $\mDi$ it follows
$\penSv[\pDi] \leq 2[3\xdfCr^2+ 2\cpen]
\daRa{\pdDi}{\xdfCw[]}$ and 
$\VnormLp{\ProjC[0]\xdf}^2\bias[\mDi]^2(\xdf)\leq
      [\xdfCr^2+4\cpen]\daRa{\sDi{\ssY}}{\xdfCw[]}$
which together with
$\daRa{\sDi{\ssY}}{\xdfCw[]}\geq\naRa{\xdfCw[]}=\daRa{\pdDi}{\xdfCw[]}\geq\ssY^{-1}$
and exploiting 
\eqref{ak:ag:ub:pnp:mm:p1} implies%
 \begin{multline}\label{ak:ag:ub:pnp:mm:p2}
   \sup_{\xdf\in\rwCxdf}\FuEx[\ssY]{\rY}\VnormLp{\txdfAg[{\erWe[]}]-\xdf}^2% \leq
 %   \tfrac{2}{7}\penSv[\pDi]
 %    +2\VnormLp{\ProjC[0]\xdf}^2\bias[\mDi]^2(\xdf)
 % + \cst{}\VnormLp{\ProjC[0]\xdf}^2\Ind{\{\mDi>1\}}\exp\big(\tfrac{-\cmiSv[\mdDi]\mdDi}{200\Vnormlp[1]{\fydf}}\big)\\
 %    +\cst{}\big[\VnormLp{\ProjC[0]\xdf}^2\Ind{\{\mDi>1\}}
 %    +\miSv[\Di_{\ydf}]^2\Di_{\ydf}^2+\miSv[\ssY_{o}]^2 \big]\ssY^{-1}\\
    \leq 
   \cst{}(\xdfCr^2\vee1)\min_{\Di\in\nset{1,\ssY}}\big[\daRa{\Di}{\xdfCw[]}\vee\exp\big(\tfrac{-\cmiSv\Di}{200\Vnorm[{\xdfCw[]}]{\edf}\xdfCr}\big)\big]\\
   +\cst{}\big[\miSv[\Di_{\edf,\xdfCr}]^2\Di_{\edf,\xdfCr}^2+\miSv[\ssY_{o}]^2 \big]\ssY^{-1}
\end{multline}
which shows the assertion \eqref{ak:ag:ub:pnp:mm:e1} and  completes the
proof of \cref{ak:ag:ub:pnp:mm}.\proEnd\end{pro}
% ....................................................................
% <<Pro upper bound ag p>>
% ....................................................................
\begin{pro}[Proof of \cref{ak:ag:ub2:pnp:mm}.] Under
  \ref{ak:ag:ub2:pnp:mm:c} holds  
  $\exp\big(\tfrac{-\cmiSv[\aDi{\ssY}({\xdfCw[]})]\aDi{\ssY}({\xdfCw[]})}{200\Vnorm[{\xdfCw[]}]{\edf}\xdfCr}\big)\leq
  \naRa{\xdfCw[]}$  for $\ssY> \ssY_{\xdfCw[],\xdfCr,\iSv}$, while 
  $\exp\big(\tfrac{-\cmiSv[\aDi{\ssY}({\xdfCw[]})]\aDi{\ssY}({\xdfCw[]})}{200\Vnorm[{\xdfCw[]}]{\edf}\xdfCr}\big)\leq1\leq
  \ssY\naRa{\xdfCw[]}\leq \ssY_{\xdfCw[],\xdfCr,\iSv}
  \naRa{{\xdfCw[]}}$ for $\ssY\in\nset{1,\ssY_{\xdfCw[],\xdfCr,\iSv}}$. Thereby, from
  \eqref{ak:ag:ub:pnp:mm:e1} with $\sDi{\ssY}:=\aDi{\ssY}({\xdfCw[]})$
  follows immediately the assertion
    $\nRi{\txdfAg[{\erWe[]}]}{\rwCxdf,\iSv}
    \leq \cst{\xdfCw[],\xdfCr,\iSv} \naRa{\xdfCw[],\iSv}$ for all
    $\ssY\in\Nz$, which  completes the
proof of \cref{ak:ag:ub2:pnp:mm}.\proEnd\end{pro}
% % ....................................................................
% % <<Re upper bound>>
% % ....................................................................
% \begin{pr}\label{re:ub}Let $\rWc\geq1$, $\cpen\geq1$, $\DipenSv=\cmiSv \Di \miSv$
%   with $\sqrt{\cmiSv}=\tfrac{\log (\Di\miSv \vee(\Di+2))}{\log(\Di+2)}\geq1$, 
% and   $\hRaDi{\Di,\xdf,\iSv}=[\bias^2(\xdf)\vee \DipenSv\,\ssY^{-1}]$
% for $\Di\in\Nz$. Given $\pdDi,\mdDi\in\nset{1,\ssY}$ let  $\mDi$ as in \eqref{de:*Di}.
% If $\dr\pdDi\geq3(\tfrac{800\Vnormlp[1]{\fydf}}{\cpen})^2$  then  there is an universal
%   numerical constant $\cst{}$ such that  for all  $\dr n\geq
%   15(\tfrac{300}{\sqrt{\cpen}})^4$ holds
% \begin{multline*}
% \FuEx[\ssY]{\rY}\VnormLp{\txdf-\xdf}^2\leq\cst{}\big\{
% [1\vee\VnormLp{\Proj[{\mHiH[0]}]^\perp\xdf}^2]\hRa{\pdDi,\xdf,\iSv}+\Vnormlp[1]{\fydf}^2
% \ssY^{-1}\}\\\hfill
% +2\VnormLp{\Proj[{\mHiH[0]^\perp}]\xdf}^2\bias[\mDi]^2(\xdf)
% +6\VnormLp{\Proj[{\mHiH[0]}]^\perp\xdf}^2 \exp\big(\tfrac{-\cpen\cmSv[\mdDi]\mdDi}{400\Vnormlp[1]{\fydf}}\big)
% \\\hfill
% \hfill+2\VnormLp{\Proj[{\mHiH[0]}]^\perp\xdf}^2\,[\mDi-1]\exp\big(-\rWc\cpen\ssY\hRa{\mdDi,\xdf,\iSv}-
%     \tfrac{\rWc\VnormLp{\Proj[{\mHiH[0]}]^\perp\xdf}^2}{4}\ssY\bias[{[\mDi-1]}]^2(\xdf)\big).
% \end{multline*}
% \end{pr}
% % ....................................................................
% % <<Pro Re upper bound>>
% % ....................................................................
% \begin{pro}[Proof of \cref{re:ub}.]
% Given $\pdDi,\mdDi\in\nset{1,\ssY}$ let $\pDi$ and $\mDi$
%   as in \eqref{de:au:*Di}. 
% From  \cref{co:agg} together with  \cref{re:SrWe} follows
%   \begin{multline*}
% \FuEx[\ssY]{\rY}\VnormLp{\txdf-\xdf}^2\leq 2\FuEx[\ssY]{\rY}\VnormLp{\txdfPr[\pDi]-\xdfPr[\pDi]}^2+2\VnormLp{\Proj[{\mHiH[0]^\perp}]\xdf}^2\bias[\mDi]^2(\xdf)+(\tfrac{2}{3\cpen\rWc^{2}}+ \tfrac{8}{\rWc})\ssY^{-1}\\\hfill
% \hfill+2\VnormLp{\Proj[{\mHiH[0]}]^\perp\xdf}^2\,[\mDi-1]\exp\big(-\rWc\cpen\ssY\hRa{\mdDi,\xdf,\iSv}-
%     \tfrac{\rWc\VnormLp{\Proj[{\mHiH[0]}]^\perp\xdf}^2}{4}\ssY\bias[{[\mDi-1]}]^2(\xdf)\big)
%     \\\hfill+2\VnormLp{\Proj[{\mHiH[0]}]^\perp\xdf}^2\FuVg[\ssY]{\rY}\big(\VnormLp{\txdfPr[\mdDi]-\xdfPr[\mdDi]}^2\geq\penSv[\mdDi]\big)\\\hfill
% +2\sum_{\Di\in\nsetlo{\pDi,\ssY}}\FuEx[\ssY]{\rY}\vectp{\VnormLp{\txdfPr-\xdfPr}^2-\cst{1}\penSv}\\
% +(24\cst{1}\cpen/\ssY)\sum_{\Di\in\nsetlo{\pDi,\ssY}}\DipenSv\FuVg[\ssY]{\rY}\vect{\VnormLp{\txdfPr-\xdfPr}^2\geq\penSv}
%  \end{multline*}
% Since $\dr\pDi\geq\pdDi\geq3(\tfrac{800\Vnormlp[1]{\fydf}}{\cpen})^2$  and  $\dr n\geq
%   15(\tfrac{300}{\sqrt{\cpen}})^4$ due to \cref{re:rest}
%   \ref{re:rest:i} and \ref{re:rest:ii} there is a finite numerical constant
%   $\cst{}$ such that
% \begin{multline*}
% \FuEx[\ssY]{\rY}\VnormLp{\txdf-\xdf}^2\leq 2\FuEx[\ssY]{\rY}\VnormLp{\txdfPr[\pDi]-\xdfPr[\pDi]}^2+2\VnormLp{\Proj[{\mHiH[0]^\perp}]\xdf}^2\bias[\mDi]^2(\xdf)+(\tfrac{2}{3\cpen\rWc^{2}}+ \tfrac{8}{\rWc})\ssY^{-1}\\\hfill
% \hfill+2\VnormLp{\Proj[{\mHiH[0]}]^\perp\xdf}^2\,[\mDi-1]\exp\big(-\rWc\cpen\ssY\hRa{\mdDi,\xdf,\iSv}-
%     \tfrac{\rWc\VnormLp{\Proj[{\mHiH[0]}]^\perp\xdf}^2}{4}\ssY\bias[{[\mDi-1]}]^2(\xdf)\big)
%     \\\hfill+2\VnormLp{\Proj[{\mHiH[0]}]^\perp\xdf}^2\FuVg[\ssY]{\rY}\big(\VnormLp{\txdfPr[\mdDi]-\xdfPr[\mdDi]}^2\geq\penSv[\mdDi]\big)\\\hfill
% +\cst{}\big[\tfrac{24\Vnormlp[1]{\fydf}^2}{\cpen}+8\big]\ssY^{-1}
% + \cst{1}\big[
% (72*\tfrac{400^2\Vnormlp[1]{\fydf}^2}{\cpen}+72*800\Vnormlp[1]{\fydf}
% + 72\cpen\big]\ssY^{-1}
%  \end{multline*}
% and together with \cref{re:conc} \ref{re:conc:ii} we obtain
% \begin{multline*}
% \FuEx[\ssY]{\rY}\VnormLp{\txdf-\xdf}^2\leq 2\FuEx[\ssY]{\rY}\VnormLp{\txdfPr[\pDi]-\xdfPr[\pDi]}^2+2\VnormLp{\Proj[{\mHiH[0]^\perp}]\xdf}^2\bias[\mDi]^2(\xdf)+(\tfrac{2}{3\cpen\rWc^{2}}+ \tfrac{8}{\rWc})\ssY^{-1}\\\hfill
% \hfill+2\VnormLp{\Proj[{\mHiH[0]}]^\perp\xdf}^2\,[\mDi-1]\exp\big(-\rWc\cpen\ssY\hRa{\mdDi,\xdf,\iSv}-
%     \tfrac{\rWc\VnormLp{\Proj[{\mHiH[0]}]^\perp\xdf}^2}{4}\ssY\bias[{[\mDi-1]}]^2(\xdf)\big)
% \\
% +6\VnormLp{\Proj[{\mHiH[0]}]^\perp\xdf}^2 \bigg[\exp\big(\tfrac{-\cpen\cmSv[\mdDi]\mdDi}{400\Vnormlp[1]{\fydf}}\big)
% +\exp\big(\tfrac{-\sqrt{n\cpen\cmSv[\mdDi]}}{100}\big)\bigg]\\\hfill
% +\cst{}\big[\tfrac{24\Vnormlp[1]{\fydf}^2}{\cpen}+8\big]\ssY^{-1}
% + \cst{1}\big[
% (72*\tfrac{400^2\Vnormlp[1]{\fydf}^2}{\cpen}+72*800\Vnormlp[1]{\fydf}
% + 72\cpen\big]\ssY^{-1}
%  \end{multline*}
% Moreover, for $\dr \ssY>
% \ssY_{\xdf,\iSv}\geq 15(\tfrac{300}{\sqrt{\cpen}})^4$  holds
% $\sqrt{n}\geq \tfrac{300}{\sqrt{\cpen}}\log(\ssY+2)\geq\tfrac{100}{\sqrt{\cpen}}\log(\ssY+2)$ which in turn
% together with $\cmiSv[\mdDi]\geq1$ implies
% $\ssY\exp\big(-\sqrt{\ssY}\tfrac{\sqrt{\cpen\cmiSv[\mdDi]}}{100}\big)\leq\exp\big(-\sqrt{\ssY}\tfrac{\sqrt{\cpen}}{100}+\log(\ssY+2)\big)\leq1$,
% and thus 
% \begin{multline*}
% \FuEx[\ssY]{\rY}\VnormLp{\txdf-\xdf}^2\leq 2\FuEx[\ssY]{\rY}\VnormLp{\txdfPr[\pDi]-\xdfPr[\pDi]}^2+2\VnormLp{\Proj[{\mHiH[0]^\perp}]\xdf}^2\bias[\mDi]^2(\xdf)+(\tfrac{2}{3\cpen\rWc^{2}}+ \tfrac{8}{\rWc})\ssY^{-1}\\\hfill
% \hfill+2\VnormLp{\Proj[{\mHiH[0]}]^\perp\xdf}^2\,[\mDi-1]\exp\big(-\rWc\cpen\ssY\hRa{\mdDi,\xdf,\iSv}-
%     \tfrac{\rWc\VnormLp{\Proj[{\mHiH[0]}]^\perp\xdf}^2}{4}\ssY\bias[{[\mDi-1]}]^2(\xdf)\big)
% \\
% +6\VnormLp{\Proj[{\mHiH[0]}]^\perp\xdf}^2 \bigg[\exp\big(\tfrac{-\cpen\cmSv[\mdDi]\mdDi}{400\Vnormlp[1]{\fydf}}\big)
% +\ssY^{-1}\bigg]\\\hfill
% +\cst{}\big[\tfrac{24\Vnormlp[1]{\fydf}^2}{\cpen}+8\big]\ssY^{-1}
% + \cst{1}\big[
% (72*\tfrac{400^2\Vnormlp[1]{\fydf}^2}{\cpen}+72*800\Vnormlp[1]{\fydf}
% + 72\cpen\big]\ssY^{-1}
%  \end{multline*}
% Recalling that $\FuEx[\ssY]{\ydf}\VnormH{\txdfPr-\xdfPr}^2\leq2\Di\oiSv/\ssY\leq
%   2\DipenSv/\ssY$ for all $\Di\in\Nz$. Taking into account the
%   definition \eqref{de:*Di} of $\pDi$ we obtain
%   $\FuEx[\ssY]{\rY}\VnormLp{\txdfPr[\pDi]-\xdfPr[\pDi]}^2 \leq 2\DipenSv[\pDi]/\ssY\leq
%   2[\tfrac{3}{12\cpen}\VnormLp{\Proj[{\mHiH[0]}]^\perp\xdf}^2+\tfrac{108}{12}]\hRa{\pdDi,\xdf,\iSv}$
%   and hence
% \begin{multline*}
% \FuEx[\ssY]{\rY}\VnormLp{\txdf-\xdf}^2\leq [\tfrac{1}{\cpen}\VnormLp{\Proj[{\mHiH[0]}]^\perp\xdf}^2+36]\hRa{\pdDi,\xdf,\iSv}+2\VnormLp{\Proj[{\mHiH[0]^\perp}]\xdf}^2\bias[\mDi]^2(\xdf)+(\tfrac{2}{3\cpen\rWc^{2}}+ \tfrac{8}{\rWc})\ssY^{-1}\\\hfill
% \hfill+2\VnormLp{\Proj[{\mHiH[0]}]^\perp\xdf}^2\,[\mDi-1]\exp\big(-\rWc\cpen\ssY\hRa{\mdDi,\xdf,\iSv}-
%     \tfrac{\rWc\VnormLp{\Proj[{\mHiH[0]}]^\perp\xdf}^2}{4}\ssY\bias[{[\mDi-1]}]^2(\xdf)\big)
% \\
% +6\VnormLp{\Proj[{\mHiH[0]}]^\perp\xdf}^2 \bigg[\exp\big(\tfrac{-\cpen\cmSv[\mdDi]\mdDi}{400\Vnormlp[1]{\fydf}}\big)
% +\ssY^{-1}\bigg]\\\hfill
% +\cst{}\big[\tfrac{24\Vnormlp[1]{\fydf}^2}{\cpen}+8\big]\ssY^{-1}
% + \cst{1}\big[
% (72*\tfrac{400^2\Vnormlp[1]{\fydf}^2}{\cpen}+72*800\Vnormlp[1]{\fydf}
% + 72\cpen\big]\ssY^{-1}
%  \end{multline*}
% Recalling that $\hRaDi{\Di,\xdf,\iSv}=[\bias^2(\xdf)\vee
% \DipenSv\,\ssY^{-1}]\geq\ssY^{-1}$ there is an universal finite numerical constant $\cst{}$ such
% that 
% \begin{multline*}
% \FuEx[\ssY]{\rY}\VnormLp{\txdf-\xdf}^2\leq\cst{}\big\{
% [1\vee\VnormLp{\Proj[{\mHiH[0]}]^\perp\xdf}^2]\hRa{\pdDi,\xdf,\iSv}+\Vnormlp[1]{\fydf}^2
% \ssY^{-1}\}\\\hfill
% +2\VnormLp{\Proj[{\mHiH[0]^\perp}]\xdf}^2\bias[\mDi]^2(\xdf)
% +6\VnormLp{\Proj[{\mHiH[0]}]^\perp\xdf}^2 \exp\big(\tfrac{-\cpen\cmSv[\mdDi]\mdDi}{400\Vnormlp[1]{\fydf}}\big)
% \\\hfill
% \hfill+2\VnormLp{\Proj[{\mHiH[0]}]^\perp\xdf}^2\,[\mDi-1]\exp\big(-\rWc\cpen\ssY\hRa{\mdDi,\xdf,\iSv}-
%     \tfrac{\rWc\VnormLp{\Proj[{\mHiH[0]}]^\perp\xdf}^2}{4}\ssY\bias[{[\mDi-1]}]^2(\xdf)\big)
% \end{multline*}
% which shows the assertion and completes the proof.\proEnd\end{pro}
% % ....................................................................
% % <<Re upper bound 1>>
% % ....................................................................
% \begin{lm}\label{re:ub:co1} If $\xdf=\bas_0$ then  there is a finite numerical
%   constant $\cst{}$ such that for all $\dr\ssY\in\Nz$ we have
%   $\FuEx[\ssY]{\rY}\VnormLp{\txdfPr[]-\xdf}^2\leq\cst{}\DipenSv[\ssY_o]\ssY^{-1}$ with  
% $\dr\ssY_o:=\ceil{15(\tfrac{300}{\sqrt{\cpen}})^4}$.
% \end{lm}
% % ....................................................................
% % <<Pro Re upper bound 1>>
% % ....................................................................
% \begin{pro}[Proof of \cref{re:ub:co1}.]
% Let $\dr\ssY_o:=\ceil{15(\tfrac{300}{\sqrt{\cpen}})^4}$. We destinguish for $\ssY\in\Nz$ the following two cases
% cases, \begin{inparaenum}[i]\renewcommand{\theenumi}{\dgrau\rm(\alph{enumi})}\item\label{pro:ub:co1:c1}
% $\ssY\in\nsetro{1,\ssY_o}$ and \item\label{pro:ub:co1:c2}
% $\ssY\geq\ssY_o$.\end{inparaenum}

% Consider \ref{pro:ub:co1:c1}. We select 
% $\dr\pDi=\ssY\leq\ssY_o$ and thus keeping in mind that $\xdf=\bas_0$,
% and hence $\VnormLp{\Proj[{\mHiH[0]}]^\perp\xdf}^2=0$  from
% \cref{co:agg} follows for all $\ssY\in\nsetro{1,\ssY_o}$
% \begin{equation}\label{pro:ub:co1:e1}
% \FuEx[\ssY]{\rY}\VnormLp{\txdf-\xdf}^2\leq2\FuEx[\ssY]{\rY}\VnormLp{\txdfPr[\ssY]-\xdfPr[\ssY]}^2\leq4\ssY\oiSv[\ssY]\ssY^{-1}\leq
% 4 \ssY_o\oiSv[\ssY_o]\ssY^{-1}\leq4\DipenSv[\ssY_o]\ssY^{-1}.
% \end{equation}

% Consider \ref{pro:ub:co1:c2}, i.e., $\ssY\geq\ssY_o$. We select
% $\dr\pdDi:=\ssY_o\in\nset{1,\ssY}$. 
% Note that $\Vnormlp[1]{\fydf}=1$ and hence, $\dr\pDi\geq\pdDi\geq
% 3(\tfrac{800\Vnormlp[1]{\fydf}}{\cpen})^2$. Therefore, for all  $\dr
% \ssY\geq \ssY_o\geq 15(\tfrac{300}{\sqrt{\cpen}})^4$ due to \cref{re:ub} 
%  follows
% \begin{multline*}
% \FuEx[\ssY]{\rY}\VnormLp{\txdf-\xdf}^2\leq\cst{}\big\{
% [1\vee\VnormLp{\Proj[{\mHiH[0]}]^\perp\xdf}^2]\hRa{\pdDi,\xdf,\iSv}+\ssY^{-1}\}\\\hfill
% +2\VnormLp{\Proj[{\mHiH[0]^\perp}]\xdf}^2\bias[\mDi]^2(\xdf)
% +6\VnormLp{\Proj[{\mHiH[0]}]^\perp\xdf}^2 \exp\big(\tfrac{-\cpen\cmSv[\mdDi]\mdDi}{400\Vnormlp[1]{\fydf}}\big)
% \\\hfill
% \hfill+2\VnormLp{\Proj[{\mHiH[0]}]^\perp\xdf}^2\,[\mDi-1]\exp\big(-\rWc\cpen\ssY\hRa{\mdDi,\xdf,\iSv}-
%     \tfrac{\rWc\VnormLp{\Proj[{\mHiH[0]}]^\perp\xdf}^2}{4}\ssY\bias[{[\mDi-1]}]^2(\xdf)\big).
% \end{multline*}
% Since
% $\VnormLp{\Proj[{\mHiH[0]}]^\perp\xdf}^2=0$, and thus
% $\hRa{\pdDi,\xdf,\iSv}=\DipenSv[\pdDi]/\ssY=\DipenSv[\ssY_o]/\ssY$,
% there is a numerical constant $\cst{}$ such that
% $\FuEx[\ssY]{\rY}\VnormLp{\txdf-\xdf}^2\leq\cst{}\DipenSv[\ssY_o]\ssY^{-1}$
% for all $\ssY\geq\ssY_o$. Combining
% the upper bounds  for the two
% cases \ref{pro:ub:co1:c1} and \ref{pro:ub:co1:c2}  we obtain the
% assertion which  completes the proof.\proEnd\end{pro}
% % ....................................................................
% % <<Re upper bound 2>>
% % ....................................................................
% \begin{lm}\label{re:ub:co2} Assume there is $K\in\Nz$
%   with   $1\geq \bias[{[K-1] }](\xdf)>0$ and $\bias[K](\xdf)=0$. Set
%  $K_{\ydf}:=K\dr\vee
% 3(\tfrac{800\Vnormlp[1]{\fydf}}{\cpen})^2$, $c_{\xdf}:=\tfrac{2\VnormLp{\Proj[{\mHiH[0]}]^\perp\xdf}^2+484\cpen}{\VnormLp{\Proj[{\mHiH[0]}]^\perp\xdf}^2\bias[{[K-1]}]^2(\xdf)}$
% and
% $\ssY_{\xdf,\iSv}=\ceil{c_{\xdf}\DipenSv[K_{\ydf}]\dr\vee15(\tfrac{300}{\sqrt{\cpen}})^4}$.\\
% If $\ssY\leq\ssY_{\xdf,\iSv}$ then let $\sDi{\ssY}:=K_{\ydf}(\log
% n)$, and otherwise if  $\ssY>\ssY_{\xdf,\iSv}$ then let
% $\sDi{\ssY}:=\max\{\Di\in\nset{K,\ssY}:c_{\xdf}\,\DipenSv<\ssY\}$
% where the defining set contains $K_{\ydf}$ and thus it is not empty.
% There is a finite numerical constant $\cst{}$ such that for all $\ssY\in\Nz$ holds
% \begin{equation}\label{re:ub:co2:e1}
% \FuEx[\ssY]{\rY}\VnormLp{\txdf-\xdf}^2
% \leq\cst{}\{\DipenSv[\ssY_{\xdf,\iSv}]+\VnormLp{\Proj[{\mHiH[0]^\perp}]\xdf}^2\ssY_{\xdf,\iSv}+ \Vnormlp[1]{\fydf}^2\}\ssY^{-1}
% + 6\VnormLp{\Proj[{\mHiH[0]}]^\perp\xdf}^2\{ \exp\big(\tfrac{-\cpen\cmSv[\sDi{\ssY}]\sDi{\ssY}}{400\Vnormlp[1]{\fydf}}\big)-\tfrac{1}{\ssY}\}.
% \end{equation}
% If there is $\widetilde{\ssY}_{\xdf,\iSv}\in\Nz$ such that for all
% $\ssY\geq\widetilde{\ssY}_{\xdf,\iSv}$ in addition
% $\cmiSv[\sDi{\ssY}]\sDi{\ssY}\geq K_{\ydf}(\log\ssY)$  holds true then  
% \begin{equation}\label{re:ub:co2:e2}
% \FuEx[\ssY]{\rY}\VnormLp{\txdf-\xdf}^2
% \leq\cst{}\{\DipenSv[{[\ssY_{\xdf,\iSv}\vee\widetilde{\ssY}_{\xdf,\iSv}]}]+\VnormLp{\Proj[{\mHiH[0]^\perp}]\xdf}^2[\ssY_{\xdf,\iSv}\vee\widetilde{\ssY}_{\xdf,\iSv}]+ \Vnormlp[1]{\fydf}^2\}\ssY^{-1}.
% \end{equation}
% \end{lm}
% % ....................................................................
% % <<Pro Re upper bound 2>>
% % ....................................................................
% \begin{pro}[Proof of \cref{re:ub:co2}.]\label{pro:ub:co2}
% Given $K\in\Nz$   with   $1\geq \bias[{[K-1] }](\xdf)>0$ and
% $\bias(\xdf)=0$ for all $\Di\geq K$ let $K_{\ydf}:=K\dr\vee
% 3(\tfrac{800\Vnormlp[1]{\fydf}}{\cpen})^2$, 
% $c_{\xdf}:=\tfrac{2\VnormLp{\Proj[{\mHiH[0]}]^\perp\xdf}^2+484\cpen}{\VnormLp{\Proj[{\mHiH[0]}]^\perp\xdf}^2\bias[{[K-1]}]^2(\xdf)}$
% and
% $\ssY_{\xdf,\iSv}=\ceil{c_{\xdf}\DipenSv[K_{\ydf}]\dr\vee15(\tfrac{300}{\sqrt{\cpen}})^4}$
% we distinguish for $\ssY\in\Nz$ the following two
% cases, \begin{inparaenum}[i]\renewcommand{\theenumi}{\dgrau\rm(\alph{enumi})}\item\label{pro:ub:co2:c1}
% $\ssY\in\nsetro{1,\ssY_{\xdf,\iSv}}$ and \item\label{pro:ub:co2:c2}
% $\ssY>\ssY_{\xdf,\iSv}$.\end{inparaenum}\\

% Firstly, consider \ref{pro:ub:co2:c1},  let
% $\ssY\in\nsetro{1,\ssY_{\xdf,\iSv}}$, then setting $\mDi=1$ and
% $\pDi=\ssY$ from \cref{co:agg} follows
% \begin{multline}\label{pro:ub:co2:e1}
% \FuEx[\ssY]{\rY} \VnormLp{\txdf-\xdf}^2\leq 2\FuEx[\ssY]{\rY}\VnormLp{\txdfPr[\ssY]-\xdfPr[\ssY]}^2+2\VnormLp{\Proj[{\mHiH[0]^\perp}]\xdf}^2\bias[1]^2(\xdf)
% \\\hfill\leq 4\ssY\oiSv[\ssY]\ssY^{-1}+2\VnormLp{\Proj[{\mHiH[0]^\perp}]\xdf}^2\leq
% 4 \ssY_{\xdf,\iSv}\oiSv[\ssY_{\xdf,\iSv}]\ssY^{-1}+2\VnormLp{\Proj[{\mHiH[0]^\perp}]\xdf}^2\ssY_{\xdf,\iSv}\ssY^{-1}\\\leq(4\DipenSv[\ssY_{\xdf,\iSv}]+2\VnormLp{\Proj[{\mHiH[0]^\perp}]\xdf}^2\ssY_{\xdf,\iSv})\ssY^{-1}.
% \end{multline}

% Secondly, consider \ref{pro:ub:co2:c2}, i.e., 
% $\ssY>\ssY_{\xdf,\iSv}$. Setting $\pdDi:=K_{\ydf}\leq\DipenSv[K_{\ydf}]\leq\ssY_{\xdf,\iSv}$, i.e., $\pdDi\in\nset{1,\ssY}$ from $\pdDi=K_{\ydf}\geq K$  follows
% $\bias[\pdDi](\xdf)=0$ and hence
% $\hRa{\pdDi,\xdf,\iSv}=\DipenSv[K_{\ydf}]\ssY^{-1}$. 
% Keeping in mind that $\dr\pdDi\geq
% 3(\tfrac{800\Vnormlp[1]{\fydf}}{\cpen})^2$ and  $\dr
% \ssY\geq \ssY_o\geq 15(\tfrac{300}{\sqrt{\cpen}})^4$ 
% from
% \cref{re:ub} follows
% \begin{multline}\label{pro:ub:co2:e2}
% \FuEx[\ssY]{\rY}\VnormLp{\txdf-\xdf}^2\leq\cst{}\big\{
% [1\vee\VnormLp{\Proj[{\mHiH[0]}]^\perp\xdf}^2]\DipenSv[K_{\ydf}]\ssY^{-1}+\Vnormlp[1]{\fydf}^2
% \ssY^{-1}\}\\\hfill
% +2\VnormLp{\Proj[{\mHiH[0]^\perp}]\xdf}^2\bias[\mDi]^2(\xdf)
% +6\VnormLp{\Proj[{\mHiH[0]}]^\perp\xdf}^2 \exp\big(\tfrac{-\cpen\cmSv[\mdDi]\mdDi}{400\Vnormlp[1]{\fydf}}\big)
% \\\hfill
% \hfill+2\VnormLp{\Proj[{\mHiH[0]}]^\perp\xdf}^2\,[\mDi-1]\exp\big(-\rWc\cpen\ssY\hRa{\mdDi,\xdf,\iSv}-
%     \tfrac{\rWc\VnormLp{\Proj[{\mHiH[0]}]^\perp\xdf}^2}{4}\ssY\bias[{[\mDi-1]}]^2(\xdf)\big).
% \end{multline}
% Since
% $\ssY>\ssY_{\xdf,\iSv}\geq c_{\xdf}\DipenSv[K_{\ydf}]$
% with $c_{\xdf}=\tfrac{2\VnormLp{\Proj[{\mHiH[0]}]^\perp\xdf}^2+484\cpen}{\VnormLp{\Proj[{\mHiH[0]}]^\perp\xdf}^2\bias[{[K-1]}]^2(\xdf)}$ the defining set of
% $\sDi{\ssY}:=\max\{\Di\in\nset{K,\ssY}:\ssY>c_{\xdf,\iSv}\DipenSv\}$
% evenutally containing $K_{\ydf}$ is not empty. Consequently, $\sDi{\ssY}\geq K$ and $\VnormLp{\Proj[{\mHiH[0]}]^\perp\xdf}^2\bias[{[K-1]}]^2(\xdf)>[2\VnormLp{\Proj[{\mHiH[0]}]^\perp\xdf}^2+484\cpen]\DipenSv[\sDi{\ssY}]/\ssY=[2\VnormLp{\Proj[{\mHiH[0]}]^\perp\xdf}^2+484\cpen]\hRa{\sDi{\ssY},\xdf,\iSv}$. Therefore,
%   setting $\mdDi:=\sDi{\ssY}$ the definition  \eqref{de:*Di} implies  $\mDi=K$ and hence
%   $\bias[\mDi]^2(\xdf)=\bias[K]^2(\xdf)=0$,
%   $\bias[{[\mDi-1]}]^2(\xdf)=\bias[{[K-1]}]^2(\xdf)>0$. From
%   \eqref{pro:ub:co2:e2} follows for all $\ssY>\ssY_{\xdf,\iSv}$  thus
% \begin{multline}\label{pro:ub:co2:e3}
% \FuEx[\ssY]{\rY}\VnormLp{\txdf-\xdf}^2\leq\cst{}\big\{
% [1\vee\VnormLp{\Proj[{\mHiH[0]}]^\perp\xdf}^2]\DipenSv[K_{\ydf}]\ssY^{-1}+\Vnormlp[1]{\fydf}^2
% \ssY^{-1}\}\\\hfill
% +6\VnormLp{\Proj[{\mHiH[0]}]^\perp\xdf}^2 \exp\big(\tfrac{-\cpen\cmSv[\sDi{\ssY}]\sDi{\ssY}}{400\Vnormlp[1]{\fydf}}\big)
% \\\hfill
% \hfill+2\VnormLp{\Proj[{\mHiH[0]}]^\perp\xdf}^2\,\sDi{\ssY}\exp\big(-\rWc\cpen\ssY\hRa{\mdDi,\xdf,\iSv}-
%     \tfrac{\rWc\VnormLp{\Proj[{\mHiH[0]}]^\perp\xdf}^2}{4}\ssY\bias[{[K-1]}]^2(\xdf)\big)\\\hfill
% \leq\cst{}\big\{
% [1\vee\VnormLp{\Proj[{\mHiH[0]}]^\perp\xdf}^2]\DipenSv[K_{\ydf}]\ssY^{-1}+\Vnormlp[1]{\fydf}^2
% \ssY^{-1}\}\\\hfill
% +6\VnormLp{\Proj[{\mHiH[0]}]^\perp\xdf}^2 \exp\big(\tfrac{-\cpen\cmSv[\sDi{\ssY}]\sDi{\ssY}}{400\Vnormlp[1]{\fydf}}\big)
% \\\hfill
% +2\VnormLp{\Proj[{\mHiH[0]}]^\perp\xdf}^2\,[K-1]\underbrace{\exp\big(-\tfrac{\rWc\VnormLp{\Proj[{\mHiH[0]}]^\perp\xdf}^2}{4}\ssY\bias[{[\mDi-1]}]^2(\xdf)\big)}_{\leq
%   \tfrac{4}{\rWc\VnormLp{\Proj[{\mHiH[0]}]^\perp\xdf}^2\bias[{[K-1]}]^2(\xdf)}\ssY^{-1}\exp(-1)}
% \end{multline}
% Note that $\DipenSv[K_{\ydf}]\leq\ssY_{\xdf,\iSv}$ and
% $\tfrac{8[K-1]}{e\rWc\bias[{[K-1]}]^2(\xdf)}\leq\tfrac{1}{\rWc}\VnormLp{\Proj[{\mHiH[0]}]^\perp\xdf}^2\ssY_{\xdf,\iSv}$. Thereby,
% we obtain 
% \begin{multline}\label{pro:ub:co2:e4}
% \FuEx[\ssY]{\rY}\VnormLp{\txdf-\xdf}^2
% \leq\cst{2}\{\VnormLp{\Proj[{\mHiH[0]}]^\perp\xdf}^2\ssY_{\xdf,\iSv}+ \Vnormlp[1]{\fydf}^2\}\ssY^{-1}\\
% + 6\VnormLp{\Proj[{\mHiH[0]}]^\perp\xdf}^2\{ \exp\big(\tfrac{-\cpen\cmSv[\sDi{\ssY}]\sDi{\ssY}}{400\Vnormlp[1]{\fydf}}\big)-\tfrac{1}{\ssY}\}
%  \end{multline}
% for some finite numerical constant $\cst{2}$.\\
% Combining
% the upper bounds 
% \eqref{pro:ub:co2:e1} and
% \eqref{pro:ub:co2:e4} for the two
% cases \ref{pro:ub:co2:c1} and \ref{pro:ub:co2:c2}  we obtain the
% assertion \eqref{re:ub:co2:e1}, that is, there is a finite numerical
% constant $\cst{}$ such that  for all
% $\ssY\in\Nz$ holds
% \begin{multline}\label{pro:ub:co2:e5}
% \FuEx[\ssY]{\rY}\VnormLp{\txdf-\xdf}^2
% \leq\cst{}\{\DipenSv[\ssY_{\xdf,\iSv}]+\VnormLp{\Proj[{\mHiH[0]^\perp}]\xdf}^2\ssY_{\xdf,\iSv}+ \Vnormlp[1]{\fydf}^2\}\ssY^{-1}\\
% + 6\VnormLp{\Proj[{\mHiH[0]}]^\perp\xdf}^2\{ \exp\big(\tfrac{-\cpen\cmSv[\sDi{\ssY}]\sDi{\ssY}}{400\Vnormlp[1]{\fydf}}\big)-\tfrac{1}{\ssY}\}
% \end{multline}

% Assume finally, that there is in addition
% $\widetilde{\ssY}_{\xdf,\iSv}\in\Nz$ such that
% $\cmiSv[\sDi{\ssY}]\sDi{\ssY}\geq K_{\ydf}(\log\ssY)$ for all
% $\ssY\geq\widetilde{\ssY}_{\xdf,\iSv}$. We shall use without further
% reference that then $\exp\big(\tfrac{-\cpen\cmSv[\sDi{\ssY}]\sDi{\ssY}}{400\Vnormlp[1]{\fydf}}\big)\leq\ssY^{-1}$ for
% all $\ssY\geq\widetilde{\ssY}_{\xdf,\iSv}$ since $K_{\ydf}\geq
% \tfrac{400\Vnormlp[1]{\fydf}}{\cpen}$. Following line by line the
% proof of \eqref{pro:ub:co2:e5} using
% $\widetilde{\ssY}_{\xdf,\iSv}\vee\ssY_{\xdf,\iSv}$  rather than
% $\ssY_{\xdf,\iSv}$  we obtain the
% assertion, that is,
% $\FuEx[\ssY]{\rY}\VnormLp{\txdf-\xdf}^2
% \leq\cst{}\{\DipenSv[{[\ssY_{\xdf,\iSv}\vee\widetilde{\ssY}_{\xdf,\iSv}]}]+\VnormLp{\Proj[{\mHiH[0]^\perp}]\xdf}^2[\ssY_{\xdf,\iSv}\vee\widetilde{\ssY}_{\xdf,\iSv}]+ \Vnormlp[1]{\fydf}^2\}\ssY^{-1}$, which completes the proof.\proEnd\end{pro}
% % ....................................................................
% % <<Re upper bound 3>>
% % ....................................................................
% \begin{lm}\label{re:ub:co3} Assume that $\bias(\xdf)>0$ for all
%   $\Di\in\Nz$.  Set
%   $\Di_{\ydf}:=\dr3(\tfrac{800\Vnormlp[1]{\fydf}}{\cpen})^2$, 
% $\tDi_{\ydf}=\min\{\Di\in\Nz:\bias[\Di_{\ydf}](\xdf)>\bias[\Di](\xdf)\}$
% and
% $\ssY_{\xdf,\iSv}:=\ceil{\tfrac{\DipenSv[\tDi_{\ydf}]}{\bias[\tDi_{\ydf}]^2(\xdf)}\vee\dr15(\tfrac{300}{\sqrt{\cpen}})^4}$. Then,
% there is a finite numerical constant $\cst{}$ such that for all
% $\ssY\in\Nz$ and $\sDi{\ssY}\in\nset{\aDi{\ssY},\ssY}$ holds 
% \begin{multline}\label{re:ub:co3:e1}
% \FuEx[\ssY]{\rY}\VnormLp{\txdf-\xdf}^2\leq
% \cst{}\{[1\vee\VnormLp{\Proj[{\mHiH[0]}]^\perp\xdf}^2]\hRaDi{\sDi{\ssY},\xdf,\iSv}
% + (\Vnormlp[1]{\fydf}^2+\DipenSv[\ssY_{\xdf,\iSv}]+\VnormLp{\Proj[{\mHiH[0]^\perp}]\xdf}^2\ssY_{\xdf,\iSv})\ssY^{-1}  \}\\ + 8\VnormLp{\Proj[{\mHiH[0]}]^\perp\xdf}^2 \big\{\exp\big(\tfrac{-\cpen}{400\Vnormlp[1]{\fydf}}\cmSv[\sDi{\ssY}]\sDi{\ssY}\big)-\hRa{\xdf,\iSv}\big\}.
% \end{multline}
% If there is  $\widetilde{\ssY}_{\xdf,\iSv}\in\Nz$ such that for all
% $\ssY\geq\widetilde{\ssY}_{\xdf,\iSv}$ in addition
% $\cmiSv[\aDi{\ssY}]\aDi{\ssY}\geq \Di_{\ydf}|\log\hRa{\xdf,\iSv}|$  holds true then  
% for all $\ssY\in\Nz$ we have
% \begin{multline}\label{re:ub:co3:e2}
% \FuEx[\ssY]{\rY}\VnormLp{\txdf-\xdf}^2\leq
% \cst{}\big\{[1\vee\VnormLp{\Proj[{\mHiH[0]}]^\perp\xdf}^2]\hRa{\xdf,\iSv}
% \\+ (\Vnormlp[1]{\fydf}^2+\DipenSv[{[\ssY_{\xdf,\iSv}\vee\widetilde{\ssY}_{\xdf,\iSv}]}]+\VnormLp{\Proj[{\mHiH[0]^\perp}]\xdf}^2[\ssY_{\xdf,\iSv}\vee\widetilde{\ssY}_{\xdf,\iSv}])\ssY^{-1}  \big\}.
% \end{multline}
% \end{lm}
% % ....................................................................
% % <<Rem upper bound 3>>
% % ....................................................................
% \begin{rem}Keep in mind that
% $\aDi{\ssY}:=\argmin\Nset[{\Di\in\nset{1,\ssY}}]{\hRaDi{\Di,\xdf,\iSv}}$
% with $\hRaDi{\Di,\xdf,\iSv}=[\bias^2(\xdf)\vee \DipenSv\,\ssY^{-1}]$
% and $\hRa{\xdf,\iSv}:=\hRaDi{\aDi{\ssY},\xdf,\iSv}$.
% Considering $\Di_{\ydf}:=\dr3(\tfrac{800\Vnormlp[1]{\fydf}}{\cpen})^2$ and
% $\tDi_{\ydf}=\min\{\Di\in\Nz:\bias[\Di_{\ydf}](\xdf)>\bias[\Di](\xdf)\}$
% as defined in \cref{re:ub:co3} the defining set is not empty since $\bias[\Di](\xdf)>0$ for all
% $\Di\in\Nz$ and $\lim_{\Di\to\infty}\bias[\Di](\xdf)=0$. Moreover, it
% holds $\tDi_{\ydf}>\Di_{\ydf}$ due to the the monotonicity of
% $\bias(\xdf)$. Defining as in \cref{re:ub:co3}
% $\ssY_{\xdf,\iSv}:=\ceil{\tfrac{\DipenSv[\tDi_{\ydf}]}{\bias[\tDi_{\ydf}]^2(\xdf)}\vee\dr15(\tfrac{300}{\sqrt{\cpen}})^4}$
% where $\ssY_{\xdf,\iSv}\geq \DipenSv[\tDi_{\ydf}]\geq\tDi_{\ydf}$ by construction,
% for all $\ssY\geq\ssY_{\xdf,\iSv}$ holds 
% $\oRaDi{\Di_{\ydf},\xdf,\iSv}\geq\bias[\Di_{\ydf}]^2(\xdf)>\bias[\tDi_{\ydf}]^2(\xdf)% =\bias[\tDi_{\Sv}]^2(\So)[1\vee
% % \tfrac{\tDi_{\Sv}\cmiSv[\tDi_{\Sv}]\miSv[\tDi_{\Sv}]/\nlIm}{\bias[\tDi_{\Sv}]^2(\So)}]
% =\oRaDi{\tDi_{\ydf},\xdf,\iSv}$
% and hence, for all
% $\ssY\geq\ssY_{\xdf,\iSv}$ we have $\aDi{\ssY}>
% \Di_{\ydf}$.   We use these
% preliminary findings in the proof of \cref{re:ub:co3} without
% further reference.\remEnd  
% \end{rem}
% % ....................................................................
% % <<Pro upper bound 3>>
% % ....................................................................
% \begin{pro}[Proof of \cref{re:ub:co3}.]
% Given $\ssY_{\xdf,\iSv}\in\Nz$ as in \cref{re:ub:co3} we distinguish for $\ssY\in\Nz$ the following two
% cases, \begin{inparaenum}[i]\renewcommand{\theenumi}{\dgrau\rm(\alph{enumi})}\item\label{pro:ub:co3:c1}
% $\ssY\in\nsetro{1,\ssY_{\xdf,\iSv}}$ and \item\label{pro:ub:co3:c2}
% $\ssY>\ssY_{\xdf,\iSv}$. \end{inparaenum} Firstly, consider \ref{pro:ub:co3:c2}. We set $\pdDi=\mdDi=\sDi{\ssY}\geq\aDi{\ssY}$, 
% then $\VnormLp{\Proj[{\mHiH[0]^\perp}]\xdf}^2\bias[\mDi]^2(\xdf)\leq[2\VnormLp{\Proj[{\mHiH[0]^\perp}]\xdf}^2+484\cpen]\hRaDi{\sDi{\ssY},\xdf,\iSv}$
% exploiting the definition \eqref{de:*Di} of $\mDi\leq\mdDi$,  
% $\dr\pDi\geq\pdDi=\sDi{\ssY}\geq\aDi{\ssY}\geq \Di_{\ydf}\geq
% 3(\tfrac{800\Vnormlp[1]{\fydf}}{\cpen})^2$  and $\dr \ssY\geq
% \ssY_{\xdf,\iSv}\geq 15(\tfrac{300}{\sqrt{\cpen}})^4$
% due to \cref{re:ub} there is a finite numerical constant $\cst{}$ such
% that 
% \begin{multline}\label{pro:ub:co3:e1} 
% \FuEx[\ssY]{\rY}\VnormLp{\txdf-\xdf}^2\leq\cst{}\big\{
% [1\vee\VnormLp{\Proj[{\mHiH[0]}]^\perp\xdf}^2]\hRa{\sDi{\ssY},\xdf,\iSv}+\Vnormlp[1]{\fydf}^2
% \ssY^{-1}\}\\\hfill
% +6\VnormLp{\Proj[{\mHiH[0]}]^\perp\xdf}^2 \exp\big(\tfrac{-\cpen\cmSv[\sDi{\ssY}]\sDi{\ssY}}{400\Vnormlp[1]{\fydf}}\big)
% +2\VnormLp{\Proj[{\mHiH[0]}]^\perp\xdf}^2\,\sDi{\ssY}\exp\big(-\rWc\cpen\ssY\hRa{\sDi{\ssY},\xdf,\iSv}\big).
% \end{multline}
% Keeping in mind that
% $\hRaDi{\sDi{\ssY},\xdf,\iSv}\geq\hRaDi{\aDi{\ssY},\xdf,\iSv}=\hRa{\xdf,\iSv}$, 
% $\ssY\hRaDi{\sDi{\ssY},\xdf,\iSv}\geq\DipenSv[\sDi{\ssY}]\geq\cmiSv[\sDi{\ssY}]\sDi{\ssY}\geq\sDi{\ssY}\geq1$
% and
% $\sDi{\ssY}\exp\big(-\rWc\cpen\cmiSv[\sDi{\ssY}]\sDi{\ssY}\big)=
% \tfrac{2}{\rWc\cpen}\tfrac{\rWc\cpen}{2}\sDi{\ssY}\exp\big(-\tfrac{\rWc\cpen}{2}\cmiSv[\sDi{\ssY}]\sDi{\ssY}\big)\exp\big(-\tfrac{\rWc\cpen}{2}\cmiSv[\sDi{\ssY}]\sDi{\ssY}\big)\leq\tfrac{2}{e\rWc\cpen}\exp\big(-\tfrac{\rWc\cpen}{2}\cmiSv[\sDi{\ssY}]\sDi{\ssY}\big)$
% it follows 
% \begin{multline}\label{pro:ub:co3:e4}
% \FuEx[\ssY]{\rY}\VnormLp{\txdf-\xdf}^2
% \leq\cst{2}\{[1\vee\VnormLp{\Proj[{\mHiH[0]}]^\perp\xdf}^2]\hRaDi{\sDi{\ssY},\xdf,\iSv}
% + \Vnormlp[1]{\fydf}^2\ssY^{-1}\}\hfill \\+ 8\VnormLp{\Proj[{\mHiH[0]}]^\perp\xdf}^2 \big\{\exp\big(\tfrac{-\cpen}{400\Vnormlp[1]{\fydf}}\cmSv[\sDi{\ssY}]\sDi{\ssY}\big)-\hRa{\xdf,\iSv}\big\}.
%  \end{multline}
% where $\cst{2}$ is a finite numerical constant. 

% Secondly, consider \ref{pro:ub:co3:c1}, exploiting
% \eqref{pro:ub:co2:e1} we have $\FuEx[\ssY]{\rY} \VnormLp{\txdf-\xdf}^2\leq(4\DipenSv[\ssY_{\xdf,\iSv}]+2\VnormLp{\Proj[{\mHiH[0]^\perp}]\xdf}^2\ssY_{\xdf,\iSv})\ssY^{-1}$.
% Combining
% the last upper bound and
% \eqref{pro:ub:co3:e4} for the two
% cases \ref{pro:ub:co2:c1} and \ref{pro:ub:co2:c2}, respectively,  we obtain the
% assertion \eqref{re:ub:co3:e1}, that is, there is a finite numerical
% constant $\cst{}$ such that  for all
% $\ssY\in\Nz$ holds
% \begin{multline}\label{pro:ub:co3:e5}
% \FuEx[\ssY]{\rY}\VnormLp{\txdf-\xdf}^2
% \cst{}\{[1\vee\VnormLp{\Proj[{\mHiH[0]}]^\perp\xdf}^2]\hRaDi{\sDi{\ssY},\xdf,\iSv}
% + (\Vnormlp[1]{\fydf}^2+\DipenSv[\ssY_{\xdf,\iSv}]+\VnormLp{\Proj[{\mHiH[0]^\perp}]\xdf}^2\ssY_{\xdf,\iSv})\ssY^{-1}  \}\\ + 8\VnormLp{\Proj[{\mHiH[0]}]^\perp\xdf}^2 \big\{\exp\big(\tfrac{-\cpen}{400\Vnormlp[1]{\fydf}}\cmSv[\sDi{\ssY}]\sDi{\ssY}\big)-\hRa{\xdf,\iSv}\big\}.
% \end{multline}
% Assume finally, that there is in addition
% $\widetilde{\ssY}_{\xdf,\iSv}\in\Nz$ such that
% $\cmiSv[\sDi{\ssY}]\sDi{\ssY}\geq \Di_{\ydf}|\log\hRa{\xdf,\iSv}|$ for all
% $\ssY\geq\widetilde{\ssY}_{\xdf,\iSv}$. We shall use without further
% reference that then $\exp\big(\tfrac{-\cpen\cmSv[\sDi{\ssY}]\sDi{\ssY}}{400\Vnormlp[1]{\fydf}}\big)\leq\hRa{\xdf,\iSv}$ for
% all $\ssY\geq\widetilde{\ssY}_{\xdf,\iSv}$ since $\Di_{\ydf}\geq
% \tfrac{400\Vnormlp[1]{\fydf}}{\cpen}$.  Following line by line the
% proof of \eqref{pro:ub:co3:e5} using
% $\widetilde{\ssY}_{\xdf,\iSv}\vee\ssY_{\xdf,\iSv}$  rather than
% $\ssY_{\xdf,\iSv}$  we obtain the
% assertion, that is,
% $\FuEx[\ssY]{\rY}\VnormLp{\txdf-\xdf}^2
% \leq\cst{}\{[1\vee\VnormLp{\Proj[{\mHiH[0]}]^\perp\xdf}^2]\hRaDi{\sDi{\ssY},\xdf,\iSv}
% + (\Vnormlp[1]{\fydf}^2+\DipenSv[{[\ssY_{\xdf,\iSv}\vee\widetilde{\ssY}_{\xdf,\iSv}]}]+\VnormLp{\Proj[{\mHiH[0]^\perp}]\xdf}^2[\ssY_{\xdf,\iSv}\vee\widetilde{\ssY}_{\xdf,\iSv}])\ssY^{-1}  \}$, which completes the proof.\proEnd\end{pro}
%%% Local Variables:
%%% mode: latex
%%% TeX-master: "_0DACD"
%%% End:






%\section{Intermediate results}
%\begin{pr}\label{PR_FREQ_CIRCDECONV_KNOWN_IID_ORACLE_NP_TALAGRAND}
%Let be any double indexed sequences $(\delta^{\star}_{k, l})_{0 \leq k < l \leq n}$ and $(\Delta^{\star}_{k, l})_{0 \leq k < l \leq n}$ such that
%\begin{alignat*}{3}
%& \delta^{\star}_{k, l} && \geq && \sum\limits_{k < j \leq l} \Lambda_{j};\\
%& \Delta^{\star}_{k, l} && \geq && \max\limits_{k < j \leq l} \Lambda_{j}.
%\end{alignat*}
%Then, with the constants $K := \frac{\sqrt{2} - 1}{21 \sqrt{2}}$ and $C > 0$, we have
%
%
%\begin{alignat}{2}
%& \E_{\theta^{\circ}}^{n} && \left[\left(\sup\limits_{t \in \mathds{B}_{k, l}} \vert \left\langle t \vert \overline{\theta} - \theta^{\circ} \right\rangle_{l^{2}} \vert^{2} - 6 \frac{\psi_{n} \delta^{\star}_{k, l}}{n}\right)_{+}\right]\notag\\
%& && \leq C \left\{\frac{\Vert \lambda \Vert_{l^{2}} \Vert \theta^{\circ} \Vert_{l^{2}} \Delta^{\star}_{k, l}}{n} \exp\left[ -\frac{\psi_{n} \delta^{\star}_{k, l}}{6 \Vert \lambda \Vert_{l^{2}} \Vert \theta^{\circ} \Vert_{l^{2}} \Delta^{\star}_{k, l}} \right] + \frac{\delta^{\star}_{k, l}}{n^{2}} \exp\left[- K \sqrt{n \psi_{n}}\right]\right\}\label{EQD.1}\\
%& \P_{\theta^{\circ}}^{n} && \left(\sup\limits_{t \in \mathds{B}_{k, l}} \vert \left\langle t \vert \overline{\theta} - \theta^{\circ} \right\rangle_{l^{2}} \vert^{2} \geq 3 \frac{\psi_{n} \delta^{\star}_{k, l}}{n}\right)\notag\\
%& && \leq 3 \exp\left[ -K \left(\frac{\psi_{n} \delta^{\star}_{k, l}}{\Delta^{\star}_{k, l} \Vert \lambda \Vert_{l^{2}} \Vert \theta^{\circ} \Vert_{l^{2}}} \wedge \sqrt{n \psi_{n}} \right)\right]\label{EQD.2}
%\end{alignat}
%\end{pr}
%
%\begin{de}\label{DE_FREQ_CIRCDECONV_KNOWN_IID_ORACLE_NP_BOUNDS}
%Define the following quantities :
%\begin{alignat*}{3}
%& G_{n}^{\dagger -} &&:=&& \min\left\{m \in \llbracket 1, m_{n}^{\dagger} \rrbracket : \quad \mathfrak{b}_{m}^{2} \leq \frac{176}{3} \mathfrak{b}_{0}^{2} \Phi^{\dagger}_{n}\right\},\\
%& G_{n}^{\dagger +} &&:=&& \max \left\{m \in \llbracket m_{n}^{\dagger}, n \rrbracket : \psi_{n} m \Lambda_{m} \leq \frac{25}{3} n \mathfrak{b}_{0}^{2} \Phi_{n}^{\dagger} \right\}.
%\end{alignat*}
%\end{de}
%
%\begin{pr}\label{PR_FREQ_CIRCDECONV_KNOWN_IID_ORACLE_NP_CONTRACTTHRESHOLD}
%For any $n$ and $\eta$ in $\N$, and constant $C_{\lambda, \theta^{\circ}} > \sum\limits_{j = 1}^{\infty} \exp\left[- \eta \frac{\psi_{n} m \Lambda_{(m)}}{2}\right]$ we have
%
%\begin{alignat}{3}
%& \E_{\theta^{\circ}}^{n}\left[\P_{M \vert Y^{n}}^{n, (\eta)}\left(\llbracket 0, G^{\dagger -}_{n} - 1 \rrbracket\right)\right] && \leq && 4 m^{\dagger}_{n} \exp\left[- K \left(\frac{\psi_{n} 2 m^{\dagger}_{n}}{\Vert \theta^{\circ} \Vert_{l^{2}} \Vert \lambda \Vert_{l^{2}}} \wedge \sqrt{n \psi_{n}}\right)\right]\label{EQD.3}\\
%& \E_{\theta^{\circ}}^{n}\left[\P_{M \vert Y^{n}}^{n, (\eta)}\left(\llbracket G^{\dagger +}_{n} + 1, n \rrbracket\right)\right] && \leq && C_{\lambda, \theta^{\circ}} \exp\left[- K \left(\frac{\psi_{n} 2 m^{\dagger}_{n}}{\Vert \theta^{\circ} \Vert_{l^{2}} \Vert \lambda \Vert_{l^{2}}} \wedge \sqrt{n \psi_{n}}\right)\right]\label{EQD.4}
%\end{alignat}
%\end{pr}
%
%\begin{pr}\label{PR_FREQ_CIRCDECONV_KNOWN_IID_ORACLE_NP_DECOMPOSITION}
%\begin{alignat}{2}
%& \sum\limits_{0 < \vert j \vert \leq n} && \Lambda_{j} \E_{\theta^{\circ}}^{n}\left[\left(\lambda_{j} \overline{\theta}_{j} - \lambda_{j} \theta^{\circ}_{j}\right)^{2} \P_{M \vert Y^{n}}^{n, (\eta)}\left(\llbracket \vert j \vert, n \rrbracket\right)\right]\notag \\
%& && \leq 28 \mathfrak{b}_{0}^{2} \Phi^{\dagger}_{n} + \frac{1}{n} 12 C_{\lambda, \theta^{\circ}} \exp\left[- K \left(\frac{\psi_{n} 2 m^{\dagger}_{n}}{\Vert \lambda \Vert_{l^{2}} \Vert \theta^{\circ} \Vert_{l^{2}}} \wedge \sqrt{n \psi_{n}}\right) + \log\left(\psi_{n} n \Lambda_{n}\right)\right]\label{EQD.5}\\
%& \sum\limits_{0 < \vert j \vert \leq n} && \left(\theta^{\circ}_{j}\right)^{2} \E_{\theta^{\circ}}^{n}\left[\P_{M \vert Y^{n}}^{n, (\eta)}\left(\llbracket 0, j-1 \rrbracket \right)\right] + \sum\limits_{ \vert j \vert > n} \left( \theta^{\circ}_{j}\right)^{2}\notag\\
%& && \leq 59 \mathfrak{b}_{0}^{2} \Phi^{\dagger}_{n} + \frac{1}{n} 4 C_{\lambda, \theta^{\circ}} \exp\left[- K \left(\frac{\psi_{n} 2 m^{\circ}_{n}}{\Vert \lambda \Vert_{l^{2}} \Vert \theta^{\circ} \Vert_{l^{2}}} \wedge \sqrt{n \psi_{n}}\right) + \log\left(n m^{\dagger}_{n}\right)\right]\label{EQD.6}
%\end{alignat}
%\end{pr}
%
%\section{Detailed proofs}
%\begin{pro}{\textsc{Proof of \nref{PR_FREQ_CIRCDECONV_KNOWN_IID_ORACLE_NP_TALAGRAND}} \\}\label{PROD.3.1}
%To prove this result, we will use \nref{LM_TALAGRAND}.
%
%Throughout the proof, $m$ and $l$ are two positive integers with $m < l$.
%
%\medskip
%
%For any $t$ in $\mathds{B}_{m,l}$, observe
%\begin{alignat*}{3}
%& \left\langle t \vert \overline{\theta} \right\rangle && = && \sum\limits_{m \leq \vert j \vert \leq l} \left(t_{j} \cdot \frac{1}{n} \sum\limits_{p = 1}^{n} \frac{e_{j}\left(-Y_{p}^{n}\right)}{\overline{\lambda_{j}}}\right)\\
%& && = && \frac{1}{n} \sum\limits_{p = 1}^{n} \sum\limits_{m \leq \vert j \vert \leq l} \left(\frac{t_{j}}{\overline{\lambda_{j}}} \cdot e_{j}\left(-Y_{p}^{n}\right)\right)\\
%& && = && \frac{1}{n} \sum\limits_{p = 1}^{n} \mathcal{F}^{-1}\left(\frac{t}{\overline{\lambda}}\right)\left(-Y_{p}^{n}\right).
%\end{alignat*}
%
%Using the same notations as in \nref{LM_TALAGRAND}, we define
%\[\nu_{t} := \sum\limits_{m \leq \vert j \vert \leq l} \left(\frac{t_{j}}{\overline{\lambda_{j}}}\right) e_{j} = \mathcal{F}^{-1}\left(\frac{t}{\overline{\lambda}}\right),\]
%
%which gives the following formulation
%
%\begin{alignat*}{3}
%& \left\langle t \vert \overline{\theta} - \theta^{\circ} \right\rangle && = && \frac{1}{n} \sum\limits_{p = 1}^{n} \left(\nu_{t}(Y_{p}^{n}) - \E_{\theta^{\circ}}^{n}\left[\nu_{t}(Y_{p}^{n})\right]\right).
%\end{alignat*}
%
%We obtain hence our value for $h^{2}$
%\begin{alignat}{3}
%& \sup\limits_{t \in \mathds{B}_{m, l}} \sup\limits_{y \in [0, 1]} \left\vert \nu_{t} \right\vert^{2} && = && \sup\limits_{t \in \mathds{B}_{m, l}} \sup\limits_{y \in [0, 1]} \left\vert \sum\limits_{m \leq \vert j \vert \leq l} \frac{t_{j}}{\overline{\lambda_{j}}} \cdot e_{j}(y) \right\vert^{2}\notag\\
%& && \leq && \sup\limits_{y \in [0, 1]} \sum\limits_{m \leq \vert j \vert \leq l} \left\vert \frac{1}{\overline{\lambda_{j}}} \cdot e_{j}(y) \right\vert^{2}\notag\\
%& && \leq && \sum\limits_{m \leq \vert j \vert \leq l} \Lambda_{j}\notag\\
%& && \leq && \delta^{\star}_{m, l} =: h^{2}. \label{EQD.7}
%\end{alignat}
%
%We now use Cauchy-Schwarz inequality to obtain $H^{2}$
%\begin{alignat*}{3}
%& \E_{\theta^{\circ}}^{n}\left[\sup\limits_{t \in \mathds{B}_{m, l}} \vert \overline{\nu}_{t} \vert^{2} \right] && = && \E_{\theta^{\circ}}^{n}\left[\sup\limits_{t \in \mathds{B}_{m, l}} \left\vert\frac{1}{n} \sum\limits_{p = 1}^{n} \nu_{t}(Y_{p}^{n}) - \E_{\theta^{\circ}}^{n}\left[\frac{1}{n} \sum\limits_{p = 1}^{n} \nu_{t}(Y_{p}^{n}) \right]\right\vert^{2} \right]\\
%& && = && \E_{\theta^{\circ}}^{n}\left[\sup\limits_{t \in \mathds{B}_{m, l}} \left\vert \left\langle t \vert \overline{\theta} - \theta^{\circ} \right\rangle \right\vert^{2} \right]\\
%& && \leq && \E_{\theta^{\circ}}^{n}\left[\sup\limits_{t \in \mathds{B}_{m, l}} \left\Vert t \right\Vert^{2} \left\Vert \Pi_{m,l}\left(\overline{\theta} - \theta^{\circ} \right) \right\Vert^{2} \right]\\
%& && \leq && \E_{\theta^{\circ}}^{n}\left[ \left\Vert \Pi_{m,l}\left(\overline{\theta} - \theta^{\circ}\right) \right\Vert^{2} \right]\\
%& && \leq && \frac{1}{n} \sum\limits_{m \leq \vert j \vert \leq l} \Lambda_{j}
%\end{alignat*}
%we hence define,
%\begin{alignat}{1}
%H^{2} := \psi_{n} \frac{1}{n} \delta^{\star}_{k,l}  \geq \frac{1}{n} \delta^{\star}_{k,l} \geq E_{\theta^{\circ}}^{n}\left[\sup\limits_{t \in \mathds{B}_{k, l}} \vert \overline{\nu}_{t} \vert^{2} \right].\label{EQD.8}
%\end{alignat}
%
%Finally, for $v$, let us consider $t$ in $\mathds{B}_{m,l}$
%\begin{alignat*}{3}
%& \E_{\theta^{\circ}}^{n}\left[ \left\vert \nu_{t}(Y_{1}^{n}) \right\vert^{2} \right] && = && \E_{\theta^{\circ}}^{n}\left[ \left\vert \sum\limits_{k \leq \vert j \vert \leq l} \left(\frac{t_{j}}{\overline{\lambda_{j}}}\right) e_{j}(Y_{1}^{n}) \right\vert^{2} \right]\\
%& && = && \E_{\theta^{\circ}}^{n}\left[ \left( \sum\limits_{k \leq \vert j \vert \leq l} \left(\frac{t_{j}}{\overline{\lambda_{j}}}\right) e_{j}(Y_{1}^{n}) \right)\left( \sum\limits_{k \leq \vert j \vert \leq l} \left(\frac{\overline{t_{j}}}{\lambda_{j}}\right) e_{j}(-Y_{1}^{n}) \right) \right]\\
%& && = && \sum\limits_{k \leq \vert j_{1} \vert, \vert j_{2} \vert \leq l} \frac{t_{j_{1}}\overline{t_{j_{2}}}}{\overline{\lambda_{j_{1}}}\lambda_{j_{2}}} \E_{\theta^{\circ}}^{n}\left[ e_{j_{1} - j_{2}}(Y_{1}^{n}) \right]\\
%& && = && \sum\limits_{k \leq \vert j_{1} \vert, \vert j_{2} \vert \leq l} t_{j_{1}}\overline{t_{j_{2}}} \frac{1}{\overline{\lambda_{j_{1}}}\lambda_{j_{2}}} \theta^{\circ}_{j_{1} - j_{2}} \lambda_{j_{1} - j_{2}}.
%\end{alignat*}
%
%
%\textcolor{red}{I spot some difference, which cancel out, (square root operator) with what you did here, am I mistaken?}
%
%Define, then, the Hermitian semi definite linear operator $M$ such that for any $j_{1}$ and $j_{2}$ in $\mathds{Z}$, we have $M_{j_{1}, j_{2}} = \frac{1}{\overline{\lambda_{j_{1}}}\lambda_{j_{2}}} \theta^{\circ}_{j_{1} - j_{2}} \lambda_{j_{1} - j_{2}}$.
%Due to semi-definitiveness, note that this operator admits a square root, $M^{1/2}$ which is self adjoint itself.
%
%Moreover, we define the spectral norm $\Vert \cdot \Vert_{s}$ on the space of linear operators from $\mathds{C}^{\mathds{Z}}$ onto itself such that for any such operator $A$, we have $\Vert A \Vert_{s} = \sup\limits_{x \in \mathcal{L}^{2} : \Vert x \Vert_{l^{2}} \leq 1 } \Vert A x \Vert_{l^{2}}$.
%Note that this norm verifies, for any Hermitian operator $A$ the following identity $\Vert A^{1/2} \Vert_{s} = \sqrt{\Vert A \Vert_{s}}$.
%
%With this notation, we have
%\begin{alignat*}{3}
%& \sup\limits_{t \in \mathds{B}_{m, l}} \frac{1}{n} \sum\limits_{p = 1}^{n} \V_{\theta^{\circ}}^{n}\left[ \nu_{t}(Y_{1}^{n}) \right] && = && \sup\limits_{t \in \mathds{B}_{m, l}} \left(\E_{\theta^{\circ}}^{n}\left[ \left\vert \nu_{t}(Y_{1}^{n}) \right\vert^{2} \right] - \vert \E_{\theta^{\circ}}^{n}\left[ \nu_{t}(Y_{1}^{n}) \right]\vert^{2}\right)\\
%& && \leq && \sup\limits_{t \in \mathds{B}_{m, l}} \E_{\theta^{\circ}}^{n}\left[ \left\vert \nu_{t}(Y_{1}^{n}) \right\vert^{2} \right]\\
%& && \leq && \sup\limits_{t \in \mathds{B}_{m, l}} \sum\limits_{m \leq \vert j_{1} \vert, \vert j_{2} \vert \leq l} t_{j_{1}}\overline{t_{j_{2}}} \frac{1}{\overline{\lambda_{j_{1}}}\lambda_{j_{2}}} \theta^{\circ}_{j_{1} - j_{2}} \lambda_{j_{1} - j_{2}}\\
%& && \leq && \sup\limits_{t \in \mathds{B}_{m, l}} \left\langle \Pi_{m, l} M \Pi_{m, l} t \vert t \right\rangle_{L_{2}}\\
%& && \leq && \sup\limits_{t \in \mathds{B}_{m, l}} \left\langle \left(\Pi_{m, l} M \Pi_{m, l}\right)^{1/2} t \vert \left(\Pi_{m, l} M \Pi_{m, l}\right)^{1/2} t \right\rangle_{L_{2}}\\
%& && \leq && \sup\limits_{t \in \mathds{B}_{m, l}} \left\Vert \left(\Pi_{m, l} M \Pi_{m, l}\right)^{1/2} t \right\Vert_{L_{2}}^{2}\\
%& && \leq && \left\Vert \left(\Pi_{m, l} M \Pi_{m, l}\right)^{1/2} \right\Vert_{s}^{2}\\
%& && \leq && \left\Vert \Pi_{m, l} M \Pi_{m, l} \right\Vert_{s}.
%\end{alignat*}
%
%Define then, for any element $t$ of $\mathcal{L}^{2}$, the diagonal operator $D_{t}$ such that for any $x$ in $\mathcal{L}^{2}$ we have, $\left((D_{t} x)_{j}\right)_{j \in \mathds{Z}} = (t_{j} x_{j})_{j \in \mathds{Z}}$, clearly, $D_{t}^{-1} = D_{t^{-1}}$.
%Notice also $\left\Vert \Pi_{m, l} D_{\lambda}^{-1} \Pi_{m, l}\right\Vert_{s}^{2} = \max\limits_{m \leq \vert j \vert \leq l}\Lambda_{j} \leq \Delta^{\star}_{m, l}$ by definition of $\Delta^{\star}_{m, l}$.
%
%In addition, define, for any element $t$ of $\mathcal{L}^{2}$, the convolution operator $C_{t}$ such that for any $x$ in $\mathds{C}^{\mathds{Z}}$ we have, $\left((C_{t} x)_{k}\right)_{k \in \mathds{Z}} = \left(\sum\limits_{j \in \mathds{Z}} t_{k - j} x_{k}\right)_{k \in \mathds{Z}}$.
%This operator verifies, using Young's inequality
%\[\left\Vert C_{t} \right\Vert_{s} = \sup\limits_{x \in \mathcal{L}^{2}, \Vert x \Vert_{l^{2}} \leq 1} \left\Vert C_{t} x \right\Vert_{l^{2}} \leq \sup\limits_{x \in \mathcal{L}^{2}, \Vert x \Vert_{l^{2}} \leq 1} \left\Vert t \right\Vert_{l^{1}} \cdot \left\Vert x \right\Vert_{l^{2}} \leq \left\Vert t \right\Vert_{l^{1}}.\]
%
%This gives us the following form for $\Pi_{m, l} M \Pi_{m, l} = \Pi_{m, l} D_{\lambda}^{-1} \Pi_{m, l} C_{\theta^{\circ} \lambda} \Pi_{m, l} D_{\overline{\lambda}}^{-1} \Pi_{m, l}$.
%
%From this, we can deduce, using \nref{A.3.2}
%\begin{alignat*}{3}
%& \left\Vert \Pi_{m, l} M \Pi_{m, l} \right\Vert_{s} && = && \left\Vert \Pi_{m, l} D_{\lambda}^{-1} \Pi_{m, l} C_{\theta^{\circ} \lambda} \Pi_{m, l} D_{\overline{\lambda}}^{-1} \Pi_{m, l} \right\Vert_{s}\\
%& && = && \left\Vert \Pi_{m, l} D_{\lambda}^{-1} \Pi_{m, l}\right\Vert_{s} \cdot \left\Vert C_{\theta^{\circ} \lambda} \right\Vert_{s} \cdot \left\Vert \Pi_{m, l} D_{\overline{\lambda}}^{-1} \Pi_{m, l} \right\Vert_{s}\\
%& && = && \left\Vert \Pi_{m, l} D_{\lambda}^{-1} \Pi_{m, l} \right\Vert_{s}^{2} \cdot \left\Vert C_{\theta^{\circ} \lambda}\right\Vert_{s}\\
%& && \leq && \Delta_{m, l}^{\star} \cdot \left\Vert \theta^{\circ} \lambda \right\Vert_{l^{1}}\\
%& && \leq && \Delta_{m, l}^{\star} \cdot \left\Vert \theta^{\circ} \right\Vert_{l^{2}} \cdot \left\Vert \lambda \right\Vert_{l^{2}}
%\end{alignat*}
%and we define
%\begin{alignat}{1}
%& v := \Delta_{m, l}^{\star} \cdot \left\Vert \theta^{\circ} \right\Vert_{l^{2}} \cdot \left\Vert \lambda \right\Vert_{l^{2}}.\label{EQD.9}
%\end{alignat}
%
%\medskip
%
%The three quantities $h$, $H$ and $v$, respectively defined in \nref{EQD.7}, \nref{EQD.8}, and \nref{EQC.9} verify the hypotheses of \nref{LM_TALAGRAND} which gives us the conclusion.
%
%\qedsymbol
%\end{pro}
%
%
%\begin{pro}{\textsc{Proof of \nref{PR_FREQ_CIRCDECONV_KNOWN_IID_ORACLE_NP_CONTRACTTHRESHOLD}} \\}\label{PROC.3.2}
%
%Before considering the two inequalities separately, let us do some observations.
%
%Throughout the proof, $l$ and $m$ will be two positive integers.
%
%Define the function $\Gamma$ from $\mathcal{L}^{2}$ onto $\mathds{C}$ such that for any $x$ in $\mathcal{L}^{2}$, we have $\Gamma(x) = \Vert x \Vert_{l^{2}}^{2} - 2 \left\langle x \vert \overline{\theta} \right\rangle_{l^{2}}$ and notice that we have $\Gamma(\Pi_{l} x) = \Vert \Pi_{l} x \Vert_{l^{2}}^{2} - 2 \left\langle \Pi_{l} x \vert \overline{\theta} \right\rangle_{l^{2}} + \Vert \overline{\theta}^{l} \Vert_{l^{2}}^{2} - \Vert \overline{\theta}^{l} \Vert_{l^{2}}^{2} = \Vert \Pi_{l} x - \overline{\theta}^{l} \Vert_{l^{2}}^{2} - \Vert \overline{\theta}^{l} \Vert_{l^{2}}^{2}$ which shows $\Gamma(\overline{\theta}^{l}) = -\frac{2}{n} \Upsilon(Y^{n}, l)$.
%
%In addition, for any $x$, $y$, and $z$ in $\mathcal{L}^{2}$, we have
%\begin{alignat}{3}
%& \Gamma(x) - \Gamma(y) && = && \Vert x \Vert_{l^{2}}^{2} - \Vert y \Vert_{l^{2}}^{2} - 2 \left\langle x - y \vert \overline{\theta} \right\rangle_{l^{2}}\notag\\
%& && = && \Vert x \Vert_{l^{2}}^{2} - 2 \left\langle x \vert z \right\rangle_{l^{2}} + \Vert z \Vert_{l^{2}}^{2} - \Vert y \Vert_{l^{2}}^{2} + 2 \left\langle y \vert z \right\rangle_{l^{2}} - \Vert z \Vert_{l^{2}}^{2} - 2 \left\langle x - y \vert \overline{\theta} - z \right\rangle_{l^{2}}\notag\\
%& && = && \Vert x - z \Vert_{l^{2}}^{2} - \Vert y - z \Vert_{l^{2}}^{2} - 2 \left\langle x - y \vert \overline{\theta} - z \right\rangle_{l^{2}}\label{EQD.10}.
%\end{alignat}
%
%Finally, notice that, for any event $\Omega$, we have the following inequality
%
%\begin{alignat}{3}
%& \P_{M \vert Y^{n}}^{n, (\eta)}(m) && = && \frac{\exp\left[\eta\left(- \pen(m) + \Upsilon(m, Y^{n})\right)\right]}{\sum\limits_{k = 1}^{n} \exp\left[\eta\left(- \pen(k) + \Upsilon(k, Y^{n})\right)\right]}\notag\\
%& && = && \frac{1}{\sum\limits_{k = 1}^{n} \exp\left[\eta\left(- (\pen(k)-\pen(m)) - \frac{n}{2} \left( \Gamma\left(\overline{\theta}^{k}\right) - \Gamma\left(\overline{\theta}^{m}\right) \right) \right)\right]}\notag\\
%& && = && \frac{1}{\sum\limits_{k = 1}^{n} \exp\left[\eta\left(- (\pen(k)-\pen(m)) - \frac{n}{2} \left( \Gamma\left(\overline{\theta}^{k}\right) - \Gamma\left(\overline{\theta}^{m}\right) \right) \right)\right]} \mathds{1}_{\Omega} +\notag\\
%& && && \frac{1}{\sum\limits_{k = 1}^{n} \exp\left[\eta\left(- (\pen(k)-\pen(m)) - \frac{n}{2} \left( \Gamma\left(\overline{\theta}^{k}\right) - \Gamma\left(\overline{\theta}^{m}\right) \right) \right)\right]} \mathds{1}_{\Omega^{c}}\notag\\
%& && \leq && \exp\left[\eta\left((\pen(l)-\pen(m)) + \frac{n}{2} \left( \Gamma\left(\overline{\theta}^{l}\right) - \Gamma\left(\overline{\theta}^{m}\right) \right) \right)\right] \mathds{1}_{\Omega} + \mathds{1}_{\Omega^{c}} \label{EQD.11}
%\end{alignat}
%
%\bigskip
%
%Assume for now, in addition, that $m < l$.
%
%Considering \nref{EQD.10} gives, for any $x$ and $z$ in $\mathcal{L}^{2}$, using 
%\begin{alignat*}{3}
%& 0 && \leq && \left(\frac{1}{2} \left\Vert \Pi_{m, l} x \right\Vert_{l^{2}} - 2  \left\langle \frac{\Pi_{m, l} x}{\left\Vert \Pi_{m, l} x \right\Vert_{l^{2}}} \vert \Pi_{m, l}(\overline{\theta} - z) \right\rangle_{l^{2}}\right)^{2} \\
%& && = && \frac{1}{4} \left\Vert \Pi_{m, l} x \right\Vert_{l^{2}}^{2} - 2 \left\Vert \Pi_{m, l} x \right\Vert_{l^{2}} \left\langle \frac{\Pi_{m, l} x}{\left\Vert \Pi_{m, l} x \right\Vert_{l^{2}}} \vert \Pi_{m, l}(\overline{\theta} - z) \right\rangle_{l^{2}} + 4 \left\vert \left\langle \frac{\Pi_{m, l} x}{\left\Vert \Pi_{m, l} x \right\Vert_{l^{2}}} \vert \Pi_{m, l}(\overline{\theta} - z) \right\rangle_{l^{2}}\right\vert^{2}
%\end{alignat*}
%
%as well as the triangular inequality and the Riesz representation theorem and its implication for the operator norm:
%
%\begin{alignat}{3}
%& \Gamma(\Pi_{l} x) - \Gamma(\Pi_{m} x) &&=&& \Vert \Pi_{l} x - z \Vert_{l^{2}}^{2} - \Vert \Pi_{m} x - z \Vert_{l^{2}}^{2} - 2 \left\langle \Pi_{l} x - \Pi_{m} x \vert \overline{\theta} - z \right\rangle_{l^{2}}\notag\\
%& && = && \Vert \Pi_{l} (x - z) \Vert_{l^{2}}^{2} + \Vert \Pi_{l}^{\perp} z \Vert_{l^{2}}^{2} - \Vert \Pi_{m} (x - z) \Vert_{l^{2}}^{2} - \Vert \Pi_{m}^{\perp} z \Vert_{l^{2}}^{2} -\notag\\
%& && && 2 \left\langle \Pi_{m, l} x \vert \overline{\theta} - z \right\rangle_{l^{2}}\notag\\
%& && = && \Vert \Pi_{m, l} (x - z) \Vert_{l^{2}}^{2} - \Vert \Pi_{m, l} z \Vert_{l^{2}}^{2} -\notag\\
%& && && 2 \left\langle \Pi_{m, l} x \vert \Pi_{m, l}(\overline{\theta} - z) \right\rangle_{l^{2}}\notag\\
%& && = && \Vert \Pi_{m, l} (x - z) \Vert_{l^{2}}^{2} - \Vert \Pi_{m, l} z \Vert_{l^{2}}^{2} -\notag\\
%& && && 2 \left\Vert \Pi_{m, l} x \right\Vert_{l^{2}} \left\langle \frac{\Pi_{m, l} x}{\left\Vert \Pi_{m, l} x \right\Vert_{l^{2}}} \vert \Pi_{m, l}(\overline{\theta} - z) \right\rangle_{l^{2}}\label{EQD.12}\\
%& && \leq && \Vert \Pi_{m, l} (x - z) \Vert_{l^{2}}^{2} - \Vert \Pi_{m, l} z \Vert_{l^{2}}^{2} +\notag\\
%& && && \frac{1}{4} \left\Vert \Pi_{m, l} x \right\Vert_{l^{2}}^{2} + 4 \left\vert \left\langle \frac{\Pi_{m, l} x}{\left\Vert \Pi_{m, l} x \right\Vert_{l^{2}}} \vert \Pi_{m, l}(\overline{\theta} - z) \right\rangle_{l^{2}}\right\vert^{2}\notag\\
%& && \leq && \left\Vert \Pi_{m, l} (x - z) \right\Vert_{l^{2}}^{2} - \left\Vert \Pi_{m, l} z \right\Vert_{l^{2}}^{2} +\notag\\
%& && && \frac{1}{4} \left\Vert \Pi_{m, l} (x - z) \right\Vert_{l^{2}}^{2} + \frac{1}{4} \left\Vert \Pi_{m, l} z \right\Vert_{l^{2}}^{2} + 4 \sup\limits_{x \in \mathds{B}_{k, l}} \left\vert \left\langle \Pi_{m, l} x \vert \Pi_{m, l}(\overline{\theta} - z) \right\rangle_{l^{2}}\right\vert^{2}\notag\\
%& && \leq && \frac{5}{4}\left\Vert \Pi_{m, l} (x - z) \right\Vert_{l^{2}}^{2} - \frac{3}{4} \left\Vert \Pi_{m, l} z \right\Vert_{l^{2}}^{2} + 4 \left\Vert \Pi_{m, l}(\overline{\theta} - z) \right\Vert_{l^{2}}^{2}\label{EQD.13}
%\end{alignat}
%
%We now use this inequality with $x = \overline{\theta}$, and $z = \theta^{\circ}$ and, yet again, the Riesz theorem:
%
%\begin{alignat}{3}
%& \Gamma(\overline{\theta}^{l}) - \Gamma(\overline{\theta}^{m}) && \leq && \frac{5}{4}\left\Vert \Pi_{m, l} (\overline{\theta} - \theta^{\circ}) \right\Vert_{l^{2}}^{2} - \frac{3}{4} \left\Vert \Pi_{m, l} \theta^{\circ} \right\Vert_{l^{2}}^{2} + 4 \left\Vert \Pi_{m, l}(\overline{\theta} - \theta^{\circ}) \right\Vert_{l^{2}}^{2}\notag\\
%& && \leq && \frac{21}{4}\left\Vert \Pi_{m, l} (\overline{\theta} - \theta^{\circ}) \right\Vert_{l^{2}}^{2} - \frac{3}{4} \left\Vert \Pi_{m, l} \theta^{\circ} \right\Vert_{l^{2}}^{2}\notag\\
%& && \leq && \frac{21}{4} \sup\limits_{x \in \mathds{B}_{m, l}} \left\vert \left\langle x \vert \Pi_{m, l} (\overline{\theta} - \theta^{\circ}) \right\rangle \right\vert_{l^{2}}^{2} - \frac{3}{4} \left\Vert \Pi_{m, l} \theta^{\circ} \right\Vert_{l^{2}}^{2}.\label{EQD.14}
%\end{alignat}
%
%We use \nref{EQD.11}, \nref{EQD.14} and \nref{PR_FREQ_CIRCDECONV_KNOWN_IID_ORACLE_NP_TALAGRAND} in order to obtain, with the event $\mathcal{A}_{m, l} := \left\{ \sup\limits_{x \in \mathds{B}_{m, l}} \left\vert \left\langle x \vert \Pi_{m, l} (\overline{\theta} - \theta^{\circ}) \right\rangle \right\vert_{l^{2}}^{2} < 3 \frac{\psi_{n} \delta_{m, l}^{\star}}{n} \right\}$
%
%\begin{alignat}{4}
%& \P_{M \vert Y^{n}}^{n, (\eta)}(m) && \leq && \exp &&\left[\eta\left((\pen(l)-\pen(m)) +\right.\right.\notag\\
%& && && && \left.\left. \frac{n}{2} \left( \frac{21}{4} \sup\limits_{x \in \mathds{B}_{m, l}} \left\vert \left\langle x \vert \Pi_{m, l} (\overline{\theta} - \theta^{\circ}) \right\rangle \right\vert_{l^{2}}^{2} - \frac{3}{4} \left\Vert \Pi_{m, l} \theta^{\circ} \right\Vert_{l^{2}}^{2} \right) \right)\right] \mathds{1}_{\mathcal{A}_{m, l}} +\notag\\
%& && && && \mathds{1}_{\mathcal{A}_{m, l}}^{c}\notag\\
%& \E_{\theta^{\circ}}^{n}\left[\P_{M \vert Y^{n}}^{n, (\eta)}(m)\right]&& \leq && \E_{\theta^{\circ}}^{n} && \left[\exp\left[\eta\left((\pen(l)-\pen(m)) +\right.\right.\right.\notag\\
%& && && && \left.\left.\left. \frac{n}{2} \left( \frac{21}{4} \sup\limits_{x \in \mathds{B}_{m, l}} \left\vert \left\langle x \vert \Pi_{m, l} (\overline{\theta} - \theta^{\circ}) \right\rangle \right\vert_{l^{2}}^{2} - \frac{3}{4} \left\Vert \Pi_{m, l} \theta^{\circ} \right\Vert_{l^{2}}^{2} \right) \right)\right]\mathds{1}_{\mathcal{A}_{m, l}} \right] + \notag\\
%& && && &&\P_{\theta^{\circ}}^{n}\left(\mathcal{A}_{m, l}^{c}\right)\notag\\
%& && \leq && \exp && \left[\eta\left((\pen(l)-\pen(m)) +\right.\right.\notag\\
%& && && && \left.\left. \frac{n}{2} \left( \frac{63}{4} \frac{\psi_{n} \delta_{m, l}^{\star}}{n} - \frac{3}{4} \left\Vert \Pi_{m, l} \theta^{\circ} \right\Vert_{l^{2}}^{2} \right) \right)\right] + \notag\\
%& && && && 3 \exp\left[ -K \left(\frac{\psi_{n} \delta^{\star}_{m, l}}{\Delta^{\star}_{m, l} \Vert \lambda \Vert_{l^{2}} \Vert \theta^{\circ} \Vert_{l^{2}}} \wedge \sqrt{n \psi_{n}} \right)\right]\notag\\
%& && \leq && \exp&&\left[\eta\left((\pen(l)-\pen(m)) +\right.\right. \notag\\
%& && && &&\left.\left. \frac{n}{2} \left( \frac{63}{4} \frac{\psi_{n} \delta_{m, l}^{\star}}{n} - \frac{3}{4} \left( \mathfrak{b}_{m}^{2}\left(\theta^{\circ}\right) - \mathfrak{b}_{l}^{2}\left(\theta^{\circ}\right) \right) \right) \right)\right] +\notag\\
%& && && && 3 \exp\left[ -K \left(\frac{\psi_{n} \delta^{\star}_{m, l}}{\Delta^{\star}_{m, l} \Vert \lambda \Vert_{l^{2}} \Vert \theta^{\circ} \Vert_{l^{2}}} \wedge \sqrt{n \psi_{n}} \right)\right].\label{EQD.15}
%\end{alignat}
%
%In particular with $l = m^{\dagger}_{n}$, set $\Delta^{\star}_{m, m^{\dagger}_{n}} = \Lambda_{(m^{\dagger}_{n})}$ and $\delta^{\star}_{m, m^{\dagger}_{n}} = 2 m^{\dagger}_{n}\Lambda_{(m^{\dagger}_{n})}$, for any $k$ we set $\pen^{\dagger}(k) = \kappa k \Lambda_{(k)} \psi_{n}$, $\Phi^{\dagger}_{n} = \left[\mathfrak{b}_{m^{\dagger}_{n}}^{2}\mathfrak{b}_{0}^{-2} \vee 2 \frac{m^{\dagger}_{n} \Lambda_{(m^{\dagger}_{n})}}{n} \psi_{n}\right]$, and finishing with $\kappa = \frac{43}{4}$,  such that they verify the hypotheses of \nref{PR_FREQ_CIRCDECONV_KNOWN_IID_ORACLE_NP_TALAGRAND} and we obtain
%
%\begin{alignat}{4}
%& \E_{\theta^{\circ}}^{n}\left[\P_{M \vert Y^{n}}^{n, (\eta)}(m)\right] && \leq && \exp&&\left[\frac{\eta n}{2}\left(43 \mathfrak{b}_{0}^{2}(\theta^{\circ}) \Phi^{\dagger}_{n} - \frac{3}{4} \mathfrak{b}_{m}^{2}\left(\theta^{\circ}\right) \right)\right] +\notag\\
%& && && && 3 \exp\left[ -K \left(\frac{\psi_{n} 2 m^{\dagger}_{n}}{\Vert \lambda \Vert_{l^{2}} \Vert \theta^{\circ} \Vert_{l^{2}}} \wedge \sqrt{n \psi_{n}} \right)\right].\label{EQD.16}
%\end{alignat}
%
%\medskip
%
%Now, we assume $m > l$.
%
%We have, starting with \nref{EQD.12} and using the triangular inequality and Riesz theorem
%\begin{alignat*}{3}
%& \Gamma(\Pi_{l} x) - \Gamma(\Pi_{m} x) && = && -\left(\Gamma(\Pi_{m} x) - \Gamma(\Pi_{l} x)\right)\\
%& && = && -\Vert \Pi_{l, m} (x - z) \Vert_{l^{2}}^{2} + \Vert \Pi_{l, m} z \Vert_{l^{2}}^{2} + 2 \left\Vert \Pi_{l, m} x \right\Vert_{l^{2}} \left\langle \frac{\Pi_{l, m} x}{\left\Vert \Pi_{l, m} x \right\Vert_{l^{2}}} \vert \Pi_{l, m}(\overline{\theta} - z) \right\rangle_{l^{2}}\\
%& && \leq && -\Vert \Pi_{l, m} (x - z) \Vert_{l^{2}}^{2} + \Vert \Pi_{l, m} z \Vert_{l^{2}}^{2} + \frac{1}{4} \left\Vert \Pi_{l, m} x \right\Vert_{l^{2}}^{2} +\\
%& && && 4 \left\vert\left\langle \frac{\Pi_{l, m} x}{\left\Vert \Pi_{l, m} x \right\Vert_{l^{2}}} \vert \Pi_{l, m}(\overline{\theta} - z) \right\rangle_{l^{2}}\right\vert^{2}\\
%& && \leq && -\Vert \Pi_{l, m} (x - z) \Vert_{l^{2}}^{2} + \Vert \Pi_{l, m} z \Vert_{l^{2}}^{2} + \frac{1}{4} \left\Vert \Pi_{l, m} (x-z) \right\Vert_{l^{2}}^{2} + \frac{1}{4} \left\Vert \Pi_{l, m} z \right\Vert_{l^{2}}^{2} +\\
%& && && 4 \sup\limits_{x \in \mathds{B}_{l, m}}\left\vert\left\langle x \vert \Pi_{l, m}(\overline{\theta} - z) \right\rangle_{l^{2}}\right\vert^{2}\\
%& && \leq && -\frac{3}{4}\Vert \Pi_{l, m} (x - z) \Vert_{l^{2}}^{2} + \frac{5}{4} \Vert \Pi_{l, m} z \Vert_{l^{2}}^{2} + 4 \left\Vert \Pi_{l, m}(\overline{\theta} - z) \right\Vert_{l^{2}}^{2}.
%\end{alignat*}
%
%We now use this inequality with $x = \overline{\theta}$, and $z = \theta^{\circ}$ and, yet again, the Riesz theorem:
%
%\begin{alignat}{3}
%& \Gamma(\overline{\theta}^{l}) - \Gamma(\overline{\theta}^{m}) && \leq && -\frac{3}{4}\Vert \Pi_{l, m} (\overline{\theta} - \theta^{\circ}) \Vert_{l^{2}}^{2} + \frac{5}{4} \Vert \Pi_{l, m} \theta^{\circ} \Vert_{l^{2}}^{2} + 4 \left\Vert \Pi_{l, m}(\overline{\theta} - \theta^{\circ}) \right\Vert_{l^{2}}^{2}\notag\\
%& && \leq && \frac{13}{4}\Vert \Pi_{l, m} (\overline{\theta} - \theta^{\circ}) \Vert_{l^{2}}^{2} + \frac{5}{4} \Vert \Pi_{l, m} \theta^{\circ} \Vert_{l^{2}}^{2}\notag\\
%& && \leq && \frac{13}{4}\sup\limits_{x \in \mathds{B}_{l, m}}\left\vert \left\langle x \vert  (\overline{\theta} - \theta^{\circ}) \right\rangle_{l^{2}} \right\vert^{2} + \frac{5}{4} \Vert \Pi_{l, m} \theta^{\circ} \Vert_{l^{2}}^{2}.\label{EQD.17}
%\end{alignat}
%
%We use \nref{EQD.11}, \nref{EQD.17} and \nref{PR_FREQ_CIRCDECONV_KNOWN_IID_ORACLE_NP_TALAGRAND} in order to obtain, with the event $\mathcal{A}_{m, l} := \left\{ \sup\limits_{x \in \mathds{B}_{l, m}} \left\vert \left\langle x \vert \Pi_{l, m} (\overline{\theta} - \theta^{\circ}) \right\rangle \right\vert_{l^{2}}^{2} < 3 \frac{\psi_{n} \delta_{l, m}^{\star}}{n} \right\}$
%
%\begin{alignat}{3}
%& \P_{M \vert Y^{n}}^{n, (\eta)}(m) && \leq && \exp\left[\eta\left((\pen(l)-\pen(m)) +\right.\right.\notag\\
%& && && \left.\left. \frac{n}{2} \left(\frac{13}{4}\sup\limits_{x \in \mathds{B}_{l, m}}\left\vert \left\langle x \vert  (\overline{\theta} - \theta^{\circ}) \right\rangle_{l^{2}} \right\vert^{2} + \frac{5}{4} \Vert \Pi_{l, m} \theta^{\circ} \Vert_{l^{2}}^{2} \right) \right)\right] \mathds{1}_{\mathcal{A}_{m, l}} +\notag\\
%& && && \mathds{1}_{\mathcal{A}_{m, l}}^{c}\notag\\
%& \E_{\theta^{\circ}}^{n}\left[\P_{M \vert Y^{n}}^{n, (\eta)}(m)\right] && \leq && \exp\left[\eta\left((\pen(l)-\pen(m)) +\right.\right.\notag\\
%& && && \left.\left. \frac{n}{2} \left( \frac{39}{4} \frac{\psi_{n} \delta_{l, m}^{\star}}{n} + \frac{5}{4} (\mathfrak{b}_{l}^{2}(\theta^{\circ}) - \mathfrak{b}_{m}^{2}(\theta^{\circ})) \right) \right)\right] +\notag\\
%& && && 3 \exp\left[ -K \left(\frac{\psi_{n} \delta^{\star}_{k, l}}{\Delta^{\star}_{k, l} \Vert \lambda \Vert_{l^{2}} \Vert \theta^{\circ} \Vert_{l^{2}}} \wedge \sqrt{n \psi_{n}} \right)\right].\label{EQD.18}
%\end{alignat}
%
%In particular with $l = m^{\dagger}_{n}$, set $\Delta^{\star}_{m^{\dagger}_{n}, m} = \Lambda_{(m)}$ and $\delta^{\star}_{m^{\dagger}_{n}, m} = 2 m \Lambda_{(m)}$, for any $k$ we set $\pen^{\dagger}(k) = \kappa k \Lambda_{(k)} \psi_{n}$, $\Phi^{\dagger}_{n} = \left[\mathfrak{b}_{m^{\dagger}_{n}}^{2}\mathfrak{b}_{0}^{-2} \vee 2 \frac{m^{\dagger}_{n} \Lambda_{(m^{\dagger}_{n})}}{n} \psi_{n}\right]$, and finishing with $\kappa = \frac{43}{4}$,  such that they verify the hypotheses of \nref{PR_FREQ_CIRCDECONV_KNOWN_IID_ORACLE_NP_TALAGRAND} and we obtain
%
%\begin{alignat}{3}
%& \E_{\theta^{\circ}}^{n}\left[\P_{M \vert Y^{n}}^{n, (\eta)}(m)\right] && \leq &&  \exp\left[\frac{n \eta}{2} \left(\left(\kappa + \frac{5}{4} \right) \mathfrak{b}_{0}^{2}(\theta^{\circ}) \Phi^{\dagger}_{n} + \left(\frac{39}{4} - \kappa\right) 2 \frac{m \Lambda_{(m)}}{n} \psi_{n} - \frac{5}{4}\mathfrak{b}_{m}^{2}(\theta^{\circ}) \right)\right] +\notag\\
%& && && 3 \exp\left[ -K \left(\frac{\psi_{n} 2 m}{ \Vert \lambda \Vert_{l^{2}} \Vert \theta^{\circ} \Vert_{l^{2}}} \wedge \sqrt{n \psi_{n}} \right)\right]\notag\\
%& && \leq && \exp\left[\frac{n \eta}{2} \left(12 \mathfrak{b}_{0}^{2}(\theta^{\circ}) \Phi^{\dagger}_{n} - 2 \frac{m \Lambda_{(m)}}{n} \psi_{n} \right)\right] +\notag\\
%& && && 3 \exp\left[ -K \left(\frac{\psi_{n} 2 m}{ \Vert \lambda \Vert_{l^{2}} \Vert \theta^{\circ} \Vert_{l^{2}}} \wedge \sqrt{n \psi_{n}} \right)\right]\label{EQD.19}.
%\end{alignat}
%
%\bigskip
%
%We will now conclude using the definitions of $G^{\dagger -}_{n}$ and $G^{\dagger +}_{n}$.
%
%Consider \nref{EQD.3}.
%As for all $m < G^{\dagger -}_{n}$, we have $\mathfrak{b}_{m}^{2}(\theta^{\circ}) > \frac{176}{3} \Phi^{\dagger}_{n} \mathfrak{b}_{0}^{2}(\theta^{\circ})$ and using \nref{EQD.16}
%
%\begin{alignat*}{3}
%& \E_{\theta^{\circ}}^{n}\left[\P_{M \vert Y^{n}}^{n, (\eta)}\left(\llbracket 0, G^{\dagger -}_{n} - 1 \rrbracket\right)\right] && \leq && G^{\dagger -}_{n} \exp\left[- \frac{\mathfrak{b}_{0}^{2}(\theta^{\circ}) n \Phi^{\dagger}_{n}}{2} \right] + 3 G^{\dagger -}_{n} \exp\left[ -K \left(\frac{\psi_{n} 2 m^{\dagger}_{n}}{\Vert \lambda \Vert_{l^{2}} \Vert \theta^{\circ} \Vert_{l^{2}}} \wedge \sqrt{n \psi_{n}} \right)\right]\\
%& && \leq && 4 m^{\dagger}_{n} \exp\left[ -K \left(\frac{\psi_{n} 2 m^{\dagger}_{n}}{\Vert \lambda \Vert_{l^{2}} \Vert \theta^{\circ} \Vert_{l^{2}}} \wedge \sqrt{n \psi_{n}} \right)\right].
%\end{alignat*}
%
%Now for \nref{EQD.4}.
%As for all $m > G^{\dagger +}_{n}$, we have $\psi_{n} m \Lambda_{(m)} > \frac{25}{3} n \Phi^{\dagger}_{n} \mathfrak{b}_{0}^{2}$, using \nref{EQD.19}
%
%\begin{alignat*}{3}
%& \E_{\theta^{\circ}}^{n}&& &&\left[\P_{M \vert Y^{n}}^{n, (\eta)}\left(\llbracket G^{\dagger +}_{n} + 1, n \rrbracket\right)\right]\\
%& && \leq && \sum\limits_{G^{\dagger +} < m \leq n}\exp\left[\frac{n \eta}{2} \left(12 \mathfrak{b}_{0}^{2}(\theta^{\circ}) \Phi^{\dagger}_{n} - 2 \frac{m \Lambda_{(m)}}{n} \psi_{n} \right)\right] +\notag\\
%& && && 3 \sum\limits_{G^{\dagger +} < m \leq n} \exp\left[ -K \left(\frac{\psi_{n} 2 m}{\Vert \lambda \Vert_{l^{2}} \Vert \theta^{\circ} \Vert_{l^{2}}} \wedge \sqrt{n \psi_{n}} \right)\right]\\
%& && \leq && \sum\limits_{G^{\dagger +} < m \leq n} \exp\left[\eta\left(-\frac{\mathfrak{b}_{0}^{2}(\theta^{\circ}) n \Phi^{\dagger}_{n}}{2} - \frac{\psi_{n} m \Lambda_{(m)}}{2}\right)\right] +\notag\\
%& && && \exp\left[ -K \left(\frac{\psi_{n} 2 G^{+}_{n}}{\Vert \lambda \Vert_{l^{2}} \Vert \theta^{\circ} \Vert_{l^{2}}} \wedge \sqrt{n \psi_{n}} \right)\right]\cdot \sum\limits_{G^{\dagger +} < m \leq n} \frac{\exp\left[ -K \left(\frac{\psi_{n} 2 m}{\Vert \lambda \Vert_{l^{2}} \Vert \theta^{\circ} \Vert_{l^{2}}} \wedge \sqrt{n \psi_{n}} \right)\right]}{\exp\left[ -K \left(\frac{\psi_{n} 2 G^{+}_{n}}{\Vert \lambda \Vert_{l^{2}} \Vert \theta^{\circ} \Vert_{l^{2}}} \wedge \sqrt{n \psi_{n}} \right)\right]}\\
%& && \leq && \exp\left[-\frac{\eta \mathfrak{b}_{0}^{2}(\theta^{\circ}) n \Phi^{\dagger}_{n}}{2}\right]\sum\limits_{1 < m \leq n} \exp\left[- \frac{\eta \psi_{n} m \Lambda_{(m)}}{2}\right] +\\
%& && && \exp\left[ -K \left(\frac{\psi_{n} 2 G^{+}_{n}}{\Vert \lambda \Vert_{l^{2}} \Vert \theta^{\circ} \Vert_{l^{2}}} \wedge \sqrt{n \psi_{n}} \right)\right]\cdot \sum\limits_{1 < m \leq n} \exp\left[ -K \left(\frac{\psi_{n} 2 m}{\Vert \lambda \Vert_{l^{2}} \Vert \theta^{\circ} \Vert_{l^{2}}} \wedge \sqrt{n \psi_{n}} \right)\right]\\
%& && \leq &&  C_{\lambda, \theta^{\circ}} \exp\left[ -K \left(\frac{\psi_{n} 2 G^{\dagger +}_{n} }{\Vert \lambda \Vert_{l^{2}} \Vert \theta^{\circ} \Vert_{l^{2}}} \wedge \sqrt{n \psi_{n}} \right)\right].
%\end{alignat*}
%
%%\begin{alignat*}{3}
%%& \E_{\theta^{\circ}}^{n}\left[\P_{M \vert Y^{n}}^{n, (\eta)}\left(\llbracket G^{-}_{n} + 1, n \rrbracket\right)\right] && \leq && \sum\limits_{G^{+}_{n} < m \leq n} \exp\left[-\eta\left( \frac{\psi_{n} m \overline{\Lambda}_{m}}{2} + \frac{\mathfrak{b}_{0}^{2}(\theta^{\circ}) n \Phi^{\dagger}_{n}}{2} \right)\right] +\\
%%& && && 3 \sum\limits_{G^{+}_{n} < m \leq n} \exp\left[ -K \left(\frac{\psi_{n} 2 m^{\dagger}_{n}\overline{\Lambda_{m^{\dagger}_{n}}}}{\Lambda_{(m^{\dagger}_{n})} \Vert \lambda \Vert_{l^{2}} \Vert \theta^{\circ} \Vert_{l^{2}}} \wedge \sqrt{n \psi_{n}} \right)\right]\\
%%& && \leq && \exp\left[- \eta \frac{\mathfrak{b}_{0}^{2}(\theta^{\circ}) n \Phi^{\dagger}_{n}}{2}\right] \sum\limits_{G^{+}_{n} < m \leq n} \exp\left[-\eta \frac{\psi_{n} m \overline{\Lambda}_{m}}{2}\right] +\\
%%& && && \exp\left[ -K \left(\frac{\psi_{n} 2 G^{+}_{n}\overline{\Lambda_{G^{+}_{n}}}}{\Lambda_{(G^{+}_{n})} \Vert \lambda \Vert_{l^{2}} \Vert \theta^{\circ} \Vert_{l^{2}}} \wedge \sqrt{n \psi_{n}} \right)\right] \sum\limits_{m = 1}^{n} \exp\left[ -K \left(\frac{\psi_{n} 2 m \overline{\Lambda_{m}}}{\Lambda_{(m)} \Vert \lambda \Vert_{l^{2}} \Vert \theta^{\circ} \Vert_{l^{2}}} \wedge \sqrt{n \psi_{n}} \right)\right]\\
%%& && \leq && C_{\lambda, \theta^{\circ}} \exp\left[ -K \left(\frac{\psi_{n} 2 m^{\dagger}_{n}\overline{\Lambda_{m^{\dagger}_{n}}}}{\Lambda_{(m^{\dagger}_{n})} \Vert \lambda \Vert_{l^{2}} \Vert \theta^{\circ} \Vert_{l^{2}}} \wedge \sqrt{n \psi_{n}} \right)\right]
%%\end{alignat*}
%
%Which completes the proof.
%
%\qedsymbol
%\end{pro}
%
%\begin{pro}{\textsc{Proof of \nref{PR_FREQ_CIRCDECONV_KNOWN_IID_ORACLE_NP_DECOMPOSITION}} \\}\label{PROD.3.3}
%
%Begin with \nref{EQD.5}.
%Use this first decomposition in order to use \nref{EQD.4} and \nref{PR_FREQ_CIRCDECONV_KNOWN_IID_ORACLE_NP_TALAGRAND}
%\begin{alignat*}{2}
%& \sum\limits_{0 < \vert j \vert \leq n} && \Lambda_{j} \left(\lambda_{j} \overline{\theta}_{j} - \lambda_{j} \theta^{\circ}_{j}\right)^{2} \P_{M \vert Y^{n}}^{n, (\eta)}\left(\llbracket \vert j \vert, n \rrbracket\right)\\
%& && \leq \sum\limits_{0 < \vert j \vert \leq G^{\dagger +}_{n}} \Lambda_{j} \left(\lambda_{j} \overline{\theta}_{j} - \lambda_{j} \theta^{\circ}_{j}\right)^{2} + \sum\limits_{G^{\dagger +}_{n} < \vert j \vert \leq n} \Lambda_{j} \left(\lambda_{j} \overline{\theta}_{j} - \lambda_{j} \theta^{\circ}_{j}\right)^{2} \P_{M \vert Y^{n}}^{n, (\eta)}\left(\llbracket G^{\dagger +}_{n} + 1, n \rrbracket\right)\\
%& && \leq \left\Vert \Pi_{G^{\dagger +}_{n}}\left(\theta^{\circ} - \overline{\theta}\right) \right\Vert_{l^{2}}^{2} + \left\Vert \Pi_{G^{\dagger +}_{n}, n}\left(\theta^{\circ} - \overline{\theta}\right)\right\Vert_{l^{2}}^{2} \P_{M \vert Y^{n}}^{n, (\eta)}\left(\llbracket G^{\dagger +}_{n} + 1, n \rrbracket\right)\\
%\end{alignat*}
%
%Considering the result from \nref{PR_FREQ_CIRCDECONV_KNOWN_IID_ORACLE_NP_TALAGRAND}, and, hence, taking $\delta^{\star}_{G^{+}_{n}, n} \geq \sum\limits_{G^{+}_{n} \leq \vert j \vert \leq n} \Lambda_{j}$, we obtain
%
%\begin{alignat*}{3}
%& \sum\limits_{0 < \vert j \vert \leq n} && && \Lambda_{j} \left(\lambda_{j} \overline{\theta}_{j} - \lambda_{j} \theta^{\circ}_{j}\right)^{2} \P_{M \vert Y^{n}}^{n, (\eta)}\left(\llbracket \vert j \vert, n \rrbracket\right)\\
%& && \leq && \left\Vert \Pi_{G^{+}_{n}}\left(\theta^{\circ} - \overline{\theta}\right) \right\Vert_{l^{2}}^{2} + \left(\sup\limits_{t \in \mathds{B}_{G^{+}_{n}, n}} \left\vert \left\langle t \vert \overline{\theta} - \theta^{\circ} \right\rangle_{l^{2}}\right\vert^{2} - 6 \frac{\psi_{n} \delta^{\star}_{G^{+}_{n}, n}}{n} \right)_{+} +\\
%& && && 6 \frac{\psi_{n} \delta^{\star}_{G^{+}_{n}, n}}{n} \P_{M \vert Y^{n}}^{n, (\eta)}\left(\llbracket G^{+}_{n} + 1, n \rrbracket\right)
%\end{alignat*}
%
%We can hence use \nref{EQD.4} and \nref{EQD.1} to obtain
%
%\begin{alignat*}{3}
%& \sum\limits_{0 < \vert j \vert \leq n} && && \Lambda_{j} \E_{\theta^{\circ}}^{n}\left[\left(\lambda_{j} \overline{\theta}_{j} - \lambda_{j} \theta^{\circ}_{j}\right)^{2} \P_{M \vert Y^{n}}^{n, (\eta)}\left(\llbracket \vert j \vert, n \rrbracket\right)\right]\notag \\
%& && \leq && \frac{2 G^{+}_{n} \overline{\Lambda}_{G^{+}_{n}}}{n} +\notag\\
%& && && C \left\{\frac{\Vert \lambda \Vert_{l^{2}} \Vert \theta^{\circ} \Vert_{l^{2}} \Delta^{\star}_{G^{+}_{n}, n}}{n} \exp\left[ -\frac{\psi_{n} \delta^{\star}_{G^{+}_{n}, n}}{6 \Vert \lambda \Vert_{l^{2}} \Vert \theta^{\circ} \Vert_{l^{2}} \Delta^{\star}_{G^{+}_{n}, n}} \right] + \frac{\delta^{\star}_{G^{+}_{n}, n}}{n^{2}} \exp\left[- K \sqrt{n \psi_{n}}\right]\right\} +\notag \\
%& && && 6 \frac{\psi_{n} \delta^{\star}_{G^{+}_{n}, n}}{n} C_{\lambda, \theta^{\circ}} \exp\left[- K \left(\frac{\psi_{n} 2 m^{\dagger}_{n}}{\Vert \theta^{\circ} \Vert_{l^{2}} \Vert \lambda \Vert_{l^{2}}} \wedge \sqrt{n \psi_{n}}\right)\right].
%\end{alignat*}
%
%Using the definition of $G^{+}_{n}$ and taking $\Delta^{\star}_{G^{+}_{n}, n} = \Lambda_{(n)}$ and $\delta^{\star}_{G^{+}_{n}, n} = 2 n \Lambda_{(n)}$ and the constraints $C_{\lambda, \theta^{\circ}} \geq C \Vert \lambda \Vert_{l^{2}} \Vert \theta^{\circ} \Vert_{l^{2}}$ we obtain
%
%\begin{alignat*}{3}
%& \sum\limits_{0 < \vert j \vert \leq n} && && \Lambda_{j} \E_{\theta^{\circ}}^{n}\left[\left(\lambda_{j} \overline{\theta}_{j} - \lambda_{j} \theta^{\circ}_{j}\right)^{2} \P_{M \vert Y^{n}}^{n, (\eta)}\left(\llbracket \vert j \vert, n \rrbracket\right)\right]\notag \\
%& && \leq && 28 \mathfrak{b}_{0}^{2}(\theta^{\circ}) \Phi^{\dagger}_{n} +\notag\\
%& && && \frac{1}{n} \left\{ C \Vert \lambda \Vert_{l^{2}} \Vert \theta^{\circ} \Vert_{l^{2}} \Lambda_{(n)} \exp\left[ -\frac{\psi_{n} 2 n }{6 \Vert \lambda \Vert_{l^{2}} \Vert \theta^{\circ} \Vert_{l^{2}}} \right] + 2 \Lambda_{(n)} \exp\left[- K \sqrt{n \psi_{n}}\right]\right\} +\notag \\
%& && && 6 \psi_{n} 2 \Lambda_{(n)} C_{\lambda, \theta^{\circ}} \exp\left[- K \left(\frac{\psi_{n} 2 m^{\dagger}_{n}}{\Vert \theta^{\circ} \Vert_{l^{2}} \Vert \lambda \Vert_{l^{2}}} \wedge \sqrt{n \psi_{n}}\right)\right]\\
%& && \leq && 28 \mathfrak{b}_{0}^{2}(\theta^{\circ}) \Phi^{\dagger}_{n} + \frac{1}{n} C_{\lambda, \theta^{\circ}} 12 \psi_{n} n \Lambda_{(n)} \exp\left[- K \left(\frac{\psi_{n} 2 m^{\dagger}_{n}}{\Vert \theta^{\circ} \Vert_{l^{2}} \Vert \lambda \Vert_{l^{2}}} \wedge \sqrt{n \psi_{n}}\right)\right];
%\end{alignat*}
%which proves the statement.
%
%\bigskip
%
%Consider now \nref{EQD.6}.
%We split the sum in a similar manner:
% 
%\begin{alignat*}{2}
%& \sum\limits_{0 < \vert j \vert \leq n} && \left(\theta^{\circ}_{j}\right)^{2} \E_{\theta^{\circ}}^{n}\left[\P_{M \vert Y^{n}}^{n, (\eta)}\left(\llbracket 0, j-1 \rrbracket \right)\right] + \sum\limits_{ \vert j \vert > n} \left( \theta^{\circ}_{j}\right)^{2}\\
%& && \leq \sum\limits_{j \in \llbracket 1, G^{\dagger -}_{n} \rrbracket} \vert \theta^{\circ}_{j}\vert^{2} \E_{\theta^{\circ}}^{n}\left[\P_{M \vert Y^{n}}^{n, (\eta)}\left(\llbracket 1, \vert j \vert-1 \rrbracket\right)\right] + \sum\limits_{\vert j \vert \in \llbracket G^{\dagger -}_{n} + 1, n\rrbracket} \vert \theta^{\circ}_{j} \vert^{2} + \sum\limits_{\vert j \vert > n} \vert \theta^{\circ}_{j} \vert^{2}\\
%& && \leq \left\Vert \theta^{\circ} \right\Vert_{l^{2}}^{2} \E_{\theta^{\circ}}^{n}\left[\P_{M \vert Y^{n}}^{n, (\eta)}\left(\llbracket 0, G^{\dagger -}_{n} + 1\rrbracket\right)\right] + \mathfrak{b}_{G^{\dagger-}_{n}}^{2}(\theta^{\circ})
%\end{alignat*}
%
%So we now use \nref{EQD.3},
%\begin{alignat*}{2}
%& \sum\limits_{0 < \vert j \vert \leq n} && \left(\theta^{\circ}_{j}\right)^{2} \E_{\theta^{\circ}}^{n}\left[\P_{M \vert Y^{n}}^{n, (\eta)}\left(\llbracket 0, j-1 \rrbracket \right)\right] + \sum\limits_{ \vert j \vert > n} \left( \theta^{\circ}_{j}\right)^{2}\\
%& && \leq \mathfrak{b}_{G^{-}_{n}}^{2}(\theta^{\circ}) + \frac{1}{n} C_{\lambda, \theta^{\circ}} 4 n m^{\dagger}_{n} \exp\left[- K \left(\frac{\psi_{n} 2 m^{\dagger}_{n}}{\Vert \theta^{\circ} \Vert_{l^{2}} \Vert \lambda \Vert_{l^{2}}} \wedge \sqrt{n \psi_{n}}\right)\right]
%\end{alignat*}
%The proof is completed using the definition of $G^{-}_{n}$.
%
%\qedsymbol
%\end{pro}
%
%\begin{pro}{\textsc{Proof of \nref{THM_FREQ_CIRCDECONV_KNOWN_IID_ORACLE_NP}} \\}\label{PRO_FREQ_CIRCDECONV_KNOWN_IID_ORACLE_NP}
%For any $j$ in $\mathds{Z}$, and $\eta$ in $\N^{\star}$ we have:
%\begin{alignat*}{3}
%& \widehat{\theta}^{(\eta)}_{j} - \theta^{\circ}_{j} && = && \left(\overline{\theta}_{j} - \theta^{\circ}_{j}\right) \P_{M \vert Y^{n}}^{n, (\eta)}(\llbracket \vert j \vert, n\rrbracket) - \theta^{\circ}_{j} \P_{M \vert Y^{n}}^{n, (\eta)}(\llbracket 1, \vert j \vert - 1 \rrbracket).
%\end{alignat*}
%Note that with $j = 0$ the previous equality gives $0$ and with $j > n$ it gives $-\theta^{\circ}_{j}$.
%
%It follows that
%\begin{alignat*}{4}
%& \E_{\theta^{\circ}}^{n}\left[\Vert \widehat{\theta}^{(\eta)} - \theta^{\circ}\Vert_{l^{2}}^{2}\right] && = && \sum\limits_{\vert j \vert \in \llbracket 1, n \rrbracket} && \E_{\theta^{\circ}}^{n}\left[\left\vert \left(\overline{\theta}_{j} - \theta^{\circ}_{j}\right) \P_{M \vert Y^{n}}^{n, (\eta)}(\llbracket \vert j \vert, n\rrbracket) - \theta^{\circ}_{j} \P_{M \vert Y^{n}}^{n, (\eta)}(\llbracket 1, \vert j \vert - 1 \rrbracket) \right\vert^{2}\right] +\\
%& && && && \sum\limits_{\vert j \vert > n} \left\vert \theta^{\circ}_{j} \right\vert^{2}\\
%& && \leq &&  \sum\limits_{\vert j \vert \in \llbracket 1, n \rrbracket} && \E_{\theta^{\circ}}^{n}\left[\left\vert \overline{\theta}_{j} - \theta^{\circ}_{j}\right\vert^{2} \P_{M \vert Y^{n}}^{n, (\eta)}(\llbracket \vert j \vert, n\rrbracket)\right] + \left\vert\theta^{\circ}_{j} \right\vert^{2} \E_{\theta^{\circ}}^{n}\left[\P_{M \vert Y^{n}}^{n, (\eta)}(\llbracket 1, \vert j \vert - 1 \rrbracket)\right] +\\
%& && && && \sum\limits_{\vert j \vert > n} \left\vert \theta^{\circ}_{j} \right\vert^{2}.
%\end{alignat*}
%The proof is completed by using \nref{PR_FREQ_CIRCDECONV_KNOWN_IID_ORACLE_NP_DECOMPOSITION}.
%
%\qedsymbol
%\end{pro}
\chapter{Proof for \nref{THM_FREQ_CIRCDECONV_KNOWN_IID_ORACLE_NP_FAST}}\label{PRO_FREQ_CIRCDECONV_KNOWN_IID_ORACLE_NP_FAST}
\input{./tex/proofs/freq_circdeconv_known_iid_oracle_np_fast.tex}
\chapter{Proof for \textsc{\nref{THM_FREQ_CIRCDECONV_KNOWN_BETA_ORACLE_NP}}}\label{PRO_FREQ_CIRCDECONV_KNOWN_BETA_ORACLE_NP}
\section{Intermediate results}

\begin{de}\label{DED.5.1}
Define the following quantities :
\begin{alignat*}{3}
& G_{n}^{-} &&:=&& \min\left\{m \in \llbracket 1, m_{n}^{\dagger} \rrbracket : \quad \mathfrak{b}_{m}^{2} \leq 141 \mathfrak{b}_{0}^{2} \Phi^{\dagger}_{n}\right\},\\
& G_{n}^{+} &&:=&& \max \left\{m \in \llbracket m_{n}^{\dagger}, n \rrbracket : \psi_{n} m \Lambda_{m} \leq \frac{25}{4} n \mathfrak{b}_{0}^{2} \Phi_{n}^{\dagger} \right\}.
\end{alignat*}
\end{de}

\begin{pr}\label{PRD.5.1}
Using the notations of \nref{DED.5.1}, we have
\begin{alignat}{4}
& \E_{\theta^{\circ}}^{n}\left[\left\Vert \widehat{\theta} - \theta^{\circ}\right\Vert_{l^{2}}^{2}\right] && \leq && \sum\limits_{0 < \vert j \vert \leq n}&& \E_{\theta^{\circ}}^{n}\left[\P_{M \vert Y^{n}}^{n, (\eta)} \left(\llbracket \vert j \vert, n \rrbracket\right) \left\vert  \left(\overline{\theta}_{j} - \theta^{\circ}_{j}\right) \right\vert^{2}\right] + \label{EQD.26}\\
& && && && \E_{\theta^{\circ}}^{n}\left[\P_{M \vert Y^{n}}^{n, (\eta)} \left(\left\llbracket 1, G^{-}_{n} - 1\right\rrbracket \right) \right] \left\Vert \theta^{\circ} \right\Vert_{l^{2}}^{2} + 141 \mathfrak{b}_{0}^{2}(\theta^{\circ}) \Phi^{\dagger}_{n}. \label{EQD.27}
\end{alignat}

\end{pr}

\begin{pr}{\textsc{Control of \nref{EQD.30}} \\}\label{PRD.5.2}
Using the notations from \nref{DED.5.1} and \nref{DE2.5.3} we have, under \nref{ASB.0.1}
\begin{alignat}{3}
& \sum\limits_{0 < \vert j \vert \leq n} \E_{\theta^{\circ}}^{n}\left[\P_{M \vert Y^{n}}^{n, (\eta)} \left(\llbracket \vert j \vert, n \rrbracket \right) \left\vert  \overline{\theta}_{j} - \theta^{\circ}_{j} \right\vert^{2}\right] && \leq && \frac{25 \mathfrak{b}_{0}^{2}(\theta^{\circ}) \Phi^{\dagger}_{n}}{2 \psi_{n}} + \notag\\
& && && \E_{\theta^{\circ}}^{n}\left[ \left(\sup\limits_{t \in \mathds{B}_{G^{+}_{n}, n}} \left\vert\left\langle t \vert \overline{\theta} - \theta^{\circ}\right\rangle_{l^{2}} \right\vert^{2} - 2 \Lambda_{(n)} \psi_{n} \right)_{+} \right]+\notag\\
& && && 2 \Lambda_{(n)} \psi_{n} \E_{\theta^{\circ}}^{n}\left[ \P_{M \vert Y^{n}}^{n, (\eta)}\left(\llbracket G^{+}_{n} + 1, n \rrbracket\right)\right]\label{EQD.28}.
\end{alignat}
\end{pr}

\begin{pr}\label{PRD.5.3}
For any $n$ and $\eta$ in $\N$, and constant $C_{\lambda, \theta^{\circ}} > \sum\limits_{j = 1}^{\infty} \exp\left[- \eta \frac{\psi_{n} m \Lambda_{(m)}}{2}\right]$ we have

%\begin{alignat}{3}
%& \E_{\theta^{\circ}}^{n}\left[\P_{M \vert Y^{n}}^{n, (\eta)}\left(\llbracket 0, G^{\dagger -}_{n} - 1 \rrbracket\right)\right] && \leq && 4 m^{\dagger}_{n} \left(\exp\left[- \frac{(m^{\dagger}_{n})\psi_{n}}{800 s \Vert \theta^{\circ} \Vert_{l^{2}} \cdot \Vert \lambda \Vert_{l^{2}}}\right] + \exp\left[ \frac{-\sqrt{r \psi_{n}}}{100 \sqrt{2 s}}\right] + r\beta_{s}\right)\label{EQD.28}\\
%& \E_{\theta^{\circ}}^{n}\left[\P_{M \vert Y^{n}}^{n, (\eta)}\left(\llbracket G^{\dagger +}_{n} + 1, n \rrbracket\right)\right] && \leq && C_{\theta^{\circ}\lambda} \left(\exp\left[- \frac{m^{\dagger}_{n}\psi_{n}}{800 s \Vert \theta^{\circ} \Vert_{l^{2}} \cdot \Vert \lambda \Vert_{l^{2}}}\right] + \exp\left[ \frac{- \sqrt{ r \psi_{n}}}{\sqrt{2 s} 100}\right]\right) + n r \beta_{s}\label{EQD.29}
%\end{alignat}

\begin{alignat}{3}
& \E_{\theta^{\circ}}^{n}\left[\P_{M \vert Y^{n}}^{n, (\eta)}\left(\left\llbracket 1, G^{-}_{n} - 1 \right\rrbracket\right)\right] && \leq && G^{-}_{n}\exp\left[\frac{- \eta n \mathfrak{b}_{0}^{2}(\theta^{\circ}) \Phi^{\dagger}_{n}}{2}\right] +\notag\\
& && && \sum\limits_{1 \leq \vert j \vert < G^{-}_{n}}\P_{\theta^{\circ}}^{n}\left(\sup\limits_{x \in \mathds{B}_{m, m^{\dagger}_{n}}} \left\vert \left\langle x \vert \Pi_{m, m^{\dagger}_{n}} (\overline{\theta} - \theta^{\circ}) \right\rangle \right\vert_{l^{2}}^{2} < \frac{2 \Lambda_{(m^{\dagger}_{n})} m^{\dagger}_{n} \psi_{n}}{n}\right)\label{EQD.28}\\
& \E_{\theta^{\circ}}^{n}\left[\P_{M \vert Y^{n}}^{n, (\eta)}(\llbracket G^{+}_{n} + 1, n \rrbracket)\right] && \leq && C_{\theta^{\circ} \lambda} \exp \left[-\frac{\eta \mathfrak{b}_{0}^{2}(\theta^{\circ}) n \Phi^{\dagger}_{n}}{2}\right] +\notag\\
& && && \sum\limits_{G^{+}_{n} < \vert j \vert \leq n}\P_{\theta^{\circ}}^{n}\left(\sup\limits_{x \in \mathds{B}_{m^{\dagger}_{n}, m}} \left\vert \left\langle x \vert \Pi_{m^{\dagger}_{n}, m} (\overline{\theta} - \theta^{\circ}) \right\rangle \right\vert_{l^{2}}^{2} < \frac{2 \Lambda_{(m)} m \psi_{n}}{n}\right)\label{EQD.29}
\end{alignat}
\end{pr}

\begin{pr}\label{PRD.5.4}
For any integer $m$ and $l$ such that $m \leq l$, define for any $t$ in $\mathds{B}_{m, l}$ the functional $\overline{\nu}_{t} = \left\langle t \vert \overline{\theta} - \theta^{\circ}\right\rangle_{l^{2}}$.
Under \nref{ASB.0.2}, we define
\begin{alignat*}{3}
& \overline{\nu}^{e, \perp}_{t} && = && \frac{1}{r} \sum\limits_{q = 1}^{r} \left(v_{t}(E^{\perp}_{q}) - \E_{\theta^{\circ}}^{n}\left[v_{t}(E^{\perp}_{q})\right]\right);\\
& v_{t}(E^{\perp}_{q}) && = && \frac{1}{s} \sum\limits_{p = 1}^{s} \nu_{t}(E^{\perp}_{q, p});\\
& \nu_{t}(E^{\perp}_{q, p}) && = && \sum\limits_{m \leq \vert j \vert \leq l} \left(\frac{t_{j}}{\overline{\lambda_{j}}} e_{j}(E^{\perp}_{q, p})\right).
\end{alignat*}
Then, for any sequence $\left(C_{n}\right)_{n \in \N}$, we have the following inequalities:
\begin{alignat}{3}
& \E_{\theta^{\circ}}^{n} \left[ \left(\sup\limits_{t \in \mathds{B}_{m, l}} \left\vert\left\langle t \vert \overline{\theta} - \theta^{\circ}\right\rangle_{l^{2}} \right\vert^{2} - C_{n} \right)_{+} \right] && \leq && 2 \cdot \E_{\theta^{\circ}}^{n}\left[ \left(\sup\limits_{t \in \mathds{B}_{m, l}} \vert \overline{\nu}^{e, \perp}_{t} \vert^{2} - C_{n} \right)_{+} \right] +\notag\\
& && && 2 \cdot \E_{\theta^{\circ}}^{n}\left[ \sup\limits_{t \in \mathds{B}_{m, l}} \vert \overline{\nu}^{e, \perp}_{t} - \overline{\nu}^{e}_{t} \vert^{2} \right]\label{EQD.26}\\
& \P_{\theta^{\circ}}^{n}\left(\sup\limits_{t \in \mathds{B}_{m, l}} \left\vert \left\langle t \vert \overline{\theta} - \theta^{\circ}\right\rangle_{l^{2}} \right\vert \geq C_{n}\right) && \leq && \P_{\theta^{\circ}}^{n}\left(\sup\limits_{t \in \mathds{B}_{m, l}} \left\vert \overline{\nu}_{t}^{e, \perp} \right\vert \geq C_{n} \right) +\notag\\
& && && 3 \P_{\theta^{\circ}}^{n}\left(\overline{\nu}_{t}^{e} \neq \overline{\nu}_{t}^{e, \perp}\right)\label{EQD.27}
\end{alignat}
\end{pr}

\begin{pr}\label{PRD.5.5}
For any integers $m$ and $l$ with $m < l$; consider a triplet $h^{2}$, $H^{2}$ and $v$ verifying
\begin{alignat*}{3}
& h^{2} && \geq &&\sum\limits_{m \leq \vert j \vert \leq l} \Lambda_{j};\\
& H^{2} && \geq && \frac{\Lambda_{(l)}(l-m+1)( \psi_{n} + 1)}{n};\\
& v && \geq &&\Lambda_{(l)} \Vert \theta^{\circ} \Vert_{l^{2}} \cdot \Vert \lambda \Vert_{l^{2}};
\end{alignat*}
then, under \nref{ASB.0.1}, for any $C > 0$, we have:
\begin{alignat}{3}
& \E_{\theta^{\circ}}^{n}\left[\left(\sup\limits_{t \in \mathds{B}_{m, l}}\vert \overline{\nu}^{e, \perp}_{t} \vert^{2} - 6 H^{2}\right)_{+}\right] && \leq && C \left[\frac{v}{r} \exp\left(\frac{-r H^{2}}{6 v}\right) + \frac{h^{2}}{r^{2}} \exp\left(\frac{- r H}{100 h}\right)\right];\label{EQD.28}\\
& \P_{\theta^{\circ}}^{n}\left(\sup\limits_{t \in \mathds{B}_{m, l}} \vert \overline{\nu}^{e, \perp}_{t} \vert \geq 6 H^{2}\right) && \leq && 3 \left(\exp\left[- \frac{r H^{2}}{400 v}\right] + \exp\left[ \frac{-r H}{100 h}\right]\right)\label{EQD.29}.
\end{alignat}
\end{pr}

\begin{pr}\label{PRD.5.6}
\begin{alignat}{3}
& \E_{\theta^{\circ}}^{n}\left[\sup\limits_{t \in \mathds{B}_{m, l}} \left\vert \overline{\nu}^{e, \perp}_{t} - \overline{\nu}^{e}_{t} \right\vert^{2} \right] && \leq && 4 \beta_{s} \sum\limits_{m \leq \vert j \vert \leq l} \Lambda_{j}\label{EQD.30}\\
& \P_{\theta^{\circ}}^{n}\left(\overline{\nu}_{t}^{e} \neq \overline{\nu}_{t}^{e, \perp}\right) && \leq && r\beta_{s}\label{EQD.31}
\end{alignat}
\end{pr}

\section{Detailed proofs}

\begin{pro}{\textsc{Proof for \nref{PRD.5.1}} \\}\label{PROD.5.1}
Using triangular inequality, linearity of the expectation, and the structure of our estimator, straightforward calculus yields:
\begin{alignat*}{4}
& \E_{\theta^{\circ}}^{n}\left[\left\Vert \widehat{\theta} - \theta^{\circ}\right\Vert_{l^{2}}^{2}\right] && = && \E_{\theta^{\circ}}^{n}&&\left[\sum\limits_{0 < \vert j \vert \leq n} \left\vert \P_{M \vert Y^{n}}^{n, (\eta)} \left(\llbracket \vert j \vert, n \rrbracket \right) \left(\overline{\theta}_{j} - \theta^{\circ}_{j}\right) - \P_{M \vert Y^{n}}^{n, (\eta)} \left(\llbracket 1, \vert j \vert - 1 \rrbracket \right) \theta^{\circ}_{j}\right\vert^{2} \right]\\
& && && && + \sum\limits_{\vert j \vert > n} \vert \theta^{\circ} \vert^{2}\\
& && \leq && \sum\limits_{0 < \vert j \vert \leq n}&& \E_{\theta^{\circ}}^{n}\left[\P_{M \vert Y^{n}}^{n, (\eta)} \left(\llbracket \vert j \vert, n \rrbracket\right) \left\vert  \left(\overline{\theta}_{j} - \theta^{\circ}_{j}\right) \right\vert^{2}\right] +\\
& && && && \sum\limits_{0 < \vert j \vert \leq n} \left\vert \theta^{\circ}_{j}\right\vert^{2} \E_{\theta^{\circ}}^{n}\left[\P_{M \vert Y^{n}}^{n, (\eta)} \left(\llbracket 1, \vert j \vert - 1 \rrbracket\right) \right] + \sum\limits_{\vert j \vert > n} \vert \theta^{\circ}_{j} \vert^{2}.
\end{alignat*}

The second term can be decomposed as follows, using the definition of $G^{-}_{n}$:

\begin{alignat*}{2}
& \sum\limits_{0 < \vert j \vert \leq n} &&\left\vert \theta^{\circ}_{j} \right\vert^{2} \E_{\theta^{\circ}}^{n}\left[\P_{M \vert Y^{n}}^{n, (\eta)} \left(\left\llbracket 1, \vert j \vert - 1\right\rrbracket \right) \right] + \sum\limits_{\vert j \vert > n} \left\vert \theta^{\circ}_{j} \right\vert^{2}\\
& && \leq \sum\limits_{0 < \vert j \vert \leq G^{-}_{n} - 1} \left\vert \theta^{\circ}_{j} \right\vert^{2} \E_{\theta^{\circ}}^{n}\left[\P_{M \vert Y^{n}}^{n, (\eta)} \left(\left\llbracket 1, \vert j \vert - 1\right\rrbracket \right) \right] + \sum\limits_{G^{-}_{n} \leq \vert j \vert \leq n} \left\vert \theta^{\circ}_{j}\right\vert^{2} + \sum\limits_{\vert j \vert > n} \left\vert \theta^{\circ}_{j} \right\vert^{2}\\
& && \leq \E_{\theta^{\circ}}^{n}\left[\P_{M \vert Y^{n}}^{n, (\eta)} \left(\left\llbracket 1, G^{-}_{n} - 1\right\rrbracket \right) \right] \left\Vert \theta^{\circ} \right\Vert_{l^{2}}^{2} + \mathfrak{b}_{G^{-}_{n}}^{2}(\theta^{\circ})\\
& && \leq \E_{\theta^{\circ}}^{n}\left[\P_{M \vert Y^{n}}^{n, (\eta)} \left(\left\llbracket 1, G^{-}_{n} - 1\right\rrbracket \right) \right] \left\Vert \theta^{\circ} \right\Vert_{l^{2}}^{2} + 141 \mathfrak{b}_{0}^{2} \Phi^{\dagger}_{n}\\.
\end{alignat*}

which proves the statement.

\qedsymbol
\end{pro}

\begin{pro}{\textsc{Proof of \nref{PRD.5.2}} \\}\label{PROD.5.2}
First decompose the sum between the "good" and "bad" values of the threshold parameter:
\begin{alignat}{3}
& \sum\limits_{0 < \vert j \vert \leq n} \E_{\theta^{\circ}}^{n}\left[\P_{M \vert Y^{n}}^{n, (\eta)} \left(\llbracket \vert j \vert, n \rrbracket \right) \left\vert  \overline{\theta}_{j} - \theta^{\circ}_{j} \right\vert^{2}\right] && \leq && \sum\limits_{0 < \vert j \vert \leq G^{+}_{n}} \E_{\theta^{\circ}}^{n}\left[\left\vert \overline{\theta}_{j} - \theta^{\circ}_{j}\right\vert^{2} \right] +\label{EQD.28}\\
& && && \sum\limits_{G^{+}_{n} < \vert j \vert \leq n} \E_{\theta^{\circ}}^{n}\left[\P_{M \vert Y^{n}}^{n, (\eta)}\left(\llbracket G^{+}_{n} + 1, n\rrbracket\right) \left\vert \overline{\theta}_{j} - \theta^{\circ}_{j}\right\vert^{2} \right]\label{EQD.29}.
\end{alignat}

We control \nref{EQD.28} using the definition of $G^{+}_{n}$ and \nref{LMB.0.1} which can be applied thanks to \nref{ASB.0.1}:

\begin{alignat*}{3}
& \sum\limits_{0 < \vert j \vert \leq G^{+}_{n}} \E_{\theta^{\circ}}^{n}\left[\left\vert \overline{\theta}_{j} - \theta^{\circ}_{j}\right\vert^{2} \right] && = && \sum\limits_{0 < \vert j \vert \leq G^{+}_{n}} \V_{\theta^{\circ}}^{n}\left[ \overline{\theta}_{j} - \theta^{\circ}_{j} \right]\\
& && = && \sum\limits_{0 < \vert j \vert \leq G^{+}_{n}} \frac{\Lambda_{j}}{n^{2}} \V_{\theta^{\circ}}^{n}\left[ \sum\limits_{p = 1}^{n} e_{j}(Y_{p}^{n})\right]\\
& && \leq && \frac{\Lambda_{(G^{+}_{n})}2 G^{+}_{n}}{n}\\
& && \leq && \frac{25 \mathfrak{b}_{0}^{2}(\theta^{\circ}) \Phi^{\dagger}_{n}}{2 \psi_{n}}\\
\end{alignat*}

Consider \nref{EQD.29}

\begin{alignat}{3}
& \sum\limits_{G^{+}_{n} < \vert j \vert \leq n} \E_{\theta^{\circ}}^{n}\left[\P_{M \vert Y^{n}}^{n, (\eta)}\left(\llbracket G^{+}_{n} + 1, n\rrbracket\right) \left\vert \overline{\theta}_{j} - \theta^{\circ}_{j}\right\vert^{2} \right] && = &&  \E_{\theta^{\circ}}^{n}\left[\P_{M \vert Y^{n}}^{n, (\eta)}\left(\llbracket G^{+}_{n} + 1, n\rrbracket\right) \sum\limits_{G^{+}_{n} < \vert j \vert \leq n} \left\vert \overline{\theta}_{j} - \theta^{\circ}_{j}\right\vert^{2} \right] \notag\\
& && = && \E_{\theta^{\circ}}^{n}\left[\P_{M \vert Y^{n}}^{n, (\eta)}\left(\llbracket G^{+}_{n} + 1, n\rrbracket\right) \left\Vert \Pi_{G^{+}_{n}, n}\left(\overline{\theta} - \theta^{\circ}\right)\right\Vert_{l^{2}}^{2} \right] \notag\\
& && = && \E_{\theta^{\circ}}^{n}\left[ \left(\sup\limits_{t \in \mathds{B}_{G^{+}_{n}, n}} \left\vert\left\langle t \vert \overline{\theta} - \theta^{\circ}\right\rangle_{l^{2}} \right\vert^{2} - 2 \Lambda_{(n)} \psi_{n} \right)_{+} \right]+ \label{EQD.34}\\
& && && 2 \Lambda_{(n)} \psi_{n} \E_{\theta^{\circ}}^{n}\left[ \P_{M \vert Y^{n}}^{n, (\eta)}\left(\llbracket G^{+}_{n} + 1, n \rrbracket\right)\right]\label{EQD.35}
\end{alignat}
\end{pro}

\begin{pro}{\textsc{Proof of \nref{PRD.5.3}} \\}\label{PROD.5.3}

Before considering the two inequalities separately, let us do some observations.

Throughout the proof, $m$ and $l$ will be two positive integers.

For now, assume $m < l$ and let $H$ be a positive real number.

We use \nref{EQD.11}, \nref{EQD.14} in order to obtain, with the event $\mathcal{A}_{m, l} := \left\{ \sup\limits_{x \in \mathds{B}_{m, l}} \left\vert \left\langle x \vert \Pi_{m, l} (\overline{\theta} - \theta^{\circ}) \right\rangle \right\vert_{l^{2}}^{2} < 6H^{2} \right\}$

\begin{alignat*}{4}
& \P_{M \vert Y^{n}}^{n, (\eta)}(m) && \leq && \exp &&\left[\eta\left((\pen(l)-\pen(m)) +\right.\right.\\
& && && && \left.\left. \frac{n}{2} \left( \frac{21}{4} \sup\limits_{x \in \mathds{B}_{m, l}} \left\vert \left\langle x \vert \Pi_{m, l} (\overline{\theta} - \theta^{\circ}) \right\rangle \right\vert_{l^{2}}^{2} - \frac{3}{4} \left\Vert \Pi_{m, l} \theta^{\circ} \right\Vert_{l^{2}}^{2} \right) \right)\right] \mathds{1}_{\mathcal{A}_{m, l}} +\\
& && && && \mathds{1}_{\mathcal{A}_{m, l}^{c}}
\end{alignat*}

which gives us

\begin{alignat*}{4}
& \E_{\theta^{\circ}}^{n}\left[\P_{M \vert Y^{n}}^{n, (\eta)}(m)\right]&& \leq && \E_{\theta^{\circ}}^{n} && \left[\exp\left[\eta\left((\pen(l)-\pen(m)) +\right.\right.\right.\notag\\
& && && && \left.\left.\left. \frac{n}{2} \left( \frac{21}{4} \sup\limits_{x \in \mathds{B}_{m, l}} \left\vert \left\langle x \vert \Pi_{m, l} (\overline{\theta} - \theta^{\circ}) \right\rangle \right\vert_{l^{2}}^{2} - \frac{3}{4} \left\Vert \Pi_{m, l} \theta^{\circ} \right\Vert_{l^{2}}^{2} \right) \right)\right]\mathds{1}_{\mathcal{A}_{m, l}} \right] + \\
& && && &&\P_{\theta^{\circ}}^{n}\left(\mathcal{A}_{m, l}^{c}\right)\\
& && \leq && \exp && \left[\eta\left((\pen(l)-\pen(m)) +\right.\right.\\
& && && && \left.\left. \frac{n}{2} \left( \frac{63}{2} H^{2} - \frac{3}{4} \left\Vert \Pi_{m, l} \theta^{\circ} \right\Vert_{l^{2}}^{2} \right) \right)\right] + \notag\\
& && && && \P_{\theta^{\circ}}^{n}\left(\sup\limits_{x \in \mathds{B}_{m, l}} \left\vert \left\langle x \vert \Pi_{m, l} (\overline{\theta} - \theta^{\circ}) \right\rangle \right\vert_{l^{2}}^{2} > 6H^{2}\right)
\end{alignat*}

%\begin{alignat*}{4}
%& \E_{\theta^{\circ}}^{n}\left[\P_{M \vert Y^{n}}^{n, (\eta)}(m)\right]&& \leq && \E_{\theta^{\circ}}^{n} && \left[\exp\left[\eta\left((\pen(l)-\pen(m)) +\right.\right.\right.\notag\\
%& && && && \left.\left.\left. \frac{n}{2} \left( \frac{21}{4} \sup\limits_{x \in \mathds{B}_{m, l}} \left\vert \left\langle x \vert \Pi_{m, l} (\overline{\theta} - \theta^{\circ}) \right\rangle \right\vert_{l^{2}}^{2} - \frac{3}{4} \left\Vert \Pi_{m, l} \theta^{\circ} \right\Vert_{l^{2}}^{2} \right) \right)\right]\mathds{1}_{\mathcal{A}_{m, l}} \right] + \notag\\
%& && && &&\P_{\theta^{\circ}}^{n}\left(\mathcal{A}_{m, l}^{c}\right)\notag\\
%& && \leq && \exp && \left[\eta\left((\pen(l)-\pen(m)) +\right.\right.\notag\\
%& && && && \left.\left. \frac{n}{2} \left( \frac{21}{2} 3H^{2} - \frac{3}{4} \left\Vert \Pi_{m, l} \theta^{\circ} \right\Vert_{l^{2}}^{2} \right) \right)\right] + \notag\\
%& && && && 3 \left(\exp\left[- \frac{r H^{2}}{400 v}\right] + \exp\left[ \frac{-r H}{100 h}\right]\right) +\notag\\
%& && && && 3 r\beta_{s}.\label{EQD.34}
%\end{alignat*}

%, $h^{2} = m^{\dagger}_{n} \Lambda_{(m^{\dagger}_{n})}$, and $v = \Lambda_{(m^{\dagger}_{n})} \Vert \theta^{\circ} \Vert_{l^{2}} \cdot \Vert \lambda \Vert_{l^{2}}$


In particular with $l = m^{\dagger}_{n}$, set $H^{2} = 2 \frac{\Lambda_{(m^{\dagger}_{n})}m^{\dagger}_{n}\psi_{n}}{n}$.
Using the definitions of $\pen$, and $\Phi^{\dagger}_{n}$, and finishing with $\kappa = \frac{41}{4}$, we obtain


\begin{alignat}{4}
& \E_{\theta^{\circ}}^{n}\left[\P_{M \vert Y^{n}}^{n, (\eta)}(m)\right]&& \leq && \exp && \left[\frac{\eta n}{2}\left(\frac{419}{4} \mathfrak{b}_{0}^{2}(\theta^{\circ}) \Phi^{\dagger}_{n} - \frac{3}{4} \mathfrak{b}_{m}^{2}(\theta^{\circ}) \right)\right] + \notag\\
& && && && \P_{\theta^{\circ}}^{n}\left(\sup\limits_{x \in \mathds{B}_{m, m^{\dagger}_{n}}} \left\vert \left\langle x \vert \Pi_{m, m^{\dagger}_{n}} (\overline{\theta} - \theta^{\circ}) \right\rangle \right\vert_{l^{2}}^{2} > 6H^{2}\right).
\end{alignat}



%\begin{alignat}{4}
%& \E_{\theta^{\circ}}^{n}\left[\P_{M \vert Y^{n}}^{n, (\eta)}(m)\right] && \leq && \exp && \left[\frac{n \eta}{2}\left(\frac{419}{4} \mathfrak{b}_{0}^{2}(\theta^{\circ}) \Phi^{\dagger}_{n} - \frac{3}{4} \mathfrak{b}_{m}^{2}(\theta^{\circ}) \right)\right] + \notag\\
%& && && && 3 \left(\exp\left[- \frac{m^{\dagger}_{n}\psi_{n}}{800 s \Vert \theta^{\circ} \Vert_{l^{2}} \cdot \Vert \lambda \Vert_{l^{2}}}\right] + \exp\left[ \frac{- \sqrt{r \psi_{n}}}{100 \sqrt{ 2 s }}\right]\right) +\notag\\
%& && && && 3 r\beta_{s}\label{EQD.35}.
%\end{alignat}

\medskip

Now, we assume $m > l$.

We use \nref{EQD.11}, and \nref{EQD.17} in order to obtain, with the event $\mathcal{A}_{m, l} := \left\{ \sup\limits_{x \in \mathds{B}_{l, m}} \left\vert \left\langle x \vert \Pi_{l, m} (\overline{\theta} - \theta^{\circ}) \right\rangle \right\vert_{l^{2}}^{2} < 6H^{2} \right\}$ for some real number $H$

\begin{alignat*}{3}
& \P_{M \vert Y^{n}}^{n, (\eta)}(m) && \leq && \exp\left[\eta\left((\pen(l)-\pen(m)) +\right.\right.\notag\\
& && && \left.\left. \frac{n}{2} \left(\frac{13}{4}\sup\limits_{x \in \mathds{B}_{l, m}}\left\vert \left\langle x \vert  (\overline{\theta} - \theta^{\circ}) \right\rangle_{l^{2}} \right\vert^{2} + \frac{5}{4} \Vert \Pi_{l, m} \theta^{\circ} \Vert_{l^{2}}^{2} \right) \right)\right] \mathds{1}_{\mathcal{A}_{m, l}} +\notag\\
& && && \mathds{1}_{\mathcal{A}_{m, l}}^{c}\\
& \E_{\theta^{\circ}}^{n}\left[\P_{M \vert Y^{n}}^{n, (\eta)}(m)\right] && \leq &&  \exp \left[\eta\left((\pen(l)-\pen(m)) +\right.\right.\notag\\
& && && \left.\left. \frac{n}{2} \left( \frac{13}{2} 3H^{2} + \frac{5}{4} \left\Vert \Pi_{l, m} \theta^{\circ} \right\Vert_{l^{2}}^{2} \right) \right)\right] + \notag\\
& && && \P_{\theta^{\circ}}^{n}\left(\sup\limits_{x \in \mathds{B}_{l, m}} \left\vert \left\langle x \vert \Pi_{l, m} (\overline{\theta} - \theta^{\circ}) \right\rangle \right\vert_{l^{2}}^{2} > 6H^{2}\right).
\end{alignat*}

In particular with $l = m^{\dagger}_{n}$, set $H^{2} = 2 \frac{\Lambda_{(m)} m \psi_{n}}{n}$.
Using the definitions of $\pen$ and $\Phi^{\dagger}_{n}$ and finishing with $\kappa = 41/4$ we obtain

\begin{alignat}{3}
& \E_{\theta^{\circ}}^{n}\left[\P_{M \vert Y^{n}}^{n, (\eta)}(m)\right] && \leq &&  \exp \left[\frac{\eta n}{2}\left(\left(\frac{169}{4}\right) \mathfrak{b}_{0}^{2}(\theta^{\circ}) \Phi^{\dagger}_{n} - 2 \frac{m \Lambda_{(m)} \psi_{n}}{n} \right)\right] + \notag\\
& && && \P_{\theta^{\circ}}^{n}\left(\sup\limits_{x \in \mathds{B}_{m^{\dagger}_{n}, m}} \left\vert \left\langle x \vert \Pi_{m^{\dagger}_{n}, m} (\overline{\theta} - \theta^{\circ}) \right\rangle \right\vert_{l^{2}}^{2} > 6H^{2}\right).\label{EQD.36}
\end{alignat}

%\begin{alignat}{3}
%& \E_{\theta^{\circ}}^{n}\left[\P_{M \vert Y^{n}}^{n, (\eta)}(m)\right] && \leq && \exp \left[\frac{n\eta}{2}\left(\left(\frac{23}{2}\right) \mathfrak{b}_{0}^{2} \Phi^{\dagger}_{n} - 2 \frac{m \Lambda_{(m)} \psi_{n}}{n} \right)\right] + \notag\\
%& && && 3 \left(\exp\left[- \frac{m\psi_{n}}{800 s \Vert \theta^{\circ} \Vert_{l^{2}} \cdot \Vert \lambda \Vert_{l^{2}}}\right] + \exp\left[ \frac{- \sqrt{ rm\psi_{n}}}{100 \sqrt{ 2 s m}}\right]\right) +\notag\\
%& && && 3 r\beta_{s} \label{EQD.37}.
%\end{alignat}

\bigskip

We will now conclude using the definitions of $G^{\dagger -}_{n}$ and $G^{\dagger +}_{n}$.

Consider \nref{EQD.32}.
As for all $m < G^{\dagger -}_{n}$, we have $\mathfrak{b}_{m}^{2}(\theta^{\circ}) > 141 \Phi^{\dagger}_{n} \mathfrak{b}_{0}^{2}(\theta^{\circ})$ and using \nref{EQD.35}


\begin{alignat*}{4}
& \E_{\theta^{\circ}}^{n}\left[\P_{M \vert Y^{n}}^{n, (\eta)}(\llbracket 1, G^{-}_{n} - 1 \rrbracket)\right]&& \leq && && G^{-}_{n} \exp \left[- \frac{\eta n}{2}\left( \mathfrak{b}_{0}^{2}(\theta^{\circ}) \Phi^{\dagger}_{n} \right)\right] + \notag\\
& && && && \sum\limits_{m = 1}^{G^{-}_{n} - 1} \P_{\theta^{\circ}}^{n}\left(\sup\limits_{x \in \mathds{B}_{m, m^{\dagger}_{n}}} \left\vert \left\langle x \vert \Pi_{m, m^{\dagger}_{n}} (\overline{\theta} - \theta^{\circ}) \right\rangle \right\vert_{l^{2}}^{2} > 6H^{2}\right).
\end{alignat*}


%\begin{alignat*}{4}
%& \E_{\theta^{\circ}}^{n}\left[\P_{M \vert Y^{n}}^{n, (\eta)}(\llbracket 1, G^{-}_{n} - 1) \rrbracket\right] && \leq && G^{-}_{n} \exp && \left[- \frac{ n \eta \mathfrak{b}_{0}^{2}(\theta^{\circ}) \Phi^{\dagger}_{n}}{2}\right] + \\
%& && && && 3 G^{-}_{n} \left(\exp\left[- \frac{(m^{\dagger}_{n})\psi_{n}}{800 s \Vert \theta^{\circ} \Vert_{l^{2}} \cdot \Vert \lambda \Vert_{l^{2}}}\right] + \exp\left[ \frac{-\sqrt{r \psi_{n}}}{100 \sqrt{2 s}}\right] + r\beta_{s}\right)\\
%& && \leq && 4 m^{\dagger}_{n} && \left(\exp\left[- \frac{(m^{\dagger}_{n})\psi_{n}}{800 s \Vert \theta^{\circ} \Vert_{l^{2}} \cdot \Vert \lambda \Vert_{l^{2}}}\right] + \exp\left[ \frac{-\sqrt{r \psi_{n}}}{100 \sqrt{2 s}}\right] + r\beta_{s}\right).
%\end{alignat*}

Consider \nref{EQD.33}.
As for all $m > G^{\dagger +}_{n}$, we have $m \Lambda_{(m)} \psi_{n} > \frac{173}{4} \mathfrak{b}_{0}^{2}(\theta^{\circ}) n \Phi^{\dagger}_{n}$, with a constant $C_{\theta^{\circ}, \lambda} > \sum\limits_{m = 1}^{\infty} \exp \left[- \eta \frac{m \Lambda_{(m)} \psi_{n}}{2}\right]$ and using \nref{EQD.37}

\begin{alignat*}{3}
& \E_{\theta^{\circ}}^{n}\left[\P_{M \vert Y^{n}}^{n, (\eta)}(\llbracket G^{+}_{n} + 1, n \rrbracket )\right] && \leq && \exp \left[- \eta \frac{\mathfrak{b}_{0}^{2}(\theta^{\circ}) n \Phi^{\dagger}_{n}}{2}\right] \sum\limits_{m = G^{+}_{n} + 1}^{n} \exp \left[- \eta \frac{m \Lambda_{(m)} \psi_{n}}{2}\right] +\\
& && && \sum\limits_{m = G^{+} + 1}^{n} \P_{\theta^{\circ}}^{n}\left(\sup\limits_{x \in \mathds{B}_{m^{\dagger}_{n}, m}} \left\vert \left\langle x \vert \Pi_{m^{\dagger}_{n}, m} (\overline{\theta} - \theta^{\circ}) \right\rangle \right\vert_{l^{2}}^{2} > 6H^{2}\right)\\
& && \leq && C_{\theta^{\circ}, \lambda} \exp \left[- \eta \frac{\mathfrak{b}_{0}^{2}(\theta^{\circ}) n \Phi^{\dagger}_{n}}{2}\right] +\\
& && && \sum\limits_{m = G^{+} + 1}^{n} \P_{\theta^{\circ}}^{n}\left(\sup\limits_{x \in \mathds{B}_{m^{\dagger}_{n}, m}} \left\vert \left\langle x \vert \Pi_{m^{\dagger}_{n}, m} (\overline{\theta} - \theta^{\circ}) \right\rangle \right\vert_{l^{2}}^{2} > 6H^{2}\right).
\end{alignat*}

%\begin{alignat*}{3}
%& \E_{\theta^{\circ}}^{n}\left[\P_{M \vert Y^{n}}^{n, (\eta)}(\llbracket G^{+}_{n} + 1 , n \rrbracket)\right] && \leq && \sum\limits_{m = G^{+}_{n} + 1}^{n}\left(\exp \left[- \frac{\psi_{n} \eta m \Lambda_{(m)}}{2} - \frac{ \mathfrak{b}_{0}^{2}(\theta^{\circ}) n \Phi^{\dagger}_{n}}{2}\right]\right) + \\
%& && && 3 \sum\limits_{m = G^{+}_{n} + 1}^{n} \left(\exp\left[- \frac{m\psi_{n}}{800 s \Vert \theta^{\circ} \Vert_{l^{2}} \cdot \Vert \lambda \Vert_{l^{2}}}\right] + \exp\left[ \frac{- \sqrt{ r \psi_{n}}}{\sqrt{2 s} 100}\right] + r\beta_{s}\right)\\
%& && \leq && \exp \left[- \frac{ \mathfrak{b}_{0}^{2}(\theta^{\circ}) n \Phi^{\dagger}_{n}}{2}\right]\underbrace{\sum\limits_{m = G^{+}_{n} + 1}^{n}\left(\exp \left[- \frac{\psi_{n} \eta m \Lambda_{(m)}}{2}\right]\right)}_{\leq 5 C_{\theta^{\circ} \lambda}/ 6} + \\
%& && && 3 \left(\exp\left[- \frac{G^{+}_{n}\psi_{n}}{800 s \Vert \theta^{\circ} \Vert_{l^{2}} \cdot \Vert \lambda \Vert_{l^{2}}}\right] + \exp\left[ \frac{- \sqrt{ r \psi_{n}}}{\sqrt{2 s} 100}\right]\right) \cdot \\
%& && &&\underbrace{\left(\sum\limits_{m = 1}^{n} \exp\left[- \frac{m\psi_{n}}{800 s \Vert \theta^{\circ} \Vert_{l^{2}} \cdot \Vert \lambda \Vert_{l^{2}}}\right] + \exp\left[ \frac{- \sqrt{ r \psi_{n}}}{\sqrt{2 s} 100}\right] \right)}_{\leq C_{\theta^{\circ} \lambda}/6} + n r\beta_{s}\\
%& && \leq && C_{\theta^{\circ}\lambda} \left(\exp\left[- \frac{m^{\dagger}_{n}\psi_{n}}{800 s \Vert \theta^{\circ} \Vert_{l^{2}} \cdot \Vert \lambda \Vert_{l^{2}}}\right] + \exp\left[ \frac{- \sqrt{ r \psi_{n}}}{\sqrt{2 s} 100}\right]\right) + n r \beta_{s}.
%\end{alignat*}

Which proves the statement.

%\begin{alignat*}{2}
%& \frac{2 \Lambda_{(m^{\dagger}_{n})}}{n} && \left(2(m^{\dagger}_{n}-m+1) \left\{1 + 2 \left[\gamma_{\theta^{\circ} \lambda}\frac{S}{\sqrt{2(m^{\dagger}_{n}-m+1)}} + 2 \sum\limits_{p = S+1}^{s-1} \beta(E^{\perp}_{1, 0},E^{\perp}_{1, p})\right]\right\}\right)\\
%& && \leq \frac{2 \Lambda_{(m^{\dagger}_{n})}}{n} \left(2(m^{\dagger}_{n} + 1) \left\{1 + 2 \left[\gamma_{\theta^{\circ} \lambda}\frac{S}{\sqrt{2(m^{\dagger}_{n}-G^{-}_{n}+1)}} + 2 \sum\limits_{p = S+1}^{s-1} \beta(E^{\perp}_{1, 0},E^{\perp}_{1, p})\right]\right\}\right)\\
%\end{alignat*}
%
%Take \textcolor{red}{$S = \left\lfloor\frac{\sqrt{2 \left(m^{\dagger}_{n} - G^{-}_{n} + 1\right)}}{\gamma_{\theta^{\circ} \lambda}}\right\rfloor \rightarrow \infty$} so for $n$ large enough, $\sum\limits_{p = S+1}^{s-1} \beta(E^{\perp}_{1, 0},E^{\perp}_{1, p}) < \frac{1}{8}$
%
%\begin{alignat*}{2}
%& \frac{2 \Lambda_{(m^{\dagger}_{n})}}{n} && \left(2(m^{\dagger}_{n}-m+1) \left\{1 + 2 \left[\gamma_{\theta^{\circ} \lambda}\frac{S}{\sqrt{2(m^{\dagger}_{n}-m+1)}} + 2 \sum\limits_{p = S+1}^{s-1} \beta(E^{\perp}_{1, 0},E^{\perp}_{1, p})\right]\right\}\right)\\
%& && \leq \frac{2 \Lambda_{(m^{\dagger}_{n})}}{n} \left(2(m^{\dagger}_{n} + 1) \left\{1 + 2 \left[\gamma_{\theta^{\circ} \lambda}\frac{S}{\sqrt{2(m^{\dagger}_{n}-G^{\dagger}_{-}+1)}} + \frac{1}{4}\right]\right\}\right)\\
%& && \leq \frac{14 (m^{\dagger}_{n} + 1)\Lambda_{(m^{\dagger}_{n})}}{n}
%\end{alignat*}
%
%\begin{alignat}{4}
%& \E_{\theta^{\circ}}^{n}\left[\P_{M \vert Y^{n}}^{n, (\eta)}(\llbracket 1, G^{-}_{n} - 1 \rrbracket)\right] && \leq && \sum\limits_{m = 1}^{G^{-}_{n} - 1} \exp&&\left[\eta\left((\pen(m^{\dagger}_{n})-\pen(m)) +\right.\right. \notag\\
%& && && &&\left.\left. \frac{n}{2} \left( \frac{21}{2} 3H^{2} - \frac{3}{4} \left( \mathfrak{b}_{m}^{2}\left(\theta^{\circ}\right) - \mathfrak{b}_{m^{\dagger}_{n}}^{2}\left(\theta^{\circ}\right) \right) \right) \right)\right] +\notag\\
%& && && && 3 \exp\left[- K n \left(\frac{\gamma^{2}}{v} \wedge \frac{\gamma}{h}\right)\right]\notag\\
%& && \leq &&\sum\limits_{m = 1}^{G^{-}_{n} - 1} \exp&&\left[\eta\left((\pen(m^{\dagger}_{n})-\pen(m)) +\right.\right. \notag\\
%& && && &&\left.\left. \frac{3 n}{4} \left( 21 \frac{14 \left(m^{\dagger}_{n} + 1\right) \Lambda_{(m^{\dagger}_{n})} \psi_{n}}{n} - \frac{1}{2} \left( \mathfrak{b}_{m}^{2}\left(\theta^{\circ}\right) - \mathfrak{b}_{m^{\dagger}_{n}}^{2}\left(\theta^{\circ}\right) \right) \right) \right)\right] +\notag\\
%& && && && 3 \exp\left[- K n \left(\frac{14 (m^{\dagger}_{n} + 1) \psi_{n}}{n \Vert \theta^{\circ} \Vert_{l^{2}} \cdot \Vert \lambda \Vert_{l^{2}}} \wedge \sqrt{\frac{14 (m^{\dagger}_{n} + 1)\psi_{n}}{n}}\right)\right]\notag\\
%\end{alignat}
%
%\begin{alignat*}{3}
%& \E_{\theta^{\circ}}^{n}\left[\P_{M \vert Y^{n}}^{n, (\eta)}\left(\llbracket 0, G^{\dagger -}_{n} - 1 \rrbracket\right)\right] && \leq && G^{\dagger -}_{n} \exp\left[- \frac{\mathfrak{b}_{0}^{2}(\theta^{\circ}) n \Phi^{\dagger}_{n}}{2} \right] + 3 G^{\dagger -}_{n} \exp\left[ -K \left(\frac{\psi_{n} 2 m^{\dagger}_{n}}{\Vert \lambda \Vert_{l^{2}} \Vert \theta^{\circ} \Vert_{l^{2}}} \wedge \sqrt{n \psi_{n}} \right)\right]\\
%& && \leq && 4 m^{\dagger}_{n} \exp\left[ -K \left(\frac{\psi_{n} 2 m^{\dagger}_{n}}{\Vert \lambda \Vert_{l^{2}} \Vert \theta^{\circ} \Vert_{l^{2}}} \wedge \sqrt{n \psi_{n}} \right)\right].
%\end{alignat*}
%
%Now for \nref{EQD.33}.
%As for all $m > G^{\dagger +}_{n}$, we have $\psi_{n} m \Lambda_{(m)} > \frac{25}{3} n \Phi^{\dagger}_{n} \mathfrak{b}_{0}^{2}$, using \nref{EQC.18}
%
%\begin{alignat*}{2}
%& \frac{2 \Lambda_{(m)}}{n} && \left(2(m-m^{\dagger}_{n}+1) \left\{1 + 2 \left[\gamma_{\theta^{\circ} \lambda}\frac{S}{\sqrt{2(m-m^{\dagger}_{n}+1)}} + 2 \sum\limits_{p = S+1}^{s-1} \beta(E^{\perp}_{1, 0},E^{\perp}_{1, p})\right]\right\}\right)\\
%& && \leq \frac{2 \Lambda_{(m)}}{n} \left(2(m + 1) \left\{1 + 2 \left[\gamma_{\theta^{\circ} \lambda}\frac{S}{\sqrt{2(G^{+}_{n}-m^{\dagger}_{n}+1)}} + 2 \sum\limits_{p = S+1}^{s-1} \beta(E^{\perp}_{1, 0},E^{\perp}_{1, p})\right]\right\}\right)\\
%\end{alignat*}
%
%Take \textcolor{red}{$S = \left\lfloor\frac{\sqrt{2 \left(G^{+}_{n} - m^{\dagger}_{n} + 1\right)}}{\gamma_{\theta^{\circ} \lambda}}\right\rfloor \rightarrow \infty$} so for $n$ large enough, $\sum\limits_{p = S+1}^{s-1} \beta(E^{\perp}_{1, 0},E^{\perp}_{1, p}) < \frac{1}{8}$
%
%\begin{alignat*}{2}
%&  \frac{2 \Lambda_{(m)}}{n} && \left(2(m-m^{\dagger}_{n}+1) \left\{1 + 2 \left[\gamma_{\theta^{\circ} \lambda}\frac{S}{\sqrt{2(m-m^{\dagger}_{n}+1)}} + 2 \sum\limits_{p = S+1}^{s-1} \beta(E^{\perp}_{1, 0},E^{\perp}_{1, p})\right]\right\}\right)\\
%& && \leq \frac{2 \Lambda_{(m)}}{n} \left(2(m + 1) \left\{1 + 2 \left[\gamma_{\theta^{\circ} \lambda}\frac{S}{\sqrt{2(G^{+}_{n}-m^{\dagger}_{n}+1)}} + \frac{1}{4}\right]\right\}\right)\\
%& && \leq \frac{14 (m + 1)\Lambda_{(m)}}{n}
%\end{alignat*}
%
%\begin{alignat*}{3}
%& \E_{\theta^{\circ}}^{n}&& &&\left[\P_{M \vert Y^{n}}^{n, (\eta)}\left(\llbracket G^{\dagger +}_{n} + 1, n \rrbracket\right)\right]\\
%& && \leq && \sum\limits_{G^{\dagger +} < m \leq n}\exp\left[\frac{n \eta}{2} \left(12 \mathfrak{b}_{0}^{2}(\theta^{\circ}) \Phi^{\dagger}_{n} - 2 \frac{m \Lambda_{(m)}}{n} \psi_{n} \right)\right] +\notag\\
%& && && 3 \sum\limits_{G^{\dagger +} < m \leq n} \exp\left[ -K \left(\frac{\psi_{n} 2 m}{\Vert \lambda \Vert_{l^{2}} \Vert \theta^{\circ} \Vert_{l^{2}}} \wedge \sqrt{n \psi_{n}} \right)\right]\\
%& && \leq && \sum\limits_{G^{\dagger +} < m \leq n} \exp\left[\eta\left(-\frac{\mathfrak{b}_{0}^{2}(\theta^{\circ}) n \Phi^{\dagger}_{n}}{2} - \frac{\psi_{n} m \Lambda_{(m)}}{2}\right)\right] +\notag\\
%& && && \exp\left[ -K \left(\frac{\psi_{n} 2 G^{+}_{n}}{\Vert \lambda \Vert_{l^{2}} \Vert \theta^{\circ} \Vert_{l^{2}}} \wedge \sqrt{n \psi_{n}} \right)\right]\cdot \sum\limits_{G^{\dagger +} < m \leq n} \frac{\exp\left[ -K \left(\frac{\psi_{n} 2 m}{\Vert \lambda \Vert_{l^{2}} \Vert \theta^{\circ} \Vert_{l^{2}}} \wedge \sqrt{n \psi_{n}} \right)\right]}{\exp\left[ -K \left(\frac{\psi_{n} 2 G^{+}_{n}}{\Vert \lambda \Vert_{l^{2}} \Vert \theta^{\circ} \Vert_{l^{2}}} \wedge \sqrt{n \psi_{n}} \right)\right]}\\
%& && \leq && \exp\left[-\frac{\eta \mathfrak{b}_{0}^{2}(\theta^{\circ}) n \Phi^{\dagger}_{n}}{2}\right]\sum\limits_{1 < m \leq n} \exp\left[- \frac{\eta \psi_{n} m \Lambda_{(m)}}{2}\right] +\\
%& && && \exp\left[ -K \left(\frac{\psi_{n} 2 G^{+}_{n}}{\Vert \lambda \Vert_{l^{2}} \Vert \theta^{\circ} \Vert_{l^{2}}} \wedge \sqrt{n \psi_{n}} \right)\right]\cdot \sum\limits_{1 < m \leq n} \exp\left[ -K \left(\frac{\psi_{n} 2 m}{\Vert \lambda \Vert_{l^{2}} \Vert \theta^{\circ} \Vert_{l^{2}}} \wedge \sqrt{n \psi_{n}} \right)\right]\\
%& && \leq &&  C_{\lambda, \theta^{\circ}} \exp\left[ -K \left(\frac{\psi_{n} 2 G^{\dagger +}_{n} }{\Vert \lambda \Vert_{l^{2}} \Vert \theta^{\circ} \Vert_{l^{2}}} \wedge \sqrt{n \psi_{n}} \right)\right].
%\end{alignat*}
%
%Which completes the proof.

\qedsymbol
\end{pro}


\begin{pro}{Proof of \nref{PRD.5.3}\\}\label{PROD.5.3}

In this part, let $m$ and $l$ be two positive integers with $m < l$.

We have, for any $t$ in $\mathds{B}_{m, l}$
\begin{alignat*}{3}
&\left\langle t \vert \overline{\theta} \right\rangle_{l^{2}} && = && \frac{1}{n} \sum\limits_{p = 1}^{n} \sum\limits_{m \leq \vert j \vert \leq l} \left(\frac{t_{j}}{\overline{\lambda}_{j}} \cdot e_{j}\left(- Y_{p}^{n}\right)\right)\\
& && = && \frac{1}{n} \sum\limits_{p = 1}^{n} \mathcal{F}^{-1}\left(\frac{t}{\overline{\lambda}}\right)\left(- Y_{p}^{n}\right).
\end{alignat*}
So we define for any $t$ in $\mathds{B}_{m, l}$ the functional $\nu_{t} := \sum\limits_{m \leq \vert j \vert \leq l} \left(\frac{t_{j}}{\overline{\lambda}_{j}}\right) e_{j} = \mathcal{F}^{-1}\left(\frac{t}{\overline{\lambda}}\right)$ and we obtain
\[\overline{\nu}_{t} := \left\langle t \vert \overline{\theta} - \theta^{\circ} \right\rangle_{l^{2}} = \frac{1}{n} \sum\limits_{p = 1}^{n} \left(\nu_{t}\left(Y_{p}\right) - \E_{\theta^{\circ}}^{n}\left[\nu_{t}\left(Y_{p}^{n}\right)\right]\right)\]

\medskip

Then, for any $t$ in $\mathds{B}_{m, l}$ and $x$ in $\mathcal{L}^{2}$ we define $v_{t}(x) = \frac{1}{s} \sum\limits_{p = 1}^{s} \nu_{t}(x_{p})$, so we can write $\frac{1}{n} \sum\limits_{p = 1}^{n} \nu_{t}(Y_{p}^{n}) = \frac{1}{2} \left\{ \frac{1}{r} \sum\limits_{q = 1}^{r} v_{t}(E_{q}) + \frac{1}{r} \sum\limits_{q = 1}^{r} v_{t}(O_{q}) \right\}$, which gives
\begin{alignat*}{3}
& \left\langle t \vert \overline{\theta} - \theta^{\circ}\right\rangle && = && \frac{1}{n} \sum\limits_{p = 1}^{n} \left(\nu_{t}(Y_{p}^{n}) - \E_{\theta^{\circ}}^{n}\left[\nu_{t}(Y_{p}^{n})\right]\right)\\
& && = && \frac{1}{2} \left(\underbrace{\frac{1}{r} \sum\limits_{q = 1}^{r} \left(v_{t}(E_{q})- \E_{\theta^{\circ}}^{n}\left[v_{t}(E_{q})\right]\right)}_{=: \overline{\nu}_{t}^{e}} + \underbrace{\frac{1}{r} \sum\limits_{q = 1}^{r} \left(v_{t}(O_{q})- \E_{\theta^{\circ}}^{n}\left[v_{t}(O_{q})\right]\right)}_{=: \overline{\nu}_{t}^{o}}\right)
\end{alignat*}
Similarly, we define for any $t$ in $\mathds{B}_{m, l}$ the quantities $\overline{\nu}_{t}^{e, \perp} := \frac{1}{r} \sum\limits_{q = 1}^{r} \left(v_{t}(E_{q}^{\perp})- \E_{\theta^{\circ}}^{n}\left[v_{t}(E_{q}^{\perp})\right]\right)$ and $\overline{\nu}_{t}^{o, \perp} := \frac{1}{r} \sum\limits_{q = 1}^{r} \left(v_{t}(O_{q}^{\perp})- \E_{\theta^{\circ}}^{n}\left[v_{t}(O_{q}^{\perp})\right]\right)$ which combined give $\overline{\nu}_{t}^{\perp} := \frac{1}{2} \left(\overline{\nu}_{t}^{e, \perp} + \overline{\nu}_{t}^{o, \perp}\right)$.


Consider first \nref{EQD.26}.

\begin{alignat*}{3}
& \E_{\theta^{\circ}}^{n} && && \left[ \left(\sup\limits_{t \in \mathds{B}_{m, l}} \left\vert\left\langle t \vert \overline{\theta} - \theta^{\circ}\right\rangle_{l^{2}} \right\vert^{2} - C_{n} \right)_{+} \right] = \E_{\theta^{\circ}}^{n}\left[ \left(\sup\limits_{t \in \mathds{B}_{m, l}} \vert \overline{\nu}_{t} \vert^{2} - C_{n} \right)_{+} \right]\notag\\
& && \leq && \E_{\theta^{\circ}}^{n}\left[ \left(\sup\limits_{t \in \mathds{B}_{m, l}} \vert \overline{\nu}^{e, \perp}_{t} \vert^{2} - C_{n} \right)_{+} \right] + \E_{\theta^{\circ}}^{n}\left[ \sup\limits_{t \in \mathds{B}_{m, l}} \vert \overline{\nu}^{e, \perp}_{t} - \overline{\nu}^{e}_{t} \vert^{2} \right] + \notag\\
& && &&\E_{\theta^{\circ}}^{n}\left[ \left(\sup\limits_{t \in \mathds{B}_{m, l}} \vert \overline{\nu}^{o, \perp}_{t} \vert^{2} - C_{n} \right)_{+} \right] + \E_{\theta^{\circ}}^{n}\left[ \sup\limits_{t \in \mathds{B}_{m, l}} \vert \overline{\nu}^{o, \perp}_{t} - \overline{\nu}^{o}_{t} \vert^{2} \right] \notag\\
& && \leq && 2 \cdot \E_{\theta^{\circ}}^{n}\left[ \left(\sup\limits_{t \in \mathds{B}_{m, l}} \vert \overline{\nu}^{e, \perp}_{t} \vert^{2} - C_{n} \right)_{+} \right] +\\
& && && 2 \cdot \E_{\theta^{\circ}}^{n}\left[ \sup\limits_{t \in \mathds{B}_{m, l}} \vert \overline{\nu}^{e, \perp}_{t} - \overline{\nu}^{e}_{t} \vert^{2} \right]
\end{alignat*}
Which proves the statement.

Consider now \nref{EQD.27}.
\begin{alignat*}{3}
& \P_{\theta^{\circ}}^{n}\left(\sup\limits_{t \in \mathds{B}_{m, l}} \left\vert \left\langle t \vert \overline{\theta} - \theta^{\circ}\right\rangle_{l^{2}} \right\vert \geq C_{n}'\right) && = && \P_{\theta^{\circ}}^{n}\left(\sup\limits_{t \in \mathds{B}_{m, l}} \left\vert \overline{\nu}_{t} \right\vert \geq C_{n}'\right)\notag\\
& && = && \P_{\theta^{\circ}}^{n}\left(\sup\limits_{t \in \mathds{B}_{m, l}} \left\vert \frac{1}{2} \left(\overline{\nu}_{t}^{e} - \overline{\nu}_{t}^{e, \perp} + \overline{\nu}_{t}^{e, \perp} + \overline{\nu}_{t}^{o} - \overline{\nu}_{t}^{o, \perp} + \overline{\nu}_{t}^{o, \perp}\right) \right\vert \geq C_{n}'\right)\notag\\
& && = && \P_{\theta^{\circ}}^{n}\left(\left\{\sup\limits_{t \in \mathds{B}_{m, l}} \left\vert \frac{1}{2} \left(\overline{\nu}_{t}^{e} - \overline{\nu}_{t}^{e, \perp} + \overline{\nu}_{t}^{e, \perp} + \overline{\nu}_{t}^{o} - \overline{\nu}_{t}^{o, \perp} + \overline{\nu}_{t}^{o, \perp} \right) \right\vert \geq C_{n}'\right\} \cap \right.\notag\\
& && && \left.\left\{ \overline{\nu}_{t}^{e} = \overline{\nu}_{t}^{e, \perp}, \overline{\nu}_{t}^{o} = \overline{\nu}_{t}^{o, \perp} \right\}\right) + \notag\\
& && && \P_{\theta^{\circ}}^{n}\left(\left\{\sup\limits_{t \in \mathds{B}_{m, l}} \left\vert \frac{1}{2} \left(\overline{\nu}_{t}^{e} - \overline{\nu}_{t}^{e, \perp} + \overline{\nu}_{t}^{e, \perp} + \overline{\nu}_{t}^{o} - \overline{\nu}_{t}^{o, \perp} + \overline{\nu}_{t}^{o, \perp}\right) \right\vert \geq C_{n}'\right\} \right.\notag\\
& && && \left.\cap \left\{ \overline{\nu}_{t}^{e} \neq \overline{\nu}_{t}^{e, \perp}, \overline{\nu}_{t}^{o} \neq \overline{\nu}_{t}^{o, \perp} \right\}\right) +\notag\\
& && && \P_{\theta^{\circ}}^{n}\left(\left\{\sup\limits_{t \in \mathds{B}_{m, l}} \left\vert \frac{1}{2} \left(\overline{\nu}_{t}^{e} - \overline{\nu}_{t}^{e, \perp} + \overline{\nu}_{t}^{e, \perp} + \overline{\nu}_{t}^{o} - \overline{\nu}_{t}^{o, \perp} + \overline{\nu}_{t}^{o, \perp}\right) \right\vert \geq C_{n}'\right\} \right.\notag\\
& && && \left.\cap \left\{ \overline{\nu}_{t}^{e} \neq \overline{\nu}_{t}^{e, \perp}, \overline{\nu}_{t}^{o} = \overline{\nu}_{t}^{o, \perp} \right\}\right) +\notag\\
& && && \P_{\theta^{\circ}}^{n}\left(\left\{\sup\limits_{t \in \mathds{B}_{m, l}} \left\vert \frac{1}{2} \left(\overline{\nu}_{t}^{e} - \overline{\nu}_{t}^{e, \perp} + \overline{\nu}_{t}^{e, \perp} + \overline{\nu}_{t}^{o} - \overline{\nu}_{t}^{o, \perp} + \overline{\nu}_{t}^{o, \perp}\right) \right\vert \geq C_{n}'\right\} \right.\notag\\
& && && \left.\cap \left\{ \overline{\nu}_{t}^{e} = \overline{\nu}_{t}^{e, \perp}, \overline{\nu}_{t}^{o} \neq \overline{\nu}_{t}^{o, \perp} \right\}\right) +\notag\\
& && \leq && \P_{\theta^{\circ}}^{n}\left(\sup\limits_{t \in \mathds{B}_{m, l}} \left\vert \frac{1}{2} \left(\overline{\nu}_{t}^{e, \perp} + \overline{\nu}_{t}^{o, \perp} \right) \right\vert \geq C_{n}'\right) + \P_{\theta^{\circ}}^{n}\left(\overline{\nu}_{t}^{e} \neq \overline{\nu}_{t}^{e, \perp}, \overline{\nu}_{t}^{o} \neq \overline{\nu}_{t}^{o, \perp}\right) +\notag\\
& && && \P_{\theta^{\circ}}^{n}\left(\overline{\nu}_{t}^{e} = \overline{\nu}_{t}^{e, \perp}, \overline{\nu}_{t}^{o} \neq \overline{\nu}_{t}^{o, \perp}\right) +\notag\\
& && && \P_{\theta^{\circ}}^{n}\left(\overline{\nu}_{t}^{e} \neq \overline{\nu}_{t}^{e, \perp}, \overline{\nu}_{t}^{o} = \overline{\nu}_{t}^{o, \perp}\right)\notag\\
& && \leq && \P_{\theta^{\circ}}^{n}\left(\sup\limits_{t \in \mathds{B}_{m, l}} \left\vert \max\left\{ \overline{\nu}_{t}^{e, \perp}, \overline{\nu}_{t}^{o, \perp} \right\} \right\vert \geq C_{n}'\right) + \notag\\
& && &&\min\left\{\P_{\theta^{\circ}}^{n}\left(\overline{\nu}_{t}^{e} \neq \overline{\nu}_{t}^{e, \perp}\right), \P\left(\overline{\nu}_{t}^{o} \neq \overline{\nu}_{t}^{o, \perp}\right)\right\} +\notag\\
& && &&\min\left\{\P_{\theta^{\circ}}^{n}\left(\overline{\nu}_{t}^{e} = \overline{\nu}_{t}^{e, \perp}\right), \P\left(\overline{\nu}_{t}^{o} \neq \overline{\nu}_{t}^{o, \perp}\right)\right\} +\notag\\
& && &&\min\left\{\P_{\theta^{\circ}}^{n}\left(\overline{\nu}_{t}^{e} \neq \overline{\nu}_{t}^{e, \perp}\right), \P\left(\overline{\nu}_{t}^{o} = \overline{\nu}_{t}^{o, \perp}\right)\right\}\notag\\
& && \leq && \P_{\theta^{\circ}}^{n}\left(\sup\limits_{t \in \mathds{B}_{m, l}} \left\vert \max\left\{ \overline{\nu}_{t}^{e, \perp}, \overline{\nu}_{t}^{o, \perp} \right\} \right\vert \geq C_{n}' \vert \overline{\nu}_{t}^{e, \perp}\geq \overline{\nu}_{t}^{o, \perp} \right)\cdot \P\left(\overline{\nu}_{t}^{e, \perp}\geq \overline{\nu}_{t}^{o, \perp}\right) +\notag\\
& && && \P_{\theta^{\circ}}^{n}\left(\sup\limits_{t \in \mathds{B}_{m, l}} \left\vert \max\left\{ \overline{\nu}_{t}^{e, \perp}, \overline{\nu}_{t}^{o, \perp} \right\} \right\vert \geq C_{n}' \vert \overline{\nu}_{t}^{e, \perp}\leq \overline{\nu}_{t}^{o, \perp} \right)\cdot \P\left(\overline{\nu}_{t}^{e, \perp}\leq \overline{\nu}_{t}^{o, \perp}\right) + \notag\\
& && && 3 \P_{\theta^{\circ}}^{n}\left(\overline{\nu}_{t}^{e} \neq \overline{\nu}_{t}^{e, \perp}\right)
\end{alignat*}

\begin{alignat*}{3}
& && \leq && \P_{\theta^{\circ}}^{n}\left(\sup\limits_{t \in \mathds{B}_{m, l}} \left\vert \overline{\nu}_{t}^{e, \perp} \right\vert \geq C_{n}' \right)\cdot \P\left(\overline{\nu}_{t}^{e, \perp}\geq \overline{\nu}_{t}^{o, \perp}\right) +\notag\\
& && && \P_{\theta^{\circ}}^{n}\left(\sup\limits_{t \in \mathds{B}_{m, l}} \left\vert \overline{\nu}_{t}^{o, \perp}  \right\vert \geq C_{n}' \right)\cdot \P\left(\overline{\nu}_{t}^{e, \perp}\leq \overline{\nu}_{t}^{o, \perp}\right) + \notag\\
& && && 3 \P_{\theta^{\circ}}^{n}\left(\overline{\nu}_{t}^{e} \neq \overline{\nu}_{t}^{e, \perp}\right)\notag\\
& && \leq && \P_{\theta^{\circ}}^{n}\left(\sup\limits_{t \in \mathds{B}_{m, l}} \left\vert \overline{\nu}_{t}^{e, \perp} \right\vert \geq C_{n}' \right) + 3 \P_{\theta^{\circ}}^{n}\left(\overline{\nu}_{t}^{e} \neq \overline{\nu}_{t}^{e, \perp}\right)
\end{alignat*}
Which completes the proof.

\qedsymbol
\end{pro}



\begin{pro}{\textsc{Proof of \nref{PRD.5.4}}\\}\label{PROD.5.4}

We will use Talagrand's inequality with, for any $t$ in $\mathds{B}_{m, l}$
\begin{alignat*}{3}
& \overline{\nu}^{e, \perp}_{t} && = && \frac{1}{r} \sum\limits_{q = 1}^{r} \left(v_{t}(E^{\perp}_{q}) - \E_{\theta^{\circ}}^{n}\left[v_{t}(E^{\perp}_{q})\right]\right)\\
& v_{t}(E^{\perp}_{q}) && = && \frac{1}{s} \sum\limits_{p = 1}^{s} \nu_{t}(E^{\perp}_{q, p})\\
& \nu_{t}(E^{\perp}_{q, p}) && = && \sum\limits_{m \leq \vert j \vert \leq l} \left(\frac{t_{j}}{\overline{\lambda_{j}}} e_{j}(E^{\perp}_{q, p})\right)
\end{alignat*}

\begin{alignat*}{3}
& \sup\limits_{t \in \mathds{B}_{m,l}} \sup\limits_{x \in [0, 1]^{s}} \vert v_{t}(x) \vert^{2} && = && \sup\limits_{t \in \mathds{B}_{m,l}} \sup\limits_{x \in [0, 1]^{s}} \left\vert \frac{1}{s} \sum\limits_{p = 1}^{s} \nu_{t}(x_{p}) \right\vert^{2}\\
& && = && \sup\limits_{t \in \mathds{B}_{m,l}} \sup\limits_{x \in [0, 1]^{s}} \left\vert \frac{1}{s} \sum\limits_{p = 1}^{s} \sum\limits_{m \leq \vert j \vert \leq l} \left(\frac{t_{j}}{\overline{\lambda_{j}}} e_{j}(x_{p})\right) \right\vert^{2}\\
& && \leq && \sup\limits_{t \in \mathds{B}_{m,l}} \sup\limits_{x \in [0, 1]^{s}} \frac{1}{s^{2}}  \sum\limits_{m \leq \vert j \vert \leq l} \left\vert t_{j} \right\vert^{2} \Lambda_{j} \underbrace{\left\vert \sum\limits_{p = 1}^{s} e_{j}(x_{p}) \right\vert^{2}}_{\leq s^{2}}\\
& && \leq && \sup\limits_{t \in \mathds{B}_{m,l}} \sum\limits_{m \leq \vert j \vert \leq l} \left\vert t_{j} \right\vert^{2} \Lambda_{j}\\
& && \leq && \sum\limits_{m \leq \vert j \vert \leq l} \Lambda_{j}.
\end{alignat*}

Hence we define \textcolor{red}{$h^{2} := \delta^{\star}_{m, l} \geq \sum\limits_{m \leq \vert j \vert \leq l} \Lambda_{j}$.}

\medskip

To define $H^{2}$, we define the following objects: $\overline{e}_{j}(E_{q}^{\perp}) := \frac{1}{s} \sum\limits_{p = 1}^{s} e_{j}(E^{\perp}_{q, p})$ and $\overline{e}(E_{q}^{\perp}) = \left(\overline{e}_{j}(E_{q}^{\perp})\right)_{j \in \mathds{Z}}$.

We use \nref{lmB.0.2} in the last line to conclude:

\begin{alignat*}{4}
& \E_{\theta^{\circ}}&&\left[\sup\limits_{t \in \mathds{B}_{m,l}} \left\vert \overline{\nu}^{e, \perp}_{t}\right\vert^{2}\right] && = && \E_{\theta^{\circ}}\left[\sup\limits_{t \in \mathds{B}_{m,l}} \left\vert \frac{1}{r} \sum\limits_{q = 1}^{r} \left(v_{t}(E^{\perp}_{q}) - \E_{\theta^{\circ}}^{n}\left[v_{t}(E^{\perp}_{q})\right]\right)\right\vert^{2}\right]\\
& && = && \E_{\theta^{\circ}}&&\left[\sup\limits_{t \in \mathds{B}_{m,l}} \left\vert \frac{1}{r} \sum\limits_{q = 1}^{r} \left(\frac{1}{s} \sum\limits_{p = 1}^{s} \nu_{t}(E^{\perp}_{q, p}) - \E_{\theta^{\circ}}^{n}\left[\frac{1}{s} \sum\limits_{p = 1}^{s} \nu_{t}(E^{\perp}_{q, p})\right]\right)\right\vert^{2}\right]\\
& && = && \E_{\theta^{\circ}}&&\left[\sup\limits_{t \in \mathds{B}_{m,l}} \left\vert \frac{1}{r} \sum\limits_{q = 1}^{r} \left(\frac{1}{s} \sum\limits_{p = 1}^{s} \sum\limits_{m \leq \vert j \vert \leq l} \left(\frac{t_{j}}{\overline{\lambda_{j}}} e_{j}(E^{\perp}_{q, p})\right) \right.\right.\right. \\
& && && && \left.\left.\left.- \E_{\theta^{\circ}}^{n}\left[\frac{1}{s} \sum\limits_{p = 1}^{s} \sum\limits_{m \leq \vert j \vert \leq l} \left(\frac{t_{j}}{\overline{\lambda_{j}}} e_{j}(E^{\perp}_{q, p})\right)\right]\right)\right\vert^{2}\right]\\
& && = && \E_{\theta^{\circ}}&&\left[\sup\limits_{t \in \mathds{B}_{m,l}} \frac{1}{r^{2}}\left\vert \sum\limits_{q = 1}^{r} \left( \sum\limits_{m \leq \vert j \vert \leq l}\frac{t_{j}}{\overline{\lambda_{j}}} \sum\limits_{p = 1}^{s}\frac{1}{s} e_{j}(E^{\perp}_{q, p}) \right.\right.\right. \\
& && && && \left.\left.\left.- \E_{\theta^{\circ}}^{n}\left[ \sum\limits_{m \leq \vert j \vert \leq l} \frac{t_{j}}{\overline{\lambda_{j}}} \sum\limits_{p = 1}^{s} \frac{1}{s} e_{j}(E^{\perp}_{q, p})\right]\right)\right\vert^{2}\right]\\
& && = && \E_{\theta^{\circ}}&&\left[\sup\limits_{t \in \mathds{B}_{m,l}} \frac{1}{r^{2}}\left\vert \sum\limits_{q = 1}^{r} \left\langle t \vert \frac{\overline{e}(-E^{\perp}_{q})}{\lambda} - \overline{\theta^{\circ}} \right\rangle \right\vert^{2}\right]\\
& && \leq && \E_{\theta^{\circ}}&&\left[\sup\limits_{t \in \mathds{B}_{m,l}} \frac{1}{r^{2}}\left\Vert t \right\Vert_{l^{2}}^{2} \sum\limits_{q = 1}^{r} \left\Vert \frac{\overline{e}(-E^{\perp}_{q})}{\lambda} - \overline{\theta^{\circ}} \right\Vert_{l^{2}}^{2}\right]\\
& && \leq && \E_{\theta^{\circ}}&&\left[\frac{1}{r^{2}} \sum\limits_{q = 1}^{r} \left\Vert \Pi_{m, l}\left(\frac{\overline{e}(-E^{\perp}_{q})}{\lambda} - \overline{\theta^{\circ}} \right)\right\Vert_{l^{2}}^{2}\right]\\
& && \leq && \frac{1}{r}\E_{\theta^{\circ}}&&\left[ \left\Vert \Pi_{m, l}\left(\frac{\overline{e}(-E^{\perp}_{1})}{\lambda} - \overline{\theta^{\circ}}\right) \right\Vert_{l^{2}}^{2}\right]\\
& && \leq && \frac{1}{r}\E_{\theta^{\circ}}&&\left[ \sum\limits_{m \leq \vert j \vert \leq l} \left\vert \frac{\overline{e}_{j}(E^{\perp}_{1})}{\lambda_{j}} - \theta^{\circ}_{j} \right\vert^{2} \right]\\
& && \leq && \frac{1}{r} \sum\limits_{m \leq \vert j \vert \leq l} \E_{\theta^{\circ}}&&\left[ \left\vert \frac{\overline{e}_{j}(E^{\perp}_{1})}{\lambda_{j}} - \theta^{\circ}_{j} \right\vert^{2} \right]\\
& && \leq && \frac{1}{r} \sum\limits_{m \leq \vert j \vert \leq l}\left( \frac{\Lambda_{j}}{s^{2}}\V_{\theta^{\circ}}\right. && \left.\left[ \sum\limits_{p = 1}^{s} e_{j}(E^{\perp}_{1, p}) \right] + \left\vert \frac{1}{\lambda_{j}} \E_{\theta^{\circ}}\left[ e_{j}(E^{\perp}_{1, 1})\right] - \theta^{\circ}_{j} \right\vert^{2}\right)\\
& && \leq && \frac{\Lambda_{(l)}}{r s} &&\left(2(l-m+1) \left\{1 + 2 \left[\gamma_{\theta^{\circ} \lambda}\frac{S}{\sqrt{2(l-m+1)}} +\right.\right.\right. \\
& && && && \left.\left.\left. 2 \sum\limits_{p = S+1}^{s-1} \beta(E^{\perp}_{1, 0},E^{\perp}_{1, K})\right]\right\}\right)
\end{alignat*}

In particular, we set \textcolor{red}{$S := \left\lfloor \frac{\psi_{n} \sqrt{2 \left(l - m + 1\right)}}{\gamma_{\theta^{\circ} \lambda}} \right\rfloor \leq s-1$} so it tends to infinity and $\sum\limits_{p = S+1}^{s-1} \beta(E^{\perp}_{1, 0},E^{\perp}_{1, K}) \leq \frac{1}{8}$ for $n \geq n^{\circ}$ and 
\begin{alignat*}{3}
& \E_{\theta^{\circ}}\left[\sup\limits_{t \in \mathds{B}_{m,l}} \left\vert \overline{\nu}^{e, \perp}_{t}\right\vert^{2}\right] && \leq && \frac{6 \Lambda_{(l)}(l-m+1)( \psi_{n} + 1)}{n}
\end{alignat*}

\textcolor{red}{So we set $H^{2} \geq \frac{6 \Lambda_{(l)}(l-m+1)( \psi_{n} + 1)}{n}$.}


Finally we control $v$ in the same way as in \nref{PROD.3.1} and hence \textcolor{red}{$v \geq \Lambda_{(l)} \Vert \theta^{\circ} \Vert_{l^{2}} \cdot \Vert \lambda \Vert_{l^{2}}$.}

Using Talagrand's inequality gives us the result.

\qedsymbol
\end{pro}



\begin{pro}{\textsc{Proof of \nref{PRD.5.5}} \\}\label{PROD.5.5}
Both inequalities are verified using $\P_{\theta^{\circ}}^{n}\left(E_{q} \neq E^{\perp}_{q}\right) \leq \beta_{s}$ and, as it was proven in \nref{PROD.5.2}, $\Vert v_{t} \Vert_{\infty}^{2} \leq \sum\limits_{m \leq \vert j \vert \leq l}\Lambda_{j} \leq h^{2}$.

\medskip

Consider first \nref{EQD.30}.
\begin{alignat*}{3}
& \E_{\theta^{\circ}}^{n}\left[\sup\limits_{t \in \mathds{B}_{m, l}} \left\vert \overline{\nu}^{e, \perp}_{t} - \overline{\nu}^{e}_{t} \right\vert^{2} \right] && = && \frac{1}{r^{2}}\E_{\theta^{\circ}}^{n}\left[\sup\limits_{t \in \mathds{B}_{m, l}} \left\vert \sum\limits_{q = 1}^{r} v_{t}(E^{\perp}_{q}) - v_{t}(E_{q})  \right\vert^{2} \right]\\
& && \leq && 4 \sup\limits_{t \in\mathds{B}_{m, l}} \Vert v_{t} \Vert_{\infty}^{2} \beta_{s}\\
& && \leq && 4 \beta_{s} \sum\limits_{m \leq \vert j \vert \leq l} \Lambda_{j}
\end{alignat*}

Consider now \nref{EQD.31}
\begin{alignat*}{3}
& \P_{\theta^{\circ}}^{n}\left(\overline{\nu}_{t}^{e} \neq \overline{\nu}_{t}^{e, \perp}\right) && \leq && \P_{\theta^{\circ}}^{n}\left(\bigcup\limits_{q = 1}^{r} E_{q}^{\perp} \neq E_{q}\right)\\
& && \leq && r \beta_{s}
\end{alignat*}
Which completes the proof.

\qedsymbol
\end{pro}

\begin{pro}{\textsc{Proof of \nref{THM3.5.1}} \\}\label{PROD.5.7}
We use \nref{EQD.35} and \nref{EQD.33} in \nref{EQD.31} to obtain for any $m$ and $l$ such that $m < l$

\begin{alignat*}{3}
& \P_{\theta^{\circ}}^{n}\left(\sup\limits_{t \in \mathds{B}_{m, l}} \left\vert \left\langle t \vert \overline{\theta} - \theta^{\circ}\right\rangle_{l^{2}} \right\vert^{2} \geq H^{2} \right) && \leq && 3 \left(\exp\left[- \frac{r H^{2}}{400 v}\right] + \exp\left[ \frac{-r H}{100 h}\right]\right) + 3 r\beta_{s}.
\end{alignat*}

We inject this result in \nref{EQD.28} with $H^{2} = 2 \frac{\Lambda_{(m^{\dagger}_{n})} m^{\dagger}_{n} \psi_{n}}{n}$, $h^{2} = \Lambda_{(m^{\dagger}_{n})}$ and $v = \Lambda_{(m^{\dagger}_{n}) \Vert \theta^{\circ} \Vert_{l^{2}} \cdot \Vert \lambda \Vert_{l^{2}}}$ to obtain

\begin{alignat}{3}
& \E_{\theta^{\circ}}^{n}\left[\P_{M \vert Y^{n}}^{n, (\eta)}\left(\left\llbracket 1, G^{-}_{n} - 1 \right\rrbracket\right)\right] && \leq && G^{-}_{n}\exp\left[\frac{- \eta n \mathfrak{b}_{0}^{2}(\theta^{\circ}) \Phi^{\dagger}_{n}}{2}\right] +\notag\\
& && && 3 \sum\limits_{1 \leq \vert j \vert < G^{-}_{n}}\left(\exp\left[- \frac{m^{\dagger}_{n} \psi_{n}}{400 s \Vert \theta^{\circ} \Vert_{l^{2}} \cdot \Vert \lambda \Vert_{l^{2}}}\right] + \exp\left[ \frac{- \sqrt{r m^{\dagger}_{n} \psi_{n}}}{100\sqrt{s}}\right] + r\beta_{s}\right)\label{EQD.42}
\end{alignat}

and in similarly in \nref{EQD.29} with $H^{2} = 2 \frac{\Lambda_{(m)} m \psi_{n}}{n}$, $h^{2} = \Lambda_{(m)}$ and $v = \Lambda_{(m) \Vert \theta^{\circ} \Vert_{l^{2}} \cdot \Vert \lambda \Vert_{l^{2}}}$ and fixing $C_{\theta^{\circ}, \lambda} \geq  \sum\limits_{1 < \vert j \vert \leq \infty} \left(\exp\left[- \frac{m \psi_{n}}{400 s \Vert \theta^{\circ} \Vert_{l^{2}} \cdot \Vert \lambda \Vert_{l^{2}}}\right] + \exp\left[ \frac{- \sqrt{r m \psi_{n}}}{100 \sqrt{s}}\right]\right)$ to obtain

\begin{alignat}{3}
& \E_{\theta^{\circ}}^{n}\left[\P_{M \vert Y^{n}}^{n, (\eta)}(\llbracket G^{+}_{n} + 1, n \rrbracket)\right] && \leq && C_{\theta^{\circ} \lambda} \exp \left[-\frac{\eta \mathfrak{b}_{0}^{2}(\theta^{\circ}) n \Phi^{\dagger}_{n}}{2}\right] +\notag\\
& && && 3 \left(\exp\left[- \frac{G^{+}_{n} \psi_{n}}{400 s \Vert \theta^{\circ} \Vert_{l^{2}} \cdot \Vert \lambda \Vert_{l^{2}}}\right] + \exp\left[ \frac{- \sqrt{r G^{+}_{n} \psi_{n}}}{100 \sqrt{s}}\right] + G^{+}_{n} r\beta_{s}\right) \cdot\notag\\
& && && \sum\limits_{1 < \vert j \vert \leq \infty} \left(\exp\left[- \frac{m \psi_{n}}{400 s \Vert \theta^{\circ} \Vert_{l^{2}} \cdot \Vert \lambda \Vert_{l^{2}}}\right] + \exp\left[ \frac{- \sqrt{r m \psi_{n}}}{100 \sqrt{s}}\right]\right)\notag\\
& && \leq && C_{\theta^{\circ} \lambda} \exp \left[-\frac{\eta \mathfrak{b}_{0}^{2}(\theta^{\circ}) n \Phi^{\dagger}_{n}}{2}\right] +\notag\\
& && && 3 C_{\theta^{\circ}, \lambda} \left(\exp\left[- \frac{G^{+}_{n} \psi_{n}}{400 s \Vert \theta^{\circ} \Vert_{l^{2}} \cdot \Vert \lambda \Vert_{l^{2}}}\right] + \exp\left[ \frac{- \sqrt{r G^{+}_{n} \psi_{n}}}{100 \sqrt{s}}\right] + G^{+}_{n} r\beta_{s}\right)\label{EQD.43}
\end{alignat}

We use first \nref{EQD.42} in \nref{EQD.27} to obtain

\begin{alignat}{3}
& \E_{\theta^{\circ}}^{n}&& &&\left[\P_{M \vert Y^{n}}^{n, (\eta)} \left(\left\llbracket 1, G^{-}_{n} - 1\right\rrbracket \right) \right] \left\Vert \theta^{\circ} \right\Vert_{l^{2}}^{2} + 141 \mathfrak{b}_{0}^{2}(\theta^{\circ}) \Phi^{\dagger}_{n}\notag\\
& && \leq && G^{-}_{n}\exp\left[\frac{- \eta n \mathfrak{b}_{0}^{2}(\theta^{\circ}) \Phi^{\dagger}_{n}}{2}\right] +\notag\\
& && && 3 \sum\limits_{1 \leq \vert j \vert < G^{-}_{n}}\left(\exp\left[- \frac{m^{\dagger}_{n} \psi_{n}}{400 s \Vert \theta^{\circ} \Vert_{l^{2}} \cdot \Vert \lambda \Vert_{l^{2}}}\right] + \exp\left[ \frac{- \sqrt{r m^{\dagger}_{n} \psi_{n}}}{100\sqrt{s}}\right] + r\beta_{s}\right).
\end{alignat}

And we use \nref{EQD.44} and \nref{EQD.31} followed by \nref{EQD.33} and \nref{EQD.35} in \nref{EQD.28} with $H^{2} = 2\frac{\Lambda_{(n)}n\psi_{n}}{r}$, $h^{2} = n \Lambda_{n}$ and $v = \Lambda_{(n)} \Vert \theta^{\circ} \Vert_{l^{2}} \cdot \Vert \lambda \Vert_{l^{2}}$ to obtain
\begin{alignat}{3}
& \sum\limits_{0 < \vert j \vert \leq n} && && \E_{\theta^{\circ}}^{n}\left[\P_{M \vert Y^{n}}^{n, (\eta)} \left(\llbracket \vert j \vert, n \rrbracket \right) \left\vert  \overline{\theta}_{j} - \theta^{\circ}_{j} \right\vert^{2}\right] \notag\\
& &&\leq&& \frac{25 \mathfrak{b}_{0}^{2}(\theta^{\circ}) \Phi^{\dagger}_{n}}{2 \psi_{n}} + \notag\\
& && && 2 \cdot C \left[\frac{\Lambda_{(n)} \Vert \theta^{\circ} \Vert_{l^{2}} \cdot \Vert \lambda \Vert_{l^{2}}}{r} \exp\left(\frac{-n \Lambda_{(n)} \psi_{n}}{3 \Vert \theta^{\circ} \Vert_{l^{2}} \cdot \Vert \lambda \Vert_{l^{2}}}\right) + \frac{n \Lambda_{(n)}}{r^{2}} \exp\left(\frac{- \sqrt{2 r \psi_{n}}}{100}\right)\right] +\notag\\
& && && 2 \cdot 4 \beta_{s} \sum\limits_{m \leq \vert j \vert \leq l} \Lambda_{j}+\notag\\
& && && 2 \Lambda_{(n)} \psi_{n} C_{\theta^{\circ} \lambda} \exp \left[-\frac{\eta \mathfrak{b}_{0}^{2}(\theta^{\circ}) n \Phi^{\dagger}_{n}}{2}\right] +\notag\\
& && && 3 C_{\theta^{\circ}, \lambda} \left(\exp\left[- \frac{G^{+}_{n} \psi_{n}}{400 s \Vert \theta^{\circ} \Vert_{l^{2}} \cdot \Vert \lambda \Vert_{l^{2}}}\right] + \exp\left[ \frac{- \sqrt{r G^{+}_{n} \psi_{n}}}{100 \sqrt{s}}\right] + G^{+}_{n} r\beta_{s}\right)\label{EQD.47}.
\end{alignat}

Finally, we combine \nref{EQD.45} and \nref{EQD.46} in \nref{PRD.5.1} to obtain the result:

\begin{alignat*}{3}
& \E_{\theta^{\circ}}^{n}\left[\left\Vert \widehat{\theta} - \theta^{\circ}\right\Vert_{l^{2}}^{2}\right] && \leq &&  \frac{25 \mathfrak{b}_{0}^{2}(\theta^{\circ}) \Phi^{\dagger}_{n}}{2 \psi_{n}} + \notag\\
& && && 2 \cdot C \left[\frac{\Lambda_{(n)} \Vert \theta^{\circ} \Vert_{l^{2}} \cdot \Vert \lambda \Vert_{l^{2}}}{r} \exp\left(\frac{-n \Lambda_{(n)} \psi_{n}}{3 \Vert \theta^{\circ} \Vert_{l^{2}} \cdot \Vert \lambda \Vert_{l^{2}}}\right) + \frac{n \Lambda_{(n)}}{r^{2}} \exp\left(\frac{- \sqrt{2 r \psi_{n}}}{100}\right)\right] +\notag\\
& && && 2 \cdot 4 \beta_{s} \sum\limits_{m \leq \vert j \vert \leq l} \Lambda_{j}+\notag\\
& && && 2 \Lambda_{(n)} \psi_{n} C_{\theta^{\circ} \lambda} \exp \left[-\frac{\eta \mathfrak{b}_{0}^{2}(\theta^{\circ}) n \Phi^{\dagger}_{n}}{2}\right] +\notag\\
& && && 3 C_{\theta^{\circ}, \lambda} \left(\exp\left[- \frac{G^{+}_{n} \psi_{n}}{400 s \Vert \theta^{\circ} \Vert_{l^{2}} \cdot \Vert \lambda \Vert_{l^{2}}}\right] + \exp\left[ \frac{- \sqrt{r G^{+}_{n} \psi_{n}}}{100 \sqrt{s}}\right] + G^{+}_{n} r\beta_{s}\right) +\\
& && && \left(G^{-}_{n}\exp\left[\frac{- \eta n \mathfrak{b}_{0}^{2}(\theta^{\circ}) \Phi^{\dagger}_{n}}{2}\right] + \right.\notag\\
& && &&\left. 3 \sum\limits_{1 \leq \vert j \vert < G^{-}_{n}}\left(\exp\left[- \frac{m^{\dagger}_{n} \psi_{n}}{400 s \Vert \theta^{\circ} \Vert_{l^{2}} \cdot \Vert \lambda \Vert_{l^{2}}}\right] + \exp\left[ \frac{- \sqrt{r m^{\dagger}_{n} \psi_{n}}}{100\sqrt{s}}\right] + r\beta_{s}\right)\right) \left\Vert \theta^{\circ} \right\Vert_{l^{2}}^{2} +\\
& && && 141 \mathfrak{b}_{0}^{2}(\theta^{\circ}) \Phi^{\dagger}_{n}.
\end{alignat*}

which completes the proof.

\qedsymbol
\end{pro}


% ....................................................................
% <<Re upper bound 1>>
% ....................................................................
\begin{lm}\label{re:ub:co1} If $\xdf=\bas_0$ then  there is a finite numerical
  constant $\cst{}$ such that for all $\dr\ssY\in\Nz$ we have
  $\FuEx[\ssY]{\rY}\VnormLp{\txdfPr[]-\xdf}^2\leq\cst{}\DipenSv[\ssY_o]\ssY^{-1}$ with  
$\dr\ssY_o:=\ceil{15(\tfrac{300}{\sqrt{\cpen}})^4}$.
\end{lm}
% ....................................................................
% <<Pro Re upper bound 1>>
% ....................................................................
\begin{pro}[Proof of \cref{re:ub:co1}.]
Let $\dr\ssY_o:=\ceil{15(\tfrac{300}{\sqrt{\cpen}})^4}$. We destinguish for $\ssY\in\Nz$ the following two cases
cases, \begin{inparaenum}[i]\renewcommand{\theenumi}{\dgrau\rm(\alph{enumi})}\item\label{pro:ub:co1:c1}
$\ssY\in\nsetro{1,\ssY_o}$ and \item\label{pro:ub:co1:c2}
$\ssY\geq\ssY_o$.\end{inparaenum}

Consider \ref{pro:ub:co1:c1}. We select 
$\dr\pDi=\ssY\leq\ssY_o$ and thus keeping in mind that $\xdf=\bas_0$,
and hence $\VnormLp{\Proj[{\mHiH[0]}]^\perp\xdf}^2=0$  from
\cref{co:agg} follows for all $\ssY\in\nsetro{1,\ssY_o}$
\begin{equation}\label{pro:ub:co1:e1}
\FuEx[\ssY]{\rY}\VnormLp{\txdf-\xdf}^2\leq2\FuEx[\ssY]{\rY}\VnormLp{\txdfPr[\ssY]-\xdfPr[\ssY]}^2\leq4\ssY\oiSv[\ssY]\ssY^{-1}\leq
4 \ssY_o\oiSv[\ssY_o]\ssY^{-1}\leq4\DipenSv[\ssY_o]\ssY^{-1}.
\end{equation}

Consider \ref{pro:ub:co1:c2}, i.e., $\ssY\geq\ssY_o$. We select
$\dr\pdDi:=\ssY_o\in\nset{1,\ssY}$. 
Note that $\Vnormlp[1]{\fydf}=1$ and hence, $\dr\pDi\geq\pdDi\geq
3(\tfrac{800\Vnormlp[1]{\fydf}}{\cpen})^2$. Therefore, for all  $\dr
\ssY\geq \ssY_o\geq 15(\tfrac{300}{\sqrt{\cpen}})^4$ due to \cref{re:ub} 
 follows
\begin{multline*}
\FuEx[\ssY]{\rY}\VnormLp{\txdf-\xdf}^2\leq\cst{}\big\{
[1\vee\VnormLp{\Proj[{\mHiH[0]}]^\perp\xdf}^2]\hRa{\pdDi,\xdf,\iSv}+\ssY^{-1}\}\\\hfill
+2\VnormLp{\Proj[{\mHiH[0]^\perp}]\xdf}^2\bias[\mDi]^2(\xdf)
+6\VnormLp{\Proj[{\mHiH[0]}]^\perp\xdf}^2 \exp\big(\tfrac{-\cpen\cmSv[\mdDi]\mdDi}{400\Vnormlp[1]{\fydf}}\big)
\\\hfill
\hfill+2\VnormLp{\Proj[{\mHiH[0]}]^\perp\xdf}^2\,[\mDi-1]\exp\big(-\rWc\cpen\ssY\hRa{\mdDi,\xdf,\iSv}-
    \tfrac{\rWc\VnormLp{\Proj[{\mHiH[0]}]^\perp\xdf}^2}{4}\ssY\bias[{[\mDi-1]}]^2(\xdf)\big).
\end{multline*}
Since
$\VnormLp{\Proj[{\mHiH[0]}]^\perp\xdf}^2=0$, and thus
$\hRa{\pdDi,\xdf,\iSv}=\DipenSv[\pdDi]/\ssY=\DipenSv[\ssY_o]/\ssY$,
there is a numerical constant $\cst{}$ such that
$\FuEx[\ssY]{\rY}\VnormLp{\txdf-\xdf}^2\leq\cst{}\DipenSv[\ssY_o]\ssY^{-1}$
for all $\ssY\geq\ssY_o$. Combining
the upper bounds  for the two
cases \ref{pro:ub:co1:c1} and \ref{pro:ub:co1:c2}  we obtain the
assertion which  completes the proof.
\end{pro}
% ....................................................................
% <<Re upper bound 2>>
% ....................................................................
\begin{lm}\label{re:ub:co2} Assume there is $K\in\Nz$
  with   $1\geq \bias[{[K-1] }](\xdf)>0$ and $\bias[K](\xdf)=0$. Set
 $K_{\ydf}:=K\dr\vee
3(\tfrac{800\Vnormlp[1]{\fydf}}{\cpen})^2$, $c_{\xdf}:=\tfrac{2\VnormLp{\Proj[{\mHiH[0]}]^\perp\xdf}^2+484\cpen}{\VnormLp{\Proj[{\mHiH[0]}]^\perp\xdf}^2\bias[{[K-1]}]^2(\xdf)}$
and
$\ssY_{\xdf,\iSv}=\ceil{c_{\xdf}\DipenSv[K_{\ydf}]\dr\vee15(\tfrac{300}{\sqrt{\cpen}})^4}$.\\
If $\ssY\leq\ssY_{\xdf,\iSv}$ then let $\sDi{\ssY}:=K_{\ydf}(\log
n)$, and otherwise if  $\ssY>\ssY_{\xdf,\iSv}$ then let
$\sDi{\ssY}:=\max\{\Di\in\nset{K,\ssY}:c_{\xdf}\,\DipenSv<\ssY\}$
where the defining set contains $K_{\ydf}$ and thus it is not empty.
There is a finite numerical constant $\cst{}$ such that for all $\ssY\in\Nz$ holds
\begin{equation}\label{re:ub:co2:e1}
\FuEx[\ssY]{\rY}\VnormLp{\txdf-\xdf}^2
\leq\cst{}\{\DipenSv[\ssY_{\xdf,\iSv}]+\VnormLp{\Proj[{\mHiH[0]^\perp}]\xdf}^2\ssY_{\xdf,\iSv}+ \Vnormlp[1]{\fydf}^2\}\ssY^{-1}
+ 6\VnormLp{\Proj[{\mHiH[0]}]^\perp\xdf}^2\{ \exp\big(\tfrac{-\cpen\cmSv[\sDi{\ssY}]\sDi{\ssY}}{400\Vnormlp[1]{\fydf}}\big)-\tfrac{1}{\ssY}\}.
\end{equation}
If there is $\widetilde{\ssY}_{\xdf,\iSv}\in\Nz$ such that for all
$\ssY\geq\widetilde{\ssY}_{\xdf,\iSv}$ in addition
$\cmiSv[\sDi{\ssY}]\sDi{\ssY}\geq K_{\ydf}(\log\ssY)$  holds true then  
\begin{equation}\label{re:ub:co2:e2}
\FuEx[\ssY]{\rY}\VnormLp{\txdf-\xdf}^2
\leq\cst{}\{\DipenSv[{[\ssY_{\xdf,\iSv}\vee\widetilde{\ssY}_{\xdf,\iSv}]}]+\VnormLp{\Proj[{\mHiH[0]^\perp}]\xdf}^2[\ssY_{\xdf,\iSv}\vee\widetilde{\ssY}_{\xdf,\iSv}]+ \Vnormlp[1]{\fydf}^2\}\ssY^{-1}.
\end{equation}
\end{lm}
% ....................................................................
% <<Pro Re upper bound 2>>
% ....................................................................
\begin{pro}[Proof of \cref{re:ub:co2}.]\label{pro:ub:co2}
Given $K\in\Nz$   with   $1\geq \bias[{[K-1] }](\xdf)>0$ and
$\bias(\xdf)=0$ for all $\Di\geq K$ let $K_{\ydf}:=K\dr\vee
3(\tfrac{800\Vnormlp[1]{\fydf}}{\cpen})^2$, 
$c_{\xdf}:=\tfrac{2\VnormLp{\Proj[{\mHiH[0]}]^\perp\xdf}^2+484\cpen}{\VnormLp{\Proj[{\mHiH[0]}]^\perp\xdf}^2\bias[{[K-1]}]^2(\xdf)}$
and
$\ssY_{\xdf,\iSv}=\ceil{c_{\xdf}\DipenSv[K_{\ydf}]\dr\vee15(\tfrac{300}{\sqrt{\cpen}})^4}$
we distinguish for $\ssY\in\Nz$ the following two
cases, \begin{inparaenum}[i]\renewcommand{\theenumi}{\dgrau\rm(\alph{enumi})}\item\label{pro:ub:co2:c1}
$\ssY\in\nsetro{1,\ssY_{\xdf,\iSv}}$ and \item\label{pro:ub:co2:c2}
$\ssY>\ssY_{\xdf,\iSv}$.\end{inparaenum}\\

Firstly, consider \ref{pro:ub:co2:c1},  let
$\ssY\in\nsetro{1,\ssY_{\xdf,\iSv}}$, then setting $\mDi=1$ and
$\pDi=\ssY$ from \cref{co:agg} follows
\begin{multline}\label{pro:ub:co2:e1}
\FuEx[\ssY]{\rY} \VnormLp{\txdf-\xdf}^2\leq 2\FuEx[\ssY]{\rY}\VnormLp{\txdfPr[\ssY]-\xdfPr[\ssY]}^2+2\VnormLp{\Proj[{\mHiH[0]^\perp}]\xdf}^2\bias[1]^2(\xdf)
\\\hfill\leq 4\ssY\oiSv[\ssY]\ssY^{-1}+2\VnormLp{\Proj[{\mHiH[0]^\perp}]\xdf}^2\leq
4 \ssY_{\xdf,\iSv}\oiSv[\ssY_{\xdf,\iSv}]\ssY^{-1}+2\VnormLp{\Proj[{\mHiH[0]^\perp}]\xdf}^2\ssY_{\xdf,\iSv}\ssY^{-1}\\\leq(4\DipenSv[\ssY_{\xdf,\iSv}]+2\VnormLp{\Proj[{\mHiH[0]^\perp}]\xdf}^2\ssY_{\xdf,\iSv})\ssY^{-1}.
\end{multline}

Secondly, consider \ref{pro:ub:co2:c2}, i.e., 
$\ssY>\ssY_{\xdf,\iSv}$. Setting $\pdDi:=K_{\ydf}\leq\DipenSv[K_{\ydf}]\leq\ssY_{\xdf,\iSv}$, i.e., $\pdDi\in\nset{1,\ssY}$ from $\pdDi=K_{\ydf}\geq K$  follows
$\bias[\pdDi](\xdf)=0$ and hence
$\hRa{\pdDi,\xdf,\iSv}=\DipenSv[K_{\ydf}]\ssY^{-1}$. 
Keeping in mind that $\dr\pdDi\geq
3(\tfrac{800\Vnormlp[1]{\fydf}}{\cpen})^2$ and  $\dr
\ssY\geq \ssY_o\geq 15(\tfrac{300}{\sqrt{\cpen}})^4$ 
from
\cref{re:ub} follows
\begin{multline}\label{pro:ub:co2:e2}
\FuEx[\ssY]{\rY}\VnormLp{\txdf-\xdf}^2\leq\cst{}\big\{
[1\vee\VnormLp{\Proj[{\mHiH[0]}]^\perp\xdf}^2]\DipenSv[K_{\ydf}]\ssY^{-1}+\Vnormlp[1]{\fydf}^2
\ssY^{-1}\}\\\hfill
+2\VnormLp{\Proj[{\mHiH[0]^\perp}]\xdf}^2\bias[\mDi]^2(\xdf)
+6\VnormLp{\Proj[{\mHiH[0]}]^\perp\xdf}^2 \exp\big(\tfrac{-\cpen\cmSv[\mdDi]\mdDi}{400\Vnormlp[1]{\fydf}}\big)
\\\hfill
\hfill+2\VnormLp{\Proj[{\mHiH[0]}]^\perp\xdf}^2\,[\mDi-1]\exp\big(-\rWc\cpen\ssY\hRa{\mdDi,\xdf,\iSv}-
    \tfrac{\rWc\VnormLp{\Proj[{\mHiH[0]}]^\perp\xdf}^2}{4}\ssY\bias[{[\mDi-1]}]^2(\xdf)\big).
\end{multline}
Since
$\ssY>\ssY_{\xdf,\iSv}\geq c_{\xdf}\DipenSv[K_{\ydf}]$
with $c_{\xdf}=\tfrac{2\VnormLp{\Proj[{\mHiH[0]}]^\perp\xdf}^2+484\cpen}{\VnormLp{\Proj[{\mHiH[0]}]^\perp\xdf}^2\bias[{[K-1]}]^2(\xdf)}$ the defining set of
$\sDi{\ssY}:=\max\{\Di\in\nset{K,\ssY}:\ssY>c_{\xdf,\iSv}\DipenSv\}$
evenutally containing $K_{\ydf}$ is not empty. Consequently, $\sDi{\ssY}\geq K$ and $\VnormLp{\Proj[{\mHiH[0]}]^\perp\xdf}^2\bias[{[K-1]}]^2(\xdf)>[2\VnormLp{\Proj[{\mHiH[0]}]^\perp\xdf}^2+484\cpen]\DipenSv[\sDi{\ssY}]/\ssY=[2\VnormLp{\Proj[{\mHiH[0]}]^\perp\xdf}^2+484\cpen]\hRa{\sDi{\ssY},\xdf,\iSv}$. Therefore,
  setting $\mdDi:=\sDi{\ssY}$ the definition  \eqref{de:*Di} implies  $\mDi=K$ and hence
  $\bias[\mDi]^2(\xdf)=\bias[K]^2(\xdf)=0$,
  $\bias[{[\mDi-1]}]^2(\xdf)=\bias[{[K-1]}]^2(\xdf)>0$. From
  \eqref{pro:ub:co2:e2} follows for all $\ssY>\ssY_{\xdf,\iSv}$  thus
\begin{multline}\label{pro:ub:co2:e3}
\FuEx[\ssY]{\rY}\VnormLp{\txdf-\xdf}^2\leq\cst{}\big\{
[1\vee\VnormLp{\Proj[{\mHiH[0]}]^\perp\xdf}^2]\DipenSv[K_{\ydf}]\ssY^{-1}+\Vnormlp[1]{\fydf}^2
\ssY^{-1}\}\\\hfill
+6\VnormLp{\Proj[{\mHiH[0]}]^\perp\xdf}^2 \exp\big(\tfrac{-\cpen\cmSv[\sDi{\ssY}]\sDi{\ssY}}{400\Vnormlp[1]{\fydf}}\big)
\\\hfill
\hfill+2\VnormLp{\Proj[{\mHiH[0]}]^\perp\xdf}^2\,\sDi{\ssY}\exp\big(-\rWc\cpen\ssY\hRa{\mdDi,\xdf,\iSv}-
    \tfrac{\rWc\VnormLp{\Proj[{\mHiH[0]}]^\perp\xdf}^2}{4}\ssY\bias[{[K-1]}]^2(\xdf)\big)\\\hfill
\leq\cst{}\big\{
[1\vee\VnormLp{\Proj[{\mHiH[0]}]^\perp\xdf}^2]\DipenSv[K_{\ydf}]\ssY^{-1}+\Vnormlp[1]{\fydf}^2
\ssY^{-1}\}\\\hfill
+6\VnormLp{\Proj[{\mHiH[0]}]^\perp\xdf}^2 \exp\big(\tfrac{-\cpen\cmSv[\sDi{\ssY}]\sDi{\ssY}}{400\Vnormlp[1]{\fydf}}\big)
\\\hfill
+2\VnormLp{\Proj[{\mHiH[0]}]^\perp\xdf}^2\,[K-1]\underbrace{\exp\big(-\tfrac{\rWc\VnormLp{\Proj[{\mHiH[0]}]^\perp\xdf}^2}{4}\ssY\bias[{[\mDi-1]}]^2(\xdf)\big)}_{\leq
  \tfrac{4}{\rWc\VnormLp{\Proj[{\mHiH[0]}]^\perp\xdf}^2\bias[{[K-1]}]^2(\xdf)}\ssY^{-1}\exp(-1)}
\end{multline}
Note that $\DipenSv[K_{\ydf}]\leq\ssY_{\xdf,\iSv}$ and
$\tfrac{8[K-1]}{e\rWc\bias[{[K-1]}]^2(\xdf)}\leq\tfrac{1}{\rWc}\VnormLp{\Proj[{\mHiH[0]}]^\perp\xdf}^2\ssY_{\xdf,\iSv}$. Thereby,
we obtain 
\begin{multline}\label{pro:ub:co2:e4}
\FuEx[\ssY]{\rY}\VnormLp{\txdf-\xdf}^2
\leq\cst{2}\{\VnormLp{\Proj[{\mHiH[0]}]^\perp\xdf}^2\ssY_{\xdf,\iSv}+ \Vnormlp[1]{\fydf}^2\}\ssY^{-1}\\
+ 6\VnormLp{\Proj[{\mHiH[0]}]^\perp\xdf}^2\{ \exp\big(\tfrac{-\cpen\cmSv[\sDi{\ssY}]\sDi{\ssY}}{400\Vnormlp[1]{\fydf}}\big)-\tfrac{1}{\ssY}\}
 \end{multline}
for some finite numerical constant $\cst{2}$.\\
Combining
the upper bounds 
\eqref{pro:ub:co2:e1} and
\eqref{pro:ub:co2:e4} for the two
cases \ref{pro:ub:co2:c1} and \ref{pro:ub:co2:c2}  we obtain the
assertion \eqref{re:ub:co2:e1}, that is, there is a finite numerical
constant $\cst{}$ such that  for all
$\ssY\in\Nz$ holds
\begin{multline}\label{pro:ub:co2:e5}
\FuEx[\ssY]{\rY}\VnormLp{\txdf-\xdf}^2
\leq\cst{}\{\DipenSv[\ssY_{\xdf,\iSv}]+\VnormLp{\Proj[{\mHiH[0]^\perp}]\xdf}^2\ssY_{\xdf,\iSv}+ \Vnormlp[1]{\fydf}^2\}\ssY^{-1}\\
+ 6\VnormLp{\Proj[{\mHiH[0]}]^\perp\xdf}^2\{ \exp\big(\tfrac{-\cpen\cmSv[\sDi{\ssY}]\sDi{\ssY}}{400\Vnormlp[1]{\fydf}}\big)-\tfrac{1}{\ssY}\}
\end{multline}

Assume finally, that there is in addition
$\widetilde{\ssY}_{\xdf,\iSv}\in\Nz$ such that
$\cmiSv[\sDi{\ssY}]\sDi{\ssY}\geq K_{\ydf}(\log\ssY)$ for all
$\ssY\geq\widetilde{\ssY}_{\xdf,\iSv}$. We shall use without further
reference that then $\exp\big(\tfrac{-\cpen\cmSv[\sDi{\ssY}]\sDi{\ssY}}{400\Vnormlp[1]{\fydf}}\big)\leq\ssY^{-1}$ for
all $\ssY\geq\widetilde{\ssY}_{\xdf,\iSv}$ since $K_{\ydf}\geq
\tfrac{400\Vnormlp[1]{\fydf}}{\cpen}$. Following line by line the
proof of \eqref{pro:ub:co2:e5} using
$\widetilde{\ssY}_{\xdf,\iSv}\vee\ssY_{\xdf,\iSv}$  rather than
$\ssY_{\xdf,\iSv}$  we obtain the
assertion, that is,
$\FuEx[\ssY]{\rY}\VnormLp{\txdf-\xdf}^2
\leq\cst{}\{\DipenSv[{[\ssY_{\xdf,\iSv}\vee\widetilde{\ssY}_{\xdf,\iSv}]}]+\VnormLp{\Proj[{\mHiH[0]^\perp}]\xdf}^2[\ssY_{\xdf,\iSv}\vee\widetilde{\ssY}_{\xdf,\iSv}]+ \Vnormlp[1]{\fydf}^2\}\ssY^{-1}$, which completes the proof.
\end{pro}
\chapter{Proof for \nref{THM_FREQ_CIRCDECONV_UNKNOWN_IID_ORACLE_P}}\label{PRO_FREQ_CIRCDECONV_UNKNOWN_IID_ORACLE_P}
%======================================================================================================================
%                                                                 
% Title:  Appendix: unknown error density
% Author: Jan JOHANNES, Institut für Angewandte Mathematik, Ruprecht-Karls Universität Heidelberg, Deutschland  
% 
% Email: johannes@math.uni-heidelberg.de
% Date: %%ts latex start%%[2018-03-29 Thu 13:25]%%ts latex end%%
%
% ======================================================================================================================
% --------------------------------------------------------------------
% section <<Appendix: Proofs of \nref{au}>>\ref{a:au}
% --------------------------------------------------------------------
% --------------------------------------------------------------------
% <<Proof of Re key argument>>
% --------------------------------------------------------------------
%\begin{pro}[Proof of \nref{co:agg:au}.]
%We start the proof with the observation that
%$\overline{\widehat{\theta}^{(\eta)}}(s)-\overline{\theta^{\circ}}(s)=\widehat{\theta}^{(\eta)}(-s)-\fxdf(-s)$ for all $s\in\Zz$, 
%$\widehat{\theta}^{(\eta)}(0)-\fxdf[(0)]=0$ and
%$\widehat{\theta}^{(\eta)}(s)-\fxdf[(s)]=-\fxdf[(s)]$ for all $s >\ssY$, while for all
%$s\in\nset{1,n}$ with $\xEv:= \{\vert\hfedf[(s)]\vert^2\geq1/\ssE\}$ and
%$\xEv^c:= \{\vert\hfedf[(s)]\vert^2<1/\ssE\}$ holds
%\begin{multline*}
%  \widehat{\theta}^{(\eta)}{(s)}-\fxdf[(s)]=(\hfedfmpI[(s)]\hfydf[(s)]-\fxdf[(s)])\We[](\nset{s,n})-\fxdf[(s)]\We[](\nsetro{1,s})% \\=
%\\
%=
%\hfedfmpI[(s)](\hfydf[(s)]-\fydf[(s)])\We[](\nset{s,n})\\
%+\hfedfmpI[(s)](\fedf[(s)]-\hfedf[(s)])\fxdf[(s)]\We[](\nset{s,n})
%-\Ind{\xEv}\fxdf[(s)]\We[](\nsetro{1,s})-\Ind{\xEv^c}\fxdf[(s)]
%\end{multline*}
%Consequently, we  have
%  \begin{multline}\label{pro:au:key:e1}
%    \Vnormlp{\widehat{\theta}^{(\eta)}-\xdf}^2% =
%%    2\sum_{s\in\nset{1,n}}\vert\hfedfmpI[(s)](\hfydf[(s)]-\fydf[(s)])\We[](\nset{s,\ssY})
%% +\hfedfmpI[(s)](\fedf[(s)]-\hfedf[(s)])\fxdf[(s)]\We[](\nset{s,\ssY})
%% -\fxdf[(s)]\We[](\nsetro{1,s})\vert^2\Ind{\xEv}\\\hfill+2\sum_{s\in\nset{1,\ssY}}\Ind{\xEv^c}\vert\fxdf[(s)]\vert^2+2\sum_{s>\ssY}\vert\fxdf[(s)]\vert^2
%\leq6\sum_{s\in\nset{1,n}}\vert\hfedfmpI[(s)]\vert^2\vert\hfydf[(s)]-\fydf[(s)]\vert^2\We[](\nset{s,n})
%\\\hfill+6\sum_{s\in\nset{1,n}}\Ind{\xEv}\vert\fxdf[(s)]\vert^2\We[](\nsetro{1,s})+2\sum_{s>n}\vert\fxdf[(s)]\vert^2\\\hfill
%+6\sum_{s\in\nset{1,n}}\vert\hfedfmpI[(s)]\vert^2\vert\fedf[(s)]-\hfedf[(s)]\vert^2\vert\fxdf[(s)]\vert^2
%+2\sum_{s\in\nset{1,n}}\Ind{\xEv^c}\vert\fxdf[(s)]\vert^2.
% \end{multline}
%Consider the first r.h.s. term in
%\eqref{pro:au:key:e1}. We split the sum into two parts which we
%bound separately.  Precisely, given
%$\dxdfPr=(\mathds{1}_{\{s \leq m\}}\hfedfmpI[(s)]\fydf[(s)])_{s \in \Z}$ where
%$\Vnormlp{\hxdfPr-\dxdfPr}^2=2\sum_{s\in\nset{1,\Di}}\vert \theta_{n, n_{\lambda}}(s) - \dxdfPr(s)\vert^2=2\sum_{s\in\nset{1,\Di}}\vert\hfedfmpI[(s)]\vert^2\vert\hfydf[(s)]-\fydf[(s)]\vert^2$
%it follows
%\begin{multline}\label{pro:au:key:e2}
%2\sum_{s\in\nset{1,\ssY}}\vert\hfedfmpI[(s)]\vert^2(\hfydf[(s)]-\fydf[(s)])^2
%\We[](\nset{s,\ssY})\\
%%=\sum_{s=1}^{\DiMa}\Ex\Vnormlp{\DiPro[(j-1)j](\hfxdf[]-\fxdf[])}^2\pM[\hw](\nset{j,\DiMa})\}\\
%% \leq2 \sum_{s\in\nset{1,\pDi}}\vert\hfedfmpI[(s)]\vert^2\vert(\hfydf[(s)]-\fydf[(s)])^2 +
%% 2\sum_{s\in\nsetlo{\pDi,n}}\vert\hfedfmpI[(s)]\vert^2(\hfydf[(s)]-\fydf[(s)])^2\sum_{l\in\nset{s,\ssY}}\We[(l)]\\
%% = 2\sum_{s\in\nset{1,\pDi}}\vert\hfedfmpI[(s)]\vert^2\vert(\hfydf[(s)]-\fydf[(s)])^2 +
%% 2\sum_{l\in\nsetlo{\pDi,\ssY}}\We[(l)]\sum_{s\in\nsetlo{\pDi,l}}\vert\hfedfmpI[(s)]\vert^2(\hfydf[(s)]-\fydf[(s)])^2\\
%\leq \Vnormlp{\hxdfPr[\pDi]-\dxdfPr[\pDi]}^2
%+\sum_{l\in\nsetlo{\pDi,\ssY}}\We[(l)]\Vnormlp{\hxdfPr[l]-\dxdfPr[l]}^2\\
%% \leq\Vnormlp{\hxdfPr[\pDi]-\dSoPr[\pDi]}^2
%% +\sum_{l\in\nsetlo{\pDi,\nS}}\We[(l)]\Vnormlp{\hxdfPr[l]-\dSoPr[l]}^2\Ind{\{\Vnormlp{\hxdfPr[l]-\dSoPr[l]}^2\geq\epenSv[l]\}}
%% \\
%% \hfill+(\cpen/\ssY)\sum_{l\in\nsetlo{\pDi,\nS}}\DiepenSv[l]\We[(l)]\Ind{\{\Vnormlp{\hxdfPr[l]-\dSoPr[l]}^2<\epenSv[l]\}}\\
%% =\Vnormlp{\hxdfPr[\pDi]-\dSoPr[\pDi]}^2
%% +\sum_{l\in\nsetlo{\pDi,\nS}}\We[(l)]\vectp{\Vnormlp{\hxdfPr[l]-\dSoPr[l]}^2-\epenSv[l]}\Ind{\{\Vnormlp{\hxdfPr[l]-\dSoPr[l]}^2\geq\epenSv[l]\}}\\
%% \hfill+(\cpen/\ssY)\sum_{l\in\nsetlo{\pDi,\nS}}\DiepenSv[l]\Ind{\{\Vnormlp{\hxdfPr[l]-\dSoPr[l]}^2\geq\epenSv[l]\}}
%% +(\cpen/\ssY)\sum_{l\in\nsetlo{\pDi,\nS}}\DiepenSv[l]\We[(l)]\Ind{\{\Vnormlp{\hxdfPr[l]-\dSoPr[l]}^2<\epenSv[l]\}}\\
%\leq\Vnormlp{\hxdfPr[\pDi]-\dxdfPr[\pDi]}^2
%+\sum_{l\in\nsetlo{\pDi,\ssY}}\We[(l)]\vectp{\Vnormlp{\hxdfPr[l]-\dxdfPr[l]}^2-\pen(l)/7}\\
%+\tfrac{1}{7}\sum_{l\in\nsetlo{\pDi,\ssY}}\We[(l)]\pen(l)\Ind{\{\Vnormlp{\hxdfPr[l]-\dxdfPr[l]}^2\geq\pen(l)/7\}}
%+\tfrac{1}{7}\sum_{l\in\nsetlo{\pDi,\ssY}}\We[(l)]\pen(l)\Ind{\{\Vnormlp{\hxdfPr[l]-\dxdfPr[l]}^2<\pen(l)/7\}}%\\
%% =\pen[\pDi] +
%% \vectp{\Vnormlp{\tSoPr[\pDi]-\SoPr[\pDi]}^2-\pen[\pDi]}+
%% \vectp{\Vnormlp{\tSoPr[n]-\SoPr[n]}^2-\pen[n]} + \cpen \We[](\nsetlo{\pDi,n})\\
%% =\Vnormlp{\tSoPr[\pDi]-\SoPr[\pDi]}^2+ 
%% \vectp{\Vnormlp{\tSoPr[\nS]-\SoPr[\nS]}^2-\cpen\DipenSv[\nS]} + \cpen\,\DipenSv[\nS]\, \We[](\nsetlo{\pDi,n}).% \\
%% \leq \pen[\peDi] +\cpen
%% \We[](\nsetlo{\pDi,n}) + 2\max\setB{\vectp{\Vnormlp{\tSoPr[k]-\SoPr[k]}^2-\pen[k]},k\in\set{\pDi,n}}
%\end{multline}
%Consider the second and third r.h.s. term in \eqref{pro:au:key:e1}.  Splitting the first sum into two parts we obtain
%\begin{multline}\label{pro:au:key:e3}
%2\sum_{s\in\nset{1,n}}\Ind{\xEv}\vert\fxdf[(s)]\vert^2\We[](\nsetro{1,s})+2\sum_{s>n}\vert\fxdf[(s)]\vert^2\\
%\hspace*{5ex}\leq  2\sum_{s\in\nset{1,\mDi}}\vert\fxdf[(s)]\vert^2\Ind{\xEv}\We[](\nsetro{1,s})+ 2\sum_{s\in\nsetlo{\mDi,n}}\vert\fxdf[(s)]\vert^2+
%  2\sum_{s>n}\vert\fxdf[(s)]\vert^2\\\hfill
%\leq \Vnormlp{\xdf_{\underline{0}}}^2\{\We[](\nsetro{1,\mDi})+\bias[\mDi]^2(\xdf)\}
%\end{multline}
%Combining  \eqref{pro:au:key:e1} and the upper bounds \eqref{pro:au:key:e2}
%and \eqref{pro:au:key:e3} we obtain   the assertion \eqref{co:agg:au:e1}, which completes the proof.\proEnd
%\end{pro}
\section{Proofs of \nref{au:rb}}\label{a:au:rb}
% ....................................................................
% Te <<Upper bound random weights>>
% ....................................................................
%\begin{te}
% Below  we state the proofs of  \nref{au:re:SrWe:ag} and \nref{au:re:SrWe:ms}. The
%  proof of \nref{au:re:SrWe:ag} is based on \nref{re:erWe} given first.
%\end{te}
%% ....................................................................
%% <<Re Random weights>>
%% ....................................................................
%\begin{lm}\label{re:erWe} Consider the data-driven aggreagtion weights
%  $\erWe[]$ as in \eqref{au:de:erWe}. Under condition
%  \nref{au:ass:pen:oo} for any $l\in\nset{1,\ssY}$ with
%  $\daRa{l}{(\xdf,\Lambda)}=[\bias[l]^2(\xdf)\vee \DipenSv[l]\, n^{-1}]$ holds
%  \begin{resListeN}[]
%  \item\label{re:erWe:i} with
%    $\aixEv[l]:=\set{1/4\leq\iSv[s]^{-1}\eiSv[(s)]\leq9/4,\;\forall\;s\in\nset{1,l}}$ for all $k\in\nsetro{1,l}$ 
%    we have\\
%   $\erWe[(k)]\Ind{\setB{\Vnormlp{\hxdfPr[l]-\dxdfPr[l]}^2<\peneSv[(l)]/7}}\Ind{\aixEv[l]}$\\\null\hfill$\leq
%  \exp\big(\rWn\big\{[\tfrac{25}{2}\cpen+\tfrac{1}{8}\Vnormlp{\xdf_{\underline{0}}}^2]\dRa{l}{\xdf,\Lambda}-\tfrac{1}{8}\Vnormlp{\xdf_{\underline{0}}}^2\bias^2(\xdf)-\tfrac{1}{50}\penSv\big\}\big)$.
%  \item\label{re:erWe:ii} with $\Vnormlp{\ProjC[l]\dxdfPr[\ssY]}^2=2\sum_{s=l+1}^{\ssY}\iSv[s]^{-1}\eiSv[(s)]\vert\fxdf[(s)]\vert^2$
%    for all $\Di\in\nsetlo{l,\ssY}$ we have\\
%    $\erWe\Ind{\setB{\Vnormlp{\hxdfPr-\dxdfPr}^2<\penSv/7}} \leq
%   \exp\big(\rWn\big\{-\tfrac{1}{2}\peneSv+\tfrac{3}{2}\Vnormlp{\ProjC[l]\dxdfPr[\ssY]}^2+\peneSv[(l)]\big\}\big)$.
%  \end{resListeN}
%\end{lm}
%% --------------------------------------------------------------------
%% <<Proof Re Random weights>>
%% --------------------------------------------------------------------
%\begin{pro}[Proof of \nref{re:erWe}.]
%Given $\Di,l\in\nset{1,n}$ and an event $\dmEv{\Di}{l}$ (to be specified below) it clearly follows
%\begin{multline}\label{p:re:erWe:e1}
% \erWe\Ind{\dmEv{\Di}{l}}=\frac{\exp(-\rWn\{-\Vnormlp{\hxdfPr}^2+\peneSv\})}{\sum_{l\in\nset{1,\ssY}}\exp(-\rWn\{-\Vnormlp{\hxdfPr[l]}^2+\peneSv[(l)]\})}\Ind{\dmEv{\Di}{l}}\\
%\leq
%\exp\big(\rWn\big\{\Vnormlp{\hxdfPr}^2-\Vnormlp{\hxdfPr[l]}^2+(\peneSv[(l)]-\peneSv)\big\}\big)\Ind{\dmEv{\Di}{l}}
%\end{multline}
%We distinguish the two cases \ref{re:erWe:i} $\Di\in\nsetro{1,l}$ and
%\ref{re:erWe:ii} $\Di\in\nsetlo{l,\ssY}$.
%Consider first  \ref{re:erWe:i} $\Di\in\nsetro{1,l}$. From \ref{re:contr:e1} in \nref{re:contr}
%(with
%$\dxdfPr[\bullet]=\hxdfPr[\ssY]$
%and  $\xdf=\dxdfPr[\ssY]=(\mathds{1}_{\{\vert s \vert \leq n\}}\hfedfmpI[(s)]\fydf[(s)])_{s \in \Z}$) follows that
%\begin{multline}\label{p:re:erWe:e2}
% \erWe\Ind{\dmEv{\Di}{l}}
%\leq
%\exp\big(\rWn\big\{\Vnormlp{\hxdfPr}^2-\Vnormlp{\hxdfPr[l]}^2+(\peneSv[(l)]-\peneSv)\big\}\big)\Ind{\dmEv{\Di}{l}}\\
%%\exp\big(\rWn\big\{\contr[](\tSoPr[l])-\contr[](\tSoPr)+\tfrac{9}{2}(\penSv[l]-\penSv)\big\}\big)\Ind{\dmEv{\Di}{l}}\\
%\leq
%\exp\big(\rWn\big\{\tfrac{11}{2}\Vnormlp{\hxdfPr[l]-\dxdfPr[l]}^2-\tfrac{1}{2}\Vnormlp{\dProj{\Di}{l}\dxdfPr[\ssY]}^2
%+(\peneSv[(l)]-\peneSv)\big\}\big)\Ind{\dmEv{k}{l}}
%\end{multline}
%Note that on the event
%$\aixEv[l]:=\set{1/2\leq\vert\fedf[(s)]\hfedfmpI[(s)]\vert\leq3/2,\;\forall\;s\in\nset{1,l}}$
%we have
%\begin{multline*}
%\Vnormlp{\dProj{\Di}{l}\dxdfPr[\ssY]}^2\Ind{\aixEv[l]}\geq
%\tfrac{1}{4}\Vnormlp{\dProj{\Di}{l}\xdf}^2=\tfrac{1}{4}\Vnormlp{\xdf_{\underline{0}}}^2(\bias^2(\xdf)-\bias[l]^2(\xdf))\hfill\\
%\meiSv[l]\Ind{\aixEv[l]}=\max\set{\eiSv[(s)]=(\hfedfmpI[(s)])^2,s\in\nset{1,l}}\Ind{\aixEv[l]}\leq\tfrac{9}{4}\max\set{\iSv[s]=\fedf[(s)]^{-2},s\in\nset{1,l}}=\tfrac{9}{4}\miSv[l]\hfill\\
%\meiSv[l]\Ind{\aixEv[l]}\geq\tfrac{1}{4}\miSv[l]\hfill
%\end{multline*}
%Thus on $\aixEv[l]$ holds  
%$\tfrac{1}{4}l\miSv[l]\vee(l+2)\leq l\meiSv[l]\vee(l+2)\leq
%\tfrac{9}{4}l\miSv[l]\vee(l+2) $. Since 
%$\sqrt{\cmiSv[l]}=\tfrac{\log (l\miSv[l] \vee (l+2))}{\log(l+2)}\geq1 $
% for all $l\in\Nz$ hold
%$\tfrac{\log (\tfrac{1}{4}l\miSv[l] \vee (l+2))}{\log(l+2)}\geq
%\sqrt{\cmiSv[l]}\tfrac{\log(3/4)}{\log 3}\geq \tfrac{3}{10}\sqrt{\cmiSv[l]}$
%and $\tfrac{\log (\tfrac{9}{4}l\miSv[l] \vee
%  (l+2))}{\log(l+2)}\leq\sqrt{\cmiSv[l]}\tfrac{\log(27/4)}{\log 3}\leq
%\tfrac{7}{4}\sqrt{\cmiSv[l]}$
%which together with $\DipenSv[l]=l\cmiSv[l]\miSv[l]$ and $\DipeneSv[l]=l\cmeiSv[l]\meiSv[l]$
%imply 
%\begin{multline}\label{pro:au:erWe:e3}
%\tfrac{3}{10}\sqrt{\cmiSv[l]}\leq\sqrt{\cmeiSv[l]}\Ind{\aixEv[l]}\leq
%\tfrac{7}{4}\sqrt{\cmiSv[l]}\hfill\\
%\tfrac{1}{50}\DipenSv[l]\leq\tfrac{9}{400}\DipenSv[l]=l\,\tfrac{9}{100}\cmiSv[l]\,\tfrac{1}{4}\miSv[l]\\
%\leq l\,\cmeiSv[l]\,\meiSv[l]\Ind{\aixEv[l]}=\DipeneSv[l]\Ind{\aixEv[l]}\leq l\,\tfrac{49}{16}\cmiSv[l]\,\tfrac{9}{4}\miSv[l]=\tfrac{441}{64}\DipenSv[l]\leq7\DipenSv[l]\hfill
%\end{multline}
%and hence for $\penSv=\cpen\DipenSv$ and $\peneSv=\cpen\DipeneSv$
%follows
%\begin{equation}\label{pro:au:erWe:peneSv}
%\tfrac{1}{50}\penSv\leq\peneSv\Ind{\aixEv[l]}\leq7\penSv\quad\text{for
%all }\Di\in\nset{1,l}\text{ and for all }l\in\Nz.
%\end{equation}
%If we define $\dmEv{k}{l}:=\{\Vnormlp{\hxdfPr[l]-\dxdfPr[l]}^2<\peneSv[(l)]/7\}\cap\aixEv[l]$
%then the last bounds imply
%\begin{multline*}
% \erWe\Ind{\setB{\Vnormlp{\hxdfPr[l]-\dxdfPr[l]}^2<\peneSv[(l)]/7}}\Ind{\aixEv[l]}
%\\\leq\exp\big(\rWn\big\{\tfrac{11}{14}\peneSv[(l)]-\tfrac{1}{8}\Vnormlp{\xdf_{\underline{0}}}^2(\bias^2(\xdf)-\bias[l]^2(\xdf))+(\peneSv[(l)]-\peneSv)\big\}\big)\Ind{\aixEv[l]}\\
%\\=\exp\big(\rWn\big\{\tfrac{25}{14}\peneSv[(l)]-\tfrac{1}{8}\Vnormlp{\xdf_{\underline{0}}}^2(\bias^2(\xdf)-\bias[l]^2(\xdf))-\peneSv\big\}\big)\Ind{\aixEv[l]}\\
%\hfill\leq\exp\big(\rWn\big\{7*\tfrac{25}{14}\penSv[l]+\tfrac{1}{8}\Vnormlp{\xdf_{\underline{0}}}^2(\bias[l]^2(\xdf)-\bias^2(\xdf))-\tfrac{1}{50}\penSv\big\}\big)% \\
%% \leq\exp\big(\rWn\big\{\tfrac{630}{16}*\penSv[l]-\tfrac{1}{8}\Vnormlp{\Proj[{\mHiH[0]^\perp}]\xdf}^2(\bias[k]^2(\xdf)-\bias[l]^2(\xdf))\}\big)
%\end{multline*}
% and hence, by exploiting \nref{ak:ass:pen:oo} for
% $\dRa{l}{\xdf,\Lambda}=[\bias[l]^2(\xdf)\vee \DipenSv[l] n^{-1}]$  follows the
% assertion \ref{re:erWe:i}, that is
%\begin{multline*}
%  \erWe[(k)]\Ind{\setB{\Vnormlp{\hxdfPr[l]-\dxdfPr[l]}^2<\peneSv[(l)]/7}}\Ind{\aixEv[l]}\\\leq
%  \exp\big(\rWn\big\{[\tfrac{25}{2}\cpen+\tfrac{1}{8}\Vnormlp{\xdf_{\underline{0}}}^2]\dRa{l}{\xdf,\Lambda}-\tfrac{1}{8}\Vnormlp{\xdf_{\underline{0}}}^2\bias^2(\xdf)-\tfrac{1}{50}\penSv\big\}\big).
% \end{multline*}
%Consider secondly \ref{re:erWe:ii} $\Di\in\nsetlo{l,\ssY}$. From 
%\ref{re:contr:e2} in \nref{re:contr} (with $\dxdfPr[\bullet]=\hxdfPr[\ssY]$
%and  $\xdf=\dxdfPr[\ssY]=(\mathds{1}_{\{\vert s \vert \leq \ssY\}}\hfedfmpI[(s)]\fydf[(s)])_{s \in \Z}$) and \eqref{p:re:erWe:e1} follows 
% \begin{multline}\label{p:re:erWe:e4}
%  \erWe\Ind{\dmEv{l}{\Di}}
%\leq\exp\big(\rWn\big\{\Vnormlp{\hxdfPr}^2-\Vnormlp{\hxdfPr[l]}^2+(\peneSv[(l)]-\peneSv)\big\}\big)\Ind{\dmEv{l}{\Di}}\\
%\leq \exp\big(\rWn\big\{\tfrac{7}{2}\Vnormlp{\hxdfPr[k]-\dxdfPr[k]}^2+\tfrac{3}{2}\Vnormlp{\dProj{l}{k}\dxdfPr[\ssY]}^2+(\peneSv[(l)]-\peneSv)\big\}\big)\Ind{\dmEv{l}{k}}
%\end{multline}
%Keep in mind that
%\begin{multline*}
%\Vnormlp{\dProj{l}{k}\dxdfPr[\ssY]}^2\Ind{\aixEv[l]}=2\sum\nolimits_{s=l+1}^k(\fedf[(s)]\hfedfmpI[(s)])^2\vert\fxdf[(s)]\vert^2\\
%\leq 2\sum\nolimits_{s=l+1}^{\ssY}(\fedf[(s)]\hfedfmpI[(s)])^2\vert\fxdf[(s)]\vert^2=\Vnormlp{\ProjC[{\mHiH[l]}]\dxdfPr[\ssY]}^2.
%\end{multline*} 
%If we set $\dmEv{l}{\Di}:=\{\Vnormlp{\hxdfPr-\dxdfPr}^2<\peneSv/7\}$
% then we clearly have \ref{re:erWe:ii}, that is
% \begin{displaymath}
%   \erWe\Ind{\setB{\Vnormlp{\hxdfPr-\dxdfPr}^2<\penSv/7}} \leq
%   \exp\big(\rWn\big\{-\tfrac{1}{2}\peneSv+\tfrac{3}{2}\Vnormlp{\ProjC[l]\dxdfPr[\ssY]}^2+\peneSv[(l)]\big\}\big)
% \end{displaymath}
% which completes the proof.\proEnd
%\end{pro}
% ....................................................................
% <<Proof Re Sum Random weights>>
% ....................................................................
%\begin{pro}[Proof of \nref{au:re:SrWe:ag}.]
%  Consider \ref{au:re:SrWe:ag:i}. For the non trivial case $\mDi>1$
%  from \nref{re:erWe} \ref{re:erWe:i} with $l=\mdDi$ follows for all
%  $\Di<\mDi\leq \mdDi$, and hence due to the definition
%  \eqref{au:de:*Di:ag}
%  $\Vnormlp{\xdf_{\underline{0}}}^2\bias^2\geq
%  \Vnormlp{\xdf_{\underline{0}}}^2\bias[\mDi-1]^2>[\Vnormlp{\xdf_{\underline{0}}}^2+8*13\cpen]\daRa{\mdDi}{(\xdf,\Lambda)}$.
%  Exploiting the last bound we obtain for each $\Di\in\nsetro{1,\mDi}$
%  \begin{multline*}
%    \erWe\Ind{\setB{\Vnormlp{\hxdfPr[\mdDi]-\dxdfPr[\mdDi]}^2<\peneSv[(\mdDi)]/7}\cap\aixEv[l]}
%    \\\hfill\leq
%    \exp\big(\rWn\big\{-\tfrac{\Vnormlp{\xdf_{\underline{0}}}^2}{8}\bias^2(\xdf)
%    +[\tfrac{25\cpen}{2}+\tfrac{\Vnormlp{\xdf_{\underline{0}}}^2}{8}]\daRaS{\mdDi}{\xdf,\Lambda}-\tfrac{1}{50}\penSv\big\}\big)\\
%    \hfill
%    \leq\exp\big(-\tfrac{1}{2}\rWc\cpen \ssY\daRaS{\mdDi}{\xdf,\Lambda}-\tfrac{1}{50}\rWn\penSv\big)
%  \end{multline*}
%  which in turn with
%  $\penSv=\cpen \Di\cmiSv\miSv \ssY^{-1}\geq \cpen\Di\ssY^{-1}$ and
%  $\sum_{\Di\in\Nz}\exp(-\mu\Di)\leq \mu^{-1}$ for any $\mu>0$
%  implies \ref{ak:re:SrWe:ag:i}, that is,
%  \begin{multline*}
%    \rWe[](\nsetro{1,\mDi})\\
%    \leq\rWe[](\nsetro{1,\mDi})\Ind{\setB{\Vnormlp{\hxdfPr[\mdDi]-\dxdfPr[\mdDi]}^2<\peneSv[(\mdDi)]/7}\cap\aixEv[\mdDi]}
%    +\Ind{\setB{\Vnormlp{\hxdfPr[\mdDi]-\dxdfPr[\mdDi]}^2\geq\peneSv[(\mdDi)]/7}\cup\aixEv[\mdDi]^c}\\
%    \hfill\leq\exp\big(-\tfrac{\rWc\cpen}{2}\ssY\daRaS{\mdDi}{\xdf,\Lambda}\big)\sum\nolimits_{k=1}^{\mDi-1}\exp(-\tfrac{\rWc\cpen}{50}\Di)
%    +\Ind{\setB{\Vnormlp{\hxdfPr[\mdDi]-\dxdfPr[\mdDi]}^2\geq\peneSv[(\mdDi)]/7}\cup\aixEv[\mdDi]^c}\\
%    \leq \tfrac{50}{\rWc\cpen}\exp\big(-\tfrac{\rWc\cpen}{2}\ssY\daRaS{\mdDi}{\xdf,\Lambda}\big)
%    +\Ind{\setB{\Vnormlp{\hxdfPr[\mdDi]-\dxdfPr[\mdDi]}^2\geq\peneSv[(\mdDi)]/7}\cup\aixEv[\mdDi]^c}.
%  \end{multline*} 
%  Consider \ref{au:re:SrWe:ag:ii}. From \nref{re:erWe} \ref{re:erWe:ii}
%  with $l=\pdDi$ follows for all $\Di>\pDi\geq \pdDi$, and hence due
%  to the definition \eqref{au:de:*Di:ag}
%  $\peneSv > 6\Vnormlp{\xdf_{\underline{0}}}^2+ 4\peneSv$. Thereby, we
%  obtain for $\Di\in\nsetlo{\mDi,n}$
%  \begin{multline*}
%    \erWe\Ind{\setB{\Vnormlp{\hxdfPr-\dxdfPr}^2<\penSv/7}} \leq
%   \exp\big(\rWn\big\{-\tfrac{1}{2}\peneSv+\tfrac{3}{2}\Vnormlp{\ProjC[\pdDi]\dxdfPr[\ssY]}^2+\peneSv[(\pdDi)]\big\}\big)
%    % \rWe\Ind{\setB{\Vnormlp{\txdfPr-\xdfPr}^2<\penSv/7}}
%    % \leq % \exp\big(\rWn\big\{-\tfrac{1}{2}\pen +[\tfrac{3}{2}\Vnormlp{\So}^2+\cpen]\dRa{\pdDi}(\So)\big\}\big)
%    % \\
%    % =
%    % \exp\big(\rWn\big\{-\tfrac{1}{4} \penSv
%    % -\tfrac{1}{4}\penSv 
%    % +[\tfrac{3}{2}\Vnormlp{\xdf_{\underline{0}}}^2+\cpen]\daRaS{\pdDi}{\xdf,\Lambda}\big\}\big)
%    \\
%    \leq \exp\big(\rWn\big\{-\tfrac{1}{4} \peneSv\big\}\big).
%  \end{multline*}
%Note that $\vert\hfedf[(s)]\vert^2\leq1$ for all $s\in\Zz$, hence if
%$\vert\hfedf[(s)]\vert^2\geq1/\ssE$ then $\eiSv[(s)]=\vert\hfedfmpI[(s)]\vert^2\geq1$. Thereby,
%$\eiSv[(s)]=\vert\hfedfmpI[(s)]\vert^2<1$ implies $\vert\hfedf[(s)]\vert^2<1/\ssE$ and hence
%$\eiSv[(s)]=\vert\hfedfmpI[(s)]\vert^2=0$. Thereby
%$1>\meiSv=\max\{\vert\hfedfmpI[(s)]\vert^2,s\in\nset{1,\Di}\}$ implies
%$\meiSv=0$, that is,
%\begin{equation}\label{p:au:re:SrWe:ag:hPhi}
%{\{\meiSv<1\}=\{\meiSv=0\}}\text{ and, }\peneSv=\cpen\cmeiSv \Di \meiSv=\peneSv\Ind{\{\meiSv\geq1\}.}
%\end{equation}
%Thereby, it   follows 
%  \begin{multline}\label{p:au:re:SrWe:ag:e1}
%\sum_{\Di\in\nsetlo{\pDi,\ssY}}\peneSv\erWe\Ind{\{\Vnormlp{\hxdfPr[k]-\dxdfPr[k]}^2\leq\peneSv/7\}}
%\leq \sum_{\Di\in\nsetlo{\pDi,\ssY}}\peneSv\exp\big(-\tfrac{\rWc}{4}\ssY\peneSv\big)
%\\\hfill=\sum_{\Di\in\nsetlo{\pDi,\ssY}}\peneSv\exp\big(-\tfrac{\rWc}{4}\ssY\peneSv\big)\{\Ind{\{\meiSv\geq1\}}+\Ind{\{\meiSv<1\}}\}\\
%\hfill=\sum_{\Di\in\nsetlo{\pDi,\ssY}}\peneSv\exp\big(-\tfrac{\rWc}{4}\ssY\peneSv\big)\Ind{\{\meiSv\geq1\}}\\
%=\cpen\ssY^{-1}\sum_{\Di\in\nsetlo{\pDi,\ssY}} \Di\cmeiSv\meiSv\exp\big(-\tfrac{\rWc\cpen}{4}\Di\cmeiSv\meiSv\big)\Ind{\{\meiSv\geq1\}}
%  \end{multline}
%  Exploiting that
%  $\sqrt{\cmeiSv}=\tfrac{\log (\Di\meiSv \vee
%    (\Di+2))}{\log(\Di+2)}\geq1$, $\cpen/4\geq2\log(3e)$ and
%  $\rWc\geq1$, then for all $k\in\Nz$ we have
%  $\tfrac{\rWc\cpen}{4} k-\log(k+2)\geq1$, and hence by
%  $a\exp(-ab)\leq \exp(-b)$ for $a,b\geq1$, it follows
%  \begin{multline}\label{p:au:re:SrWe:ag:e2}
%    \cmeiSv\Di \meiSv\exp\big(-\tfrac{\rWc\cpen}{4}\cmeiSv\Di\meiSv\big)\Ind{\{\meiSv\geq1\}}\\
%    \leq\cmeiSv\exp\big(-\tfrac{\rWc\cpen}{4}\cmeiSv\Di\meiSv + \sqrt{\cmeiSv}\log(\Di+2)\big)\Ind{\{\meiSv\geq1\}}
%    \\\hfill\leq
%    \cmeiSv\exp\big(-\cmeiSv(\tfrac{\rWc\cpen}{4}\Di-\log(\Di+2))\big)\Ind{\{\meiSv\geq1\}}
%    \leq\exp\big(-(\tfrac{\rWc\cpen}{4}\Di-\log(\Di+2))\big)\Ind{\{\meiSv\geq1\}}\\
%    =(\Di+2)\exp\big(-\tfrac{\rWc\cpen}{4}\Di\big)\Ind{\{\meiSv\geq1\}}\leq (\Di+2)\exp\big(-\tfrac{\rWc\cpen}{4}\Di\big).
%  \end{multline}
%  Exploiting $\sum_{\Di\in\Nz}\mu\Di\exp(-\mu\Di)\leq \mu^{-1}$ und
%  $\sum_{\Di\in\Nz}\mu\exp(-\mu\Di)\leq 1$ we obtain
%  \begin{displaymath}
%    \sum_{k=\pDi+1}^{\ssY}\cmeiSv\Di \meiSv\exp\big(-\tfrac{\rWc\cpen}{4}\cmeiSv\Di\meiSv\big)\Ind{\{\meiSv\geq1\}}
%    \leq \sum_{k=\pDi+1}^\infty(\Di+2)\exp\big(-\tfrac{\rWc\cpen}{4}\Di\big)
%    \leq \tfrac{16}{\cpen^2\rWc^{2}}+ \tfrac{8}{\cpen\rWc}.
%  \end{displaymath}
%  Combining the last bound and \eqref{p:au:re:SrWe:ag:e1} we obtain the
%  assertion \ref{au:re:SrWe:ag:ii}, that is
%  \begin{displaymath}
%    \sum_{\Di\in\nsetlo{\pDi,n}}\peneSv\erWe\Ind{\{\Vnormlp{\hxdfPr-\dxdfPr}^2\leq\peneSv/7\}}
%    \leq \ssY^{-1}\{\tfrac{16}{\cpen\rWc^{2}}+ \tfrac{8}{\rWc}\}
%  \end{displaymath}
%  which completes the proof.\proEnd
%\end{pro}
% --------------------------------------------------------------------
% <<Proof Re Sum MS Random weights>>
% --------------------------------------------------------------------
%\begin{pro}[Proof of \nref{au:re:SrWe:ms}.]
%  By definition of $\hDi$ it holds
%  $-\Vnormlp{\hxdfPr[\hDi]}^2+\peneSv[(\hDi)]\leq
%  -\Vnormlp{\hxdfPr}^2+\peneSv$ for all $\Di\in\nset{1,\ssY}$, and
%  hence
%  \begin{equation}\label{au:re:SrWe:ms:pr:e1}
%    \Vnormlp{\hxdfPr[\hDi]}^2-\Vnormlp{\hxdfPr}^2\geq
%    \peneSv[(\hDi)]-\peneSv\text{ for all }\Di\in\nset{1,\ssY}.
%  \end{equation}
%  Consider \ref{ak:re:SrWe:ms:i}. It is sufficient to show, that 
%  $\{\hDi\in\nsetro{1,\mDi}\}\cap\aixEv[\mdDi]\subseteq
%  \{\Vnormlp{\hxdfPr[\mdDi]-\dxdfPr[\mdDi]}^2\geq\peneSv[(\mdDi)]/7\}$ holds for $\mDi>1$.  On the event $\{\hDi\in\nsetro{1,\mDi}\}$ holds
%  $1\leq\hDi<\mDi\leq\mdDi$ and thus by definition
%  \eqref{au:de:*Di:ag}
%  \begin{equation}\label{au:re:SrWe:ms:pr:e2}
%    \Vnormlp{\xdf_{\underline{0}}}^2\bias[\hDi]^2(\xdf)>
%    [\Vnormlp{\xdf_{\underline{0}}}^2+104\cpen]\daRa{\mdDi}{(\xdf,\Lambda)}
%  \end{equation}
%  and due to  \nref{re:contr} \ref{re:contr:e1} 
%(with
%$\dxdfPr[\bullet]=\hxdfPr[\ssY]$
%and  $\xdf=\dxdfPr[\ssY]=(\mathds{1}_{\{\vert s \vert \leq \ssY\}}\hfedfmpI[(s)]\fydf[(s)])_{s \in \Z}$) also
%  \begin{equation}\label{au:re:SrWe:ms:pr:e3}
%    \Vnormlp{\hxdfPr[\hDi]}^2-\Vnormlp{\hxdfPr[\mdDi]}^2\leq
%    \tfrac{11}{2}\Vnormlp{\hxdfPr[\mdDi]-\dxdfPr[\mdDi]}^2
%    -\tfrac{1}{2}\Vnormlp{\dProj{\hDi}{\mdDi}\dxdfPr[\ssY]}^2.
%  \end{equation}
%  On $\{\hDi\in\nsetro{1,\mDi}\}\cap\aixEv[\mdDi]$ we have
%\begin{equation}\label{au:re:SrWe:ms:pr:e3:b}
%\Vnormlp{\dProj{\hDi}{\mdDi}\dxdfPr[\ssY]}^2\Ind{\aixEv[\mdDi]}\geq
%\tfrac{1}{4}\Vnormlp{\dProj{\hDi}{\mdDi}\xdf}^2=\tfrac{1}{4}\Vnormlp{\xdf_{\underline{0}}}^2(\bias[\hDi]^2(\xdf)-\bias[\mdDi]^2(\xdf))
%\end{equation}
%Combining, \eqref{au:re:SrWe:ms:pr:e1}, \eqref{au:re:SrWe:ms:pr:e3}  and
%  \eqref{au:re:SrWe:ms:pr:e3:b} it follows that
%  \begin{multline*}
%    \tfrac{11}{2}\Vnormlp{\hxdfPr[\mdDi]-\dxdfPr[\mdDi]}^2\geq
%    \peneSv[(\hDi)]-\peneSv[(\mdDi)]
%    +\tfrac{1}{8}\Vnormlp{\xdf_{\underline{0}}}^2\{\bias[\hDi]^2(\xdf)-\bias[\mdDi]^2(\xdf)\}\hfill
%  \end{multline*}
%  and hence together with $\peneSv[(\hDi)]\geq0$,  $\peneSv[(\mdDi)]\Ind{\aixEv[\mdDi]}\leq7\penSv[\mdDi]$ 
%  by \eqref{pro:au:erWe:peneSv},  \eqref{au:re:SrWe:ms:pr:e2}
%  and \nref{au:ass:pen:oo} we obtain the claim, that is
%  \begin{multline*}
%    \tfrac{11}{2}\Vnormlp{\hxdfPr[\mdDi]-\dxdfPr[\mdDi]}^2\geq
%    \tfrac{1}{8}\Vnormlp{\xdf_{\underline{0}}}^2\bias[\hDi]^2(\xdf)-
%    \tfrac{1}{8}\Vnormlp{\xdf_{\underline{0}}}^2\bias[\mdDi]^2(\xdf)
%    -\peneSv[(\mdDi)]\\
%    >\tfrac{1}{8}[\Vnormlp{\xdf_{\underline{0}}}^2+104\cpen]\daRa{\mdDi}{(\xdf,\Lambda)}
%    -\tfrac{1}{8}\Vnormlp{\xdf_{\underline{0}}}^2\bias[\mdDi]^2(\xdf)-\peneSv[(\mdDi)]\\
%    \geq 13\cpen \daRa{\mdDi}{(\xdf,\Lambda)}-\peneSv[(\mdDi)]\geq\tfrac{26}{14}\peneSv[(\mdDi)]-\peneSv[(\mdDi)]  \geq\tfrac{11}{14}\peneSv[(\mdDi)],
%  \end{multline*}
%which  shows \ref{au:re:SrWe:ms:i}.  Consider \ref{ak:re:SrWe:ms:ii}. It is sufficient to show that,
%  $\{\hDi\in\nsetlo{\pDi,\ssY}\}\subseteq
%  \{\Vnormlp{\hxdfPr[\hDi]-\dxdfPr[\hDi]}^2\geq\peneSv[(\hDi)]/7\}$.  On the
%  event $\{\hDi\in\nsetlo{\pDi,\ssY}\}$ holds $\hDi>\pDi\geq\pdDi$ and
%  thus by definition \eqref{au:de:*Di:ag}
%  \begin{equation}\label{au:re:SrWe:ms:pr:e4}
%    \peneSv[(\hDi)] > 2[3\Vnormlp{\ProjC[\pdDi]\dxdfPr[\ssY]}^2+2\peneSv[(\pdDi)]]
%  \end{equation}
%  and due to \nref{re:contr} \ref{re:contr:e2} 
%(with
%$\dxdfPr[\bullet]=\hxdfPr[\ssY]$
%and
%$\xdf=\dxdfPr[\ssY]=(\mathds{1}_{\{\vert s \vert \leq n\}}\hfedfmpI[(s)]\fydf[(s)])_{s \in \Z}$)
%also
%\begin{equation}\label{au:re:SrWe:ms:pr:e5}
%  \Vnormlp{\dxdfPr[\hDi]}^2-\Vnormlp{\dxdfPr[\pdDi]}^2\leq
%  \tfrac{7}{2}\Vnormlp{\dxdfPr[\hDi]-\xdfPr[\hDi]}^2+\tfrac{3}{2}\Vnormlp{\dProj{\pdDi}{\hDi}\dxdfPr[\ssY]}^2.
%  \end{equation}
%  Combining, \eqref{au:re:SrWe:ms:pr:e1} and \eqref{au:re:SrWe:ms:pr:e5} it
%  follows that
%  \begin{multline*}
%    \tfrac{7}{2}\Vnormlp{\hxdfPr[\hDi]-\dxdfPr[\hDi]}^2\geq
%    \peneSv[(\hDi)]-\peneSv[(\pdDi)]  -\tfrac{3}{2}\Vnormlp{\dProj{\pdDi}{\hDi}\dxdfPr[\ssY]}^2\hfill
%  \end{multline*}
%  and hence together with
%  $\Vnormlp{\dProj{\pdDi}{\hDi}\dxdfPr[\ssY]}^2\leq\Vnormlp{\ProjC[\pdDi]\dxdfPr[\ssY]}^2$ and
%  \eqref{au:re:SrWe:ms:pr:e4} we obtain the claim,
%  that is
%  \begin{multline*}
%    \tfrac{7}{2}\Vnormlp{\hxdfPr[\hDi]-\dxdfPr[\hDi]}^2\geq
%    (\tfrac{1}{2}+\tfrac{1}{2})\peneSv[(\hDi)]-\peneSv[(\pdDi)]  -\tfrac{3}{2}\Vnormlp{\ProjC[\pdDi]\dxdfPr[\ssY]}^2\\
%    >\tfrac{1}{2}\penSv[\hDi]+\tfrac{1}{2}2[3\Vnormlp{\ProjC[\pdDi]\dxdfPr[\ssY]}^2+2\peneSv[(\pdDi)]]-\peneSv[(\pdDi)]-\tfrac{3}{2}\Vnormlp{\ProjC[\pdDi]\dxdfPr[\ssY]}^2
%    \geq\tfrac{1}{2}\peneSv[(\hDi)],
%  \end{multline*}
%  which shows \ref{ak:re:SrWe:ms:ii} and completes the proof.\proEnd
%\end{pro}
% ....................................................................
% <<Re rest>>
% ....................................................................
%\begin{lm}\label{au:re:rest}
%Consider  
%$\hxdfPr-\dxdfPr=\sum_{s\in\nset{-\Di,\Di}}\hfedfmpI[(s)](\hfydf[(s)]-\fydf[(s)])$.
%Conditionally on $\rE_1,\dotsc,\rE_{\ssE}$ the r.v.'s
%$\rY_1,\dotsc,\rY_{\ssY}$ are \iid and we  denote by $\P{\rY\vert\rE}$ and $\E_{\rY\vert\rE}$ their conditional
% distribution and expectation, respectively. 
%Let $\eiSv[(s)]=\vert\hfedfmpI[(s)]\vert^2$,
%$\oeiSv=\tfrac{1}{\Di}\sum_{s\in\nset{1,\Di}}\eiSv[(s)]$, $\meiSv=
%  \max\{\eiSv[(s)],s\in\nset{1,\Di}\}$, $\cpen\geq1$, $\DiepenSv=\cmeiSv\Di \meiSv$
% and $\sqrt{\cmeiSv}=\tfrac{\log (\Di\meiSv \vee (\Di+2))}{\log(\Di+2)}\geq1$.  Then there is a numerical constant $\cst{}$ such
% that for all $\ssY\in\Nz$ and $\Di\in\nset{1,\ssY}$ holds
%  \begin{resListeN}[]
%  \item\label{au:re:rest:i} let $\Di_{\ydf}:=\floor{  3(6\Vnormlp[1]{\fydf})^2}$ and $ \ssY_{o}:={15(200)^4}$ then\\ 
%    $ \sum_{\Di=1}^{\ssY}\E_{\rY\vert\rE} \vectp{\Vnormlp{\hxdfPr-\dxdfPr}^2 - 12\DiepenSv\ssY^{-1}} 
%\leq \cst{}\ssY^{-1}\big[(1\vee\meiSv[\Di_{\ydf}])\Di_{\ydf}+(1\vee\meiSv[\ssY_{o}])\big]$
%  \item\label{au:re:rest:ii} let
%    $\Di_{\ydf}:=\floor{3(400\Vnormlp[1]{\fydf})^2}$ and
%    $\ssY_{o}:=15({600})^4$ then\\
%    $\sum_{\Di=1}^{\ssY}\cmeiSv \Di \meiSv\P_{\rY\vert\rE}\big(\Vnormlp{\hxdfPr-\dxdfPr}^2 \geq 12\DiepenSv\ssY^{-1}\big)
%\leq\cst{}\big[(1\vee\meiSv[\Di_{\ydf}]^2)\Di_{\ydf}^2+(1\vee\meiSv[\ssY_{o}]^2)\big]$
%  \item\label{au:re:rest:iii} 
%  $\P_{\rY\vert\rE}\big(\Vnormlp{\hxdfPr-\dxdfPr}^2 \geq 12\DiepenSv\ssY^{-1}\big)\leq 3 \big[\exp\big(\tfrac{-\cmeiSv[\Di]\Di}{200\Vnormlp[1]{\fydf}}\big)
%    +(200)^2\ssY^{-1}\big] $.
%  \end{resListeN}
%\end{lm}
% --------------------------------------------------------------------
% <<Proof Re ND rest>>
% --------------------------------------------------------------------
%\begin{pro}[Proof of \nref{au:re:nd:rest}.]
%  Since $\dr\cpen/7\geq 12$ and $\dr\peneSv/7\geq12\DipeneSv\ssY^{-1}$,
%  $\Di\in\nset{1,n}$, by exploiting \nref{au:re:rest}
%  \ref{au:re:rest:i}, \ref{au:re:rest:ii} and \ref{au:re:rest:iii} we
%  obtain immediately the claim \ref{au:re:nd:rest1},
%  \ref{au:re:nd:rest2} and \ref{au:re:nd:rest3}, respectively, which  completes the proof.
%\proEnd\end{pro}
% --------------------------------------------------------------------
% <<Proof Re upper bound ag>>
% --------------------------------------------------------------------
%\begin{pro}[Proof of \nref{au:ag:ub}.]
%  Consider firstly the aggregation using the aggregation weights
%  $\erWe[]$ as in \eqref{au:de:erWe}.  Combining
%  \nref{ak:re:nd:rest} and the upper bound given in \eqref{co:agg:au:ag}
%  we obtain
%  \begin{multline}\label{au:ag:ub:p1}
%  \E_{\rY\vert\rE}\Vnormlp{\widehat{\theta}^{(\eta)}-\xdf}^2\leq 
%    3\E_{\rY\vert\rE}\Vnormlp{\hxdfPr[\pDi]-\dxdfPr[\pDi]}^2
%    +3 \Vnormlp{\xdf_{\underline{0}}}^2\bias[\mDi]^2(\xdf)\\\hfill
%    +\tfrac{150}{\rWc\cpen}\Vnormlp{\xdf_{\underline{0}}}^2\Ind{\{\mDi>1\}}
%    \exp\big(-\tfrac{3\rWc\cpen}{14}n\daRa{\mdDi}{(\xdf,\Lambda)}\big)
%    \\\hfill
%    +\cst{} \Vnormlp{\xdf_{\underline{0}}}^2\Ind{\{\mDi>1\}} \big[ \exp\big(\tfrac{-\cmeiSv[\mdDi]\mdDi}{200\Vnormlp[1]{\fydf}}\big) \Ind{\aixEv[\mdDi]} + \Ind{\aixEv[\mdDi]^c}\big]
%    \\\hfill
%    +6\sum_{s\in\nset{1,\ssY}}\vert\hfedfmpI[(s)]\vert^2\vert\fedf[(s)]-\hfedf[(s)]\vert^2\vert\fxdf[(s)]\vert^2
%    +2\sum_{s\in\nset{1,\ssY}}\Ind{\xEv^c}\vert\fxdf[(s)]\vert^2\\
%    +\cst{}\ssY^{-1}\{(1\vee\meiSv[\Di_{\ydf}]^2)\Di_{\ydf}^2+(1\vee\meiSv[\ssY_{o}]^2)+\Vnormlp{\xdf_{\underline{0}}}^2\Ind{\{\mDi>1\}} +\tfrac{16}{\cpen\rWc^{2}}+
%    \tfrac{8}{\rWc}\}
%  \end{multline}
% Consider
%  \begin{multline*}
%  \E_{\rY\vert\rE}\Vnormlp{\hxdfPr[\pDi]-\dxdfPr[\pDi]}^2\\
%  =2\sum_{s=1}^{\pDi}(\hfedfmpI[(s)])^2/\ssY=2\sum_{s=1}^{\pDi}\eiSv[(s)]/\ssY=2\pDi\oeiSv[(\pDi)]/\ssY\leq2\DipeneSv\ssY^{-1},
%  \end{multline*}
%  where  by construction $\peneSv/7\geq12\DipeneSv\ssY^{-1}$ and hence
%  we have
%  $\E_{\rY\vert\rE}\Vnormlp{\hxdfPr[\pDi]-\dxdfPr[\pDi]}^2\leq\tfrac{1}{42}\peneSv[(\pDi)]$. Moreover, exploiting
%  $\max_{s\in\nset{1,n}}\eiSv[(s)]\leq\ssE$ and $\pDi\leq\ssY$ it holds also 
%  $\E_{\rY\vert\rE}\Vnormlp{\hxdfPr[\pDi]-\dxdfPr[\pDi]}^2\leq2\ssE$.
%Considering the event $\aixEv[\pdDi]$ and its complement
%$\aixEv[\pdDi]^c$  it follows
%$\E_{\rY\vert\rE}\Vnormlp{\hxdfPr[\pDi]-\dxdfPr[\pDi]}^2\leq
%2\ssE\Ind{\aixEv[\pdDi]^c}+\tfrac{1}{42}\peneSv[(\pDi)]\Ind{\aixEv[\pdDi]}$.
%Taking into account the definition
%  \eqref{au:de:*Di:ag} of
%  $\pDi$ we obtain
%  $\E_{\rY\vert\rE}\Vnormlp{\hxdfPr[\pDi]-\dxdfPr[\pDi]}^2\leq
%  2\ssE\Ind{\aixEv[\pdDi]^c}+\tfrac{1}{42}[6\Vnormlp{\ProjC[\pdDi]\dxdfPr[\ssY]}^2+4\peneSv[(\pdDi)]]\Ind{\aixEv[\pdDi]}$ 
%Thereby,
%with $\rWc\geq1$ and $\cpen\geq1$
%from \eqref{au:ag:ub:p1} follows (keep in mind that $\ssY_{o}$ is a numerical constant)
%  \begin{multline}\label{au:ag:ub:p2}
%  \E_{\rY\vert\rE}\Vnormlp{\widehat{\theta}^{(\eta)}-\xdf}^2\leq  \tfrac{2}{7}\peneSv[(\pdDi)]\Ind{\aixEv[\pdDi]}+\tfrac{3}{7}\Vnormlp{\ProjC[\pdDi]\dxdfPr[\ssY]}^2\Ind{\aixEv[\pdDi]}
%    +3 \Vnormlp{\xdf_{\underline{0}}}^2\bias[\mDi]^2(\xdf)\\\hfill
%    +\cst{}\Vnormlp{\xdf_{\underline{0}}}^2\Ind{\{\mDi>1\}}\big[
%    \exp\big(-\tfrac{3\rWc\cpen}{14}n\daRa{\mdDi}{(\xdf,\Lambda)}\big)
%    +
%    \exp\big(\tfrac{-\cmeiSv[\mdDi]\mdDi}{200\Vnormlp[1]{\fydf}}\big)
%    \Ind{\aixEv[\mdDi]}\big]\\ \hfill+ \cst{}\big[
%    \Vnormlp{\xdf_{\underline{0}}}^2\Ind{\{\mDi>1\}} \Ind{\aixEv[\mdDi]^c}+\ssE\Ind{\aixEv[\pdDi]^c} + \ssY^{-1}\{\Di_{\ydf}^2\ssE^2\Ind{\aixEv[\Di_{\ydf}]^c}+\ssE^2\Ind{\aixEv[\ssY_{o}]^c}\}\big]
%    \\\hfill
%    +6\sum_{s\in\nset{1,\ssY}}\vert\hfedfmpI[(s)]\vert^2\vert\fedf[(s)]-\hfedf[(s)]\vert^2\vert\fxdf[(s)]\vert^2
%    +2\sum_{s\in\nset{1,\ssY}}\Ind{\xEv^c}\vert\fxdf[(s)]\vert^2\\
%    +\cst{}\ssY^{-1}\{(1\vee\meiSv[\Di_{\ydf}]^2)\Di_{\ydf}^2\Ind{\aixEv[\Di_{\ydf}]}+(1\vee\meiSv[\ssY_{o}]^2)\Ind{\aixEv[\ssY_{o}]}+\Vnormlp{\xdf_{\underline{0}}}^2\Ind{\{\mDi>1\}}\}
%  \end{multline}
%Employing \eqref{pro:au:erWe:e3}  and \eqref{pro:au:erWe:peneSv}
%follows $\meiSv\Ind{\aixEv[\Di]}\leq \tfrac{9}{4}\miSv$,
%$\cmeiSv\Ind{\aixEv[\Di]}\geq \tfrac{9}{100}\cmiSv$ and
%$\peneSv\Ind{\aixEv[\Di]}\leq 7\penSv$ for all $\Di\in\Nz$. Thereby,
%\eqref{au:ag:ub:p2}  implies
%  \begin{multline}\label{au:ag:ub:p3}
%  \E_{\rY\vert\rE}\Vnormlp{\widehat{\theta}^{(\eta)}-\xdf}^2\leq  2\penSv[\pdDi] +\tfrac{3}{7}\Vnormlp{\ProjC[\pdDi]\dxdfPr[\ssY]}^2+3 \Vnormlp{\xdf_{\underline{0}}}^2\bias[\mDi]^2(\xdf)\\\hfill
%    +\cst{}\Vnormlp{\xdf_{\underline{0}}}^2\Ind{\{\mDi>1\}}\big[
%    \exp\big(-\tfrac{3\rWc\cpen}{14}n\daRa{\mdDi}{(\xdf,\Lambda)}\big)
%    +
%    \exp\big(\tfrac{-9\cmiSv[\mdDi]\mdDi}{20000\Vnormlp[1]{\fydf}}\big)
%    \big]\\ \hfill+ \cst{}\big[
%    \Vnormlp{\xdf_{\underline{0}}}^2\Ind{\{\mDi>1\}} \Ind{\aixEv[\mdDi]^c}+\ssE\Ind{\aixEv[\pdDi]^c} + \ssY^{-1}\{\Di_{\ydf}^2\ssE^2\Ind{\aixEv[\Di_{\ydf}]^c}+\ssE^2\Ind{\aixEv[\ssY_{o}]^c}\}\big]
%    \\\hfill
%    +6\sum_{s\in\nset{1,\ssY}}\vert\hfedfmpI[(s)]\vert^2\vert\fedf[(s)]-\hfedf[(s)]\vert^2\vert\fxdf[(s)]\vert^2
%    +2\sum_{s\in\nset{1,\ssY}}\Ind{\xEv^c}\vert\fxdf[(s)]\vert^2\\
%    +\cst{}\ssY^{-1}\{\miSv[\Di_{\ydf}]^2\Di_{\ydf}^2+\miSv[\ssY_{o}]^2+\Vnormlp{\xdf_{\underline{0}}}^2\Ind{\{\mDi>1\}}\}
%  \end{multline}
%Exploiting \nref{oSv:re} we obtain from 
%\ref{oSv:re:i}
%\begin{displaymath}
%\E\Vnormlp{\ProjC[\pdDi]\dxdfPr[\ssY]}^2\leq4\sum_{\vert s\vert\in\nsetlo{\pDi,\ssY}}\vert\fxdf[(s)]\vert^2\leq
%4\Vnormlp{\xdf_{\underline{0}}}^2\bias[\pdDi]^2(\xdf)
%\end{displaymath}
%from \ref{oSv:re:iii} and $\mRa{\xdf,\Lambda}:=\sum_{s\in\Nz}\fxdf[(s)]^2[1\wedge
%\iSv[s]/\ssE]$ as defined in \eqref{oo:de:mra} 
%\begin{displaymath}
%\sum_{s\in\nset{1,\ssY}}\vert\fxdf[(s)]\vert^2\E\vert\hfedfmpI[(s)]\vert^2\vert\fedf[(s)]-\hfedf[(s)]\vert^2\leq4\cst{4}\mRa{\xdf,\Lambda}
%\end{displaymath}
%and from
%\ref{oSv:re:ii}
%\begin{displaymath}
%\sum_{s\in\nset{1,n}}\P(\xEv^c)\vert\fxdf[(s)]\vert^2
%\leq4\mRa{\xdf,\Lambda}
%\end{displaymath}
%The last bounds imply together with 
%  \begin{multline}\label{au:ag:ub:p4}
%  \E\Vnormlp{\widehat{\theta}^{(\eta)}-\xdf}^2\leq
%  2\penSv[\pdDi] +\tfrac{12}{7}\Vnormlp{\xdf_{\underline{0}}}^2\bias[\pdDi]^2(\xdf)+3 \Vnormlp{\xdf_{\underline{0}}}^2\bias[\mDi]^2(\xdf)\\\hfill
%    +\cst{}\Vnormlp{\xdf_{\underline{0}}}^2\Ind{\{\mDi>1\}}\big[
%    \exp\big(-\tfrac{3\rWc\cpen}{14}n\daRa{\mdDi}{(\xdf,\Lambda)}\big)
%    +
%    \exp\big(\tfrac{-9\cmiSv[\mdDi]\mdDi}{20000\Vnormlp[1]{\fydf}}\big)
%    \big]\\ \hfill+ \cst{}\big[
%    \Vnormlp{\xdf_{\underline{0}}}^2\Ind{\{\mDi>1\}} \P(\aixEv[\mdDi]^c)+\ssE\P(\aixEv[\pdDi]^c) + \ssY^{-1}\{\Di_{\ydf}^2\ssE^2\P(\aixEv[\Di_{\ydf}]^c)+\ssE^2\P(\aixEv[\ssY_{o}]^c)\}\big]
%    \\\hfill
%    +24\cst{4}\mRa{\xdf,\Lambda}
%    +8\mRa{\xdf,\Lambda}\\
%    +\cst{}\ssY^{-1}\{\miSv[\Di_{\ydf}]^2\Di_{\ydf}^2+\miSv[\ssY_{o}]^2+\Vnormlp{\xdf_{\underline{0}}}^2\Ind{\{\mDi>1\}}\}
%  \end{multline}
%Moreover, since $\ssY\daRa{\mdDi}{(\xdf,\Lambda)}\geq\cmiSv[\mdDi]\mdDi$. From
%\eqref{au:ag:ub:p4} with $\tfrac{3\rWc\cpen}{14}>\tfrac{9}{20000\Vnormlp[1]{\fydf}}$
%(since $\cpen,\rWc\geq1$ and $\Vnormlp[1]{\fydf}\geq\vert\fydf[(0)]\vert=1$)
%follows
%  \begin{multline}\label{au:ag:ub:p5}
%  \E\Vnormlp{\widehat{\theta}^{(\eta)}-\xdf}^2\leq
%  2\penSv[\pdDi] +\tfrac{12}{7}\Vnormlp{\xdf_{\underline{0}}}^2\bias[\pdDi]^2(\xdf)+3 \Vnormlp{\xdf_{\underline{0}}}^2\bias[\mDi]^2(\xdf)\\\hfill
%    +\cst{}\Vnormlp{\xdf_{\underline{0}}}^2\Ind{\{\mDi>1\}}
%    \exp\big(\tfrac{-9\cmiSv[\mdDi]\mdDi}{20000\Vnormlp[1]{\fydf}}\big)
%    \\ \hfill+ \cst{}\big[
%    \Vnormlp{\xdf_{\underline{0}}}^2\Ind{\{\mDi>1\}} \P(\aixEv[\mdDi]^c)+\ssE\P(\aixEv[\pdDi]^c) + \ssY^{-1}\{\Di_{\ydf}^2\ssE^2\P(\aixEv[\Di_{\ydf}]^c)+\ssE^2\P(\aixEv[\ssY_{o}]^c)\}\big]
%    \\\hfill
%    +\cst{}\mRa{\xdf,\Lambda}
%    +\cst{}\ssY^{-1}\{\miSv[\Di_{\ydf}]^2\Di_{\ydf}^2+\miSv[\ssY_{o}]^2+\Vnormlp{\xdf_{\underline{0}}}^2\Ind{\{\mDi>1\}}\}
%  \end{multline}
%Exploiting \nref{re:evrest} \ref{re:evrest:ii} there is a
%numerical constant $\cst{}$ such that for all  $\ssE,\Di\in\Nz$ holds
%$\P(\aixEv^c)\leq\cst{}\Di\miSv^2\ssE^{-2}$
%and hence, $\ssE^2\P(\aixEv[\Di_{\ydf}]^c)\leq
%\cst{}\Di_{\ydf}\miSv[\Di_{\ydf}]^2$ and $\ssE^2\P(\aixEv[\ssY_{o}]^c)\leq
%\cst{}\ssY_{o}\miSv[\ssY_{o}]^2$, thereby from \eqref{au:ag:ub:p4}
%follows the assertion \eqref{au:ag:ub:e1}, that is, (keep in mind that $\ssY_{o}$ is a numerical constant)
%  \begin{multline}\label{au:ag:ub:p6}
%  \E\Vnormlp{\widehat{\theta}^{(\eta)}-\xdf}^2\leq
%  2\penSv[\pdDi] +\tfrac{12}{7}\Vnormlp{\xdf_{\underline{0}}}^2\bias[\pdDi]^2(\xdf)+3 \Vnormlp{\xdf_{\underline{0}}}^2\bias[\mDi]^2(\xdf)\\\hfill
%    +\cst{}\Vnormlp{\xdf_{\underline{0}}}^2\Ind{\{\mDi>1\}}
%    \exp\big(\tfrac{-9\cmiSv[\mdDi]\mdDi}{20000\Vnormlp[1]{\fydf}}\big)
%    + \cst{}\big[
%    \Vnormlp{\xdf_{\underline{0}}}^2\Ind{\{\mDi>1\}} \P(\aixEv[\mdDi]^c)+\ssE\P(\aixEv[\pdDi]^c) \big]
%    \\\hfill
%    +\cst{}\mRa{\xdf,\Lambda}
%    +\cst{}\ssY^{-1}\{\miSv[\Di_{\ydf}]^2\Di_{\ydf}^3+\miSv[\ssY_{o}]^2+\Vnormlp{\xdf_{\underline{0}}}^2\Ind{\{\mDi>1\}}\}
%  \end{multline}
%  Consider secondly the aggregation using the model selection weights $\msWe[]$
%  as in \eqref{au:de:msWe}. Combining
%  \nref{au:re:nd:rest} and the upper bound given in \eqref{co:agg:ms}
%  we obtain
%  \begin{multline}\label{au:ag:ub:p7}
%    \E\Vnormlp{\txdfAg[{\msWe[]}]-\xdf}^2\leq \tfrac{2}{7}\penSv[\pDi]
%    +2\Vnormlp{\xdf_{\underline{0}}}^2\bias[\mDi]^2(\xdf)
%    \\\hfill
% + \cst{}\Vnormlp{\xdf_{\underline{0}}}^2\Ind{\{\mDi>1\}}
% \exp\big(\tfrac{-1}{200\Vnormlp[1]{\fydf}}\ssY\daRaS{\mdDi}{\xdf,\Lambda}\miSv[\mdDi]^{-1}\big)
% \\
% +\cst{}\big[
% \Vnormlp{\xdf_{\underline{0}}}^2\Ind{\{\mDi>1\}}
%+\miSv[\Di_{\ydf}]^2\Di_{\ydf}^2+\miSv[\ssY_{o}]^2 \big]\ssY^{-1}.
%  \end{multline}  
%From \eqref{ak:ag:ub:p2} and \eqref{ak:ag:ub:p3} together with
%$\ssY\daRaS{\mdDi}{\xdf,\Lambda}\miSv[\mdDi]^{-1}\geq\cmiSv[\mdDi]\mdDi$
%follows the claim \eqref{ak:ag:ub:e1}, which  completes the proof.
%\proEnd\end{pro}
% ....................................................................
% <<Pro upper bound ag p>>
% ....................................................................
%\begin{pro}[Proof of \nref{au:ag:ub:pnp}.]
%From
%\eqref{au:ag:ub:e1} follows for any $\mdDi,\pdDi\in\nset{1,n}$ and associated
%$\mDi,\pDi\in\nset{1,n}$ as defined in  \eqref{au:de:*Di:ag}%
% \begin{multline}\label{au:ag:ub:pnp:p1}
%   \E\Vnormlp{\widehat{\theta}^{(\eta)}-\xdf}^2\leq
%  2\penSv[\pdDi] +\tfrac{12}{7}\Vnormlp{\xdf_{\underline{0}}}^2\bias[\pdDi]^2(\xdf)+3 \Vnormlp{\xdf_{\underline{0}}}^2\bias[\mDi]^2(\xdf)\\\hfill
%    +\cst{}\Vnormlp{\xdf_{\underline{0}}}^2\Ind{\{\mDi>1\}}
%    \exp\big(\tfrac{-9\cmiSv[\mdDi]\mdDi}{20000\Vnormlp[1]{\fydf}}\big)
%    + \cst{}\big[
%    \Vnormlp{\xdf_{\underline{0}}}^2\Ind{\{\mDi>1\}} \P(\aixEv[\mdDi]^c)+\ssE\P(\aixEv[\pdDi]^c) \big]
%    \\\hfill
%    +\cst{}\mRa{\xdf,\Lambda}
%    +\cst{}\ssY^{-1}\{\miSv[\Di_{\ydf}]^2\Di_{\ydf}^3+\miSv[\ssY_{o}]^2+\Vnormlp{\xdf_{\underline{0}}}^2\Ind{\{\mDi>1\}}\}
%\end{multline}
%We destinguish the two cases \ref{au:ag:ub:pnp:p} and
%\ref{au:ag:ub:pnp:np}. Consider first \ref{au:ag:ub:pnp:p}, and hence there is $K\in\Nz_0$   with   $1\geq \bias[{[K-1] }](\xdf)>0$ and
%$\bias(\xdf)=0$ for all $\Di\geq K$.
%Consider first $K=0$, then $\bias[0](\xdf)=0$
%and hence $\Vnormlp{\xdf_{\underline{0}}}^2=0$ and $\mRa{\xdf,\Lambda}=0$ (see \nref{oo:rem:nm}). From \eqref{au:ag:ub:pnp:p1}
%follows 
% \begin{equation}\label{au:ag:ub:pnp:p2}
%    \E\Vnormlp{\widehat{\theta}^{(\eta)}-\xdf}^2\leq
%  2\penSv[\pdDi] 
%    + \cst{}\ssE\P(\aixEv[\pdDi]^c) 
%    +\cst{}\ssY^{-1}\{\miSv[\Di_{\ydf}]^2\Di_{\ydf}^3+\miSv[\ssY_{o}]^2\}
%\end{equation}
%Setting  $\pdDi:=1$ it follows
%$\penSv[\pdDi]=\cpen\DipenSv[1]\ssY^{-1}=\cpen\cmSv[1]
%\miSv[1]\ssY^{-1}\leq\cpen\miSv[1]^2\ssY^{-1}$ and exploiting \nref{re:evrest} \ref{re:evrest:ii} there is a
%numerical constant $\cst{}$ such that for all  $\ssE\in\Nz$ holds
%$\P(\aixEv[\pdDi]^c)\leq\cst{}\miSv[1]^2\ssE^{-2}$. Thereby with numerical
%constant $\cpen\geq84$, \eqref{au:ag:ub:pnp:p2} implies for all $\ssY,\ssE\in\Nz$
% \begin{equation}\label{au:ag:ub:pnp:p3}
%    \E\Vnormlp{\widehat{\theta}^{(\eta)}-\xdf}^2\leq
% \cst{}\miSv[1]^2\ssE^{-1} +\cst{}\ssY^{-1}\{\miSv[1]^2+\miSv[\Di_{\ydf}]^2\Di_{\ydf}^3+\miSv[\ssY_{o}]^2\}
%\end{equation}
%Consider now $K\in\Nz$, and hence $\Vnormlp{\xdf_{\underline{0}}}^2>0$. Let 
%$ c_{\xdf}:=\tfrac{\Vnormlp{\xdf_{\underline{0}}}^2+104\cpen}{\Vnormlp{\xdf_{\underline{0}}}^2\bias[{[K-1]}]^2(\xdf)}>1$
%and $\ssY_{\xdf}:=\floor{c_{\xdf}\DipenSv[K]}\in\Nz$. We distinguish for $n\in\Nz$ the following two
% cases, \begin{inparaenum}[i]\renewcommand{\theenumi}{\dgrau\rm(\alph{enumi})}\item\label{au:ag:ub:pnp:p:n1}
%$\ssY\in\nset{1,\ssY_{\xdf}}$ and \item\label{au:ag:ub:pnp:p:n2}
%$\ssY> \ssY_{\xdf}$. \end{inparaenum} Firstly, consider
%\ref{au:ag:ub:pnp:p:n1} with $\ssY\in\nset{1,\ssY_{\xdf}}$, then setting $\mdDi:=1$, $\pdDi:=1$ we have
%$\mDi=1$, $1\geq\bias[\mDi]$ and  $1\leq\DipenSv[1]=\cmSv[1]
%\miSv[1]\leq\miSv[1]^2$. Thereby,  from \eqref{au:ag:ub:pnp:p1} 
%follows
% \begin{multline*}
%    \E\Vnormlp{\widehat{\theta}^{(\eta)}-\xdf}^2\leq
%  2\cpen\miSv[1]^2\ssY^{-1} +\tfrac{33}{7}\Vnormlp{\xdf_{\underline{0}}}^2 + \cst{}\big[\ssE\P(\aixEv[1]^c) \big]
%    \\\hfill
%    +\cst{}\mRa{\xdf,\Lambda}
%    +\cst{}\ssY^{-1}\{\miSv[\Di_{\ydf}]^2\Di_{\ydf}^3+\miSv[\ssY_{o}]^2\}
%  \end{multline*}
%  Exploiting \nref{re:evrest} \ref{re:evrest:ii} there is a
%numerical constant $\cst{}$ such that for all  $\ssE\in\Nz$ holds
%$\P(\aixEv[1]^c)\leq\cst{}\miSv[1]^2\ssE^{-2}$, which
%together with
%$\mRa{\xdf,\Lambda}\leq
%\Vnormlp{\xdf_{\underline{0}}}^2\miSv[K]\ssE^{-1}$ implies
% \begin{multline*}
%    \E\Vnormlp{\widehat{\theta}^{(\eta)}-\xdf}^2\leq
%  2\cpen\miSv[1]^2\ssY^{-1} +\tfrac{33}{7}\Vnormlp{\xdf_{\underline{0}}}^2 +
%  \cst{}\big[\miSv[1]^2 +\Vnormlp{\xdf_{\underline{0}}}^2\miSv[K] \big]\ssE^{-1}
%    \\\hfill
%    +\cst{}\ssY^{-1}\{\miSv[\Di_{\ydf}]^2\Di_{\ydf}^3+\miSv[\ssY_{o}]^2\}
%  \end{multline*}
%Moreover, for all $\ssY\in\nset{1,\ssY_{\xdf}}$ with
%$\ssY_{\xdf}=\floor{c_{\xdf}\DipenSv[K]}$ and
%$\DipenSv[K]=K\cmSv[K] \miSv[K]\leq K^2\miSv[K]^2$ holds
%$\ssY\leq\cst{}\tfrac{(\Vnormlp{\xdf_{\underline{0}}}^2\vee1)}{\Vnormlp{\xdf_{\underline{0}}}^2\bias[{[K-1]}]^2(\xdf)}
%K^2\miSv[K]^2$ and thereby, for all $\ssY\in\nset{1,\ssY_{\xdf}}$ and
%for all $\ssE\in\Nz$
%\begin{multline}\label{au:ag:ub:pnp:p4}
%  \E\Vnormlp{\widehat{\theta}^{(\eta)}-\xdf}^2\leq
%  \cst{}\big[(\Vnormlp{\xdf_{\underline{0}}}^2\vee1)\miSv[1]^2\tfrac{K^2\miSv[K]^2}{\Vnormlp{\xdf_{\underline{0}}}^2\bias[{[K-1]}]^2(\xdf)}+\miSv[\Di_{\ydf}]^2\Di_{\ydf}^3+\miSv[\ssY_{o}]^2\big]\ssY^{-1}\\
%  +
%  \cst{}\big[\miSv[1]^2 +\Vnormlp{\xdf_{\underline{0}}}^2\miSv[K] \big]\ssE^{-1}.
%\end{multline}
%Secondly, consider \ref{au:ag:ub:pnp:p:n2}, i.e., $\ssY>
%\ssY_{\xdf}$. Setting
%$\pdDi:=K< \floor{c_{\xdf}\DipenSv[K]}=\ssY_{\xdf}$, i.e.,
%$\pdDi\in\nset{1,\ssY}$, it follows $\bias[\pdDi](\xdf)=0$ and
%$\pen[\pdDi]=\cpen\DipenSv[K]\ssY^{-1}\leq
%\cpen K^2\miSv[K]^2\ssY^{-1}$. From
%\eqref{au:ag:ub:pnp:p1} follows for all $\ssY> \ssY_{\xdf}$ thus
%\begin{multline*}
%   \E\Vnormlp{\widehat{\theta}^{(\eta)}-\xdf}^2\leq
%  3 \Vnormlp{\xdf_{\underline{0}}}^2\bias[\mDi]^2(\xdf)\\\hfill
%    +\cst{}\Vnormlp{\xdf_{\underline{0}}}^2\Ind{\{\mDi>1\}}
%    \exp\big(\tfrac{-9\cmiSv[\mdDi]\mdDi}{20000\Vnormlp[1]{\fydf}}\big)
%    + \cst{}\big[
%    \Vnormlp{\xdf_{\underline{0}}}^2\Ind{\{\mDi>1\}} \P(\aixEv[\mdDi]^c)+\ssE\P(\aixEv[K]^c) \big]
%    \\\hfill
%    +\cst{}\mRa{\xdf,\Lambda}
%    +\cst{}\ssY^{-1}\{K^2\miSv[K]^2\ssY^{-1}+\miSv[\Di_{\ydf}]^2\Di_{\ydf}^3+\miSv[\ssY_{o}]^2+\Vnormlp{\xdf_{\underline{0}}}^2\Ind{\{\mDi>1\}}\}.
%  \end{multline*}
%Exploiting \nref{re:evrest} \ref{re:evrest:ii} there is a
%numerical constant $\cst{}$ such that for all  $\ssE\in\Nz$ holds
%$\P(\aixEv[K]^c)\leq\cst{}K\miSv[K]^2\ssE^{-2}$, which
%together with
%$\mRa{\xdf,\Lambda}\leq
%\Vnormlp{\xdf_{\underline{0}}}^2\miSv[K]\ssE^{-1}$ implies
%\begin{multline}\label{au:ag:ub:pnp:p5}
%  \E\Vnormlp{\widehat{\theta}^{(\eta)}-\xdf}^2\leq \cst{}\ssY^{-1}\{K^2\miSv[K]^2\ssY^{-1}+\miSv[\Di_{\ydf}]^2\Di_{\ydf}^3+\miSv[\ssY_{o}]^2+\Vnormlp{\xdf_{\underline{0}}}^2\Ind{\{\mDi>1\}}\}\\\hfill
%+ 3 \Vnormlp{\xdf_{\underline{0}}}^2\bias[\mDi]^2(\xdf)
%    +\cst{}\Vnormlp{\xdf_{\underline{0}}}^2\Ind{\{\mDi>1\}}\{
%    \exp\big(\tfrac{-9\cmiSv[\mdDi]\mdDi}{20000\Vnormlp[1]{\fydf}}\big)
%    +  \P(\aixEv[\mdDi]^c)\}
%    \\\hfill
%    +\cst{}\ssE^{-1}\{K\miSv[K]^2+ \Vnormlp{\xdf_{\underline{0}}}^2\miSv[K]\}
%  \end{multline}
%In order to control the terms
%involving $\mdDi$ and $\mDi$ we destinguish for $\ssE\in\Nz$ with
% $\ssE(\xdf,\Lambda):=\floor{289\log(K+2)\cmiSv[K]\miSv[K]}$
%the following two cases
%cases, \begin{inparaenum}[i]\renewcommand{\theenumi}{\dgrau\rm(b-\roman{enumi})}\item\label{au:ag:ub:pnp:p:m1}
%$\ssE\in\nset{1,\ssE(\xdf,\Lambda)}$ and \item\label{au:ag:ub:pnp:p:m2}
%$\ssE>\ssE(\xdf,\Lambda)$. \end{inparaenum}
%Consider first \ref{au:ag:ub:pnp:p:m1} $\ssE\in\nset{1,\ssE(\xdf,\Lambda)}$.
%We set $\mdDi=1$ and hence
%$\mDi=1$. Thereby, with $\bias[1]^2(\xdf)\leq1$, $\log(K+2)\leq
%\tfrac{K+2}{e}\leq 2K$, $\cmiSv[\Di]\miSv[\Di]\leq K\miSv[K]^2$, and hence $\ssE(\xdf,\Lambda)\leq\cst{}K^2\miSv[K]^2$,
%from \eqref{au:ag:ub:pnp:p5} follows for all $\ssE\in\nsetro{1,\ssE(\xdf,\Lambda)}$
%\begin{multline}\label{au:ag:ub:pnp:p6}
%  \E\Vnormlp{\widehat{\theta}^{(\eta)}-\xdf}^2\leq \cst{}\ssY^{-1}\{K^2\miSv[K]^2\ssY^{-1}+\miSv[\Di_{\ydf}]^2\Di_{\ydf}^3+\miSv[\ssY_{o}]^2\}\\\hfill
%    +\cst{}\ssE^{-1}\{K\miSv[K]^2+ \Vnormlp{\xdf_{\underline{0}}}^2(K^2\miSv[K]^2+\miSv[K])\}
%  \end{multline}
%Consider now \ref{au:ag:ub:pnp:p:m2} $\ssE>\ssE(\xdf,\Lambda)$. Note that for all $\ssE> \ssE(K,\Lambda)$ the defining set of
%$\sDi{\ssE}:=\max\{\Di\in\nset{K,\ssE}:289\log(\Di+2)\cmiSv[\Di]\miSv[\Di]\leq\ssE\}$
%is not empty, where obviously for each
%$\mdDi\in\nset{K,\sDi{\ssE}}$ holds 
%$\ssE\geq289\log(\mdDi+2)\cmiSv[\mdDi]\miSv[\mdDi]$, and thus from
%\nref{re:evrest} \ref{re:evrest:iii} follows
%$\P(\aixEv[\mdDi]^c)\leq 53\ssE^{-1}$.
%Since also
%$\ssY> \ssY_{\xdf}:=\floor{c_{\xdf}\DipenSv[K]}$ with
%$c_{\xdf}:=\tfrac{\Vnormlp{\xdf_{\underline{0}}}^2+104\cpen}{\Vnormlp{\xdf_{\underline{0}}}^2\bias[{[K-1]}]^2(\xdf)}>1$
%the defining set of
%$\sDi{\ssY}:=\max\{\Di\in\nset{K,\ssY}:\ssY>c_{\xdf}\DipenSv\}$ is not
%empty. Consequently,  for all $\mdDi\in\nset{K,\sDi{\ssY}}$ holds $\mdDi \geq K$ and, hence 
%$\bias[\mdDi](\xdf)=0$, and
%$\daRaS{\mdDi}{\xdf,\Lambda}=\DipenSv[\mdDi]\ssY^{-1}<c_{\xdf}^{-1}=\tfrac{\Vnormlp{\xdf_{\underline{0}}}^2\bias[{[K-1]}]^2(\xdf)}{\Vnormlp{\xdf_{\underline{0}}}^2+104\cpen}$,
%it follows
%$\Vnormlp{\xdf_{\underline{0}}}^2\bias[{[K-1]}]^2(\xdf)>[\Vnormlp{\xdf_{\underline{0}}}^2+104\cpen]\daRaS{\mdDi}{\xdf,\Lambda}$
%and trivially
%$\Vnormlp{\xdf_{\underline{0}}}^2\bias[{K}]^2(\xdf)=0<[\Vnormlp{\xdf_{\underline{0}}}^2+104\cpen]\daRaS{\mdDi}{\xdf,\Lambda}$. Therefore, the definition \eqref{au:de:*Di:ag}
%implies $\mDi=K$ and hence
%$\bias[\mDi]^2(\xdf)=\bias[K]^2(\xdf)=0$. Selecting
%  $\mdDi:=\sDi{\ssY}\wedge\sDi{\ssE}$ we have 
%$\P(\aixEv[\mdDi]^c)\leq 53\ssE^{-1}$, 
%$\mDi=K$ and $\bias[\mDi]^2(\xdf)=0$, such that 
% from  \eqref{au:ag:ub:pnp:p5} follows for all $\ssE>\ssE(\xdf,\Lambda)$
% and $\ssY>\ssY_{\xdf,\Lambda}$
%\begin{multline}\label{au:ag:ub:pnp:p7}
%  \E\Vnormlp{\widehat{\theta}^{(\eta)}-\xdf}^2\leq \cst{}\ssY^{-1}\{K^2\miSv[K]^2\ssY^{-1}+\miSv[\Di_{\ydf}]^2\Di_{\ydf}^3+\miSv[\ssY_{o}]^2+\Vnormlp{\xdf_{\underline{0}}}^2\}\\\hfill
%    +\cst{}\Vnormlp{\xdf_{\underline{0}}}^2\{
%    \exp\big(\tfrac{-9\cmiSv[\sDi{\ssY}\wedge\sDi{\ssE}]\sDi{\ssY}\wedge\sDi{\ssE}}{20000\Vnormlp[1]{\fydf}}\big)\}    +\cst{}\ssE^{-1}\{K\miSv[K]^2+ \Vnormlp{\xdf_{\underline{0}}}^2\miSv[K]\}
%  \end{multline}
%Combining \eqref{au:ag:ub:pnp:p6} and \eqref{au:ag:ub:pnp:p7}
%for   the cases \ref{au:ag:ub:pnp:p:m1}
%$\ssE\in\nset{1,\ssE(\xdf,\Lambda)}$ and \ref{au:ag:ub:pnp:p:m2}
%$\ssE>\ssE(\xdf,\Lambda)$ we obtain for all $\ssY>\ssY_{\xdf,\Lambda}$ and for all $\ssE\in\Nz$
%\begin{multline}\label{au:ag:ub:pnp:p8}
%  \E\Vnormlp{\widehat{\theta}^{(\eta)}-\xdf}^2\leq
%  \cst{}\Vnormlp{\xdf_{\underline{0}}}^2\big[\ssY^{-1}\vee \ssE^{-1} \vee\exp\big(\tfrac{-9\cmiSv[\sDi{\ssY}\wedge\sDi{\ssE}]\sDi{\ssY}\wedge\sDi{\ssE}}{20000\Vnormlp[1]{\fydf}}\big)\big]\\
%+  \cst{}\ssY^{-1}\{K^2\miSv[K]^2+\miSv[\Di_{\ydf}]^2\Di_{\ydf}^3+\miSv[\ssY_{o}]^2\}
%    +\cst{}\ssE^{-1}(1\vee\Vnormlp{\xdf_{\underline{0}}}^2)K\miSv[K]^2
%  \end{multline}
%Combining \eqref{au:ag:ub:pnp:p4} and \eqref{au:ag:ub:pnp:p8} for $K\in\Nz$
%   with \ref{au:ag:ub:pnp:p:n1}
%$\ssY\in\nset{1,\ssY_{\xdf,\Lambda}}$ and \ref{au:ag:ub:pnp:p:n2}
%$\ssY>\ssY_{\xdf,\Lambda}$, respectively, and \eqref{au:ag:ub:pnp:p3}
%for $K=0$, we obtain for all $K\in\Nz_0$ and for all $\ssY,\ssE\in\Nz$
%\begin{multline}\label{au:ag:ub:pnp:p9}
%  \E\Vnormlp{\widehat{\theta}^{(\eta)}-\xdf}^2\leq
%  \cst{}\Vnormlp{\xdf_{\underline{0}}}^2\big[\ssY^{-1}\vee \ssE^{-1} \vee\exp\big(\tfrac{-9\cmiSv[\sDi{\ssY}\wedge\sDi{\ssE}]\sDi{\ssY}\wedge\sDi{\ssE}}{20000\Vnormlp[1]{\fydf}}\big)\big]\\
%+  \cst{}\ssY^{-1}\{\miSv[1]^2\{\tfrac{(\Vnormlp{\xdf_{\underline{0}}}^2\vee1)K^2\miSv[K]^2}{\Vnormlp{\xdf_{\underline{0}}}^2\bias[{[K-1]}]^2(\xdf)}\Ind{\{K\geq1\}}+\Ind{\{K=0\}}\}+\miSv[\Di_{\ydf}]^2\Di_{\ydf}^3+\miSv[\ssY_{o}]^2\}\\
%+\cst{}\ssE^{-1}\{(1\vee\Vnormlp{\xdf_{\underline{0}}}^2)K\miSv[K]^2\Ind{\{K\geq1\}}+\miSv[1]^2\Ind{\{K=0\}}\}.
%  \end{multline}
%  Consider the case \ref{au:ag:ub:pnp:np}. We destinguish for $\ssE\in\Nz$ with
% $\ssE(\Lambda):=\floor{289\log(3)\cmiSv[1]\miSv[1]}$
%the following two
%cases, \begin{inparaenum}[i]\renewcommand{\theenumi}{\dgrau\rm(\alph{enumi})}\item\label{au:ag:ub:pnp:np:m1}
%$\ssE\in\nset{1,\ssE(\Lambda)}$ and \item\label{au:ag:ub:pnp:np:m2}
%$\ssE>\ssE(\Lambda)$. \end{inparaenum}
%Consider firstly the case \ref{au:ag:ub:pnp:np:m1}
%$\ssE\in\nset{1,\ssE(\Lambda)}$. We set $\pdDi=\mdDi=1$, and hence
%$\mDi=1$, $\bias[1]^2(\xdf)\leq1$,
%$\penSv[1]\leq\cpen\miSv[1]^2\ssY^{-1}$, $\miSv[1]^2\leq \miSv[\ssY_{o}]^2$, $\ssE(\Lambda)\leq \cst{}\miSv[1]^2$ and due to
%\nref{re:evrest} \ref{re:evrest:ii} $\P(\aixEv[1]^c)\leq
%\cst{}\miSv[1]^2\ssE^{-2}$. Thereby, 
% \eqref{au:ag:ub:pnp:p1} implies for all $\ssY\in\Nz$ and $\ssE\in\nset{1,\ssE(\Lambda)}$
% \begin{multline}\label{au:ag:ub:pnp:p10}
%  %  \E\Vnormlp{\widehat{\theta}^{(\eta)}-\xdf}^2\leq
%  % 2\penSv[1] +\tfrac{12}{7}\Vnormlp{\xdf_{\underline{0}}}^2\bias[1]^2(\xdf)+3 \Vnormlp{\xdf_{\underline{0}}}^2\bias[1]^2(\xdf)\\\hfill
%  %   +\cst{}\miSv[1]^2\ssE^{-1} 
%  %   \\\hfill
%  %   +\cst{}\mRa{\xdf,\Lambda}
%  %   +\cst{}\ssY^{-1}\{\miSv[\Di_{\ydf}]^2\Di_{\ydf}^3+\miSv[\ssY_{o}]^2\}\\
%   \E\Vnormlp{\widehat{\theta}^{(\eta)}-\xdf}^2\leq
%   \cst{}\mRa{\xdf,\Lambda}
%    + \cst{}(1\vee\Vnormlp{\xdf_{\underline{0}}}^2)\miSv[1]^2\ssE^{-1}  
%    +\cst{}\{\miSv[\Di_{\ydf}]^2\Di_{\ydf}^3+\miSv[\ssY_{o}]^2\}\ssY^{-1}
%  \end{multline}
%Consider secondly \ref{au:ag:ub:pnp:np:m2}
%$\ssE>\ssE(\Lambda)$.  We set  
%$\sDi{\ssE}:=\max\{\Di\in\nset{1,\ssE}:289\log(\Di+2)\cmiSv[\Di]\miSv[\Di]\leq\ssE\}$, where the defining set containing $1$ is not
%empty. For each
%$\Di\in\nset{1,\sDi{\ssE}}$ holds 
%$\ssE\geq289\log(\Di+2)\cmiSv\miSv$, and thus from
%\nref{re:evrest} \ref{re:evrest:iii} follows
%$\P(\aixEv[\Di]^c)\leq 11226\ssE^{-2}$. For $\aDi{\ssY}\in\nset{1,n}$
%as in \nref{ak:ass:pen:oo} let
%$\pdDi:=\aDi{\ssY}\wedge\sDi{\ssE}$, where
%$\penSv[\aDi{\ssY}\wedge\sDi{\ssE}]\leq\penSv[\aDi{\ssY}]\leq \daRa{\aDi{\ssY}}{(\xdf,\Lambda)}$, then from \eqref{au:ag:ub:pnp:p1} follows
% \begin{multline}\label{au:ag:ub:pnp:p11}
%   \E\Vnormlp{\widehat{\theta}^{(\eta)}-\xdf}^2\leq
%   2\daRa{\aDi{\ssY}}{(\xdf,\Lambda)}+\tfrac{12}{7}\Vnormlp{\xdf_{\underline{0}}}^2\bias[\aDi{\ssY}\wedge\sDi{\ssE}]^2(\xdf)
%   +3 \Vnormlp{\xdf_{\underline{0}}}^2\bias[\mDi]^2(\xdf)\\\hfill
%    +\cst{}\Vnormlp{\xdf_{\underline{0}}}^2\Ind{\{\mDi>1\}}
%    \exp\big(\tfrac{-9\cmiSv[\mdDi]\mdDi}{20000\Vnormlp[1]{\fydf}}\big)
%    + \cst{}\big[
%    \Vnormlp{\xdf_{\underline{0}}}^2\Ind{\{\mDi>1\}} \P(\aixEv[\mdDi]^c)+\ssE^{-1} \big]
%    \\\hfill
%    +\cst{}\mRa{\xdf,\Lambda}
%    +\cst{}\ssY^{-1}\{\miSv[\Di_{\ydf}]^2\Di_{\ydf}^3+\miSv[\ssY_{o}]^2+\Vnormlp{\xdf_{\underline{0}}}^2\Ind{\{\mDi>1\}}\}
%\end{multline}
%Let
%$\sDi{\ssY}:=\argmin\{\daRa{\Di}{(\xdf,\Lambda)}\vee\exp\big(\tfrac{-\cmiSv[\Di]\Di}{\Di_{\ydf}}\big):\Di\in\nset{1,\ssY}\}$,
%where $\sDi{\ssY}\in\nset{\aDi{\ssY},1}$ by definition of
%$\aDi{\ssY}$. Setting  
%$\mdDi:=\sDi{\ssY}\wedge\sDi{\ssE}$ from
%\nref{re:evrest} \ref{re:evrest:iii} follows
%$\P(\aixEv[\mdDi]^c)\leq 53\ssE^{-1}$, while  $\mDi$ as in  definition
%\eqref{au:de:*Di:ag} satisfies    $\Vnormlp{\xdf_{\underline{0}}}^2\bias[\mDi]^2(\xdf) \leq
%  [\Vnormlp{\xdf_{\underline{0}}}^2+104\cpen]\dRa{\sDi{\ssY}\wedge\sDi{\ssE}}{\xdf,\Lambda}$,
%  where
%  $\dRa{\sDi{\ssY}\wedge\sDi{\ssE}}{\xdf,\Lambda}\leq\dRa{\sDi{\ssY}}{\xdf,\Lambda}+\bias[\aDi{\ssY}\wedge\sDi{\ssE}]^2(\xdf)$, $\dRa{\aDi{\ssY}}{\xdf,\Lambda}\leq\dRa{\sDi{\ssY}}{\xdf,\Lambda}$
%  and $\bias[\sDi{\ssY}\wedge\sDi{\ssE}]^2(\xdf)\leq
%  \bias[\aDi{\ssY}\wedge\sDi{\ssE}]^2(\xdf)$. Thereby, we obtain for all $\ssY\in\Nz$ and $\ssE>\ssE(\Lambda)$
% \begin{multline}\label{au:ag:ub:pnp:p12}
%   \E\Vnormlp{\widehat{\theta}^{(\eta)}-\xdf}^2\leq\cst{}(1\vee\Vnormlp{\xdf_{\underline{0}}}^2)\{\daRa{\sDi{\ssY}}{(\xdf,\Lambda)}+\bias[\aDi{\ssY}\wedge\sDi{\ssE}]^2(\xdf)\}\\\hfill
%    +\cst{}\Vnormlp{\xdf_{\underline{0}}}^2\Ind{\{\mDi>1\}}
%    \exp\big(\tfrac{-9\cmiSv[\sDi{\ssY}\wedge\sDi{\ssE}]\sDi{\ssY}\wedge\sDi{\ssE}}{20000\Vnormlp[1]{\fydf}}\big)
%    + \cst{}\big[
%    \Vnormlp{\xdf_{\underline{0}}}^2\Ind{\{\mDi>1\}}\ssE^{-1}+\ssE^{-1} \big]
%    \\\hfill
%    +\cst{}\mRa{\xdf,\Lambda}
%    +\cst{}\ssY^{-1}\{\miSv[\Di_{\ydf}]^2\Di_{\ydf}^3+\miSv[\ssY_{o}]^2+\Vnormlp{\xdf_{\underline{0}}}^2\Ind{\{\mDi>1\}}\}
%\end{multline}
%Since $\daRa{\sDi{\ssY}}{(\xdf,\Lambda)}\geq\ssY^{-1}$ and
%$\mRa{\xdf,\Lambda}\geq\tfrac{1}{2}\Vnormlp{\xdf_{\underline{0}}}^2 \ssE^{-1}$
%(see \nref{oo:rem:nm}) it follows  for all $\ssY\in\Nz$ and $\ssE>\ssE(\Lambda)$
%\begin{multline}\label{au:ag:ub:pnp:p13}
%  \E\Vnormlp{\widehat{\theta}^{(\eta)}-\xdf}^2\leq\cst{}(1\vee\Vnormlp{\xdf_{\underline{0}}}^2)\min_{\Di\in\nset{1,\ssY}}\{\daRa{\Di}{(\xdf,\Lambda)}\vee\exp\big(\tfrac{-\cmiSv[\Di]\Di}{\Di_{\ydf}}\big)\}\\\hfill
%+\cst{}(1\vee\Vnormlp{\xdf_{\underline{0}}}^2)\{\bias[\aDi{\ssY}\wedge\sDi{\ssE}]^2(\xdf)\vee\exp\big(\tfrac{-\cmiSv[\sDi{\ssE}]\sDi{\ssE}}{\Di_{\ydf}}\big)\} \\\hfill
%+\cst{}\mRa{\xdf,\Lambda}+\cst{}\ssE^{-1}+\cst{}\ssY^{-1}\{\miSv[\Di_{\ydf}]^2\Di_{\ydf}^3+\miSv[\ssY_{o}]^2\}
%\end{multline}
%Combining \eqref{au:ag:ub:pnp:p10} and \eqref{au:ag:ub:pnp:p13}
%for   the cases \ref{au:ag:ub:pnp:np:m1}
%$\ssE\in\nset{1,\ssE(\Lambda)}$ and \ref{au:ag:ub:pnp:np:m2}
%$\ssE>\ssE(\Lambda)$ we obtain for all $\ssY,\ssE\in\Nz$
%\begin{multline}\label{au:ag:ub:pnp:p14}
%    \E\Vnormlp{\widehat{\theta}^{(\eta)}-\xdf}^2\leq\cst{}(1\vee\Vnormlp{\xdf_{\underline{0}}}^2)\min_{\Di\in\nset{1,\ssY}}\{\daRa{\Di}{(\xdf,\Lambda)}\vee\exp\big(\tfrac{-\cmiSv[\Di]\Di}{\Di_{\ydf}}\big)\}\Ind{\{\ssE>\ssE(\Lambda)\}}\\\hfill
%+\cst{}(1\vee\Vnormlp{\xdf_{\underline{0}}}^2)\{\bias[\aDi{\ssY}\wedge\sDi{\ssE}]^2(\xdf)\vee\exp\big(\tfrac{-\cmiSv[\sDi{\ssE}]\sDi{\ssE}}{\Di_{\ydf}}\big)\}\Ind{\{\ssE>\ssE(\Lambda)\}} \\\hfill
% +\cst{}\mRa{\xdf,\Lambda}   + \cst{}(1\vee\Vnormlp{\xdf_{\underline{0}}}^2)\miSv[1]^2\ssE^{-1}  
%    +\cst{}\{\miSv[\Di_{\ydf}]^2\Di_{\ydf}^3+\miSv[\ssY_{o}]^2\}\ssY^{-1}
%\end{multline}
%which shows the assertion \eqref{au:ag:ub:pnp:e2} and  completes the
%proof of \nref{au:ag:ub:pnp}.\proEnd\end{pro}

\section{Proofs of \nref{au:mrb}}\label{a:au:mrb}
% ....................................................................
% Te <<Upper bound random weights>>
% ....................................................................
\begin{te}
 Below  we state the proofs of  \nref{au:mrb:re:SrWe:ag}. The  proof of \nref{au:mrb:re:SrWe:ag} is based on \nref{mrb:re:erWe} given first.
\end{te}
% ....................................................................
% <<Re Random weights>>
% ....................................................................
\begin{cor}\label{mrb:re:erWe} Consider the data-driven aggreagtion weights
  $\erWe[]$ as in \eqref{au:de:erWe}. Under condition
  \nref{au:ass:pen:oo} for any $l\in\nset{1,\ssY}$ with
  $\daRa{l}{(\xdfCw[],\Lambda)}=[\xdfCw[(l)]\vee \DipenSv[l]\, n^{-1}]$ holds
  \begin{resListeN}[]
  \item\label{mrb:re:erWe:i} with
    $\aixEv[l]:=\set{1/4\leq\iSv[s]^{-1}\eiSv[(s)]\leq9/4,\;\forall\;s\in\nset{1,l}}$ for all $k\in\nsetro{1,l}$ 
    we have\\
   $\erWe[(k)]\Ind{\setB{\Vnormlp{\hxdfPr[l]-\dxdfPr[l]}^2<\peneSv[(l)]/7}}\Ind{\aixEv[l]}$\\\null\hfill$\leq
  \exp\big(\rWn\big\{[\tfrac{25}{2}\cpen+\tfrac{1}{8}\xdfCr^2]\dRa{l}{\xdfCw[],\Lambda}-\tfrac{1}{8}\Vnormlp{\xdf_{\underline{0}}}^2\bias^2(\xdf)-\tfrac{1}{50}\penSv\big\}\big)$.
  \item\label{mrb:re:erWe:ii} with $\Vnormlp{\ProjC[l]\dxdfPr[\ssY]}^2=2\sum_{s=l+1}^{\ssY}\iSv[s]^{-1}\eiSv[(s)]\vert\fxdf[(s)]\vert^2$
    for all $\Di\in\nsetlo{l,\ssY}$ we have\\
    $\erWe\Ind{\setB{\Vnormlp{\hxdfPr-\dxdfPr}^2<\penSv/7}} \leq
   \exp\big(\rWn\big\{-\tfrac{1}{2}\peneSv+\tfrac{3}{2}\Vnormlp{\ProjC[l]\dxdfPr[\ssY]}^2+\peneSv[(l)]\big\}\big)$.
  \end{resListeN}
\end{cor}
% --------------------------------------------------------------------
% <<Proof Re Random weights>> angepasst
% --------------------------------------------------------------------
\begin{pro}[Proof of \nref{mrb:re:erWe}.]The assertion
  \ref{mrb:re:erWe:i} follows from \nref{re:erWe} \ref{re:erWe:i}
  using that $\xdfCr^2\daRa{\Di}{(\xdfCw[],\Lambda)}\geq\Vnormlp{\xdf_{\underline{0}}}^2\bias^2(\xdf)$
uniformely for all $\xdf\in\rwCxdf$ and for all
$\Di\in\Nz$.  The assertion
  \ref{mrb:re:erWe:ii} equals \nref{re:erWe} \ref{re:erWe:ii}.\proEnd
\end{pro}
% ....................................................................
% <<Proof Re Sum Random weights>>
% ....................................................................
\begin{pro}[Proof of \nref{au:mrb:re:SrWe:ag}.]
The proof follows line by line the proof of \nref{au:re:SrWe:ag} using
\nref{mrb:re:erWe} rather than \nref{re:erWe}, and we omit the details.\proEnd  
\end{pro}

%%% Local Variables:
%%% mode: latex
%%% TeX-master: "_0DACD"
%%% End:

\chapter{Proofs of \nref{THM_FREQ_CIRCDECONV_UNKNOWN_IID_ORACLE_NP}}\label{PRO_FREQ_CIRCDECONV_UNKNOWN_IID_ORACLE_NP}
%======================================================================================================================
%                                                                 
% Title:  Appendix: unknown error density
% Author: Jan JOHANNES, Institut für Angewandte Mathematik, Ruprecht-Karls Universität Heidelberg, Deutschland  
% 
% Email: johannes@math.uni-heidelberg.de
% Date: %%ts latex start%%[2018-03-29 Thu 13:25]%%ts latex end%%
%
% ======================================================================================================================
% --------------------------------------------------------------------
% section <<Appendix: Proofs of \cref{au}>>\ref{a:au}
% --------------------------------------------------------------------
\subsection{Proofs of \cref{au}}\label{a:au}
\begin{te}
For each
  $\Di\in\Nz$ the projections $\xdfPr=\sum_{j=-\Di}^{\Di}\fxdf[j]\bas_j$,
  $\hxdfPr=\sum_{j=-\Di}^{\Di}\hfedfmpI[j]\hfydf[j]\bas_j$ and
  $\dr\dxdfPr:=\sum_{j=-\Di}^{\Di}\hfedfmpI[j]\fydf[j]\bas_j$
  are constructed using
the sequences  $\fxdf=\Nsuite{\fxdf[j]}$, $\fydf=\Nsuite{\fydf[j]}$,
$\hfedfmpI[]=\Nsuite{\hfedfmpI[j]}$ with $\hfedfmpI[j]:=\hfedf[j]^{-1}\Ind{\{|\hfedf[j]|^2\geq1/\ssE\}}$  and  $\hfydf=\Nsuite{\hfydf[j]}$. 
\end{te}
% --------------------------------------------------------------------
% <<Proof of Re key argument>>
% --------------------------------------------------------------------
\begin{pro}[Proof of \cref{co:agg:au}.]
We start the proof with the observation that
$\ofhxdfPr[{\We[]}]{j}-\ofxdf[j]=\fhxdfPr[{\We[]}]{-j}-\fxdf[-j]$ for all $j\in\Zz$, 
$\fhxdf{0}-\fxdf[0]=0$ and
$\fhxdfPr[{\We[]}]{j}-\fxdf[j]=-\fxdf[j]$ for all $j>\ssY$, while for all
$j\in\nset{1,n}$ with $\xEv:= \{|\hfedf[j]|^2\geq1/\ssE\}$ and
$\xEv^c:= \{|\hfedf[j]|^2<1/\ssE\}$ holds
\begin{multline*}
  \fhxdfPr[{\We[]}]{j}-\fxdf[j]=(\hfedfmpI[j]\hfydf[j]-\fxdf[j])\FuVg{\We[]}(\nset{j,n})-\fxdf[j]\FuVg{\We[]}(\nsetro{1,j})% \\=
\\
=
\hfedfmpI[j](\hfydf[j]-\fydf[j])\FuVg{\We[]}(\nset{j,n})
+\hfedfmpI[j](\fedf[j]-\hfedf[j])\fxdf[j]\FuVg{\We[]}(\nset{j,n})
-\Ind{\xEv}\fxdf[j]\FuVg{\We[]}(\nsetro{1,j})-\Ind{\xEv^c}\fxdf[j]
\end{multline*}
Consequently, we  have
  \begin{multline}\label{pro:au:key:e1}
    \VnormLp{\hxdfPr[{\We[]}]-\xdf}^2% =
%    2\sum_{j\in\nset{1,n}}|\hfedfmpI[j](\hfydf[j]-\fydf[j])\FuVg{\We[]}(\nset{j,\ssY})
% +\hfedfmpI[j](\fedf[j]-\hfedf[j])\fxdf[j]\FuVg{\We[]}(\nset{j,\ssY})
% -\fxdf[j]\FuVg{\We[]}(\nsetro{1,j})|^2\Ind{\xEv}\\\hfill+2\sum_{j\in\nset{1,\ssY}}\Ind{\xEv^c}|\fxdf[j]|^2+2\sum_{j>\ssY}|\fxdf[j]|^2
\leq6\sum_{j\in\nset{1,n}}|\hfedfmpI[j]|^2|\hfydf[j]-\fydf[j]|^2\FuVg{\We[]}(\nset{j,n})
\\\hfill+6\sum_{j\in\nset{1,n}}\Ind{\xEv}|\fxdf[j]|^2\FuVg{\We[]}(\nsetro{1,j})+2\sum_{j>n}|\fxdf[j]|^2\\\hfill
+6\sum_{j\in\nset{1,n}}|\hfedfmpI[j]|^2|\fedf[j]-\hfedf[j]|^2|\fxdf[j]|^2
+2\sum_{j\in\nset{1,n}}\Ind{\xEv^c}|\fxdf[j]|^2.
 \end{multline}
Consider the first r.hs. term in
\eqref{pro:au:key:e1}. We split the sum into two parts which we
bound separately.  Precisely, given
$\dxdfPr=\sum_{j=-\Di}^{\Di}\hfedfmpI[j]\fydf[j]\bas_j$ where
$\VnormLp{\hxdfPr-\dxdfPr}^2=2\sum_{j\in\nset{1,\Di}}|\fhxdfPr{j}-\fdxdfPr{j}|^2=2\sum_{j\in\nset{1,\Di}}|\hfedfmpI[j]|^2|\hfydf[j]-\fydf[j]|^2$
it follows
\begin{multline}\label{pro:au:key:e2}
2\sum_{j\in\nset{1,\ssY}}|\hfedfmpI[j]|^2(\hfydf[j]-\fydf[j])^2
\FuVg{\We[]}(\nset{j,\ssY})\\
%=\sum_{j=1}^{\DiMa}\Ex\Vnormlp{\DiPro[(j-1)j](\hfxdf[]-\fxdf[])}^2\pM[\hw](\nset{j,\DiMa})\}\\
% \leq2 \sum_{j\in\nset{1,\pDi}}|\hfedfmpI[j]|^2|(\hfydf[j]-\fydf[j])^2 +
% 2\sum_{j\in\nsetlo{\pDi,n}}|\hfedfmpI[j]|^2(\hfydf[j]-\fydf[j])^2\sum_{l\in\nset{j,\ssY}}\We[l]\\
% = 2\sum_{j\in\nset{1,\pDi}}|\hfedfmpI[j]|^2|(\hfydf[j]-\fydf[j])^2 +
% 2\sum_{l\in\nsetlo{\pDi,\ssY}}\We[l]\sum_{j\in\nsetlo{\pDi,l}}|\hfedfmpI[j]|^2(\hfydf[j]-\fydf[j])^2\\
\leq \VnormLp{\hxdfPr[\pDi]-\dxdfPr[\pDi]}^2
+\sum_{l\in\nsetlo{\pDi,\ssY}}\We[l]\VnormLp{\hxdfPr[l]-\dxdfPr[l]}^2\\
% \leq\VnormH{\hxdfPr[\pDi]-\dSoPr[\pDi]}^2
% +\sum_{l\in\nsetlo{\pDi,\nS}}\We[l]\VnormH{\hxdfPr[l]-\dSoPr[l]}^2\Ind{\{\VnormH{\hxdfPr[l]-\dSoPr[l]}^2\geq\epenSv[l]\}}
% \\
% \hfill+(\cpen/\ssY)\sum_{l\in\nsetlo{\pDi,\nS}}\DiepenSv[l]\We[l]\Ind{\{\VnormH{\hxdfPr[l]-\dSoPr[l]}^2<\epenSv[l]\}}\\
% =\VnormH{\hxdfPr[\pDi]-\dSoPr[\pDi]}^2
% +\sum_{l\in\nsetlo{\pDi,\nS}}\We[l]\vectp{\VnormH{\hxdfPr[l]-\dSoPr[l]}^2-\epenSv[l]}\Ind{\{\VnormH{\hxdfPr[l]-\dSoPr[l]}^2\geq\epenSv[l]\}}\\
% \hfill+(\cpen/\ssY)\sum_{l\in\nsetlo{\pDi,\nS}}\DiepenSv[l]\Ind{\{\VnormH{\hxdfPr[l]-\dSoPr[l]}^2\geq\epenSv[l]\}}
% +(\cpen/\ssY)\sum_{l\in\nsetlo{\pDi,\nS}}\DiepenSv[l]\We[l]\Ind{\{\VnormH{\hxdfPr[l]-\dSoPr[l]}^2<\epenSv[l]\}}\\
\leq\VnormLp{\hxdfPr[\pDi]-\dxdfPr[\pDi]}^2
+\sum_{l\in\nsetlo{\pDi,\ssY}}\We[l]\vectp{\VnormLp{\hxdfPr[l]-\dxdfPr[l]}^2-\pen[l]/7}\\
+\tfrac{1}{7}\sum_{l\in\nsetlo{\pDi,\ssY}}\We[l]\pen[l]\Ind{\{\VnormLp{\hxdfPr[l]-\dxdfPr[l]}^2\geq\pen[l]/7\}}
+\tfrac{1}{7}\sum_{l\in\nsetlo{\pDi,\ssY}}\We[l]\pen[l]\Ind{\{\VnormLp{\hxdfPr[l]-\dxdfPr[l]}^2<\pen[l]/7\}}%\\
% =\pen[\pDi] +
% \vectp{\VnormH{\tSoPr[\pDi]-\SoPr[\pDi]}^2-\pen[\pDi]}+
% \vectp{\VnormH{\tSoPr[n]-\SoPr[n]}^2-\pen[n]} + \cpen \FuVg{\We[]}(\nsetlo{\pDi,n})\\
% =\VnormH{\tSoPr[\pDi]-\SoPr[\pDi]}^2+ 
% \vectp{\VnormH{\tSoPr[\nS]-\SoPr[\nS]}^2-\cpen\DipenSv[\nS]} + \cpen\,\DipenSv[\nS]\, \FuVg{\We[]}(\nsetlo{\pDi,n}).% \\
% \leq \pen[\peDi] +\cpen
% \FuVg{\We[]}(\nsetlo{\pDi,n}) + 2\max\setB{\vectp{\VnormH{\tSoPr[k]-\SoPr[k]}^2-\pen[k]},k\in\set{\pDi,n}}
\end{multline}
Consider the second and third r.hs. term in \eqref{pro:au:key:e1}.  Splitting the first sum into two parts we obtain
\begin{multline}\label{pro:au:key:e3}
2\sum_{j\in\nset{1,n}}\Ind{\xEv}|\fxdf[j]|^2\FuVg{\We[]}(\nsetro{1,j})+2\sum_{j>n}|\fxdf[j]|^2\\
\hspace*{5ex}\leq  2\sum_{j\in\nset{1,\mDi}}|\fxdf[j]|^2\Ind{\xEv}\FuVg{\We[]}(\nsetro{1,j})+ 2\sum_{j\in\nsetlo{\mDi,n}}|\fxdf[j]|^2+
  2\sum_{j>n}|\fxdf[j]|^2\\\hfill
\leq \VnormLp{\ProjC[0]\xdf}^2\{\FuVg{\We[]}(\nsetro{1,\mDi})+\bias[\mDi]^2(\xdf)\}
\end{multline}
Combining  \eqref{pro:au:key:e1} and the upper bounds \eqref{pro:au:key:e2}
and \eqref{pro:au:key:e3} we obtain   the assertion \eqref{co:agg:au:e1}, which completes the proof.\proEnd
\end{pro}
\subsubsection{Proofs of \cref{au:rb}}\label{a:au:rb}
% ....................................................................
% Te <<Upper bound random weights>>
% ....................................................................
\begin{te}
 Below  we state the proofs of  \cref{au:re:SrWe:ag} and \cref{au:re:SrWe:ms}. The
  proof of \cref{au:re:SrWe:ag} is based on \cref{re:erWe} given first.
\end{te}
% ....................................................................
% <<Re Random weights>>
% ....................................................................
\begin{lm}\label{re:erWe} Consider the data-driven aggreagtion weights
  $\erWe[]$ as in \eqref{au:de:erWe}. Under condition
  \ref{au:ass:pen:oo} for any $l\in\nset{1,\ssY}$ with
  $\daRa{l}{\xdf,\iSv}=[\bias[l]^2(\xdf)\vee \DipenSv[l]\, n^{-1}]$ holds
  \begin{resListeN}[]
  \item\label{re:erWe:i} with
    $\dr\aixEv[l]:=\set{1/4\leq\iSv[j]^{-1}\eiSv[j]\leq9/4,\;\forall\;j\in\nset{1,l}}$ for all $k\in\nsetro{1,l}$ 
    we have\\
   $\erWe[k]\Ind{\setB{\VnormLp{\hxdfPr[l]-\dxdfPr[l]}^2<\peneSv[l]/7}}\Ind{\aixEv[l]}$\\\null\hfill$\leq
  \exp\big(\rWn\big\{[\tfrac{25}{2}\cpen+\tfrac{1}{8}\VnormLp{\ProjC[0]\xdf}^2]\dRa{l}{\xdf,\iSv}-\tfrac{1}{8}\VnormLp{\ProjC[0]\xdf}^2\bias^2(\xdf)-\tfrac{1}{50}\penSv\big\}\big)$.
  \item\label{re:erWe:ii} with $\VnormLp{\ProjC[l]\dxdfPr[\ssY]}^2=2\sum_{j=l+1}^{\ssY}\iSv[j]^{-1}\eiSv[j]|\fxdf[j]|^2$
    for all $\Di\in\nsetlo{l,\ssY}$ we have\\
    $\erWe\Ind{\setB{\VnormLp{\hxdfPr-\dxdfPr}^2<\penSv/7}} \leq
   \exp\big(\rWn\big\{-\tfrac{1}{2}\peneSv+\tfrac{3}{2}\VnormLp{\ProjC[l]\dxdfPr[\ssY]}^2+\peneSv[l]\big\}\big)$.
  \end{resListeN}
\end{lm}
% --------------------------------------------------------------------
% <<Proof Re Random weights>>
% --------------------------------------------------------------------
\begin{pro}[Proof of \cref{re:erWe}.]
Given $\Di,l\in\nset{1,n}$ and an event $\dmEv{\Di}{l}$ (to be specified below) it clearly follows
\begin{multline}\label{p:re:erWe:e1}
 \erWe\Ind{\dmEv{\Di}{l}}=\frac{\exp(-\rWn\{-\VnormLp{\hxdfPr}^2+\peneSv\})}{\sum_{l\in\nset{1,\ssY}}\exp(-\rWn\{-\VnormLp{\hxdfPr[l]}^2+\peneSv[l]\})}\Ind{\dmEv{\Di}{l}}\\
\leq
\exp\big(\rWn\big\{\VnormLp{\hxdfPr}^2-\VnormLp{\hxdfPr[l]}^2+(\peneSv[l]-\peneSv)\big\}\big)\Ind{\dmEv{\Di}{l}}
\end{multline}
We distinguish the two cases \ref{re:erWe:i} $\Di\in\nsetro{1,l}$ and
\ref{re:erWe:ii} $\Di\in\nsetlo{l,\ssY}$.
Consider first  \ref{re:erWe:i} $\Di\in\nsetro{1,l}$. From \ref{re:contr:e1} in \cref{re:contr}
(with
$\dxdfPr[\bullet]=\hxdfPr[\ssY]$
and  $\xdf=\dxdfPr[\ssY]=\sum_{j\in\nset{-\ssY,\ssY}}\hfedfmpI[j]\fydf[j]\bas_j$) follows that
\begin{multline}\label{p:re:erWe:e2}
 \erWe\Ind{\dmEv{\Di}{l}}
\leq
\exp\big(\rWn\big\{\VnormLp{\hxdfPr}^2-\VnormLp{\hxdfPr[l]}^2+(\peneSv[l]-\peneSv)\big\}\big)\Ind{\dmEv{\Di}{l}}\\
%\exp\big(\rWn\big\{\contr[](\tSoPr[l])-\contr[](\tSoPr)+\tfrac{9}{2}(\penSv[l]-\penSv)\big\}\big)\Ind{\dmEv{\Di}{l}}\\
\leq
\exp\big(\rWn\big\{\tfrac{11}{2}\VnormLp{\hxdfPr[l]-\dxdfPr[l]}^2-\tfrac{1}{2}\VnormLp{\dProj{\Di}{l}\dxdfPr[\ssY]}^2
+(\peneSv[l]-\peneSv)\big\}\big)\Ind{\dmEv{k}{l}}
\end{multline}
Note that on the event
$\aixEv[l]:=\set{1/2\leq|\fedf[j]\hfedfmpI[j]|\leq3/2,\;\forall\;j\in\nset{1,l}}$
we have
\begin{multline*}
\VnormLp{\dProj{\Di}{l}\dxdfPr[\ssY]}^2\Ind{\aixEv[l]}\geq
\tfrac{1}{4}\VnormLp{\dProj{\Di}{l}\xdf}^2=\tfrac{1}{4}\VnormLp{\ProjC[0]\xdf}^2(\bias^2(\xdf)-\bias[l]^2(\xdf))\hfill\\
\meiSv[l]\Ind{\aixEv[l]}=\max\set{\eiSv[j]=(\hfedfmpI[j])^2,j\in\nset{1,l}}\Ind{\aixEv[l]}\leq\tfrac{9}{4}\max\set{\iSv[j]=\fedf[j]^{-2},j\in\nset{1,l}}=\tfrac{9}{4}\miSv[l]\hfill\\
\meiSv[l]\Ind{\aixEv[l]}\geq\tfrac{1}{4}\miSv[l]\hfill
\end{multline*}
Thus on $\aixEv[l]$ holds  
$\tfrac{1}{4}l\miSv[l]\vee(l+2)\leq l\meiSv[l]\vee(l+2)\leq
\tfrac{9}{4}l\miSv[l]\vee(l+2) $. Since 
$\sqrt{\cmiSv[l]}=\tfrac{\log (l\miSv[l] \vee (l+2))}{\log(l+2)}\geq1 $
 for all $l\in\Nz$ hold
$\tfrac{\log (\tfrac{1}{4}l\miSv[l] \vee (l+2))}{\log(l+2)}\geq
\sqrt{\cmiSv[l]}\tfrac{\log(3/4)}{\log 3}\geq \tfrac{3}{10}\sqrt{\cmiSv[l]}$
and $\tfrac{\log (\tfrac{9}{4}l\miSv[l] \vee
  (l+2))}{\log(l+2)}\leq\sqrt{\cmiSv[l]}\tfrac{\log(27/4)}{\log 3}\leq
\tfrac{7}{4}\sqrt{\cmiSv[l]}$
which together with $\DipenSv[l]=l\cmiSv[l]\miSv[l]$ and $\DipeneSv[l]=l\cmeiSv[l]\meiSv[l]$ imply 
\begin{multline}\label{pro:au:erWe:e3}
\tfrac{3}{10}\sqrt{\cmiSv[l]}\leq\sqrt{\cmeiSv[l]}\Ind{\aixEv[l]}\leq
\tfrac{7}{4}\sqrt{\cmiSv[l]}\hfill\\
\tfrac{1}{50}\DipenSv[l]\leq\tfrac{9}{400}\DipenSv[l]=l\,\tfrac{9}{100}\cmiSv[l]\,\tfrac{1}{4}\miSv[l]\leq
l\,\cmeiSv[l]\,\meiSv[l]\Ind{\aixEv[l]}=\DipeneSv[l]\Ind{\aixEv[l]}\leq l\,\tfrac{49}{16}\cmiSv[l]\,\tfrac{9}{4}\miSv[l]=\tfrac{441}{64}\DipenSv[l]\leq7\DipenSv[l]\hfill
\end{multline}
and hence  for $\penSv=\cpen\DipenSv$ and $\peneSv=\cpen\DipeneSv$
follows
\begin{equation}\label{pro:au:erWe:peneSv}
\tfrac{1}{50}\penSv\leq\peneSv\Ind{\aixEv[l]}\leq7\penSv\quad\text{for
all }\Di\in\nset{1,l}\text{ and for all }l\in\Nz.
\end{equation}
If we define $\dmEv{k}{l}:=\{\VnormLp{\hxdfPr[l]-\dxdfPr[l]}^2<\peneSv[l]/7\}\cap\aixEv[l]$
then the last bounds imply
\begin{multline*}
 \erWe\Ind{\setB{\VnormLp{\hxdfPr[l]-\dxdfPr[l]}^2<\peneSv[l]/7}}\Ind{\aixEv[l]}
\\\leq\exp\big(\rWn\big\{\tfrac{11}{14}\peneSv[l]-\tfrac{1}{8}\VnormLp{\ProjC[0]\xdf}^2(\bias^2(\xdf)-\bias[l]^2(\xdf))+(\peneSv[l]-\peneSv)\big\}\big)\Ind{\aixEv[l]}\\
\\=\exp\big(\rWn\big\{\tfrac{25}{14}\peneSv[l]-\tfrac{1}{8}\VnormLp{\ProjC[0]\xdf}^2(\bias^2(\xdf)-\bias[l]^2(\xdf))-\peneSv\big\}\big)\Ind{\aixEv[l]}\\
\hfill\leq\exp\big(\rWn\big\{7*\tfrac{25}{14}\penSv[l]+\tfrac{1}{8}\VnormLp{\ProjC[0]\xdf}^2(\bias[l]^2(\xdf)-\bias^2(\xdf))-\tfrac{1}{50}\penSv\big\}\big)% \\
% \leq\exp\big(\rWn\big\{\tfrac{630}{16}*\penSv[l]-\tfrac{1}{8}\VnormLp{\Proj[{\mHiH[0]^\perp}]\xdf}^2(\bias[k]^2(\xdf)-\bias[l]^2(\xdf))\}\big)
\end{multline*}
 and hence, by exploiting \ref{ak:ass:pen:oo} for
 $\dRa{l}{\xdf,\iSv}=[\bias[l]^2(\xdf)\vee \DipenSv[l] n^{-1}]$  follows the
 assertion \ref{re:erWe:i}, that is
\begin{multline*}
  \erWe[k]\Ind{\setB{\VnormLp{\hxdfPr[l]-\dxdfPr[l]}^2<\peneSv[l]/7}}\Ind{\aixEv[l]}\\\leq
  \exp\big(\rWn\big\{[\tfrac{25}{2}\cpen+\tfrac{1}{8}\VnormLp{\ProjC[0]\xdf}^2]\dRa{l}{\xdf,\iSv}-\tfrac{1}{8}\VnormLp{\ProjC[0]\xdf}^2\bias^2(\xdf)-\tfrac{1}{50}\penSv\big\}\big).
 \end{multline*}
Consider secondly \ref{re:erWe:ii} $\Di\in\nsetlo{l,\ssY}$. From 
\ref{re:contr:e2} in \cref{re:contr} (with $\dxdfPr[\bullet]=\hxdfPr[\ssY]$
and  $\xdf=\dxdfPr[\ssY]=\sum_{j\in\nset{-\ssY,\ssY}}\hfedfmpI[j]\fydf[j]\bas_j$) and \eqref{p:re:erWe:e1} follows 
 \begin{multline}\label{p:re:erWe:e4}
  \erWe\Ind{\dmEv{l}{\Di}}
\leq\exp\big(\rWn\big\{\VnormLp{\hxdfPr}^2-\VnormLp{\hxdfPr[l]}^2+(\peneSv[l]-\peneSv)\big\}\big)\Ind{\dmEv{l}{\Di}}\\
\leq \exp\big(\rWn\big\{\tfrac{7}{2}\VnormLp{\hxdfPr[k]-\dxdfPr[k]}^2+\tfrac{3}{2}\VnormLp{\dProj{l}{k}\dxdfPr[\ssY]}^2+(\peneSv[l]-\peneSv)\big\}\big)\Ind{\dmEv{l}{k}}
\end{multline}
Keep in mind that 
$\VnormLp{\dProj{l}{k}\dxdfPr[\ssY]}^2\Ind{\aixEv[l]}%=\sum_{j=l+1}^k(\hfedfmpI[j])^2|\fydf[j]|^2
=2\sum_{j=l+1}^k(\fedf[j]\hfedfmpI[j])^2|\fxdf[j]|^2\leq2\sum_{j=l+1}^{\ssY}(\fedf[j]\hfedfmpI[j])^2|\fxdf[j]|^2=\VnormLp{\ProjC[{\mHiH[l]}]\dxdfPr[\ssY]}^2$. 
If we set $\dmEv{l}{\Di}:=\{\VnormLp{\hxdfPr-\dxdfPr}^2<\peneSv/7\}$
 then we clearly have \ref{re:erWe:ii}, that is
 \begin{displaymath}
   \erWe\Ind{\setB{\VnormLp{\hxdfPr-\dxdfPr}^2<\penSv/7}} \leq
   \exp\big(\rWn\big\{-\tfrac{1}{2}\peneSv+\tfrac{3}{2}\VnormLp{\ProjC[l]\dxdfPr[\ssY]}^2+\peneSv[l]\big\}\big)
 \end{displaymath}
 which completes the proof.\proEnd
\end{pro}
% ....................................................................
% <<Proof Re Sum Random weights>>
% ....................................................................
\begin{pro}[Proof of \cref{au:re:SrWe:ag}.]
  Consider \ref{au:re:SrWe:ag:i}. For the non trivial case $\mDi>1$
  from \cref{re:erWe} \ref{re:erWe:i} with $l=\mdDi$ follows for all
  $\Di<\mDi\leq \mdDi$, and hence due to the definition
  \eqref{au:de:*Di:ag}
  $\VnormLp{\ProjC[0]\xdf}^2\bias^2\geq
  \VnormLp{\ProjC[0]\xdf}^2\bias[\mDi-1]^2>[\VnormLp{\ProjC[0]\xdf}^2+8*13\cpen]\daRa{\mdDi}{\xdf,\iSv}$.
  Exploiting the last bound we obtain for each $\Di\in\nsetro{1,\mDi}$
  \begin{multline*}
    \erWe\Ind{\setB{\VnormLp{\hxdfPr[\mdDi]-\dxdfPr[\mdDi]}^2<\peneSv[\mdDi]/7}\cap\aixEv[l]}
    \\\hfill\leq
    \exp\big(\rWn\big\{-\tfrac{\VnormLp{\ProjC[0]\xdf}^2}{8}\bias^2(\xdf)
    +[\tfrac{25\cpen}{2}+\tfrac{\VnormLp{\ProjC[0]\xdf}^2}{8}]\daRaS{\mdDi}{\xdf,\iSv}-\tfrac{1}{50}\penSv\big\}\big)\\
    \hfill
    \leq\exp\big(-\tfrac{1}{2}\rWc\cpen \ssY\daRaS{\mdDi}{\xdf,\iSv}-\tfrac{1}{50}\rWn\penSv\big)
  \end{multline*}
  which in turn with
  $\dr\penSv=\cpen \Di\cmiSv\miSv \ssY^{-1}\geq \cpen\Di\ssY^{-1}$ and
  $\dr\sum_{\Di\in\Nz}\exp(-\mu\Di)\leq \mu^{-1}$ for any $\mu>0$
  implies \ref{ak:re:SrWe:ag:i}, that is,
  \begin{multline*}
    \FuVg{\rWe[]}(\nsetro{1,\mDi})\leq
    \FuVg{\rWe[]}(\nsetro{1,\mDi})\Ind{\setB{\VnormLp{\hxdfPr[\mdDi]-\dxdfPr[\mdDi]}^2<\peneSv[\mdDi]/7}\cap\aixEv[\mdDi]}
    +\Ind{\setB{\VnormLp{\hxdfPr[\mdDi]-\dxdfPr[\mdDi]}^2\geq\peneSv[\mdDi]/7}\cup\aixEv[\mdDi]^c}\\
    \hfill\leq\exp\big(-\tfrac{\rWc\cpen}{2}\ssY\daRaS{\mdDi}{\xdf,\iSv}\big)\sum_{k=1}^{\mDi-1}\exp(-\tfrac{\rWc\cpen}{50}\Di)
    +\Ind{\setB{\VnormLp{\hxdfPr[\mdDi]-\dxdfPr[\mdDi]}^2\geq\peneSv[\mdDi]/7}\cup\aixEv[\mdDi]^c}\\
    \leq \tfrac{50}{\rWc\cpen}\exp\big(-\tfrac{\rWc\cpen}{2}\ssY\daRaS{\mdDi}{\xdf,\iSv}\big)
    +\Ind{\setB{\VnormLp{\hxdfPr[\mdDi]-\dxdfPr[\mdDi]}^2\geq\peneSv[\mdDi]/7}\cup\aixEv[\mdDi]^c}.
  \end{multline*} 
  Consider \ref{au:re:SrWe:ag:ii}. From \cref{re:erWe} \ref{re:erWe:ii}
  with $l=\pdDi$ follows for all $\Di>\pDi\geq \pdDi$, and hence due
  to the definition \eqref{au:de:*Di:ag}
  $\peneSv > 6\VnormLp{\ProjC[0]\xdf}^2+ 4\peneSv$. Thereby, we
  obtain for $\Di\in\nsetlo{\mDi,n}$
  \begin{multline*}
    \erWe\Ind{\setB{\VnormLp{\hxdfPr-\dxdfPr}^2<\penSv/7}} \leq
   \exp\big(\rWn\big\{-\tfrac{1}{2}\peneSv+\tfrac{3}{2}\VnormLp{\ProjC[\pdDi]\dxdfPr[\ssY]}^2+\peneSv[\pdDi]\big\}\big)
    % \rWe\Ind{\setB{\VnormLp{\txdfPr-\xdfPr}^2<\penSv/7}}
    % \leq % \exp\big(\rWn\big\{-\tfrac{1}{2}\pen +[\tfrac{3}{2}\VnormLp{\So}^2+\cpen]\dRa{\pdDi}(\So)\big\}\big)
    % \\
    % =
    % \exp\big(\rWn\big\{-\tfrac{1}{4} \penSv
    % -\tfrac{1}{4}\penSv 
    % +[\tfrac{3}{2}\VnormLp{\ProjC[0]\xdf}^2+\cpen]\daRaS{\pdDi}{\xdf,\iSv}\big\}\big)
    \\
    \leq \exp\big(\rWn\big\{-\tfrac{1}{4} \peneSv\big\}\big).
  \end{multline*}
Note that $|\hfedf[j]|^2\leq1$ for all $j\in\Zz$, hence if
$|\hfedf[j]|^2\geq1/\ssE$ then $\eiSv[j]=|\hfedfmpI[j]|^2\geq1$. Thereby,
$\eiSv[j]=|\hfedfmpI[j]|^2<1$ implies $|\hfedf[j]|^2<1/\ssE$ and hence
$\eiSv[j]=|\hfedfmpI[j]|^2=0$. Thereby
$1>\meiSv=\max\{|\hfedfmpI[j]|^2,j\in\nset{1,\Di}\}$ implies
$\meiSv=0$, that is,
\begin{equation}\label{p:au:re:SrWe:ag:hPhi}
{\dr\{\meiSv<1\}=\{\meiSv=0\}}\text{ and consequently, }\peneSv=\cpen\cmeiSv \Di \meiSv=\peneSv\Ind{\{\meiSv\geq1\}}
\end{equation}
Thereby, it   follows 
  \begin{multline}\label{p:au:re:SrWe:ag:e1}
\sum_{\Di\in\nsetlo{\pDi,\ssY}}\peneSv\erWe\Ind{\{\VnormLp{\hxdfPr[k]-\dxdfPr[k]}^2\leq\peneSv/7\}}
\leq \sum_{\Di\in\nsetlo{\pDi,\ssY}}\peneSv\exp\big(-\tfrac{\rWc}{4}\ssY\peneSv\big)
\\\hfill=\sum_{\Di\in\nsetlo{\pDi,\ssY}}\peneSv\exp\big(-\tfrac{\rWc}{4}\ssY\peneSv\big)\{\Ind{\{\meiSv\geq1\}}+\Ind{\{\meiSv<1\}}\}\\
\hfill=\sum_{\Di\in\nsetlo{\pDi,\ssY}}\peneSv\exp\big(-\tfrac{\rWc}{4}\ssY\peneSv\big)\Ind{\{\meiSv\geq1\}}\\
=\cpen\ssY^{-1}\sum_{\Di\in\nsetlo{\pDi,\ssY}} \Di\cmeiSv\meiSv\exp\big(-\tfrac{\rWc\cpen}{4}\Di\cmeiSv\meiSv\big)\Ind{\{\meiSv\geq1\}}
  \end{multline}
  Exploiting that
  $\dr\sqrt{\cmeiSv}=\tfrac{\log (\Di\meiSv \vee
    (\Di+2))}{\log(\Di+2)}\geq1$, $\dr\cpen/4\geq2\log(3e)$ and
  $\dr\rWc\geq1$, then for all $k\in\Nz$ we have
  $\tfrac{\rWc\cpen}{4} k-\log(k+2)\geq1$, and hence by
  $a\exp(-ab)\leq \exp(-b)$ for $a,b\geq1$, it follows
  \begin{multline}\label{p:au:re:SrWe:ag:e2}
    \cmeiSv\Di \meiSv\exp\big(-\tfrac{\rWc\cpen}{4}\cmeiSv\Di\meiSv\big)\Ind{\{\meiSv\geq1\}}
    \leq\cmeiSv\exp\big(-\tfrac{\rWc\cpen}{4}\cmeiSv\Di\meiSv + \sqrt{\cmeiSv}\log(\Di+2)\big)\Ind{\{\meiSv\geq1\}}
    \\\hfill\leq
    \cmeiSv\exp\big(-\cmeiSv(\tfrac{\rWc\cpen}{4}\Di-\log(\Di+2))\big)\Ind{\{\meiSv\geq1\}}
    \leq\exp\big(-(\tfrac{\rWc\cpen}{4}\Di-\log(\Di+2))\big)\Ind{\{\meiSv\geq1\}}\\
    =(\Di+2)\exp\big(-\tfrac{\rWc\cpen}{4}\Di\big)\Ind{\{\meiSv\geq1\}}\leq (\Di+2)\exp\big(-\tfrac{\rWc\cpen}{4}\Di\big).
  \end{multline}
  Exploiting $\sum_{\Di\in\Nz}\mu\Di\exp(-\mu\Di)\leq \mu^{-1}$ und
  $\sum_{\Di\in\Nz}\mu\exp(-\mu\Di)\leq 1$ we obtain
  \begin{displaymath}
    \sum_{k=\pDi+1}^{\ssY}\cmeiSv\Di \meiSv\exp\big(-\tfrac{\rWc\cpen}{4}\cmeiSv\Di\meiSv\big)\Ind{\{\meiSv\geq1\}}
    \leq \sum_{k=\pDi+1}^\infty(\Di+2)\exp\big(-\tfrac{\rWc\cpen}{4}\Di\big)
    \leq \tfrac{16}{\cpen^2\rWc^{2}}+ \tfrac{8}{\cpen\rWc}.
  \end{displaymath}
  Combining the last bound and \eqref{p:au:re:SrWe:ag:e1} we obtain the
  assertion \ref{au:re:SrWe:ag:ii}, that is
  \begin{displaymath}
    \sum_{\Di\in\nsetlo{\pDi,n}}\peneSv\erWe\Ind{\{\VnormLp{\hxdfPr-\dxdfPr}^2\leq\peneSv/7\}}
    \leq \ssY^{-1}\{\tfrac{16}{\cpen\rWc^{2}}+ \tfrac{8}{\rWc}\}
  \end{displaymath}
  which completes the proof.\proEnd
\end{pro}
% --------------------------------------------------------------------
% <<Proof Re Sum MS Random weights>>
% --------------------------------------------------------------------
\begin{pro}[Proof of \cref{au:re:SrWe:ms}.]
  By definition of $\hDi$ it holds
  $-\VnormLp{\hxdfPr[\hDi]}^2+\peneSv[\hDi]\leq
  -\VnormLp{\hxdfPr}^2+\peneSv$ for all $\Di\in\nset{1,\ssY}$, and
  hence
  \begin{equation}\label{au:re:SrWe:ms:pr:e1}
    \VnormLp{\hxdfPr[\hDi]}^2-\VnormLp{\hxdfPr}^2\geq
    \peneSv[\hDi]-\peneSv\text{ for all }\Di\in\nset{1,\ssY}.
  \end{equation}
  Consider \ref{ak:re:SrWe:ms:i}. It is sufficient to show, that 
  $\{\hDi\in\nsetro{1,\mDi}\}\cap\aixEv[\mdDi]\subseteq
  \{\VnormLp{\hxdfPr[\mdDi]-\dxdfPr[\mdDi]}^2\geq\peneSv[\mdDi]/7\}$ holds for $\mDi>1$.  On the event $\{\hDi\in\nsetro{1,\mDi}\}$ holds
  $1\leq\hDi<\mDi\leq\mdDi$ and thus by definition
  \eqref{au:de:*Di:ag}
  \begin{equation}\label{au:re:SrWe:ms:pr:e2}
    \VnormLp{\ProjC[0]\xdf}^2\bias[\hDi]^2(\xdf)>
    [\VnormLp{\ProjC[0]\xdf}^2+104\cpen]\daRa{\mdDi}{\xdf,\iSv}
  \end{equation}
  and due to  \cref{re:contr} \ref{re:contr:e1} 
(with
$\dxdfPr[\bullet]=\hxdfPr[\ssY]$
and  $\xdf=\dxdfPr[\ssY]=\sum_{j\in\nset{-\ssY,\ssY}}\hfedfmpI[j]\fydf[j]\bas_j$) also
  \begin{equation}\label{au:re:SrWe:ms:pr:e3}
    \VnormLp{\hxdfPr[\hDi]}^2-\VnormLp{\hxdfPr[\mdDi]}^2\leq
    \tfrac{11}{2}\VnormLp{\hxdfPr[\mdDi]-\dxdfPr[\mdDi]}^2
    -\tfrac{1}{2}\VnormLp{\dProj{\hDi}{\mdDi}\dxdfPr[\ssY]}^2.
  \end{equation}
  On $\{\hDi\in\nsetro{1,\mDi}\}\cap\aixEv[\mdDi]$ we have
\begin{equation}\label{au:re:SrWe:ms:pr:e3:b}
\VnormLp{\dProj{\hDi}{\mdDi}\dxdfPr[\ssY]}^2\Ind{\aixEv[\mdDi]}\geq
\tfrac{1}{4}\VnormLp{\dProj{\hDi}{\mdDi}\xdf}^2=\tfrac{1}{4}\VnormLp{\ProjC[0]\xdf}^2(\bias[\hDi]^2(\xdf)-\bias[\mdDi]^2(\xdf))
\end{equation}
Combining, \eqref{au:re:SrWe:ms:pr:e1}, \eqref{au:re:SrWe:ms:pr:e3}  and
  \eqref{au:re:SrWe:ms:pr:e3:b} it follows that
  \begin{multline*}
    \tfrac{11}{2}\VnormLp{\hxdfPr[\mdDi]-\dxdfPr[\mdDi]}^2\geq
    \peneSv[\hDi]-\peneSv[\mdDi]
    +\tfrac{1}{8}\VnormLp{\ProjC[0]\xdf}^2\{\bias[\hDi]^2(\xdf)-\bias[\mdDi]^2(\xdf)\}\hfill
  \end{multline*}
  and hence together with $\peneSv[\hDi]\geq0$,  $\peneSv[\mdDi]\Ind{\aixEv[\mdDi]}\leq7\penSv[\mdDi]$ 
  by \eqref{pro:au:erWe:peneSv},  \eqref{au:re:SrWe:ms:pr:e2}
  and \ref{au:ass:pen:oo} we obtain the claim, that is
  \begin{multline*}
    \tfrac{11}{2}\VnormLp{\hxdfPr[\mdDi]-\dxdfPr[\mdDi]}^2\geq
    \tfrac{1}{8}\VnormLp{\ProjC[0]\xdf}^2\bias[\hDi]^2(\xdf)-
    \tfrac{1}{8}\VnormLp{\ProjC[0]\xdf}^2\bias[\mdDi]^2(\xdf)
    -\peneSv[\mdDi]\\
    >\tfrac{1}{8}[\VnormLp{\ProjC[0]\xdf}^2+104\cpen]\daRa{\mdDi}{\xdf,\iSv}
    -\tfrac{1}{8}\VnormLp{\ProjC[0]\xdf}^2\bias[\mdDi]^2(\xdf)-\peneSv[\mdDi]\\
    \geq 13\cpen \daRa{\mdDi}{\xdf,\iSv}-\peneSv[\mdDi]\geq\tfrac{26}{14}\peneSv[\mdDi]-\peneSv[\mdDi]  \geq\tfrac{11}{14}\peneSv[\mdDi],
  \end{multline*}
which  shows \ref{au:re:SrWe:ms:i}.  Consider \ref{ak:re:SrWe:ms:ii}. It is sufficient to show that,
  $\{\hDi\in\nsetlo{\pDi,\ssY}\}\subseteq
  \{\VnormLp{\hxdfPr[\hDi]-\dxdfPr[\hDi]}^2\geq\peneSv[\hDi]/7\}$.  On the
  event $\{\hDi\in\nsetlo{\pDi,\ssY}\}$ holds $\hDi>\pDi\geq\pdDi$ and
  thus by definition \eqref{au:de:*Di:ag}
  \begin{equation}\label{au:re:SrWe:ms:pr:e4}
    \peneSv[\hDi] > 2[3\VnormLp{\ProjC[\pdDi]\dxdfPr[\ssY]}^2+2\peneSv[\pdDi]]
  \end{equation}
  and due to   \cref{re:contr} \ref{re:contr:e2} 
(with
$\dxdfPr[\bullet]=\hxdfPr[\ssY]$
and
$\xdf=\dxdfPr[\ssY]=\sum_{j\in\nset{-\ssY,\ssY}}\hfedfmpI[j]\fydf[j]\bas_j$)
also
\begin{equation}\label{au:re:SrWe:ms:pr:e5}
  \VnormLp{\dxdfPr[\hDi]}^2-\VnormLp{\dxdfPr[\pdDi]}^2\leq
  \tfrac{7}{2}\VnormLp{\dxdfPr[\hDi]-\xdfPr[\hDi]}^2+\tfrac{3}{2}\VnormLp{\dProj{\pdDi}{\hDi}\dxdfPr[\ssY]}^2.
  \end{equation}
  Combining, \eqref{au:re:SrWe:ms:pr:e1} and \eqref{au:re:SrWe:ms:pr:e5} it
  follows that
  \begin{multline*}
    \tfrac{7}{2}\VnormLp{\hxdfPr[\hDi]-\dxdfPr[\hDi]}^2\geq
    \peneSv[\hDi]-\peneSv[\pdDi]  -\tfrac{3}{2}\VnormLp{\dProj{\pdDi}{\hDi}\dxdfPr[\ssY]}^2\hfill
  \end{multline*}
  and hence together with
  $\VnormLp{\dProj{\pdDi}{\hDi}\dxdfPr[\ssY]}^2\leq\VnormLp{\ProjC[\pdDi]\dxdfPr[\ssY]}^2$ and
  \eqref{au:re:SrWe:ms:pr:e4} we obtain the claim,
  that is
  \begin{multline*}
    \tfrac{7}{2}\VnormLp{\hxdfPr[\hDi]-\dxdfPr[\hDi]}^2\geq
    (\tfrac{1}{2}+\tfrac{1}{2})\peneSv[\hDi]-\peneSv[\pdDi]  -\tfrac{3}{2}\VnormLp{\ProjC[\pdDi]\dxdfPr[\ssY]}^2\\
    >\tfrac{1}{2}\penSv[\hDi]+\tfrac{1}{2}2[3\VnormLp{\ProjC[\pdDi]\dxdfPr[\ssY]}^2+2\peneSv[\pdDi]]-\peneSv[\pdDi]-\tfrac{3}{2}\VnormLp{\ProjC[\pdDi]\dxdfPr[\ssY]}^2
    \geq\tfrac{1}{2}\peneSv[\hDi],
  \end{multline*}
  which shows \ref{ak:re:SrWe:ms:ii} and completes the proof.\proEnd
\end{pro}
% ....................................................................
% <<Re rest>>
% ....................................................................
\begin{lm}\label{au:re:rest}
Consider  
$\hxdfPr-\dxdfPr=\sum_{j\in\nset{-\Di,\Di}}\hfedfmpI[j](\hfydf[j]-\fydf[j])\bas_j$.
Conditionally on $\rE_1,\dotsc,\rE_{\ssE}$ the r.v.'s
$\rY_1,\dotsc,\rY_{\ssY}$ are \iid and we  denote by $\FuVg[\ssY]{\rY|\rE}$ and $\FuEx[\ssY]{\rY|\rE}$ their conditional
 distribution and expectation, respectively. 
Let $\eiSv[j]=|\hfedfmpI[j]|^2$,
$\oeiSv=\tfrac{1}{\Di}\sum_{j\in\nset{1,\Di}}\eiSv[j]$, $\meiSv=
  \max\{\eiSv[j],j\in\nset{1,\Di}\}$, $\cpen\geq1$, $\DiepenSv=\cmeiSv\Di \meiSv$
 and $\sqrt{\cmeiSv}=\tfrac{\log (\Di\meiSv \vee (\Di+2))}{\log(\Di+2)}\geq1$.  Then there is a numerical constant $\cst{}$ such
 that for all $\ssY\in\Nz$ and $\Di\in\nset{1,\ssY}$ holds
  \begin{resListeN}[]
  \item\label{au:re:rest:i} let $\dr\Di_{\ydf}:=\floor{  3(6\Vnormlp[1]{\fydf})^2}$ and $\dr \ssY_{o}:={15(200)^4}$ then\\ 
    $ \sum_{\Di=1}^{\ssY}\FuEx[\ssY]{\rY|\rE} \vectp{\VnormLp{\hxdfPr-\dxdfPr}^2 - 12\DiepenSv\ssY^{-1}} 
\leq \cst{}\ssY^{-1}\big[(1\vee\meiSv[\Di_{\ydf}])\Di_{\ydf}+(1\vee\meiSv[\ssY_{o}])\big]$
  \item\label{au:re:rest:ii} let
    $\dr\Di_{\ydf}:=\floor{3(400\Vnormlp[1]{\fydf})^2}$ and
    $\dr \ssY_{o}:=15({600})^4$ then\\
    $\sum_{\Di=1}^{\ssY}\cmeiSv \Di \meiSv\FuVg[\ssY]{\rY|\rE}\big(\VnormLp{\hxdfPr-\dxdfPr}^2 \geq 12\DiepenSv\ssY^{-1}\big)
\leq\cst{}\big[(1\vee\meiSv[\Di_{\ydf}]^2)\Di_{\ydf}^2+(1\vee\meiSv[\ssY_{o}]^2)\big]$
  \item\label{au:re:rest:iii} 
  $\FuVg[\ssY]{\rY|\rE}\big(\VnormLp{\hxdfPr-\dxdfPr}^2 \geq 12\DiepenSv\ssY^{-1}\big)\leq 3 \big[\exp\big(\tfrac{-\cmeiSv[\Di]\Di}{200\Vnormlp[1]{\fydf}}\big)
    +(200)^2\ssY^{-1}\big] $.
  \end{resListeN}
\end{lm}
% % ....................................................................
% % <<Proof Re rest>>
% % ....................................................................
% \begin{pro}[Proof of \cref{au:re:rest}.]Consider \ref{au:re:rest:i}.
%   Since $\cmeiSv\geq1$ for
%   $\dr\Di\geq3({6\Vnormlp[1]{\fydf}})^2$ holds
%   $\tfrac{\sqrt{\cmeiSv}\Di}{6\Vnormlp[1]{\fydf}}-\log(\Di+2)\geq0$
%   and%
%   \begin{multline*}
%     \meiSv\exp\big(\tfrac{-\cmeiSv\Di}{3\Vnormlp[1]{\fydf}}\big)\leq
%     \exp\big(\tfrac{-\cmeiSv\Di}{6\Vnormlp[1]{\fydf}}\big)
%     \exp\big(-\sqrt{\cmeiSv}[\tfrac{\sqrt{\cmeiSv}\Di}{6\Vnormlp[1]{\fydf}}-\log(\Di+2)]\big)\\
%     \leq\exp\big(\tfrac{-\cmeiSv\Di}{6\Vnormlp[1]{\fydf}}\big)
%     \leq\exp\big(-\tfrac{1}{6\Vnormlp[1]{\fydf}}\Di\big)
%   \end{multline*}
%   consequently, for
%   $\dr\Di_{\ydf}:=\floor{3({6\Vnormlp[1]{\fydf}})^2}$ then exploiting
%   $\sum_{\Di\in\Nz}\exp(-\mu\Di)\leq \mu^{-1}$ follows
%   \begin{displaymath}
%     \sum_{\Di=1+\Di_{\ydf}}^{\ssY}\meiSv\exp\big(\tfrac{-\cmeiSv\Di}{3\Vnormlp[1]{\fydf}}\big)\leq
%     \sum_{\Di=1+\Di_{\ydf}}^{\ssY}\exp\big(-\tfrac{1}{6\Vnormlp[1]{\fydf}}\Di\big)
%     \leq {6\Vnormlp[1]{\fydf}}
%   \end{displaymath}
%   while
%   \begin{displaymath}
%    \sum_{\Di=1}^{\Di_{\ydf}}\meiSv\exp\big(\tfrac{-\cmeiSv\Di}{3\Vnormlp[1]{\fydf}}\big)\leq
%    \meiSv[\Di_{\ydf}]\sum_{\Di=1}^{\Di_{\ydf}}\exp\big(\tfrac{-\Di}{3\Vnormlp[1]{\fydf}}\big)
%    \leq \meiSv[\Di_{\ydf}]{3\Vnormlp[1]{\fydf}}
%  \end{displaymath}
%  hence
%  \begin{displaymath}
%    \sum_{\Di=1}^{\ssY}\meiSv\exp\big(\tfrac{-\cmeiSv\Di}{3\Vnormlp[1]{\fydf}}\big)\leq
%    {6\Vnormlp[1]{\fydf}}+3\meiSv[\Di_{\ydf}]\Vnormlp[1]{\fydf}\leq 3(2+\meiSv[\Di_{\ydf}]){\Vnormlp[1]{\fydf}}
%  \end{displaymath}
%  Using for all $\dr \ssY>\ssY_{o}:=15({200})^4$ holds 
%  $\sqrt{n}\geq{200}\log(n+2)$ it follows for all $\Di\in\nset{1,n}$
%  \begin{displaymath}
%    \tfrac{\Di\meiSv}{\ssY}\exp\big(\tfrac{-\sqrt{n\cmeiSv}}{200}\big)
%    \leq
%    \tfrac{1}{\ssY}\exp\big(-\sqrt{\cmeiSv}[\tfrac{\sqrt{\ssY}}{200}-\log(\Di+2)]\big)\leq \tfrac{1}{\ssY}
%  \end{displaymath}
%  consequently, 
%  \begin{equation*}
%    \sum_{\Di=1}^{\ssY}\tfrac{\Di\meiSv}{\ssY}\exp\big(\tfrac{-\sqrt{n\cmeiSv}}{200}\big)
%    \leq\sum_{\Di=1}^{\ssY}\tfrac{1}{\ssY}\leq1
%  \end{equation*}
%  while for $\ssY\leq \ssY_{o}$ with
%  $\meiSv[\ssY]\leq\meiSv[\ssY_{o}]$ follows
% \begin{equation*}
%    \sum_{\Di=1}^{\ssY}\tfrac{\Di\meiSv}{\ssY}\exp\big(\tfrac{-\sqrt{\ssY\cmeiSv}}{200}\big)\leq 
%     \meiSv[\ssY]\ssY\exp\big(\tfrac{-\sqrt{\ssY}}{200}\big)\leq\ssY_{o}\meiSv[\ssY_{o}]
%   \end{equation*}
%  consequently, for all $\ssY\in\Nz$ holds
%  \begin{displaymath}
%  \sum_{\Di=1}^{\ssY}\tfrac{\Di\meiSv}{\ssY}\exp\big(\tfrac{-\sqrt{\ssY\cmeiSv}}{200}\big)\leq (1\vee\meiSv[\ssY_{o}]\ssY_{o})
% \end{displaymath}
% Combining the last two bounds and \cref{re:cconc} \ref{re:cconc:i}  we
% obtain \ref{au:re:rest:i} (keep in mind that
% $\Vnormlp[1]{\fydf}^2\leq\Di_{\ydf}$ and $\ssY_{o}=15({200})^4$ is a numerical constant), that is 
% \begin{multline*}
%  \sum_{\Di=1}^{\ssY}\FuEx[\ssY]{\rY|\rE} \vectp{\VnormLp{\hxdfPr-\dxdfPr}^2 - 12\DiepenSv\ssY^{-1}} \\\hfill \leq 
%     \cst{} \bigg[\tfrac{\Vnormlp[1]{\fydf}}{\ssY}\sum_{\Di=1}^{\ssY}\meiSv
%     \exp\big(\tfrac{-\cmeiSv\Di}{3\Vnormlp[1]{\fydf}}\big)
%     +\tfrac{1}{\ssY}\sum_{\Di=1}^{\ssY}\tfrac{\Di\meiSv}{n}\exp\big(\tfrac{-\sqrt{\ssY\cmeiSv}}{200}\big) \bigg]\\
% \leq \cst{}\ssY^{-1}\big[(1\vee\meiSv[\Di_{\ydf}])\Di_{\ydf}+(1\vee\meiSv[\ssY_{o}])\big]
% \end{multline*}
% Consider  \ref{au:re:rest:ii}. If   $\dr\Di\geq 3({400\Vnormlp[1]{\fydf}})^2$ then 
% $\Di\geq  ({400\Vnormlp[1]{\fydf}})\log(\Di+2)$ and
% hence
% $\Di-{200\Vnormlp[1]{\fydf}}\log(\Di+2)\geq{200\Vnormlp[1]{\fydf}}\log(\Di+2)$
% or equivalently,
% $\tfrac{\Di}{200\Vnormlp[1]{\fydf}}-\log(\Di+2)\geq\log(\Di+2)\geq1$
% and thus similar to \eqref{p:au:re:SrWe:ag:e2} it follows
% \begin{multline*}
% \Di\cmeiSv\meiSv\exp\big(\tfrac{-\cmSv\Di}{200\Vnormlp[1]{\fydf}}\big)\leq
% \cmSv\exp\big(-\cmSv\,[\tfrac{\Di}{200\Vnormlp[1]{\fydf}}-\log(\Di+2)]\big)\\\leq
% (\Di+2)\exp\big(-\tfrac{\Di}{200\Vnormlp[1]{\fydf}}\big)% =
% % \\\cmSv\exp\big(-\tfrac{\cpen}{800\Vnormlp[1]{\fydf}}\cmSv\Di\big)\exp\big(-\sqrt{\cmSv}[\tfrac{\cpen\sqrt{\cmSv}}{800\Vnormlp[1]{\fydf}}\Di-\log(\Di+2)]\big)\leq\\
% \end{multline*}
% consequently, for $\dr\Di_{\ydf}:=\floor{3({400\Vnormlp[1]{\fydf}})^2}$ exploiting $\sum_{\Di\in\Nz}(\Di+2)\exp(-\mu\Di)\leq \mu^{-2}+ 2\mu^{-1}$
% follows
% \begin{multline*}
% \sum_{\Di=1+\Di_{\ydf}}^{\ssY}\Di\cmeiSv\meiSv\exp\big(\tfrac{-\cmeiSv\Di}{200\Vnormlp[1]{\fydf}}\big)\leq
% \sum_{\Di=1+\Di_{\ydf}}^{\ssY}(k+2)\exp\big(-\tfrac{\Di}{200\Vnormlp[1]{\fydf}}\big)
% \\\leq
% ({200\Vnormlp[1]{\fydf}})^2+{400\Vnormlp[1]{\fydf}}\leq \Di_{\ydf}^2
% \end{multline*}
% while  $\log(\Di\meiSv)\Ind{\{\meiSv\geq1\}}\leq
% \tfrac{1}{e}\Di\meiSv\Ind{\{\meiSv\geq1\}}$ implies with \eqref{p:au:re:SrWe:ag:hPhi}
% $\cmeiSv\meiSv=\cmeiSv\meiSv\Ind{\{\meiSv\geq1\}}\leq\Di\meiSv^2\Ind{\{\meiSv\geq1\}}=\Di\meiSv^2$ it follows
% \begin{multline*}
%   \sum_{\Di=1}^{\Di_{\ydf}}\Di\cmeiSv\meiSv\exp\big(\tfrac{-\cmeiSv\Di}{200\Vnormlp[1]{\fydf}}\big)\leq
%   \cmeiSv[\Di_{\ydf}]\meiSv[\Di_{\ydf}]\sum_{\Di=1}^{\Di_{\ydf}}\Di\exp\big(\tfrac{-\Di}{200\Vnormlp[1]{\fydf}}\big)\\\leq
%   \cmeiSv[\Di_{\ydf}]\meiSv[\Di_{\ydf}]({200\Vnormlp[1]{\fydf}})^2\leq\meiSv[\Di_{\ydf}]^2\Di_{\ydf}^2
% \end{multline*}
% consequently for all $\ssY\in\Nz$ we have
% \begin{displaymath}
%   \sum_{\Di=1}^{\ssY}\Di\cmeiSv\meiSv\exp\big(\tfrac{-\cmeiSv\Di}{200\Vnormlp[1]{\fydf}}\big)\leq(1+\meiSv[\Di_{\ydf}]^2)\Di_{\ydf}^2%\leq 2\meiSv[\Di_{\ydf}]^2\Di_{\ydf}^2
% \end{displaymath}
% Since  $\cmeiSv\meiSv\leq\Di\meiSv^2$,
% and  for all $\dr \ssY>\ssY_{o}:=\floor{15({600})^4}$ holds $\sqrt{\ssY}\geq{600}\log(\ssY+2)$
% \begin{multline*}
% \Di\cmeiSv\meiSv\exp\big(\tfrac{-\sqrt{\ssY\cmeiSv}}{200}\big)\leq
% \Di^2\meiSv^2\exp\big(\tfrac{-\sqrt{\ssY\cmeiSv}}{200}\big)\\\leq
% \tfrac{1}{\ssY}\exp\big(-\sqrt{\cmeiSv}[\tfrac{\sqrt{\ssY}}{200}-2\log(\Di+2)]+\log(\ssY+2)\big)
% \leq\tfrac{1}{\ssY}\exp\big(-3\sqrt{\cmeiSv}[\tfrac{\sqrt{\ssY}}{600}-\log(\ssY+2)]\big)
% \\
% \leq \tfrac{1}{\ssY}
%   \end{multline*}
% consequently, 
% \begin{equation*}
% \sum_{\Di=1}^{\ssY}\Di\cmeiSv\meiSv\exp\big(\tfrac{-\sqrt{\ssY\cmeiSv}}{200}\big)\leq\sum_{\Di=1}^{\ssY}\tfrac{1}{\ssY}\leq1
% \end{equation*}
% On the other hand side for $\ssY\leq\ssY_{o}$ with  $\ssY^b\exp(-a\ssY^{1/c})\leq (\tfrac{cb}{ea})^{cb}$ for all $c>0$ and $a,b\geq0$  follows
% \begin{multline*}
% \sum_{\Di=1}^{\ssY}\Di\cmeiSv\meiSv\exp\big(\tfrac{-\sqrt{\ssY\cmeiSv}}{200}\big)\leq\ssY^2\cmeiSv[\ssY]\meiSv[\ssY]\exp\big(\tfrac{-\sqrt{\ssY}}{200}\big)\leq
% \meiSv[\ssY]^2\ssY^3\exp\big(\tfrac{-\sqrt{\ssY}}{200}\big)\\\leq \meiSv[\ssY_{o}]^2\big({600}\big)^6\leq\meiSv[\ssY_{o}]^2\ssY_{o}^2
% \end{multline*}
%  consequently, for all $\ssY\in\Nz$ holds
%  \begin{displaymath}
%  \sum_{\Di=1}^{\ssY}\Di\cmeiSv\miSv\exp\big(\tfrac{-\sqrt{\ssY\cmeiSv}}{200}\big)\leq 1+\meiSv[\ssY_{o}]^2\ssY_{o}^2
% \end{displaymath}
% Combining the last two bounds and \cref{re:cconc} \ref{re:cconc:ii} we
% obtain \ref{au:re:rest:ii} (keep in mind that $\ssY_{o}=15({200})^4$ is a numerical constant), that is 
% \begin{multline*}
% \sum_{\Di=1}^{\ssY}\cmeiSv \Di \meiSv\FuVg[\ssY]{\rY|\rE}\big(\VnormLp{\hxdfPr-\dxdfPr}^2 \geq 12\DiepenSv\ssY^{-1}\big)\\\leq 
% 3 \bigg[\sum_{\Di=1}^{\ssY}\cmeiSv \Di \meiSv\exp\big(\tfrac{-\cmeiSv\Di}{200\Vnormlp[1]{\fydf}}\big)
% +\sum_{\Di=1}^{\ssY}\cmeiSv \Di \meiSv\exp\big(\tfrac{-\sqrt{\ssY\cmeiSv}}{200}\big)\bigg]\\
% \leq3\bigg[(1+\meiSv[\Di_{\ydf}]^2)\Di_{\ydf}^2+1+\meiSv[\ssY_{o}]^2\ssY_{o}^2\bigg]
% \end{multline*}
% Consider \ref{ak:re:rest:iii}. Since
% $\tfrac{\ssY\sqrt{\daRa{\Di}{\xdf,\eiSv}}}{200\sqrt{\Di\meiSv}}\geq\tfrac{\sqrt{\ssY\cmeiSv}}{200}\geq\tfrac{\sqrt{\ssY}}{200}$
% and $\ssY\exp(-\tfrac{\sqrt{\ssY}}{200})\leq(200)^2$ 
% from \cref{re:cconc} \ref{re:cconc:ii} follows \ref{au:re:rest:iii}, that is 
% \begin{multline*}\FuVg[\ssY]{\rY|\rE}\big(\VnormLp{\hxdfPr-\dxdfPr}^2 \geq 12\DiepenSv\ssY^{-1}\big)\leq 
%     3 \big[\exp\big(\tfrac{-\cmeiSv\Di}{200\Vnormlp[1]{\fydf}}\big)
%     +\exp\big(\tfrac{-\sqrt{\ssY\cmeiSv}}{200}\big)\big]
% \\\leq 3 \big[\exp\big(\tfrac{-\cmeiSv\Di}{200\Vnormlp[1]{\fydf}}\big)
%     +(200)^2\ssY^{-1}\big] 
% \end{multline*}
% which  completes the proof.\proEnd\end{pro}
% --------------------------------------------------------------------
% <<Proof Re ND rest>>
% --------------------------------------------------------------------
\begin{pro}[Proof of \cref{au:re:nd:rest}.]
  Since $\dr\cpen/7\geq 12$ and $\dr\peneSv/7\geq12\DipeneSv\ssY^{-1}$,
  $\Di\in\nset{1,n}$, by exploiting \cref{au:re:rest}
  \ref{au:re:rest:i}, \ref{au:re:rest:ii} and \ref{au:re:rest:iii} we
  obtain immediately the claim \ref{au:re:nd:rest1},
  \ref{au:re:nd:rest2} and \ref{au:re:nd:rest3}, respectively, which  completes the proof.
\proEnd\end{pro}
% --------------------------------------------------------------------
% <<Proof Re upper bound ag>>
% --------------------------------------------------------------------
\begin{pro}[Proof of \cref{au:ag:ub}.]
  Consider firstly the aggregation using the aggregation weights
  $\erWe[]$ as in \eqref{au:de:erWe}.  Combining
  \cref{ak:re:nd:rest} and the upper bound given in \eqref{co:agg:au:ag}
  we obtain
  \begin{multline}\label{au:ag:ub:p1}
  \FuEx[\ssY]{\rY|\rE}\VnormLp{\hxdf[{\erWe[]}]-\xdf}^2\leq 
    3\FuEx[\ssY]{\rY|\rE}\VnormLp{\hxdfPr[\pDi]-\dxdfPr[\pDi]}^2
    +3 \VnormLp{\ProjC[0]\xdf}^2\bias[\mDi]^2(\xdf)\\\hfill
    +\tfrac{150}{\rWc\cpen}\VnormLp{\ProjC[0]\xdf}^2\Ind{\{\mDi>1\}}
    \exp\big(-\tfrac{3\rWc\cpen}{14}n\daRa{\mdDi}{\xdf,\iSv}\big)
    \\\hfill
    +\cst{} \VnormLp{\ProjC[0]\xdf}^2\Ind{\{\mDi>1\}} \big[ \exp\big(\tfrac{-\cmeiSv[\mdDi]\mdDi}{200\Vnormlp[1]{\fydf}}\big) \Ind{\aixEv[\mdDi]} + \Ind{\aixEv[\mdDi]^c}\big]
    \\\hfill
    +6\sum_{j\in\nset{1,\ssY}}|\hfedfmpI[j]|^2|\fedf[j]-\hfedf[j]|^2|\fxdf[j]|^2
    +2\sum_{j\in\nset{1,\ssY}}\Ind{\xEv^c}|\fxdf[j]|^2\\
    +\cst{}\ssY^{-1}\{(1\vee\meiSv[\Di_{\ydf}]^2)\Di_{\ydf}^2+(1\vee\meiSv[\ssY_{o}]^2)+\VnormLp{\ProjC[0]\xdf}^2\Ind{\{\mDi>1\}} +\tfrac{16}{\cpen\rWc^{2}}+
    \tfrac{8}{\rWc}\}
  \end{multline}
 Consider
  $\FuEx[\ssY]{\rY|\rE}\VnormLp{\hxdfPr[\pDi]-\dxdfPr[\pDi]}^2=2\sum_{j=1}^{\pDi}(\hfedfmpI[j])^2/\ssY=2\sum_{j=1}^{\pDi}\eiSv[j]/\ssY=2\pDi\oeiSv[\pDi]/\ssY\leq2\DipeneSv\ssY^{-1}$,
  where  by construction $\dr\peneSv/7\geq12\DipeneSv\ssY^{-1}$ and hence
  we have
  $\FuEx[\ssY]{\rY|\rE}\VnormLp{\hxdfPr[\pDi]-\dxdfPr[\pDi]}^2\leq\tfrac{1}{42}\peneSv[\pDi]$. Moreover, exploiting
  $\dr\max_{j\in\nset{1,n}}\eiSv[j]\leq\ssE$ and $\pDi\leq\ssY$ it holds also 
  $\FuEx[\ssY]{\rY|\rE}\VnormLp{\hxdfPr[\pDi]-\dxdfPr[\pDi]}^2\leq2\ssE$.
Considering the event $\aixEv[\pdDi]$ and its complement
$\aixEv[\pdDi]^c$  it follows
$\FuEx[\ssY]{\rY|\rE}\VnormLp{\hxdfPr[\pDi]-\dxdfPr[\pDi]}^2\leq
2\ssE\Ind{\aixEv[\pdDi]^c}+\tfrac{1}{42}\peneSv[\pDi]\Ind{\aixEv[\pdDi]}$.
Taking into account the definition
  \eqref{au:de:*Di:ag} of
  $\pDi$ we obtain
  $\FuEx[\ssY]{\rY|\rE}\VnormLp{\hxdfPr[\pDi]-\dxdfPr[\pDi]}^2\leq
  2\ssE\Ind{\aixEv[\pdDi]^c}+\tfrac{1}{42}[6\VnormLp{\ProjC[\pdDi]\dxdfPr[\ssY]}^2+4\peneSv[\pdDi]]\Ind{\aixEv[\pdDi]}$ 
Thereby,
with $\rWc\geq1$ and $\cpen\geq1$
from \eqref{au:ag:ub:p1} follows (keep in mind that $\ssY_{o}$ is a numerical constant)
  \begin{multline}\label{au:ag:ub:p2}
  \FuEx[\ssY]{\rY|\rE}\VnormLp{\hxdf[{\erWe[]}]-\xdf}^2\leq  \tfrac{2}{7}\peneSv[\pdDi]\Ind{\aixEv[\pdDi]}+\tfrac{3}{7}\VnormLp{\ProjC[\pdDi]\dxdfPr[\ssY]}^2\Ind{\aixEv[\pdDi]}
    +3 \VnormLp{\ProjC[0]\xdf}^2\bias[\mDi]^2(\xdf)\\\hfill
    +\cst{}\VnormLp{\ProjC[0]\xdf}^2\Ind{\{\mDi>1\}}\big[
    \exp\big(-\tfrac{3\rWc\cpen}{14}n\daRa{\mdDi}{\xdf,\iSv}\big)
    +
    \exp\big(\tfrac{-\cmeiSv[\mdDi]\mdDi}{200\Vnormlp[1]{\fydf}}\big)
    \Ind{\aixEv[\mdDi]}\big]\\ \hfill+ \cst{}\big[
    \VnormLp{\ProjC[0]\xdf}^2\Ind{\{\mDi>1\}} \Ind{\aixEv[\mdDi]^c}+\ssE\Ind{\aixEv[\pdDi]^c} + \ssY^{-1}\{\Di_{\ydf}^2\ssE^2\Ind{\aixEv[\Di_{\ydf}]^c}+\ssE^2\Ind{\aixEv[\ssY_{o}]^c}\}\big]
    \\\hfill
    +6\sum_{j\in\nset{1,\ssY}}|\hfedfmpI[j]|^2|\fedf[j]-\hfedf[j]|^2|\fxdf[j]|^2
    +2\sum_{j\in\nset{1,\ssY}}\Ind{\xEv^c}|\fxdf[j]|^2\\
    +\cst{}\ssY^{-1}\{(1\vee\meiSv[\Di_{\ydf}]^2)\Di_{\ydf}^2\Ind{\aixEv[\Di_{\ydf}]}+(1\vee\meiSv[\ssY_{o}]^2)\Ind{\aixEv[\ssY_{o}]}+\VnormLp{\ProjC[0]\xdf}^2\Ind{\{\mDi>1\}}\}
  \end{multline}
Employing \eqref{pro:au:erWe:e3}  and \eqref{pro:au:erWe:peneSv}
follows $\meiSv\Ind{\aixEv[\Di]}\leq \tfrac{9}{4}\miSv$,
$\cmeiSv\Ind{\aixEv[\Di]}\geq \tfrac{9}{100}\cmiSv$ and
$\peneSv\Ind{\aixEv[\Di]}\leq 7\penSv$ for all $\Di\in\Nz$. Thereby,
\eqref{au:ag:ub:p2}  implies
  \begin{multline}\label{au:ag:ub:p3}
  \FuEx[\ssY]{\rY|\rE}\VnormLp{\hxdf[{\erWe[]}]-\xdf}^2\leq  2\penSv[\pdDi] +\tfrac{3}{7}\VnormLp{\ProjC[\pdDi]\dxdfPr[\ssY]}^2+3 \VnormLp{\ProjC[0]\xdf}^2\bias[\mDi]^2(\xdf)\\\hfill
    +\cst{}\VnormLp{\ProjC[0]\xdf}^2\Ind{\{\mDi>1\}}\big[
    \exp\big(-\tfrac{3\rWc\cpen}{14}n\daRa{\mdDi}{\xdf,\iSv}\big)
    +
    \exp\big(\tfrac{-9\cmiSv[\mdDi]\mdDi}{20000\Vnormlp[1]{\fydf}}\big)
    \big]\\ \hfill+ \cst{}\big[
    \VnormLp{\ProjC[0]\xdf}^2\Ind{\{\mDi>1\}} \Ind{\aixEv[\mdDi]^c}+\ssE\Ind{\aixEv[\pdDi]^c} + \ssY^{-1}\{\Di_{\ydf}^2\ssE^2\Ind{\aixEv[\Di_{\ydf}]^c}+\ssE^2\Ind{\aixEv[\ssY_{o}]^c}\}\big]
    \\\hfill
    +6\sum_{j\in\nset{1,\ssY}}|\hfedfmpI[j]|^2|\fedf[j]-\hfedf[j]|^2|\fxdf[j]|^2
    +2\sum_{j\in\nset{1,\ssY}}\Ind{\xEv^c}|\fxdf[j]|^2\\
    +\cst{}\ssY^{-1}\{\miSv[\Di_{\ydf}]^2\Di_{\ydf}^2+\miSv[\ssY_{o}]^2+\VnormLp{\ProjC[0]\xdf}^2\Ind{\{\mDi>1\}}\}
  \end{multline}
Exploiting \cref{oSv:re} we obtain from 
\ref{oSv:re:i}
\begin{displaymath}
\FuEx[\ssE]{\rE}\VnormLp{\ProjC[\pdDi]\dxdfPr[\ssY]}^2\leq4\sum_{|j|\in\nsetlo{\pDi,\ssY}}|\fxdf[j]|^2\leq
4\VnormLp{\ProjC[0]\xdf}^2\bias[\pdDi]^2(\xdf)
\end{displaymath}
from \ref{oSv:re:iii} and $\mRa{\xdf,\iSv}:=\sum_{j\in\Nz}\fxdf[j]^2[1\wedge
\iSv[j]/\ssE]$ as defined in \eqref{oo:de:mra} 
\begin{displaymath}
\sum_{j\in\nset{1,\ssY}}|\fxdf[j]|^2\FuEx[\ssE]{\rE}|\hfedfmpI[j]|^2|\fedf[j]-\hfedf[j]|^2\leq4\cst{4}\mRa{\xdf,\iSv}
\end{displaymath}
and from
\ref{oSv:re:ii}
\begin{displaymath}
\sum_{j\in\nset{1,n}}\FuVg[\ssE]{\rE}(\xEv^c)|\fxdf[j]|^2
\leq4\mRa{\xdf,\iSv}
\end{displaymath}
The last bounds imply together with 
  \begin{multline}\label{au:ag:ub:p4}
  \FuEx[\ssY,\ssE]{\rY,\rE}\VnormLp{\hxdf[{\erWe[]}]-\xdf}^2\leq
  2\penSv[\pdDi] +\tfrac{12}{7}\VnormLp{\ProjC[0]\xdf}^2\bias[\pdDi]^2(\xdf)+3 \VnormLp{\ProjC[0]\xdf}^2\bias[\mDi]^2(\xdf)\\\hfill
    +\cst{}\VnormLp{\ProjC[0]\xdf}^2\Ind{\{\mDi>1\}}\big[
    \exp\big(-\tfrac{3\rWc\cpen}{14}n\daRa{\mdDi}{\xdf,\iSv}\big)
    +
    \exp\big(\tfrac{-9\cmiSv[\mdDi]\mdDi}{20000\Vnormlp[1]{\fydf}}\big)
    \big]\\ \hfill+ \cst{}\big[
    \VnormLp{\ProjC[0]\xdf}^2\Ind{\{\mDi>1\}} \FuVg[\ssE]{\rE}(\aixEv[\mdDi]^c)+\ssE\FuVg[\ssE]{\rE}(\aixEv[\pdDi]^c) + \ssY^{-1}\{\Di_{\ydf}^2\ssE^2\FuVg[\ssE]{\rE}(\aixEv[\Di_{\ydf}]^c)+\ssE^2\FuVg[\ssE]{\rE}(\aixEv[\ssY_{o}]^c)\}\big]
    \\\hfill
    +24\cst{4}\mRa{\xdf,\iSv}
    +8\mRa{\xdf,\iSv}\\
    +\cst{}\ssY^{-1}\{\miSv[\Di_{\ydf}]^2\Di_{\ydf}^2+\miSv[\ssY_{o}]^2+\VnormLp{\ProjC[0]\xdf}^2\Ind{\{\mDi>1\}}\}
  \end{multline}
Moreover, since $\ssY\daRa{\mdDi}{\xdf,\iSv}\geq\cmiSv[\mdDi]\mdDi$. From
\eqref{au:ag:ub:p4} with $\tfrac{3\rWc\cpen}{14}>\tfrac{9}{20000\Vnormlp[1]{\fydf}}$
(since $\cpen,\rWc\geq1$ and $\Vnormlp[1]{\fydf}\geq|\fydf[0]|=1$)
follows
  \begin{multline}\label{au:ag:ub:p4}
  \FuEx[\ssY,\ssE]{\rY,\rE}\VnormLp{\hxdf[{\erWe[]}]-\xdf}^2\leq
  2\penSv[\pdDi] +\tfrac{12}{7}\VnormLp{\ProjC[0]\xdf}^2\bias[\pdDi]^2(\xdf)+3 \VnormLp{\ProjC[0]\xdf}^2\bias[\mDi]^2(\xdf)\\\hfill
    +\cst{}\VnormLp{\ProjC[0]\xdf}^2\Ind{\{\mDi>1\}}
    \exp\big(\tfrac{-9\cmiSv[\mdDi]\mdDi}{20000\Vnormlp[1]{\fydf}}\big)
    \\ \hfill+ \cst{}\big[
    \VnormLp{\ProjC[0]\xdf}^2\Ind{\{\mDi>1\}} \FuVg[\ssE]{\rE}(\aixEv[\mdDi]^c)+\ssE\FuVg[\ssE]{\rE}(\aixEv[\pdDi]^c) + \ssY^{-1}\{\Di_{\ydf}^2\ssE^2\FuVg[\ssE]{\rE}(\aixEv[\Di_{\ydf}]^c)+\ssE^2\FuVg[\ssE]{\rE}(\aixEv[\ssY_{o}]^c)\}\big]
    \\\hfill
    +\cst{}\mRa{\xdf,\iSv}
    +\cst{}\ssY^{-1}\{\miSv[\Di_{\ydf}]^2\Di_{\ydf}^2+\miSv[\ssY_{o}]^2+\VnormLp{\ProjC[0]\xdf}^2\Ind{\{\mDi>1\}}\}
  \end{multline}
Exploiting \cref{re:evrest} \ref{re:evrest:ii} there is a
numerical constant $\cst{}$ such that for all  $\ssE,\Di\in\Nz$ holds
$\FuVg[\ssE]{\rE}(\aixEv^c)\leq\cst{}\Di\miSv^2\ssE^{-2}$
and hence, $\ssE^2\FuVg[\ssE]{\rE}(\aixEv[\Di_{\ydf}]^c)\leq
\cst{}\Di_{\ydf}\miSv[\Di_{\ydf}]^2$ and $\ssE^2\FuVg[\ssE]{\rE}(\aixEv[\ssY_{o}]^c)\leq
\cst{}\ssY_{o}\miSv[\ssY_{o}]^2$, thereby from \eqref{au:ag:ub:p4}
follows the assertion \eqref{au:ag:ub:e1}, that is, (keep in mind that $\ssY_{o}$ is a numerical constant)
  \begin{multline}\label{au:ag:ub:p5}
  \FuEx[\ssY,\ssE]{\rY,\rE}\VnormLp{\hxdf[{\erWe[]}]-\xdf}^2\leq
  2\penSv[\pdDi] +\tfrac{12}{7}\VnormLp{\ProjC[0]\xdf}^2\bias[\pdDi]^2(\xdf)+3 \VnormLp{\ProjC[0]\xdf}^2\bias[\mDi]^2(\xdf)\\\hfill
    +\cst{}\VnormLp{\ProjC[0]\xdf}^2\Ind{\{\mDi>1\}}
    \exp\big(\tfrac{-9\cmiSv[\mdDi]\mdDi}{20000\Vnormlp[1]{\fydf}}\big)
    + \cst{}\big[
    \VnormLp{\ProjC[0]\xdf}^2\Ind{\{\mDi>1\}} \FuVg[\ssE]{\rE}(\aixEv[\mdDi]^c)+\ssE\FuVg[\ssE]{\rE}(\aixEv[\pdDi]^c) \big]
    \\\hfill
    +\cst{}\mRa{\xdf,\iSv}
    +\cst{}\ssY^{-1}\{\miSv[\Di_{\ydf}]^2\Di_{\ydf}^3+\miSv[\ssY_{o}]^2+\VnormLp{\ProjC[0]\xdf}^2\Ind{\{\mDi>1\}}\}
  \end{multline}
  Consider secondly the aggregation using the model selection weights $\erWe[]:=\msWe[]$
  as in \eqref{au:de:msWe}. Combining
  \cref{au:re:nd:rest} and the upper bound given in \eqref{co:agg:ms}
  we obtain
  \begin{multline}\label{au:ag:ub:p3}
    \FuEx[\ssY]{\rY}\VnormLp{\txdfAg[{\msWe[]}]-\xdf}^2\leq \tfrac{2}{7}\penSv[\pDi]
    +2\VnormLp{\ProjC[0]\xdf}^2\bias[\mDi]^2(\xdf)
    \\\hfill
 + \cst{}\VnormLp{\ProjC[0]\xdf}^2\Ind{\{\mDi>1\}}
 \exp\big(\tfrac{-1}{200\Vnormlp[1]{\fydf}}\ssY\daRaS{\mdDi}{\xdf,\iSv}\miSv[\mdDi]^{-1}\big)
 \\
 +\cst{}\big[
 \VnormLp{\ProjC[0]\xdf}^2\Ind{\{\mDi>1\}}
+\miSv[\Di_{\ydf}]^2\Di_{\ydf}^2+\miSv[\ssY_{o}]^2 \big]\ssY^{-1}.
  \end{multline}  
From \eqref{ak:ag:ub:p2} and \eqref{ak:ag:ub:p3} together with
$\ssY\daRaS{\mdDi}{\xdf,\iSv}\miSv[\mdDi]^{-1}\geq\cmiSv[\mdDi]\mdDi$
follows the claim \eqref{ak:ag:ub:e1}, which  completes the proof.
\proEnd\end{pro}
% ....................................................................
% <<Pro upper bound ag p>>
% ....................................................................
\begin{pro}[Proof of \cref{au:ag:ub:pnp}.]
From
\eqref{au:ag:ub:e1} follows for any $\mdDi,\pdDi\in\nset{1,n}$ and associated
$\mDi,\pDi\in\nset{1,n}$ as defined in  \eqref{au:de:*Di:ag}%
 \begin{multline}\label{au:ag:ub:pnp:p1}
   \FuEx[\ssY,\ssE]{\rY,\rE}\VnormLp{\hxdf[{\erWe[]}]-\xdf}^2\leq
  2\penSv[\pdDi] +\tfrac{12}{7}\VnormLp{\ProjC[0]\xdf}^2\bias[\pdDi]^2(\xdf)+3 \VnormLp{\ProjC[0]\xdf}^2\bias[\mDi]^2(\xdf)\\\hfill
    +\cst{}\VnormLp{\ProjC[0]\xdf}^2\Ind{\{\mDi>1\}}
    \exp\big(\tfrac{-9\cmiSv[\mdDi]\mdDi}{20000\Vnormlp[1]{\fydf}}\big)
    + \cst{}\big[
    \VnormLp{\ProjC[0]\xdf}^2\Ind{\{\mDi>1\}} \FuVg[\ssE]{\rE}(\aixEv[\mdDi]^c)+\ssE\FuVg[\ssE]{\rE}(\aixEv[\pdDi]^c) \big]
    \\\hfill
    +\cst{}\mRa{\xdf,\iSv}
    +\cst{}\ssY^{-1}\{\miSv[\Di_{\ydf}]^2\Di_{\ydf}^3+\miSv[\ssY_{o}]^2+\VnormLp{\ProjC[0]\xdf}^2\Ind{\{\mDi>1\}}\}
\end{multline}
We destinguish the two cases \ref{au:ag:ub:pnp:p} and
\ref{au:ag:ub:pnp:np}. Consider first \ref{au:ag:ub:pnp:p}, and hence there is $K\in\Nz_0$   with   $1\geq \bias[{[K-1] }](\xdf)>0$ and
$\bias(\xdf)=0$ for all $\Di\geq K$. Consider first $K=0$, then $\bias[0](\xdf)=0$
and hence $\VnormLp{\ProjC[0]\xdf}^2=0$ and $\mRa{\xdf,\iSv}=0$ (see \cref{oo:rem:nm}). From \eqref{au:ag:ub:pnp:p1}
follows 
 \begin{equation}\label{au:ag:ub:pnp:p2}
    \FuEx[\ssY,\ssE]{\rY,\rE}\VnormLp{\hxdf[{\erWe[]}]-\xdf}^2\leq
  2\penSv[\pdDi] 
    + \cst{}\ssE\FuVg[\ssE]{\rE}(\aixEv[\pdDi]^c) 
    +\cst{}\ssY^{-1}\{\miSv[\Di_{\ydf}]^2\Di_{\ydf}^3+\miSv[\ssY_{o}]^2\}
\end{equation}
Setting  $\pdDi:=1$ it follows
$\penSv[\pdDi]=\cpen\DipenSv[1]\ssY^{-1}=\cpen\cmSv[1]
\miSv[1]\ssY^{-1}\leq\cpen\miSv[1]^2\ssY^{-1}$ and exploiting \cref{re:evrest} \ref{re:evrest:ii} there is a
numerical constant $\cst{}$ such that for all  $\ssE\in\Nz$ holds
$\FuVg[\ssE]{\rE}(\aixEv[\pdDi]^c)\leq\cst{}\miSv[1]^2\ssE^{-2}$. Thereby with numerical
constant $\cpen\geq84$, \eqref{au:ag:ub:pnp:p2} implies for all $\ssY,\ssE\in\Nz$
 \begin{equation}\label{au:ag:ub:pnp:p3}
    \FuEx[\ssY,\ssE]{\rY,\rE}\VnormLp{\hxdf[{\erWe[]}]-\xdf}^2\leq
 \cst{}\miSv[1]^2\ssE^{-1} +\cst{}\ssY^{-1}\{\miSv[1]^2+\miSv[\Di_{\ydf}]^2\Di_{\ydf}^3+\miSv[\ssY_{o}]^2\}
\end{equation}
Consider now $K\in\Nz$, and hence $\VnormLp{\ProjC[0]\xdf}^2>0$. Let 
$\dr c_{\xdf}:=\tfrac{\VnormLp{\ProjC[0]\xdf}^2+104\cpen}{\VnormLp{\ProjC[0]\xdf}^2\bias[{[K-1]}]^2(\xdf)}>1$
and $\ssY_{\xdf}:=\floor{c_{\xdf}\DipenSv[K]}\in\Nz$. We distinguish for $n\in\Nz$ the following two
 cases, \begin{inparaenum}[i]\renewcommand{\theenumi}{\dgrau\rm(\alph{enumi})}\item\label{au:ag:ub:pnp:p:n1}
$\ssY\in\nset{1,\ssY_{\xdf}}$ and \item\label{au:ag:ub:pnp:p:n2}
$\ssY> \ssY_{\xdf}$. \end{inparaenum} Firstly, consider
\ref{au:ag:ub:pnp:p:n1} with $\ssY\in\nset{1,\ssY_{\xdf}}$, then setting $\mdDi:=1$, $\pdDi:=1$ we have
$\mDi=1$, $1\geq\bias[\mDi]$ and  $1\leq\DipenSv[1]=\cmSv[1]
\miSv[1]\leq\miSv[1]^2$. Thereby,  from \eqref{au:ag:ub:pnp:p1} 
follows
 \begin{multline*}
    \FuEx[\ssY,\ssE]{\rY,\rE}\VnormLp{\hxdf[{\erWe[]}]-\xdf}^2\leq
  2\cpen\miSv[1]^2\ssY^{-1} +\tfrac{33}{7}\VnormLp{\ProjC[0]\xdf}^2 + \cst{}\big[\ssE\FuVg[\ssE]{\rE}(\aixEv[1]^c) \big]
    \\\hfill
    +\cst{}\mRa{\xdf,\iSv}
    +\cst{}\ssY^{-1}\{\miSv[\Di_{\ydf}]^2\Di_{\ydf}^3+\miSv[\ssY_{o}]^2\}
  \end{multline*}
  Exploiting \cref{re:evrest} \ref{re:evrest:ii} there is a
numerical constant $\cst{}$ such that for all  $\ssE\in\Nz$ holds
$\FuVg[\ssE]{\rE}(\aixEv[1]^c)\leq\cst{}\miSv[1]^2\ssE^{-2}$, which
together with
$\mRa{\xdf,\iSv}\leq
\VnormLp{\ProjC[0]\xdf}^2\miSv[K]\ssE^{-1}$ implies
 \begin{multline*}
    \FuEx[\ssY,\ssE]{\rY,\rE}\VnormLp{\hxdf[{\erWe[]}]-\xdf}^2\leq
  2\cpen\miSv[1]^2\ssY^{-1} +\tfrac{33}{7}\VnormLp{\ProjC[0]\xdf}^2 +
  \cst{}\big[\miSv[1]^2 +\VnormLp{\ProjC[0]\xdf}^2\miSv[K] \big]\ssE^{-1}
    \\\hfill
    +\cst{}\ssY^{-1}\{\miSv[\Di_{\ydf}]^2\Di_{\ydf}^3+\miSv[\ssY_{o}]^2\}
  \end{multline*}
Moreover, for all $\ssY\in\nset{1,\ssY_{\xdf}}$ with
$\ssY_{\xdf}=\floor{c_{\xdf}\DipenSv[K]}$ and
$\DipenSv[K]=K\cmSv[K] \miSv[K]\leq K^2\miSv[K]^2$ holds
$\ssY\leq\cst{}\tfrac{(\VnormLp{\ProjC[0]\xdf}^2\vee1)}{\VnormLp{\ProjC[0]\xdf}^2\bias[{[K-1]}]^2(\xdf)}
K^2\miSv[K]^2$ and thereby, for all $\ssY\in\nset{1,\ssY_{\xdf}}$ and
for all $\ssE\in\Nz$
\begin{multline}\label{au:ag:ub:pnp:p4}
  \FuEx[\ssY,\ssE]{\rY,\rE}\VnormLp{\hxdf[{\erWe[]}]-\xdf}^2\leq
  \cst{}\big[(\VnormLp{\ProjC[0]\xdf}^2\vee1)\miSv[1]^2\tfrac{K^2\miSv[K]^2}{\VnormLp{\ProjC[0]\xdf}^2\bias[{[K-1]}]^2(\xdf)}+\miSv[\Di_{\ydf}]^2\Di_{\ydf}^3+\miSv[\ssY_{o}]^2\big]\ssY^{-1}\\
  +
  \cst{}\big[\miSv[1]^2 +\VnormLp{\ProjC[0]\xdf}^2\miSv[K] \big]\ssE^{-1}.
\end{multline}
Secondly, consider \ref{au:ag:ub:pnp:p:n2}, i.e., $\ssY>
\ssY_{\xdf}$. Setting
$\pdDi:=K< \floor{c_{\xdf}\DipenSv[K]}=\ssY_{\xdf}$, i.e.,
$\pdDi\in\nset{1,\ssY}$, it follows $\bias[\pdDi](\xdf)=0$ and
$\pen[\pdDi]=\cpen\DipenSv[K]\ssY^{-1}\leq
\cpen K^2\miSv[K]^2\ssY^{-1}$. From
\eqref{au:ag:ub:pnp:p1} follows for all $\ssY> \ssY_{\xdf}$ thus
\begin{multline*}
   \FuEx[\ssY,\ssE]{\rY,\rE}\VnormLp{\hxdf[{\erWe[]}]-\xdf}^2\leq
  3 \VnormLp{\ProjC[0]\xdf}^2\bias[\mDi]^2(\xdf)\\\hfill
    +\cst{}\VnormLp{\ProjC[0]\xdf}^2\Ind{\{\mDi>1\}}
    \exp\big(\tfrac{-9\cmiSv[\mdDi]\mdDi}{20000\Vnormlp[1]{\fydf}}\big)
    + \cst{}\big[
    \VnormLp{\ProjC[0]\xdf}^2\Ind{\{\mDi>1\}} \FuVg[\ssE]{\rE}(\aixEv[\mdDi]^c)+\ssE\FuVg[\ssE]{\rE}(\aixEv[K]^c) \big]
    \\\hfill
    +\cst{}\mRa{\xdf,\iSv}
    +\cst{}\ssY^{-1}\{K^2\miSv[K]^2\ssY^{-1}+\miSv[\Di_{\ydf}]^2\Di_{\ydf}^3+\miSv[\ssY_{o}]^2+\VnormLp{\ProjC[0]\xdf}^2\Ind{\{\mDi>1\}}\}.
  \end{multline*}
Exploiting \cref{re:evrest} \ref{re:evrest:ii} there is a
numerical constant $\cst{}$ such that for all  $\ssE\in\Nz$ holds
$\FuVg[\ssE]{\rE}(\aixEv[K]^c)\leq\cst{}K\miSv[K]^2\ssE^{-2}$, which
together with
$\mRa{\xdf,\iSv}\leq
\VnormLp{\ProjC[0]\xdf}^2\miSv[K]\ssE^{-1}$ implies
\begin{multline}\label{au:ag:ub:pnp:p5}
  \FuEx[\ssY,\ssE]{\rY,\rE}\VnormLp{\hxdf[{\erWe[]}]-\xdf}^2\leq \cst{}\ssY^{-1}\{K^2\miSv[K]^2\ssY^{-1}+\miSv[\Di_{\ydf}]^2\Di_{\ydf}^3+\miSv[\ssY_{o}]^2+\VnormLp{\ProjC[0]\xdf}^2\Ind{\{\mDi>1\}}\}\\\hfill
+ 3 \VnormLp{\ProjC[0]\xdf}^2\bias[\mDi]^2(\xdf)
    +\cst{}\VnormLp{\ProjC[0]\xdf}^2\Ind{\{\mDi>1\}}\{
    \exp\big(\tfrac{-9\cmiSv[\mdDi]\mdDi}{20000\Vnormlp[1]{\fydf}}\big)
    +  \FuVg[\ssE]{\rE}(\aixEv[\mdDi]^c)\}
    \\\hfill
    +\cst{}\ssE^{-1}\{K\miSv[K]^2+ \VnormLp{\ProjC[0]\xdf}^2\miSv[K]\}
  \end{multline}
In order to control the terms
involving $\mdDi$ and $\mDi$ we destinguish for $\ssE\in\Nz$ with
 $\ssE(\xdf,\iSv):=\floor{289\log(K+2)\cmiSv[K]\miSv[K]}$
the following two cases
cases, \begin{inparaenum}[i]\renewcommand{\theenumi}{\dgrau\rm(b-\roman{enumi})}\item\label{au:ag:ub:pnp:p:m1}
$\ssE\in\nset{1,\ssE(\xdf,\iSv)}$ and \item\label{au:ag:ub:pnp:p:m2}
$\ssE>\ssE(\xdf,\iSv)$. \end{inparaenum}
Consider first \ref{au:ag:ub:pnp:p:m1} $\ssE\in\nset{1,\ssE(\xdf,\iSv)}$.
We set $\mdDi=1$ and hence
$\mDi=1$. Thereby, with $\bias[1]^2(\xdf)\leq1$, $\log(K+2)\leq
\tfrac{K+2}{e}\leq 2K$, $\cmiSv[\Di]\miSv[\Di]\leq K\miSv[K]^2$, and hence $\ssE(\xdf,\iSv)\leq\cst{}K^2\miSv[K]^2$,
from \eqref{au:ag:ub:pnp:p5} follows for all $\ssE\in\nsetro{1,\ssE(\xdf,\iSv)}$
\begin{multline}\label{au:ag:ub:pnp:p6}
  \FuEx[\ssY,\ssE]{\rY,\rE}\VnormLp{\hxdf[{\erWe[]}]-\xdf}^2\leq \cst{}\ssY^{-1}\{K^2\miSv[K]^2\ssY^{-1}+\miSv[\Di_{\ydf}]^2\Di_{\ydf}^3+\miSv[\ssY_{o}]^2\}\\\hfill
    +\cst{}\ssE^{-1}\{K\miSv[K]^2+ \VnormLp{\ProjC[0]\xdf}^2(K^2\miSv[K]^2+\miSv[K])\}
  \end{multline}
Consider now \ref{au:ag:ub:pnp:p:m2} $\ssE>\ssE(\xdf,\iSv)$. Note that for all $\ssE> \ssE(K,\iSv)$ the defining set of
$\sDi{\ssE}:=\max\{\Di\in\nset{K,\ssE}:289\log(\Di+2)\cmiSv[\Di]\miSv[\Di]\leq\ssE\}$
is not empty, where obviously for each
$\mdDi\in\nset{K,\sDi{\ssE}}$ holds 
$\ssE\geq289\log(\mdDi+2)\cmiSv[\mdDi]\miSv[\mdDi]$, and thus from
\cref{re:evrest} \ref{re:evrest:iii} follows
$\FuVg[\ssE]{\rE}(\aixEv[\mdDi]^c)\leq 53\ssE^{-1}$.
Since also
$\ssY> \ssY_{\xdf}:=\floor{c_{\xdf}\DipenSv[K]}$ with
$\dr c_{\xdf}:=\tfrac{\VnormLp{\ProjC[0]\xdf}^2+104\cpen}{\VnormLp{\ProjC[0]\xdf}^2\bias[{[K-1]}]^2(\xdf)}>1$
the defining set of
$\sDi{\ssY}:=\max\{\Di\in\nset{K,\ssY}:\ssY>c_{\xdf}\DipenSv\}$ is not
empty. Consequently,  for all $\mdDi\in\nset{K,\sDi{\ssY}}$ holds $\mdDi>\geq
K$ and, hence 
$\bias[\mdDi](\xdf)=0$, and
$\daRaS{\mdDi}{\xdf,\iSv}=\DipenSv[\mdDi]\ssY^{-1}<c_{\xdf}^{-1}=\tfrac{\VnormLp{\ProjC[0]\xdf}^2\bias[{[K-1]}]^2(\xdf)}{\VnormLp{\ProjC[0]\xdf}^2+104\cpen}$,
it follows
$\VnormLp{\ProjC[0]\xdf}^2\bias[{[K-1]}]^2(\xdf)>[\VnormLp{\ProjC[0]\xdf}^2+104\cpen]\daRaS{\mdDi}{\xdf,\iSv}$
and trivially
$\VnormLp{\ProjC[0]\xdf}^2\bias[{K}]^2(\xdf)=0<[\VnormLp{\ProjC[0]\xdf}^2+104\cpen]\daRaS{\mdDi}{\xdf,\iSv}$. Therefore, the definition \eqref{au:de:*Di:ag}
implies $\mDi=K$ and hence
$\bias[\mDi]^2(\xdf)=\bias[K]^2(\xdf)=0$. Selecting
  $\mdDi:=\sDi{\ssY}\wedge\sDi{\ssE}$ we have 
$\FuVg[\ssE]{\rE}(\aixEv[\mdDi]^c)\leq 53\ssE^{-1}$, 
$\mDi=K$ and $\bias[\mDi]^2(\xdf)=0$, such that 
 from  \eqref{au:ag:ub:pnp:p5} follows for all $\ssE>\ssE(\xdf,\iSv)$
 and $\ssY>\ssY_{\xdf,\iSv}$
\begin{multline}\label{au:ag:ub:pnp:p7}
  \FuEx[\ssY,\ssE]{\rY,\rE}\VnormLp{\hxdf[{\erWe[]}]-\xdf}^2\leq \cst{}\ssY^{-1}\{K^2\miSv[K]^2\ssY^{-1}+\miSv[\Di_{\ydf}]^2\Di_{\ydf}^3+\miSv[\ssY_{o}]^2+\VnormLp{\ProjC[0]\xdf}^2\}\\\hfill
    +\cst{}\VnormLp{\ProjC[0]\xdf}^2\{
    \exp\big(\tfrac{-9\cmiSv[\sDi{\ssY}\wedge\sDi{\ssE}]\sDi{\ssY}\wedge\sDi{\ssE}}{20000\Vnormlp[1]{\fydf}}\big)\}    +\cst{}\ssE^{-1}\{K\miSv[K]^2+ \VnormLp{\ProjC[0]\xdf}^2\miSv[K]\}
  \end{multline}
Combining \eqref{au:ag:ub:pnp:p6} and \eqref{au:ag:ub:pnp:p7}
for   the cases \ref{au:ag:ub:pnp:p:m1}
$\ssE\in\nset{1,\ssE(\xdf,\iSv)}$ and \ref{au:ag:ub:pnp:p:m2}
$\ssE>\ssE(\xdf,\iSv)$ we obtain for all $\ssY>\ssY_{\xdf,\iSv}$ and for all $\ssE\in\Nz$
\begin{multline}\label{au:ag:ub:pnp:p8}
  \FuEx[\ssY,\ssE]{\rY,\rE}\VnormLp{\hxdf[{\erWe[]}]-\xdf}^2\leq
  \cst{}\VnormLp{\ProjC[0]\xdf}^2\big[\ssY^{-1}\vee \ssE^{-1} \vee\exp\big(\tfrac{-9\cmiSv[\sDi{\ssY}\wedge\sDi{\ssE}]\sDi{\ssY}\wedge\sDi{\ssE}}{20000\Vnormlp[1]{\fydf}}\big)\big]\\
+  \cst{}\ssY^{-1}\{K^2\miSv[K]^2+\miSv[\Di_{\ydf}]^2\Di_{\ydf}^3+\miSv[\ssY_{o}]^2\}
    +\cst{}\ssE^{-1}(1\vee\VnormLp{\ProjC[0]\xdf}^2)K\miSv[K]^2
  \end{multline}
Combining \eqref{au:ag:ub:pnp:p4} and \eqref{au:ag:ub:pnp:p8} for $K\in\Nz$
   with \ref{au:ag:ub:pnp:p:n1}
$\ssY\in\nset{1,\ssY_{\xdf,\iSv}}$ and \ref{au:ag:ub:pnp:p:n2}
$\ssY>\ssY_{\xdf,\iSv}$, respectively, and \eqref{au:ag:ub:pnp:p3}
for $K=0$, we obtain for all $K\in\Nz_0$ and for all $\ssY,\ssE\in\Nz$
\begin{multline}\label{au:ag:ub:pnp:p9}
  \FuEx[\ssY,\ssE]{\rY,\rE}\VnormLp{\hxdf[{\erWe[]}]-\xdf}^2\leq
  \cst{}\VnormLp{\ProjC[0]\xdf}^2\big[\ssY^{-1}\vee \ssE^{-1} \vee\exp\big(\tfrac{-9\cmiSv[\sDi{\ssY}\wedge\sDi{\ssE}]\sDi{\ssY}\wedge\sDi{\ssE}}{20000\Vnormlp[1]{\fydf}}\big)\big]\\
+  \cst{}\ssY^{-1}\{\miSv[1]^2\{\tfrac{(\VnormLp{\ProjC[0]\xdf}^2\vee1)K^2\miSv[K]^2}{\VnormLp{\ProjC[0]\xdf}^2\bias[{[K-1]}]^2(\xdf)}\Ind{\{K\geq1\}}+\Ind{\{K=0\}}\}+\miSv[\Di_{\ydf}]^2\Di_{\ydf}^3+\miSv[\ssY_{o}]^2\}\\
+\cst{}\ssE^{-1}\{(1\vee\VnormLp{\ProjC[0]\xdf}^2)K\miSv[K]^2\Ind{\{K\geq1\}}+\miSv[1]^2\Ind{\{K=0\}}\}.
  \end{multline}
  Consider the case \ref{au:ag:ub:pnp:np}. We destinguish for $\ssE\in\Nz$ with
 $\dr\ssE(\iSv):=\floor{289\log(3)\cmiSv[1]\miSv[1]}$
the following two
cases, \begin{inparaenum}[i]\renewcommand{\theenumi}{\dgrau\rm(\alph{enumi})}\item\label{au:ag:ub:pnp:np:m1}
$\ssE\in\nset{1,\ssE(\iSv)}$ and \item\label{au:ag:ub:pnp:np:m2}
$\ssE>\ssE(\iSv)$. \end{inparaenum}
Consider firstly the case \ref{au:ag:ub:pnp:np:m1}
$\ssE\in\nset{1,\ssE(\iSv)}$. We set $\pdDi=\mdDi=1$, and hence
$\mDi=1$, $\bias[1]^2(\xdf)\leq1$,
$\penSv[1]\leq\cpen\miSv[1]^2\ssY^{-1}$, $\miSv[1]^2\leq \miSv[\ssY_{o}]^2$, $\ssE(\iSv)\leq \cst{}\miSv[1]^2$ and due to
\cref{re:evrest} \ref{re:evrest:ii} $\FuVg[\ssE]{\rE}(\aixEv[1]^c)\leq
\cst{}\miSv[1]^2\ssE^{-2}$. Thereby, 
 \eqref{au:ag:ub:pnp:p1} implies for all $\ssY\in\Nz$ and $\ssE\in\nset{1,\ssE(\iSv)}$
 \begin{multline}\label{au:ag:ub:pnp:p10}
  %  \FuEx[\ssY,\ssE]{\rY,\rE}\VnormLp{\hxdf[{\erWe[]}]-\xdf}^2\leq
  % 2\penSv[1] +\tfrac{12}{7}\VnormLp{\ProjC[0]\xdf}^2\bias[1]^2(\xdf)+3 \VnormLp{\ProjC[0]\xdf}^2\bias[1]^2(\xdf)\\\hfill
  %   +\cst{}\miSv[1]^2\ssE^{-1} 
  %   \\\hfill
  %   +\cst{}\mRa{\xdf,\iSv}
  %   +\cst{}\ssY^{-1}\{\miSv[\Di_{\ydf}]^2\Di_{\ydf}^3+\miSv[\ssY_{o}]^2\}\\
   \FuEx[\ssY,\ssE]{\rY,\rE}\VnormLp{\hxdf[{\erWe[]}]-\xdf}^2\leq
   \cst{}\mRa{\xdf,\iSv}
    + \cst{}(1\vee\VnormLp{\ProjC[0]\xdf}^2)\miSv[1]^2\ssE^{-1}  
    +\cst{}\{\miSv[\Di_{\ydf}]^2\Di_{\ydf}^3+\miSv[\ssY_{o}]^2\}\ssY^{-1}
  \end{multline}
Consider secondly \ref{au:ag:ub:pnp:np:m2}
$\ssE>\ssE(\iSv)$.  We set  
$\sDi{\ssE}:=\max\{\Di\in\nset{1,\ssE}:289\log(\Di+2)\cmiSv[\Di]\miSv[\Di]\leq\ssE\}$, where the defining set containing $1$ is not
empty. For each
$\Di\in\nset{1,\sDi{\ssE}}$ holds 
$\ssE\geq289\log(\Di+2)\cmiSv\miSv$, and thus from
\cref{re:evrest} \ref{re:evrest:iii} follows
$\FuVg[\ssE]{\rE}(\aixEv[\Di]^c)\leq 11226\ssE^{-2}$. For $\aDi{\ssY}\in\nset{1,n}$
as in \ref{ak:ass:pen:oo} let
$\pdDi:=\aDi{\ssY}\wedge\sDi{\ssE}$, where
$\penSv[\aDi{\ssY}\wedge\sDi{\ssE}]\leq\penSv[\aDi{\ssY}]\leq \daRa{\aDi{\ssY}}{\xdf,\iSv}$, then from \eqref{au:ag:ub:pnp:p1} follows
 \begin{multline}\label{au:ag:ub:pnp:p11}
   \FuEx[\ssY,\ssE]{\rY,\rE}\VnormLp{\hxdf[{\erWe[]}]-\xdf}^2\leq
   2\daRa{\aDi{\ssY}}{\xdf,\iSv}+\tfrac{12}{7}\VnormLp{\ProjC[0]\xdf}^2\bias[\aDi{\ssY}\wedge\sDi{\ssE}]^2(\xdf)
   +3 \VnormLp{\ProjC[0]\xdf}^2\bias[\mDi]^2(\xdf)\\\hfill
    +\cst{}\VnormLp{\ProjC[0]\xdf}^2\Ind{\{\mDi>1\}}
    \exp\big(\tfrac{-9\cmiSv[\mdDi]\mdDi}{20000\Vnormlp[1]{\fydf}}\big)
    + \cst{}\big[
    \VnormLp{\ProjC[0]\xdf}^2\Ind{\{\mDi>1\}} \FuVg[\ssE]{\rE}(\aixEv[\mdDi]^c)+\ssE^{-1} \big]
    \\\hfill
    +\cst{}\mRa{\xdf,\iSv}
    +\cst{}\ssY^{-1}\{\miSv[\Di_{\ydf}]^2\Di_{\ydf}^3+\miSv[\ssY_{o}]^2+\VnormLp{\ProjC[0]\xdf}^2\Ind{\{\mDi>1\}}\}
\end{multline}
Let
$\sDi{\ssY}:=\argmin\{\daRa{\Di}{\xdf,\iSv}\vee\exp\big(\tfrac{-\cmiSv[\Di]\Di}{\Di_{\ydf}}\big):\Di\in\nset{1,\ssY}\}$,
where $\sDi{\ssY}\in\nset{\aDi{\ssY},1}$ by definition of
$\aDi{\ssY}$. Setting  
$\mdDi:=\sDi{\ssY}\wedge\sDi{\ssE}$ from
\cref{re:evrest} \ref{re:evrest:iii} follows
$\FuVg[\ssE]{\rE}(\aixEv[\mdDi]^c)\leq 53\ssE^{-1}$, while  $\mDi$ as in  definition
\eqref{au:de:*Di:ag} satisfies    $\VnormLp{\ProjC[0]\xdf}^2\bias[\mDi]^2(\xdf) \leq
  [\VnormLp{\ProjC[0]\xdf}^2+104\cpen]\dRa{\sDi{\ssY}\wedge\sDi{\ssE}}{\xdf,\iSv}$,
  where
  $\dRa{\sDi{\ssY}\wedge\sDi{\ssE}}{\xdf,\iSv}\leq\dRa{\sDi{\ssY}}{\xdf,\iSv}+\bias[\aDi{\ssY}\wedge\sDi{\ssE}]^2(\xdf)$, $\dRa{\aDi{\ssY}}{\xdf,\iSv}\leq\dRa{\sDi{\ssY}}{\xdf,\iSv}$
  and $\bias[\sDi{\ssY}\wedge\sDi{\ssE}]^2(\xdf)\leq
  \bias[\aDi{\ssY}\wedge\sDi{\ssE}]^2(\xdf)$. Thereby, we obtain for all $\ssY\in\Nz$ and $\ssE>\ssE(\iSv)$
 \begin{multline}\label{au:ag:ub:pnp:p12}
   \FuEx[\ssY,\ssE]{\rY,\rE}\VnormLp{\hxdf[{\erWe[]}]-\xdf}^2\leq\cst{}(1\vee\VnormLp{\ProjC[0]\xdf}^2)\{\daRa{\sDi{\ssY}}{\xdf,\iSv}+\bias[\aDi{\ssY}\wedge\sDi{\ssE}]^2(\xdf)\}\\\hfill
    +\cst{}\VnormLp{\ProjC[0]\xdf}^2\Ind{\{\mDi>1\}}
    \exp\big(\tfrac{-9\cmiSv[\sDi{\ssY}\wedge\sDi{\ssE}]\sDi{\ssY}\wedge\sDi{\ssE}}{20000\Vnormlp[1]{\fydf}}\big)
    + \cst{}\big[
    \VnormLp{\ProjC[0]\xdf}^2\Ind{\{\mDi>1\}}\ssE^{-1}+\ssE^{-1} \big]
    \\\hfill
    +\cst{}\mRa{\xdf,\iSv}
    +\cst{}\ssY^{-1}\{\miSv[\Di_{\ydf}]^2\Di_{\ydf}^3+\miSv[\ssY_{o}]^2+\VnormLp{\ProjC[0]\xdf}^2\Ind{\{\mDi>1\}}\}
\end{multline}
Since $\daRa{\sDi{\ssY}}{\xdf,\iSv}\geq\ssY^{-1}$ and
$\mRa{\xdf,\iSv}\geq\tfrac{1}{2}\VnormLp{\ProjC[0]\xdf}^2 \ssE^{-1}$
(see \cref{oo:rem:nm}) it follows  for all $\ssY\in\Nz$ and $\ssE>\ssE(\iSv)$
\begin{multline}\label{au:ag:ub:pnp:p13}
  \FuEx[\ssY,\ssE]{\rY,\rE}\VnormLp{\hxdf[{\erWe[]}]-\xdf}^2\leq\cst{}(1\vee\VnormLp{\ProjC[0]\xdf}^2)\min_{\Di\in\nset{1,\ssY}}\{\daRa{\Di}{\xdf,\iSv}\vee\exp\big(\tfrac{-\cmiSv[\Di]\Di}{\Di_{\ydf}}\big)\}\\\hfill
+\cst{}(1\vee\VnormLp{\ProjC[0]\xdf}^2)\{\bias[\aDi{\ssY}\wedge\sDi{\ssE}]^2(\xdf)\vee\exp\big(\tfrac{-\cmiSv[\sDi{\ssE}]\sDi{\ssE}}{\Di_{\ydf}}\big)\} \\\hfill
+\cst{}\mRa{\xdf,\iSv}+\cst{}\ssE^{-1}+\cst{}\ssY^{-1}\{\miSv[\Di_{\ydf}]^2\Di_{\ydf}^3+\miSv[\ssY_{o}]^2\}
\end{multline}
Combining \eqref{au:ag:ub:pnp:p10} and \eqref{au:ag:ub:pnp:p13}
for   the cases \ref{au:ag:ub:pnp:np:m1}
$\ssE\in\nset{1,\ssE(\iSv)}$ and \ref{au:ag:ub:pnp:np:m2}
$\ssE>\ssE(\iSv)$ we obtain for all $\ssY,\ssE\in\Nz$
\begin{multline}\label{au:ag:ub:pnp:p14}
    \FuEx[\ssY,\ssE]{\rY,\rE}\VnormLp{\hxdf[{\erWe[]}]-\xdf}^2\leq\cst{}(1\vee\VnormLp{\ProjC[0]\xdf}^2)\min_{\Di\in\nset{1,\ssY}}\{\daRa{\Di}{\xdf,\iSv}\vee\exp\big(\tfrac{-\cmiSv[\Di]\Di}{\Di_{\ydf}}\big)\}\Ind{\{\ssE>\ssE(\iSv)\}}\\\hfill
+\cst{}(1\vee\VnormLp{\ProjC[0]\xdf}^2)\{\bias[\aDi{\ssY}\wedge\sDi{\ssE}]^2(\xdf)\vee\exp\big(\tfrac{-\cmiSv[\sDi{\ssE}]\sDi{\ssE}}{\Di_{\ydf}}\big)\}\Ind{\{\ssE>\ssE(\iSv)\}} \\\hfill
 +\cst{}\mRa{\xdf,\iSv}   + \cst{}(1\vee\VnormLp{\ProjC[0]\xdf}^2)\miSv[1]^2\ssE^{-1}  
    +\cst{}\{\miSv[\Di_{\ydf}]^2\Di_{\ydf}^3+\miSv[\ssY_{o}]^2\}\ssY^{-1}
\end{multline}
which shows the assertion \eqref{au:ag:ub:pnp:e2} and  completes the
proof of \cref{au:ag:ub:pnp}.\proEnd\end{pro}
\subsubsection{Proofs of \cref{au:mrb}}\label{a:au:mrb}
% ....................................................................
% Te <<Upper bound random weights>>
% ....................................................................
\begin{te}
 Below  we state the proofs of  \cref{au:mrb:re:SrWe:ag} and \cref{au:mrb:re:SrWe:ms}. The  proof of \cref{au:mrb:re:SrWe:ag} is based on \cref{mrb:re:erWe} given first.
\end{te}
% ....................................................................
% <<Re Random weights>>
% ....................................................................
\begin{cor}\label{mrb:re:erWe} Consider the data-driven aggreagtion weights
  $\erWe[]$ as in \eqref{au:de:erWe}. Under condition
  \ref{au:ass:pen:oo} for any $l\in\nset{1,\ssY}$ with
  $\daRa{l}{\xdfCw[],\iSv}=[\xdfCw[l]\vee \DipenSv[l]\, n^{-1}]$ holds
  \begin{resListeN}[]
  \item\label{mrb:re:erWe:i} with
    $\dr\aixEv[l]:=\set{1/4\leq\iSv[j]^{-1}\eiSv[j]\leq9/4,\;\forall\;j\in\nset{1,l}}$ for all $k\in\nsetro{1,l}$ 
    we have\\
   $\erWe[k]\Ind{\setB{\VnormLp{\hxdfPr[l]-\dxdfPr[l]}^2<\peneSv[l]/7}}\Ind{\aixEv[l]}$\\\null\hfill$\leq
  \exp\big(\rWn\big\{[\tfrac{25}{2}\cpen+\tfrac{1}{8}\xdfCr^2]\dRa{l}{\xdfCw[],\iSv}-\tfrac{1}{8}\VnormLp{\ProjC[0]\xdf}^2\bias^2(\xdf)-\tfrac{1}{50}\penSv\big\}\big)$.
  \item\label{mrb:re:erWe:ii} with $\VnormLp{\ProjC[l]\dxdfPr[\ssY]}^2=2\sum_{j=l+1}^{\ssY}\iSv[j]^{-1}\eiSv[j]|\fxdf[j]|^2$
    for all $\Di\in\nsetlo{l,\ssY}$ we have\\
    $\erWe\Ind{\setB{\VnormLp{\hxdfPr-\dxdfPr}^2<\penSv/7}} \leq
   \exp\big(\rWn\big\{-\tfrac{1}{2}\peneSv+\tfrac{3}{2}\VnormLp{\ProjC[l]\dxdfPr[\ssY]}^2+\peneSv[l]\big\}\big)$.
  \end{resListeN}
\end{cor}
% --------------------------------------------------------------------
% <<Proof Re Random weights>> angepasst
% --------------------------------------------------------------------
\begin{pro}[Proof of \cref{mrb:re:erWe}.]The assertion
  \ref{mrb:re:erWe:i} follows from \cref{re:erWe} \ref{re:erWe:i}
  using that $\xdfCr^2\daRa{\Di}{\xdfCw[],\iSv}\geq\VnormLp{\ProjC[0]\xdf}^2\bias^2(\xdf)$
uniformely for all $\xdf\in\rwCxdf$ and for all
$\Di\in\Nz$.  The assertion
  \ref{mrb:re:erWe:ii} equals \cref{re:erWe} \ref{re:erWe:ii}.\proEnd
\end{pro}
% ....................................................................
% <<Proof Re Sum Random weights>>
% ....................................................................
\begin{pro}[Proof of \cref{au:mrb:re:SrWe:ag}.]
The proof follows line by line the proof of \cref{au:re:SrWe:ag} using
\cref{mrb:re:erWe} rather than \cref{re:erWe}, and we omit the details.\proEnd  
\end{pro}

%%% Local Variables:
%%% mode: latex
%%% TeX-master: "_0DACD"
%%% End:

\chapter{Simulation skim}\label{SIMULATION}
%% In this part, we will illustrate the computational properties of the presented methods.
In particular, we are interested in computing an estimation of the quadratic error of the estimator given by the posterior mean for different values of $\eta$, the iteration parameter, and compare it to the error obtained with oracle estimators.
We also compare the adaptive estimators to their oracle counterpart, obtained by truncating them at the index minimizing the mean square error.
\textcolor{red}{THE PROPER NAME IS NOT ORACLE HERE AS IT DEPENDS ON $\omega$ HOW SHOULD I CALL IT ? SHOULD I FORMULATE IT FOR EACH $\omega$ ?}
In other words, given an estimator $\widehat{\theta}$ of $\theta^{\circ}$, we define $\widehat{\theta}^{\circ}$ for each realization $y$ of $Y$ as follows :
\begin{align*}
m_{y}^{\circ} &:= \min \left\{\argmin\limits_{m \in \left\llbracket 1, G_{\epsilon} \right\rrbracket} \left\{\sum\limits_{j = 1}^{\infty} \left( \left(\widehat{\theta}_{j} \mathds{1}_{\left\{j \leq m\right\}} + \theta^{\times} \mathds{1}_{\left\{j > m\right\}}\right) - \theta^{\circ}\right)^{2}\right\}\right\},\\
\widehat{\theta}^{\circ} &:= \widehat{\theta}_{j} \mathds{1}_{\left\{j \leq m_{y}^{\circ}\right\}} + \theta^{\times} \mathds{1}_{\left\{j > m_{y}^{\circ}\right\}}.
\end{align*}

In particular we call "oracle projection estimator" the sequence obtained this way when $\left(\widehat{\theta}_{j}\right)_{j \in \mathbb{N}} = \left(\frac{Y_{j}}{\lambda_{j}} \mathds{1}_{\left\{j \in \left\llbracket 1, G_{\epsilon} \right\rrbracket \right\}} + \theta^{\times}_{j} \mathds{1}_{\left\{j \in \left\llbracket G_{\epsilon}, \infty \right\rrbracket \right\}}\right)_{j \in \mathbb{N}};$ and "oracle Bayes estimator" the sequence obtained this way when $\left(\widehat{\theta}_{j}\right)_{j \in \mathbb{N}} = \mathbb{E}_{\boldsymbol{\theta}, M \vert Y^{1}}\left[\boldsymbol{\theta}\right].$

\medskip

The simulation protocole is the following :

\begin{algorithm}[H]
\KwData{$\epsilon_{1}$ maximal value for $\epsilon$, $\epsilon_{2}$ minimal value for $\epsilon$, $\theta^{\circ}$, $\lambda$, $\theta^{\times}$, $s$, $\eta$, $N_{exp}$ number of Monte-Carlo experiences, $N$ number of intermediate values for $\epsilon$ between $\epsilon_{1}$ and $\epsilon_{2}$.}
\KwResult{$\widehat{\mathcal{R}_{\mathbb{L}^{2}}}\left(\widehat{\theta}, \theta^{\circ}\right)$ estimation of the mean square error of $\widehat{\theta}$ for each value of $\epsilon$ with the specified true parameters $\theta^{\circ}$, $\widehat{\mathbb{P}}_{M\vert Y}$ estimation of the average distribution of $M \vert Y$ for each value of $\epsilon$ for the specified true parameter $\theta^{\circ}$.}

Generate $\left(\boldsymbol{e}_{j}\right)_{j \in \llbracket 1, N_{exp}\rrbracket}$ vectors of length $\left\lfloor\frac{1}{\epsilon_{2}}\right\rfloor$ whose elements are realizations of $\mathcal{N}(0, 1)$

Initiate $\widehat{\mathcal{R}_{\mathbb{L}^{2}}}\left(\widehat{\theta}, \theta^{\circ}\right)$ as a vector of zeros of length $N$

Initiate $\widehat{\mathbb{P}}_{M\vert Y}$ as a vector of zeros of length $N$

\For{$n \in 1 : N$}{
	$\epsilon := \epsilon_{2} - n \cdot \frac{\epsilon_{2} - \epsilon_{1}}{N}$
	
	\For{$k \in 1 : N_{exp}$}{
		\begin{align*}
		y &:= \left(\lambda_{j} \cdot \theta^{\circ}_{j}\right)_{j \in \left\llbracket1, \left\lfloor\frac{1}{\epsilon}\right\rfloor \right\rrbracket} + \sqrt{\epsilon} \cdot \left(\boldsymbol{e}_{k, j}\right)_{j \in \left\llbracket 1, \left\lfloor\frac{1}{\epsilon}\right\rfloor \right\rrbracket}\\
		\widehat{\mathbb{P}}_{M\vert Y} &= \widehat{\mathbb{P}}_{M\vert Y} + \frac{\left(\mathbb{P}_{M\vert Y = y}(M = m)\right)_{m \in \llbracket1, G_{\epsilon}\rrbracket}}{N_{exp}}\\
		\widehat{\theta} &:= \mathbb{E}\left[\boldsymbol{\theta}^{M}\vert Y=y\right]\\
		\widehat{\mathcal{R}_{\mathbb{L}^{2}}}\left(\widehat{\theta}, \theta^{\circ}\right)_{n} &= \widehat{\mathcal{R}_{\mathbb{L}^{2}}}\left(\widehat{\theta}, \theta^{\circ}\right)_{n} + \frac{\left(\sum\limits_{j = 1}^{\infty} \left(\theta^{\circ}_{j} - \widehat{\theta}_{j}\right)^{2}\right)}{N_{exp}}
		\end{align*}
	}
}
\end{algorithm}

Results of those simulations are represented in \textsc{\autoref{EQM}}.
We see that in every situation, the Bayesian method outperforms the model selection.
This might be due to the choice of the penalization constant fixed to $3$ here.

\medskip

In cases where $\theta^{\circ}$ is polynomial, the Bayesian method does not perform badly compared to the oracle estimates, even in the case where $\lambda$ is exponentially decreasing, which let suppose that the goal to weaken the hypothesis on $\lambda$ is reachable.

However, when $\theta^{\circ}$ decreases exponentially, the adaptive estimates are easily outperformed by the oracle ones.
The adaptive methods estimate way too many parameters in this case.
This phenomenon is followed by an important numerical instability at values of $\epsilon$ where $G_{\epsilon}$ increases (note that the bayesian method seems to be more resistant to it).


\section{Simulations for the inverse Gaussian sequence space model}\label{D.1}

\begin{figure}
\centering
\begin{subfigure}{.5\textwidth}
  \centering
  \includegraphics[width=1.\linewidth]{gauss/old/empirical_EQM/EQM1.pdf}
  \caption{$\theta^{\circ}$ polynomial and $\lambda$ polynomial}
  \label{fig3:sub1}
\end{subfigure}%
\begin{subfigure}{.5\textwidth}
  \centering
  \includegraphics[width=1.\linewidth]{gauss/old/empirical_EQM/EQM2.pdf}
  \caption{$\theta^{\circ}$ polynomial and $\lambda$ exponential}
  \label{fig3:sub2}
\end{subfigure}
\begin{subfigure}{.5\textwidth}
  \centering
  \includegraphics[width=1.\linewidth]{gauss/old/empirical_EQM/EQM3}
  \caption{$\theta^{\circ}$ exponential and $\lambda$ polynomial}
  \label{fig3:sub3}
\end{subfigure}%
\begin{subfigure}{.5\textwidth}
  \centering
  \includegraphics[width=1.\linewidth]{gauss/old/empirical_EQM/EQM4}
  \caption{$\theta^{\circ}$ exponential and $\lambda$ exponential}
  \label{fig3:sub4}
\end{subfigure}
\caption{Performance of the model selection; oracle projection; oracle shrinkage; and shrinkage estimators for different values of $\eta$ }
\label{EQM}
\end{figure}




\begin{figure}
\centering
\begin{subfigure}{.5\textwidth}
  \centering
  \includegraphics[width=1.\linewidth]{gauss/old/distribution_M/influence_n/M1.pdf}
  \caption{$\theta^{\circ}$ polynomial and $\lambda$ polynomial}
  \label{fig3:sub1}
\end{subfigure}%
\begin{subfigure}{.5\textwidth}
  \centering
  \includegraphics[width=1.\linewidth]{gauss/old/distribution_M/influence_n/M2.pdf}
  \caption{$\theta^{\circ}$ polynomial and $\lambda$ exponential}
  \label{fig3:sub2}
\end{subfigure}
\begin{subfigure}{.5\textwidth}
  \centering
  \includegraphics[width=1.\linewidth]{gauss/old/distribution_M/influence_n/M3}
  \caption{$\theta^{\circ}$ exponential and $\lambda$ polynomial}
  \label{fig3:sub3}
\end{subfigure}%
\begin{subfigure}{.5\textwidth}
  \centering
  \includegraphics[width=1.\linewidth]{gauss/old/distribution_M/influence_n/M4}
  \caption{$\theta^{\circ}$ exponential and $\lambda$ exponential}
  \label{fig3:sub4}
\end{subfigure}
\caption{Estimated median of the quadratic error of the estimator given by the posterior mean for different classes of $\theta^{\circ}$ and $\lambda$ with $\theta^{\times} \equiv 0$ and $s \equiv 1$.}
\label{EQM}
\end{figure}






\begin{figure}
\centering
\begin{subfigure}{.5\textwidth}
  \centering
  \includegraphics[width=1.\linewidth]{gauss/old/distribution_M/influence_iteration/IterationExpoT.png}
  \caption{$\theta^{\circ}$ polynomial and $\lambda$ polynomial}
  \label{fig3:sub1}
\end{subfigure}%
\begin{subfigure}{.5\textwidth}
  \centering
  \includegraphics[width=1.\linewidth]{gauss/old/distribution_M/influence_iteration/IterationExpoTL.png}
  \caption{$\theta^{\circ}$ polynomial and $\lambda$ exponential}
  \label{fig3:sub2}
\end{subfigure}
\begin{subfigure}{.5\textwidth}
  \centering
  \includegraphics[width=1.\linewidth]{gauss/old/distribution_M/influence_iteration/IterationExpoTL'.png}
  \caption{$\theta^{\circ}$ exponential and $\lambda$ polynomial}
  \label{fig3:sub3}
\end{subfigure}%
\begin{subfigure}{.5\textwidth}
  \centering
  \includegraphics[width=1.\linewidth]{gauss/old/distribution_M/influence_iteration/IterationsExpoL.png}
  \caption{$\theta^{\circ}$ exponential and $\lambda$ exponential}
  \label{fig3:sub4}
\end{subfigure}
\caption{Estimated median of the quadratic error of the estimator given by the posterior mean for different classes of $\theta^{\circ}$ and $\lambda$ with $\theta^{\times} \equiv 0$ and $s \equiv 1$.}
\label{EQM}
\end{figure}






\begin{figure}
\centering
\begin{subfigure}{.5\textwidth}
  \centering
  \includegraphics[width=1.\linewidth]{gauss/old/distribution_M/influence_iteration/iteration.pdf}
  \caption{$\theta^{\circ}$ polynomial and $\lambda$ polynomial}
  \label{fig3:sub1}
\end{subfigure}%
\caption{Estimated median of the quadratic error of the estimator given by the posterior mean for different classes of $\theta^{\circ}$ and $\lambda$ with $\theta^{\times} \equiv 0$ and $s \equiv 1$.}
\label{EQM}
\end{figure}



\begin{figure}
\centering
\begin{subfigure}{.5\textwidth}
  \centering
  \includegraphics[width=1.\linewidth]{gauss/adaptive/aggregation_surv.png}
  \caption{$\theta^{\circ}$ polynomial and $\lambda$ polynomial}
  \label{fig3:sub1}
\end{subfigure}%
\begin{subfigure}{.5\textwidth}
  \centering
  \includegraphics[width=1.\linewidth]{gauss/adaptive/aggregation.png}
  \caption{$\theta^{\circ}$ polynomial and $\lambda$ exponential}
  \label{fig3:sub2}
\end{subfigure}
\begin{subfigure}{.5\textwidth}
  \centering
  \includegraphics[width=1.\linewidth]{gauss/adaptive/model_selection.png}
  \caption{$\theta^{\circ}$ exponential and $\lambda$ polynomial}
  \label{fig3:sub3}
\end{subfigure}%
\begin{subfigure}{.5\textwidth}
  \centering
  \includegraphics[width=1.\linewidth]{gauss/adaptive/penalised_contrast.png}
  \caption{$\theta^{\circ}$ exponential and $\lambda$ exponential}
  \label{fig3:sub4}
\end{subfigure}
\caption{Estimated median of the quadratic error of the estimator given by the posterior mean for different classes of $\theta^{\circ}$ and $\lambda$ with $\theta^{\times} \equiv 0$ and $s \equiv 1$.}
\label{EQM}
\end{figure}

\begin{figure}
\centering
\begin{subfigure}{.5\textwidth}
  \centering
  \includegraphics[width=1.\linewidth]{gauss/error/aggregation_error.png}
  \caption{$\theta^{\circ}$ polynomial and $\lambda$ polynomial}
  \label{fig3:sub1}
\end{subfigure}%
\begin{subfigure}{.5\textwidth}
  \centering
  \includegraphics[width=1.\linewidth]{gauss/error/model_selection_error.png}
  \caption{$\theta^{\circ}$ polynomial and $\lambda$ exponential}
  \label{fig3:sub2}
\end{subfigure}
\begin{subfigure}{.5\textwidth}
  \centering
  \includegraphics[width=1.\linewidth]{gauss/error/oracle_error.png}
  \caption{$\theta^{\circ}$ exponential and $\lambda$ polynomial}
  \label{fig3:sub3}
\end{subfigure}%
\begin{subfigure}{.5\textwidth}
  \centering
  \includegraphics[width=1.\linewidth]{gauss/error/oracle_estimator.png}
  \caption{$\theta^{\circ}$ exponential and $\lambda$ exponential}
  \label{fig3:sub4}
\end{subfigure}
\begin{subfigure}{.5\textwidth}
  \centering
  \includegraphics[width=1.\linewidth]{gauss/error/oracle_threshold.png}
  \caption{$\theta^{\circ}$ exponential and $\lambda$ exponential}
  \label{fig3:sub4}
\end{subfigure}
\caption{Estimated median of the quadratic error of the estimator given by the posterior mean for different classes of $\theta^{\circ}$ and $\lambda$ with $\theta^{\times} \equiv 0$ and $s \equiv 1$.}
\label{EQM}
\end{figure}


\begin{figure}
\centering
\begin{subfigure}{.5\textwidth}
  \centering
  \includegraphics[width=1.\linewidth]{gauss/estimation/bayes.png}
  \caption{$\theta^{\circ}$ polynomial and $\lambda$ polynomial}
  \label{fig3:sub1}
\end{subfigure}%
\begin{subfigure}{.5\textwidth}
  \centering
  \includegraphics[width=1.\linewidth]{gauss/estimation/frequentist.png}
  \caption{$\theta^{\circ}$ polynomial and $\lambda$ exponential}
  \label{fig3:sub2}
\end{subfigure}
\caption{Estimated median of the quadratic error of the estimator given by the posterior mean for different classes of $\theta^{\circ}$ and $\lambda$ with $\theta^{\times} \equiv 0$ and $s \equiv 1$.}
\label{EQM}
\end{figure}



\begin{figure}
\centering
\begin{subfigure}{.5\textwidth}
  \centering
  \includegraphics[width=1.\linewidth]{gauss/MCMC/error_aggregation.png}
  \caption{$\theta^{\circ}$ polynomial and $\lambda$ polynomial}
  \label{fig3:sub1}
\end{subfigure}%
\begin{subfigure}{.5\textwidth}
  \centering
  \includegraphics[width=1.\linewidth]{gauss/MCMC/error_model_selection.png}
  \caption{$\theta^{\circ}$ polynomial and $\lambda$ exponential}
  \label{fig3:sub2}
\end{subfigure}
\begin{subfigure}{.5\textwidth}
  \centering
  \includegraphics[width=1.\linewidth]{gauss/MCMC/error_oracle.png}
  \caption{$\theta^{\circ}$ polynomial and $\lambda$ exponential}
  \label{fig3:sub2}
\end{subfigure}
\begin{subfigure}{.5\textwidth}
  \centering
  \includegraphics[width=1.\linewidth]{gauss/MCMC/evolution_error_aggregation.png}
  \caption{$\theta^{\circ}$ polynomial and $\lambda$ exponential}
  \label{fig3:sub2}
\end{subfigure}
\begin{subfigure}{.5\textwidth}
  \centering
  \includegraphics[width=1.\linewidth]{gauss/MCMC/evolution_error_model_selection.png}
  \caption{$\theta^{\circ}$ polynomial and $\lambda$ exponential}
  \label{fig3:sub2}
\end{subfigure}
\caption{Estimated median of the quadratic error of the estimator given by the posterior mean for different classes of $\theta^{\circ}$ and $\lambda$ with $\theta^{\times} \equiv 0$ and $s \equiv 1$.}
\label{EQM}
\end{figure}

\begin{figure}
\centering
\begin{subfigure}{.5\textwidth}
  \centering
  \includegraphics[width=1.\linewidth]{gauss/observation/observation.png}
  \caption{$\theta^{\circ}$ polynomial and $\lambda$ polynomial}
  \label{fig3:sub1}
\end{subfigure}%
\caption{Estimated median of the quadratic error of the estimator given by the posterior mean for different classes of $\theta^{\circ}$ and $\lambda$ with $\theta^{\times} \equiv 0$ and $s \equiv 1$.}
\label{EQM}
\end{figure}


\begin{figure}
\centering
\begin{subfigure}{.5\textwidth}
  \centering
  \includegraphics[width=1.\linewidth]{gauss/truncation/truncation.png}
  \caption{$\theta^{\circ}$ polynomial and $\lambda$ polynomial}
  \label{fig3:sub1}
\end{subfigure}%
\caption{Estimated median of the quadratic error of the estimator given by the posterior mean for different classes of $\theta^{\circ}$ and $\lambda$ with $\theta^{\times} \equiv 0$ and $s \equiv 1$.}
\label{EQM}
\end{figure}


\begin{figure}
\centering
\begin{subfigure}{.5\textwidth}
  \centering
  \includegraphics[width=1.\linewidth]{gauss/truth/truth.png}
  \caption{$\theta^{\circ}$ polynomial and $\lambda$ polynomial}
  \label{fig3:sub1}
\end{subfigure}%
\caption{Estimated median of the quadratic error of the estimator given by the posterior mean for different classes of $\theta^{\circ}$ and $\lambda$ with $\theta^{\times} \equiv 0$ and $s \equiv 1$.}
\label{EQM}
\end{figure}


\begin{figure}
\centering
\begin{subfigure}{.5\textwidth}
  \centering
  \includegraphics[width=1.\linewidth]{gauss/bayesian/tirage-direct.pdf}
  \caption{$\theta^{\circ}$ polynomial and $\lambda$ polynomial}
  \label{fig3:sub1}
\end{subfigure}%
\begin{subfigure}{.5\textwidth}
  \centering
  \includegraphics[width=1.\linewidth]{gauss/bayesian/tirage-inverse.pdf}
  \caption{$\theta^{\circ}$ polynomial and $\lambda$ polynomial}
  \label{fig3:sub1}
\end{subfigure}%
\caption{Estimated median of the quadratic error of the estimator given by the posterior mean for different classes of $\theta^{\circ}$ and $\lambda$ with $\theta^{\times} \equiv 0$ and $s \equiv 1$.}
\label{EQM}
\end{figure}

\section{Simulations for circular density deconvolution}\label{D.2}

\begin{figure}
\centering
\begin{subfigure}{.4\textwidth}
  \centering
  \includegraphics[width=1.\linewidth]{density/anim-proj/1.png}
  \caption{$\theta^{\circ}$ polynomial and $\lambda$ polynomial}
  \label{fig3:sub1}
\end{subfigure}%
\begin{subfigure}{.4\textwidth}
  \centering
  \includegraphics[width=1.\linewidth]{density/anim-proj/20.png}
  \caption{$\theta^{\circ}$ polynomial and $\lambda$ exponential}
  \label{fig3:sub2}
\end{subfigure}
\begin{subfigure}{.4\textwidth}
  \centering
  \includegraphics[width=1.\linewidth]{density/anim-proj/40.png}
  \caption{$\theta^{\circ}$ polynomial and $\lambda$ exponential}
  \label{fig3:sub2}
\end{subfigure}
\begin{subfigure}{.4\textwidth}
  \centering
  \includegraphics[width=1.\linewidth]{density/anim-proj/70.png}
  \caption{$\theta^{\circ}$ polynomial and $\lambda$ exponential}
  \label{fig3:sub2}
\end{subfigure}
\begin{subfigure}{.4\textwidth}
  \centering
  \includegraphics[width=1.\linewidth]{density/anim-proj/90.png}
  \caption{$\theta^{\circ}$ polynomial and $\lambda$ exponential}
  \label{fig3:sub2}
\end{subfigure}
\caption{Estimated median of the quadratic error of the estimator given by the posterior mean for different classes of $\theta^{\circ}$ and $\lambda$ with $\theta^{\times} \equiv 0$ and $s \equiv 1$.}
\label{EQM}
\end{figure}





\begin{figure}
\centering
\begin{subfigure}{.5\textwidth}
  \centering
  \includegraphics[width=1.\linewidth]{density/anim-proj-expo/1.png}
  \caption{$\theta^{\circ}$ polynomial and $\lambda$ polynomial}
  \label{fig3:sub1}
\end{subfigure}%
\begin{subfigure}{.5\textwidth}
  \centering
  \includegraphics[width=1.\linewidth]{density/anim-proj-expo/10.png}
  \caption{$\theta^{\circ}$ polynomial and $\lambda$ exponential}
  \label{fig3:sub2}
\end{subfigure}
\begin{subfigure}{.4\textwidth}
  \centering
  \includegraphics[width=1.\linewidth]{density/anim-proj-expo/20.png}
  \caption{$\theta^{\circ}$ polynomial and $\lambda$ exponential}
  \label{fig3:sub2}
\end{subfigure}
\begin{subfigure}{.4\textwidth}
  \centering
  \includegraphics[width=1.\linewidth]{density/anim-proj-expo/30.png}
  \caption{$\theta^{\circ}$ polynomial and $\lambda$ exponential}
  \label{fig3:sub2}
\end{subfigure}
\caption{Estimated median of the quadratic error of the estimator given by the posterior mean for different classes of $\theta^{\circ}$ and $\lambda$ with $\theta^{\times} \equiv 0$ and $s \equiv 1$.}
\label{EQM}
\end{figure}





\begin{figure}
\centering
\begin{subfigure}{.7\textwidth}
  \centering
  \includegraphics[width=1.\linewidth]{density/compare-direct/1.png}
  \caption{$\theta^{\circ}$ polynomial and $\lambda$ polynomial}
  \label{fig3:sub1}
\end{subfigure}%

\begin{subfigure}{.7\textwidth}
  \centering
  \includegraphics[width=1.\linewidth]{density/compare-direct/20.png}
  \caption{$\theta^{\circ}$ polynomial and $\lambda$ exponential}
  \label{fig3:sub2}
\end{subfigure}
\end{figure}

\begin{figure}\ContinuedFloat
\begin{subfigure}{.7\textwidth}
  \centering
  \includegraphics[width=1.\linewidth]{density/compare-direct/40.png}
  \caption{$\theta^{\circ}$ polynomial and $\lambda$ exponential}
  \label{fig3:sub2}
\end{subfigure}

\begin{subfigure}{.7\textwidth}
  \centering
  \includegraphics[width=1.\linewidth]{density/compare-direct/60.png}
  \caption{$\theta^{\circ}$ polynomial and $\lambda$ exponential}
  \label{fig3:sub2}
\end{subfigure}
\end{figure}

\begin{figure}\ContinuedFloat
\begin{subfigure}{.7\textwidth}
  \centering
  \includegraphics[width=1.\linewidth]{density/compare-direct/80.png}
  \caption{$\theta^{\circ}$ polynomial and $\lambda$ exponential}
  \label{fig3:sub2}
\end{subfigure}

\begin{subfigure}{.7\textwidth}
  \centering
  \includegraphics[width=1.\linewidth]{density/compare-direct/100.png}
  \caption{$\theta^{\circ}$ polynomial and $\lambda$ exponential}
  \label{fig3:sub2}
\end{subfigure}
\end{figure}

\begin{figure}\ContinuedFloat
\begin{subfigure}{.7\textwidth}
  \centering
  \includegraphics[width=1.\linewidth]{density/compare-direct/120.png}
  \caption{$\theta^{\circ}$ polynomial and $\lambda$ exponential}
  \label{fig3:sub2}
\end{subfigure}
\caption{Estimated median of the quadratic error of the estimator given by the posterior mean for different classes of $\theta^{\circ}$ and $\lambda$ with $\theta^{\times} \equiv 0$ and $s \equiv 1$.}
\label{EQM}
\end{figure}


\begin{figure}
\centering
\begin{subfigure}{.45\textwidth}
  \centering
  \includegraphics[width=1.\linewidth]{density/distM/1.png}
  \caption{$\theta^{\circ}$ polynomial and $\lambda$ polynomial}
  \label{fig3:sub1}
\end{subfigure}%
\begin{subfigure}{.45\textwidth}
  \centering
  \includegraphics[width=1.\linewidth]{density/distM/10.png}
  \caption{$\theta^{\circ}$ polynomial and $\lambda$ polynomial}
  \label{fig3:sub1}
\end{subfigure}%

\begin{subfigure}{.45\textwidth}
  \centering
  \includegraphics[width=1.\linewidth]{density/distM/20.png}
  \caption{$\theta^{\circ}$ polynomial and $\lambda$ polynomial}
  \label{fig3:sub1}
\end{subfigure}%
\begin{subfigure}{.45\textwidth}
  \centering
  \includegraphics[width=1.\linewidth]{density/distM/30.png}
  \caption{$\theta^{\circ}$ polynomial and $\lambda$ polynomial}
  \label{fig3:sub1}
\end{subfigure}%

\begin{subfigure}{.45\textwidth}
  \centering
  \includegraphics[width=1.\linewidth]{density/distM/40.png}
  \caption{$\theta^{\circ}$ polynomial and $\lambda$ polynomial}
  \label{fig3:sub1}
\end{subfigure}%
\begin{subfigure}{.45\textwidth}
  \centering
  \includegraphics[width=1.\linewidth]{density/distM/50.png}
  \caption{$\theta^{\circ}$ polynomial and $\lambda$ polynomial}
  \label{fig3:sub1}
\end{subfigure}%
\end{figure}

\begin{figure}\ContinuedFloat
\begin{subfigure}{.45\textwidth}
  \centering
  \includegraphics[width=1.\linewidth]{density/distM/60.png}
  \caption{$\theta^{\circ}$ polynomial and $\lambda$ polynomial}
  \label{fig3:sub1}
\end{subfigure}%
\begin{subfigure}{.45\textwidth}
  \centering
  \includegraphics[width=1.\linewidth]{density/distM/70.png}
  \caption{$\theta^{\circ}$ polynomial and $\lambda$ polynomial}
  \label{fig3:sub1}
\end{subfigure}%

\begin{subfigure}{.45\textwidth}
  \centering
  \includegraphics[width=1.\linewidth]{density/distM/80.png}
  \caption{$\theta^{\circ}$ polynomial and $\lambda$ polynomial}
  \label{fig3:sub1}
\end{subfigure}%
\begin{subfigure}{.45\textwidth}
  \centering
  \includegraphics[width=1.\linewidth]{density/distM/90.png}
  \caption{$\theta^{\circ}$ polynomial and $\lambda$ polynomial}
  \label{fig3:sub1}
\end{subfigure}%

\caption{Estimated median of the quadratic error of the estimator given by the posterior mean for different classes of $\theta^{\circ}$ and $\lambda$ with $\theta^{\times} \equiv 0$ and $s \equiv 1$.}
\label{EQM}
\end{figure}




\begin{figure}
\centering
  \centering
  \includegraphics[width=1.\linewidth]{density/EQMdirect.pdf}
\caption{Estimated median of the quadratic error of the estimator given by the posterior mean for different classes of $\theta^{\circ}$ and $\lambda$ with $\theta^{\times} \equiv 0$ and $s \equiv 1$.}
\label{EQM}
\end{figure}



\begin{figure}
\centering
\begin{subfigure}{.40\textwidth}
  \centering
  \includegraphics[width=1.\linewidth]{density/illu-deconv1/test-1.png}
  \caption{$\theta^{\circ}$ polynomial and $\lambda$ polynomial}
  \label{fig3:sub1}
\end{subfigure}%
\begin{subfigure}{.40\textwidth}
  \centering
  \includegraphics[width=1.\linewidth]{density/illu-deconv1/test-5.png}
  \caption{$\theta^{\circ}$ polynomial and $\lambda$ exponential}
  \label{fig3:sub2}
\end{subfigure}

\begin{subfigure}{.40\textwidth}
  \centering
  \includegraphics[width=1.\linewidth]{density/illu-deconv1/test-10.png}
  \caption{$\theta^{\circ}$ polynomial and $\lambda$ exponential}
  \label{fig3:sub2}
\end{subfigure}
\begin{subfigure}{.40\textwidth}
  \centering
  \includegraphics[width=1.\linewidth]{density/illu-deconv1/test-15.png}
  \caption{$\theta^{\circ}$ polynomial and $\lambda$ exponential}
  \label{fig3:sub2}
\end{subfigure}

\begin{subfigure}{.40\textwidth}
  \centering
  \includegraphics[width=1.\linewidth]{density/illu-deconv1/test-20.png}
  \caption{$\theta^{\circ}$ polynomial and $\lambda$ exponential}
  \label{fig3:sub2}
\end{subfigure}
\begin{subfigure}{.40\textwidth}
  \centering
  \includegraphics[width=1.\linewidth]{density/illu-deconv1/test-25.png}
  \caption{$\theta^{\circ}$ polynomial and $\lambda$ exponential}
  \label{fig3:sub2}
\end{subfigure}
\caption{Estimated median of the quadratic error of the estimator given by the posterior mean for different classes of $\theta^{\circ}$ and $\lambda$ with $\theta^{\times} \equiv 0$ and $s \equiv 1$.}
\label{EQM}
\end{figure}




\begin{figure}
\centering
\begin{subfigure}{.4\textwidth}
  \centering
  \includegraphics[width=1.\linewidth]{density/illu-deconv2/800.png}
  \caption{$\theta^{\circ}$ polynomial and $\lambda$ polynomial}
  \label{fig3:sub1}
\end{subfigure}%
\begin{subfigure}{.4\textwidth}
  \centering
  \includegraphics[width=1.\linewidth]{density/illu-deconv2/850.png}
  \caption{$\theta^{\circ}$ polynomial and $\lambda$ exponential}
  \label{fig3:sub2}
\end{subfigure}

\begin{subfigure}{.4\textwidth}
  \centering
  \includegraphics[width=1.\linewidth]{density/illu-deconv2/900.png}
  \caption{$\theta^{\circ}$ polynomial and $\lambda$ exponential}
  \label{fig3:sub2}
\end{subfigure}
\begin{subfigure}{.4\textwidth}
  \centering
  \includegraphics[width=1.\linewidth]{density/illu-deconv2/950.png}
  \caption{$\theta^{\circ}$ polynomial and $\lambda$ exponential}
  \label{fig3:sub2}
\end{subfigure}

\begin{subfigure}{.4\textwidth}
  \centering
  \includegraphics[width=1.\linewidth]{density/illu-deconv2/997.png}
  \caption{$\theta^{\circ}$ polynomial and $\lambda$ exponential}
  \label{fig3:sub2}
\end{subfigure}
\caption{Estimated median of the quadratic error of the estimator given by the posterior mean for different classes of $\theta^{\circ}$ and $\lambda$ with $\theta^{\times} \equiv 0$ and $s \equiv 1$.}
\label{EQM}
\end{figure}

\section{General illustrations}\label{D.3}

\begin{figure}
\centering
\begin{subfigure}{.5\textwidth}
  \centering
  \includegraphics[width=1.\linewidth]{general/direct-case.pdf}
  \caption{$\theta^{\circ}$ polynomial and $\lambda$ polynomial}
  \label{fig3:sub1}
\end{subfigure}%
\begin{subfigure}{.5\textwidth}
  \centering
  \includegraphics[width=1.\linewidth]{general/mildly-illposed.pdf}
  \caption{$\theta^{\circ}$ polynomial and $\lambda$ exponential}
  \label{fig3:sub2}
\end{subfigure}
\caption{Estimated median of the quadratic error of the estimator given by the posterior mean for different classes of $\theta^{\circ}$ and $\lambda$ with $\theta^{\times} \equiv 0$ and $s \equiv 1$.}
\label{EQM}
\end{figure}








%
\chapter{R Code}\label{CODE}
%
%\chapter*{List of notations}

\section*{Spaces}
\subsection*{General case}
\begin{alignat*}{3}
& \left(\mathds{Y}, \mathcal{Y}\right) &&: && \quad \text{the measurable space of observations};\\
& \mathds{T} \subset \R &&: && \quad \text{time domain};\\
& \mathds{D} \subset \R &&: && \quad \text{intensity domain};\\
& \Xi = \left\{f : \mathds{T} \rightarrow \mathds{D}\right\} &&: && \quad \text{the parameter space represented in time domain};\\
& \mathds{F} \subset \R &&: && \quad \text{frequency domain};\\
& \mathds{A} \subset \C &&: && \quad \text{amplitude in frequency domain};\\
& \left(\Theta, \mathcal{B}\right) = \left(\left\{\theta : \mathds{F} \rightarrow \mathds{A}\right\}, \mathcal{B}\right) &&: && \quad \text{the parameter space represented in frequency domain};
\end{alignat*}

\subsection*{Inverse Gaussian sequence space model}
\begin{alignat*}{3}
%& \left([0, 1[, \mathcal{A}\right) &&: && \quad \text{the measurable space of observations};\\
%& \mathcal{D}_{\mu}([0,1[) && : && \quad \text{space of densities on } [0, 1[ \text{ with respect to } \mu ;\\
& \mathcal{M}([0, 1[) && : && \quad \text{space of probability measures on } [0, 1[\\
& \mathcal{S}^{+}(\mathds{Z}) && : && \quad \text{ set of all positive definite, complex valued functions } [f] \text{ on } \mathds{Z} \text{ with } [f](0) = 1;\\
\end{alignat*}

\subsection*{Circular density deconvolution model}
\begin{alignat*}{3}
%& \left([0, 1[, \mathcal{A}\right) &&: && \quad \text{the measurable space of observations};\\
%& \mathcal{D}_{\mu}([0,1[) && : && \quad \text{space of densities on } [0, 1[ \text{ with respect to } \mu ;\\
& \mathcal{M}([0, 1[) && : && \quad \text{space of probability measures on } [0, 1[\\
& \mathcal{S}^{+}(\mathds{Z}) && : && \quad \text{ set of all positive definite, complex valued functions } [f] \text{ on } \mathds{Z} \text{ with } [f](0) = 1;\\
\end{alignat*}

\section*{Measures and densities}

\subsection*{General case}
\begin{alignat*}{5}
& \P_{\boldsymbol{\theta}^{m}} &&:&& \mathcal{B} \rightarrow [0,1])_{f \in \Xi} &&:&& \quad \text{Gaussian sieve prior with threshold parameter }m\\
\end{alignat*}


$\P_{\boldsymbol{\theta}^{M}}$

$\P_{M}$

$\P_{M \vert Y^{n}}^{n, (\eta)}$

$\P_{\boldsymbol{\theta}^{m}\vert Y^{n}}^{n, (\eta)}$

$\P_{\boldsymbol{\theta}^{M}\vert Y^{n}}^{n, (\eta)}$

\subsection*{Inverse Gaussian sequence space model}
\subsection*{Circular density deconvolution model}

\begin{alignat*}{5}
& (\mathds{P}_{Y \vert f} &&:&& \mathcal{Y} \rightarrow [0,1])_{f \in \Xi} &&:&& \quad \text{family to which the data distribution belongs}\\
& \mathds{P}^{X} &&:&& \mathcal{A} \rightarrow [0,1] &&: && \quad \text{distribution of interest};\\
& \mathds{P}^{\epsilon}&&:&& \mathcal{A} \rightarrow [0,1] &&: && \quad \text{distribution of the noise};\\
& \mu&&:&& \mathcal{A} \rightarrow \mathds{R}_{+} &&: && \quad \text{ a sigma-finite measure, dominating both } \mathds{P}^{X} \text{ and } \mathds{P}^{\epsilon};\\
& f^{X}&&:&& [0, 1[ \rightarrow \overline{\mathds{R}_{+}} &&: && \quad \text{density of } \mathds{P}^{X} \text{ with respect to } \mu;\\
& f^{\epsilon}&&:&& [0, 1[ \rightarrow \overline{\mathds{R}_{+}} &&: && \quad \text{density of } \mathds{P}^{\epsilon} \text{ with respect to } \mu;\\
& \mathds{P}^{Y} (=\mathds{P}^{X}*\mathds{P}^{\epsilon})&&:&& \mathcal{A} \rightarrow [0,1] &&: && \quad \text{distribution of the observations};\\
& f^{Y}(=f^{X}*f^{\epsilon})&&:&& [0, 1[ \rightarrow \overline{\mathds{R}_{+}} &&: && \quad \text{density of } \mathds{P}^{Y} \text{ with respect to } \mu;\\
& \# && : && \sigma(\mathcal{P}(G)) \rightarrow \N && : && \text{counting measure on the sigma-algebra generated by the set of subsets of} G
\end{alignat*}


\section*{Random variables}
\subsection*{General case}
\subsection*{Inverse Gaussian sequence space model}
\subsection*{Circular density deconvolution model}
\begin{alignat*}{5}
& X &&:&& (\Omega, \mathcal{B}) \rightarrow ([0,1[, \mathcal{A}) &&: && \quad \text{a random variable with distribution } \mathds{P}^{X};\\
& \epsilon &&:&& (\Omega, \mathcal{B}) \rightarrow ([0,1[, \mathcal{A})  &&: && \quad \text{a random variable with distribution } \mathds{P}^{\epsilon};\\
& Y (=X \Box \epsilon) &&:&& (\Omega, \mathcal{B}) \rightarrow ([0,1[, \mathcal{A})  &&: && \quad \text{a random variable with distribution } \mathds{P}^{Y};\\
& X^{n} (=(X^{n}_{i})_{i \in \llbracket 1, n \rrbracket}) &&:&& (\Omega, \mathcal{B}) \rightarrow ([0,1[^{n}, \mathcal{A}^{\otimes n}) &&: && \quad \text{a } n \text{-vector of i.i.d. replications of } X;\\
& \epsilon^{n} (=(\epsilon^{n}_{i})_{i \in \llbracket 1, n \rrbracket})&&:&& (\Omega, \mathcal{B}) \rightarrow ([0,1[^{n}, \mathcal{A}^{\otimes n}) &&: && \quad \text{a } n \text{-vector of i.i.d. replications of } \epsilon;\\
& Y^{n} (=(Y^{n}_{i})_{i \in \llbracket 1, n \rrbracket}) &&:&& (\Omega, \mathcal{B}) \rightarrow ([0,1[^{n}, \mathcal{A}^{\otimes n})  &&: && \quad \text{a } n \text{-vector of i.i.d. replications of } Y;\\
& \boldsymbol{\theta}^{m} &&:&& (\Omega, \mathcal{B}) \rightarrow ([0,1[, \mathcal{A}) &&: && \quad \text{a random variable with distribution } \mathds{P}^{X};\\
& \boldsymbol{\theta}^{M} &&:&& (\Omega, \mathcal{B}) \rightarrow ([0,1[, \mathcal{A}) &&: && \quad \text{a random variable with distribution } \mathds{P}^{X};\\
& \widetilde{\theta}^{m, (\eta)} &&:&& (\Omega, \mathcal{B}) \rightarrow ([0,1[, \mathcal{A}) &&: && \quad \text{a random variable with distribution } \mathds{P}^{X};\\
& \widehat{\theta}^{(\eta)} &&:&& (\Omega, \mathcal{B}) \rightarrow ([0,1[, \mathcal{A}) &&: && \quad \text{a random variable with distribution } \mathds{P}^{X};\\
& \bar{\theta}^{m, (\eta)} &&:&& (\Omega, \mathcal{B}) \rightarrow ([0,1[, \mathcal{A}) &&: && \quad \text{a random variable with distribution } \mathds{P}^{X};
\end{alignat*}

\section*{Unary operators}
\subsection*{General case}
\subsection*{Inverse Gaussian sequence space model}
\subsection*{Circular density deconvolution model}
\begin{alignat*}{6}
& &&\mathds{E}[\cdot] && && &&:&& \quad \text{the expectation operator when the distribution is obvious};\\
& \forall \mathds{P} \text{ distribution } && \mathds{E}_{\mathds{P}}[\cdot] && && &&:&& \quad \text{the expected value under } \mathds{P};\\
& && ^{*}\cdot &&:&& \mathcal{M}([0, 1[) && \rightarrow && \mathcal{M}([0, 1[);\\
& && && && \mathds{P} && \mapsto && ^{*}\mathds{P} = \mathds{P}*\mathds{P}^{\epsilon}\\
& && ^{*}\cdot &&:&& \mathcal{D}_{\mu}([0, 1[) && \rightarrow && \mathcal{D}_{\mu}([0, 1[);\\
& && && && f && \mapsto && ^{*}f = f*f^{\epsilon}\\
& \forall j \in \mathds{Z} && e_{j}(\cdot) &&:&& [0,1[ && \rightarrow && \mathds{C};\\
& && && && x && \mapsto && \exp[-2 i \pi j x]\\
& && \mathcal{F}_{\mu}(\cdot) &&:&& \mathcal{D}_{\mu}([0, 1[) && \rightarrow && \mathcal{S}^{+}(\mathds{Z});\\
& && && && f && \mapsto && [f] = \left(j \mapsto \int_{0}^{1} f(x) e_{j}(x) d\mu(x)\right)\\
& && \mathcal{F}(\cdot) &&:&& \mathcal{M}([0, 1[) && \rightarrow && \mathcal{S}^{+}(\mathds{Z});\\
& && && && \mathds{P} && \mapsto && [\mathds{P}] = \left(j \mapsto \int_{0}^{1} e_{j}(x) d\mathds{P}(x)\right)\\
& && \mathcal{F}_{\mu}^{-1}(\cdot) &&:&& \mathcal{S}^{+}(\mathds{Z}) && \rightarrow && \mathcal{D}_{\mu}([0, 1[);\\
& && && && [f] && \mapsto && f = \left(x \mapsto \sum\limits_{j \in \mathds{Z}} [f]_{j} e_{j}(x) \right)\\
& && \mathcal{F}^{-1}(\cdot) &&:&& \mathcal{S}^{+}(\mathds{Z}) && \rightarrow && \mathcal{M}([0, 1[);\\
& && && && [\mathds{P}] && \mapsto && \mathds{P} = \left(A \mapsto \int_{A} \sum\limits_{j \in \mathds{Z}} [\mathds{P}]_{j} e_{j}(x) dx\right)\\
\end{alignat*}

\section*{Binary operators}
\subsection*{General case}
\subsection*{Inverse Gaussian sequence space model}
\subsection*{Circular density deconvolution model}
\begin{alignat*}{7}
& \cdot \Box \cdot &&:&& [0,1[^{2} &&\rightarrow&& [0,1[ &&: && \quad \text{the modular addition binary operator on the unit segment};\\
& && && (x,y) &&\mapsto&& x+y-\lfloor x+y \rfloor && &&\\
\end{alignat*}

\section*{Miscellaneous}
\subsection*{General case}
\subsection*{Inverse Gaussian sequence space model}
\subsection*{Circular density deconvolution model}
For any set $S$, subset $s \subseteq S$ we note $\mathds{1}_{s}$ the indicatrix function
\begin{alignat*}{5}
	&\mathds{1}_{s} &&:&& S &&\rightarrow&& \{0, 1\};\\
	& && && x && \mapsto &&
		\begin{cases}
			0 \text{ if } x \notin s\\
			1 \text{ if } x \in s
		\end{cases}
\end{alignat*}

















\bibliography{biblio.bib}{}
\nocite{*}
\bibliographystyle{apalike}
%

\end{document}
%%% Local Variables:
%%% mode: latex
%%% TeX-master: t
%%% End:
