\section{Statistical ill-posed inverse problems}\label{INTRO_INVERSE}

\subsection{Statistical model}\label{INTRO_INVERSE_STATMOD}

Consider a measurable observation space $(\mathds{Y}, \mathcal{Y})$, a parameter space $\Xi$ and a family of probability distribution on $(\mathds{Y}, \mathcal{Y})$ indexed by $\Xi$, which we denote $(\P_{Y \vert f})_{f \in \Xi}$.
That is to say, for any $f$ in $\Xi$, $\P_{Y \vert f}$ is a probability measure.

\bigskip

Throughout this thesis, we are interested in the estimation of the parameter $f$, under the two paradigmata of frequentist and Bayesian statistics, specified hereafter, and the quantification of the quality of such estimations.

From a frequentist point of view, one specifies a measurable application from $\mathds{Y}$ to $\Xi$, called estimator.
Ideally, an estimator solely depends of the observations and needs no knowledge whatsoever about the parameter of interest in order to be implemented properly.
Once this application specified the duty of the statistician is to study properties such as consistency, rate of convergence, and asymptotic distribution of the estimator.
These properties are presented in more details in \nref{INTRO_FREQ}.

\medskip

In the Bayesian paradigm, one defines a $\sigma$-algebra $\mathcal{B}$ on $\Xi$ and a probability distribution $\P_{\boldsymbol{f}}$ on $(\Xi, \mathcal{B})$ called prior distribution which represents the prior knowledge about the parameter, for example gathered by experts or prior experiments.
One is then interested in the posterior distribution, that is, the distribution of the parameter of interest given the observations.
From a purely Bayesian point of view, being able to define a prior distribution on $\Xi$ and compute (or approach) the posterior distribution is all that is needed as one does not assume existence of a true underlying parameter.
However, from a frequentist Bayesian (also called pragmatic Bayesian) approach, described in details in \nref{INTRO_BAYES}, one assumes existence of a true parameter and wonders if the posterior distribution contracts around this true parameter.

\bigskip

We are particularly interested in a specific class of models where $\Xi$ is a subset of a function (or sequence) space.
More specifically, let $\mathds{T}$ and $\mathds{D}$ be a subsets of $\R$ ($\mathds{T}$ might be referred to as time domain and $\mathds{D}$ as intensity domain).
Denote $\Vert \cdot \Vert_{\Xi}$ a norm on the space of functions $\mathds{T} \rightarrow \mathds{D}$.
Then $\Xi$ is the space of functions $\{f : \mathds{T} \rightarrow \mathds{D}, \Vert f \Vert_{\Xi} < \infty\}$.

\bigskip

We will assume moreover that there exist a measure $\P^{\circ}$ on $(\mathds{Y}, \mathcal{Y})$ dominating the family $(\P_{Y \vert f})_{f \in \Xi}$ and we denote $L : \left(\Xi \times \mathds{Y}, \mathcal{B} \otimes \mathcal{Y}\right) \rightarrow \left(\overline{\R}_{+}, \mathcal{B}(\R)\right)$ the likelihood with respect to $\P^{\circ}$:
\[\frac{\mathds{P}_{Y \vert f}}{\P^{\circ}}(f, y) = L(f, y).\]

\subsection{(Compact operator)}\label{INTRO_INVERSE_COMPACTOPE}
We will give particular interest to inverse problems, a family of models where one wants to infer on a parameter $f$ but the data we observe is generated through the distribution with parameter $T(f)$ where $T$ is an operator from $\Xi$ to itself.

Hence we have:

\begin{alignat*}{6}
& T && : && \Xi && \rightarrow && \Xi && ; \\
& && && f && \rightarrow && T(f) && \\
& && && Y && \sim && \P_{Y \vert T(f)} &&.
\end{alignat*}

These models gathered interest for a long time because many of them have the particularity to be ill-posed in the sense of \citet{cite:hadamard}.
That is to say, if we build an estimator $\widehat{T(f)}$ of $T(f)$ from the data $Y$ and try to apply $T^{-1}$ to this estimator in order to estimate $f$, one of the following problems might arise:
\begin{itemize}
\item non existence (the equation $T(x) = \widehat{T(f)}$ does not have a solution);
\item non unicity (the equation $T(x) = \widehat{T(f)}$ has multiple solutions);
\item non stability (the solutions to the equations $T(x) = \widehat{T(f)}$ and $T(x) = \widehat{T(f)} + \epsilon$ are arbitrary far for $\epsilon$ arbitrary small with respect to $\Vert \cdot \Vert$).
\end{itemize}

Though Hadamard thought that inverse problems do not arise in practical situations and that problems of our realm only are of the well-posed kind; evolution of science proved him wrong and ill-posed problems now have many applications and the specific challenges they represent gather a lot of interest as we will show through two examples in \nref{INTRO_IGSSM} and \nref{INTRO_CIRCULARDECONVOLUTION}.

\medskip

More-Penrose inverse.

Self adjoint.

Orthogonal basis of eigen functions  -> notation $(e_{j})_{j \in \mathds{F}}, \left(\lambda_{j}\right)_{j \in \mathds{F}}$.

$\mathcal{F}$ the application $\Xi \rightarrow \Theta, f \mapsto \theta = (\langle f \vert e_{j} \rangle)_{j \in \mathds{F}}$.
$\mathcal{F}^{-1}$ the application $\Theta \rightarrow \Xi, \theta \mapsto f = (\int_{j \in \mathds{F}} \theta_{j} e_{j}(x))_{x \in \mathds{T}}$.
Plancherel theorem.
$\P_{Y \vert \theta}$, $\P_{\boldsymbol{\theta}}$, $L(\theta, y)$

Compactness, $\lambda_{j} \rightarrow 0$ -> inverse unbounded (condition 3 not check).
$\mathcal{F}(T(f)) = \lambda \theta$

\subsection{Ill-posed inverse problem with known operator}\label{INTRO_INVERSE_KNOWN}
$e_{j}$ and $\lambda_{j}$ known

\subsection{Ill-posed inverse problem with partially known operator}\label{INTRO_INVERSE_UNKNOWN}
$e_{j}$ known but only an estimator $\widehat{\lambda_{j}}$ of $\lambda_{j}$ is available.
%
%\subsection{(Ill-posed inverse problem with unknown operator)}\label{1.1.5}
%$e_{j}$ and $\lambda_{j}$ estimated.

\subsection{Popular regularisation methods}\label{INTRO_INVERSE_REGULARISE}
\begin{itemize}
\item spectral cut-off
\item Tikhonov
\item entropy minimisation
\item ...?
\end{itemize}