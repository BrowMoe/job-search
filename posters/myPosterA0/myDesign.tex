%
%
%
% LENGTHS
%
\newlength{\myTopToPretitleVSpace}
\setlength{\myTopToPretitleVSpace}{0mm}
\newlength{\myPreTitleInnerYSep}
\setlength{\myPreTitleInnerYSep}{15mm}
\newlength{\preTitleSpaceLength}
\setlength{\preTitleSpaceLength}{8mm}
\newlength{\myMainTitleInnerYSep}
\setlength{\myMainTitleInnerYSep}{15mm}
\newlength{\myMainTitleFooterHeight}
\setlength{\myMainTitleFooterHeight}{5mm}
\newlength{\myMainTitleToBodyVSpace}
\setlength{\myMainTitleToBodyVSpace}{25mm}
\newlength{\myFooterHeaderHeight}
\setlength{\myFooterHeaderHeight}{5mm}
\newlength{\myFooterBodyInnerYSep}
\setlength{\myFooterBodyInnerYSep}{10mm}
\newlength{\myFooterBodyColumnSize}
\setlength{\myFooterBodyColumnSize}{400mm}
\newlength{\myBlockColHSpace}
\setlength{\myBlockColHSpace}{15mm}
\newlength{\myBlockToBlockVSpace}
\setlength{\myBlockToBlockVSpace}{15mm}
\newlength{\myBlockTitleToBodyVSpace}
\setlength{\myBlockTitleToBodyVSpace}{0mm}
\newlength{\myBlockInnerXSep}
\setlength{\myBlockInnerXSep}{0mm}
\newlength{\myBlockInnerYSep}
\setlength{\myBlockInnerYSep}{0mm}
\newlength{\myBlockBodyInnerXSep}
\setlength{\myBlockBodyInnerXSep}{10mm}
\newlength{\myBlockBodyInnerYSep}
\setlength{\myBlockBodyInnerYSep}{10mm}
\newlength{\myBlockTextWidth}
\setlength{\myBlockTextWidth}{385mm}
%
%
%
% COLORS
%
\definecolor{colorBg}{HTML}{FFFFFF}
\definecolor{RTGblue}{HTML}{003466}
\definecolor{RTGred}{HTML}{C80B33}
\definecolor{RTGgray}{HTML}{3C3C3C}
\definecolor{RTGlightgray}{HTML}{DCDCDC}
\colorlet{colMainBg}{RTGlightgray}
\colorlet{colPreTitleBg}{colorBg}
\colorlet{colPreTitleFg}{RTGgray}
\colorlet{colMainTitleBg}{RTGblue}
\colorlet{colMainTitleFg}{white}
\colorlet{colFooterBodyBg}{RTGblue}
\colorlet{colFooterBodyFg}{white}
\colorlet{colFooterHeaderBg}{RTGred}
\colorlet{colFooterHeaderFg}{white}
\colorlet{colBlockBg}{RTGlightgray}
\colorlet{colBlockFg}{black}
\colorlet{colBlockTitleBg}{RTGblue}
\colorlet{colBlockTitleFg}{white}
\colorlet{colBlockBodyBg}{white}
\colorlet{colBlockBodyFg}{RTGgray}
%
%
%
% FONT STYLE
%
% available font sizes:
%\tiny          12pt
%\scriptsize    14.4pt
%\footnotesize  17.28pt
%\small         20.74pt
%\normalsize    24.88pt
%\large         29.86pt
%\Large         35.83pt
%\LARGE         43pt
%\huge          51.6pt
%\Huge          61.92pt
%\veryHuge      74.3pt
%\VeryHuge      89.16pt
%\VERYHuge      107p
\newcommand*{\fsPreTitle}{\fontsize{74.3}{89}\selectfont\bf}
\newcommand*{\fsPreTitleRTG}{\LARGE\bf}
\newcommand*{\fsMainTitle}{\Huge}
\newcommand*{\fsName}{\LARGE\bf}
\newcommand*{\fsSuperv}{\LARGE}
\newcommand*{\fsBlockTitle}{\LARGE\bf}
\newcommand*{\fsBlockBody}{\normalsize}
\newcommand*{\fsFooterLeft}{\Large}
\newcommand*{\fsFooterRight}{\LARGE\bf}
%
%
%
% PRETITLE: a block for the pre-title at the very top
%
\newcommand{\myPreTitle}{%
	\vspace*{\myTopToPretitleVSpace}\\
	\begin{minipage}[c]{18cm}
		\centering
		\includegraphics[height=5cm]{logoMAt} 
		%\includegraphics[height=7cm]{logoMA} 
	\end{minipage}
	\begin{minipage}[c]{40cm}%
		\centering%
		{\fsPreTitle%
		\spaceskip=0.25em
		Statistical Modeling of\\Complex Systems and Processes\par}
	\end{minipage}
	\hspace*{-1cm}
	\begin{minipage}[c]{18cm}
		\centering
		\includegraphics[height=7cm]{logoHDt} 
		%\includegraphics[height=7cm]{logoHD} 
	\end{minipage}%
}
%
%
%
% MAINTITLE: a block for the title and name
%
\newcommand{\myMainTitle}{%
	\begin{tikzpicture}[]
		\node[
			colMainTitleFg,
			fill=colMainTitleBg,
			anchor=north,
			align=center,
			inner xsep=0,
			inner ysep=\myMainTitleInnerYSep,
		] (mainTitleBody) {%
			{\fsMainTitle%
			\spaceskip=0.25em%
			\myTitle}%
			\\%
			\vspace*{10mm}%
			{\fsName%
			\spaceskip=0.25em%
			\myName}%
			\\%
			\vspace*{10mm}%
			{\fsSuperv%
				\spaceskip=0.25em%
				Supervisors: \mySuperv}%
		};
		\node[
			fill=RTGred,
			anchor=north,
			align=center,
			inner xsep=0,
			inner ysep=0,
			minimum height=\myMainTitleFooterHeight,
			below =0of mainTitleBody,
		] (mainTitleFooter) {};
	\end{tikzpicture}
}
%
%
%
% FOOTER: block at the very bottom of the poster
%
\newcommand{\myFooter}{%
	\begin{tikzpicture}[]
		\node[
			colFooterBodyFg,
			fill=colFooterBodyBg,
			anchor=south,
			align=center,
			inner xsep=0,
			inner ysep=\myFooterBodyInnerYSep,
		] (footerBody) {
			\begin{minipage}[c]{\myFooterBodyColumnSize}
				\fsFooterLeft
				Supported by\hspace{0.5em}
				\includegraphics[height=15mm]{logoDFG}
			\end{minipage}
			\begin{minipage}[c]{\myFooterBodyColumnSize}
				\flushright
				\fsFooterRight
				RTG 1953
			\end{minipage}	
		};
		\node[
			colFooterHeaderFg,
			fill=colFooterHeaderBg,
			anchor=south,
			align=center,
			inner xsep=0,
			inner ysep=0,
			minimum height=\myFooterHeaderHeight,
			above =0of footerBody,
		] (footerHeader) {};
	\end{tikzpicture}
}
%
%
%
% INNERBLOCK: Content of a Block: BlockTitle, BlockBody
%
\newcommand{\myInnerBlock}[2]{%
	\begin{tikzpicture}[]
	\ifx&#1&
	\node[
	colBlockBodyFg,
	fill=colBlockBodyBg,
	anchor=center,
	align=justify,
	inner xsep=\myBlockBodyInnerXSep,
	inner ysep=\myBlockBodyInnerYSep,
	text width=\myBlockTextWidth-2.*\myBlockInnerXSep-2.*\myBlockBodyInnerXSep,
	] (body) {%
		\fsBlockBody%
		#2%
		\par%
	};
	\else
	\node[
	colBlockTitleFg,
	fill=colBlockTitleBg,
	anchor=center,
	align=center,
	inner xsep=\myBlockBodyInnerXSep,
	inner ysep=0mm,
	text width=\myBlockTextWidth-2.*\myBlockInnerXSep-2.*\myBlockBodyInnerXSep,
	] (title) at (0,0) {%
		\vspace*{12mm}%
		\\%
		\fsBlockTitle%
		\spaceskip=0.25em
		#1%
		\vspace*{10mm}%
	};
	\node[
	colBlockBodyFg,
	fill=colBlockBodyBg,
	anchor=center,
	align=justify,
	inner xsep=\myBlockBodyInnerXSep,
	inner ysep=\myBlockBodyInnerYSep,
	text width=\myBlockTextWidth-2.*\myBlockInnerXSep-2.*\myBlockBodyInnerXSep,
	below =\myBlockTitleToBodyVSpace of title
	] (body) {%
		\fsBlockBody%
		#2%
		\par%
	};
	\fi
	\end{tikzpicture}
}
%
%
%
% BLOCKSTYLE: style for blocks, which surround BlockTitle and BlockBody
%
\tikzset{BlockStyle/.style={%
		colBlockFg,
		fill=colBlockBg,
		anchor=north west,
		inner sep=0,
		rectangle, 
		text width=\myBlockTextWidth-2.*\myBlockInnerXSep,
		inner xsep=\myBlockInnerXSep,
		inner ysep=\myBlockInnerYSep,
	}
}
%
%
%
% BLOCK: command for placing a new block below the previous one
%
\newcounter{myColumnCounter}
\newcounter{myBlockCounter}[myColumnCounter]
%
\newcommand{\myFirstBlock}[2]{%
	\node[%
		BlockStyle,
	] (a) {%
		\myInnerBlock{#1}{#2}%
	};%
}%
%
\newcommand{\myFurtherBlock}[2]{%
	\node[%
		BlockStyle,
		below =\myBlockToBlockVSpace of a,
	] (a) {%
		\myInnerBlock{#1}{#2}%
	};%
}%
%
\newcommand{\myBlock}[2]{%
	\ifnum\themyBlockCounter=0%
		\myFirstBlock{#1}{#2}%
	\else%
		\myFurtherBlock{#1}{#2}%
	\fi%
	\stepcounter{myBlockCounter}%
}
%
%
%
% THEOREM STYLES
%
\declaretheoremstyle[
	bodyfont=\normalfont,
	postfoothook={\vspace*{30pt}},
	preheadhook={},
	mdframed={
		backgroundcolor = RTGlightgray!40!white,
		innertopmargin = 20pt,
		innerbottommargin = 20pt,
		innerrightmargin = 20pt,
		innerleftmargin = 20pt,
		skipabove = 30pt,
		skipbelow = 30pt,
		leftmargin = 30pt,
		rightmargin = 30pt,
		%		roundcorner = 0pt,
		%		topline = true,
		%		bottomline = true,
		leftline = false,
		rightline = false,
		%		hidealllines = true,
		linewidth = 5pt,
		%		innerlinewidth = 5pt,
		%		middlelinewidth = 6pt,	
		%		outerlinewidth = 2pt,
		linecolor = RTGred,
		%		innerlinecolor = orange,
		%		middlelinecolor = green,	
		%		outerlinecolor = RTGblue,
		%		fontcolor = black,
		%		font = ?,
		%		shadow = true,
		%		shadowsize = 8pt,
		%		shadowcolor = black!50,
		%		frametitle = {Title},
		%		frametitle<Option...>,
		%		%\mdfsubtitle{SubTitle}
		%		subtitle<Option...>,
		%		tikzsetting = {},
		%		align =center,
		%		extra = {},		
	}]{redHLineStyle}
%
%
% numbered
\declaretheorem[style=redHLineStyle,name=Definition]{definition}
\declaretheorem[style=redHLineStyle,name=Lemma,numberlike=definition]{lemma}
\declaretheorem[style=redHLineStyle,name=Theorem,numberlike=definition]{theorem}
\declaretheorem[style=redHLineStyle,name=Corollary,numberlike=definition]{corollary}
\declaretheorem[style=redHLineStyle,name=Remark,numberlike=definition]{remark}
\declaretheorem[style=redHLineStyle,name=Example,numberlike=definition]{example}
%
% no numbering
\declaretheorem[style=redHLineStyle,name=Definition,numbered=no]{definition*}
\declaretheorem[style=redHLineStyle,name=Lemma,numbered=no]{lemma*}
\declaretheorem[style=redHLineStyle,name=Theorem,numbered=no]{theorem*}
\declaretheorem[style=redHLineStyle,name=Corollary,numbered=no]{corollary*}
\declaretheorem[style=redHLineStyle,name=Remark,numbered=no]{remark*}
\declaretheorem[style=redHLineStyle,name=Example,numbered=no]{example*}
%
%
%