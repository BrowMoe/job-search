%%%%%%%%%%%%%%%%%%%%%%%%%%%%%%%%%%%%%%%%%
\documentclass[11pt,a4paper,sans,colorlinks,linkcolor=blue]{moderncv} % Font sizes: 10, 11, or 12; paper sizes: a4paper, letterpaper, a5paper, legalpaper, executivepaper or landscape; font families: sans or roman

\moderncvstyle{banking} % CV theme - options include: 'casual' (default), 'classic', 'oldstyle' and 'banking'
\moderncvcolor{black} % CV color - options include: 'blue' (default), 'orange', 'green', 'red', 'purple', 'grey' and 'black'
\moderncvicons{awesome}
\usepackage{multicol}

\usepackage[scale=0.75]{geometry}
\usepackage{hyperref} % Reduce document margins
%\setlength{\hintscolumnwidth}{3cm} % Uncomment to change the width of the dates column
%\setlength{\makecvtitlenamewidth}{10cm} % For the 'classic' style, uncomment to adjust the width of the space allocated to your name

\renewcommand*{\cventry}[7][.25em]{
  \begin{tabular*}{\textwidth}{l@{\extracolsep{\fill}}r}%
	  {\bfseries #4} & {\bfseries #5} \\%
	  {#3\ifthenelse{\equal{#6}{}}{}{#6}}&{#2}\\%
  \end{tabular*}%
  \ifx&#7&%
    \else{\\\vbox{\small#7}}\fi%
\par\addvspace{#1}}

\renewcommand*{\cvitem}[3][.25em]{%
  \ifthenelse{\equal{#2}{}}{}{\hintstyle{#2} }{#3}%
\par\addvspace{#1}}
%----------------------------------------------------------------------------------------
%	NAME AND CONTACT INFORMATION SECTION
%----------------------------------------------------------------------------------------

\firstname{Xavier} % Your first name
\familyname{Loizeau} % Your last name
% All information in this block is optional, comment out any lines you don't need
\title{PhD, \href{https://ima.org.uk/}{MIMA}}
\address{Flat 1, The Lamb, 16 Acre Road}{KT2 6EF Kingston upon Thames, England}
\mobile{(+44) 7845 079595}
\email{subscription@loizeau.eu}
\photo[70pt][0.6pt]{picture} % The first bracket is the picture height, the second is the thickness of the frame around the picture (0pt for no frame)
%\quote{
%Mathematicien de formation,
%}

%----------------------------------------------------------------------------------------

\begin{document}
\thispagestyle{empty}

\makecvtitle % Print the CV title

\section{Experiences professionelles}
\cventry{$21$ janvier $2019$ -- actuel}{Higher Research Scientist}{NPL (National Physical Laboratory)}{Teddington, Angleterre}{}{
\begin{multicols}{2}
  \begin{itemize}
    \item \textbf{Recherche:} modelisation mathematique de processus physiques, chimiques, et biologiques.
      Conception d'estimateurs statistiques et etude de leurs proprietes theoriques et empiriques, developpement de
      logiciels pour l'analyse de donnees et l'apprentissage machine, interpretation de modeles mathematiques bases sur
      des donnees et prise de decision rationelle;
    \item \textbf{Developpement de logiciels:} creation de templates de d\'epots git, templates SLURM, preparation
      d'audit par TickIT+, creation de template \LaTeX{} et Beamer;
    \item \textbf{Management:} co-supervision de doctorants, supervision d'\'etudiants (niveau L3) en stage de 12 mois,
      direction technique de scientifiques moins experiment\é$\bullet$e$\bullet$s,
    \item \textbf{Strat\'egie:} redaction de demandes de financement (interne, nationale, et Europ\'eenne), cr\'eation
      de collaborations entre d\'epartements du laboratoire, contribution aux politiques internes du laboratoire et \'à
      la preparation d'une candidature \'a l'atestation Stonewall en tant que membre du commit\'e LGBTQ+, r\'ealisation
      d'entretiens pour le recrutement.
  \end{itemize}
\end{multicols}
}
\cventry{2015--2018}{Doctorant en math\'ematiques}{Ruprecht-Karls-Universit\"at, Institut de math\'ematiques appliqu\'ees,}{Heidelberg, Allemagne}{}{
  \begin{multicols}{2}
  \begin{itemize}
    \item \textbf{Recherche:} M\'ethodes Bay\'esiennes hierarchiques and aggr\'egation fr\'equentistes pour les
      probl\`emes inverses: demonstration de l'optimalit\'e d'a priori hierarchiques et d'estimateurs par aggregation en
      terme de taux de contraction et convergence oracle et minimax pour des modeles de problemes inverses
      (equation de chaleur, deconvolution de densites) lorsque l'operateur est inconnu;
    \item \textbf{Enseignement:} encadrement de travaux dirig\'es, preparation et correction d'examens, co-encadrement
      de m\'emoires (Licence et Master);
    \item \textbf{Strat\'egie:} Contribution \'a la r\'edaction de \href{http://rtg1953.uni-heidelberg.uni-mannheim.de/research-training-group-1953.html}{demandes de financement}.
  \end{itemize}
}
\cventry{2015}{Stage}{ONERA (Office National d'\'etudes et de recherche en a\'erospatial)}{Palaiseau, France}{}{
  Creation d'un modele de substitution a partir de donnees multi-fidelite pour les signatures infra-rouge de moteur
  d'engins aerospatiaux.}
\cventry{2014}{Stage}{CREST (Centre de recherche pour l'\'economie et la statistique)}{Rennes, France}{}{
  Etude, implementation et comparaison de methodes de correction de biais d'illumination pour les images de microscopie
  electronique.}

\medskip

\subsection{Relevant technical skills}
\begin{multicols}{2}
\underline{\textsc{Applications:}}
Projects pluridisciplinaires (communication, capacite d'adaptation), image satellitaires, altimetry par satellite,
pathologie, conception de capteurs, imagerie par spectrometrie de masse, analyse d'images.

\underline{\textsc{Mathematiques:}}
problemes inverses mal poses, statistique non-parametrique bayesienne et frequentiste, analyse de Fourier,
theorie minimax, processus ponctuels, analyse de donnees fonctionnelles, processus stochastiques, series temporelles,
theorie des probabilites, algorithmes d'approximation, quantification d'incertitude pour modeles parcimonieux,
application du transport optimal a l'analyse de donnees, methodes MCMC;

\underline{\textsc{Informatique:}}
Python (usage avance), R (usage avance), C++ (intermediaire), MatLab (intermediaire),
git avec GitHub/GitLab (usage avance), IDEs (emacs, visual studio, ...), environement Unix.

\end{multicols}

\subsection{Projets}
\href{https://www.npl.co.uk/news/next-generation-of-digital-pathology}{Pathologie digitale} (direction technique, plusieurs articles en preparation),
\href{https://www.bipmwmo22.org/submissions/submission/137}{Niveau moyen de la mer} (direction technique, article en preparation),
\href{https://www.bipmwmo22.org/submissions/submission/138}{Propagation d'incertitude et re-meshing} (direction technique, article en preparation),
\href{https://www.npl.co.uk/earth-observation/truths}{TRUTHS} (conception d'un capteur de calibration),
\href{https://www.npl.co.uk/grand-challenge}{CRUK Rosetta Grand challenge} (analyse de donnees),
\href{https://arxiv.org/pdf/2102.01037}{Taux de convergence minimax-optimal d'estimateurs par aggregation dans un model de deconvolution de densites} (article redige durant le doctorat),
\href{https://pubs.acs.org/doi/abs/10.1021/acs.analchem.1c02470}{Resolution spatial en imagerie par spectrometrie de masse} (co-auteur, co-supervision de doctorant).

%----------------------------------------------------------------------------------------
%	EDUCATION SECTION
%----------------------------------------------------------------------------------------

\section{Education}
\cventry{2012--2015}{MSc in }{ENSAI - Ecole National de la Statistique et de l'Analyse de l'Information}{Rennes, France}{Filiere genie statistique;}{}
\cventry{2015}{MSc in }{Universite Rennes 1, department de mathematiques}{Rennes, France}{Statistique mathematique;}{}  % Arguments not required can be left empty
\cventry{2012}{BSc in }{Universite Rennes 1, department de mathematiques}{Rennes, France}{Mathematiques;}{}
\cventry{2010--2012}{Classe preparatoire}{Lyc\'ee Clemenceau}{Nantes, France}{Filiere MPSI - MP}{}

%\medskip
%
%\subsection{Subjects in study program}
%
%\begin{multicols}{2}
%\underline{\textsc{Mathematics:}}
%probability theory, complex analysis, topology, functional analysis, measure theory, numerical analysis, group theory, arithmetic, linear algebra;
%
%\underline{\textsc{stochastic:}}
%survival analysis, Le Cam theory, test theory, generalized additive models, non linear regression, time series;
%
%\underline{\textsc{Data science:}} neural networks, SVM, random forests, classification/regression trees, CART, BAGGING, MCMC algorithms, image processing (filtering, Markov fields, MAP classification, MAP reconstruction), kNN, surrogate models, design of experiments, quality control of industrial processes.
%\end{multicols}
%------------------------------------------------

\medskip

\subsection{Apprentissage en ligne}
Initiation \`a la th\'eorie des distributions (Coursera, \'Ecole Polytechnique),
Approximation Algorithms Part I (Coursera, \'Ecole Normale Sup\'erieure),
Introduction to Complex Analysis (Coursera, Wesleyan University)

%\section{Programing languages}
%\textbf{Python} (wrote several packages for analysis of data - MSI, satellite imaging, altimetry data, high resolution microscopy, ...), \textbf{R} (many projects during studies, intensive use during internships for simulations and real data applications, data simulations during my PhD, and since joining NPL), \textbf{C++} (using for video game development with Unreal Engine), \textbf{Matlab} (used in research context, during studies for image treatment projects).

%----------------------------------------------------------------------------------------
%	LANGUAGES SECTION
%----------------------------------------------------------------------------------------

\section{Languages}
Francais (langue natale), Anglais (bilingue), Allemand (basique), Espagnol (basique).

%----------------------------------------------------------------------------------------
%	INTERESTS SECTION
%----------------------------------------------------------------------------------------

\section{Centres d'interet}
\textbf{Musique} (guitare, chant, composition),
\textbf{dance} (pole, classique),
\textbf{sport} (escalade de bloc et en voie, velo),
\textbf{lecture} (romans, philosophie, sociologie, surveillance technologique).

% \section{Personal Details}
% \cvitem{Date of Birth:}{$13^{th}$ of January, 1992}
% \cvitem{\faEnvelope :}{Flat 1, 16 Acre Road, KT2 6EF Kingston upon Thames, England}
% \cvitem{\faLinkedin :}{\url{www.linkedin.com/in/x-loizeau}}
% \cvitem{\faGlobe :}{\url{www.xavierloizeau.eu}}

\end{document}
\grid
\grid
\grid
\grid
