%%%%%%%%%%%%%%%%%%%%%%%%%%%%%%%%%%%%%%%%%
\documentclass[11pt,a4paper,sans]{moderncv} % Font sizes: 10, 11, or 12; paper sizes: a4paper, letterpaper, a5paper, legalpaper, executivepaper or landscape; font families: sans or roman

\moderncvstyle{banking} % CV theme - options include: 'casual' (default), 'classic', 'oldstyle' and 'banking'
\moderncvcolor{black} % CV color - options include: 'blue' (default), 'orange', 'green', 'red', 'purple', 'grey' and 'black'
\moderncvicons{awesome}
\usepackage{fontawesome5}
\usepackage{multicol}
\usepackage[french]{babel}
\usepackage[utf8]{inputenc}
\usepackage[T1]{fontenc}
\usepackage[scale=0.75]{geometry}
% \usepackage{hyperref} % Reduce document margins
%\setlength{\hintscolumnwidth}{3cm} % Uncomment to change the width of the dates column
%\setlength{\makecvtitlenamewidth}{10cm} % For the 'classic' style, uncomment to adjust the width of the space allocated to your name

\renewcommand*{\cventry}[7][.25em]{
\begin{tabular*}{\textwidth}{l@{\extracolsep{\fill}}r}%
{\bfseries #4} & {\bfseries #5} \\%
{#3\ifthenelse{\equal{#6}{}}{}{#6}}&{#2}\\%
\end{tabular*}%
\ifx&#7&%
\else{\\\vbox{\small#7}}\fi%
\par\addvspace{#1}}

\renewcommand*{\cvitem}[3][.25em]{%
\ifthenelse{\equal{#2}{}}{}{\hintstyle{#2} }{#3}%
\par\addvspace{#1}}
%----------------------------------------------------------------------------------------
%	NAME AND CONTACT INFORMATION SECTION
%----------------------------------------------------------------------------------------

\firstname{Xavier} % Your first name
\familyname{Loizeau} % Your last name
% All information in this block is optional, comment out any lines you don't need
\title{Dr, \href{https://ima.org.uk/}{MIMA}\textsubscript{\normalsize\faIcon{link}}}
\address{Flat 1, The Lamb, 16 Acre Road}{KT2 6EF Kingston upon Thames, England}
\mobile{(+44) 7845 079595}
\email{subscription@loizeau.eu}
\photo[70pt][0.6pt]{picture} % The first bracket is the picture height, the second is the thickness of the frame around the picture (0pt for no frame)
%\quote{
%Mathematicien de formation,
%}
%----------------------------------------------------------------------------------------

\begin{document}
\thispagestyle{empty}

\makecvtitle % Print the CV title

\section{\faIcon{briefcase} Expériences professionnelles}
\cventry{$21$ janvier $2019$ -- actuel}{Higher Research Scientist}{NPL (National Physical Laboratory)}{Teddington, Angleterre}{}{\vspace{-.5cm}
    \begin{multicols}{2}
        \begin{itemize}
            \item \textbf{Recherche:}
                modélisation mathématique de processus physiques, chimiques, et biologiques.
                Conception d'estimateurs statistiques et étude de leurs propriétés théoriques et empiriques.
                Développement de logiciels pour l'analyse de données et l'apprentissage machine.
                Interprétation de modèles mathématiques et aide à la prise de décision.
                Rédaction d'articles.
                Présentations en conférences.
            \item \textbf{Qualité logiciels:}
                création de gabarits de dépôts git, de scripts SLURM, et de rapports, articles et présentations en \LaTeX{} et Beamer.
                Préparation d'audits par TickIT+;
            \item \textbf{Management:}
                co-supervision de doctorants, supervision d'étudiants (niveau L3) en stage de 12 mois.
                Direction technique de scientifiques moins expérimenté\textbullet e\textbullet s.
            \item \textbf{Stratégie:}
                rédaction de demandes de financement (interne, nationale, et Européenne).
                Création de collaborations entre départements du laboratoire.
                Contribution aux politiques internes du laboratoire dans des \href{https://www.npl.co.uk/news/npl-achieves-new-diversity-and-inclusion-awards}{démarches de diversité et d'inclusivité (prix Stonewall)}\textsubscript{\faIcon{link}}.
                Réalisation d'entretiens de recrutement.
        \end{itemize}
    \end{multicols}
    \vspace{-.3cm}
}
\cventry{2015--2018}{Doctorant en mathématiques}{Ruprecht-Karls-Universität}{Heidelberg, Allemagne}{}{\vspace{-.5cm}
    \begin{multicols}{2}
        \begin{itemize}
            \item \textbf{Recherche:}
                Preuve de l'optimalité oracle et minimax d'a priori hiérarchiques en terme du taux de contraction a posteriori, et d'estimateurs par agrégation en terme du taux de convergence.
                Application à des problèmes inverses mal posés lorsque l'opérateur est inconnu (équation de chaleur, dé-convolution de densités).
                Rédaction d'articles.
                Présentations en conférences.
            \item \textbf{Enseignement:}
                encadrement de travaux dirigés, préparation et correction d'examens, co-encadrement de mémoires (licence et master);
            \item \textbf{Stratégie:}
                Contribution à la rédaction de \href{http://rtg1953.uni-heidelberg.uni-mannheim.de/research-training-group-1953.html}{demandes de financement}\textsubscript{\faIcon{link}}.
        \end{itemize}
    \end{multicols}
    \vspace{-.3cm}
}
\cventry{2015}{Stage}{ONERA (Office National d'études et de recherche en aérospatial)}{Palaiseau, France}{}{
Création d'un modèle de substitution à partir de données multi-fidélités pour les signatures infra-rouges de moteurs d'engins aérospatiaux.}
\cventry{2014}{Stage}{CREST (Centre de recherche pour l'économie et la statistique)}{Rennes, France}{}{
Étude, implémentation et comparaison de méthodes de correction de biais d'illumination pour les images de microscopie électronique.}

\subsection{\faIcon{rocket} \faIcon{dna} Projets}
\href{https://www.npl.co.uk/news/next-generation-of-digital-pathology}{Pathologie digitale} \textsubscript{\faIcon{link}} (direction technique, articles en préparation),
\href{https://www.bipmwmo22.org/submissions/submission/137}{Hausse du niveau moyen de la mer} \textsubscript{\faIcon{link}} (direction technique, article en préparation),
\href{https://www.bipmwmo22.org/submissions/submission/138}{Propagation d'incertitude et re-meshing}\textsubscript{\faIcon{link}} (direction technique, article en préparation),
\href{https://www.npl.co.uk/earth-observation/truths}{TRUTHS}\textsubscript{\faIcon{link}} (conception d'un capteur de calibration),
\href{https://www.npl.co.uk/grand-challenge}{CRUK Rosetta Grand challenge}\textsubscript{\faIcon{link}} (analyse de données),
\href{https://arxiv.org/pdf/2102.01037}{Taux de convergence minimax-optimal d'estimateurs par agrégation dans un modèle de de-convolution de densités}\textsubscript{\faIcon{link}},
\href{https://pubs.acs.org/doi/abs/10.1021/acs.analchem.1c02470}{Résolution spatiale en imagerie par spectrométrie de masse}\textsubscript{\faIcon{link}} (coauteur, co-supervision de doctorant).

\newpage

\subsection{\faIcon{comment-dots} \faIcon{bezier-curve} \faIcon{code} Compétences techniques pertinentes}
\begin{multicols}{2}
\underline{\textsc{Applications:}}
Projets pluridisciplinaires (communication, capacité d'adaptation),
analyse d'images (imagerie spatiale, histopathologie, imagerie par spectrométrie de masse),
altimétrie par satellite,
optimisation de capteurs.

\underline{\textsc{Mathématiques:}}
problèmes inverses mal posés,
statistique non-paramétrique Bayesienne et fréquentiste,
analyse de Fourier,
théorie minimax,
processus ponctuels,
analyse de données fonctionnelles,
processus stochastiques,
séries temporelles,
théorie des probabilités,
algorithmes d'approximation,
quantification d'incertitude pour modèles parcimonieux,
application du transport optimal à l'analyse de données (calibration de couleurs, morphométrie, normalisation de données),
méthodes Markov Chain - Monte Carlo;

\underline{\textsc{Informatique:}}
Python (usage avancé),
R (usage avancé),
C++ (intermédiaire),
MatLab (intermédiaire),
git avec GitHub / GitLab (usage avancé),
IDEs (Emacs, visual studio, pycharm...),
environnement Unix.
\end{multicols}

%----------------------------------------------------------------------------------------
%	EDUCATION SECTION
%----------------------------------------------------------------------------------------

\section{\faIcon{university} Formation}
\cventry{2012--2015}{Titre d'ingénieur en }{ENSAI - École nationale de la statistique et de l'analyse de l'information}{Rennes, France}{génie statistique;}{}
\cventry{2015}{Master de }{Université Rennes 1, département de mathématiques}{Rennes, France}{statistique mathématique;}{}  % Arguments not required can be left empty
\cventry{2012}{Licence de }{Université Rennes 1, département de mathématiques}{Rennes, France}{mathématiques;}{}
\cventry{2010--2012}{Classe préparatoire}{Lycée Clemenceau}{Nantes, France}{Filière MPSI - MP}{}

\medskip

\subsection{\faIcon{globe} Apprentissage en ligne}
Initiation à la théorie des distributions (Coursera, école Polytechnique),
Approximation Algorithms Part I (Coursera, école Normale Supérieure),
Introduction to Complex Analysis (Coursera, Wesleyan University)

%\section{Programing languages}
%\textbf{Python} (wrote several packages for analysis of data - MSI, satellite imaging, altimetry data, high resolution microscopy, ...), \textbf{R} (many projects during studies, intensive use during internships for simulations and real data applications, data simulations during my PhD, and since joining NPL), \textbf{C++} (using for video game development with Unreal Engine), \textbf{Matlab} (used in research context, during studies for image treatment projects).

%----------------------------------------------------------------------------------------
%	LANGUAGES SECTION
%----------------------------------------------------------------------------------------

\section{\faIcon{language} Langages}
Français (langue natale), Anglais (bilingue), Allemand (basique), Espagnol (basique).

%----------------------------------------------------------------------------------------
%	INTERESTS SECTION
%----------------------------------------------------------------------------------------

\section{\faIcon{heart} Centres d'intérêt}
\textbf{Musique} (guitare, chant, composition),
\textbf{danse},
\textbf{sport} (escalade de bloc et en voie, vélo, badminton),
\textbf{lecture} (science-fiction, philosophie, sociologie, veille technologique).

% \section{Personal Details}
% \cvitem{Date of Birth:}{$13^{th}$ of January, 1992}
% \cvitem{\faEnvelope :}{Flat 1, 16 Acre Road, KT2 6EF Kingston upon Thames, England}
% \cvitem{\faLinkedin :}{\url{www.linkedin.com/in/x-loizeau}}
% \cvitem{\faGlobe :}{\url{www.xavierloizeau.eu}}

\end{document}
\grid
\grid
\grid
\grid
